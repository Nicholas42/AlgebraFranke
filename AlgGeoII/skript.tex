\documentclass[a4paper,parskip=half,numbers=enddot, DIV=12]{scrreprt}
%\usepackage[utf8]{inputenc}

\usepackage{../header}
\usepackage{../frankenumbering}
\usepackage{../shortcuts}

\usepackage{eurosym}
\usetikzlibrary{fadings}

\usepackage{csquotes}
%\usepackage{tikz-cd}%I cannot draw diagrams without it - Felix. %well, I can - Ferdinand
\usepackage[backend=biber,style=numeric,sorting=none]{biblatex}
\setcounter{biburlnumpenalty}{7000}
\setcounter{biburllcpenalty}{7000}
\setcounter{biburlucpenalty}{8000}
\addbibresource{../literatur.bib}

% Title Page
\title{Algebraic Geometry II}
\author{Ferdinand Wagner}
\date{Sommersemester 2018}

\displaywidowpenalty=8000
%\postdisplaypenalty=8000
\widowpenalty=8000
\clubpenalty=8000

\newcommand{\vC}{\v{C}}

\begin{document}
\pagenumbering{Alph}
\maketitle
\pagenumbering{roman}

\thispagestyle{plain}
This text consists of notes on the lecture Algebraic Geometry II, taught at the University of Bonn by Professor Jens Franke in the summer term (Sommersemester) 2018. 

Please report bugs, typos etc. through the \emph{Issues} feature of github.

\tableofcontents

\addchap{Introduction}
\pagenumbering{arabic}
This lecture will develop the cohomology of (quasi)coherent sheaves of modules. Professor Franke assumes familiarity with the contents of last term's Algebraic Geometry I. In particular, this includes the category of (pre)schemes, equalizers and fibre products of preschemes as well as in arbitrary categories and quasi-coherent $\Oo_X$-modules. If you want to brush up your knowledge about these topics, the \href{https://github.com/Nicholas42/AlgebraFranke/tree/master/AlgGeoI}{\emph{lecture notes from Algebraic Geometry I}} \cite{alggeo1} might be your friend.

Professor Franke started the lecture with an example of sheaf cohomology entering the game. Let $X$ be a topological space, $\Cc_X$ the sheaf of continuous $\IC$-valued functions on $X$ and $\underline{\IZ}_X$ the sheaf of locally constant (i.e., continuous) functions on $X$ with values in $\IZ$. Then there is a short exact sequence
\begin{align*}
	0\morphism \underline{\IZ}_X\morphism[\cdot 2\pi\mathrm{i}] \Cc_X\morphism[\exp]\Cc_X^\times\morphism 0
\end{align*}
of sheaves of abelian groups. In general, taking global section doesn't preserve exactness but gives rise to a long exact sequence
\begin{diagram*}
	\node[ob] (0o) at (0,1.5) {$0$};
	\node[ob] (0u) at (0,0) {$0$};
	\node[ob] (H0Z) [right=0.5 of 0u] {$H^0(X,\underline{\IZ}_X)$};
	\node[ob] (H0CX) [right=0.5 of H0Z] {$H^0(X,\Cc_X)$};
	\node[ob] (H0CXx) [right=0.5 of H0CX] {$H^0(X,\Cc_X^\times)$};
	\node[ob] (Z) at (0o -| H0Z) {$\underline{\IZ}_X(X)$};
	\node[ob] (CX) at (0o -| H0CX) {$\Cc_X(X)$};
	\node[ob] (CXx) at (0o -| H0CXx) {$\Cc_X^\times(X)$};
	\node[ob, shift={(0,1.5)}] (H1Z) [right=0.75 of H0CXx] {$H^1(X,\underline{\IZ}_X)$};
	\node[ob] (H1CX) [right=0.5 of H1Z] {$H^1(X,\Cc_X)$};
	\node[ob] (dots) [right=0.5 of H1CX] {$\ldots$};
	\scriptsize
	\draw[->] (0o) -- (Z);
	\draw[->] (0u) -- (H0Z);
	\draw[->] (Z) -- (CX);
	\draw[->] (H0Z) -- (H0CX);
	\draw[->] (CX) -- (CXx);
	\draw[->] (H0CX) -- (H0CXx);
	\draw[->] (CXx) -- (H1Z) node[pos=0.5, above] {$d$};
	\draw[->] (H0CXx) -- (H1Z);
	\draw[->] (H1Z) -- (H1CX);
	\draw[->] (Z) -- (H0Z) node[pos=0.5, above=-0.25ex, sloped] {$\sim$};
	\draw[->] (CX) -- (H0CX) node[pos=0.5, above=-0.25ex, sloped] {$\sim$};
	\draw[->] (CXx) -- (H0CXx) node[pos=0.5, above=-0.25ex, sloped] {$\sim$};
	\draw[->] (H1CX) -- (dots);
\end{diagram*}
in which the $H^k(X,\underline{\IZ}_X)$, $H^k(X,\Cc_X)$, and $H^k(X,\Cc_X^\times)$ are \emph{sheaf cohomology groups}. There is the more general notion of \emph{derived functors} (Grothendieck, T\^{o}hoku paper), but this won't appear in the lecture.

Background in homological algebra is not required safe for cohomology groups of cochain complexes, the long exact cohomology sequence and the following famous lemma.
\begin{lem*}[Five lemma] Given a diagram
	\begin{diagram*}
		\node[ob] (A) at (0,1.5) {$A$};
		\node[ob] (B) at (1.5,1.5) {$B$};
		\node[ob] (C) at (3,1.5) {$C$};
		\node[ob] (D) at (4.5,1.5) {$D$};
		\node[ob] (E) at (6,1.5) {$E$};
		\node[ob] (A') at (0,0) {$A'$};
		\node[ob] (B') at (1.5,0) {$B'$};
		\node[ob] (C') at (3,0) {$C'$};
		\node[ob] (D') at (4.5,0) {$D'$};
		\node[ob] (E') at (6,0) {$E'$};
		\scriptsize
		\draw[->] (A) -- (B);
		\draw[->] (B) -- (C);
		\draw[->] (C) -- (D);
		\draw[->] (D) -- (E);
		\draw[->] (A') -- (B');
		\draw[->] (B') -- (C');
		\draw[->] (C') -- (D');
		\draw[->] (D') -- (E');
		\draw[->] (A) -- (A') node[pos=0.5, left] {$\alpha$};
		\draw[->] (B) -- (B') node[pos=0.5, left] {$\beta$};
		\draw[->] (C) -- (C') node[pos=0.5, left] {$\gamma$};
		\draw[->] (D) -- (D') node[pos=0.5, left] {$\delta$};
		\draw[->] (E) -- (E') node[pos=0.5, left] {$\epsilon$};
	\end{diagram*}
	of (abelian) groups/$R$-modules/etc.\ with exact rows, in which $\beta$ and $\delta$ are isomorphisms, $\alpha$ is an epimorphism and $\epsilon$ a monomorphism, then $\gamma$ is an isomorphism as well.
\end{lem*}
\begin{proof}
	Easy diagram chase.
\end{proof}

\chapter{Cohomology of quasi-coherent sheaves of modules}
\section{Recollection of basic definitions and results}
\begin{defi}[{\cite[Definition~1.5.2 and Definition~1.5.9\itememph{b}]{alggeo1}}]
	\begin{alphanumerate}
		\item A \defemph{prescheme} (Franke uses ``EGA termology'') is a locally ringed space $(X,\Oo_X)$ which locally has the form $\Spec R$ for some rings $R$.
		\item A prescheme $X$ is called a \defemph{scheme}, if, for any prescheme $T$ and any pair of morphisms $T\doublemorphism[a][b]X$, the equalizer $\Eq\Big(T\doublemorphism[a][b]X\Big)$ is a closed subprescheme of $X$.
	\end{alphanumerate}
\end{defi}
\begin{rem*}
	Equivalently, a prescheme $X$ is a scheme iff the diagonal $\Delta\colon X\xrightarrow{(\id_X,\id_X)}X\times X$ is a closed immersion (cf.\ \cite[Fact~1.5.8]{alggeo1}). In other words, schemes are \emph{separated} preschemes
\end{rem*}
\begin{prop}\lbl{prop:SchemesAffineIntersec}
	If $U$ and $V$ are affine open subsets of a scheme $X$, then their intersection $U\cap V$ is again affine (and open of course).
\end{prop}
\begin{proof}
	This was proved in \cite[Proposition~1.5.4]{alggeo1}.
\end{proof}
	Note that open subsets of the form $\Spec(R_f)\simeq \Spec R\setminus V(f)$ form a topology base on $\Spec R$ and that the \emph{saturation} of $\{1,f,f^2,\ldots\}$ (i.e.\ the largest multiplicative subset of $R$ which still gives the same localization) depends only on $\Spec R\setminus V(f)$. Hence, for any $R$-module $M$, the localization $M_f$ depends (up to canonical isomorphism) only on $\Spec R\setminus V(f)$ and $M$. One defines a sheaf of modules $\snake{M}$ on $\Spec R$ as the sheafification of $\Spec (R_f)\mapsto M_f$. Then
	\begin{align*}
		\snake{M}(U)=\left\{(m_\pp)\in\prod_{\pp\in U}M_\pp\st
		\begin{array}{c}
			\text{for every }\qq\in U\text{ there are }f\in R\setminus\qq\text{ and }\mu\in M_f\text{ such that}\\
			m_\pp=\left(\text{image of }\mu\text{ under }M_f\morphism M_\pp\right)\text{ for all }\pp\in\Spec (R_f)
		\end{array}
		\right\}\;.
	\end{align*}
	In the following, I will deviate from Franke's numbering in the lecture, but I simply refuse to call a definition ``remark''.
\begin{defi}[{\cite[Definition~1.4.2]{alggeo1}}]\lbl{def:QuasicoherentSpecR}
	A sheaf of modules $\Mm$ on $\Spec R$ is called \defemph{quasi-coherent} if $\Mm\simeq \snake{M}$ for some $R$-module $M$.
\end{defi}
\begin{defi}[{\cite[Definition~1.5.3]{alggeo1}}]
	A prescheme is called \defemph{quasi-compact} if the underlying topological space is quasi-compact and \defemph{quasi-separated} if the intersection of any two quasi-compact open subsets is quasi-compact.
\end{defi}

\begin{defi}[{\cite[Definition~1.5.4]{alggeo1}}]\lbl{def:Quasicoherent}
	Let $X$ be a prescheme. An $\Oo_X$-module $\Mm$ is called \defemph{quasi-coherent} if it satisfies the following equivalent conditions.
	\begin{alphanumerate}
		\item $X$ may be covered by affine open subsets $U$ such that $\Mm|_U$ is quasi-coherent in the sense of Definition~\reff{def:QuasicoherentSpecR}.
		\item For any affine open $U\subseteq X$, $\Mm|_U$ is quasi-coherent.
		\item For any quasi-compact and quasi-separated open $U\subseteq X$ and $f\in\Oo_X(U)$, the canonical morphism
		\begin{align}\lbl{eq:qcIso}
		\Mm(U)_f\morphism\Mm(U\setminus V(f))
		\end{align}
		(coming from the universal property of localization) is an isomorphism.
		\item The morphism \eqreff{eq:qcIso} is an isomorphism when $U$ is quasi-compact and quasi-separated and injective when $U$ is only quasi-compact.
		\item When $U\subseteq X$ is affine, the canonical morphism
		\begin{align}\lbl{eq:qcIso2}
		\Mm(U)_{\pp_x}\morphism\Mm_x
		\end{align}
		is an isomorphism for all $x\in U$, where $\pp_x=\left\{f\in\Oo_X(U)\st x\in V(f)\right\}$ is the prime ideal in $\Spec \Oo_X(U)$ corresponding to $x$.
	\end{alphanumerate}
\end{defi}
\begin{rem*}
	To be fair: Despite Professor Franke's usual fondness of proving definitions (or rather stating definitions in a way they need a proof), the equivalence of \itememph{a} to \itememph{d} wasn't proposed as a definition in Algebraic Geometry I. 
\end{rem*}
\begin{proof}[Proof of Definition~\reff{def:Quasicoherent}]
	 The equivalence of  \itememph{a} to \itememph{d} was proved in \cite[Proposition~1.5.1]{alggeo1}, but property \itememph{e} is something we haven't seen yet. Recall the \emph{adjunction} (cf.\ Definition~\reff{def:AdjointFunctors})
	\begin{align}\lbl{eq:ROXAdjunction}
		\Hom_{\Oo_{\Spec R}}(\snake{M}, \Nn)\isomorphism\Hom_R(M,\Nn(\Spec R))
	\end{align}
	for $M$ an $R$-module and $\Nn$ a sheaf of $\Oo_{\Spec R}$-modules (cf.\ \cite[Proposition~1.4.3]{alggeo1}). When \eqreff{eq:qcIso2} is an isomorphism for all $x\in U\simeq\Spec R$, it follows that the canonical morphism $\snake{M}\morphism\Mm|_U$ (with $M=\Mm(U)$) coming from \eqreff{eq:ROXAdjunction} is an isomorphism on stalks, hence an isomorphism. This shows \itememph{e} $\Rightarrow$ \itememph{b}.
	
	Conversely, if \itememph{b} holds, then $\snake{M}\morphism\Mm|_U$ (with $M=\Mm(U)$) is an isomorphism for all affine open $U\subseteq X$, hence induces isomorphisms on stalks, which shows \itememph{b} $\Rightarrow$ \itememph{e}. Hence, \itememph{e} is equivalent to the other properties.
\end{proof}

Let $\Aa$ be the category $R\cat{-Mod}$ and $\Bb$ be $\Oo_{\Spec R}\cat{-Mod}$, then the functor $L$ given by $M\mapsto\snake{M}$ and the functor $\Mm\mapsto\Mm(\Spec R)$ are an adjoint pair of functors by \eqreff{eq:ROXAdjunction}. It follows that $L$ commutes with cokernels and coproducts. In particular, the full subcategory $\cat{QCoh}(X)\subseteq\Oo_X\cat{-Mod}$ of quasi-coherent $\Oo_X$-modules is closed under taking cokernels and direct sums for $X=\Spec R$, and by locality of quasi-coherentness this holds for all preschemes $X$.
\begin{defi}[{\cite[Definition~2.1.1 and Definition~2.1.2]{alggeo1}}]\lbl{def:qcqs}
	A morphism $X\morphism[f]Y$ of preschemes is \defemph{quasi-compact} if it satisfies the following equivalent conditions.
	\begin{alphanumerate}
		\item For quasi-compact open $U\subseteq Y$, $f^{-1}(U)$ is quasi-compact.
		\item For affine open $U\subseteq Y$, $f^{-1}(U)$ is quasi-compact.
		\item One can cover $Y$ by affine open $U$ such that $f^{-1}(U)$ is quasi-compact.
	\end{alphanumerate}
	It is called \defemph{quasi-separated} if it satisfies the following equivalent conditions.
	\begin{alphanumerate}\setcounter{enumi}{3}
		\item For an open quasi-separated $U\subseteq Y$, $f^{-1}(U)$ is quasi-separated again.
		\item For affine open subsets $U\subseteq Y$, $f^{-1}(U)$ is quasi-separated.
		\item It is possible to cover $Y$ by affine open $U$ such that $f^{-1}(U)$ is quasi-separated.
	\end{alphanumerate}
\end{defi}
\begin{proof}
	Equivalence was proved in \cite[Fact~2.1.1]{alggeo1} for quasi-compactness and \cite[Lemma~2.1.1]{alggeo1} for quasi-separatedness.
\end{proof}

\begin{prop}\lbl{prop:PushforwardOfQcIsQc}
	If $X\morphism[f]Y$ is quasi-compact and quasi-separated morphism of preschemes and $\Mm\in\Ob(\cat{QCoh}(X))$, then $f_*\Mm\in\Ob\left(\cat{QCoh}(Y)\right)$.
\end{prop}
\begin{proof}
	This is \cite[Proposition~1.5.2\itememph{b}]{alggeo1}.
\end{proof}
\begin{prop}
	\begin{alphanumerate}
		\item \lbl{prop:QCohHasKerCoker}The full subcategory $\cat{QCoh}(X)\subseteq\Oo_X\cat{-Mod}$ of quasi-coherent sheaves of $\Oo_X$-modules on a prescheme $X$ is closed under taking kernels and cokernels of morphisms and under taking (finite) direct sums.
		\item If $\Mm$ is a quasi-coherent $\Oo_X$-module and $U\subseteq X$ open, then $\Mm|_U\in\Ob\left(\cat{QCoh}(U)\right)$.
	\end{alphanumerate}
\end{prop}
\begin{proof}
	Part \itememph{a}. For cokernels and finite direct sums (which are finite coproducts since $\Oo_X\cat{-Mod}$ is an abelian category by Proposition~\reff{prop:RmodAbelian}), consider the case $X=\Spec R$ first. Then $R\cat{-Mod}\morphism[L]\Oo_X\cat{-Mod}$, $M\mapsto\snake{M}$ and $\Oo_X\cat{-Mod}\morphism R\cat{-Mod}$, $\Mm\mapsto\Mm(X)$ are adjoint functors by \eqreff{eq:ROXAdjunction}. By Remark~\reff{rem:AdjunctionPreservesStuff}, $L$ preserves cokernels and coproducts. By locality of quasi-coherentness, this follows for all preschemes $X$.
	
	Closedness under taking kernels was proved in \cite[Fact~1.5.3]{alggeo1}. It's worth pointing out that in fact, the proof given there shows that $M\mapsto\snake{M}$ preserves kernels as well. 
	
	Part \itememph{b} follows immediately from (e.g.) Definition~\reff{def:Quasicoherent}\itememph{b}.
\end{proof}
\begin{cor}\lbl{cor:kerAndCokerForQcModules}
	Let $X$ be a prescheme, $\Mm\morphism[f]\Nn$ a morphism of quasi-coherent $\Oo_X$-modules and $U\subseteq X$ open, then
	\begin{align*}
		\ker\Big(\Mm(U)\morphism[f]\Nn(U)\Big)\simeq\ker(f)(U)\;.
	\end{align*}
	If $U$ is, in addition, affine, then
	\begin{align*}
		\coker\Big(\Mm(U)\morphism[f]\Nn(U)\Big)\simeq\coker(f)(X)\;.
	\end{align*}
\end{cor}
\begin{proof}
	The first assertion holds by our explicit construction of $\ker(f)$ in Lemma~\reff{lem:RModHasKernels}. For the second one, we may assume $X=U=\Spec R$. Denoting $M=\Mm(X)$, $N=\Nn(X)$, then
	\begin{align*}
		\coker\Big(\Mm\morphism[f]\Nn\Big)=\coker\Big(\snake{M}\morphism[f]\snake{N}\Big)\simeq\Big(\coker\Big(M\morphism[f]N\Big)\Big)^\sim
	\end{align*}
	as $M\mapsto\snake{M}$ preserves cokernels.
\end{proof}
\begin{cor}\lbl{cor:GlobalSectionsExactOnAffine}
	Let $0\morphism\Mm'\morphism\Mm\morphism\Mm''\morphism0$ be a short exact sequence of quasi-coherent $\Oo_X$-modules on a prescheme $X$ and $U\subseteq X$ be affine open, then 
	\begin{align*}
		0\morphism\Mm'(U)\morphism\Mm(U)\morphism\Mm''(U)\morphism 0
	\end{align*}
	is exact as well.
\end{cor}
\begin{proof}
	Follows from Corollary~\reff{cor:kerAndCokerForQcModules}.
\end{proof}
\begin{rem*}
	It turns out to be sufficient to assume that two of the above three sheaves $\Mm',\Mm,\Mm''$ are quasi-coherent. Indeed, we proved in Proposition~\reff{prop:QCohHasKerCoker} that kernels and cokernels of morphisms between quasi-coherent sheaves are quasi-coherent again, so the only case in question is where $\Mm$ is not required to be quasi-coherent. This case, however, will be treated by cohomological methods.
\end{rem*}
Our plan is to associate to any quasi-coherent $\Oo_X$-module on a scheme $X$ \emph{cohomology groups} $H^i(X,\Mm)$ such that
\begin{itemize}
	\item $H^0(X,\Mm)\simeq\Mm(X)$.
	\item when $0\morphism\Mm'\morphism\Mm\morphism\Mm''\morphism0$ is an exact sequence of $\Oo_X$-modules, we have a canonical long exact sequence
	\begin{multline*}
		0\morphism H^0(X,\Mm')\morphism H^0(X,\Mm)\morphism H^0(X,\Mm'')\\
		\morphism[d] H^1(X,\Mm')\morphism H^1(X,\Mm)\morphism H^1(X,\Mm'')\morphism[d]H^2(X,\Mm')\morphism\ldots
	\end{multline*}
\end{itemize}
But before we do this, we to introduce the notion of \emph{coherent} $\Oo_X$-modules.
\begin{prop}
	If $X$ is a prescheme, associating to (the isomorphism class of) a closed embedding $Y\morphism[i]X$ the sheaf of ideals $\Jj=\ker\Big(\Oo_X\morphism[i^*]i_*\Oo_X\Big)$ gives a bijection between the set of closed subpreschemes of $X$ and the quasi-coherent sheaves of ideals in $\Oo_X$.
\end{prop}
\begin{proof}
	This is \cite[Proposition~1.5.3]{alggeo1}.
\end{proof}
\begin{lem}\lbl{lem:locFinGen}
	For a quasi-coherent $\Oo_X$-module $\Mm$ on a prescheme $X$, the following conditions are equivalent.
	\begin{alphanumerate}
		\item For any affine open $U\subseteq X$, $\Mm(U)$ is a finitely generated $\Oo_X(U)$-module.
		\item It is possible to cover $X$ by affine open subsets $U\subseteq X$, for which $\Mm(U)$ is a finitely generated $\Oo_X(U)$-module.
	\end{alphanumerate}
\end{lem}
\begin{proof}
	This will follow from Lemma~\reff{lem:localProperty} and Lemma~\reff{lem:FinitelyGeneratedLocal} below.
\end{proof}
\begin{lem}\lbl{lem:localProperty}
	Let $\Pp$ be a property of affine open subsets of a prescheme $X$ such that
	\begin{alphanumerate}
		\item[\itememph{\alpha}] If $U\subseteq X$ is affine and $f\in\Oo_X(U)$, then $\Pp(U)$ implies $\Pp(U\setminus V(f))$.
		\item[\itememph{\beta}] If $U$ is affine and $f_1,\ldots,f_n\in\Oo_X(U)$ are such that $\bigcap_{i=1}^nV(f_i)=\emptyset$ and such that $\Pp(U\setminus V(f_i))$ holds for all $i=1,\ldots,n$, then $\Pp(U)$ holds.
	\end{alphanumerate}
	Then the following assertions about $X$ are equivalent.
	\begin{alphanumerate}
		\item If $U\subseteq X$ is affine open, $\Pp(U)$ holds.
		\item $X$ may be covered by affine open $U$ for which $\Pp(U)$ holds.
	\end{alphanumerate}
\end{lem}
\begin{proof}
	We proved this in \cite[Lemma~2.2.2]{alggeo1}.
\end{proof}
\begin{lem}
	\begin{alphanumerate}
		\item \lbl{lem:FinitelyGeneratedLocal}If $M$ is a finitely generated $R$-module, then $M_f$ is a finitely generated $R_f$-module.
		\item If $M$ is an $R$-module and $f_1,\ldots,f_n\in R$ such that $\bigcap_{i=1}^nV(f_i)=\emptyset$ in $\Spec R$ and such that $M_{f_i}$ is finitely generated over $R_{f_i}$, then $M$ is finitely generated over $R$.
	\end{alphanumerate}
\end{lem}
\begin{proof}
	Part \itememph{a} is trivial, as the images of $R$-generators of $M$ in $M_f$ generate it as an $R_f$-module.
	
	Now for part \itememph{b}. As $M_{f_i}$ is finitely generated over $R_{f_i}$, there are $k\in\IN$ and $m_{i,j}\in M$, $j=1,\ldots,N_i$ such that $m_{i,j}f^{-k}$ generate $M_{f_i}$ over $R_{f_i}$ (as there are only finitely many generators, we can choose a common exponent $k$ for all of them). Then also the $m_{i,j}$ generate $M_{f_i}$ since $f_i$ is a unit in $R_{f_i}$. We claim that the $\left\{m_{i,j}\st i=1,\ldots, n\text{ and }j=1,\ldots, N_i\right\}$ generate $M$ as an $R$-module. Indeed, let $m\in M$, then
	\begin{align*}
		m=\sum_{j=1}^{N_i}\frac{r_{i,j}}{f_i^\ell}m_{i,j}\quad\text{in }M_{f_i}\;,
	\end{align*}
	where $r_{i,j}\in R$ and $\ell\in\IN$ (again, we can choose a common exponent $\ell$). Then there is some $\ell'\in \IN$ such that
	\begin{align*}
		f_i^{\ell+\ell'}m=\sum_{j=1}^{N_i}r_{i,j}f_i^{\ell'}m_{i,j}\quad\text{in }M\;.
	\end{align*}
	Replacing $\ell$ by $\ell+\ell'$ and $r_{i,j}$ by $f_i^{\ell'}r_{i,j}$ we may assume $\ell'=0$, i.e.
	\begin{align*}
		f_i^{\ell}m=\sum_{j=1}^{N_i}r_{i,j}m_{i,j}\quad\text{in }M\;.
	\end{align*}
	We now have $\bigcap_{i=1}^nV(f_i^\ell)=\bigcap_{i=1}^nV(f_i)=\emptyset$, hence the ideal generated by the $f_i^\ell$ is $R$ and we thus find $g_1,\ldots,g_n\in R$ such that $\sum_{i=1}^{n}f_i^\ell g_i=1$ in $R$. It follows that
	\begin{align*}
		m=\sum_{i=1}^nf_i^\ell g_im=\sum_{i=1}^{n}\sum_{j=1}^{N_i}r_{i,j}g_im_{i,j}
	\end{align*}
	is an element of the submodule generated by the $m_{i,j}$.
\end{proof}
\begin{defi}
	\begin{alphanumerate}
		\item \lbl{def:locFinGenerated}We call a quasi-coherent $\Oo_X$-module \defemph{locally finitely generated} it it satisfies the equivalent conditions from Lemma~\reff{lem:locFinGen}.
		\item When $X$ is locally Noetherian (cf.\ \cite[Definition~2.2.2]{alggeo1}), an $\Oo_X$-module is called \defemph{coherent} if it is quasi-coherent and locally finitely generated.
	\end{alphanumerate}
\end{defi}
\begin{rem*}
	There is a general definition of \emph{coherent} sheaves of modules on arbitrary ringed spaces, which in the case of a locally Noetherian prescheme is equivalent to the above.
\end{rem*}
%Composite letter `\textasciicaron+C'(hyperref) not defined in PD1 encoding,(hyperref) removing `\textasciicaron'
\section{\vC ech cohomology}
Let $\Uu\colon X=\bigcup_{i\in I}U_i$ be an open cover of a topological space $X$. In the following, we will use the convention
\begin{align}\lbl{eq:capConvention}
U_{i_0,\ldots,i_n}=\bigcap_{k=0}^nU_{i_k}\;.
\end{align}
\begin{defi}\lbl{def:cech}
	For an open cover $\Uu$ of a topological space $X$ (e.g., a prescheme) and $\Mm$ a presheaf of abelian groups (e.g., a quasi-coherent $\Oo_X$-module) on $X$ the \defemph{\vC ech complex} $\check{C}^\bullet (\Uu,\Mm)$ is the cochain complex defined as follows. Let
	\begin{align*}
		\check{C}^n(\Uu,\Mm)\coloneqq\prod_{(i_0,\ldots,i_n)\in I^{n+1}}\Mm(U_{i_0,\ldots,i_n})\;.
	\end{align*}
	Let the elements of $\check{C}^n(\Uu,\Mm)$ be denoted $\psi=(\psi_{i_0,\ldots,i_n})_{(i_0,\ldots,i_n)\in I^{n+1}}$. The differentials $\check{C}^n(\Uu,\Mm)\morphism[\hacek{d}^n]\check{C}^{n+1}(\Uu,\Mm)$ are defined by
	\begin{align*}
		(\hacek{d}^n\psi)_{i_0,\ldots,i_{n+1}}=\sum_{j=0}^{n+1}(-1)^j\psi_{i_0,\ldots,\hat{i}_j,\ldots,i_{n+1}}|_{U_{i_0,\ldots,i_{n+1}}}
	\end{align*}
	where $\hat{i}_j$ denotes the omission of the index $i_j$. For instance,
	\begin{align*}
		(\hacek{d}^0\psi)_{i,j}=\psi_j|_{U_{i,j}}-\psi_i|_{U_{i,j}}\quad\text{and}\quad (\hacek{d}^1\psi)_{i,j,k}=\psi_{j,k}|_{U_{i,j,k}}-\psi_{i,k}|_{U_{i,j,k}}+\psi_{i,j}|_{U_{i,j,k}}\;.
	\end{align*}
	The \defemph{\vC ech cohomology} $\check{H}^\bullet (\Uu,\Mm)$ is defined as the cohomology of the \vC ech complex, i.e.,
	\begin{align*}
		\check{H}^i(\Uu,\Mm)=H^i\left(\check{C}^\bullet (\Uu,\Mm)\right)\;.
	\end{align*}	
\end{defi}
To see that $\check{C}^\bullet (\Uu,\Mm)$ is indeed a cochain complex, we need to prove $\hacek{d}^2=0$ -- and we won't do this in a remark!
\begin{proof}[Proof of Definition~\reff{def:cech}]
	For $\ell=0,\ldots,n+1$ let $\check{C}^n(\Uu,\Mm)\morphism[d_\ell]\check{C}^{n+1}(\Uu,\Mm)$ be given by
	\begin{align*}
		(d_\ell\psi)_{i_0,\ldots,i_{n+1}}=\psi_{i_0,\ldots,\hat{i}_\ell,\ldots,i_{n+1}}|_{U_{i_0,\ldots,i_{n+1}}}\;.
	\end{align*}
	Again, $\hat{i}_\ell$ denotes the omission of the index $i_\ell$.
	
	\emph{Step 1.} We prove that
	\begin{align}\lbl{eq:simplicialStuff}
		d_md_\ell=d_{\ell+1}d_m\quad\text{when }\ell\geq m\;.
	\end{align}
	Indeed, we have $(d_md_\ell\psi)_{\boldsymbol{i}}=\psi_{\boldsymbol{j}}|_{U_{\boldsymbol{i}}}$, where $\boldsymbol{j}$ is obtained from $\boldsymbol{i}$ by omitting the indices $i_\ell$ and $i_m$ when $\ell<m$ and the indices $i_{\ell+1}$ and $i_m$ when $\ell\geq m$. The assertion follows.
	
	\emph{Step 2.} We prove the following. Let $C^\bullet $ be any family of abelian groups (or objects of an abelian category) and $C^n\morphism[d_\ell]C^{n+1}$ morphisms for $\ell=0,\ldots,n+1$. Suppose that $C^n=0$ for $n<0$ and that \eqreff{eq:simplicialStuff} holds. Then $d^\bullet$ with
	\begin{align*}
		d^n=\sum_{i=0}^{n+1}(-1)^id_i
	\end{align*}
	satisfies $d^{n+1}d^n=0$. Indeed,
	\begin{align*}
		d^{n+1}d^n&=\sum_{m=0}^{n+2}\sum_{\ell=0}^{n+1}(-1)^{\ell+m}d_md_\ell=\sum_{m=0}^{n+2}\sum_{\ell=0}^{m-1}(-1)^{\ell+m}d_md_\ell+\sum_{m=0}^{n+2}\sum_{\ell=m}^{n+1}(-1)^{\ell+m}d_md_\ell\\
		&=\sum_{m=0}^{n+2}\sum_{\ell=0}^{m-1}(-1)^{\ell+m}d_md_\ell+\sum_{m=0}^{n+1}\sum_{\ell=m}^{n+1}(-1)^{\ell+m}d_{\ell+1}d_m\\
		&=\sum_{i>j}(-1)^{i+j}d_id_j+\sum_{i>j}(-1)^{i+j-1}d_id_j=0\;,
	\end{align*}
	as required.
\end{proof}
\begin{rem*}
	Our program is to show that $\check{H}^\bullet (\Uu,\Mm)$ is independent of $\Uu$ and has the desired properties, when $X$ is a scheme, $\Mm\in\Ob(\cat{QCoh}(X))$ and $\Uu$ is an affine open cover.
\end{rem*}
\begin{rem*}
	For instance, the cohomology of $\IP_R^1=\Proj(R[X_0,X_1])$ can be calculated using the affine open cover 
	\begin{align*}
		U_i=\IP_R^1\setminus V(X_i)\simeq \Spec \left(R[X_0,X_1]_{X_i}\right)_0\simeq \Spec R[t_i]\quad\text{where }t_i=
		\begin{cases}
			X_1\cdot X_0^{-1} & \text{if }i=0\\
			X_0\cdot X_1^{-1} & \text{if }i=1
		\end{cases}\;.
	\end{align*}
	Unfortunately, calculations become complicated by the fact that there are infinitely many non-zero terms in $\check{C}^\bullet (\Uu,\Mm)$.
\end{rem*}
	Let $\check{C}_\alt^n(\Uu,\Mm)\subseteq\check{C}^n(\Uu,\Mm)$ be the subgroup containing all $\psi\in\check{C}^n(\Uu,\Mm)$ such that
	\begin{align*}
		\psi_{i_{\pi(0)},,\ldots,i_{\pi(n)}}=\sgn(\pi)\psi_{i_0,\ldots,i_n}\in \Mm(U_{i_0,\ldots,i_n})\quad\text{and}\quad\psi_{i_0,\ldots,i_{n-1},i_{n-1}}=0\in\Mm(U_{i_0,\ldots,i_{n-1}})
	\end{align*}
	for all permutations $\pi\in\SS_n$. Note that $U_{i_0,\ldots,i_n}=U_{i_{\pi(0)},\ldots,i_{\pi(n)}}$ as permuting indices doesn't change intersections, so the first property makes sense. Also note that both properties together imply that $\psi_{i_0,\ldots,i_n}=0$ whenever $(i_0,\ldots,i_n)$ contains a repeated index.
	
	\begin{defi}\lbl{def:alternatingCech}
		$\check{C}_\alt^\bullet (\Uu,\Mm)\subseteq\check{C}^\bullet (\Uu,\Mm)$ is a subcomplex, called the \defemph{alternating \vC ech complex}.
	\end{defi}
	\begin{proof}
		We need to confirm that the differential $\hacek{d}^n$ maps $\check{C}_\alt^n(\Uu,\Mm)$ into $\check{C}_\alt^{n+1}(\Uu,\Mm)$. To do this, define \emph{codegeneracy maps} 
		\begin{align*}
			\check{C}^n(\Uu,\Mm)\morphism[s_\ell]\check{C}^{n-1}(\Uu,\Mm)\;,\quad (s_\ell\psi)_{i_0,\ldots,i_{n-1}}=\psi_{i_0,\ldots,i_\ell,i_\ell,\ldots,i_{n-1}}\quad\text{for }\ell=0,\ldots,n-1
		\end{align*}
		(i.e., $s_\ell$ repeats the $\ell\ordinalth$ index) as well as \emph{transposition maps}
		\begin{align*}
			\check{C}^n(\Uu,\Mm)\morphism[t_\ell]\check{C}^n(\Uu,\Mm)\;,\quad (t_\ell\psi)_{i_0,\ldots,i_n}=\psi_{i_0,\ldots,i_{\ell-1},i_{\ell+1},i_\ell,i_{\ell+2},\ldots,i_{n-1}}\quad\text{for }\ell=0,\ldots,n-1
		\end{align*}
		(i.e., $t_\ell$ swaps the $\ell\ordinalth$ and $(\ell+1)\ordinalst$ index). As any permutation may be expressed as a composition of elementary transpositions, $\check{C}_\alt^n(\Uu,\Mm)\subseteq\check{C}^n(\Uu,\Mm)$ is given by the relations
		\begin{align*}
			s_\ell\psi=0\quad\text{and}\quad t_\ell\psi=-\psi\quad\text{for }\ell=0,\ldots,n-1
		\end{align*}
		So what we need to check to confirm that $\check{C}_\alt^\bullet (\Uu,\Mm)$ is indeed a subcomplex of $\check{C}^\bullet (\Uu,\Mm)$ is that the above relations are preserved by the differential $\hacek{d}$.
		
		One may easily check the relations
		\begin{align}\lbl{eq:cosimplicialDegeneracies}
			s_\ell d_i=\begin{cases}
				d_is_{\ell-1} & \text{if }i<\ell\\
				\id &  \text{if }i=\ell\text{ or }i=\ell+1\\
				d_{i-1}s_\ell &  \text{if }i>\ell+1
			\end{cases}
		\end{align}
		and
		\begin{align}\lbl{eq:transpositions}
			t_\ell d_j=\begin{cases}
				d_jt_{\ell} & \text{if }\ell<j-1\\
				d_\ell &  \text{if }\ell=j-1\\
				d_{\ell+1} &  \text{if }\ell=j\\
				d_jt_{\ell-1} & \text{if }\ell>j
			\end{cases}\;.
		\end{align}
		Now let $\psi\in\check{C}^n(\Uu,\Mm)$ such that $t_j\psi=-\psi$ for all $j=0,\ldots,n-1$. Using \eqreff{eq:transpositions}, we get
		\begin{align*}
			t_\ell\hacek{d}\psi&=\sum_{j=0}^{\ell-1}(-1)^jt_\ell d_j\psi+(-1)^\ell t_\ell d_\ell\psi+(-1)^{\ell+1} t_\ell d_{\ell+1}\psi+\sum_{j=\ell+2}^n(-1)^jt_\ell d_j\psi\\
			&=\sum_{j=0}^{\ell-1}(-1)^jd_jt_{\ell-1}\psi+(-1)^\ell d_{\ell+1}\psi+(-1)^{\ell+1} d_\ell\psi+\sum_{j=\ell+2}^{n}(-1)^j d_jt_\ell\psi\\
			&=-\sum_{j=0}^{\ell-1}(-1)^jd_j\psi-(-1)^\ell d_\ell\psi-(-1)^{\ell+1} d_{\ell+1}\psi-\sum_{j=\ell+2}^{n}(-1)^j d_j\psi\\
			&=-\hacek{d}\psi\;.
		\end{align*}
		Similarly, one can check that $s_\ell\hacek{d}\psi=0$ when $s_j\psi=0$ for all $j=0,1,\ldots,n-1$. This shows that $\hacek{d}$ restricts to a differential on $\check{C}_\alt
		^n(\Uu,\Mm)$, as required.
	\end{proof}
	It will eventually turn out that the cohomology groups $\check{H}_\alt^i(\Uu,\Mm)=H^i\left(\check{C}_\alt^\bullet (\Uu,\Mm)\right)$ obtained from the alternating \vC ech complex are the same as the regular \vC ech cohomology groups $\check{H}^i(\Uu,\Mm)$.
\begin{rem}
	A \emph{cosimplicial object} of a category $\Aa$ is a sequence of objects $(X^n)_{n\geq 0}$ with morphisms $d_j\colon X^n\morphism X^{n+1}$ for $j=0,\ldots,n+1$ satisfying \eqreff{eq:simplicialStuff} and $s_j\colon X^n\morphism X^{n-1}$ for $j=0,\ldots,n$ satisfying a version of \eqreff{eq:simplicialStuff} together with \eqreff{eq:cosimplicialDegeneracies}. In other words, a cosimplicial object is a covariant functor from the \emph{simplex category} $\Delta$ to $\Aa$.
	
	There is a \emph{Dold--Puppe correspondence} between cochain complexes concentrated in nonnegative degrees and cosimplicial objects of an abelian category.  
\end{rem}
\begin{example}
	\begin{alphanumerate}
		\item \lbl{ex:FirstCechComputations}By the sheaf axiom, 
		\begin{align*}
			\Mm(X)&\simeq\left\{(m_i)_{i\in I}\in\prod_{i\in I}\Mm(U_i)\st m_i|_{U_{i,j}}=m_j|_{U_{i,j}}\right\}\\
			&=\ker\Big(\check{C}^0(\Uu,\Mm)\morphism[\hacek{d}^0]\check{C}^1(\Uu,\Mm)\Big)\simeq \check{H}^0(\Uu,\Mm)\\
			&=\ker\Big(\check{C}_\alt^0(\Uu,\Mm)\morphism[\hacek{d}^0]\check{C}_\alt^1(\Uu,\Mm)\Big)\simeq \check{H}_\alt^0(\Uu,\Mm)
		\end{align*}
		\item For the trivial cover $\Uu_0\colon X=X$, the \vC ech complex $\check{C}^\bullet (\Uu_0,\Mm)$ has the form
		\begin{align*}
			\Mm(X)\morphism 0\morphism 0\morphism0\morphism\ldots
		\end{align*}
		and the alternating \vC ech complex $\check{C}_\alt^\bullet (\Uu_0,\Mm)$ looks like
		\begin{align*}
			\Mm(X)\morphism[0]\Mm(X)\morphism[\id]\Mm(X)\morphism[0]\ldots\;,
		\end{align*}
		so
		\begin{align*}
			\check{H}^n(\Uu_0,\Mm)=\check{H}_\alt^n(\Uu_0,\Mm)=\begin{cases}
				\Mm(X) & \text{if }n=0\\
				0 & \text{else}
			\end{cases}\;.
		\end{align*}
		\item If $\Mm$ and $\Nn$ are presheaves of modules on $X$ and $\Uu$ is an open cover of $X$, then 
		\begin{align*}
			\check{C}^\bullet (\Uu,\Mm\oplus\Nn)=\check{C}^\bullet (\Uu,\Mm)\oplus\check{C}^\bullet (\Uu,\Nn)
		\end{align*}
		and, more general, 
		\begin{align*}
			\check{C}^\bullet \bigg(\Uu,\prod_{i\in I}\Mm_i\bigg)=\prod_{i\in I}\check{C}^\bullet (\Uu,\Mm_i)\;.
		\end{align*}
		The same holds for $\check{C}_\alt^\bullet (\Uu,-)$.
	\end{alphanumerate}
\end{example}
If $\Uu\colon X=\bigcup_{i\in I}U_i$ is a cover of a scheme $X$ by affine open subsets $U_i$, then all intersections $U_{i_0,\ldots,i_n}$ are affine again by Proposition~\reff{prop:SchemesAffineIntersec}. If $0\morphism\Mm'\morphism\Mm\morphism\Mm''\morphism 0$ is a short exact sequence in $\cat{QCoh}(X)$, Corollary~\reff{cor:GlobalSectionsExactOnAffine} provides short exact sequences
\begin{align*}
	0\morphism \check{C}^\bullet (\Uu,\Mm')\morphism \check{C}^\bullet (\Uu,\Mm)\morphism \check{C}^\bullet (\Uu,\Mm'')\morphism 0
\end{align*}
and
\begin{align*}
	0\morphism \check{C}_\alt^\bullet (\Uu,\Mm')\morphism \check{C}_\alt^\bullet (\Uu,\Mm)\morphism \check{C}_\alt^\bullet (\Uu,\Mm'')\morphism 0
\end{align*}
of chain complexes. For $\check{C}^\bullet (\Uu,-)$ this is immediate from Definition~\reff{def:cech} and from the fact that products of short exact sequences are short exact again. To see this for $\check{C}_\alt^\bullet (\Uu,-)$, choose any linear ordering of $I$ and note that 
\begin{align*}
	\check{C}_\alt^n(\Uu,\Mm)\simeq\prod_{i_0<\ldots<i_n\in I}\Mm(U_{i_0,\ldots,i_n})\;,
\end{align*}
then the same argument may be applied.

Taking long exact cohomology sequences we just proved
\begin{prop}\lbl{prop:LongExactCechSequence}
	If $\Uu$ is an affine open cover of a scheme $X$ and $0\morphism \Mm'\morphism\Mm\morphism\Mm''\morphism 0$ a short exact sequence of quasi-coherent sheaves of $\Oo_X$-modules, then there is a long exact cohomology sequence
	\begin{diagram*}
		\node[ob] (0o) at (0,1.5) {$0$};
		\node[ob] (0u) at (0,0) {$0$};
		\node[ob] (H0Z) [right=0.5 of 0o] {$\check{H}^0(\Uu,\Mm')$};
		\node[ob] (H0CX) [right=0.5 of H0Z] {$\check{H}^0(\Uu,\Mm)$};
		\node[ob] (H0CXx) [right=0.5 of H0CX] {$\check{H}^0(\Uu,\Mm'')$};
		\node[ob] (Z) at (0u -| H0Z) {$\Mm'(X)$};
		\node[ob] (CX) at (0u -| H0CX) {$\Mm(X)$};
		\node[ob] (CXx) at (0u -| H0CXx) {$\Mm''(X)$};
		\node[ob] (H1Z) [right=0.75 of H0CXx] {$\check{H}^1(\Uu,\Mm')$};
		\node[ob] (H1CX) [right=0.5 of H1Z] {$\check{H}^1(\Uu,\Mm)$};
		\node[ob] (dots) [right=0.5 of H1CX] {$\ldots$};
		\scriptsize
		\draw[->] (0u) -- (Z);
		\draw[->] (0o) -- (H0Z);
		\draw[->] (Z) -- (CX);
		\draw[->] (H0Z) -- (H0CX);
		\draw[->] (CX) -- (CXx);
		\draw[->] (H0CX) -- (H0CXx);
		\draw[->] (H0CXx) -- (H1Z)node[pos=0.5, above] {$d$};;
		\draw[->] (H1Z) -- (H1CX);
		\draw[->] (H0Z) -- (Z) node[pos=0.5, above=-0.25ex, sloped] {$\sim$};
		\draw[->] (H0CX) -- (CX) node[pos=0.5, above=-0.25ex, sloped] {$\sim$};
		\draw[->] (H0CXx) -- (CXx) node[pos=0.5, above=-0.25ex, sloped] {$\sim$};
		\draw[->] (H1CX) -- (dots);
	\end{diagram*}
	and similar for $\check{H}_\alt^\bullet (\Uu,-)$.
\end{prop}
\begin{rem}
	For arbitrary preschemes, the situation is more difficult (cf.\ Thomason, \emph{The Grothendieck Festschrift}).
\end{rem}
\begin{defi}
	An open cover $\Vv\colon X=\bigcup_{j\in J}V_j$ is a \defemph{refinement} of $\Uu\colon X=\bigcup_{i\in I}U_i$ if there is a map $v\colon J\morphism I$ such that $V_j\subseteq U_{v(j)}$ for all $j\in J$. Such a $v$ is called a \defemph{refinement map} for the pair $(\Vv,\Uu)$. 
\end{defi}
Note that Professor Franke isn't sure whether \emph{refinement map} is the usual term. Assuming the axiom of choice, the existence of $v$ is equivalent to every $V_j$ being contained in some $U_i$. 

A refinement map $v$ induces a morphism
\begin{align*}
	\check{C}^\bullet (\Uu,\Mm)\morphism[v^*]\check{C}^\bullet (\Vv,\Mm)\;,\quad(v^n\psi)_{j_0,\ldots,j_n}=\psi_{v(j_0),\ldots,v(j_n)}|_{V_{j_0,\ldots,j_n}}\quad\text{for }\psi\in\check{C}^n(\Uu,\Mm)\;.
\end{align*}
Clearly, $v^*$ commutes with the $d_j$, $s_j$, and $t_j$, hence restricts to a morphism of chain complexes $\check{C}_\alt^\bullet (\Uu,\Mm)\morphism[v^*]\check{C}_\alt^\bullet (\Vv,\Mm)$.
\begin{lem}
	\begin{alphanumerate}
		\item \lbl{lem:refinementPullback}A refinement $\Ww$ of a refinement $\Vv$ of $\Uu$ is a refinement of $\Uu$ and if $v$ and $w$ are associated refinement maps for $(\Vv,\Uu)$ and $(\Ww,\Uu)$, then $vw$ is a refinement map for $(\Ww,\Uu)$. Moreover, $(vw)^*=w^*v^*$ and the identity $\id_I$ is a refinement map for $(\Uu,\Uu)$ and $\id_I^*=\id_{\check{C}^*(\Uu,-)}$.
		\item Two arbitrary open covers have a common refinement. When $X$ is a prescheme, this common refinement can be chosen affine.
		\item If $v_1,v_2\colon J\morphism I$ are two refinement maps for $(\Vv,\Uu)$, then $v_1^*$ and $v_2^*$ induce the same morphism on \vC ech cohomology.
	\end{alphanumerate}
\end{lem}
\begin{proof}
	Part \itememph{a} is obvious. For \itememph{b}, let $\Uu$ and $\Vv$ be open covers of $X$. Then $X=\bigcup_{(i,j)\in I\times J}U_i\cap V_j$ is a common refinement. When $X$ is a prescheme, we may cover each $U_i\cap V_j$ by affine open subsets and thus obtain a common affine refinement of $\Uu$ and $\Vv$.
	
	Now for part \itememph{c}. Define maps $h^n\colon\check{C}^n(\Uu,\Mm)\morphism\check{C}^{n-1}(\Vv,\Mm)$ as follows: We put 
	\begin{align*}
		h^n=\sum_{\ell=0}^{n-1}(-1)^\ell h_\ell\;,
	\end{align*}
	where $h_\ell\colon\check{C}^n(\Uu,\Mm)\morphism\check{C}^{n-1}(\Vv,\Mm)$ is given by
	\begin{align*}
		(h_\ell\psi)_{j_0,\ldots,j_{n-1}}=\psi_{v_1(j_0),\ldots,v_1(j_\ell),v_2(j_\ell),\ldots,v_2(j_{n-1})}|_{V_{j_0,\ldots,j_{n-1}}}\quad\text{for }\psi\in\check{C}^n(\Uu,\Mm)\;.
	\end{align*}
	Then it's a straightforward but tedious check that the following relations hold:
	\begin{align}\lbl{eq:SHIT}
		h_\ell d_k=\begin{cases}
			d_kh_{\ell-1} & \text{if }0\leq k<\ell\\
			h_{\ell-1}d_{k} & \text{if }0< k=\ell\\
			v_2^n & \text{if }0=k=\ell\\
			h_{\ell+1}d_{k} & \text{if }k=\ell+1<n\\
			v_1^n & \text{if }k=\ell+1=n\\
			d_{k-1}h_\ell & \text{if }k>\ell+1
		\end{cases}\;.
	\end{align}
	(I tried my best to get the indices right and I claim my hit ratio is way higher than Franke's). Our goal is to show 
	\begin{align*}
		\hacek{d}^{n-1}h^n+h^{n+1}\hacek{d}^n=v_2^n-v_1^n\;,
	\end{align*}
	for then $h$ is a cochain homotopy between $v_1^*$ and $v_2^*$ and it's a well-known fact from homological algebra that cochain homotopic maps induce the same morphisms on cohomology. Indeed, using \eqreff{eq:SHIT} we get
	\begin{align*}
		h^{n+1}\hacek{d}^n&=\sum_{\ell=0}^{n}\sum_{k=0}^{n+1}(-1)^{\ell+k}h_\ell d_k\\
		&=\begin{multlined}[t]
			\sum_{\ell=0}^{n}\sum_{k=0}^{\ell-1}(-1)^{\ell+k}h_\ell d_k+(-1)^0h_0d_0+\sum_{\ell=1}^{n}(-1)^{2\ell}h_\ell d_\ell\\
			+\sum_{\ell=0}^{n-1}(-1)^{2\ell+1}h_\ell d_{\ell+1}+(-1)^{2n+1}h_nd_{n+1}+\sum_{\ell=0}^{n}\sum_{k=\ell+2}^{n+1}(-1)^{\ell+k}h_\ell d_k
		\end{multlined}\\
		&=\begin{multlined}[t]
			\sum_{\ell=0}^{n}\sum_{k=0}^{\ell-1}(-1)^{\ell+k}d_k h_{\ell-1}+v_2^n+\sum_{\ell=1}^{n}h_{\ell-1} d_\ell\\
			-\sum_{\ell=0}^{n-1}h_\ell d_{\ell+1}-v_1^n+\sum_{\ell=0}^{n}\sum_{k=\ell+2}^{n+1}(-1)^{\ell+k}d_{k-1} h_{\ell}
		\end{multlined}\\
		&=\sum_{\ell=0}^{n-1}\sum_{k=0}^{\ell}(-1)^{\ell+1+k}d_k h_{\ell}+\sum_{\ell=0}^{n}\sum_{k=\ell+1}^{n}(-1)^{\ell+k+1}d_{k} h_{\ell}+v_2^n-v_1^n\\
		&=-\sum_{k=0}^{n}\sum_{\ell=0}^{n-1}(-1)^{\ell+k}d_kh_\ell +v_2^n-v_1^n\\
		&=-\hacek{d}^{n-1}h^n+v_2^n-v_1^n\;,
	\end{align*}
	as required.
\end{proof}
\begin{cor}\lbl{cor:refinementPullback}
	Let $\Mm$ be a presheaf of modules on a space $X$ and $\Uu,\Vv$ be open covers of $X$.
	\begin{alphanumerate} 
		\item If $\Vv$ is a refinement of $\Uu$, we have a canonical morphism 
		\begin{align*}
			\tau_{\Uu,\Vv}\colon \check{H}^\bullet (\Uu,\Mm)\morphism\check{H}^\bullet (\Vv,\Mm)
		\end{align*}
		satisfying $\tau_{\Vv,\Ww}\tau_{\Uu,\Vv}=\tau_{\Uu,\Ww}$ if $\Ww$ is another open cover of $X$ which is a refinement of $\Vv$, as well as $\tau_{\Uu,\Uu}=\id$.
		\item If $\Uu$ is a refinement of $\Vv$ and $\Vv$ a refinement of $\Uu$, then $\tau_{\Uu,\Vv}$ and $\tau_{\Vv,\Uu}$ are isomorphisms which are inverse to each other.
		\item If there is an $i^*\in I$ such that $U_{i^*}=X$, then $\check{H}^n(\Uu,\Mm)=0$ for $n\geq 1$.
	\end{alphanumerate}
\end{cor}
\begin{proof}
	Part \itememph{a} follows from Lemma~\reff{lem:refinementPullback}. Part \itememph{b} follows from \itememph{a} as $\tau_{\Vv,\Uu}\tau_{\Uu,\Vv}=\tau_{\Uu,\Uu}=\id$. Part \itememph{c} follows from \itememph{b} and Example~\reff{ex:FirstCechComputations} as $\Uu$ and the trivial cover $\Uu_0\colon X=X$ are refinements of each other in this case.
\end{proof}
\begin{rem}\lbl{rem:trivialCalt}
	In general, the cochain homotopy used in the proof of Lemma~\reff{lem:refinementPullback} won't preserve the subcomplex $\check{C}_\alt^\bullet (-,\Mm)\subseteq\check{C}^\bullet (-,\Mm)$. However, Corollary~\reff{cor:refinementPullback}\itememph{c} for $\check{H}_\alt^\bullet (\Uu,\Mm)$ can still be obtained, using the \emph{cochain contraction} $h\colon \check{C}^\bullet (\Uu,\Mm)\morphism\check{C}^\bullet (\Uu,\Mm)$, where $h^n\colon \check{C}^n(\Uu,\Mm)\morphism\check{C}^{n-1}(\Uu,\Mm)$ for $n\geq 1$ is given by
	\begin{align*}
		(h^n\psi)_{i_0,\ldots,i_n}=\psi_{i^*,i_0,\ldots,i_n}\;.
	\end{align*}
	It's a straightforward check that $h$ preserves $\check{C}_\alt^\bullet (\Uu,\Mm)\subseteq \check{C}^\bullet (\Uu,\Mm)$. Moreover, one has the relations
	\begin{align*}
		h^nd_k=\begin{cases}
			\id &\text{if }k=0\\
			d_{k-1}h^{n-1} & \text{if }k>0
		\end{cases}\;,
	\end{align*}
	hence
	\begin{align*}
		\hacek{d}^{n-1}h^n+h^{n+1}\hacek{d}^n&=\sum_{k=0}^{n}(-1)^kd_kh^n+\sum_{k=0}^{n+1}(-1)^kd_kh^{n+1}\\
		&=\id+\sum_{k=0}^{n}\left((-1)^k+(-1)^{k+1}\right)d_kh^n\\
		&=\id\;,
	\end{align*}
	proving that $h$ is indeed a cochain contraction, hence $\check{H}_\alt^i(\Uu,\Mm)=0$ for $i>0$. For our purposes, this will turn out to be sufficient.
\end{rem}
We now arrive at the main result of this section.
\begin{prop}\lbl{prop:CechCohoOnScheme}
	 Let $X$ be a quasi-compact scheme, $\Mm$ be a quasi-coherent $\Oo_X$-module and $\Uu$ and affine open cover of $X$.
	\begin{alphanumerate}
		\item If $\Vv$ is another affine open cover of $X$ which is a refinement of $\Uu$, then
		\begin{align*}
			\tau_{\Uu,\Vv}\colon \check{H}^\bullet (\Uu,\Mm)\isomorphism\check{H}^\bullet (\Vv,\Mm)\;.
		\end{align*}
		\item The inclusion $\check{C}_\alt^\bullet (\Uu,\Mm)\subseteq \check{C}^\bullet (\Uu,\Mm)$ of cochain complexes induces an isomorphism
		\begin{align*}
			\check{H}_\alt^\bullet (\Uu,\Mm)\isomorphism\check{H}^\bullet (\Uu,\Mm)\;.
		\end{align*}
		\item If $X$ is affine and $i>0$, then $\check{H}_\alt^i(\Uu,\Mm)=\check{H}^i(\Uu,\Mm)=0$.
	\end{alphanumerate}
\end{prop}
Before the proof, a lemma.
\begin{lem}\lbl{lem:CohomologyOfPushforward}
	For an open cover $\Uu\colon X=\bigcup_{i\in I}\Uu_i$ of a topological space $X$ and a continuous map $f\colon Y\morphism X$, let $f^{-1}(\Uu)$ be the cover $Y=\bigcup_{i\in I}f^{-1}(U_i)$. Let $\Ff$ be a sheaf of abelian groups on $Y$.
	\begin{alphanumerate}
		\item We have an isomorphism of cochain complexes $\check{C}^\bullet (f^{-1}(\Uu),\Ff)\simeq \check{C}^\bullet (\Uu,f_*\Ff)$ and this isomorphism restricts to an isomorphism $\check{C}_\alt^\bullet (f^{-1}(\Uu),\Ff)\simeq \check{C}_\alt^\bullet (\Uu,f_*\Ff)$.
		\item If the image of $f$ is contained in one of the open subsets $U_i$, then $\check{H}^i(\Uu,f_*\Ff)=\check{H}_\alt^i(\Uu,f_*\Ff)=0$ for $i>0$.
	\end{alphanumerate}
\end{lem}
\begin{proof}
	Part \itememph{a} is pretty much tautological. Let $\Vv=f^{-1}(\Uu)$, $V_i=f^{-1}(U_i)$, then we have $V_{i_0,\ldots,i_n}=f^{-1}(U_{i_0,\ldots,i_n})$, hence $f_*\Ff(V_{i_0,\ldots,i_n})=\Ff(U_{i_0,\ldots,i_n})$ by definition of the \emph{direct image} $f_*\Ff$. The differentials $\hacek{d}$ as well as the degeneracy and transposition maps $s_\ell,t_\ell$ of $\check{C}^\bullet (f^{-1}(\Uu),\Ff)$ and $\check{C}^\bullet (\Uu,f_*\Ff)$ clearly coincide, proving the asserted isomorphisms of cochain complexes.
	
	Part \itememph{b} follows from \itememph{a}, Corollary~\reff{cor:refinementPullback}\itememph{c} and Remark~\reff{rem:trivialCalt}.
\end{proof}
\begin{proof}[Proof of Proposition~\reff{prop:CechCohoOnScheme}]
	Part \itememph{a}. We fix the covers $\Uu$ and $\Vv$ and consider the following conditions on $\Mm$:
	\begin{quote}
		$\Mm$ satisfies $A_i(\Mm)$ iff $\check{H}^j(\Uu,\Mm)\morphism[\tau_{\Uu,\Vv}]\check{H}^j(\Vv,\Mm)$ is an isomorphism for $j<i$ and injective for $i=j$.
	\end{quote}
	\begin{claim}\lbl{claim:AiAi+1}
		For a short exact sequence $0\morphism \Mm'\morphism \Mm\morphism \Mm''\morphism 0$ of quasi-coherent $\Oo_X$-modules one has the implication
		\begin{align*}
		\left[A_{i+1}(\Mm)\text{ and }A_i(\Mm'')\right] \Longrightarrow A_{i+1}(\Mm')\;.
		\end{align*}
	\end{claim}
	To prove Claim~\reff{claim:AiAi+1}, consider the commutative diagram
	\begin{diagram*}
		\node[ob] (A) at (0,1.5) {$\check{H}^{j-1}(\Uu,\Mm)$};
		\node[ob] (B) at (3,1.5) {$\check{H}^{j-1}(\Uu,\Mm'')$};
		\node[ob] (C) at (6,1.5) {$\check{H}^{j}(\Uu,\Mm')$};
		\node[ob] (D) at (9,1.5) {$\check{H}^j(\Uu,\Mm)$};
		\node[ob] (E) at (12,1.5) {$\check{H}^{j}(\Uu,\Mm'')$};
		\node[ob] (A') at (0,0) {$\check{H}^{j-1}(\Vv,\Mm)$};
		\node[ob] (B') at (3,0) {$\check{H}^{j-1}(\Vv,\Mm'')$};
		\node[ob] (C') at (6,0) {$\check{H}^{j}(\Vv,\Mm')$};
		\node[ob] (D') at (9,0) {$\check{H}^{j}(\Vv,\Mm)$};
		\node[ob] (E') at (12,0) {$\check{H}^{j}(\Vv,\Mm'')$};
		\scriptsize
		\draw[->] (A) -- (B);
		\draw[->] (B) -- (C);
		\draw[->] (C) -- (D);
		\draw[->] (D) -- (E);
		\draw[->] (A') -- (B');
		\draw[->] (B') -- (C');
		\draw[->] (C') -- (D');
		\draw[->] (D') -- (E');
		\draw[->] (A) -- (A') node[pos=0.5, left] {$\alpha$};
		\draw[->] (B) -- (B') node[pos=0.5, left] {$\beta$};
		\draw[->] (C) -- (C') node[pos=0.5, left] {$\gamma$};
		\draw[->] (D) -- (D') node[pos=0.5, left] {$\delta$};
		\draw[->] (E) -- (E') node[pos=0.5, left] {$\epsilon$};
	\end{diagram*}
	When $j\leq i$, then $\alpha,\beta$, and $\delta$ are isomorphisms, while $\epsilon$ is injective, proving $\gamma$ to be an isomorphism by the five lemma. When $j=i+1$, $\alpha$ is an isomorphism by $A_{i+1}(\Mm)$ while $\beta$ and $\delta$ are injective, proving $\gamma$ to be injective by the four lemma. Claim~\reff{claim:AiAi+1} follows.
	\begin{claim}\lbl{claim:BigN}
		For any quasi-coherent $\Oo_X$-module $\Mm$ there are a quasi-coherent $\Oo_X$-module $\Nn$ such that $A_i(\Nn)$ holds for arbitary $i$, and a monomorphism $\Mm\monomorphism\Nn$.
	\end{claim}
	To show this, let $j_W\colon W\morphism X$ be the embedding of any affine open subset $W$ contained in one of the open subsets forming the cover $\Vv$. As $X$ is a scheme, the morphism $j_W$ is affine, i.e., preimages of affine open subsets are affine (cf.\ \cite[Definition~2.5.1]{alggeo1}). Indeed, since preimages in this case are intersections with $W$, this follows from Proposition~\reff{prop:SchemesAffineIntersec}. Being an open embedding, $j_W$ is also quasi-separated. By Proposition~\reff{prop:PushforwardOfQcIsQc} and Proposition~\reff{prop:QCohHasKerCoker}\itememph{b}, $(j_W)_*\Mm|_W$ is a quasi-coherent $\Oo_X$-module. Since $X$ is quasi-compact, we find finitely many affine open subsets $W_1,\ldots,W_n$ covering $X$, each $W_i$ contained in an element of $\Vv$ (that is, $X=\bigcup_{i=1}^nW_i$ is a finite affine refinement of $\Vv$). Then
	\begin{align*}
		\Nn\coloneqq\bigoplus_{i=1}^n(j_{W_i})_*\Mm|_{W_i}
	\end{align*}
	is quasi-coherent by Proposition~\reff{prop:QCohHasKerCoker}\itememph{a}.
	
	We have morphisms
	\begin{align*}
		\Mm(U)&\morphism\Nn(U)=\bigoplus_{i=1}^n(j_{W_i})_*(\Mm|_{W_i})(U)=\bigoplus_{i=1}^n\Mm(U\cap W_i)\\
	m&\longmapsto (m|_{U\cap W_i})_{i=1}^n
	\end{align*}
	which are injective as $\Mm$ satisfies the sheaf axiom and form a morphism of $\Oo_X$-modules $\Mm\morphism\Nn$, which is thus a monomorphism.
	
	By Example~\reff{ex:FirstCechComputations}\itememph{c}, 
	\begin{align*}
		\check{H}^k(\Uu,\Nn)=\bigoplus_{i=1}^n\check{H}^k(\Uu,(j_{W_i})_*\Mm|_{W_i})
	\end{align*}
	and same for $\Vv$. But Example~\reff{ex:FirstCechComputations}\itememph{a} and  Lemma~\reff{lem:CohomologyOfPushforward}\itememph{b} (which applies here as every $W_i$ is contained in an element of $\Vv$, thus also of $\Uu$) show that 
	\begin{align*}
		\check{H}^k(\Uu,(j_{W_i})_*\Mm|_{W_i})=\check{H}^k(\Vv,(j_{W_i})_*\Mm|_{W_i})=\begin{cases}
			\Mm(W_i) &\text{if }k=0\\
			0 & \text{else}
		\end{cases}\;,
	\end{align*}
	hence $(j_{W_i})_*\Mm|_{W_i}$ has property $A_i$ for arbitrary $i$. Now $\Nn$ inherits the property $A_i$ from its summands, proving Claim~\reff{claim:BigN}.
		
	We are now able to finish the proof of Proposition~\reff{prop:CechCohoOnScheme}\itememph{a}. We prove by induction on $i$ that $A_i(\Mm)$ holds for arbitrary quasi-coherent $\Oo_X$-modules $\Mm$.  For $i=0$, note that $\check{H}^0(\Uu,\Mm)\simeq \check{H}^0(\Vv,\Mm)\simeq \Mm(X)$ for all $\Oo_X$-modules $\Mm$ (quasi-coherent or not), i.e., $A_0(\Mm)$ holds. 
	
	Now suppose $i\geq 1$ and $A_{i-1}$ holds for every quasi-coherent $\Oo_X$-module. Using Claim~\reff{claim:BigN}, we choose an embedding $\Mm\monomorphism[\iota]\Nn$ where $\Nn$ is quasi-coherent and satisfies $A_i$, then apply Claim~\reff{claim:AiAi+1} to the exact sequence
	\begin{align*}
		0\morphism\Mm\morphism[\iota]\Nn\morphism\coker(\iota)\morphism 0
	\end{align*}
	to derive $A_i(\Mm)$ from $A_i(\Nn)$ and $A_{i-1}(\coker(\iota))$. Here, Proposition~\reff{prop:QCohHasKerCoker}\itememph{a} ensures that $\coker(\iota)$ is quasi-coherent as well.
	
	Part \itememph{b}. The proof of part \itememph{a} can be carried over with the following modifications.
	\begin{itemize}
		\item $A_i(\Mm)$ is now the condition that $\check{H}_\alt^j(\Uu,\Mm)\morphism \check{H}^j(\Uu,\Mm)$ is an isomorphism when $j<i$ and injective for $j=i$.
		\item Claim~\reff{claim:AiAi+1} also holds for the new $A_i$.
		\item If $W\subseteq X$ is affine and contained in one of the $V_i$ and $\Ff\in\Ob(\cat{Qcoh}(W))$, then $(j_W)_*\Ff$ satisfies all $A_i$ (which follows like above from Lemma~\reff{lem:CohomologyOfPushforward}\itememph{b} and Example~\reff{ex:FirstCechComputations}).
	\end{itemize}
	
	Part \itememph{c}. Apply \itememph{a} with $\Uu$ the trivial cover $X=X$ and $\Vv$ any affine open cover of $X$ to conclude $\check{H}^i(\Vv,\Mm)=0$ when $i>0$ and $\Mm$ is a quasi-coherent $\Oo_X$-module. Then one uses \itememph{b} to show that $\check{H}_\alt^\bullet (\Vv,\Mm)$ also vanishes in positive degrees.
\end{proof}
\begin{defi}
	Let $X$ be a quasi-compact scheme and $\Mm$ a quasi-coherent $\Oo_X$-module. We put
	\begin{align*}
		H^i(X,\Mm)\coloneqq \check{H}^i(\Vv_0,\Mm)\;,
	\end{align*}
	where $\Vv_0$ is the affine open cover of $X$ consisting of all affine open subsets of $X$.
\end{defi}
Summing up our previous work, we obtain the following Theorem~\reff{thm:CohoOnSchemes} in full glory.
\begin{thm}\lbl{thm:CohoOnSchemes}
	Let $X$ be a quasi-compact scheme and $\Mm$ a quasi-coherent $\Oo_X$-module.
	\begin{alphanumerate}
		\item If $\Uu$ is any affine open over of $X$, then
		\begin{align*}
			H^\bullet (X,\Mm)\isomorphism[\tau_{\Uu,\Vv_0}]\check{H}^\bullet (\Uu,\Mm)\lisomorphism \check{H}_\alt^\bullet (\Uu,\Mm)
		\end{align*}
		are isomorphisms compatible with the (iso)morphisms $\check{H}^\bullet (\Uu,\Mm)\morphism[\tau_{\Uu,\Vv}]\check{H}^\bullet (\Vv,\Mm)$ when $\Vv$ is an affine refinement of $\Uu$.
		\item When $X$ is affine, $H^i(X,\Mm)=0$ when $i>0$.
		\item There is a canonical isomorphism $H^0(X,\Mm)\simeq \Mm(X)$.
		\item When $0\morphism \Mm'\morphism\Mm\morphism\Mm''\morphism0$ is a short exact sequence of quasi-coherent $\Oo_X$-modules, one has a long exact cohomology sequence
		\begin{multline*}
			\begin{tikzpicture}[line cap=round, line join=round]
			\node[ob] (0o) at (0,0) {$0$};
			\node[ob] (0u) at (0,1.5) {$0$};
			\node[ob] (H0Z) [right=0.575 of 0o] {$H^0(X,\Mm')$};
			\node[ob] (H0CX) [right=0.575 of H0Z] {$H^0(X,\Mm)$};
			\node[ob] (H0CXx) [right=0.575 of H0CX] {$H^0(X,\Mm'')$};
			\node[ob] (Z) at (0u -| H0Z) {$\Mm'(X)$};
			\node[ob] (CX) at (0u -| H0CX) {$\Mm(X)$};
			\node[ob] (CXx) at (0u -| H0CXx) {$\Mm''(X)$};
			\node[ob] (H1Z) [right=0.575 of H0CXx] {$H^1(X,\Mm')$};
			\node[ob] (H1CX) [right=0.575 of H1Z] {$\ldots$};
			\scriptsize
			\draw[->] (0u) -- (Z);
			\draw[->] (0o) -- (H0Z);
			\draw[->] (Z) -- (CX);
			\draw[->] (H0Z) -- (H0CX);
			\draw[->] (CX) -- (CXx);
			\draw[->] (H0CX) -- (H0CXx);
			\draw[->] (H0CXx) -- (H1Z)node[pos=0.5, above] {$d$};
			\draw[->] (H1Z) -- (H1CX);
			\draw[->] (H0Z) -- (Z) node[pos=0.5, above=-0.25ex, sloped] {$\sim$};
			\draw[->] (H0CX) -- (CX) node[pos=0.5, above=-0.25ex, sloped] {$\sim$};
			\draw[->] (H0CXx) -- (CXx) node[pos=0.5, above=-0.25ex, sloped] {$\sim$};
			\end{tikzpicture}\\
			\ldots \morphism H^{i-1}(X,\Mm'')\morphism[d]H^i(X,\Mm')\morphism H^i(X,\Mm)\morphism H^i(X,\Mm'')\morphism[d]\ldots
		\end{multline*}
	\end{alphanumerate}
\end{thm}
\begin{proof}
	Part \itememph{a} follows from Proposition~\reff{prop:CechCohoOnScheme} and the fact that every affine open cover $\Uu$ is a refinement of $\Vv_0$. Part \itememph{b} is Proposition~\reff{prop:CechCohoOnScheme}\itememph{c}. Part \itememph{c} was seen in Example~\reff{ex:FirstCechComputations}\itememph{a} and \itememph{d} follows from Proposition~\reff{prop:LongExactCechSequence}.
\end{proof}
\begin{cor}
	Let $0\morphism \Mm'\morphism[\alpha]\Mm\morphism[\beta]\Mm''\morphism0$ be a short exact sequence of $\Oo_X$-modules, where $X$ is any prescheme. Then if two of $\Mm',\Mm,\Mm''$ are quasi-coherent, so is the third, and for any affine open subset $U\subseteq X$ the sequence
	\begin{align}\lbl{eq:exactAffineSectionSequence}
		0\morphism \Mm'(U)\morphism[\alpha]\Mm(U)\morphism[\beta]\Mm''(U)\morphism 0\tag{$*$}
	\end{align}
	is exact.
\end{cor}
\begin{proof}
	We have seen in Proposition~\reff{prop:QCohHasKerCoker}\itememph{a} that the category $\cat{QCoh}(X)$ of quasi-coherent $\Oo_X$-modules is stable under kernels and cokernels. Therefore, $\Mm'$ is quasi-coherent when $\Mm$ and $\Mm''$ are and $\Mm''$ is quasi-coherent when $\Mm'$ and $\Mm$ are. We may thus assume that $\Mm'$ and $\Mm''$ are the quasi-coherent ones.
	
	To show exactness of \eqreff{eq:exactAffineSectionSequence}, we may replace $X$ by $U$ and the involved sheaves by their restriction to $U$ and thus assume that $X=U=\Spec R$ is affine. This may also be assumed in the proof of quasi-coherentness of $\Mm$, which is a local question. We then only need to show surjectivity of $\Mm(X)\morphism[\beta]\Mm''(X)$. If we had the long exact cohomology sequence available, this would be an immediate consequence of the vanishing of $H^1(X,\Mm')$ -- but we haven't, so what the proof does is mimicking the cohomological argument on foot. For this, let $m''\in\Mm''(X)$. As $\Mm\morphism[\beta]\Mm''$ is an epimorphism and the affine open subsets form a topology base in $X$, there is an affine open cover $\Vv\colon X=\bigcup_{i\in I}V_i$ such that $m''|_{V_i}$ is in the image of $\beta(m_i)$ for some $m_i\in \Mm(V_i)$.
	
	The $m_i$ satisfy $\beta(m_i|_{V_{i,j}}-m_j|_{V_{i,j}})=0\in\Mm''(V_{i,j})$, hence there are $m'_{i,j}\in\Mm(V_{i,j})$ such that $\alpha(m'_{i,j})=m_i|_{V_{i,j}}-m_j|_{V_{i,j}}$, the sequence $0\morphism\Mm'(V_{i,j})\morphism[\alpha]\Mm(V_{i,j})\morphism[\beta]\Mm''(V_{i,j})$ being exact. Consider the elements $\mu=(m_i)_{i\in I}\in \check{C}^0(\Vv,\Mm)$ and $\mu'=(m'_{i,j})_{(i,j)\in I^2}\in \check{C}^1(\Vv,\Mm')$. Then $\alpha(\mu')=\hacek{d}\mu$. Thus 
	\begin{align*}
		\alpha(\hacek{d}\mu')=\hacek{d}\alpha(\mu')=\hacek{d}^2\mu=0\;,
	\end{align*}
	hence $\hacek{d}\mu'=0$ by injectivity of $\alpha$. This means that $\mu'$ is a cocycle, and by the vanishing of $\check{H}^1(\Vv,\Mm')$ (due to Proposition~\reff{prop:CechCohoOnScheme}\itememph{c}), also a coboundary. That is, $\mu'=\hacek{d}\lambda'$ for some $\lambda\in\check{C}^0(\Vv,\Mm')$ (note that at this point Professor Franke run out of primes, though there are apparently infinitely many of them). Let $\lambda=\alpha(\lambda')$. As $\hacek{d}(\mu-\lambda)=\alpha(\mu')-\alpha(\mu')=0$, the $\mu_i-\lambda_i$ lift by the sheaf axiom to a single element $m^*\in\Mm(X)$ such that $m^*|_{V_i}=\mu_i-\lambda_i$. Since $\beta(\mu-\lambda)=\beta(\mu)-\beta(\alpha(\lambda'))=\beta(\mu)$, we have $\beta(m^*)=m''$. This proves the surjectivity of $\Mm(X)\morphism\Mm''(X)$ and thus \eqreff{eq:exactAffineSectionSequence}.
	
	To show that $\Mm$ is quasi-coherent, consider $0\morphism M'\morphism M\morphism M''\morphism 0$ where $M'=\Mm'(X)$, $M=\Mm(X)$, and $M''=\Mm''(X)$ which is an exact sequence of $R$-modules (as we just proved). As pointed out in the proof of Proposition~\reff{prop:QCohHasKerCoker}\itememph{a},  the functor $R\cat{-Mod}\morphism\cat{QCoh}(\Spec R)$ mapping $N\mapsto \snake{N}$ preserves kernels and cokernels, hence the diagram
	\begin{diagram*}
		\node[ob] (A) at (0,1.5) {$0$};
		\node[ob] (B) at (1.5,1.5) {$\snake{M}'$};
		\node[ob] (C) at (3,1.5) {$\snake{M}$};
		\node[ob] (D) at (4.5,1.5) {$\snake{M}''$};
		\node[ob] (E) at (6,1.5) {$0$};
		\node[ob] (A') at (0,0) {$0$};
		\node[ob] (B') at (1.5,0) {$\Mm'$};
		\node[ob] (C') at (3,0) {$\Mm$};
		\node[ob] (D') at (4.5,0) {$\Mm''$};
		\node[ob] (E') at (6,0) {$0$};
		\scriptsize
		\draw[->] (A) -- (B);
		\draw[->] (B) -- (C);
		\draw[->] (C) -- (D);
		\draw[->] (D) -- (E);
		\draw[->] (A') -- (B');
		\draw[->] (B') -- (C');
		\draw[->] (C') -- (D');
		\draw[->] (D') -- (E');
		\draw[->] (B) -- (B') node[pos=0.5, above=-0.25ex, sloped] {$\sim$};
		\draw[->] (C) -- (C');% node[pos=0.5, left] {$\gamma$};
		\draw[->] (D) -- (D') node[pos=0.5, above=-0.25ex, sloped] {$\sim$};
	\end{diagram*}
	has exact rows. Since $\snake{M}'\morphism\Mm'$ and $\snake{M}''\morphism\Mm''$ are isomorphisms, so is $\snake{M}\morphism\Mm$ by the five lemma.
\end{proof}
In the following, if $\Uu\colon X=\bigcup_{i\in I}U_i$ is an open cover and $Y\subseteq X$, we will write $\Uu\cap Y$ for the open cover $Y=\bigcup_{i\in I}(Y\cap U_i)$.
\begin{cor}\lbl{cor:CohomologyOfPushforward}
	Let $X$ be a quasi-compact scheme.
	\begin{alphanumerate}
		\item If $W\monomorphism[j]X$ is the embedding of the affine open subset $W$ and $\Mm\in\Ob(\cat{QCoh}(W))$, then $H^p(X,j_*\Mm)=0$ for $p>0$.
		\item Let $W_i\monomorphism[j_i]X$ be the embeddings of affine open subsets $W_i$ for $i=1,\ldots,n$. Let $W=\bigcup_{i=1}^nW_i$. If $\Mm\in\Ob(\cat{QCoh}(X))$, then
		\begin{align*}
			H^p\bigg(X,\bigoplus_{i=1}^n(j_i)_*\Mm|_{W_i}\bigg)=0\quad\text{for }p>0
		\end{align*}
		and if $\Kk$ denotes the kernel of $\Mm\morphism\bigoplus_{i=1}^n(j_i)_*\Mm|_{W_i}$ sending $m\in\Mm(U)$ to
		\begin{align*}
			\left(m|_{U\cap W_i}\right)_{i=1}^n\in\bigoplus_{i=1}^n\Mm(U\cap W_i)=\bigoplus_{i=1}^n(j_i)_*\Mm|_{W_i}(U)\;,
		\end{align*}
		then $\Kk$ satisfies $\Kk|_W=0$.
	\end{alphanumerate}
\end{cor}
\begin{proof}
	Part \itememph{a}. Let $\Uu$ be any affine open cover of $X$. By Proposition~\reff{prop:SchemesAffineIntersec}, the $U_i\cap W$ are affine and we have $\check{H}^p(\Uu\cap W,\Mm)\simeq \check{H}^p(\Uu,j_*\Mm)=0$ by Lemma~\reff{lem:CohomologyOfPushforward}\itememph{a} and Proposition~\reff{prop:CechCohoOnScheme}\itememph{c}.
	
	Part \itememph{b} follows from the fact that each $j_i$ is a monomorphism and the sheaf axiom.
\end{proof}
\begin{lem}\lbl{lem:LocalizedCechComplex}
	Let $\Uu$ be a finite open cover of the quasi-separated prescheme $X$ by quasi-compact open subsets and $f\in\Oo_X(X)$. By the universal property of localization, we have morphisms $\Mm(U)_f\morphism\Mm(U\setminus V(f))$ for arbitrary $\Oo_X$-modules $\Mm$ and open $U\subseteq X$. When applied to $U=U_{i_0,\ldots,i_n}$, this gives a morphism
	\begin{align*}
		\check{C}^\bullet (\Uu,\Mm)_f\morphism\check{C}^\bullet \Big(\Uu\cap\big(X\setminus V(f)\big),\Mm\Big)
	\end{align*}
	of \vC ech complexes. When $\Mm$ is quasi-coherent, this is an isomorphism.
\end{lem}
\begin{rem*}
	The structure of an $\Oo_X(X)$-module on $\check{C}^\bullet (\Uu,\Mm)$ is obtained using the $\Oo_X(U_{i_0,\ldots,i_n})$-module structures on $\Mm(U_{i_0,\ldots,i_n})$ followed by $\Oo_X(X)\morphism \Oo_X(U_{i_0,\ldots,i_n})$. Taking the product over all $(i_0,\ldots,i_n)\in I^{n+1}$ gives the structure of a cochain complex of $\Oo_X(X)$-modules on $\check{C}^\bullet (\Uu,\Mm)$, which is used in the formulation of Lemma~\reff{lem:LocalizedCechComplex}. The morphism under investigation is a morphism of cochain complexes  of $\Oo_X(X)_f$-modules.
\end{rem*}
\begin{proof}[Proof of Lemma~\reff{lem:LocalizedCechComplex}]
	As $\check{C}^n(\Uu,\Mm)=\prod_{(i_0,\ldots,i_n)\in I^{n+1}}\Mm(U_{i_0,\ldots,i_n})$ is a finite product and localization commutes with finite products, it is sufficient to show that
	\begin{align*}
		\Mm(U_{i_0,\ldots,i_n})_f\isomorphism\Mm(U_{i_0,\ldots,i_n}\setminus V(f))\;.
	\end{align*}
	By Definition~\reff{def:Quasicoherent}\itememph{c}, this will hold if $U_{i_0,\ldots,i_n}$ is quasi-compact and quasi-separated. It is quasi-separated as an open subset of the quasi-separated prescheme $X$. It is quasi-compact as an intersection of the quasi-compact subsets $U_{i_0},\ldots,U_{i_n}$ using quasi-separatedness of $X$.
\end{proof}
\begin{prop}\lbl{prop:localiedCoho}
	Let $X$ be a quasi-compact scheme, $\Mm$ a quasi-coherent $\Oo_X$-module, and $f\in\Oo_X(X)$. Then we have a canonical isomorphism
	\begin{align*}
		H^\bullet (X,\Mm)_f\isomorphism H^\bullet \left(X\setminus V(f),\Mm|_{X\setminus V(f)}\right)\;.
	\end{align*}
\end{prop}
\begin{proof}
	Follows from Lemma~\reff{lem:LocalizedCechComplex}. Note that the open subset $X\setminus V(f)\subseteq X$ is a scheme again by \cite[Fact~1.5.9\itememph{c}]{alggeo1} and quasi-compact by \cite[Fact~2.1.2]{alggeo1}.
\end{proof}
\begin{rem*}
	In general, there is the notion of \emph{derived functors} (from Grothendieck's famous \emph{T\^{o}hoku paper}) applicable to the left-exact functors on an abelian category with sufficiently many injective objects, like $\Rr\cat{-Mod}$ ($\Rr$ being any sheaf of rings on any topological space) and $\cat{QCoh}(X)$ ($X$ being any prescheme). When $\Mm$ is an $\Rr$-module on any topological space $X$, the derived functors taken on the category of $\Rr$-modules or sheaves of abelian groups are canonically isomorphic. When $X$ is quasi-compact (or paracompact, this is actually sufficient) scheme, these are isomorphic to the cohomology introduced here and also to the derived functor of the global section functor taken for $\cat{QCoh}(X)$.
\end{rem*}
\section{The affinity criterion of Serre}
\begin{prop}[Serre's affinity criterion]\lbl{prop:serreAffinity}
	For a quasi-compact scheme $X$, the following conditions are equivalent.
	\begin{alphanumerate}
		\item $X$ is affine.
		\item $H^p(X,\Mm)=0$ for $p>0$ when $\Mm$ is a quasi-coherent $\Oo_X$-module.
		\item $H^1(X,\Jj)=0$ when $\Jj$ is a quasi-coherent sheaf of ideals on $X$.
	\end{alphanumerate}
\end{prop}
\begin{rem*}
	In EGA III, this is shown under the weaker assumptions that $X$ is quasi-compact and quasi-separated (instead of separated), using $H^\bullet $ defined as a derived functor on $\Oo_X\cat{-Mod}$.
\end{rem*}
To prove Proposition~\reff{prop:serreAffinity} we need the following.
\begin{prop}
	\begin{alphanumerate}
		\item\lbl{prop:closedPoint} If $Z$ is a non-empty quasi-compact closed subset of a prescheme $X$, it contains a closed point.
		\item\lbl{prop:qcSheafOfIdealsDefinedByVf} If $Z\subseteq X$ is a closed subset of a prescheme $X$, then $\Jj(U) \coloneqq \{f \in \Oo_X(U) \mid Z\cap U \subseteq V(f)\}$
defines a quasi-coherent sheaf of ideals.
	\item\lbl{prop:intersectQc} If $\Nn_1,\Nn_2\subseteq \Mm$ are quasi-coherent subsheaves of the quasi-coherent sheaf of modules $\Mm$, $(\Nn_1 \cap \Nn_2)(U) \coloneqq \Nn_1(U)\cap
\Nn_2(U)$ defines a quasi-coherent sheaf of modules.
	\end{alphanumerate}
\end{prop}
\begin{proof}
	Part \itememph{a} is proved in \cite[Proposition~2.1.1]{alggeo1}

	Part \itememph{b}. Let $U\subseteq X$ be quasi-compact and quasi-separated, $f\in\Oo_X(U)$ and $\varphi \in \Jj(U)$ such
	that $\varphi\vert_{U\cap V(f)}=0$. Then, as $\Oo_X$ is quasi-coherent, there exists $n\in\IN$ such that $f^n\varphi = 0$ by Definition~\reff{def:Quasicoherent}\itememph{c}.
	It follows that the canonical map $\Jj(U)_f\rightarrow \Jj(U\setminus V(f))$ is injective.

	If $\psi \in \Jj(U\setminus V(f))$, then as $\Oo_X$ is quasi-coherent, there is $n\in\IN$ such that $f^n\psi$ extends to some
	$g\in \Oo_X(U)$ (using Definition~\reff{def:Quasicoherent}\itememph{c} again). Now $fg$ extends $f^{n+1}\psi$ and is in $\Jj(U)$. Indeed, we need to show $Z\cap U \subseteq V(fg)$ and this follows from
	\begin{align*}
		(Z\cap U)\setminus V(f)\subseteq V(\psi)\subseteq V\left(f^{n+1} \psi\right)=V\left(fg\vert_{U\setminus V(f)}\right)\subseteq V(fg)
	\end{align*}
	(the first inclusion coming from $\psi\in \Jj(U\setminus V(f))$) and $V(f)\subseteq V(fg)$. Thus we get an isomorphism $\Jj(U)_f\isomorphism \Jj(U\setminus V(f))$, which is the criterion from Definition~\reff{def:Quasicoherent}\itememph{c} once again.

	Part \itememph{c}. It is directly possible to verify $(\Nn_1\cap \Nn_2)(U)_f\isomorphism (\Nn_1\cap \Nn_2)(U\setminus V(f))$ similar
	as above. Alternatively, note that $\Nn_1\cap \Nn_2 = \ker(\Nn_1\rightarrow \Mm\rightarrow\Mm/\Nn_2)$ and use that $\cat{QCoh}(X)$ has kernels and cokernels by Proposition~\reff{prop:QCohHasKerCoker}\itememph{a}.
\end{proof}
\begin{proof}[Proof of Proposition \reff{prop:serreAffinity}]
 The implication \itememph{a} $\Rightarrow$ \itememph{b} follows from Theorem \reff{thm:CohoOnSchemes}\itememph{c} and \itememph{b} $\Rightarrow$ \itememph{c} is Trivial.

Let's do \itememph{c} $\Rightarrow$ \itememph{a}. We first derive \itememph{c} $\Rightarrow$ \itememph{c^{++}}, where \itememph{c^{++}} is the condition
\begin{description}
	\item[\itememph{c^{++}}] If $\Mm\subseteq \Oo_X^n$ is a quasi-coherent submodule, then $H^1(X,\Mm)=0$.
\end{description}
We prove \itememph{c^{++}} by induction on $n$. The case $n=0$ is trivial. Now let $n\geq 1$ and assume that \itememph{c^{++}} holds for $n-1$. If $\Oo_X^{n-1}(U)$ is identified with
$\{(f_1,\ldots,f_{n-1},0) \in \Oo_X(V)^n\} \subseteq \Oo_X(U)^n$, then
$\Mm' = \Mm\cap \Oo_X^{n-1}$ is quasi-coherent by Proposition \reff{prop:intersectQc}\itememph{c}.
Moreover,
\begin{align*}
	\Mm'' \coloneqq \Mm/\Mm'&\morphism \Oo_X\\
	(f_1,\ldots,f_n)\bmod\Mm&\longmapsto f_n
\end{align*}
is a monomorphism. Hence $H^1(X,\Mm')=0$ by induction and $H^1(X,\Mm'')=0$ by \itememph{c}. Now looking at the piece
\begin{align*}
\ldots\morphism H^1(X,\Mm')\morphism H^1(X,\Mm)\morphism H^1(X,\Mm'')\morphism\ldots
\end{align*}
of the long exact cohomology sequence, we get $H^1(X,\Mm)=0$, completing the induction.

Now we prove \itememph{c^{++}} $\Rightarrow$ \itememph{a}. Let $R=\Oo_X(X)$, so we have a morphism $X\morphism[p]\Spec R$ corresponding to $\id_R$
under the adjunction
\begin{align*}
\Hom_{\cat{LRS}}(X,\Spec R) \simeq \Hom_{\cat{Ring}}(R,\Oo_X(X))
\end{align*}
from \cite[Proposition~1.4.2]{alggeo1}. We have to show that $p$ is an isomorphism. For this it is sufficient to show the following.
\begin{alphanumerate}
	\item[\itememph{\alpha}] If $f\in R$ is chosen such that $X\setminus V(f)$ is affine, then $X\setminus V(f) = p^{-1}(\Spec R\setminus V(f))$
	is isomorphically mapped to $\Spec R\setminus V(f)\simeq \Spec(R_f)$ by $p$.
	\item[\itememph{\beta}] The open subsets $X\setminus V(f)$ with $f$ as in \itememph{\alpha} cover $X$.
	\item[\itememph{\gamma}] If $f_1,\ldots,f_n \in R$ are as in \itememph{\alpha} and $X=\bigcup_{i=1}^n (X\setminus V(f_i))$ then
	\begin{align*}
		\Spec R = \bigcup_{i=1}^n (\Spec R\setminus V(f_i))\;.
	\end{align*}
\end{alphanumerate}
By \itememph{\beta} and the quasi-compactness of $X$, it is possible to find
$f_1,\ldots,f_n\in R$ to which \itememph{\gamma} may be applied. Then the open subsets
$U_i = (\Spec R\setminus V(f_i))$ cover $\Spec R$
and $\smash{p^{-1}(U_i) \isomorphism[p] U_i}$. Then we get
$\Spec R\isomorphism[p] X$.

\emph{Proof of \itememph{\alpha}}.
The fact $X\setminus V(f)=p^{-1}(\Spec R\setminus V(f))$ follows from the explicit construction of $p$.
Namely, $p$ maps a point $x\in X$ to the prime ideal $\pp \subseteq R$ which is the preimage of the maximal ideal $\mm_{X,x}\subseteq \Oo_{X,x}$ under $R=\Oo_X(X)\morphism\Oo_{X,x}$.
Hence $x\in X\setminus V(f) \Leftrightarrow f \notin p(x) \Leftrightarrow p(x) \in \Spec R\setminus V(f)$, showing that $p$ is a bijection on the underlying sets.

If $X\setminus V(f)$ is affine, we get
\begin{diagram*}%Let's try it
\node[ob] (Xf) at (0,1.5) {${X\setminus V(f)}$};
\node[ob] (SpecOX) at (4,1.5) {${\Spec\big(\Oo_X(X\setminus V(f))\big)}$};
\node[ob] (SpecRf) at (8,1.5) {$\Spec(R_f)$};
\node[ob] (X) at (0,0) {$X$};
\node[ob] (SpecO) at (4,0) {${\Spec \Oo_X(X)}$};
\node[ob] (SpecR) at (8,0) {$\Spec R$};
\scriptsize
\draw[->] (Xf) -- (X);
\draw[->] (SpecRf) -- (SpecR);
\draw[transform canvas={yshift=1pt}] (SpecO) -- (SpecR);
\draw[transform canvas={yshift=-1pt}] (SpecO) -- (SpecR);
\draw[->] (X) -- (SpecO) node[pos=0.5, above] {$p$};
\draw[->] (SpecOX) -- (SpecO);
\draw[->] (Xf) -- (SpecOX) node[pos=0.5, above=-0.25ex, sloped] {$\sim$};
\draw[->] (SpecOX) -- (SpecRf) node[pos=0.5, above=-0.25ex, sloped] {$\sim$};
\end{diagram*}
The first isomorphism comes from the fact that $X\setminus V(f)$ is affine.
For the second, we use $\Oo_X(X\setminus V(f)) \isomorphism \Oo_X(X)_f= R_f$,
which follows from Definition~\reff{def:Quasicoherent}\itememph{c} using
that $\Oo_X$ is quasi-coherent, and $X$ is both quasi-compact and quasi-separated.
This shows \itememph{\alpha}.

\emph{Proof of \itememph{\beta}}.
Let $U\subseteq X$ be the union of the affine open subsets of the form $X\setminus V(f)$
and $Z=X\setminus U$. If $Z\neq \emptyset$, there is a closed point $z\in Z$ by Proposition \reff{prop:closedPoint}\itememph{a}.
Choose an affine open neighbourhood $V$ of $z$. Let $Y_1\subseteq X$ be the closed subset $X\setminus V$, let
$Y_2 = \{z\}$ and $Y = Y_1\cup Y_2$. The sheaf of ideals $\Jj$ of functions $f\in \Oo_X(W)$ such
that $Y\cap W\subseteq V(f)$ is quasi-coherent by Proposition \reff{prop:qcSheafOfIdealsDefinedByVf}\itememph{b}.
By \itememph{c}, we have $H^1(X,\Jj)=0$. Note that $Y$ is the closed subprescheme of $X$ defined
by $\Jj$ and denote by $Y\monomorphism[i]X$ its embedding, i.e., $\Jj = \ker\big(\Oo_X\xrightarrow{i^\ast} i_\ast \Oo_Y\big)$. We get
some $\varphi\in\Oo_Y(Y) = i_\ast \Oo_Y(X)$ such that
$\varphi\vert_{Y_1}=0$ and $\varphi\vert_{Y_2}=1$ as $Y$ is the disjoint union of its open subsets $Y_1,Y_2$.

By the cohomology sequence
\begin{align*}
0\morphism \Jj(X)\morphism\Oo_X(X)\morphism[i^*] i_\ast \Oo_Y(X)\morphism H^1(X,\Jj)=0\;,
\end{align*}
there is some $f\in\Oo_X(X)$ such that $i^*f=\varphi$. Hence
$Y_1\subseteq V(f)$ and $z\notin V(f)$, so
$z\in X\setminus V(f) = (X\setminus Y_1)\setminus V(f) = V\setminus V(f)$, which is an affine open subset of $X$.
We get $z\in U$, a contradiction.

\emph{Proof of \itememph{\gamma}}. Let $f_1,\ldots,f_n$ be as requested. Consider the morphisms $\Oo_X\morphism[\cdot f_i]\Oo_X$ given by multiplication with $f_i$ and let $f\colon\Oo_X^n\morphism \Oo_X$ be the morphism induced by the $\cdot f_i$. Then $\cdot f_i$ is an isomorphism on $U_i = X\setminus V(f_i)$, hence induces isomorphisms on stalks $\Oo_{X,x}$ for $x\in U_i$, and the $U_i$ cover $X$. Thus $f$
is an epimorphism on stalks, hence an epimorphism of sheaves. Let $\Mm$ be the kernel of this epimorphism. We have
\begin{diagram*}
\node[ob] (Rn) at (0,1.5) {$\vphantom{R^n}$\scriptsize$\hphantom{(r_1,\ldots,r_n)}$};
\node[ob] at (0,1.5) {$R^n$};
\node[ob] (R) at (3,1.5) {$\vphantom{R}$\scriptsize$\hphantom{\sum r_i f_i}$};
\node[ob] at (3,1.5) {$R$};
\node[ob] (OxXn) at (0,0) {$\Oo_X(X)^n$};
\node[ob] (OxX) at (3,0) {$\Oo_X(X)$};
\node[ob] (H1M) at (5.5,0) {$H^1(X,\Mm)$};
\scriptsize
\node[ob] (ri) at ($(Rn.east)+(0,0.5)$) [left] {$(r_1,\ldots,r_n)$};
\node[ob] (sum) at ($(R.west)+(0,0.5)$) [right] {$\sum r_i f_i$};
\draw[|->] (ri) -- (sum);
\draw[->] (Rn) -- (OxXn) node[pos=0.5, above=-0.25ex, sloped] {$\sim$};
\draw[->] (R) -- (OxX) node[pos=0.5, above=-0.25ex, sloped] {$\sim$};
\draw[->] (Rn) -- (R);
\draw[->] (OxXn) -- (OxX) node[pos=0.5, above] {$f$};
\draw[->] (OxX) -- (H1M);
\end{diagram*}
The lower row is an extract of a long exact cohomology sequence.
Note that $H^1(X,\Mm)=0$ by \itememph{c^{++}}, hence the $f_i$ generate $R$ as an ideal of $R$ and $\bigcap_{i=1}^n V(f_i)=\emptyset$
in $\Spec R$ as stated.
\end{proof}
\begin{rem*}
$H^1(X,\Oo_X)=0$ is not enough (even when $X$ is a quasi-compact scheme) for affinity of $X$ as it holds, e.g., for $X=\IP^n_R$.
\end{rem*}
\section{Cohomological dimension}
\begin{prop}[Grothendieck]\lbl{prop:CohoDimension}
Let $X$ be a quasi-compact scheme and $Z\subseteq X$ a closed subset which is Noetherian as topological space. %
%\footnote{It is possible that we will later add the further assumption that $X$ is Noetherian as a topological space. The necessity of such an assumption will hopefully be clear once a full proof of this proposition is presented.}
Let $\Mm$ be a quasi-coherent sheave of $\Oo_X$-modules satisfying $\Mm\vert_{X\setminus Z}=0$ and let $p>\dim(Z)$ (the Krull dimension of $Z$, which may be infinite). Then $H^p(X,\Mm)=0$.
\end{prop}
\begin{rem*}
	Grothendieck shows $H^p(Z,\Gg)=0$ when $Z$ is a Noetherian topological space, $\Gg$ a sheaf of abelian groups on it and $p>\dim(Z)$. This, however, requires the construction of $H^p$ as a derived functor.
	
	In our situation, the bound may fail to be sharp.
\end{rem*}
\begin{proof}[Proof of Proposition~\reff{prop:CohoDimension}]
There is nothing to show if $\dim(Z)=\pm \infty$, so we may show this by induction on $\dim(Z)$.

Let $\dim(Z)=0$ and $Z=\bigcup_{i=1}^n Z_i$ the decomposition of $Z$ into its irreducible components. Each $Z_i$ is Noetherian (as $Z$ is), hence quasi-compact, and closed, thus contains a closed point $z_i$ by Proposition~\reff{prop:closedPoint}\itememph{a}. But then $Z_i = \{z_i\}$ as $\{z_i\}$ is irreducible and $\dim(Z_i)=0$.

As the affine open subsets form a topology base on $X$, there are affine open neighbourhoods $W_i\subseteq X$ of $z_i$
such that $Z\cap W_i = \{z_i\}$. Let $\Mm_i := (\iota_i)_\ast\left(\Mm\vert_{W_i}\right)$ where 
$W_i\monomorphism[\iota_i] X$ is the corresponding open immersion.
Let $\Ff= \bigoplus_{i=1}^n \Mm_i$. We have seen in Corollary~\reff{cor:CohomologyOfPushforward}\itememph{b} that
$H^p(X,\Ff)=0$ when $p>0$% and that $\ker(M\rigtharrow\snake\Mm)$ vanishes on $W=\bigcup_{i=1}^n W_i$.
%As $\Mm$ and $\snake\Mm$ vanish on $U=X\setminus Z$ and $X=U\cup W$, it follows that $\Mm\rightarrow \snake\Mm$ is a monomorphism
.
Moreover, $(\Mm_j)_{z_i}=0$ when $i\neq j$ while $\Ff\rightarrow \Mm_i$ is an isomorphism on $W_i$, hence induces an isomorphism $\Ff_{z_i}\isomorphism(\Mm_i)_{z_i}=\Mm_{z_i}$ on stalks at $z_i$.
Moreover, since $\Mm$ and $\Mm_i$ vanish on $X\setminus Z$,
we see that $\Ff\isomorphism\Mm$ and the assertion follows.

For the inductive step, let $d=\dim(Z)$ be finite and the assertion be proved for quasi-coherent $\Mm$ supported on closed
subsets of smaller Krull dimension. Let $Z=\bigcup_{i=1}^n Z_i$ be the decomposition into irreducible components $Z_i$. Then $Z_i\setminus \bigcup_{j\neq i}Z_j$ is non-empty, as $Z_i\subseteq Z_j$ otherwise by irreducibility of $Z_i$. Hence we find affine open subsets $W_i\subseteq X$ such that $W_i$ intersects $Z_i$ but no $Z_j$ fpr $j\neq i$.

Let $\Nn_i = (j_i)_\ast\Mm\vert_{W_i}$ where $W_i\monomorphism[j_i]X$ is the inclusion of the open subset $W_i\subseteq X$ and let $\Nn=\bigoplus_{i=1}^n\Nn_i$. We consider the morphism $\Mm\morphism\Nn$ given by
\begin{align*}
\Mm(U)&\morphism\Nn(U)=\bigoplus_{i=1}^n(j_{i})_*(\Mm|_{W_i})(U)=\bigoplus_{i=1}^n\Mm(U\cap W_i)\\
m&\longmapsto (m|_{U\cap W_i})_{i=1}^n\;.
\end{align*}
As was seen in Corollary~\reff{cor:CohomologyOfPushforward}\itememph{b}, $H^p(X,\Nn)=0$ when $p>0$ and also $\Mm\morphism\Nn$ is a monomorphism outside $\snake{X}=X\setminus\bigcup_{i=1}^nW_i$ (well, this is the same construction as in Claim~\reff{claim:BigN}). Let $\snake{Z}=Z\cap \snake{X}$. 
%Remark, what monomorphism _outside_ means
We have $\Mm|_{X\setminus Z}=0=\Nn|_{X\setminus Z}$, hence $\Mm\morphism \Nn$ is also a monomorphism outside $Z$, hence a monomorphism outside $\snake{Z}$. We claim:
\begin{claim}\lbl{claim:epicClaim}
	The morphism $\Mm\morphism\Nn$ is also an epimorphism outside $\snake{Z}$.
\end{claim}
Assuming this for the moment, we have an exact sequence
\begin{align}\lbl{eq:4termsequence}
	0\morphism\Kk\morphism\Mm\morphism\Nn\morphism\Cc\morphism 0\tag{\S}
\end{align}
(with $\Kk=\ker(\Mm\morphism\Nn)$ and $\Cc=\coker(\Mm\morphism\Nn)$), where $\Kk$ and $\Cc$ vanish outside $\snake{Z}$. Since $W_i$ intersects $Z_i$, we have $Z_i\not\subseteq \snake{X}$, hence no irreducible component of $Z$ is contained in $\snake{Z}$ and we obtain $\dim(\snake{Z})<\dim(Z)=d$. If $d=0$, this implies $\snake{Z}=0$, hence $\Mm\simeq \Nn$ has vanishing cohomology in positive degrees. Otherwise, the induction assumption implies $H^p(X,\Kk)=0=H^p(X,\Cc)$ when $p\geq d$. We split \eqreff{eq:4termsequence} into $0\morphism\Bb\morphism\Nn\morphism\Cc\morphism 0$ and $0\morphism \Kk\morphism\Mm\morphism\Bb\morphism 0$, where $\Bb=\ker(\Nn\morphism\Cc)$ which is also the image of $\Mm\morphism\Nn$. These give
\begin{align*}
	\ldots\morphism H^{p-1}(X,\Cc)\morphism H^p(X,\Bb)\morphism H^p(X,\Nn)\morphism \ldots\;,
\end{align*}
hence $H^p(X,\Bb)=0$ when $p>d$, and then
\begin{align*}
	\ldots\morphism H^p(X,\Kk)\morphism H^p(X,\Mm)\morphism H^p(X,\Bb)\morphism \ldots\;,
\end{align*}
implying $H^p(X,\Mm)=0$ when $p>d$. The vanishing assertion follows.

It remains to prove Claim~\reff{claim:epicClaim} that $\Cc|_{X\setminus \snake{Z}}$ vanishes. For this, we use that
\begin{align}\lbl{eq:stalksOfNn}
	(\Nn_i)_x\simeq\begin{cases}
		\Mm_x & \text{if }x\in Z_i\cap W_i\\
		0 & \text{if }x\notin Z_i
	\end{cases}\;.\tag{\%}
\end{align}
When $x\in Z$ this implies $\Nn_x=0$, hence $\Mm_x\morphism\Nn_x$ is surjective. When $x\in Z\setminus \snake{Z}$, then $x\in Z\cap W_i$ for some $i$, hence $x\in Z_i$ since $W_i$ doesn't intersect $Z_j$ for $j\neq i$. Thus
\begin{align*}
	(\Nn_j)_x\simeq\begin{cases}
	\Mm_x & \text{if }j=i\\
	0 & \text{else}
	\end{cases}
\end{align*}
by \eqreff{eq:stalksOfNn} and $\Mm_x\isomorphism\Nn_x$. To see \eqreff{eq:stalksOfNn}, note that $\Nn_i(U)=\Mm(U)$ when $U\subseteq W_i$ and $x\in Z_i\cap W_i$, hence $(\Nn_i)_x\simeq \Mm_x$ in this case. Suppose $x\notin Z_i$. When $x\notin W_i$, then clearly $(\Nn_i)_x=0$. When $x\in W_i\setminus Z_i$, then also $x\notin Z$ as $W_i$ doesn't intersect the other irreducible components, hence $(\Nn_i)_x=0$ follows from the fact that $\Mm|_{X\setminus Z}=0$.
\end{proof}

\section{Cohomology of morphisms and functorial properties of cohomology}
Let $X\morphism[f]Y$ be a quasi-compact and separated morphism of preschemes. The set $\Bb$ of quasi-compact open subsets $U\subseteq Y$ which are schemes form a topology base on $Y$ (well, it contains all affine open subsets). By well-known properties of quasi-compact and of separated morphisms, $f^{-1}(U)$ is a quasi-compact scheme when $U\in \Bb$ (quasi-compactness is just Definition~\reff{def:qcqs} and $f^{-1}(U)\morphism\Spec\IZ$ is separated as the composition of the separated morphisms $f$ and $U\morphism\Spec \IZ$, cf.\ \cite[Fact~1.5.7\itememph{c}]{alggeo1}). We thus have for any quasi-coherent $\Oo_X$-module $\Mm$ a presheaf
\begin{align}\lbl{eq:cohoPresheaf}
	\Phi^p\Mm\colon U\mapsto H^p\left(f^{-1}(U),\Mm\right)\tag{$*$}
\end{align}
on $\Bb$. The presheaf structure on \eqreff{eq:cohoPresheaf} is defined as follows. Let $V\subseteq U$ be elements of $\Bb$ and $\Uu,\Vv$ the covers of $f^{-1}(U)$ and $f^{-1}(V)$ by their affine open subsets. Then the morphism of restricting to the intersection with $f^{-1}(V)$ gives $\check{C}^\bullet (\Uu,\Mm)\morphism\check{C}^\bullet (\Uu\cap f^{-1}(V),\Mm)$ which may be followed by any morphism $\check{C}^\bullet (\Uu\cap f^{-1}(V),\Mm)\morphism\check{C}^\bullet (\Vv,\Mm)$ coming from a refinement map for the pair $(\Vv,\Uu\cap f^{-1}(\Vv))$ 
(we proved in Lemma~\reff{lem:refinementPullback}\itememph{c} that any two refinement maps induce the same morphism on cohomology, so this choice doesn't matter), which gives
\begin{align*}
	H^p\left(f^{-1}(U),\Mm\right)=\check{H}^p(\Uu,\Mm)\morphism\check{H}^p(\Vv,\Mm)=H^p\left(f^{-1}(V),\Mm\right)\;,
\end{align*}
and this is the desired presheaf structure.

\begin{defi}
	Sheafifying $\Phi^p\Mm$ gives an $\Oo_Y$-module called the \defemph{$p\ordinalth$ direct image} of $\Mm$ under $f$ and denoted $R^pf_*\Mm$.
\end{defi}
 We have a morphism
 \begin{align}\lbl{eq:HpToRp}
 	H^p\left(f^{-1}(U),\Mm\right)\morphism R^pf_*\Mm(U)=H^0(U,R^pf_*\Mm)
 \end{align}
 (for $U\in\Bb$) as a special case of the morphism from a presheaf to its sheafification.
 
 \begin{prop}\lbl{prop:HigherDirectImage}
 	Let $X\morphism[f]Y$ is a quasi-compact and separated morphism of preschemes and $\Mm$ a quasi-coherent $\Oo_X$-module.
 	\begin{alphanumerate}
 		\item $R^0f_*\Mm\simeq f_*\Mm$ canonically.
 		\item The $\Oo_X$-modules $R^pf_*\Mm$ are quasi-coherent.
 		\item For a short exact sequence $0\morphism\Mm'\morphism\Mm\morphism\Mm''\morphism 0$ of quasi-coherent $\Oo_X$-modules one has a long exact cohomology sequence 
 		\begin{multline*}
 		\begin{tikzpicture}[line cap=round, line join=round]
 		\node[ob] (0o) at (0,0) {$0$};
 		\node[ob] (0u) at (0,1.5) {$0$};
 		\node[ob] (H0Z) [right=0.575 of 0o] {$R^0f_*\Mm'$};
 		\node[ob] (H0CX) [right=0.575 of H0Z] {$R^0f_*\Mm$};
 		\node[ob] (H0CXx) [right=0.575 of H0CX] {$R^0f_*\Mm''$};
 		\node[ob] (Z) at (0u -| H0Z) {$f_*\Mm'$};
 		\node[ob] (CX) at (0u -| H0CX) {$f_*\Mm$};
 		\node[ob] (CXx) at (0u -| H0CXx) {$f_*\Mm''$};
 		\node[ob] (H1Z) [right=0.575 of H0CXx] {$R^1f_*\Mm'$};
 		\node[ob] (H1CX) [right=0.575 of H1Z] {$\ldots$};
 		\scriptsize
 		\draw[->] (0u) -- (Z);
 		\draw[->] (0o) -- (H0Z);
 		\draw[->] (Z) -- (CX);
 		\draw[->] (H0Z) -- (H0CX);
 		\draw[->] (CX) -- (CXx);
 		\draw[->] (H0CX) -- (H0CXx);
 		\draw[->] (H0CXx) -- (H1Z)node[pos=0.5, above] {$\delta$};
 		\draw[->] (H1Z) -- (H1CX);
 		\draw[->] (H0Z) -- (Z) node[pos=0.5, above=-0.25ex, sloped] {$\sim$};
 		\draw[->] (H0CX) -- (CX) node[pos=0.5, above=-0.25ex, sloped] {$\sim$};
 		\draw[->] (H0CXx) -- (CXx) node[pos=0.5, above=-0.25ex, sloped] {$\sim$};
 		\end{tikzpicture}\\
 		\ldots \morphism R^{p-1}f_*\Mm''\morphism[\delta]R^pf_*\Mm'\morphism R^pf_*\Mm\morphism R^pf_*\Mm''\morphism[\delta]\ldots
 		\end{multline*}
 		\item When $U\subseteq Y$ is affine, \eqreff{eq:HpToRp} becomes an isomorphism $H^p\left(f^{-1}(U),\Mm\right)\isomorphism R^pf_*\Mm(U)$.
 	\end{alphanumerate}
 \end{prop}
 \begin{proof}
 	Part \itememph{a}. We have $\Phi^0(U)=H^0(f^{-1}(U),\Mm)\simeq \Mm(f^{-1}(U))=f_*\Mm(U)$, hence $\Phi^0\simeq f_*\Mm|_\Bb$. Then sheafifying gives $f_*\Mm$ again by \cite[Proposition~1.2.1\itememph{c},\itememph{d}]{alggeo1}.
 	
 	Part \itememph{c}. By Theorem~\reff{thm:CohoOnSchemes} we have a similar long exact cohomology sequence of presheaves on $\Bb$. Sheafifying it gives the desired sequence, as sheafification  is an exact functor (exactness may be tested on stalks and sheafification preserves stalks by \cite[Proposition1.2.1\itememph{a}]{alggeo1}).
 	
 	It is easy to see that our construction of $R^pf_*$ is base-local, i.e.,
 	\begin{align*}
 		R^pf_*\Mm|_U\simeq R^p\left(f|_{f^{-1}(U)}\right)_*\Mm|_{f^{-1}(U)}\;.
 	\end{align*}
 	Thus we may assume $Y=\Spec R$ to be affine (quasi-coherentness is a local property) for \itememph{b} and $Y=U=\Spec R$ in \itememph{d}. 
 	
 	Let $\Bb=\left\{U\subseteq Y\st U\text{ open and a quasi-compact scheme}\right\}$ the topology base on which $\Phi^p$ is defined, then $\snake{\Bb}=\left\{Y\setminus V(\lambda)\st\lambda\in R\right\}$ is another topology base of $Y$, and $\left(\Gg|_{\snake{\Bb}}\right)^\sh=\Gg^\sh$ for any presheaf $\Gg$ on $\Bb$ by \cite[Proposition~1.2.1\itememph{c}]{alggeo1}. But for $U=Y\setminus V(\lambda)\in\snake{\Bb}$ we have
 	\begin{align*}
 		H^p\left(f^{-1}(U),\Mm\right)=H^p\left(f^{-1}(Y\setminus V(\lambda)),\Mm\right)= H^p(X\setminus V(f^*\lambda),\Mm)\simeq H^p(X,\Mm)_{f^*\lambda} =M_\lambda
 	\end{align*}
 	by Proposition~\reff{prop:localiedCoho}, where we consider $M=H^p(X,\Mm)$ as an $R=\Oo_Y(Y)$-module via $f^*\colon \Oo_Y(Y)\morphism f_*\Oo_X(Y)=\Oo_X(X)$. We also used $f^{-1}(Y\setminus V(\lambda))=X\setminus V(f^*\lambda)$ as $f$ is a morphism of locally ringed spaces (cf.\ \cite[Definition~1.3.4\itememph{b}]{alggeo1}). It follows that $R^pf_*\Mm$ is canonically isomorphic to the sheafififation of $U\setminus V(\lambda)\mapsto M_\lambda$, which is just $\snake{M}$ by definition of the twiddlelization (\cite[Definition~1.5.1]{alggeo1}, actually). As $\snake{M}$ is a quasi-coherent $\Oo_Y$-module, \itememph{b} follows. Then $M\simeq \snake{M}(Y)$, which shows \itememph{d}.
 \end{proof}
 In general, the morphism \eqreff{eq:HpToRp} fails to be an isomorphism: The \emph{restriction to fibres} (for $Y$ a quasi-compact scheme)
 \begin{align}\lbl{eq:RestrictionToFibres}
 	H^p(X,\Mm)\morphism R^pf_*\Mm=E_2^{0,p}
 \end{align}
 as a special case of \eqreff{eq:HpToRp} is normally not an isomorphism when $Y$ is not affine. There is another morphism, the \emph{pull-back} (for $Y$ again a quasi-compact scheme)
 \begin{align}\lbl{eq:pull-back}
 	E_2^{p,0}=H^p(Y,f_*\Mm)\morphism H^p(X,\Mm)
 \end{align}
(also not an isomorphism unless $f$ is affine) which can be constructed as follows. Let $\Uu_Y$ be the open cover of $Y$ by its affine subsets and $\Vv$ any affine refinement of $f^{-1}(\Uu_Y)$. Then, using Lemma~\reff{lem:CohomologyOfPushforward}\itememph{b} and Proposition~\reff{prop:CechCohoOnScheme}\itememph{a} we obtain
\begin{align*}
	H^p(Y,f_*\Mm)=\check{H}^p(\Uu_Y,f_*\Mm)=\check{H}^p\left(f^{-1}(\Uu_Y),\Mm\right)\morphism \check{H}^p(\Vv,\Mm)\lisomorphism H^p(X,\Mm)
\end{align*}
defining the above pull-back morphism \eqreff{eq:pull-back}.

There is an also inverse image functor $f^*\colon \Oo_Y\cat{-Mod}\morphism \Oo_X\cat{-Mod}$ which preserves quasi-coherence and is left-adjoint to $f_*$\footnote{\lbl{footnote:f^*}We constructed $f^*$ on exercise sheet \#10 of Algebraic Geometry~I and also proved the relevant properties there. It is given by $f^*\Mm=f^\sharp\Mm\otimes_{f^\sharp \Oo_Y}\Oo_X$, where $f^\sharp$ is constructed as in \cite[Remark~1.2.4]{alggeo1}.}. Thus, for $\Nn$ a quasi-coherent $\Oo_Y$-module, there is a canonical morphism $\Nn\morphism f_*f^*\Mm$ which gives another type of pull-back morphism
\begin{align*}
	H^p(Y,\Nn)\morphism H^p(Y,f_*f^*\Nn)\morphism[\text{\eqreff{eq:pull-back}}]H^p(X,f^*\Nn)\;.
\end{align*}
In general, \eqreff{eq:RestrictionToFibres} and \eqreff{eq:pull-back} will fail to be isomorphisms but are part of the \emph{Leray spectral sequence}
\begin{align*}
	E_2^{p,q}=H^p(Y,R^qf_*\Mm)\converge H^{p+q}(X,\Mm)
\end{align*}
converging ``to $H^{p+q}(X,\Mm)$'' (actually to some limit filtration of this). 

This means the following: One has a sequence $\left(E_k^{\bullet,\bullet}\right)_{k=2}^\infty$ of doubly graded abelian groups (or $R$-modules or objects of any other abelian category) equipped with morphisms (the \emph{differentials}) $d_k=d_k^{p,q}\colon E_k^{p,q}\morphism E_k^{p+k,q+1-k}$ such that 
\begin{align*}
	E_k^{p,q}\morphism[d_k]E_k^{p+k,q+1-k}\morphism[d_k]E_k^{p+2k,q+1-2k}
\end{align*}
vanishes and such that $E_{k+1}^{\bullet,\bullet}=H^\bullet (E_k^{\bullet,\bullet},d_k^{\bullet,\bullet})$ in the sense that
\begin{align}\lbl{eq:eta}
	E_{k+1}^{p,q}=\ker\Big(E_k^{p,q}\morphism[d_k]E_k^{p+k,q+1-k}\Big)/\Im\Big(E_k^{p-k,q+k-1}\morphism[d_k]E_k^{p,q}\Big)\tag{$\eta$}\;.
\end{align}
Thus $E_k^{p,q}=0$ for all $k\geq 2$ when $E_2^{p,q}=0$. When $E_2^{\bullet,\bullet}$ is supported in the $1\ordinalst$ quadrant (i.e., $E_2^{p,q,}=0$ if $p<0$ or $q<0$), the same is true for all $E_k^{\bullet,\bullet}$ and \eqreff{eq:eta} becomes
\begin{align}\lbl{eq:eta'}
	E_{k+1}^{p,q}=\coker\Big(E_k^{p-k,q+k-1}\morphism[d_k]E_k^{p,q}\Big)\quad\text{for }k>q+1\tag{$\eta'$}
\end{align}
and 
\begin{align}\lbl{eq:eta''}
	E_{k+1}^{p,q}=\ker\Big(E_k^{p,q}\morphism[d_k]E_k^{p+k,q-k+1}\Big)\quad\text{for }k>p\tag{$\eta''$}
\end{align}
and we deduce $E_{k+1}^{p,q}=E_k^{p,q}$ for $k>\max\{p,q+1\}$.

For $1\ordinalst$ quadrant sequences (like Leray's) we put $E_\infty^{p,q}=E_k^{p,q}$ where $k$ is large enough. When the spectral sequence \emph{converges to some filtration on its limit} $L^\bullet $ (we have $L^\bullet =H^\bullet (X,\Mm)$ for Leray), it is meant that there is a filtration $F^\bullet L^\bullet \colon L^\bullet =F^0L^\bullet \supseteq F^1L^\bullet\supseteq \ldots$ with $E_\infty^{p,q}\simeq F^pL^{p+q}/F^{p+1}L^{p+q}$. Thus, for $1\ordinalst$ quadrant sequences,
\begin{align*}
	F^1L^p=\ker\left(L^p\morphism E_\infty^{0,p}\right)=\ker\Big(L^p\morphism E_2^{p,q}\Big)
\end{align*}
(where we use $E_\infty^{0,p}\subseteq E_2^{0,p}$ by \eqreff{eq:eta''}) has \eqreff{eq:RestrictionToFibres} as a special case. Also, using $F^{p+1}L^p=0$ for $1\ordinalst$ quadrant sequences we obtain an epimorphism
\begin{align*}
	F^pL^p\simeq E_\infty^{p,q}\lepimorphism E_2^{p,0}
\end{align*}
by \eqreff{eq:eta'}. The resulting morphism $E_2^{p,0}\morphism L^p$ has \eqreff{eq:pull-back} as a special case.

There is also a Leray spectral sequence $R^pg_*R^qf_*\Mm\converge R^{p+q}(gf)_*\Mm$.

\section{Affine morphisms and the relative \texorpdfstring{$\SPEC$}{Spec} functor}
\begin{prop}\lbl{prop:AffineMorphism}
	Let $X\morphism[f]Y$ be a morphism of preschemes, then the following conditions are equivalent.
	\begin{alphanumerate}
		\item If $U\subseteq Y$ is an affine open subset, then so is $f^{-1}(U)\subseteq X$.
		\item It is possible to cover $Y$ by affine open $U$ such that $f^{-1}(U)$ is affine open.
	\end{alphanumerate}
	When $X$ is, in addition, quasi-separated, this is moreover equivalent to
	\begin{alphanumerate}\setcounter{enumi}{2}
		\item $f$ is separated and quasi-compact and we have $R^pf_*\Mm=0$ when $\Mm$ is a quasi-coherent $\Oo_X$-module and $p>0$.
	\end{alphanumerate}
\end{prop}
\begin{proof}
	An unconditional proof of \itememph{a} $\Leftrightarrow$ \itememph{b} was given in \cite[Lemma~2.5.1]{alggeo1}.
	
	The implication \itememph{a} $\Rightarrow$ \itememph{b} is trivial. For \itememph{b} $\Rightarrow$ \itememph{c}, it is obvious that $f$ is quasi-compact and separatedness was proved in \cite[Fact~2.6.1]{alggeo1}. To show $R^pf_*\Mm=0$, note that if $U\subseteq Y$ is affine open such that $f^{-1}(U)$ is affine and $\lambda\in\Oo_X(U)$, then $f^{-1}(U\setminus V(\lambda))=f^{-1}(U)\setminus V(f^*\lambda)$ is affine as well. In particular, the $U\subseteq Y$ such that $f^{-1}(U)$ is affine form a topology base on $Y$.  By Theorem~\reff{thm:CohoOnSchemes}\itememph{b} and Proposition~\reff{prop:HigherDirectImage}\itememph{b}, $\Phi^p\Mm$ thus vanishes on a topology base of $Y$. Hence $R^pf_*\Mm=0$ as the sheafification of  $\Phi^p\Mm$. In particular, we didn't need the additional assumption.
	
	Now for \itememph{c} $\Rightarrow$ \itememph{a} assuming that $X$ is quasi-separated. If $U\subseteq Y$ is affine and $\Mm\in\Ob(\cat{QCoh}(f^{-1}(U)))$, then $\Mm$ extends to $X$ by Lemma~\reff{lem:weirdLemma} and by Proposition~\reff{prop:HigherDirectImage}\itememph{d}
	\begin{align*}
		H^p\left(f^{-1}(U),\Mm\right)\simeq R^pf_*\Mm(U)=0\;.
	\end{align*}
	By Serre's affinity criterion (Proposition~\reff{prop:serreAffinity}), $f^{-1}(U)$ is affine.
\end{proof}
\begin{lem}\lbl{lem:weirdLemma}
	Let $X$ be a quasi-separated prescheme and $\Nn$ a quasi-coherent $\Oo_U$-module for some quasi-compact open subset $U\subseteq X$. Then there is a quasi-coherent $\Oo_X$-module $\Mm$ such that $\Mm|_U\simeq \Nn$.
\end{lem}
\begin{proof}
	The open immersion $U\monomorphism[j]X$ is quasi-separated (in fact, separated as is any monomorphism between preschemes, cf.\ \cite[Fact~1.5.7\itememph{a}]{alggeo1}) and quasi-compact since $X$ is quasi-separated and $U$ is quasi-compact. Hence $\Mm=j_*\Nn$ is quasi-coherent by Proposition~\reff{prop:PushforwardOfQcIsQc} and $\Mm|_U\simeq \Nn$.
\end{proof}
\begin{defi}
	\begin{alphanumerate}
		\item \lbl{def:AffineMorphism}A morphism $X\morphism[f]Y$ between preschemes is called \defemph{affine} if it satisfies the equivalent conditions from Proposition~\reff{prop:AffineMorphism}\itememph{a} and \itememph{b}.
		\item It is called \defemph{finite} if in addition for any affine open $U\subseteq Y$, $\Oo_X(f^{-1}(U))$ is a finitely generated $\Oo_Y(U)$-module, or, equivalently, if this holds for sufficiently many $U$ to cover $Y$ (by Lemma~\reff{lem:locFinGen}).
	\end{alphanumerate}
\end{defi}
\begin{rem*}
	In other words, $f$ is finite iff $f$ is affine and $f_*\Oo_X$ is a locally finitely generated $\Oo_Y$-module in the sense of Definition~\reff{def:locFinGenerated}\itememph{a}. If $Y$ is Noetherian, this is equivalent to $f$ being affine and $f_*\Oo_X$ a \emph{coherent} $\Oo_Y$-module in the sense of Definition~\reff{def:locFinGenerated}\itememph{b}.
\end{rem*}
\begin{cor}\lbl{cor:AffinePushforwardIso}
	If $X\morphism[f]Y$ is an affine morphism and $\Mm$ a quasi-coherent $\Oo_X$-module, then $R^pf_*\Mm=0$ when $p>0$. When $Y$ is a quasi-compact scheme, the morphism
	\begin{align*}
		H^p(Y,f_*\Mm)\xrightarrow{\text{\eqreff{eq:pull-back}}}H^p(X,\Mm)
	\end{align*} 
	is an isomorphism.
\end{cor}
\begin{proof}
	The first assertion follows from the fact that $R^pf_*\Mm$ is the sheafification of $U\mapsto H^p(f^{-1}(U),\Mm)$, which vanishes on the topology base of affine open subsets $U\subseteq Y$ when $p>0$ by Theorem~\reff{thm:CohoOnSchemes}\itememph{b}.
	
	When $Y$ is a quasi-compact scheme, then $X$ is a scheme as well since $X\morphism\Spec \IZ$ is separated as the composition of the separated morphisms $f$ and $Y\morphism\Spec \IZ$. Also, $X$ is quasi-compact as $f$ and $Y$ are quasi-compact. Let $\Uu_Y$ be the open cover of $Y$ by its affine open subsets, then
	\begin{align*}
		H^p\left(Y,f_*\Mm\right)\simeq \check{H}^p\left(\Uu_Y,f_*\Mm\right)\simeq \check{H}^p\left(f^{-1}(\Uu_Y),\Mm\right)\simeq H^p\left(X,\Mm\right)
	\end{align*}
	(which was our description of \eqreff{eq:pull-back}) where the last isomorphism holds as $f$ is affine, hence $f^{-1}(\Uu)$ is an affine open cover of $X$.
	
	This proves the second assertion under the assumptions which we constructed the relevant cohomology groups.
\end{proof}
\begin{rem*}
	Using the general derived functor construction it follows from the Leray spectral sequence
	\begin{align*}
		E_2^{p,q}=H^p(X,R^qf_*\Mm)\converge H^{p+q}(Y,\Mm)
	\end{align*}
	and the vanishing of $R^qf_*\Mm$ when $q>0$, $\Mm$ is quasi-coherent and $f$ affine.
\end{rem*}
Let $X$ be any prescheme. Recall the construction of the $\SPEC_X(-)$ functor from \cite[Subsection~2.5.1]{alggeo1}, which is given as follows. When $\Aa$ is a quasi-coherent $\Oo_X$-algebra, one has schemes $\Spec \Aa(U)\morphism \Spec \Oo_X(U)\simeq U$ when $U\subseteq X$ is an affine open subset. When $V\subseteq U$ is another affine open subset, then $\Spec A(V)$ is isomorphic to the preimage of $V$ under $\Spec \Aa(U)\morphism U$. This allows one to glue the different $\Spec \Aa(U)$ together to one prescheme $\SPEC_X(\Aa)\morphism X$ which also admits the following description:
\begin{itemize}
	\item \emph{Points.} As a set,
	\begin{align*}
	\SPEC_X(\Aa) = \left\{(x,\pp) \st 
	\begin{array}{c}
	x\in X\text{, }\pp \in \Spec(\Aa_x) \text{ such that the preimage of }\pp\\
	\text{under } \Oo_{X,x}\morphism \Aa_x \text{ is } \mm_{X,x}\text{ (the maximal ideal of }\Aa_x)
	\end{array}
	\right\}
	\end{align*}
	\item \emph{Topology base.} Let
	\begin{align*}
	\Omega(U,\lambda) = \left\{ (x,\pp) \st x\in U\text{ and }\pp\not\ni (\text{image of } \lambda \text{ under } \Aa(U)\to \Aa_x)\right\}
	\end{align*}
	where $U\subseteq X$ is open and $\lambda \in \Aa(U)$. As $\Omega(U,\lambda) \cap \Omega(V,\vartheta) = \Omega(U\cap V, \lambda|_{U\cap V} \cdot \vartheta|_{U\cap V})$ this is indeed a base for some topology.
	\item \emph{Local rings.} There is only one reasonable choice and this is $\Oo_{Y,[x,\pp]} \coloneqq (\Aa_x)_\pp$. Here, $Y=\SPEC_X(\Aa)$ for brevity.
	\item \emph{Structure sheaf.} For $\Omega\subseteq Y$ open, let
	\begin{align*}
	\Oo_Y(\Omega) = \left\{\alpha=(\alpha_{x,\pp})\in \prod_{(x,\pp)\in \Omega} \Oo_{Y,[x,\pp]}\st 
	\begin{array}{c}
	\alpha\text{ fulfills the \emph{cohe-}}\\
	\text{\emph{rence condition}\texttrademark}
	\end{array}
	\right\}\;.
	\end{align*}
	Here, the \emph{coherence condition}\texttrademark\ says that every $\omega\in \Omega$ has a neighbourhood $\Omega(U,\lambda)$ together with an $a\in \Aa(U)_\lambda$ such that whenever $(x,\pp)\in \Omega\cap \Omega(U,\lambda)$, we have
	\begin{align*}
	\alpha_{x,\pp}=\left(
	\begin{array}{c}
	\text{image of }a\text{ under }\\
	\Aa(U)_\lambda \morphism (\Aa_x)_\lambda \morphism (\Aa_x)_\pp = \Oo_{Y,[x,\pp]}
	\end{array}\right)\;.
	\end{align*}
\end{itemize}
\begin{prop}\lbl{prop:SPECAdjunction}
	Let $Y$ be a prescheme, $\Aa$ an $\Oo_Y$-algebra and $X\morphism[\xi]Y$ a morphism of preschemes.
	\begin{alphanumerate}
		\item The morphism $\Spec_Y(\Aa)\morphism Y$ is affine.
		\item One has a bijection
		\begin{align}\lbl{eq:SPECAdjunction}
			\Hom_{\Oo_Y\cat{-Alg}}(\Aa,\xi_*\Oo_X)\lisomorphism\Hom_{Y\!\cat{-PreSch}}(X,\SPEC_Y(\Aa))
		\end{align}
		by gluing the adjunctions 
		\begin{align*}
			\Hom_{\Oo_Y(U)\cat{-Mod}}\left(\Aa(U),\Oo_X(\xi^{-1}U)\right)\simeq \Hom_{\cat{PreSch}}\left(\xi^{-1}U,\Spec \Aa(U)\right)
		\end{align*}
		together.
		\item The morphism $\xi$ is affine iff $\xi_*\Oo_X$ is a quasi-coherent $\Oo_Y$-algebra and the morphism $X\morphism\SPEC_Y(\xi_*\Oo_X)$ corresponding to $\id_{\xi_*\Oo_X}$ under \eqreff{eq:SPECAdjunction} is an isomorphism.
	\end{alphanumerate}
\end{prop}
\begin{proof}[Sketch of a proof]
	For \itememph{a}, the preimage of an affine open $U\subseteq Y$ is $\Spec \Aa(U)$, hence affine. Part \itememph{b} was sketched in \cite[Proposition~2.5.1]{alggeo1} and part \itememph{c} is \cite[Lemma~2.5.1]{alggeo1}.
\end{proof}
\begin{cor}
	Let $\snake{Y}\morphism[f] Y$ be a morphism of preschemes. Let $f^*\colon \Oo_Y\cat{-Alg}\morphism\Oo_{\snake{Y}}\cat{-Alg}$ be the inverse image functor from \hyperref[footnote:f^*]{this} footnote. If $\Aa$ is a $\Oo_Y$-algebra, we have $\SPEC_{\snake{Y}}(f^*\Aa)\simeq \SPEC_Y(\Aa)\times_Y\snake{Y}$.
\end{cor}
\begin{proof}
	Recall that $f^*$ is left-adjoint to $f_*$ Let $X\morphism[\xi]\snake{Y}$ be a $\snake{Y}$-prescheme. Repeatedly applying Proposition~\reff{prop:SPECAdjunction}\itememph{b} we get
	\begin{align*}
		\Hom_{\snake{Y}\cat{-PreSch}}\left(X,\SPEC_{\snake{Y}}(f^*\Aa)\right)&\simeq \Hom_{\Oo_{\snake{Y}}\cat{-Alg}}\left(f^*\Aa,\xi_*\Oo_X\right)\\
		&\simeq\Hom_{\Oo_Y\cat{-Alg}}\left(\Aa,f_*\xi_*\Oo_X\right)\\
		&\simeq \Hom_{Y\!\cat{-PreSch}}\left(X,\SPEC_Y(\Aa)\right)\;.
	\end{align*}
	In the last line, the $Y$-prescheme structure on $X$ is given by $X\morphism[\xi]\snake{Y}\morphism[f]Y$. If you think about it for a moment, a morphism of $Y$-preschemes from $X$ to $\Spec_Y(\Aa)$, where the $Y$-prescheme structure on $X$ factors over $\snake{Y}$ is exactly the commuting square from the universal property of the fibre product $\SPEC_Y(\Aa)\times_Y\snake{Y}$ and we're done.
\end{proof}

\section{The relation between \texorpdfstring{$H^1$}{H} and torsors}
The usual term for these things seems to be \emph{torsor} rather than \emph{torseur}, so we will deviate from the lecture's notation here.
\begin{defi}\lbl{def:torsor}
	Let $X$ be any topological space, $\Gg$ a sheaf of groups on $X$. A \defemph{$\Gg$-torsor} or \defemph{$\Gg$-principal homogeneous space} is a sheaf of sets $\Xx$ on $X$ with a morphism
	\begin{align*}
		\Gg\times \Xx\morphism\Xx
	\end{align*}
	of sheaves of sets indicated by $(g,\xi)\mapsto g\cdot \xi$, with the property that $1_{\Gg(U)}\cdot \xi=\xi$ and $g\cdot(h\cdot \xi)=(gh)\cdot \xi$ for $g,h\in\Gg(U)$ and $\xi\in\Xx(U)$ and such that $\Xx_x\neq \emptyset$ for any $x\in X$ and the action of $\Gg_x$ on $\Xx_x$ is simply transitive for each $x\in X$.
	
	The torsor is called \defemph{trivial} if $\Xx(X)\neq \emptyset$ (and \emph{yes}, the $\neq$ is correct here).
\end{defi}
\begin{example}
	$\Gg$ with its group multiplication is a $\Gg$-torsor and trivial as $1_{\Gg(X)}\in\Gg(X)$.
\end{example}
\begin{rem}
	\begin{alphanumerate}
		\item \lbl{rem:torsorIsoStuff}If $\xi\in\Xx(X)$ then 
		\begin{align*}
			\Gg&\morphism\Xx\\
			g\in\Gg(U)&\longmapsto g\cdot \xi|_U
		\end{align*}
		is an isomorphism (by the simple transitivity stuff it is an isomorphism on stalks).
		\item If $\Xx$ and $\Yy$ are $\Gg$-torsors then any morphism $\Xx\morphism[f]\Yy$ of sheaves of sets compatible with the $\Gg$-actions induces isomorphisms on stalks, i.e., is automatically an isomorphism of sheaves. Thus, the category of $\Gg$-torsors is a \emph{groupoid} -- a category, in which every morphism is an isomorphism.
		\item I think the following should be pointed out separately (though it wasn't in the lecture). Let $\Xx$ be a $\Gg$-torsor, $U\subseteq X$ open. Then $\Gg(U)$ acts simply transitively on $\Xx(U)$. Indeed, If $\Xx(U)=\emptyset$ this is trivial. Otherwise, $\Xx|_U$ clearly is a trivial $\Gg|_U$-torsor by \itememph{a} and the assertion is just as trivial. 
		
		For $\xi,\upsilon\in\Gg(U)$ the unique element $g$ such that $g\cdot \upsilon=\xi$ will be denoted $g=\frac{\xi}{\upsilon}$ (or $\xi-\upsilon$, if the $\Gg$-action is written additively).
	\end{alphanumerate}
\end{rem}
\begin{example}
	\begin{alphanumerate}
		\item \lbl{ex:torsorsVectorBundles}Let $(X,\Oo_X)$ be a ringed space. If $\Ll$ is a \emph{line bundle} on $X$ (i.e., a locally free $\Oo_X$-module of rank $1$) then
		\begin{align*}
			U\mapsto \Ll^\times(U)&=\left\{\ell\in\Ll(U)\st \Oo_U\morphism[\cdot \ell]\Ll|_U\text{ is an isomorphism}\right\}\\
			&=\left\{\ell\in\Ll(U)\st 
			\begin{array}{c}
				\text{the image of }\ell \text{ in }\Ll_x\text{ is a generator of}\\
				\text{that free }\Oo_{X,x}\text{-module for all }x\in U
			\end{array}
			\right\}
		\end{align*}
		is an $\Oo_X^\times$-torsor when equipped with the $\Oo_X^\times$-action given by $(f,\ell)\mapsto f\cdot \ell$.
		
		If $(X,\Oo_X)$ is locally ringed with maximal ideals $\mm_x\subseteq \Oo_{X,x}$ and $\ell\in\Ll(U)$ for some open subset $U\subseteq X$, we put
		\begin{align*}
			V(\ell)=\left\{x\in U\st (\text{image of }\ell\text{ in }\Ll_x)\in\mm_x\Ll_x\right\}\;.
		\end{align*}
		If $\lambda\in\Ll(U)$ happens to be a free generator of $\Ll|_U$, then $V(\ell)$ coincides with $V\left(\frac{\ell}{\lambda}\right)$, the latter vanishing set being taken in $\Oo_X$ as usual (that is, as in \cite[Definition~1.3.3]{alg1}). Then
		\begin{align*}
			\Ll^\times (U)=\left\{\ell\in\Ll(U)\st V(\ell)=\emptyset\right\}\;.
		\end{align*}
		The line bundle $\Ll$ is trivial iff $\Ll^\times$ has a global section, i.e., iff it's a \emph{trivial} torsor in  the sense of Definition~\reff{def:torsor}. It is obvious that for an isomorphism $\phi\colon \Ll\isomorphism\Mm$ of line bundles, we have an isomorphism $\Phi\colon \Ll^\times \isomorphism\Mm^\times$ sending $\ell\in\Ll^\times(U)$ to $\phi(\ell)\in\Mm^\times (U)$. Vice versa, if $\Phi\colon \Ll^\times \isomorphism\Mm^\times $ is an isomorphism of $\Oo_X^\times$-torsors (or just a morphism -- it's automatically an isomorphism by Remark~\reff{rem:torsorIsoStuff}\itememph{b}), there is a unique isomorphism $\phi\colon \Ll\isomorphism\Mm$ such that $\Phi(\ell)=\phi(\ell)$ for all open $U\subseteq X$ and $\ell\in\Ll^\times(U)$.
		
		We thus get a bijection between the isomorphism classes of line bundles and the isomorphism classes of $\Oo_X^\times$-torsors. As was pointed out, we need to check that every $\Oo_X^\times$-torsor is given as $\Ll^\times$ for some line bundle $\Ll$. If $\Xx$ is an $\Oo_X^\times$-torsor, define $\Ll$ as the sheafification of 
		\begin{align*}
			U\mapsto (f,\xi)/_\sim\quad\text{where }f\in\Oo_X(U)\text{, }\xi\in\Xx(U)
		\end{align*}
		and the equivalence relation $\sim$ is defined by $(f,\xi)\sim (g,\upsilon)$ iff $f=g\cdot\frac{\xi}{\upsilon}$. $\Ll$ becomes an $\Oo_X$-module via $h\cdot (f,\xi)/_\sim=(fh,\xi)/_\sim$ and $(f,\xi)/_\sim+(g,\upsilon)/_\sim=\left(g+f\cdot \frac{\xi}{\upsilon},\upsilon\right)/_\sim$.
		\item Let $(X,\Oo_X)$ be a locally ringed space again. Define a sheaf $\GL_n(\Oo_X)$ by
		\begin{align*}
			\GL_n(\Oo_X)(U)&=\left\{g\in\Oo_X(U)^{n\times n}\st V(\det g)=\emptyset\right\}\\
			&=\left\{g\in\Oo_X(U)^{n\times n}\st g\text{ has an inverse matrix}\right\}
		\end{align*}
		Then there is a similar bijection
		\begin{align*}
			\left\{\begin{array}{c}
				\text{isomorphism classes of }n\text{-dimen-}\\
				\text{sional vector bundles on }X
			\end{array}
			\right\} & \isomorphism\left\{\begin{array}{c}
			\text{isomorphism classes of}\\
			\GL_n(\Oo_X)\text{-torsors}
			\end{array}
			\right\}\\
			[\Ee]&\longmapsto \left[\left\{e\in\Ee(U)^n\st 
			\begin{array}{c}
			e\text{ is a vector of free}\\
			\text{generators of }\Ee|_U
			\end{array}
			\right\}\right]\;,
		\end{align*}
		where $\GL_n(\Oo_X)$ acts by right multiplication with the column vector $e$.
	\end{alphanumerate}
\end{example}
Let $\Gg$ be a sheaf of abelian groups, $\Xx$ a $\Gg$-torsor, $\Uu\colon X=\bigcup_{i\in I}U_i$ an open cover such that $\Xx|_{U_i}$ is trivial (we will then say that $\Xx$ is trivial on $\Uu$). To be compatible with our notation for \vC ech cohomology we will write the $\Gg$-action on $\Xx$ additively. If $\xi_i\in\Xx(U_i)$ for all $i\in I$ are given, there are unique $\psi_{i,j}\coloneqq \xi_i|_{U_{i,j}}-\xi_j|_{U_{i,j}}\in\Gg(U_{i,j})$ as in Remark~\reff{rem:torsorIsoStuff}\itememph{c}. Let $\psi=(\psi_{i,j})_{(i,j)\in I^2}\in\check{C}^1(\Uu,\Gg)$. One easily verifies the correctness of the formal calculation
\begin{align*}
	(\hacek{d}^1\psi)_{i,j,k}&=\psi_{j,k}|_{U_{i,j,k}}-\psi_{i,k}|_{U_{i,j,k}}+\psi_{i,j}|_{U_{i,j,k}}\\
	&=\left(\xi_j|_{U_{i,j,k}}-\xi_k|_{U_{i,j,k}}\right)-\left(\xi_i|_{U_{i,j,k}}-\xi_k|_{U_{i,j,k}}\right)+\left(\xi_i|_{U_{i,j,k}}-\xi_j|_{U_{i,j,k}}\right)\\
	&=0\;,
\end{align*}
that is, $\psi$ is a cocycle.

\begin{prop}
	\begin{alphanumerate}
		\item \lbl{prop:torsorsH1Bij}The cohomology class $[\psi]\in \check{H}^1(\Uu,\Gg)$ depends only on the isomorphism class of the $\Gg$-torsor $\Xx$, not on the choice of trivializations.
		\item One gets a bijection between the isomorphism classes of $\Gg$-torsors trivial on $\Uu$ and $\check{H}^1(\Uu,\Gg)$.
	\end{alphanumerate}
\end{prop}
\begin{proof}
	Part \itememph{a}. If $\snake{\xi}$ is another trivialization yielding $\snake{\psi}_{i,j}=\snake{\xi}_i|_{U_{i,j}}-\snake{\xi}_j|_{U_{i,j}}$, then $\snake{\xi}_i=\gamma_i+\xi_i$ where $\gamma_i=\snake{\xi}_i-\xi_i\in\Gg(U_i)$ (Remark~\reff{rem:torsorIsoStuff}\itememph{c}) and putting $\gamma=(\gamma_i)_{i\in I}\in \check{C}^0(\Uu,\Gg)$ we get $\snake{\psi}=\hacek{d}^0\gamma+\psi$.
	
	Part \itememph{b}. By \itememph{a} it is clear that we have a map from isomorphism classes of $\Gg$-torsors trivial on $\Uu$ to $\check{H}^1(\Uu,\Gg)$. To construct a converse map, let $\eta\in \check{H}^1(\Uu,\Gg)$ and let $\psi\in\check{C}^1(\Uu,\Gg)$ be a representative of $\eta$ (so that $\hacek{d}^1\psi =0$) and let $\Xx_\psi$ be the sheaf of sets on $X$ given by
	\begin{align*}
		\Xx_\psi(V)=\left\{\gamma\in \check{C}^0(\Uu\cap V,\Gg)\st \hacek{d}^0\gamma=\psi|_V\right\}\;.
	\end{align*}
	We get a $\Gg$-action on $\Xx_\psi$ via $g+\gamma=(g|_{U_i\cap V}+\gamma_i)_{i\in I}$ for $g\in\Gg(V)$, $\gamma\in\check{C}^0(\Uu\cap V,\Gg)$. We verify that $\Gg$ acts simply transitively on stalks and that $\Gg$ trivializes on $\Uu$. Note that if $\gamma,\gamma'\in \Xx(V)$, then $\hacek{d}^0(\gamma-\gamma')=0$, hence $\gamma-\gamma'\in\check{C}^0(\Uu\cap V,\Gg)$ corresponds to (unique) element of $\Gg(V)$. So the action on $\Xx_\psi(V)$ is simply transitive for every $V\subseteq X$ open and the only thing left to check is triviality on $\Uu$. For this, note that $\check{H}^1(\Uu\cap V,\Gg|_V)=0$ by Corollary~\reff{cor:refinementPullback}\itememph{c}, so $\psi|_V$ has a preimage in $\check{C}^0(\Uu\cap V,\Gg)$ and consequently $\Xx_\psi(V)\neq\emptyset$. In particular, the stalks $(\Xx_\psi)_x$ are non-empty and $\Xx_\psi$ is a $\Gg$-torsor trivial on $\Uu$.
	
	If $\snake{\psi}=\psi+\hacek{d}^0\phi$ with $\phi\in\check{C}^0(\Uu,\Gg)$ then we have an isomorphism of $\Gg$-torsors
	\begin{align*}
		\Xx_\psi&\isomorphism\Xx_{\snake{\psi}}\\
		\gamma\in\Xx_\psi(V) &\mapsto \gamma+\phi|_V\;.
	\end{align*}
	Therefore, we get a map in the opposite direction
	\begin{align*}
		\check{H}^1(\Uu,\Gg)&\morphism\left\{\begin{array}{c}
			\text{isomorphism classes of}\\
			\Gg\text{-torsors trivial on }\Uu
		\end{array}\right\}\\
		\eta=[\psi]&\longmapsto[\Xx_\psi]\;.
	\end{align*}
	We claim that the maps we constructed from $\check{H}^1(\Uu,\Gg)$ to the set of isomorphism classes of $\Uu$-trivial $\Gg$-torsors and vice versa are inverse to each other. 
	
	Indeed, we may choose $\xi_i\in\Xx_\psi(U_i)$ such that $(\xi_i)_j=\psi_{ij}$ -- using $\hacek{d}^1\psi=0$ they indeed satisfy $\hacek{d}^0\xi_i=\psi|_{U_{i,j}}$ -- and using $\hacek{d}^1\psi=0$ again we find $\xi_i|_{U_{i,j}}-\xi_j|_{U_{i,j}}=\psi_{i,j}$. So we send $[\Xx_\psi]$ to the cohomology class $[\psi]$ and then back to $[\Xx_\psi]$.
	
	And for the other direction, if an arbitrary $\Uu$-trivial $\Gg$-torsor $\Xx$ is given, choose trivializations $\xi_i\in\Xx(U_i)$ and put $\psi=(\xi_i|_{U_{i,j}}-\xi_j|_{U_{i,j}})_{(i,j)\in I^2}\in\check{C}^1(\Uu,\Gg)$. Then we get an isomorphism $\Xx\isomorphism\Xx_\psi$ via
	\begin{align*}
		\Xx(V)&\morphism \Xx_\psi(V)\\
		\upsilon &\longmapsto \left(\upsilon|_{U_i\cap V}-\xi_i|_{U_i\cap V}\right)_{i\in I}\in\check{C}^1(\Uu\cap V,\Gg)
	\end{align*}
	(this is a morphism of $\Gg$-torsors, hence an isomorphism by Remark~\reff{rem:torsorIsoStuff}\itememph{b}), so $[\Xx]$ is mapped to $[\psi]$ and then back to $[\Xx_\psi]=[\Xx]$ again.
\end{proof}
\begin{rem*}
	When $\Gg$ is any sheaf of groups on $X$, one can define $H^1(X,\Gg)$ as the set of isomorphism classes of $\Gg$-torsors on $X$. This is a pointed set, the base point being the isomorphism class of the trivial torsor. Thus, the isomorphism classes of line bundles and of $n$-dimensional vector bundles on $X$ are in canonical bijection with $H^1(X,\Oo_X^\times)$ and $H^1(X,\GL_n(\Oo_X))$ respectively by Example~\reff{ex:torsorsVectorBundles}.
\end{rem*}
\begin{cor}
	If $X$ is an affine scheme, $\Mm$ a quasi-coherent $\Oo_X$-module and $\Xx$ an $\Mm$-torsor on $X$, then $\Xx$ is trivial.
\end{cor}
\begin{proof}
	Let $\Uu$ be an open cover of $X$ on which $\Xx$ trivializes. We may replace $\Uu$ by an affine refinement. Then $\check{H}^1(\Uu,\Mm)=H^1(X,\Mm)=0$ by Theorem~\reff{thm:CohoOnSchemes}\itememph{a} and \itememph{b}, proving that $\Xx$ is trivial by Proposition~\reff{prop:torsorsH1Bij}\itememph{b}.
\end{proof}
If $X$ is an $S$-prescheme, we have a homomorphism of abelian groups
\begin{align*}
	d\log\colon\Oo_X^\times &\morphism \Omega_{X/S}\\
	f&\longmapsto d\log f=\frac{d_{X/S}f}{f}
\end{align*}
defining a morphism of \vC ech cohomology groups $\check{H}^1(X,\Oo_X^\times)\morphism \check{H}^1(X,\Omega_{X/S})$. When no open cover is specified, we put $\check{H}^1(X,\Ff)=\colimit[\Uu]\check{H}^1(\Uu,\Ff)$ for a presheaf $\Ff$ on $X$, where the colimit is taken over all open covers $\Uu$ of $X$, partially ordered by refinement. We have to use \vC ech cohomology here since we don't have any other available, unless $X$ is a quasi-compact scheme and $\Ff$ a quasi-coherent $\Oo_X$-module. In this case, the affine open covers are cofinal in the poset of open covers and $\check{H}^1(X,\Ff)=H^1(X,\Ff)$ is what we expect.
\begin{defi}
	Let $\Ll$ be line bundle on $X$. The image under the above map of the element of $\check{H}^1(X,\Oo_X^\times)$ associated to the torsor $\Ll^\times$ by Proposition~\reff{prop:torsorsH1Bij}\itememph{b} is called the \defemph{first Chern class} of $\Ll$ and is denoted
	\begin{align*}
		c_1(\Ll)\in \check{H}^1(X,\Omega_{X/S})\;.
	\end{align*}
\end{defi}
\begin{defi}
	Let $\Vv$ and $\Ww$ be vector bundles on the locally ringed space $(X,\Oo_X)$. An \defemph{extension} of $\Ww$ by $\Vv$ is a short exact sequence 
	\begin{align*}
		0\morphism \Vv\morphism[a]\Ee\morphism[b]\Ww\morphism 0\;.
	\end{align*}
	A \defemph{morphism of extensions} is a commutative diagram
	\begin{diagram*}
		\node[ob] (A) at (0,1.5) {$0$};
		\node[ob] (B) at (1.5,1.5) {$\Vv$};
		\node[ob] (C) at (3,1.5) {$\Ee$};
		\node[ob] (D) at (4.5,1.5) {$\Ww$};
		\node[ob] (E) at (6,1.5) {$0$};
		\node[ob] (A') at (0,0) {$0$};
		\node[ob] (B') at (1.5,0) {$\Vv$};
		\node[ob] (C') at (3,0) {$\Ee'$};
		\node[ob] (D') at (4.5,0) {$\Ww$};
		\node[ob] (E') at (6,0) {$0$};
		\scriptsize
		\draw[->] (A) -- (B);
		\draw[->] (B) -- (C) node[pos=0.5,above] {$a$};
		\draw[->] (C) -- (D) node[pos=0.5,above] {$b$};
		\draw[->] (D) -- (E);
		\draw[->] (A') -- (B');
		\draw[->] (B') -- (C') node[pos=0.5,above] {$a'$};
		\draw[->] (C') -- (D') node[pos=0.5,above] {$b'$};
		\draw[->] (D') -- (E');
		\draw[transform canvas={xshift=1pt}] (B) -- (B');
		\draw[transform canvas={xshift=-1pt}] (B) -- (B');
		\draw[->] (C) -- (C') node[pos=0.5, left] {$\epsilon$};
		\draw[transform canvas={xshift=1pt}](D) -- (D');
		\draw[transform canvas={xshift=-1pt}](D) -- (D');
	\end{diagram*}
\end{defi}
Note that $\epsilon$ in the above diagram is automatically an isomorphism. Indeed, this follows immediately from the five lemma in the abelian category $\Oo_X\cat{-Mod}$.

For sheaves $\Ff$ and $\Gg$ on a topological space $X$, we get another sheaf on $X$ via 
\begin{align*}
	U\longmapsto \Hom_{\cat{Shv}(U)}(\Ff|_U,\Gg|_U)\quad\text{for open subsets }U\subseteq X\;.
\end{align*}
This sheaf is denoted $\Hhom(\Ff,\Gg)$. If $\Rr$ is  a sheaf of rings on $X$ and $\Ff$ and $\Gg$ are $\Rr$-modules, we can restrict to homomorphisms respecting the $\Rr$-module structures and obtain a sheaf $\Hhom_\Rr(\Ff,\Gg)$ by $\Hhom_{\Rr}(\Ff,\Gg)(U)=\Hom_{\Rr|_U\cat{-Mod}}(\Ff|_U,\Gg|_U)$ for $U\subseteq X$ open.
\begin{prop}
	Let $(X,\Oo_X)$ be a locally ringed space and $\Vv,\Ww$ vector bundles on $X$. Associating to any extension $\Ee$ of $\Ww$ by $\Vv$ the $\Hhom_{\Oo_X}(\Ww,\Vv)$-torsor $\Xx$ given by
	\begin{align*}
		\Xx(U)=\left\{\text{morphisms }\Ee|_U\morphism[\pi]\Vv|_U\text{ of }\Oo_U\text{-modules}\st \pi a=\id_{\Vv|_U}\right\}
	\end{align*}
	for $U\subseteq X$ open, whose $\Hhom_{\Oo_X}(\Ww,\Vv)$-action is given by $h\in\Hom_{\Oo_U\cat{-Mod}}(\Ww|_U,\Vv|_U)$ sending $\pi$ to $\pi+h b$) is an equivalence of groupoids between extensions of $\Ww$ by $\Vv$ and $\Hhom_{\Oo_X}(\Ww,\Vv)$-torsors on $X$.
\end{prop}
Franke didn't prove this in the lecture. For the sake of completeness, however, I will do my best to sketch the important steps.
\begin{proof}[Sketch of a proof]
	Denote $\Hh=\Hhom_{\Oo_X}(\Ww,\Vv)$ for short and let $\Xx_\Ee$ be the $\Hh$-torsor associated to an extension $\Ee$. Let's first give an argument why $\Xx_\Ee$ is even an $\Hh$-torsor. If $\pi,\pi'\in\Xx_\Ee(U)$ are two sections of $a$ on $U$, then $(\pi-\pi')a=0$, hence $\pi-\pi'$ factors uniquely over the cokernel of $\Vv_U\morphism[a]\Ee|_U$, which is $\Ww|_U$. That is, there is a unique morphism $\Ww|_U\morphism[h]\Vv|_U$ of $\Oo_U$-modules such that $\pi-\pi'=hb$. This shows that $\Hh(U)$ acts simply transitively on $\Xx_\Ee(U)$. It remains to show that the stalks $\Xx_{\Ee,x}$ are non-empty. But $\Vv$ and $\Ww$ are vector bundles, so every $x\in X$ has a neighbourhood $U$ in which $\Ww|_U$ and $\Vv|_U$ are free $\Oo_U$-modules. In particular, $\Ww|_U$ is projective and the extension $\Ee|_U$ is split, i.e., $\Ee|_U\simeq \Ww|_U\oplus \Vv|_U$. Then $\Xx_\Ee(V)$ is non-empty for all open subsets $V\subseteq U$, hence $\Xx_{\Ee,x}\neq \emptyset$.
	
	\newlength{\currentparskip}
	\setlength{\currentparskip}{\parskip}\begin{tikzfadingfrompicture}[name=premium access]
		\shade[top color=transparent!50, bottom color=transparent!6.25] (0,0) rectangle (2,2);
	\end{tikzfadingfrompicture}
	\begin{tikzpicture}[inner sep=0pt]
		\node (wallOfText) at (0,0) {
			\begin{minipage}{\textwidth}
			\setlength{\parskip}{\currentparskip}
				Now about the functoriality properties. To a morphism $\Ee\morphism[\epsilon]\Ee'$ of extensions we associate the morphism $\Xx_\Ee\morphism\Xx_{\Ee'}$ which takes $\pi\in\Xx_\Ee(U)$ to $\epsilon^{-1}\pi\in\Xx_{\Ee'}(U)$ (recall that $\epsilon$ is automatically an isomorphism; we could have taken $\epsilon\pi$ instead to get a contravariant equivalence of groupoids). This gives a functor $\Phi$ from the groupoid of extensions of $\Ww$ by $\Vv$ to the groupoid of $\Hh$-torsors. 
				
				We show fully faithfulness first. Faithfulness of $\Phi$ is straightforward: If $\epsilon,\epsilon'\colon \Ee\morphism\Ee'$ are morphisms of extensions such that $\Phi(\epsilon)=\Phi(\epsilon')$, then choose an open cover $\Uu\colon X=\bigcup_{i\in I}U_i$ on which $\Vv$ and $\Ww$ trivialize (i.e., $\Vv|_{U_i}$ is a free $\Oo_{U_i}$-module for all $i\in U$ and same for $\Ww$). Then $\Ee|_{U_i}\simeq \Ww|_{U_i}\oplus \Vv|_{U_i}$ is a split extension (and same for $\Ee'|_{U_i}$), hence the $\Xx_\Ee(U_i)$ are non-empty and we may choose $\pi_i\in \Xx_\Ee(U_i)$ for all $i\in I$. From $\epsilon^{-1}\pi_i=(\epsilon')^{-1}\pi_i$ and splitness of $\Ee|_{U_i},\Ee'|_{U_i}$ it's easy to deduce $\epsilon=\epsilon'$ on $U_i$. Since the $U_i$ cover $X$, this holds on all of $X$ and we are done.
				
				Now suppose $\Xx_\Ee\morphism[\phi]\Xx_{\Ee'}$ is a morphism of $\Hh$-torsors. We need to construct a morphism $\Ee\morphism[\epsilon]\Ee'$ of extensions such that $\Phi(\epsilon)=\phi$. If $\pi\in\Xx_\Ee(U)$ is a section of $a$ on $U$, then $\pi$ and $\phi(\pi)$ induce splittings $\Ee|_U\simeq \Ww|_U\oplus\Vv|_U\simeq \Ee'|_U$, which immediately gives a morphism $\Ee|_U\morphism[\epsilon_U]\Ee'|_U$ of extensions of $\Ww|_U$ by $\Vv|_U$. It can be easily checked that $\epsilon_U$ is actually independent of $\pi$. So choosing a trivializing open cover $\Uu$ as above, we get $\epsilon_{U_i}$ for all $i\in I$ which can be glued together (by naturality of their construction) to a morphism $\Ee\morphism[\epsilon]\Ee'$ which indeed satisfies $\Phi(\epsilon)=\phi$.
				
				And essential surjectivity. Let $\Xx$ be an $\Hh$-torsor. To construct an extension $\Ee$ such that $\Xx\simeq \Xx_\Ee$, the idea is -- of course -- to choose $\Ee|_U\simeq \Ww|_U\oplus\Vv|_U$ on small neighbourhoods where $\Xx$ trivializes, and then to glue them together somehow. Actually we will take the smallest possible ``neighbourhoods'' -- the stalks $\Ee_x\simeq\Ww_x\oplus\Vv_x$ and impose a suitable coherence condition. To do this, choose an element $\pi_x\in\Xx_x$ for every $x\in X$ and put
				\begin{align*}
				\Ee(U)=\left\{e=(w_x,v_x)_{x\in U}\in \prod_{x\in U}\Ww_x\oplus\Vv_x\st\begin{array}{c}
				e\text{ fulfills the \emph{cohe-}}\\
				\text{\emph{rence condition}\texttrademark}
				\end{array}\right\}\;,
				\end{align*}
				where the \emph{coherence condition}\texttrademark\ says that every $x\in U$ has a neighbourhood $V\subseteq U$ on which $\pi_x$, $w_x$ and $v_x$ have preimages $\pi_V\in \Xx(V)$, $w_V\in\Ww(V)$ and $v_V\in\Vv(V)$ such that the following holds: $\pi_{V,y}$, $w_{V,y}$ and $v_{V,y}$ denote the respective images in $\Xx_y$, $\Ww_y$ and $\Vv_y$, then $w_y=w_{V,y}$ and $v_y=v_{V,y}+h(w_y)$, where $h=\pi_y-\pi_{V,y}\in\Hh_y$. It is possible to show that this indeed defines an extension of $\Ww$ by $\Vv$ and that $\Xx\simeq\Xx_\Ee$.
			\end{minipage}};
		\coordinate (dummy) at ($(wallOfText.north east)!1.0/2!(wallOfText.south east)$);
		\fill[white,path fading=premium access] (wallOfText.north west) rectangle (dummy);
		\fill[white,opacity=0.9375] (dummy) rectangle (wallOfText.south west);
		\node[rectangle, fill=white,draw=black!50!white, inner sep=3ex, rounded corners] at (0,0) {\begin{minipage}{0.5\textwidth}
			\sffamily
			\begin{center}
			\LARGE\textbf{Want to read more?}
			\end{center}
			\vspace{2ex}
			Subscribe to continue reading (or wait till exercise sheet \#7 has been collected).
			\begin{itemize}
			\item[{\tikz[inner sep=0ex,anchor=base, baseline]{\node (a) {\phantom{h}};\draw[red!70!black] (a) circle (5pt);}}] Purchase this proof for 0,99\euro.
			\item[{\tikz[inner sep=0ex,anchor=base, baseline]{\node (a) {\phantom{h}};\draw[red!70!black] (a) circle (5pt);\fill[red!70!black] (a) circle (3pt);}}] Unlock AlgGeo II PREMIUM. Unlimited access on all devices. Free 30 days trial, then 4,99\euro\ per month.
			\end{itemize}
			\begin{center}
			\tikz\fill[inner sep=0ex,red!70!black, rounded corners]  (0,0.75) rectangle (\textwidth,0) node[pos=0.5,white] {\large\textbf{Get access now}};
			\end{center}
			\vspace{2ex}
			Already have an account? \textbf{\underline{Si\smash{g}n in}} here.
			\end{minipage}};
	\end{tikzpicture}
\end{proof}

\chapter{Cohomology of projective spaces}
\section{Regular sequences and the Koszul complex}
We will use the following conventions: 
\begin{defi}
	\begin{alphanumerate}
		\item \lbl{def:cone}For a cochain complex $(C^\bullet, d_{C}^\bullet)$ let the \defemph{shift} $C[p]^\bullet$ be the cochain complex defined by $C[p]^q=C^{p+q}$ and whose differential $d_{C[p]}^\bullet$ is given by $d_{C[p]}^q=(-1)^pd_{C}^{p+q}$.
		\item Let $C^\bullet\morphism[\phi]\snake{C}^\bullet$ be a morphism of cochain complexes. Then the \defemph{cone} $\Cone(\phi)$ is the cochain complex defined by 
		\begin{align}\lbl{eq:cone}
			\Cone(\phi)^p=\snake{C}^p\oplus C^{p+1}\quad\text{and}\quad d_{\Cone(\phi)}^p(\snake{c},c)=\left(d_{\snake{C}}^p(\snake{c})+\phi(c),-d_{C}^{p+1}(c)\right)\;.
		\end{align}
	\end{alphanumerate}
\end{defi}
\begin{rem*}
	\begin{alphanumerate}
		\item Obviously, $H^q(C[p]^\bullet)=H^{p+q}(C^\bullet)$.
		\item Note that $d^\bullet=d_{\Cone(\phi)}^ \bullet$ is indeed a differential of cochain complexes, as
		\begin{align*}
			d^{p+1}d^p(\snake{c},c)&=d^{p+1}\left(d_{\snake{C}}^p(\snake{c})+\phi(c),-d_{C}^{p+1}(c)\right)\\
			&=\left(d_{\snake{C}}^{p+1}d_{\snake{C}}^p(\snake{c})+d_{\snake{C}}^{p+1}\phi(c)-\phi d_{C}^{p+1}(c),d_{C}^{p+2}d_{C}^{p+1}(c)\right)\\
			&=0\;.
		\end{align*}
		We get a short exact sequence
		\begin{align*}
			0\morphism \snake{C}^\bullet\morphism\Cone(\phi)\morphism C[1]^\bullet\morphism 0
		\end{align*}
		of cochain complexes. The connecting homomorphism $H^{p+1}(C^\bullet)=H^p(C[1]^\bullet)\morphism[\partial]H^{p+1}(\snake{C}^\bullet)$ coincides with the morphism induced by $\phi$. We thus get a long exact cohomology sequence
		\begin{align}\lbl{eq:MappingConeSeq}
			\ldots\morphism H^p(C^\bullet)\morphism[\phi]H^p(\snake{C}^\bullet)\morphism H^p(\Cone(\phi))\morphism H^{p+1}(C^\bullet)\morphism\ldots\;.
		\end{align}
		In particular, $\phi$ induces isomorphisms on cohomology iff $H^\bullet(\Cone(\phi))$ vanishes.
	\end{alphanumerate}
\end{rem*}
Naturally arising in the computation of cohomology of projective spaces, \emph{regular sequences} are interesting enough to be studied on their own.
\begin{defi}
	Let $R$ be a ring and $M$ an $R$-module. A sequence $(x_0,\ldots,x_n)$ of elements of $R$ is called \defemph{$M$-regular} if for $i=0,\ldots,n$
	\begin{align*}
		M/(x_0M+\ldots+x_{i-1}M)\morphism[x_i\cdot]M/(x_0M+\ldots+x_{i-1}M)
	\end{align*}
	is an injective map. A \defemph{regular sequence} in $R$ is an $M$-regular sequence where $M=R$.
\end{defi}
\begin{example}
	\begin{alphanumerate}
		\item \lbl{ex:(X0Xn)regular}If $S$ is any ring and $R=S[X_0,\ldots,X_n]$, then the sequence $(X_0,\ldots,X_n)$ is $R$-regular. Indeed, $R/(X_0R+\ldots+X_{i-1}R)\simeq S[X_i,\ldots,X_n]$ and multiplication by $X_i$ in the polynomial ring $S[X_i,\ldots,X_n]$ is clearly injective.
		\item The sequence $(1,0)$ is always regular, but $(0,1)$ is only regular when $R=0$.
	\end{alphanumerate}
\end{example}
The following definition was lacking the $m=0$ exception in the lecture (let alone the indices), but I think the following is what Professor Franke intended.
\begin{defi}\lbl{def:KoszulComplex}
	Let $R$ be a ring, $M$ be an $R$-module, $(x_0,\ldots,x_n)$ a sequence of elements of $R$. The \defemph{Koszul complex} $K^\bullet\big((x_0,\ldots,x_n),M\big)$ is a cochain complex defined as follows. Let $[n]=\{0,\ldots,n\}$. For $m>1$, $K^m\big((x_0,\ldots,x_n),M\big)$ is the collection of maps $f\colon [n]^m\morphism M$ with the following properties:
	\begin{alphanumerate}
		\item $f(i_1,\ldots,i_m)=0$ when $i_k=i_\ell$ for some integers $0\leq k<\ell\leq m$.
		\item $f(i_{\pi(1)},\ldots,i_{\pi(m)})=\sgn(\pi)f(i_1,\ldots,i_m)$ for every permutation $\pi\in\SS_m$.
	\end{alphanumerate}
	The differential $d^m\colon K^m\big((x_0,\ldots,x_n),M\big)\morphism K^{m+1}\big((x_0,\ldots,x_n),M\big)$ is given by 
	\begin{align*}
		d^mf=\sum_{j=0}^m(-1)^jd_jf\quad\text{where}\quad d_jf(i_1,\ldots,i_{m+1})=x_{i_{j+1}}f(i_1,\ldots,\hat{i}_{j+1},\ldots,i_{m+1})\;.
	\end{align*}
	As in Definition~\reff{def:cech}, the hat $\hat{i}_{j+1}$ denotes the omission of $i_{j+1}$. Moreover, for $m=0$ we put $K^0\big((x_0,\ldots,x_n),M\big)=M$ and $d^0\mu$ for $\mu\in M$ is the coaugmentation given by $d^0\mu(i)=x_i\mu$ for $i=0,\ldots,n$.
\end{defi}
\begin{rem*}
	 Before we check that $K^\bullet\big((x_0,\ldots,x_n),M\big)$ is indeed a cochain complex, let's answer the question \emph{how on earth} Definition~\reff{def:KoszulComplex} is related to the definition you may find in the literature (e.g., \cite[p.~423]{eisenbudCommAlg}) ? Well, they are equivalent (duh!) and here is how.
	 
	 Let $0\leq\ell_1<\ldots< \ell_m\leq n$ and $\delta_{\ell_1,\ldots,\ell_m}\colon [n]^m\morphism R$ be the map satisfying 
	 \begin{align*}
	 	\delta_{\ell_1,\ldots,\ell_m}(i_1,\ldots,i_m)=\begin{cases}
		 	\sgn(\pi) & \text{when }(i_1,\ldots,i_m)=(\ell_{\pi(1)},\ldots,\ell_{\pi(m)})\text{ for }\pi\in\SS_{m}\\
		 	0 & \text{when }(i_1,\ldots,i_m)\text{ is no permutation of }(\ell_1,\ldots,\ell_m)
	 	\end{cases}\;.
	 \end{align*}
	  Clearly, $K^m\big((x_0,\ldots,x_n),M\big)$ is generated by the $\delta_{\ell_1,\ldots,\ell_m}\cdot \mu$ for $\mu\in M$. For $K^m(x_0,\ldots,x_n)\coloneqq K^m\big((x_0,\ldots,x_n),R\big)$ the $\delta_{\ell_1,\ldots,\ell_m}$ even form a basis and we easily get an isomorphism
	  \begin{align*}
	  	K^\bullet\big((x_0,\ldots,x_n),M\big)\simeq K^\bullet(x_0,\ldots,x_n)\otimes_R M
	  \end{align*}
	  of cochain complexes. Moreover, if $e_0,\ldots,e_n$ denote the standard basis vectors of $R^{n+1}$, mapping $\delta_{\ell_1,\ldots,\ell_m}\mapsto e_{\ell_1}\wedge \cdots\wedge e_{\ell_m}$ sends a basis of $K^m(x_0,\ldots,x_n)$ to a basis of $\bigwedge^mR^{n+1}$ and we find that they are isomorphic.
	  
	  Let us also find out which map $d^m$ corresponds to. Note that $d_j\delta_{\ell_1,\ldots,\ell_m}(i_1,\ldots,i_{m+1})$ is only non-zero when $(i_1,\ldots,i_{m+1})$ is a permutation of $(k,\ell_1,\ldots,\ell_m)$ for some $k\in[n]$ with $k$ at the $(j+1)\ordinalst$ position. In this case, it takes the value $\pm x_k$, depending on the sign of the permutation of the $\ell_i$. If you think about this for a while, $d^m\delta_{\ell_1,\ldots,\ell_m}$ corresponds to $\sum_{k=0}^nx_ke_k\wedge e_{\ell_1}\wedge \ldots \wedge e_{\ell_m}$. So $d^m$ in $K^\bullet(x_0,\ldots,x_n)$ simply corresponds to the map
	  \begin{align*}
	  	(x_0,\ldots,x_n)\wedge -\colon \bigwedge^mR^{n+1}\morphism \bigwedge^{m+1}R^{n+1}\;.
	  \end{align*}
	  In $K^\bullet\big((x_0,\ldots,x_n),M\big)\simeq K^\bullet(x_0,\ldots,x_n)\otimes_R M$, this just gets tensored with $M$.
	  
	  Also note that $\bigwedge^0N=R$ for every $R$-module $N$ by definition, so the above special case $m=0$ now fits naturally into the rest.
\end{rem*}
\begin{rem}
	\begin{alphanumerate}
		\item \lbl{rem:FirstKoszulFacts}It is easy to check that the $d_j$ satisfy relations similar to \eqreff{eq:simplicialStuff}, \eqreff{eq:cosimplicialDegeneracies}, and \eqreff{eq:transpositions}. As in the proofs of Definition~\reff{def:cech} and Definition~\reff{def:alternatingCech}, we deduce that $d^\bullet$ preserves the antisymmetry conditions and is a differential in a cochain complex. 
		
		Alternatively, from the above considerations it's immediately clear that $d^\bullet$ is well defined and satisfies $d^{m+1}d^m=0$, since wedging with $(x_0,\ldots,x_n)$ two times in a row obviously gives $0$.
		\item A short exact sequence of $R$-modules induces a short exact sequence of Koszul complexes. Indeed, in the representation $K^m\big((x_0,\ldots,x_n),M\big)\simeq \bigwedge^mR^{n+1}\otimes_RM\simeq M^{\binom{n+1}{m}}$ this is pretty obvious.
		\item Let $\phi$ be the map $\phi\colon K^\bullet\big((x_0,\ldots,x_{n-1}),M\big)\morphism[x_n\cdot]K^\bullet\big((x_0,\ldots,x_{n-1}),M\big)$. We have a canonical isomorphism
		\begin{align}\lbl{eq:KoszulCone}
			\begin{split}
				K^\bullet\big((x_0,\ldots,x_n),M\big)&\isomorphism\Cone(\phi)[-1]\\
				f\in K^m\big((x_0,\ldots,x_n),M\big)&\longmapsto \left(f(n,-)|_{[n-1]^{m-1}}, -f|_{[n-1]^{m}}\right)
			\end{split}			
		\end{align}
		(I'm pretty sure there should be a minus in the second component for the differentials to be compatible, i.e., consistent with the \emph{two} sign conventions from Definition~\reff{def:cone}\itememph{a} and \itememph{b} which are in play here, though we didn't put a minus there in the lecture). This isomorphism corresponds to the decomposition $\bigwedge^mR^{n+1}=e_n\wedge \bigwedge^{m-1}R^n\oplus \bigwedge^mR^n\simeq \bigwedge^{m-1}R^n\oplus \bigwedge^mR^n$, sending $e_n\wedge y+z$ to $(y,-z)$ (and you have to tensor it with $M$, of course). If you think about this long enough, this is compatible.
	\end{alphanumerate}
\end{rem}
In the following we put $H^j\big((x_0,\ldots,n),M\big)\coloneqq H^j\big(K^\bullet((x_0,\ldots,x_n),M)\big)$ for ``short''.
\begin{example}
	\begin{alphanumerate}
		\item \lbl{ex:firstKoszulComputations}For small values of $n$, we have
		\begin{align*}
			K^\bullet\big(\emptyset,M\big)&=\Big(0\morphism M\morphism 0\morphism\ldots\Big)\\
			K^\bullet\big((x_0),M\big)&=\Big(0\morphism M\morphism[x_0\cdot] M\morphism 0\morphism\ldots\Big)\\
			K^\bullet\big((x_0,x_1),M\big)&=\Big(0\morphism M\xrightarrow{(x_0,x_1)} M\oplus M\xrightarrow{\left(\begin{smallmatrix}
			-x_1\\
			x_0
			\end{smallmatrix}\right)} M\morphism 0\morphism\ldots\Big)\;.
		\end{align*}
		\item In general, $K^\bullet((x_0,\ldots,x_n),M)$ vanishes in degrees $<0$ and $>n+1$ (as $\bigwedge^mR^{n+1}$ is only non-zero for $0\leq m\leq n+1$). The differentials  $d^0,d^n$ are given by $d^0\mu=\sum_{j=0}^nx_je_j\otimes \mu$ and $d^n\left(\sum_{j=0}^n(e_0\wedge\cdots\wedge\roof{e}_j\wedge\cdots\wedge e_n)\otimes \mu_i\right)=\sum_{j=0}^n(-1)^jx_j (e_0\wedge\cdots\wedge e_n)\otimes\mu_j$. This shows
		\begin{align*}
			H^0\big((x_0,\ldots,x_n),M\big)\simeq\bigcap_{i=0}^n\ker\left(M\morphism[x_i\cdot ]M\right)
		\end{align*}
		and
		\begin{align*}
			H^{n+1}\big((x_0,\ldots,x_n),M\big)\simeq M/(x_0M+\ldots+x_nM)\;.
		\end{align*}
	\end{alphanumerate}
\end{example}
\begin{fact}\lbl{fact:RegularSeqKoszulComplex}
	Let $R$ be a ring, $(x_0,\ldots,x_n)$ a sequence of elements of $R$ and $M$ an $R$-module.
	\begin{alphanumerate}
		\item Let $0\leq i\leq n$, then $(x_0,\ldots,x_n)$ is $M$-regular iff $(x_0,\ldots,x_{i-1})$ is $M$-regular and $(x_i,\ldots,x_n)$ is $M/(x_0M+\ldots+x_{i-1}M)$-regular.
		\item A sequence $(x_0,\ldots,x_n)$ is $M$-regular iff $H^j\big((x_0,\ldots,x_i),M\big)=0$ for all $i,j$ such that $0\leq i\leq n$ and $j\neq i+1$.
		\item Let $0\morphism M'\morphism M\morphism M''\morphism 0$ be a short exact sequence of $R$-modules. If $(x_0,\ldots,x_n)$ is $M'$-regular and $M''$-regular, then it is $M$-regular.
		\item If $(x_0,\ldots,x_n)$ is $M$-regular and $k_0,\ldots,k_n\in \IN$ are natural exponents, then $(x_0^{k_0},\ldots,x_n^{k_n})$ is also $M$-regular.
	\end{alphanumerate}
\end{fact}
\begin{proof}
	Part \itememph{a} is obvious. For \itememph{b}, we use induction on $n$. For $n=-1$, there is nothing to show. Now suppose that $n\geq 0$ and the assertion holds for $n-1$. From \eqreff{eq:MappingConeSeq} and \eqreff{eq:KoszulCone} we get a long exact sequence
	\begin{multline}\lbl{eq:KoszulSeq}
		\ldots\morphism H^{j-1}\big((x_0,\ldots,x_{n-1}),M\big)\morphism[x_n\cdot ] H^{j-1}\big((x_0,\ldots,x_{n-1}),M\big)\morphism H^j\big((x_0,\ldots,x_n),M\big)\\
		\morphism H^j\big((x_0,\ldots,x_{n-1}),M\big)\morphism[x_n\cdot] H^j\big((x_0,\ldots,x_{n-1}),M\big)\morphism\ldots\;.\tag{$*$}
	\end{multline}
	Suppose that $H^j\big((x_0,\ldots,x_i),M\big)$ vanishes for $0\leq i\leq n$ and $j\neq i+1$. Then $(x_0,\ldots,x_{n-1})$ is $M$-regular by induction. Using $H^{n-1}\big((x_0,\ldots,x_{n-1}),M\big)=0$ and Example~\reff{ex:firstKoszulComputations}\itememph{b}, for $j=n$ the sequence \eqreff{eq:KoszulSeq} becomes
	\begin{align}\lbl{eq:KoszulSeq2}
		0\morphism H^n\big((x_0,\ldots,x_n),M\big)\morphism M/(x_0M+\ldots+x_{n-1}M)\morphism[ x_n\cdot]M/(x_0M+\ldots+x_{n-1}M)\tag{\#}
	\end{align}
	The vanishing of $H^n\big((x_0,\ldots,x_n),M\big)$ in \eqreff{eq:KoszulSeq2} and exactness show that $x_n\cdot$ is injective on $M/(x_0M+\ldots+x_{n-1}M)$, hence $(x_0,\ldots,x_n)$ is $M$-regular.
	
	Conversely, assume that $(x_0,\ldots,x_n)$ is $M$-regular. Vanishing of $H^j\big((x_0,\ldots,x_i),M\big)$ for $0\leq i\leq n-1$ and $j\neq i+1$ follows from induction (in particular, we still get \eqreff{eq:KoszulSeq2}) and for $i=n$ and $j>n+1$ this is also clear. From \eqreff{eq:KoszulSeq} we get $H^j\big((x_0,\ldots,x_n),M\big)=0$ for $j<n$ and for $j=n$ this follows from \eqreff{eq:KoszulSeq2} and injectivity of $x_n\cdot$ on $M/(x_0M+\ldots+x_{n-1}M)$.
	
	
	Part \itememph{c} follows from \itememph{b} and the long exact cohomology sequence associated to the short exact sequence $0\morphism K^\bullet\big((x_0,\ldots,x_n),M'\big)\morphism K^\bullet\big((x_0,\ldots,x_n),M\big)\morphism K^\bullet\big((x_0,\ldots,x_n),M''\big)\morphism 0$ of Koszul complexes due to Remark~\reff{rem:FirstKoszulFacts}\itememph{b}.
	
	Part \itememph{d}. We use induction on $n$. For $n=-1$, there is nothing to show. Let $n\geq 0$ and assume that \itememph{d} holds for $n-1$ and any $R$-module $M$. Then $M/x_0^{k_0}M$ has a filtration by submodules $x_0^iM/x_0^{k_0}M$ for $i=0,\ldots,k_0$ and with filtration quotients $x_0^iM/x_0^{i+1}M$. Since $x_0\cdot$ is injective on $M$, $x_0^i\cdot$ induces an isomorphism $x_0^iM/x_0^{i+1}M\lisomorphism M/x_0M$. Since $(x_1,\ldots,x_n)$ is $M/x_0M$-regular by \itememph{a}, repeatedly applying \itememph{c} implies that $(x_1,\ldots,x_n)$ is $M/x_0^{k_0}M$-regular. By the induction assumption, $(x_1^{k_1},\ldots,x_n^{k_n})$ is $(M/x_0^{k_0})$-regular as well. An application of \itememph{a} completes the proof.
\end{proof}
\begin{defi}
	Let $R$ be a ring, $M$ and $N$ be finitely generated free (or projective) $R$-modules. A \defemph{non-degenerate pairing} $M\times N\morphism R$ is a $R$-bilinear map $M\times N\morphism R$ such that one of the induced maps $M\morphism\Hom_R(N,R)$ or $N\morphism \Hom_R(M,R)$ is an isomorphism.
\end{defi}
\begin{cor}\lbl{cor:KoszulCoho}
	Let $R$ be any ring, $S=R[X_0,\ldots,X_n]$. The Koszul complex $K^\bullet\big((X_0^\ell,\ldots,X_n^\ell),S\big)$ is acyclic in cohomological degree $\neq n+1$. Its cohomology in dimension $n+1$ is a free $R$-module.
	
	The Koszul complex becomes a complex of graded $S$-modules via
	\begin{align*}
		K^m\big((X_0^\ell,\ldots,X_n^\ell),S\big)_k\coloneqq \left\{f\in K^m\big((X_0^\ell,\ldots,X_n^\ell),S\big)\st f([n]^m)\subseteq S_{k+\ell m}\right\}\;,
	\end{align*}
	where $S_{k+\ell m}$ denotes the degree-$(k+\ell m)$ part of $S=R[X_0,\ldots,X_n]$, and
	\begin{alphanumerate}
		\item $H^{n+1}\big((X_0^\ell,\ldots,X_n^\ell),S\big)_k=0$ when $k>-n-1$.
		\item We have an isomorphism
		\begin{align}\lbl{eq:Hn+1Iso}
			\begin{split}
				\tau\colon H^{n+1}\big((X_0^\ell,\ldots,X_n^\ell),S\big)_{-n-1}&\isomorphism R\\
				[f]&\longmapsto \text{coefficient of }(X_0X_1\cdots X_n)^{\ell-1}\text{ in }f(0,1,\ldots,n)
			\end{split}
		\end{align}
		(where $[f]$ denotes the cohomology class of $f\in K^{n+1}\big((X_0^\ell,\ldots,X_n^\ell),S\big)$ which is a cocycle).
		\item For $0<k<\ell$ we have a non-degenerate pairing
		\begin{align}\lbl{eq:KoszulNonDegPairing}
			S_k\times H^{n+1}\big((X_0^\ell,\ldots,X_n^\ell),S\big)_{-n-1-k}\morphism H^{n+1}\big((X_0^\ell,\ldots,X_n^\ell),S\big)_{-n-1}\overset{\text{\eqreff{eq:Hn+1Iso}}}{\simeq} R.
		\end{align}
		(simply given by multiplication).
	\end{alphanumerate}
\end{cor}
\begin{proof}
	By Example~\reff{ex:(X0Xn)regular}\itememph{a}, $(X_0,\ldots,X_n)$ is an $S$-regular sequence, hence so is $(X_0^\ell,\ldots,X_n^\ell)$ by Fact~\reff{fact:RegularSeqKoszulComplex}\itememph{d}. Hence the vanishing of $H^i\big((X_0^\ell,\ldots,X_n^\ell),S\big)$ is a consequence of Fact~\reff{fact:RegularSeqKoszulComplex}\itememph{b}. Moreover, by Example~\reff{ex:firstKoszulComputations}\itememph{b} we have 
	\begin{align*}
		H^{n+1}\big((X_0^\ell,\ldots,X_n^\ell),S\big)\simeq S/(X_0^\ell,\ldots,X_n^\ell) 
	\end{align*}
	which is a free $R$-module with basis $[X_0^{\alpha_0}\cdots X_n^{\alpha_n}]=X_0^{\alpha_0}\cdots X_n^{\alpha_n}+(X_0^\ell,\ldots,X_n^\ell)$ for $\alpha_0,\ldots,\alpha_n\in \IN_0$ such that $\alpha_i<\ell$. The elements of this basis are homogeneous for the above grading and to get an element of homogeneous degree $-n-1$ we need $\alpha_0+\ldots+\alpha_n=-n-1+\ell(n+1)=(\ell-1)(n+1)$ and there is no other way to have this than $\alpha_i=\ell-1$ for $i=0,\ldots,n$. This proves \itememph{b}.
	
	Along the same lines we see that \itememph{a} holds. Indeed, for an element of degree $k>-n-1$ we would need $\alpha_0+\ldots+\alpha_n>(\ell-1)(n+1)$ and then there is some $\alpha_i\geq \ell$, contradiction.
	
	The pairing from \itememph{c} pairs $X_0^{\beta_0}\cdot X_n^{\beta_n}$ with $[X_0^{\alpha_0}\cdots X_n^{\alpha_n}]\in H^{n+1}\big((X_0^\ell,\ldots,X_n^\ell),S\big)_{-n-1-k}$ to a non-zero element (which is then automatically $(X_0\cdots X_n)^{\ell-1}$) iff $\alpha+\beta=(\ell-1,\ldots,\ell-1)$. For every $\alpha$ contributing to our basis of $H^{n+1}\big((X_0^\ell,\ldots,X_n^\ell),S\big)_{-n-1-k}$ the corresponding $\beta=(\ell-1,\ldots,\ell-1)-\alpha$ is an element of $\IN_0^{n+1}$ and $\beta_0+\ldots+\beta_n=k$. If $k<\ell$ and $\beta\in\IN_0^{n+1}$ satisfies $\beta_0+\ldots+\beta_n=k$, then $\beta_i\leq k\leq\ell-1$ for all components of $\beta$, hence $\alpha=(\ell-1,\ldots,\ell-1)-\beta\in\IN_0^{n+1}$ contributes to our basis of $H^{n+1}\big((X_0^\ell,\ldots,X_n^\ell),S\big)_{-n-1-k}$, showing the non-degeneracy of the pairing.
\end{proof}
And finally for a lemma which is perhaps worthwhile to know.
\begin{lem}
	Let $R$ be a Noetherian local ring with maximal ideal $\mm$, $M$ a finitely generated $R$-module and $x_0,\ldots,x_n\in\mm$. Then $(x_0,\ldots,x_n)$ is $M$-regular iff $H^i\big((x_0,\ldots,x_n),M\big)=0$ for $i\neq n+1$. In particular, reordering $(x_0,\ldots,x_n)$ will again produce an $M$-regular sequence.
\end{lem}
\begin{tikzpicture}[inner sep=0pt]
	\node (wallOfText) at (0,0) {
		\begin{minipage}{\textwidth}
			\setlength{\parskip}{\currentparskip}
			\begin{proof}
			The \emph{only if} part being trivial after Fact~\reff{fact:RegularSeqKoszulComplex}\itememph{b}, we assume now that $H^i\big((x_0,\ldots,x_n),M\big)$ vanishes for $i\neq n+1$. From \eqreff{eq:KoszulCone} and the cone sequence \eqreff{eq:MappingConeSeq} we obtain a part of a long exact sequence
			\begin{multline*}
			H^j\big((x_0,\ldots,x_n),M\big)\morphism H^j\big((x_0,\ldots,x_{n-1}),M\big)\morphism[x_n\cdot] H^j\big((x_0,\ldots,x_{n-1}),M\big)\\
			\morphism H^{j+1}\big((x_0,\ldots,x_n),M\big)\;.
			\end{multline*}
			For $j<n$ the outer terms vanish by assumption, hence
			\begin{align*}
			H^j\big((x_0,\ldots,x_{n-1}),M\big)\morphism[x_n\cdot] H^j\big((x_0,\ldots,x_{n-1}),M\big)
			\end{align*}
			is an isomorphism. But we are working over a Noetherian ring $R$ and $M$ is finitely generated, so all involved constructions are finitely generated as well. In particular, $H^j\big((x_0,\ldots,x_{n-1}),M\big)$ is finitely generated. As $x_n\in\mm$, Nakayama's lemma shows $H^j\big((x_0,\ldots,x_{n-1}),M\big)=0$ for $j<n$. For $j>n$ this holds as well, obviously. Inductively, this shows $H^j\big((x_0,\ldots,x_i),M\big)=0$ for all $i<n$ and $j\neq i+1$, hence $(x_0,\ldots,x_n)$ is regular by Fact~\reff{fact:RegularSeqKoszulComplex}\itememph{b}.			
			\end{proof}
		\end{minipage}};
	\coordinate (dummy) at ($(wallOfText.north east)!1.0/2!(wallOfText.south east)$);
	\fill[white,path fading=premium access] (wallOfText.north west) rectangle (dummy);
	\fill[white,opacity=0.9375] (dummy) rectangle (wallOfText.south west);
	\node[rectangle, fill=white,draw=black!50!white, inner sep=3ex, rounded corners] at (0,0) {\begin{minipage}{0.5\textwidth}
		\sffamily
		\begin{center}
		\LARGE\textbf{Want to read more?}
		\end{center}
		\vspace{2ex}
		Subscribe to continue reading (or wait till exercise sheet \#7 has been collected).
		\begin{itemize}
		\item[{\tikz[inner sep=0ex,anchor=base, baseline]{\node (a) {\phantom{h}};\draw[red!70!black] (a) circle (5pt);}}] Purchase this proof for 0,99\euro.
		\item[{\tikz[inner sep=0ex,anchor=base, baseline]{\node (a) {\phantom{h}};\draw[red!70!black] (a) circle (5pt);\fill[red!70!black] (a) circle (3pt);}}] Unlock AlgGeo II PREMIUM. Unlimited access on all devices. Free 30 days trial, then 4,99\euro\ per month.
		\end{itemize}
		\begin{center}
		\tikz\fill[inner sep=0ex,red!70!black, rounded corners]  (0,0.75) rectangle (\textwidth,0) node[pos=0.5,white] {\large\textbf{Get access now}};
		\end{center}
		\vspace{2ex}
		Already have an account? \textbf{\underline{Si\smash{g}n in}} here.
		\end{minipage}};
\end{tikzpicture}

\section{The scheme \texorpdfstring{$\IP_R^n$}{P} and its cohomology}
Let $R$ be an $\IN_0$-graded ring. Let $R_+\subseteq R$ be the ideal of elements of $R$ of positive homogeneous degree. Recall that there is a bijection
\begin{align*}
	\left\{\text{homogeneous }\pp\in\Spec R\st \pp\supseteq R_+\right\}&\isomorphism \Spec R_0\\
	\pp&\longmapsto \qq=\pp\cap R_0\\
	\pp=\qq+R_+&\longmapsfrom \qq\;.
\end{align*}
We recapitulate the construction and basic properties of $\Proj R$, which has been introduced in \cite[Definition~2.6.2]{alggeo1} and exercise sheet \#11 of Algebraic Geometry I. As a set, put
\begin{align*}
	\Proj R=\left\{\text{homogeneous }\pp\in\Spec R\st \pp\not\supseteq R_+\right\}
\end{align*}
and for $I\subseteq R$ a homogeneous ideal, let
\begin{align*}
	V_\proj(I)=V(I)=\left\{\pp\in\Proj R\st \pp\supseteq I\right\}\;.
\end{align*}
As $\bigcap_{\lambda\in\Lambda}V(I_\lambda)=V\left(\sum_{\lambda\in\Lambda}I_\lambda\right)$ and $V(I)\cup V(J)=V(I\cap J)=V(IJ)$ and $V(R)=V(R_+)=\emptyset$ and $V(0)=\Proj R$ the $V(I)$ are the closed subsets of some topology on $\Proj R$.

We define a sheaf of graded rings $\Oo=\bigoplus_{k\in\IZ}\Oo(k)$ on $P=\Proj R$ as follows. Let $\Oo(k)_{[\pp]}=(R_\pp)_k$ for homogeneous prime ideals $\pp\subseteq R$, where we use the convention that $R_\pp$ is obtained from $R$ by inverting the \defemph{homogeneous elements of $R\setminus \pp$ only (!)}. Note that the subsets of the form $P\setminus V(f)$ with $f\in R$ homogenous form a topology base of $P$. We put
\begin{align*}
	\Oo(k)(U)=\left\{(g_\pp)\in\prod_{\pp\in U}(R_\pp)_k\st\begin{array}{c}
		\text{every }x\in U\text{ has a neighbourhood }V=P\setminus V(f)\subseteq U\text{,}\\
		\text{ where }f\text{ is homogeneous,  such that there is a }\gamma\in (R_f)_k\\
		\text{with }g_\pp=\left(\text{image of }\gamma\text{ under }R_f\morphism R_\pp\right)\text{ for all }\pp\in V
	\end{array}\right\}\;.
\end{align*}
Then there are operations
\begin{align*}
	\Oo(k)\times \Oo(k)&\morphism \Oo(k)\\
	\left((f_\pp)_{\pp\in U},(g_\pp)_{\pp\in U}\right)&\longmapsto (f_\pp)_{\pp\in U}+(g_\pp)_{\pp\in U}=(f_\pp+g_\pp)_{\pp\in U}
\end{align*}
and
\begin{align*}
	\Oo(k)\times \Oo(\ell)&\morphism \Oo(k+\ell)\\
	\left((f_\pp)_{\pp\in U},(g_\pp)_{\pp\in U}\right)&\longmapsto (f_\pp)_{\pp\in U}\cdot (g_\pp)_{\pp\in U}=(f_\pp g_\pp)_{\pp\in U}	
\end{align*}
making $\Oo$ into a sheaf of graded rings with $\Oo(0)=\Oo_{\Proj R}$. Moreover, we get a map
\begin{align}\lbl{eq:ProjPretenderStalk}
	\begin{split}
		\Oo(k)_\pp&\morphism\Oo(k)_{[\pp]}\\
		\left(\begin{array}{c}
			\text{image of}\\
			(g_\qq)\in\Oo(k)(U)\text{ in }\Oo(k)_\pp
		\end{array}\right)&\longmapsto g_\pp
	\end{split}\tag{$*$}
\end{align}
\begin{fact}
	The map \eqreff{eq:ProjPretenderStalk} is an isomorphism.
\end{fact}
\begin{proof}
	By Proposition~\reff{prop:FirstProjProperties}\itememph{a} and \itememph{c} below, 
	\begin{align*}
		\Oo(k)_\pp=\colimit[f\notin \pp]\Oo(k)\big(\Proj R\setminus V(f)\big)\simeq \colimit[f\notin\pp](R_f)_k=(R_\pp)_k=\Oo(k)_{[\pp]}\;,
	\end{align*}
	the colimits ranging over all $f$ which are homogeneous of positive degree.
\end{proof}
\begin{prop}
	\begin{alphanumerate}
		\item \lbl{prop:FirstProjProperties}Open subsets of the form $\Proj R\setminus V(f)$ where $f\in R_k$ with $k>0$ are affine and form a topology base.
		\item Every finite subset of $\Proj R$ is contained in an affine open subset. In particular, $\Proj R$ is a scheme.
		\item The sheaf of modules $\Oo(k)$ on $\Proj R$ is quasi-coherent, and for $f\in R$ which is homogeneous of positive degree there is a homeomorphism
		\begin{align}\lbl{eq:ProjLocal}
			\begin{split}
				\Proj R\setminus V(f)&\isomorphism \Spec (R_f)_0\\
				\pp &\longmapsto \qq=\pp R_f\cap (R_f)_0\\
				\pp=\sqrt{\qq R_f}\cap R&\longmapsfrom \qq\in\Spec (R_f)_0
			\end{split}
		\end{align}
		under which $(R_\pp)_0\simeq ((R_f)_0)_\qq$. We thus get an identification 
		\begin{align*}
			\Oo(0)|_{\Proj R\setminus V(f)}\simeq \Oo_{\Spec (R_f)_0}\;.
		\end{align*}		
		Moreover, under \eqreff{eq:ProjLocal} there is a similar isomorphism $(R_\pp)_k\simeq ((R_f)_k)_\qq$ (the latter denotes the localization of the $(R_f)_0$-module $(R_f)_k$ with respect to $\qq$, and here we don't need to restrict to homogeneous elements). This gives an identification
		\begin{align}\lbl{eq:O(k)isQC}
			\Oo(k)|_{\Proj R\setminus V(f)}\simeq ((R_f)_k)^\sim\;.
		\end{align}
		\item When $R$ is generated by $R_1$ as an $R_0$-algebra (or equivalently, $R_+$ by $R_1$ as an ideal in $R$) the sheaves of modules $\Oo(k)$ are line bundles and the morphism 
		\begin{align*}
			\Oo(k)\otimes_{\Oo_{\Proj R}}\Oo(\ell)&\morphism\Oo(k+\ell)\\
			f\otimes g&\longmapsto fg
		\end{align*}
		is an isomorphism (the tensor product of sheaves is introduced in Remark~\reff{rem:tensorProductSheaves}).
		\item We have a morphism of preschemes $\pi\colon\Proj R\morphism \Spec R_0$ whose topological component is given by
		\begin{align}\lbl{eq:Proj(R)toSpecR0}
			\begin{split}
				\pi\colon\Proj R&\morphism \Spec R_0\\
				\pp &\longmapsto \pp\cap R_0
			\end{split}
		\end{align}
		and the algebraic component $\pi^*\colon \Oo_{\Spec R_0}\morphism \pi_*\Oo_{\Proj R}$ is defined by sending an element $rf^{-k}\in (R_0)_f\simeq \Oo_{\Spec R_0}(\Spec R_0\setminus V(f))$ to precisely that element in $\Oo(0)(\Proj R\setminus V(f))$. The morphism $\pi$ is of finite type (cf.\ \cite[Definition~2.2.1]{alggeo1}) if $R$ is an $R_0$-algebra of finite type.
		\item If $R$ is Noetherian (which is the case iff $R_0$ is Noetherian and $R$ of finite type over $R_0$) then $\Proj R$ is Noetherian (cf.\ \cite[Definition~2.2.2]{alggeo1}) and the $\Oo(k)$ are coherent.
	\end{alphanumerate}
\end{prop}
\begin{proof}
	Most of this follows from exercise sheet~\#11 of Algebraic Geometry I. We show the topology base part of \itememph{a} first: Let $I$ be a homogeneous ideal and $\pp\notin V(I)$ a homogeneous prime ideal. Then $I\setminus \pp$ contains a homogeneous element $f$. Clearly $\Proj R\setminus V(f)\subseteq \Proj R\setminus V(I)$. If $f$ has positive degree, we are done. Otherwise choose an element $g\in R_+\setminus \pp$ and replace $f$ by $fg$. The fact that the $\Proj R\setminus V(f)$ are affine follows from \eqreff{eq:ProjLocal}.
	
	For \itememph{c}, do exercise~1 and apply this to \eqreff{eq:ProjLocal}. From this, it is clear that for $M$ a graded $R$-module the sheafification of the presheaf 
	\begin{align*}
		\Proj R\setminus V(f)\longmapsto (M_f)_0\quad\text{for }f\text{ homogeneous of positive degree}
	\end{align*}
	is quasi-coherent as it equals $((M_f)_0)^\sim$ on $\Proj R\setminus V(f)$ (this is actually the first part of exercise~3). Applying this to $M=R[k]$ (the shift of $R$ as in Definition~\reff{def:cone}\itememph{a}) shows that $\Oo(k)$ is quasi-coherent and \eqreff{eq:O(k)isQC} holds.
	
	Part \itememph{b}. Let $\pp_1,\ldots,\pp_n\in\Proj R$. Since $R_+$ isn't contained in any $\pp_i$, it is also not contained in the union $\bigcup_{i=1}^n\pp_i$ by the prime avoidance lemma (cf.\ \cite[Lemma~2.5.1]{alg1}). Then for any $f\in R_+\setminus \bigcup_{i=1}^n\pp_i$, $\Proj R\setminus V(f)\simeq \Spec (R_f)_0$ is a common affine open neighbourhood of $\pp_1.\ldots,\pp_n$ by \eqreff{eq:ProjLocal}. From \cite[Fact~1.5.9\itememph{a}]{alggeo1} we get that $\Proj R$ is a scheme.
	
	Part \itememph{d}. If $R_+$ is generated by $R_1$ as an ideal in $R$, a homogeneous prime ideal $\pp$ contains $R_+$ iff it contains $R_1$. Hence the $\Proj R\setminus V(f)$ for $f\in R_1$ cover $\Proj R$. On $U=\Proj R\setminus V(f)$ however, an isomorphism $\Oo_{\Proj R}|_U=\Oo(0)|_U\isomorphism \Oo(k)|_U$ is given by multiplication with $f^k$. To show that $\Oo(k)\otimes_{\Oo_{\Proj R}}\Oo(\ell)\isomorphism\Oo(k+\ell)$, it is enough to check this stalk-wise. But $\Oo(k)_\pp\simeq \Oo_{\Proj R,\pp}\simeq (R_\pp)_0$ holds for $\pp\in\Proj R$ as we just have seen, and similar for $\Oo(\ell)_\pp$ and $\Oo(k+\ell)_\pp$, so the isomorphism
	\begin{align*}
		\Oo(k)_\pp\otimes_{\Oo_{\Proj R,\pp}}\Oo(\ell)_\pp\isomorphism \Oo(k+\ell)_\pp
	\end{align*}
	is pretty much obvious.
	
	Part \itememph{e}. Let $\alpha_1,\ldots,\alpha_n$ be generators of $R$ as an $R_0$-algebra. Decomposing them into homogeneous components, we may assume that they are homogeneous of degrees $d_1,\ldots,d_n>0$. By \cite[Proposition~2.2.1\itememph{b}]{alggeo1}  and \itememph{c} it is enough to show that $(R_f)_0\simeq\Oo_{\Proj R}(\Proj R\setminus V(f))$ is of finite type over $R_0$ for all $f\in R_+$. Let $d$ be the homogeneous degree of $f$. One may check that $(R_f)_0$ is indeed generated as an $R_0$-algebra by those
	\begin{align*}
	\alpha_1^{k_1}\cdots \alpha_n^{k_n}\cdot f^{-k}\quad\text{where }0\leq k_i<d\text{ and }0\leq k<d\cdot(d_1+\ldots+d_n)
	\end{align*}
	which have homogeneous degree $0$. 
	
	To show that $\pi$ is quasi-compact, it suffices to show quasi-compactness of $\Proj R$ by \cite[Fact~2.1.2]{alggeo1}. Since $R$ is of finite type over $R_0$, the augmentation ideal $R_+$ is finitely generated. Let $f_1,\ldots,f_n$ be homogeneous generators of $R_+$. Then the $\Proj R\setminus V(f_i)\simeq \Spec (R_{f_i})_0$ cover $\Proj R$, so $\Proj R$ is a finite union of quasi-compact sets and thus quasi-compact itself.
	
	And finally part \itememph{f}. The stuff in parentheses was proved in \cite[Proposition~2.2.1]{alg2}, by the way. In \itememph{e} we have seen that $\Proj R$ is finite type over $\Spec R_0$ which is Noetherian, hence $\Proj R$ is Noetherian by Hilbert's Basissatz. By an argument similar to \itememph{e}, $(R_f)_k$ is finitely generated as an $(R_f)_0$-module for all $f\in R_+$, hence $\Oo(k)$ is locally finitely generated over $\Oo_{\Proj R}$ by \itememph{c} and thus coherent in the sense of Definition~\reff{def:locFinGenerated}\itememph{b}.
\end{proof}
\begin{rem}
	For a graded $R$-module $M$ and open $U\subseteq P=\Proj R$ let 
	\begin{align*}
		\snake{M}(U)=\left\{(\mu_\pp)\in\prod_{\pp\in U}(M_\pp)_0\st\begin{array}{c}
		\text{every }x\in U\text{ has a neighbourhood }V=P\setminus V(f)\subseteq U\text{,}\\
		\text{ where }f\text{ is homogeneous,  such that there is a }m\in (M_f)_0\\
		\text{with }\mu_\pp=\left(\text{image of }m\text{ under }M_f\morphism M_\pp\right)\text{ for all }\pp\in V
		\end{array}\right\}\;.
	\end{align*}
	 This is actually the sheafification of the presheaf 
	 \begin{align*}
		 \Proj R\setminus V(f)\longmapsto (M_f)_0\quad\text{for }f\text{ homogeneous of positive degree}
	 \end{align*}
	 we have already seen in the proof if Proposition~\reff{prop:FirstProjProperties}\itememph{c}. Note that the localizations $M_\pp$ appearing here are understood to be obtained by inverting the \emph{homogeneous} elements of $R\setminus \pp$ only. On open subsets of the form $\Proj R\setminus V(f)\simeq \Spec (R_f)_0$ for $f\in R_+$ homogeneous, $\snake{M}$ equals $((M_f)_0)^\sim$, proving that it is quasi-coherent. In the case $\snake{M}=R[k]$ we obtain $\snake{M}\simeq \Oo(k)$.
\end{rem}
\begin{rem*}
	If $R\morphism[\phi]S$ is a ring morphism such that there is a positive integer $d$ satisfying $\phi(R_k)\subseteq S_{dk}$ for all $k$ and such that $S_+=\sqrt{S\phi(R_+)}$, then we have a continuous map
	\begin{align*}
		\Proj S&\morphism\Proj R\\
		\pp &\longmapsto\phi^{-1}(\pp)
	\end{align*}
	together with morphisms $(R_{\phi^{-1}(\pp)})_0\morphism[\phi](S_\pp)_0$, defining a morphism $\Proj S\morphism \Proj R$ of preschemes. 
	
	For instance, there is a morphism (a closed immersion even) from $\IP_A^m\coloneqq\Proj A[X_0,\ldots,X_m]$ to $\IP_A^n$ for $m\leq n$ defined by the ring morphism $A[X_0,\ldots,X_n]\morphism A[X_0,\ldots,X_m]$ sending $X_{m+1},\ldots,X_n$ to $0$. By contrast, there is no morphism $\IP_A^m\morphism\IP_A^n$ defined by $A[X_0,\ldots,X_n]\subseteq A[X_0,\ldots,X_m]$ when $n<m$ (because the radical property fails).
	
	As an example with $d>1$ we may take $R=A[X_0,X_1,X_2]$, $S=A[Y_0,Y_1]$ and $\phi(X_0)=Y_0^2$, $\phi(X_1)=Y_0Y_1$, $\phi(X_2)=Y_1^2$. The condition $S_+=\sqrt{S\phi(R_+)}$ holds as all monomials of even degree in $Y_0,Y_1$ can be decomposed as a product of the three monomials $Y_0^2,Y_0Y_1,Y_1^2$. We obtain a closed immersion $\IP_A^1=\Proj S\morphism \IP_A^2=\Proj R$ identifying $\IP_A^1$ with the closed subprescheme $V(X_0X_2-X_1^2)\subseteq \IP_A^2$.
\end{rem*}
\begin{rem}\lbl{rem:tensorProductSheaves}
	We will construct a sheaf tensor product. Let $\Rr$ be a sheaf of rings on some space $X$ and $\Mm,\Nn$ be $\Rr$-modules. We define $\Mm\otimes_\Rr\Nn$ to be the sheafification of the presheaf 
	\begin{align*}
		\Pp\colon U\longmapsto \Mm(U)\otimes_{\Rr(U)}\Nn(U)\quad\text{for open subsets }U\subseteq X\;.
	\end{align*}
	It is ``easy to see'' that the stalk $\Pp_x$ is given by $\Mm_x\otimes_{\Rr_x}\Nn_x$\footnote{If you really want to do this, you could use that both the free module and its submodule of relations-to-be-modded-out appearing in the explicit construction of the tensor product commute with colimits, as does taking the quotient. In fact, since $\Mm_x=\colimit[U\ni x] \Mm(U)$ is a \emph{filtered} colimit, it is the same as being taken in the category of sets, and the free module functor is left-adjoint to the forgetful functor, hence preserves colimits (Remark~\reff{rem:AdjunctionPreservesStuff}). So does taking quotients, since colimits commute with colimits (Lemma~\reff{lem:ColimitsCommute}).}. Then the stalk at $x$ of $\Mm\otimes_\Rr\Nn$ is given by $\Mm_x\otimes_{\Rr_x}\Nn_x$ as well and we get
	\begin{align*}
		(\Mm\otimes_\Rr\Nn)(U)=\left\{(f_x)\in\prod_{x\in U}\Mm_x\otimes_{\Rr_x}\Nn_x\st\begin{array}{c}
			\text{every }x\in U\text{ has a neighbourhood }V\subseteq U\\
			\text{and }\phi\in \Mm(V)\otimes_{\Rr(V)}\Nn(V)\text{ such that }f_y\text{ is}\\
			\text{the image of }\phi\text{ in }\Mm_x\otimes_{\Rr_x}\Nn_x\text{ for all }y\in V
		\end{array}\right\}
	\end{align*}
	This satisfies the universal property 
	\begin{align*}
		\Hom_{\Rr\cat{-Mod}}(\Mm\otimes_\Rr\Nn,\Tt)&\isomorphism \operatorname{Bil}_\Rr(\Mm\times \Nn,\Tt)\\
		\phi&\longmapsto\big(\beta\colon (m,n)\longmapsto \phi(m\otimes n)\big)
	\end{align*}
	for every $\Rr$-module $\Tt$ (where the set $\operatorname{Bil}_\Rr(\Mm\times \Nn,\Tt)$ of bilinear morphisms of sheaves $\Mm\times \Nn\morphism \Tt$ is defined in the obvious way).
	
	Now let $X=\Spec R$ and $\Mm=\snake{M}$, $\Nn=\snake{N}$ for some $R$-modules $M,N$. Then $(M\otimes_RN)_\pp=M\otimes_RN\otimes_RR_\pp=M_\pp\otimes_{R_\pp}N_\pp$ shows that $\Mm\otimes _{\Oo_X}\Nn=(M\otimes_RN)^\sim$. Hence the tensor product over arbitrary preschemes preserves quasi-coherentness of modules.
	
	When $\Ll$ is a line bundle on a ringed space $(X,\Oo_X)$ and $\Mm$ an $\Oo_X$-module, then $\Mm\otimes_{\Oo_X}\Ll$ is locally (non-canonically) isomorphic to $\Mm$, a local isomorphism being given by 
	\begin{align*}
		\Mm|_U&\isomorphism (\Mm\otimes_{\Oo_X}\Ll)|_U\\
		m&\longmapsto m\otimes \lambda\;,
	\end{align*}
	where $\lambda\in\Ll(U)$ is a free generator of $\Ll|_U$. In that sense, $\Mm\otimes_{\Oo_X}\Ll$ may be viewed as a \emph{twist} of $\Mm$. Similarly, when $\Vv$ is an $n$-dimensional vector bundle on $X$, then $\Mm\otimes_{\Oo_X}\Vv$ is locally (non-canonically) isomorphic to  $\Mm^n$.
	
	Now let $R$ be a graded ring such that $R_+$ is generated by $R_1$ over $R$. By Proposition~\reff{prop:FirstProjProperties}\itememph{d}. Then the $\Oo(k)$ are line bundles and for a graded $R$-module $M$ we get an isomorphism
	\begin{align*}
		\snake{M}\otimes_{\Oo_{\Proj R}}\Oo(k)&\isomorphism (M[k])^\sim\\
		m\otimes r&\longmapsto rm
	\end{align*}
	(which is easily checked stalk-wise). In particular, this generalizes Proposition~\reff{prop:FirstProjProperties}\itememph{d} (which is the special case $M=R[\ell]$).
\end{rem}
Now let $A$ be a ring, $R=A[X_0,\ldots,X_n]$ be equipped with the usual grading and put $\IP_A^n=\Proj R$. Then $\Uu\colon \IP_A^n=\bigcup_{i=0}^nU_i$ with $U_i=\IP_R^n\setminus V(X_i)$ is an affine open cover of $\IP_A^n$ and by \eqreff{eq:ProjLocal} we have
\begin{align*}
	U_{i_0,\ldots,i_\ell}=\IP_A^n\setminus V\left(X_{i_0}\cdots X_{i_\ell}\right)\simeq \Spec R[(X_{i_0}\cdots X_{i_\ell})^{-1}]_0\;.
\end{align*}
\begin{thm}\lbl{thm:CohoOfIP^n}
	Let $A$ be a ring, $R=A[X_0,\ldots,X_n]$ and $\Uu$ be the affine open cover of $X=\IP_A^n$ from above. 
	\begin{alphanumerate}
		\item  For $k\geq 0$ there is an isomorphism
		\begin{align}\lbl{eq:Oo(k)(IP^n)}
			\begin{split}
				R_k&\isomorphism \Oo(k)(X)\\
				r&\longmapsto \big(\text{image of }s\text{ in }(R_\pp)_k\big)_{\pp\in X}\;.
			\end{split}
		\end{align}
		For $k<0$ we have $\Oo(k)(X)=0$.
		\item For $0<p<n$ we have $H^p(X,\Oo(k))=0$ for all integers $k$.
		\item $H^n(X,\Oo(k))=0$ for $k>-n-1$ and $n>0$.
		\item There is an isomorphism $H^n(X,\Oo(-n-1))\isomorphism A$ defined (up to sign) by
		\begin{align}\lbl{eq:H^nIP^nO(-n-1)}
			\begin{split}
				\check{C}_\alt^n\big(\Uu,\Oo(-n-1)\big)\simeq R[(X_0\cdots X_n)^{-1}]_{-n-1}&\morphism A\\
				f=\sum_{\alpha\in\IZ^n}f_\alpha X_0^{\alpha_0}\cdots X_n^{\alpha_n}&\longmapsto f_{(-1,\ldots,-1)}\;.
			\end{split}
		\end{align}
		\item For all $k\in\IZ$, the groups $H^0(X,\Oo(k))$ and $H^n(X,\Oo(k))$ are finitely generated free $A$-modules and there is a non-degenerate pairing
		\begin{align}\lbl{eq:nonDegCohoPairing}
			\begin{split}
				H^0\big(X,\Oo(k)\big)\times H^n\big(X,\Oo(-n-1-k)\big)&\morphism H^n(X,\Oo(-n-1))\overset{\text{\itememph{d}}}{\simeq}A\\
				(f,[g])&\longmapsto [fg]\;,
			\end{split}
		\end{align}
		where for $f\in\Oo(k)(X)$ and $g\in\check{C}_\alt^n(\Uu,\Oo(-n-1-k))$,  $fg\in\check{C}_\alt^n(\Uu,\Oo(-n-1))$ is defined by $(fg)_{i_0,\ldots,i_n}=f|_{U_{i_0,\ldots,i_n}}\cdot g_{i_0,\ldots,i_n}$ for all indices $i_0,\ldots,i_n$.
	\end{alphanumerate}
\end{thm}
\begin{rem*}
	The pairing in \eqreff{eq:nonDegCohoPairing} is canonical but the isomorphism $H^n(X,\Oo(-n-1))\simeq A$ is not.
\end{rem*}
\begin{proof}[Proof of Theorem~\reff{thm:CohoOfIP^n}]
	Surprisingly, the proof will be almost trivial once we discover the connection between the \vC ech complex and the Koszul complex. By \eqreff{eq:O(k)isQC}, on $U_{i_0,\ldots,i_\ell}= \IP_A^n\setminus V(X_{i_0}\cdots X_{i_\ell})$ we have
	\begin{align*}
		\Oo(k)(U_{i_0,\ldots,i_\ell})\simeq R[(X_{i_0}\cdots X_{i_\ell})^{-1}]_k
	\end{align*}
	for all $k\in \IZ$. Considering the sheaf of graded rings $\Oo=\bigoplus_{k\in\IZ}\Oo(k)$ as before, this can be written in a more compact way as
	\begin{align*}
		\Oo(U_{i_0,\ldots,i_\ell})\simeq R[(X_{i_0}\cdots X_{i_\ell})^{-1}]\;.
	\end{align*}
	Note that we have isomorphisms
	\begin{align}\lbl{eq:CechKoszulIso1}
			R[(X_{i_0}\cdots X_{i_\ell})^{-1}]\lisomorphism \colimit \left(R\xrightarrow{\cdot(X_{i_0}\cdots X_{i_\ell})}R\xrightarrow{\cdot(X_{i_0}\cdots X_{i_\ell})}R\xrightarrow{\cdot(X_{i_0}\cdots X_{i_\ell})}\ldots\right)\tag{$*$}
	\end{align}
	sending the image of $r\in R$ (as an element of the $j\ordinalth$ member of the sequence) in the colimit to $r\cdot (X_{i_0}\cdots X_{i_\ell})^{-j}$.
	
	For every $j\in\IN$ we get a morphism
	\begin{align*}
		\check{C}_\alt^\ell(\Uu,\Oo)\simeq \prod_{0\leq i_0<\cdots<i_\ell\leq n}R[(X_{i_0}\cdots X_{i_\ell})^{-1}]\lmorphism \bigwedge^{\ell+1}R^{n+1}\simeq K^{\ell+1}\big((X_0^j,\ldots,X_n^j),R\big)
	\end{align*}
	sending $e_{i_0}\wedge \cdots \wedge e_{i_\ell}$ to $(0,\ldots,(X_{i_0}\cdots X_{i_\ell})^{-j},\ldots,0)$ (with the only non-zero entry at position $(i_0,\ldots,i_\ell)$, obviously). Using $\bigwedge^{\ell+1}R^{n+1}\simeq \bigoplus _{0\leq i_0<\cdots <i_\ell\leq n}R$ and \eqreff{eq:CechKoszulIso1}, these morphisms assemble to an isomorphism
	\begin{align*}
		\check{C}_\alt^\ell(\Uu,\Oo)\lisomorphism \colimit[j\in \IN] K^{\ell+1}\big((X_0^j,\ldots,X_n^j),R\big)\;,
	\end{align*}
	where the transition morphisms $K^{\ell+1}\big((X_0^j,\ldots,X_n^j),R\big)\morphism K^{\ell+1}\big((X_0^{j+1},\ldots,X_n^{j+1}),R\big)$ (with respect to which the colimit is taken) sends an element $f\in K^{\ell+1}\big((X_0^j,\ldots,X_n^j),R\big)$ to $\phi\in K^{\ell+1}\big((X_0^{j+1},\ldots,X_n^{j+1}),R\big)$ defined by $\phi(t_1,\ldots,t_{\ell+1})=X_{t_1}\cdots X_{t_{\ell+1}}f(t_1,\ldots,t_{\ell+1})$. Convince yourself that this is compatible with the Koszul differential, so in fact we get a transition map $K^\bullet\big((X_0^j,\ldots,X_n^j),R\big)\morphism K^\bullet\big((X_0^{j+1},\ldots,X_n^{j+1}),R\big)$, and using this an isomorphism
	\begin{align}\lbl{eq:CechKoszulIso2}
		\left(R\morphism\check{C}_\alt^0(\Uu,\Oo)\morphism\check{C}_\alt^1(\Uu,\Oo)\morphism\ldots\right)\lisomorphism \colimit[j\in\IN]K^\bullet\big((X_0^j,\ldots,X_n^j),R\big)[-1]
	\end{align}
	of cochain complexes. The degree-$(-1)$ term $S$ comes from the fact that the $K^\bullet\big((X_0^j,\ldots,X_n^j),R\big)$ for all $j\in\IN$ have such a term in degree $0$.
	
	Note that if $n=0$, then $\IP_A^0=\Proj A[X_0]\simeq \Spec A$ and everything follows from well-known properties of affine schemes. Now assume $n>0$. By Example~\reff{ex:(X0Xn)regular}\itememph{a}, the sequence $(X_0,\ldots,X_n)$ is a regular sequence in $R$. By Fact~\reff{fact:RegularSeqKoszulComplex}\itememph{d}, so are $(X_0^j,\ldots,X_n^j)$ for $j\in\IN$. We can now reduce the various assertions about the cohomology of $X=\IP_A^n$ to their Koszul counterparts.
	
	Part~\itememph{a}. By Fact~\reff{fact:RegularSeqKoszulComplex}\itememph{b}, $H^p\big((X_0^j,\ldots,X_n^j),R\big)=0$ in degree $p\neq n+1$. In particular, by exactness of filtered colimits, equation \eqreff{eq:CechKoszulIso2} gives
	\begin{align*}
		H^0\left(R\morphism\check{C}_\alt^\bullet(\Uu,\Oo)\right)\lisomorphism\colimit[j\in\IN]H^1\big((X_0^j,\ldots,X_n^j),R\big)=0\;,
	\end{align*}
	hence the kernel $\ker\left(\check{C}_\alt^0(\Uu,\Oo)\morphism\check{C}_\alt^1(\Uu,\Oo)\right)\simeq H^0(X,\Oo)\simeq \Oo(X)$ equals the image of $S$ in $\check{C}_\alt^0(\Uu,\Oo)$. Thus $\Oo(X)\simeq S$ and \itememph{a} follows by comparing degrees.
	
	Part~\itememph{b}. Just as in \itememph{a}, for $0<p<n$ we obtain
	\begin{align*}
		H^p\left(R\morphism\check{C}_\alt^\bullet(\Uu,\Oo)\right)\lisomorphism\colimit[j\in\IN]H^{p+1}\big((X_0^j,\ldots,X_n^j),R\big)=0\;,
	\end{align*}
	from Fact~\reff{fact:RegularSeqKoszulComplex}\itememph{b}, hence $H^p(X,\Oo(k))=0$ for all $k\in\IZ$ by comparing degrees.
	
	And finally parts~\itememph{c}, \itememph{d}, and \itememph{e} follow in a similar colimit-taking fashion from Corollary~\reff{cor:KoszulCoho}\itememph{a}, \itememph{b}, and \itememph{c}.
\end{proof}
Our next aim is to investigate the trivialization of the $A$-module $H^n(\IP_A^n,\Oo(-n-1))$ by Theorem~\reff{thm:CohoOfIP^n}\itememph{d}, which is non-canonical unless the coordinates and their order are fixed. What happens when the coordinates are changed by an element of $\GL_{n+1}(A)$ (inducing an automorphism of $R$, hence of $\Proj R=\IP_A^n$) with a compatible action on the line bundles $\Oo(k)$?
\begin{lem}
	An element $g\in\GL_{n+1}(A)$ acts on $H^n(\IP_A^n,\Oo(-n-1))$ by multiplication with $\det(g)^{-1}$.
\end{lem}
\begin{proof}
	We will reduce this to the case where $A=K$ is an algebraically closed field. It is easy to see that changing the trivializations of $H^n(\IP_A^n,\Oo(-n-1))$ gives an action of $\GL_{n+1}(A)$ on $A$, hence a group homomorphism 
	\begin{align*}
		\phi_A\colon \GL_{n+1}(A)\morphism\Aut_A(A)\simeq A^\times\;,
	\end{align*}
	where $\Aut_A(A)$ is the automorphism group of $A$ as an $A$-module. 
	
	If $A\morphism[\alpha]B$ is a ring morphism, we get a graded ring morphism $A[X_0,\ldots,X_n]\morphism[\alpha] B[X_0,\ldots,X_n]$ defining a morphism of preschemes $\IP_B^n\morphism[p]\IP_A^n$ together with morphisms $\Oo_{\IP_A^n}(k)\morphism p_*\Oo_{\IP_B^n}(k)$ defined by the above ring morphism $\alpha$ on open subsets of the form $\IP_A^n\setminus V(X_i)$. 
	
	Note that $\IP_B^n\morphism[p]\IP_A^n$ is an affine morphism by Proposition~\reff{prop:AffineMorphism}\itememph{b}, since the preimage of $\IP_A^n\setminus V(X_i)$ (which is affine by Proposition~\reff{prop:FirstProjProperties}\itememph{c}) is $\IP_B^n\setminus V(X_i)$ (which is affine). In particular, we get
	\begin{align*}
		H^n\left(\IP_A^n,\Oo_{\IP_A^n}(-n-1)\right)\morphism[p^*]H^n\left(\IP_A^n,p_*\Oo_{\IP_B^n}(-n-1)\right)\isomorphism H^n\left(\IP_B^n,\Oo_{\IP_B^n}(-n-1)\right)\;,
	\end{align*}
	the isomorphism coming from Corollary~\reff{cor:AffinePushforwardIso}. By naturality of our cohomology constructions\footnote{If $\Uu\colon \IP_A^n=\bigcup_{i=0}^n\IP_A^n\setminus V(X_i)$ is our standard affine open cover, let $g(\Uu)\colon \IP_A^n=\bigcup_{i=0}^n\IP_A^n\setminus V(g(X_i))$ be its image under $g$. Then $g$ induces a map of \vC ech complexes $\check{C}^\bullet (\Uu,\Oo(k))\morphism\check{C}^\bullet (g(\Uu),\Oo(k))$ and this is all natural and stuff; and both \vC ech complexes give $H^n(\IP_A^n,\Oo(k))$ by Theorem~\reff{thm:CohoOnSchemes}\itememph{a}.}
	\begin{diagram*}
		\node[ob] (a) at (0,1.5) {$H^n\left(\IP_A^n,\Oo_{\IP_A^n}(-n-1)\right)$};
		\node[ob] (c) at (0,0) {$H^n\left(\IP_A^n,\Oo_{\IP_A^n}(-n-1)\right)$};
		\node[ob] (b) at (5,1.5) {$H^n\left(\IP_B^n,\Oo_{\IP_B^n}(-n-1)\right)$};
		\node[ob] (d) at (5,0) {$H^n\left(\IP_B^n,\Oo_{\IP_B^n}(-n-1)\right)$};
		\scriptsize
		\draw[->] (a) -- (b) node[pos=0.5, above] {$p^*$};
		\draw[->] (c) -- (d) node[pos=0.5, above] {$p^*$};
		\draw[->] (a) -- (c) node[pos=0.5, left] {$g$};
		\draw[->] (b) -- (d) node[pos=0.5, right] {$\alpha(g)$};
	\end{diagram*} 
	is a commutative diagram, showing that $\phi_B(\alpha(g))=\alpha(\phi_A(g))$. In general, there is a ring morphism
	\begin{align*}
		B=\IZ\left[Y_{i,j}\st i,j=0,\ldots,n\right][\det(Y_{i,j})^{-1}]\morphism A
	\end{align*}
	sending $Y_{i,j}$ to the corresponding entry in the matrix $g$ and $\det(Y_{i,j})^{-1}$ to $\det(g)^{-1}$. This reduces the assertion to the case of the infinite domain $B$. Embedding $B$ into an algebraic closure $K$ of its field of quotients, we did the advertised reduction.
	
	In this case, the morphism $\phi_K\colon \GL_{n+1}(K)\morphism K^\times$ factors over the abelianization of $\GL_{n+1}(K)$. For algebraically closed fields (actually, for fields with more than $3$ elements) the commutator subgroup of $\GL_{n+1}(K)$ is $\SL_{n+1}(K)$. Moreover, every element $C\in\GL_{n+1}(K)$ can be written as the product of a matrix $\lambda\id_{K^{n+1}}$ and a matrix in $\SL_{n+1}(K)$ (using that $\det C$ has an $(n+1)\ordinalst$ root in the algebraically closed field $K$). This reduces to the case $g=\lambda\id_{K^{n+1}}$. In this case, it is easily seen from Proposition~\reff{prop:FirstProjProperties}\itememph{c} that $g$ acts as the identity on $\IP_K^n$ and as multiplication by $\lambda^k$ on $\Oo_{\IP_K^n}(k)$. This proves our claim.
\end{proof}
\begin{defi}\lbl{def:AmpleLineBundle}
	A line bundle $\Ll$ on a quasi-compact prescheme $X$ is called \defemph{ample} if it satisfies the following equivalent conditions.
	\begin{alphanumerate}
		\item For every locally finitely generated quasi-coherent $\Oo_X$-module $\Mm$ there is an integer $m(\Mm)$ such that for all $m<m(\Mm)$, there is an epimorphism
		\begin{align*}
		(\Ll^{\otimes m})^{\oplus n}\epimorphism \Mm\;,
		\end{align*}
		for some natural number $n=n(m)$.
		\item For every quasi-coherent locally finitely generated $\Oo_X$-module $\Mm$ there is an integer $k(\Mm)$ such that for all $k>k(\Mm)$, $\Ll^{\otimes k}\otimes_{\Oo_X}\Mm$ is \defemph{generated by its global sections}. This shall mean the following: There is a family of global sections $(\lambda_i)_{i\in I}$ such that each stalk $\Ll_x^{\otimes k}\otimes_{\Oo_{X,x}}\Mm_x$ is generated as an $\Oo_{X,x}$-module by the images of the $\lambda_i$.
	\end{alphanumerate}
\end{defi}
\begin{rem*}
	\begin{alphanumerate}
		\item In Definition~\reff{def:AmpleLineBundle} we put $\Ll^{-1}=\Hhom_{\Oo_X}(\Ll,\Oo_X)$ and $\Ll^{\otimes m}=(\Ll^{-1})^{\otimes (-m)}$ if $m\in\IZ$ is negative. These are line bundles again, because in the case $\Ll=\Oo_X$ they are canonically isomorphic to $\Oo_X$, which locally and up to isomorphism is always the case. For the same reason, the morphism
		\begin{align*}
			\Ll\otimes_{\Oo_X}\Ll^{-1}\morphism\Oo_X
		\end{align*}
		defined by the bilinear map of evaluation $\Ll\times\Hhom_{\Oo_X}(\Ll,\Oo_X)\morphism\Oo_X$, $(\lambda,\eta)\mapsto\eta(\lambda)$ and the universal property of the (sheaf) tensor product (cf.\ Remark~\reff{rem:tensorProductSheaves}) is an isomorphism. 
		
		More generally, evaluation gives an isomorphism 
		\begin{align*}
			\Ll\otimes_{\Oo_X}\Hhom_{\Oo_X}(\Ll,\Mm)\isomorphism\Mm
		\end{align*}
		for every quasi-coherent $\Oo_X$-module $\Mm$. Isomorphism classes of line bundles on $X$ get an abelian group structure with $-\otimes_{\Oo_X}-$ defining multiplication, $\Oo_X$ the neutral element and $\Ll^{-1}$ the inverse to $\Ll$.
		
		Ok, actually, we have yet to check that our construction of inverse elements is compatible with the group structure, i.e., $(\Ll\otimes_{\Oo_X}\Mm)^{-1}\lisomorphism \Ll^{-1}\otimes_{\Oo_X}\Mm^{-1}$. To obtain a sheaf isomorphism 
		\begin{align*}
			\Hhom_{\Oo_X}(\Ll,\Oo_X)\otimes_{\Oo_X}\Hhom_{\Oo_X}(\Mm,\Oo_X)\isomorphism\Hhom_{\Oo_X}(\Ll\otimes_{\Oo_X}\Mm,\Oo_X)
		\end{align*}
		we can map $f\otimes g\in\Hom_{\Oo_U\cat{-Mod}}(\Ll|_U,\Oo_U)\otimes_{\Oo_X(U)}\Hom_{\Oo_U\cat{-Mod}}(\Mm|_U,\Oo_U)$ (for an open subset $U\subseteq X$) to the morphism $(\Ll\otimes_{\Oo_X}\Mm)|_U\morphism\Oo_U$ sending $\lambda\otimes\mu$ to $f(\lambda)\cdot g(\mu)$. It is clear that this is locally (where $\Ll$ and $\Mm$ trivialize) an isomorphism, hence an isomorphism in total.
		\item On $X=\IP_A^n$, there is an isomorphism
		\begin{align*}
			\Oo(k)&\isomorphism\Hhom_{\Oo_{\IP_A^n}}\big(\Oo(\ell),\Oo(\ell+k)\big)\\
			f&\longmapsto (g\mapsto f\cdot g)\;.
		\end{align*}
		\item Definition~\reff{def:AmpleLineBundle}\itememph{b} seems way more common in the literature.
		\item Don't confuse Definition~\reff{def:AmpleLineBundle} with a \emph{relative} notion of ampleness.
	\end{alphanumerate}
\end{rem*}
\begin{proof}[Proof of Definition~\reff{def:AmpleLineBundle} (this has to be)]
	First note that $\Ll^{\otimes k}\otimes_{\Oo_X}\Mm$ is still locally finitely generated, so by quasi-compactness of $X$ the family in \itememph{b} can be chosen finite (I will leave that to you).
	
	We show \itememph{a} $\Rightarrow$ \itememph{b} first. Let $k>-m(\Mm)$. Then there is an epimorphism $(\Ll^{\otimes (-k)})^{\oplus n}\epimorphism \Mm$. Tensoring with $\Ll^{\otimes k}$ gives an epimorphism $\Oo_X^{\oplus n}\epimorphism \Ll^{\otimes k}\otimes_{\Oo_X}\Mm$ (tensor products of sheaves preserve epimorphisms, since this clearly holds on stalks). Then the images of the standard unit vectors in $\Oo_X(X)^{\oplus n}$ generate $\Ll^{\otimes k}\otimes_{\Oo_X}\Mm$.
	
	And \itememph{b} $\Rightarrow$ \itememph{a}. Let $m<-k(\Mm)$. Since $\Ll^{\otimes(-m)}\otimes_{\Oo_X}\Mm$ is generated by finitely many global sections, say, $n$ of them, there is an epimorphism $\Oo_X^{\oplus n}\epimorphism\Ll^{\otimes(-m)}\otimes_{\Oo_X}\Mm$. Tensoring with $\Ll^{\otimes m}$ gives an epimorphism $(\Ll^{\otimes m})^{\oplus n}\epimorphism\Mm$.
\end{proof}
\begin{defi}
	Let $\Ll$ be a line bundle on a locally ringed space $(X,\Oo_X)$, with $\mm_x\subseteq \Oo_{X,x}$ the maximal ideal of the local ring at $x$. If $U\subseteq X$ is open and $\lambda\in\Ll(U)$, its \defemph{vanishing set} is
	\begin{align*}
		V(\lambda)=\left\{x\in U\st\text{the image of }\lambda\text{ in }\Ll_x\text{ is in }\mm_x\Ll_x\right\}\;.
	\end{align*}
\end{defi}
\begin{rem*}
	When $\Ll=\Oo_X$, this coincides with the vanishing sets introduced in \cite[Definition~1.3.3]{alggeo1}. Since locally and up to isomorphism this is always the case, $V(\lambda)$ is closed in $U$ by \cite[Fact~1.3.1]{alggeo1} and the usual trick in such situations (\cite[Remark~1.1.4\itememph{a}]{alggeo1}). Similarly, $V(\lambda\otimes \mu)=V(\lambda)\cup V(\mu)$ for the section $\lambda\otimes\mu$ of $\Ll\otimes_{\Oo_X}\Mm$ when $\Ll$ and $\Mm$ are line bundles. 
\end{rem*}
\begin{lem}\lbl{lem:powersOfLl}
	Let $X$ be a quasi-compact prescheme, $\Ll$ a line bundle on $X$, $\lambda\in\Ll(X)$ and $\Mm$ a quasi-coherent $\Oo_X$-module.
	\begin{alphanumerate}
		\item If $m\in\Mm(X)$ with $m|_{X\setminus V(\lambda)}=0$, then there is $k\in\IN$ such that the section $\lambda^{\otimes k}\otimes m=0$ in $\Ll^{\otimes k}\otimes_{\Oo_X}\Mm$.
		\item If $X$ is in addition quasi-separated and $m\in\Mm(X\setminus V(\lambda))$, then there is $k\in\IN$ such that $\lambda^{\otimes k}\otimes m\in(\Ll^{\otimes k}\otimes_{\Oo_X}\Mm)(X\setminus V(\lambda))$ extends to a global section of $\Ll^{\otimes k}\otimes_{\Oo_X}\Mm$ on $X$.
	\end{alphanumerate}
\end{lem}
\begin{rem*}
	When $\Ll=\Oo_X$, this is identical with one of our definitions of quasi-coherentness for sheaves of $\Oo_X$-modules, as we will immediately see.
\end{rem*}
\begin{proof}[Proof of Lemma~\reff{lem:powersOfLl}]
	When $\Ll=\Oo_X$, assertion \itememph{a} is equivalent to the injectivity and \itememph{b} to the surjectivity of 
	\begin{align*}
		\Mm(X)_\lambda\xrightarrow{\text{\eqreff{eq:qcIso}}}\Mm(X\setminus V(\lambda))\;.
	\end{align*}
	Since this holds when $\Mm$ is quasi-coherent and $X$ is quasi-compact (resp.\ quasi-compact and quasi-separated) by Definition~\reff{def:Quasicoherent}\itememph{d}, the assertion is true for $\Ll=\Oo_X$.
	
	Since $X$ is quasi-compact, we have a finite cover $X=\bigcup_{i=1}^nU_i$ by affine open subsets $U_i$ such that $\Ll|_{U_i}$ is trivial. Also put $U_{i,j}=U_i\cap U_j$ as in our \vC ech complex convention.
	
	For \itememph{a}, we have $(m|_{U_i})|_{U_i\setminus V(\lambda)}=0$, hence by the special case treated above there is $k_i\in\IN$ such that $\lambda^{\otimes k_i}\otimes m=0$ on $U_i$. Taking $k=\max_ik_i$ we get $\lambda^{\otimes k}\otimes m=0$ on all of $X$.
	
	For \itememph{b} we use the above special case again to obtain $\ell_i\in\IN$ and $\mu_i\in(\Ll^{\otimes \ell_i}\otimes_{\Oo_X}\Mm)(U_i)$ such that $\lambda^{\otimes \ell_i}\otimes m=\mu_i$ on $U_i$. Putting $\ell=\max_i \ell_i$ and replacing $\mu_i$ by $\lambda^{\otimes (\ell-\ell_i)}\otimes \mu_i$ we may assume $\ell_1=\ldots=\ell_n=\ell$. As $X$ is quasi-separated, the $U_{i,j}=U_i\cap U_j$ are quasi-compact. Since $\mu_i|_{U_{i,j}\setminus V(\lambda)}=\mu_j|_{U_{i,j}\setminus V(\lambda)}$, by \itememph{a} there are $k_{i,j}$ such that $\lambda^{\otimes k_{i,j}}\otimes \mu_i=\lambda^{\otimes k_{i,j}}\otimes \mu_j$ on $U_{i,j}$. Put $k=\max_{i,j}k_{i,j}$, then $\lambda^{\otimes k}\otimes \mu_i|_{U_{i,j}}=\lambda^{\otimes k}\otimes \mu_j|_{U_{i,j}}$, hence by the sheaf axiom the $\lambda^{\otimes k}\otimes \mu_i$ extend to a global section $\mu\in\Ll^{\otimes k}\otimes_{\Oo_X}\Mm$. Then $\mu|_{X\setminus V(\lambda)}=\lambda^{\otimes(k+\ell)}\otimes m$, so $\mu$ extends $\lambda^{\otimes(k+\ell)}\otimes m$.
\end{proof}
\begin{prop}\lbl{prop:AmplenessCriterion}
	Let $X$ be a quasi-compact prescheme and $\Ll$ a line bundle on $X$ such that $X$ may be covered by affine open subsets of the form $X\setminus V(\lambda)$ where $\lambda\in\Ll(X)$. Then $\Ll$ is ample.
\end{prop}
\begin{cor}
	\begin{alphanumerate}
		\item The line bundle $\Oo(1)$ on $\IP_A^n$ is ample.
		\item If $X$ is quasi-affine (i.e., an open subscheme of an affine scheme), then $\Oo_X$ is ample.
	\end{alphanumerate}
\end{cor}
\begin{proof}
	For part \itememph{a}, note that the $\IP_A^n\setminus V(X_i)$ are affine and cover $\IP_A^n$. For \itememph{b}, write $X=\ov{X}\setminus Z$ with $\ov{X}$ affine and $Z$ closed, then the open subsets of the form $X\setminus V(f|_X)=\ov{X}\setminus V(f)$ with $f\in\Oo_{\ov{X}}(\ov{X})$ such that $V(f)\supseteq Z$ are affine and cover $X$. In both cases we win by Proposition~\reff{prop:AmplenessCriterion}.
\end{proof}

\appendix
\chapter{Appendix -- category theory corner}
\setcounter{thm}{0}
\renewcommand*{\thethm}{\Alph{thm}}
\section{Towards abelian categories}
\begin{defi}
	\begin{alphanumerate}
		\item \lbl{def:additiveCategory}A \defemph{pointed} category is a category with initial and final objects, such that the canonical (unique) morphism from the initial to the final object is an isomorphism.
		\item An \defemph{additive} category $\Aa$ is a pointed category which has a product $X\times Y$ (i.e., a fibre product over the final object $*$) and coproduct $X\amalg Y$ (i.e., a dual fibre product with respect to the initial object $*$) such that the canonical morphism $X\amalg Y\morphism X\times Y$ is an isomorphism for all objects $X,Y\in\Ob(\Aa)$ and such that the resulting addition law on $\Hom_\Aa(X,Y)$ defines a group structure for all $X,Y\in\Ob(\Aa)$.
	\end{alphanumerate}
\end{defi}
\begin{rem*}
	 \begin{alphanumerate}
	 	\item When $\Aa$ is a pointed category and $X,Y\in\Ob(\Aa)$, let the \emph{zero morphism} (which we denote $0$) $X\morphism[0]Y$ be defined by $X\morphism *\morphism Y$, where $*$ is the both initial and final object.
	 	\item We will construct the canonical morphism $X\amalg Y\morphism[c]X\times Y$ from Definition~\reff{def:additiveCategory}\itememph{b}. The product $X\times Y$ comes with canonical projections $X\lmorphism[p_1]X\times Y\morphism[p_2]Y$ such that given morphisms $T\morphism[\xi]X$ and $T\morphism[\upsilon]Y$ there is a unique $T\morphism[\xi\times\upsilon]X\times Y$ such that
	 	\begin{diagram*}
	 		\node (XY) at (0,0) {$X\times Y$};
	 		\node (X) at (-1,1.25) {$X$};
	 		\node (Y) at (-1,-1.25) {$Y$};
	 		\node (T) at (2.5,0) {$T$};
	 		\scriptsize
	 		\draw[->] (XY) -- (X) node[pos=0.5, above right] {$p_1$};
	 		\draw[->] (XY) -- (Y) node[pos=0.5, below right] {$p_2$};
	 		\draw[->, dashed] (T) -- (XY) node[pos = 0.5, above] {$\exists!\ \xi\times \upsilon$};
	 		\draw[->, bend right] (T) to node[pos=0.5,below left] {$\xi$} (X);
	 		\draw[->, bend left] (T) to node[pos=0.5,above left] {$\upsilon$} (Y);
	 	\end{diagram*}
	 	commutes.
	 	
	 	Similarly, the coproduct $X\amalg Y$ has morphisms $X\morphism[i_1]X\amalg Y\lmorphism[i_2]Y$ such that given morphisms $X\morphism[\xi]T$ and $Y\morphism[\upsilon]T$ there is a unique morphism $X\amalg Y\morphism[\xi\amalg\upsilon]T$ such that
	 	\begin{diagram*}
	 		\node (XY) at (0,0) {$X\amalg Y$};
	 		\node (X) at (-1,1.25) {$X$};
	 		\node (Y) at (-1,-1.25) {$Y$};
	 		\node (T) at (2.5,0) {$T$};
	 		\scriptsize
	 		\draw[<-] (XY) -- (X) node[pos=0.5, above right] {$i_1$};
	 		\draw[<-] (XY) -- (Y) node[pos=0.5, below right] {$i_2$};
	 		\draw[->, dashed] (XY) -- (T) node[pos = 0.5, above] {$\exists!\ \xi\amalg \upsilon$};
	 		\draw[<-, bend right] (T) to node[pos=0.5,below left] {$\xi$} (X);
	 		\draw[<-, bend left] (T) to node[pos=0.5,above left] {$\upsilon$} (Y);
	 	\end{diagram*}
	 	commutes.
	 	
	 	Using the universal property of $X\times Y$, we get a unique morphism $X\morphism[\alpha]X\times Y$ such that $p_1\alpha=\id_X$, $p_2\alpha=0$ and a unique morphism $Y\morphism[\beta]X\times Y$ such that $p_1\beta=0$ and $p_2\beta=\id_Y$. Then 
	 	\begin{align*}
	 		c\colon X\amalg Y\xrightarrow{\alpha\amalg\beta}X\times Y
	 	\end{align*}
	 	is the morphism we are looking for. It is unique with the property that $p_1 c i_1=\id_X$, $p_1 c i_2=0$, $p_2 c i_1=0$, and $p_2 c i_2=\id_Y$.
	 	\item For abelian groups and modules over a ring, both $X\amalg Y$ and $X\times Y$ are given by $\left\{(x,y)\st x\in X,\ y\in Y\right\}$ with component-wise operations and $p_1(x,y)=x$, $p_2(x,y)=y$, $i_1(x)=(x,0)$, and $i_2(y)=(0,y)$.
	 	\item For an additive category $\Aa$, it follows that finite products $\prod_{i=1}^nX_i$ and coproducts $\coprod_{i=1}^nX_i$ (of some objects $X_1,\ldots,X_n\in\Ob(\Aa)$) exist and are canonically isomorphic. We typically denote both by $\bigoplus_{i=1}^nX_i$ in that case.
	 	\item We would like to describe the addition on $\Hom_\Aa(X,Y)$. For a pair of morphisms $X\doublemorphism[a][b]Y$ we denote the composition
	 	\begin{align*}
	 		X\xrightarrow{\id_X\times\id_X}X\oplus X\xrightarrow{a\amalg b}Y
	 	\end{align*}
	 	by $a+b$. Then $0$ is a neutral element and associativity holds, but the existence of inverse elements needs to be imposed to obtain indeed a group structure.
	 	
	 	One can show that this is the only possible group structure behaving bilinearly under composition.
	 	\item It is, however, automatically abelian. What we need to show is $(a\amalg b)\circ\Delta=(b\amalg a)\circ\Delta$ with $\Delta=\id_X\times \id_X$. The universal property of coproducts gives a unique $X\oplus X\morphism[\sigma]X\oplus X$ such that
	 	\begin{diagram*}
	 		\node (XX2) at (0,0) {$X\oplus X$};
	 		\node (X1) at (-1,1.25) {$X$};
	 		\node (X2) at (-1,-1.25) {$X$};
	 		\node (XX1) at (-2,0) {$X\oplus X$};
	 		\node (Y) at (2.5,0) {$Y$};
	 		\scriptsize
	 		\draw[<-] (XX2) -- (X1) node[pos=0.5, above right] {$i_1$};
	 		\draw[<-] (XX2) -- (X2) node[pos=0.5, below right] {$i_2$};
	 		\draw[->] (XX2) -- (Y) node[pos = 0.5, above] {$a\amalg b$};
	 		\draw[<-, bend right] (Y) to node[pos=0.5,below left] {$a$} (X1);
	 		\draw[<-, bend left] (Y) to node[pos=0.5,above left] {$b$} (X2);
	 		\draw[<-] (XX1) -- (X1) node[pos=0.5, above left] {$i_2$};
	 		\draw[<-] (XX1) -- (X2) node[pos=0.5, below left] {$i_1$};
	 		\draw[->, dashed] (XX1) -- (XX2) node[pos=0.5, above] {$\exists!\ \sigma$};
	 	\end{diagram*}
	 	commutes. Then $\sigma$ is easily seen to be an isomorphism and $b\amalg a=(a\amalg b)\circ \sigma$ by the uniqueness of $b\amalg a$. It thus suffices to show $\sigma\Delta=\Delta$. By the uniqueness of $\Delta$, this is equivalent to $p_1\sigma\Delta=\id_X$ and $p_2\sigma\Delta=\id_X$. We claim that $p_1\sigma=p_2$ and vice versa, which would finish the proof. To see this, note that $p_1\sigma=p_2$ is equivalent to $p_1\sigma i_1=p_2i_1=0$ and $p_1\sigma i_2=p_2i_2=\id_X$ by the universal property of the coproduct $X\oplus X$. This follows from $\sigma i_1=i_2$ and $\sigma i_2=i_1$ by definition of $\sigma$.
	 \end{alphanumerate}
\end{rem*}
\begin{example*} The following are additive categories.
	\begin{alphanumerate}
		\item Modules over a given ring $R$ (in particular, abelian groups).
		\item Sheaves of modules.
		\item Banach spaces with bounded linear maps as morphisms. The common initial and final object is the zero space and $A\oplus B=\left\{(a,b)\st a\in A,\ b\in B\right\} $ with $\max\{\|a\|,\|b\|\}$ or $\|a\|+\|b\|$ as norm (this category will turn out not to be abelian).
		\item Free or projective modules over a ring $R$.
	\end{alphanumerate}
\end{example*}
\begin{defi}\lbl{def:Kernel}
	Let $A\morphism[\alpha]B$ be a morphism in an additive category $\Aa$. The \defemph{kernel} $\ker(A\morphism[\alpha]B)$ of $\alpha$ (if it exists) comes with a morphism $\ker(\alpha)\morphism[\iota]A$ satisfying the universal property
	\begin{align*}
	\Hom_\Aa\left(T,\ker\Big(A\morphism[\alpha]B\Big)\right)&\isomorphism \left\{f\in \Hom_\Aa\st \alpha f=0\right\}\\
	\left(T\morphism[\tau]A\right)&\longmapsto f=\iota\tau
	\end{align*}
	for any test object $T\in\Ob(\Aa)$. 
\end{defi}
\begin{varthm}{defi}\lbl{def:Cokernel}
	Similarly, the \defemph{cokernel} of $\alpha$ (if existent) comes with a morphism $B\morphism[\pi]\coker(\alpha)$ and satisfies
	\begin{align*}
	\Hom_\Aa\left(\coker\Big(A\morphism[\alpha]B\Big),T\right)&\isomorphism \left\{g\in\Hom_\Aa(B,T)\st g\alpha=0\right\}\\
	\left(\coker(\alpha)\morphism[\tau]T\right)&\longmapsto g=\tau\pi
	\end{align*}
	for any test object $T\in\Ob(\Aa)$.
\end{varthm}
	
	
\begin{rem}\lbl{rem:KernelCokernel}
	Kernels and cokernels in an additive category $\Aa$ are special cases of \emph{equalizers} and \emph{coequalizers} (cf.\ \cite[Definition~A.3.2 and Definition~A.3.4]{alggeo1}), respectively. Indeed, we have
	\begin{align*}
		\ker\Big(A\morphism[\alpha]B\Big)=\Eq\Big(A\doublemorphism[\alpha][0]B\Big)\quad\text{and}\quad\coker\Big(A\morphism[\alpha]B\Big)=\Coeq\Big(A\doublemorphism[\alpha][0]B\Big)\;. 
	\end{align*}
	But we can reconstruct equalizers and coequalizers from kernels and cokernels via
	\begin{align*}
		\Eq\Big(A\doublemorphism[\alpha][\beta]B\Big)=\ker\Big(A\xrightarrow{\alpha-\beta}B\Big)\quad\text{and}\quad \Coeq\Big(A\doublemorphism[\alpha][\beta]B\Big)=\coker\Big(A\xrightarrow{\alpha-\beta}B\Big)
	\end{align*}
	(the minus here is the one obtained from additivity of $\Aa$).
	
	\begin{defi}\lbl{def:EffectiveMono}
		A morphism $A\morphism[i]B$ is an \defemph{effective monomorphism}, if the following equivalent conditions hold.
		\begin{alphanumerate}
			\item (In any category) We have a bijection
			\begin{align*}
			\Hom_\Aa(T,A)&\isomorphism\left\{f\in\Hom_\Aa(T,B)\st 
			\begin{array}{c}
			\alpha f=\beta f\text{ if }B\doublemorphism[\alpha][\beta]S\text{ is any pair of}\\ 
			\text{morphisms such that }\alpha i=\beta i
			\end{array}
			\right\}\\
			t\in\Hom_\Aa(T,A) &\longmapsto f=it\;.
			\end{align*}
			\item (If the category has finite colimits) $i$ is an equalizer of something.
			\item (In additive categories with kernels and cokernels) $i$ is the kernel of an appropriate morphism.
			\item (In additive categories with kernels and cokernels) $i$ is the kernel of its cokernel.
		\end{alphanumerate}
	\end{defi}
	\begin{varthm}{defi}\lbl{def:EffectiveEpi}
		Dually, $A\morphism[p]B$ is an \defemph{effective epimorphism} if the following equivalent conditions hold.
		\begin{alphanumerate}
			\item (In any category) We have a bijection
			\begin{align*}
			\Hom_\Aa(B,T)&\isomorphism\left\{f\in\Hom_\Aa(A,T)\st 
			\begin{array}{c}
			f\alpha =f\beta\text{ if }S\doublemorphism[\alpha][\beta]A\text{ is any pair of}\\
			\text{morphisms such that }p\alpha =p\beta 
			\end{array}
			\right\}\\
			t\in\Hom_\Aa(B,T) &\longmapsto f=tp\;.
			\end{align*}
			\item (If the category has finite limits) $p$ is a coequalizer of something.
			\item (In additive categories with kernels and cokernels) $p$ is the cokernel of an appropriate morphism.
			\item (In additive categories with kernels and cokernels) $p$ is the cokernel of its kernel.
			\item $B^\op\morphism[p^\op]A^\op$ is an effective monomorphism in the dual category $\Aa^\op$.
		\end{alphanumerate}
	\end{varthm}
	In any category, a morphism which is mono and effectively epi (or epi and effectively mono) is an isomorphism, but there are examples of morphisms which are simultaneously mono and epi but not an isomorphism (e.g.\ $\IZ\monomorphism\IQ$ in the category of rings). This needs to be ruled out by a definition, and that's what is happening now! 
\end{rem}
\begin{defi}
	A category $\Aa$ is \defemph{abelian}, if it is additive, has kernels and cokernels and such that every monomorphism is effectively mono, every epimorphism is effectively epi, and (thus) any morphism which is both a mono- and an epimorphism is an isomorphism.
\end{defi}
\begin{rem*}
	\begin{alphanumerate}
		\item The three conditions on mono- and epimorphisms are not independent. The last condition, i.e.\ that every morphism which is both a mono- and an epimorphism is an isomorphism, follows from either of the former two.
		\item Since equalizers and coequalizers in an abelian category $\Aa$ can be constructed from kernels and cokernels (cf.\ Remark~\reff{rem:KernelCokernel}) and we already have finite products and coproducts from additivity of $\Aa$, we deduce that $\Aa$ has arbitrary finite limits and colimits. Indeed, we proved on exercise sheet \#7 from Algebraic Geometry I that finite limits can be constructed from equalizers and finite products, and, given coequalizers and finite coproducts instead, it's just the same for finite colimits.
		
		Conversely, the existence of finite limits and colimits guarantees that $\Aa$ has equalizers, coequalizers, finite products, and finite products, all of them being special cases of finite limits and colimits.
	\end{alphanumerate}
	
\end{rem*}
The category of modules (over a ring $R$) or sheaves of modules are abelian categories (as we are going to prove in a moment), but not Banach spaces or projective modules over most rings.
\begin{prop}\lbl{prop:RmodAbelian}
	The category $\Rr\cat{-Mod}$ of sheaves of modules (over a sheaf of rings $\Rr$ on some topological space $X$) is abelian.
\end{prop}
For clarity (and to better distinguish between the proof and Professor Franke's remarks about it), we will chop the proof into some lemmas.
\begin{lem}
	The category $\Rr\cat{-Mod}$ is additive.
\end{lem}
\begin{proof}
	First note that the zero sheaf $0$ is a common initial and final object. A direct sum of $\Mm,\Nn\in\Ob(\Rr\cat{-Mod})$ is given by
	\begin{align*}
	(\Mm\oplus\Nn)(U)=\left\{(m,n)\st m\in\Mm(U), n\in\Nn(U)\right\}\quad\text{for all }U\subseteq X\text{ open}
	\end{align*}
	(it's clear that this is a presheaf and it inherits the sheaf axiom from $\Aa$ and $\Nn$) with component-wise module operations and with $\Mm\lmorphism[p]\Mm\oplus\Nn\morphism[q]\Nn$ and $\Mm\morphism[i]\Mm\oplus\Nn\lmorphism[j]\Nn$ given by $p(m,n)=m$, $q(m,n)=n$, $i(m)=(m,0)$, and $j(n)=(0,n)$ on open subsets $U\subseteq X$ and $m\in\Mm(U)$, $n\in\Nn(U)$.
	
	If $\Mm\morphism[\mu]\Tt\lmorphism[\nu]\Nn$ are given, $\Mm\oplus\Nn\morphism[\mu\amalg\nu]\Tt$ sending $(m,n)\in(\Mm\oplus\Nn)(U)$ to $\mu(m)+\nu(n)$ verifies the universal property of the coproduct for $\Mm\oplus\Nn$. Similarly, $\Tt\morphism[\mu\times\nu]\Mm\oplus\Nn$ given by $(\mu\times\nu)(t)=(\mu(t),\nu(t))$ for $t\in\Tt(U)$ confirms the universal property of the product for $\Mm\oplus\Nn$. Also, $c=\id_{\Mm\oplus\Nn}$ is the unique endomorphism $c$ of that object such that $pci=\id_\Mm$, $qcj=\id_\Nn$, $pcj=0$, and $qci=0$. Thus, $\Rr\cat{-Mod}$ is additive (the group structure on $\Hom$ sets being easily verified).
\end{proof}
\begin{lem}\lbl{lem:RModHasKernels}
	The category $\Rr\cat{-Mod}$ has kernels.
\end{lem}
\begin{proof}
	Let $\Mm\morphism[f]\Nn$ be a morphism of sheaves of $\Rr$-modules and $\Kk$ be the sheaf given by 
	\begin{align*}
		\Kk(U)=\ker\Big(\Mm\morphism[f]\Nn\Big)(U)\coloneqq\ker\Big(\Mm(U)\morphism[f]\Nn(U)\Big)
	\end{align*}
	(you should convince yourself that this indeed satisfies the sheaf axiom). Then the inclusion $\Kk\morphism[\kappa]\Mm$ is a monomorphism as $\Kk(U)\monomorphism\Mm(U)$ is injective for every open subset $U\subseteq X$. 
	
	If $\Tt\morphism[\tau]\Mm$ is a morphism of $\Rr$-modules such that $f\tau=0$, then, for every $t\in\Tt(U)$, we have $f(\tau(t))=0$, hence $\snake{\tau}(t)\coloneqq\tau(t)\in\ker\Big(\Mm(U)\morphism[f]\Nn(U)\Big)=\Kk(U)$ and $\tau$ factors over 
	\begin{diagram*}
		\node[ob](R) at (0,1.25) {$\Tt$};
		\node[ob](A) at (2.5,1.25) {$\Mm$};
		\node[ob](RS) at (1.25,0) {$\Kk$};
		\scriptsize
		\draw[->] (R) -- (A) node[pos=0.5, above] {$\tau$};
		\draw[->, dashed] (R) -- (RS) node[pos=0.5, below left] {$\exists!\ \snake\tau$};
		\draw[right hook->] (RS) -- (A) node[pos=0.5, below right] {$\kappa$};
	\end{diagram*}
	This proves that $\Kk$ is indeed a kernel of $f$ in the category $\Rr\cat{-Mod}$.
\end{proof}
\begin{rem}
	\begin{alphanumerate}
		\item \lbl{rem:SheafMonomorphism}It is a consequence of the exactness of the $\colimit$ functor (for filtered systems of abelian groups; exactness of $\colimit$ does \emph{not} hold in general, not even for filtered colimits in abelian categories), that 
		\begin{align*}
		\Kk_x=\colimit[U\ni x]\ker\Big(\Mm(U)\morphism[f]\Nn(U)\Big)\simeq\ker\Big(\Mm_x\morphism[f]\Nn_x\Big)\;.
		\end{align*}
		This isomorphism can also be seen in a straightforward way.
		
		\item One may check that in any additive category (with kernels), a morphism $i$ is a monomorphism iff $\ker(i)=0$. Thus, in our example we have the equivalent conditions
		\begin{alphanumerate}
			\item[\itememph{\alpha}] $\Mm\morphism[f]\Nn$ is a monomorphism.
			\item[\itememph{\beta}] $\Mm(U)\morphism[f]\Nn(U)$ is injective for all open subsets $U\subseteq X$.
			\item[\itememph{\gamma}] $\ker(f)=0$ (the zero sheaf).
			\item[\itememph{\delta}] $\Mm_x\morphism[f]\Nn_x$ is injective for all $x\in X$.
		\end{alphanumerate}
	\end{alphanumerate}
	The construction of cokernels won't be that straightforward (duh!), related to the fact that epimorphisms in categories of sheaves aren't as simple as you might think. If $\Gg$ and $\Hh$ are sheaves on some topological space $X$ and $f$ is a morphism between them such that $\Gg(U)\morphism[f]\Hh(U)$ is surjective for all open $U$, then $f$ is an epimorphism, but there are epimorphisms $f$ for which this fails. 
	
	However, it follows from the fact that a sheaf $\Gg$ is canonically isomorphic to its sheafification $\Gg^\sh$ (cf.\ \cite[Proposition~1.2.1\itememph{d}]{alggeo1}) that a morphism between sheaves (of sets, groups, \ldots) is uniquely determined by the maps it induces on stalks. Thus, $\Gg\morphism\Hh$ is an epimorphism if $\Gg_x\morphism\Hh_x$ is an epimorphism in the respective target category for all $x\in X$.
\end{rem}
\begin{lem}\lbl{lem:RmodHasCokernels}
	The category $\Rr\cat{-Mod}$ has cokernels.
\end{lem}
\begin{proof}
	For a morphism $\Mm\morphism[f]\Nn$ of sheaves of $\Rr$-modules, the map 
	\begin{align*}
		U\mapsto \coker\Big(\Mm(U)\morphism[f]\Nn(U)\Big)=\Mm(U)/\Nn(U)\quad\text{for }U\subseteq X\text{ open}
	\end{align*}
	defines a presheaf $\Ff$ of $\Rr$-modules, but in general, $\Ff$ will fail to be a sheaf. We put $\Cc=\Ff^\sh$ (the \emph{sheafification} of $\Ff$, cf.\ \cite[Definition~1.2.3]{alggeo1}) and claim that $\Nn\morphism\Cc$ is a cokernel of $f$.
	
	Our first goal is to show that
	\begin{align}\lbl{eq:SheafCokernel}
		\Cc_x\simeq\coker\Big(\Mm_x\morphism[f]\Nn_x\Big)\;.\tag{$*$}
	\end{align}
	In the lecture, we did a direct proof, which was somewhat ugly and (in my opinion) lacking the essential step. From \cite[Proposition~1.2.1\itememph{a}]{alggeo1}, we get that $\Cc_x\simeq\Ff_x$ (which is basically what we proved in the lecture for this particular special case), so we need to show that
	\begin{align*}
		\Ff_x=\colimit[U\ni x]\coker\Big(\Mm(U)\morphism[f]\Nn(U)\Big)\simeq \coker\Big(\Mm_x\morphism[f]\Nn_x\Big)\;.
	\end{align*}
	Since $\Mm_x=\colimit[U\ni x]\Mm(U)$ and similar for $\Nn_x$, this amounts to showing that cokernels and certain colimits commute. But by Remark~\reff{rem:KernelCokernel}, cokernels are just a special case of colimits, so what we are actually going to show is that colimits commute with colimits -- in the following sense.
	\begin{lem}\lbl{lem:ColimitsCommute}
		Let $(X_{i,j})_{i\in I,j\in J}$ be objects of a category $\Aa$. For each $i_1,i_2\in I$ let there be an indexing set $I_{i_1,i_2}$ and for each $\alpha\in I_{i_1,i_2}$ and $j\in J$ a morphism
		\begin{align*}
			f_\alpha^j\colon X_{i_1,j}\morphism X_{i_2,j}\;.
		\end{align*}
		Similarly, for each $j_1,j_2\in J$ let there be an indexing set $J_{j_1,j_2}$ and for each $\beta\in J_{j_1,j_2}$ and $i\in I$ a morphism
		\begin{align*}
			g_\beta^i\colon X_{i,j_1}\morphism X_{i,j_2}\;.
		\end{align*}
		Moreover, suppose that for each $i_1,i_2\in I$ and $j_1,j_2\in J$ and $\alpha\in I_{i_1,i_2}$ and $\beta\in J_{j_1,j_2}$ the diagram
		\begin{diagram}[baseline=0cm-0.5ex][\lbl{diag:CokernelsCommute}]
			\node[ob] (a) at (0,0.75) {$X_{i_1,j_1}$};
			\node[ob] (b) at (0,-0.75) {$X_{i_1,j_2}$};
			\node[ob] (c) at (2.5,0.75) {$X_{i_2,j_1}$};
			\node[ob] (d) at (2.5,-0.75) {$X_{i_2,j_2}$};
			\scriptsize
			\draw[->] (a) -- (b) node[pos=0.5,left] {$g_\beta^{i_1}$};
			\draw[->] (a) -- (c) node[pos=0.5,above] {$f_\alpha^{j_1}$};
			\draw[->] (c) -- (d) node[pos=0.5,right] {$g_\beta^{i_2}$};
			\draw[->] (b) -- (d) node[pos=0.5,above] {$f_\alpha^{j_2}$};
			\tag{\#}
		\end{diagram}
		commutes. Then there is an isomorphism
		\begin{align*}
			\colimit[i\in I]\colimit[j\in J]X_{i,j}\simeq\colimit[j\in J]\colimit[i\in I]X_{i,j}\simeq\colimit[(i,j)\in I\times J]X_{i,j}\;.
		\end{align*}
	\end{lem}
	\begin{proof}
		Clearly, it is enough to show the rightmost isomorphism. What we need to show is that $L\coloneqq\colimit[j]\colimit[i]X_{i,j}$ satisfies the universal property of $L'\coloneqq\colimit[(i,j)]X_{i,j}$.
		
		 Let $T$ be an object of $\Aa$ and $\big(X_{i,j}\morphism[\tau_{i,j}]T\big)_{i\in I,j\in J}$ be a cocone below the diagram $(X_{i,j})_{i,j}$. That is, for every $\alpha\in I_{i_1,i_2}$ and $j\in J$ the diagram 
		\begin{diagram*}
			\node[ob](R) at (0,1.25) {$X_{i_1,j}$};
			\node[ob](A) at (2.5,1.25) {$X_{i_2,j}$};
			\node[ob](RS) at (1.25,0) {$T$};
			\scriptsize
			\draw[->] (R) -- (A) node[pos=0.5, above] {$f_\alpha^j$};
			\draw[->] (R) -- (RS) node[pos=0.5, below left] {$\tau_{i_1,j}$};
			\draw[<-] (RS) -- (A) node[pos=0.5, below right] {$\tau_{i_2,j}$};
		\end{diagram*}
		commutes. By the universal properties of the $L_j\coloneqq\smash{\colimit[i]X_{i,j}}$, the $\tau_{i,j}$ factor over some maps $L_j\morphism[\tau_j]T$. Moreover, for each $j_1,j_2\in J$ and $\beta\in J_{j_1,j_2}$, the compositions 
		\begin{align*}
			X_{i,j_1}\morphism[g_\beta^i]X_{i,j_2}\morphism L_{j_2}
		\end{align*}
		induce a map $L_{j_1}\morphism[g_\beta]L_{j_2}$ by the universal property of $L_{j_1}$ (here, we silently used the commutativity of \eqreff{diag:CokernelsCommute}, otherwise the above compositions wouldn't be a cocone below $(X_{i,j_1})_{i\in I}$). We thus get a diagram
		\begin{diagram*}
			\node[ob](x1) at (0,3) {$X_{i,j_1}$};
			\node[ob](x2) at (5,3) {$X_{i,j_2}$};
			\node[ob](l1) at (1.25,1.5) {$L_{j_1}$};
			\node[ob](l2) at (3.75,1.5) {$L_{j_2}$};
			\node[ob](t) at (2.5,0) {$T$};
			\scriptsize
			\draw[->] (x1) -- (x2) node[pos=0.5, above] {$g_\beta^i$};
			\draw[dashed,->] (l1) -- (l2) node[pos=0.5, above] {$g_\beta$};
			\draw[->] (x1) -- (l1);
			\draw[->] (x2) -- (l2);
			\draw[->, bend right=45] (x1) to node[pos=0.5, below left] {$\tau_{i,j_1}$} (t);
			\draw[->, bend left=45] (x2) to node[pos=0.5, below right] {$\tau_{i,j_2}$} (t);
			\draw[->] (l1) -- (t) node[pos=0.5, below left] {$\tau_{j_1}$};
			\draw[->] (l2) -- (t) node[pos=0.5, below right] {$\tau_{j_2}$};
		\end{diagram*}
	in which everything but the bottom-middle triangle commutes. We show that this triangle commutes as well. Indeed, by the universal property of $L_{j_1}$, $\tau_{j_1}$ is the unique morphism $L_{j_1}\morphism T$ making each
	\begin{diagram*}
		\node[ob](R) at (0,1.25) {$X_{i,j_1}$};
		\node[ob](A) at (2.5,1.25) {$L_{j_1}$};
		\node[ob](RS) at (1.25,0) {$T$};
		\scriptsize
		\draw[->] (R) -- (A);
		\draw[->] (R) -- (RS) node[pos=0.5, below left] {$\tau_{i,j_1}$};
		\draw[<-] (RS) -- (A);
	\end{diagram*}
	commute. But apparently, $\tau_{j_2}g_\beta$ has this property as well, proving $\tau_{j_1}=\tau_{j_2}g_\beta$. Then the morphisms $\big(L_{j}\morphism[\tau_j]T\big)_{j\in J}$ form a cocone below the diagram $(L_j)_{j\in J}$, hence factor uniquely over some $L\morphism[\tau]T$ by the universal property of $L$.	
	
	It remains to prove uniqueness of $\tau$. If $L\morphism[\tau]T$ is a morphism over which each $X_{i,j}\morphism[\tau_{i,j}]T$ factors, then the composition $L_j\morphism L\morphism[\tau]T$ must equal $\tau_j$ since $\tau_j$ is uniquely determined by the universal property of $L_j$. But $\tau$ is uniquely determined by the $\tau_j$, proving uniqueness.
\end{proof}
	Having thus proved \eqreff{eq:SheafCokernel}, we now proceed with the proof of Lemma~\reff{lem:RmodHasCokernels}. We have a morphism $\Nn\morphism\Cc$ sending $n\in \Nn(U)$ to
	\begin{align*}
	\bigg(\text{image of }n\text{ under }\Nn(U)\morphism\Nn_x\morphism\coker\Big(\Mm_x\morphism[f]\Nn_x\Big)\bigg)_{x\in U}\;.
	\end{align*}
	Since $\Cc_x\simeq\coker\Big(\Mm_x\morphism[f]\Nn_x\Big)$, this morphism $\Nn\morphism\Cc$ induces surjections on stalks, hence is an epimorphism of sheaves. We show that the morphism $\Nn\morphism\Cc$ satisfies the universal property of the cokernel.
	
	Let $\Nn\morphism[\tau]\Tt$ be a morphism of sheaves of $\Rr$-modules such that $\tau f=0$. Let $U\subseteq X$ be open. For 
	\begin{align*}
		\nu=(\nu_x)_{x\in U}\in\Cc(U)\subseteq\prod_{x\in U}\coker\Big(\Mm_x\morphism[f]\Nn_x\Big)
	\end{align*}
	we define $\tau_1(\nu)\in\prod_{x\in U}\Tt_x$ by selecting $n\in\Nn_x$ whose image in $\coker\Big(\Mm_x\morphism[f]\Nn_x\Big)$ equals $\nu_x$, then put $\tau_1(\nu)_x=\tau(n)_x$ which is independent of the choice of $n$ as $\tau f=0$. It follows from the coherence condition for $\Cc$ that $\tau_1(\nu)$ satisfies the coherence condition for $\Tt^\sh$, i.e.\ $\tau_1(\nu)\in\Tt^\sh(U)\subseteq\prod_{x\in U}\Tt_x$. Hence there is $\Cc\morphism[\tau_2]\Tt$ such that $\tau_1=\Big(\Tt\isomorphism\Tt^\sh\Big)\circ\tau_2$ and $\tau_2$ makes
	\begin{diagram*}
		\node[ob](R) at (0,1.25) {$\Nn$};
		\node[ob](A) at (2.5,1.25) {$\Tt$};
		\node[ob](RS) at (1.25,0) {$\Cc$};
		\scriptsize
		\draw[->] (R) -- (A) node[pos=0.5, above] {$\tau$};
		\draw[->, dashed] (R) -- (RS) node[pos=0.5, below left] {$\exists!\ \tau_2$};
		\draw[->] (RS) -- (A);
	\end{diagram*}
	commutative. Uniqueness of $\tau_2$ is easy to see stalk-wise. It follows that $\Nn\morphism\Cc$ is ineed a cokernel of $f$.
\end{proof}
\begin{rem}\lbl{rem:SheafEpimorphism}
	One may check that in any additive category (with cokernels) a morphism $f$ is an epimorphism if $\coker(f)=0$. By our previous construction of cokernels and the description of stalks, we have equivalent conditions
	\begin{alphanumerate}
		\item $\Mm\morphism[f]\Nn$ is an epimorphism of sheaves of $\Rr$-modules.
		\item $\Mm_x\morphism[f]\Nn_x$ is surjective for all $x\in X$.
		\item For every open $U\subseteq X$ and $n\in\Nn(U)$ there are an open covering $U=\bigcup_{\lambda\in\Lambda}U_\lambda$ and $m_\lambda\in\Mm(U_\lambda)$ such that $n|_{U_\lambda}=f(m_\lambda)$
	\end{alphanumerate}
	\ldots but \itememph{c} does \emph{not} imply the surjectivity of $\Mm(U)\morphism[f]\Nn(U)$, unless, e.g., $f$ is also a monomorphism.
\end{rem}
\begin{proof}[Proof of Proposition~\reff{prop:RmodAbelian}]
	We verify the rest of the abelianness conditions. First, let $\Mm\morphism[f]\Nn$ be a mono- and epimorphism. Then it induces isomorphisms on stalks (by Remark~\reff{rem:SheafMonomorphism}\itememph{b} and Remark~\reff{rem:SheafEpimorphism}), hence is an isomorphism itself.
	
	Let $\Mm\morphism[i]\Nn$ be a monomorphism and $\Nn\morphism\Cc$ be its cokernel. Then 
	\begin{align*}
		\ker\left(\Nn\morphism\Cc\right)_x=\ker\left(\Nn_x\morphism\Cc_x\right)=\ker\left(\Nn_x\morphism\coker\Big(\Mm_x\morphism[i]\Nn_x\Big)\right)\simeq\Mm_x
	\end{align*}
	as $\Mm_x\morphism[i]\Nn_x$ is injective. Hence $\Mm\morphism\ker\left(\Nn\morphism\Cc\right)$ induces isomorphisms on stalks and thus is an isomorphism itself. It follows by Definition~\reff{def:EffectiveMono}\itememph{d} that any monomorphism is an effective monomorphism.
	
	Similar arguments apply to epimorphisms.	
\end{proof}
Recall the definition of an adjoint pair of functors.
\begin{defi}[{\cite[Definition~A.2.3]{alggeo1}}]\lbl{def:AdjointFunctors}
	Let $\Aa,\Bb$ be categories. A pair $\Aa\doublelrmorphism[L][R]\Bb$ of (covariant) functors is called \defemph{adjoint}, if there is a canonical bijection
	\begin{align*}
		\Hom_\Aa(X,RY)\isomorphism\Hom_\Bb(LX,Y)
	\end{align*}
	which is functorial in both $X\in\Ob(\Aa)$ and $Y\in\Ob(\Bb)$.
\end{defi}
\begin{rem}\lbl{rem:AdjunctionPreservesStuff}
	It can be easily seen that $L$ preserves colimits (in particular, coproducts, and in particular again, initial objects) and $R$ preserves limits (in particular, products, and in particular again, final objects). When $\Aa$ and $\Bb$ are additive, it follows that both $L$ and $R$ map $0$ to $0$ and are compatible with finite direct sums. Moreover, $L$ preserves cokernels and $R$ preserves kernels since these are special cases of colimits and limits, respectively (in particular, I have no idea what the purpose of Franke's extra calculation was).%We don't need this.%Because of
	%\begin{align*}
	%	\Hom_\Aa\left(T,R\Big[\ker\Big(X\morphism[f]Y\Big)\Big]\right)&\simeq \Hom_\Bb\left(LT,\ker\Big(X\morphism[f]Y\Big)\right)\\
	%	&\simeq\ker\left(\Hom_\Bb(LT,X)\morphism[f\circ]\Hom_\Bb(LT,Y)\right)\\
	%	&\simeq\ker\left(\Hom_\Aa(T,RX)\morphism[Rf\circ]\Hom_\Aa(T,RY)\right)\\
	%	&\simeq\Hom_\Aa\left(T,\ker\Big(RX\morphism[Rf]RY\Big)\right)\;,
	%\end{align*}
\end{rem}

\printbibliography

\end{document}          

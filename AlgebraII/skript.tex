\documentclass[a4paper,parskip=full,numbers=enddot]{scrreprt}
\usepackage[utf8]{inputenc}

\usepackage{../header}
\usepackage{../frankenumbering}
\usepackage{../shortcuts}

\usepackage{csquotes}
\usepackage[backend=biber,style=numeric,sorting=none]{biblatex}
\addbibresource{../literatur.bib}
% Title Page
\title{Algebra II}
\author{Nicholas Schwab \& Ferdinand Wagner}
\date{Wintersemester 2017/18}

\widowpenalty=10000
\clubpenalty=10000

\begin{document}
\pagenumbering{Alph}
\maketitle
\pagenumbering{roman}
 
This text consists of notes of the lecture Algebra II taught at the University of Bonn by Professor Jens Franke in the winter term (Wintersemester) 2017/18. 

Please report bugs, typos etc. through the \emph{Issues} feature of github.

\tableofcontents

\chapter*{Introduction}
\addcontentsline{toc}{chapter}{Introduction}
\pagenumbering{arabic}
After a slight delay due to the Professor being confused by the large attendance to his lecture, Franke briefly recaps the contents of his lecture course Algebra I. Our notes to this lecture can be found \href{https://github.com/Nicholas42/AlgebraFranke/tree/master/AlgebraI}{here}\cite{alg1}. He mentions specifically
\begin{itemize}
 \item Hilbert's Basissatz and Nullstellensatz,
 \item the Noether Normalization Theorem,
 \item the Zariski-topology on $k^n$,
 \item irreducible topological spaces and their correspondence to the prime ideals of $k[X_1, \ldots, X_n]$,
 \item Noetherian topological spaces and their unique decomposition into irreducible subsets,
 \item the dimension of topological spaces and codimension of their irreducible subsets,
 \item catenary topological spaces,
 \item the fact that $k^n$ is catenary and $\dim(k^n) = n$,
 \item quasi-affine varieties,
 \item structure sheaves,
 \item the fact that quasi-affine varieties $X$ are catenary and $\dim(X) = \deg\tr(K(X)/k)$, where $K(X)$ is the quotient field of $\Oo(X)$. By the way, there is a nice alternative characterization as a direct limit (or colimit)
 \begin{align*}
 	K(X)=\colimit[\substack{\emptyset\not=U\subseteq X\\U\text{ open}}]\Oo(U)\;.
 \end{align*}
 \item going up and going down for integral ring extensions,
 \item localizations.
\end{itemize}
The following definition won't appear in the lecture, but in Algebraic Geometry I instead.
\begin{defi} \lbl{def:scheme}
    A \textbf{(pre)scheme} is a \emph{locally ringed space} $X$ with a structure sheaf $\Oo_X$ such that each $x\in X$ has an open neighbourhood $U$ which is \emph{isomorphic} to $\Spec R$ for some ring $R$. That is, $U\cong \Spec R$ as topological spaces and $\Oo_X|_U$ restricts to $\Oo_{\Spec R}$.
\end{defi}

Exercises will be held on Wednesday from 16 to 18 and Friday from 12 to 14 in Room 0.008. It is necessary to have achieved at least half the points on the exercise sheets in order to attend the exams.

Professor Franke recapitulated on some topics of his previous lecture, Algebra I.
\begin{defi}
 A topological spcae $X$ is called quasi-compact if every open covering $X = \bigcup_{\lambda\in\Lambda} U_\lambda$ admits a finite subcovering.
 
 X is Noetherian if it satisifies the following equivalent conditions:
 \begin{alphanumerate}
    \item Every open subset is quasi-compact.
    \item There is no infinite properly descending chain of closed subsets.
    \item Every set of closed subsets of $X$ has a $\subseteq$-minimal element.
 \end{alphanumerate}

\end{defi}

\begin{defi}[Irreducible Space]
 A topological space $X$ is \defemph{irreducible} if it satisifies the following equivalent conditions:
 \begin{alphanumerate}
    \item 
        If $X= \bigcup_{i=1}^n X_i$ is a finite covering by closed subsets, there is $i$ such that $1\leq i\leq n$ and $X=X_i$.
    \item
        If $X = X_1\cup X_2$ where $X_1$ and $X_2$ are closed subsets, then $X=X_1$ or $X=X_2$. Also, $X\neq\emptyset$.
 \end{alphanumerate}

\end{defi}
\begin{prop}
 \begin{alphanumerate}
    \item 
        Any open or closed subset of a Noetherian topological space is Noetherian.
    \item 
        If $X$ is Noetherian, there is a unique (up to permutation of the $X_i$) decomposition $X = \bigcup_{i=1}^n X_i$ where the $X_i\subseteq X$ are irreducible and closed and $X_i\not\subseteq X_J$ for $i\neq j$.
 \end{alphanumerate}

\end{prop}
\begin{defi}
 Let $X$ be a topological space, $Z\subseteq X$ irreducible. We put 
 \begin{align*}
    \codim(Z,X) &= \sup\left\{\ell\st \text{There are irreducible } Z_i\subseteq X \text{ such that } Z=Z_0 \subsetneq Z_1\subsetneq \subsetneq \ldots \subsetneq Z_\ell\right\}\\
    \dim(X) &= \sup\left\{\codim(Z,X)\st Z\subseteq X \text{ irreducible}\right\}
 \end{align*}

\end{defi}
\begin{example}
  Let $k =\overline{k}$. For any ideal $I\subseteq R = k[X_1,\ldots,X_n]$ let $V(I) = \left\{x\in k^n\st f(x) \forall f\in I\right\}$ be the set of zeroes of $I$. By the Hilbert Nullstellensatz, $V(I) \neq \emptyset$ when $I\subsetneq R$. Moreover 
  \begin{align*}
    V(I)& = V(\sqrt{I})\\
    V(I\cdot J ) &= V(I) \cup V(J)\\
    V\left(\sum_{\lambda\in\Lambda} I_\lambda\right) &\bigcap_{\lambda\in\Lambda} V(I_\lambda).
 \end{align*}
 It follows that there is a topology (called the \emph{Zariski topology}) on $k^n$ containing precisely the subsets of the form $V(I)$ as closed subsets. By a version of the Nullstellensatz follows
 \begin{align*}
  \left\{f\in R\st f(x) = 0\forall f\in I\right\} = \left\{f\in R\st V(f)\supseteq V(I)\right\} = \sqrt{I}.
 \end{align*}
 This means that there is strictly antimonotonic bijective correspondence between the ideals $I$ of $R$ with $I\sqrt{I}$ and the Zariski-closed subsets $A\subseteq k^n$ by
 \begin{align*}
    \left\{f\in R\st V(f)\supseteq A\right\} &\longot A\\
    I&\longto V(I)
 \end{align*}
 A $R$ is Noetherian, any strictly ascending chain of ideals in $R$ terminates, implying that $k^n$ is a Noetherian topological space. Under the above correspondence the prime ideals correspond to the irreducible subsets.  
\end{example}
\begin{rem}
 In general for $A\subseteq B\subseteq C\subseteq X$
 \begin{align*}\lbl{eq:codimIneq}
    \codim(A,B) +\codim(B,C) &\leq \codim(A,C) \tag{@1}\\ 
    \codim(A,X)+\dim A &\leq \dim X.\lbl{eq:codimIneq2}\tag{@2}
 \end{align*}
 may be strict. A Noetherian topological space is called \emph{catenary} if \eqreff{eq:codimIneq} is an equality whenever $A$, $B$ and $C$ are irreducible.

\end{rem}
\begin{thm}
 The space $X=k^n$ is catenary and equality will always occur in \eqreff{eq:codimIneq2} for this $X$.
\end{thm}
\begin{example}
 For $k^1$, the closed subsets are $k$ and the finite subsets. Since $k$ is infinite, the points and $k$ are the irreducible subsets, implying $\dim(k) = 1 $ and the other assertions for $n=1$.
\end{example}
\begin{example}
 The irreducible subsets are $k^2$, points and $V(f)$ where $f\in k[X,Y]$ is a prime element.
\end{example}








\chapter{Krull's Principal Ideal Theorm}\lbl{ch:krullPrincipalIdealThm}
\section{Formulation}
\begin{thm}\lbl{thm:krullPrincipalIdeal}
    %Should be Theorem 11, because we continue from Algebra I, Ferdi, that's your job
    Let $R$ be Noetherian, $f\in R$, $\pp\in \Spec R$ minimal among all prime ideals containing $f$, then $\hoehe(\pp) \leq 1$. In other words, $\pp$ is a minimal prime ideal ($\hoehe(\pp) =0$) or all prime ideals strictly contained in $\pp$ are minimal.
\end{thm}
\begin{rem*}
    \begin{alphanumerate}
    \item
        By definition
        \begin{align*}
            \hoehe(\pp) = \sup\left\{\ell\in\IN\st \exists \pp_1,\ldots,\pp_\ell \in \Spec R: \pp \supsetneq\pp_1\supsetneq\ldots\supsetneq \pp_\ell\right\}.
        \end{align*}
    \item
        Recall the Zariski topology on $\Spec R$: 
        
        For any ideal $I\subseteq R$ let $V(I) = \left\{\pp\in\Spec R\st I\subseteq \pp\right\}$. We have 
        \begin{align*}
            V(I)& = V(\sqrt{I})\\
            V(I\cdot J ) &= V(I) \cup V(J)\\
            V\left(\sum_{\lambda\in\Lambda} I_\lambda\right) &= \bigcap_{\lambda\in\Lambda} V(I_\lambda).
        \end{align*}
        This implies (together with $V(O) = \Spec R$ and $V(R) = \emptyset$) that $\Spec R$ can be equipped with a topology in which the closed subsets are precisely the subsets of them form $V(I)$ where $I$ is some ideal in $R$. This topology is Noetherian when $R$ is, hence any closed subset can be decomposed into irreducible components. For $V(f) = V(f\cdot R)$, there are $V(\pp)$ where $\pp$ is minimal among all prime ideals containing $f$. The Theorem \reff{thm:krullPrincipalIdeal} thus states that all irreducible components of $V(f)$ have codimension smaller or equal to 1 in $\Spec R$.
    \end{alphanumerate}
\end{rem*}

\printbibliography
\end{document}          

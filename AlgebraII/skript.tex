\documentclass[a4paper,parskip=full,numbers=enddot]{scrreprt}
\usepackage[utf8]{inputenc}

\usepackage{../header}
\usepackage{../frankenumbering}
\usepackage{../shortcuts}

\usepackage{csquotes}
\usepackage[backend=biber,style=numeric,sorting=none]{biblatex}
\addbibresource{../literatur.bib}
% Title Page
\title{Algebra II}
\author{Nicholas Schwab \& Ferdinand Wagner}
\date{Wintersemester 2017/18}

\widowpenalty=10000
\clubpenalty=10000

\begin{document}
\pagenumbering{Alph}
\maketitle
\pagenumbering{roman}
 
This text consists of notes of the lecture Algebra II taught at the University of Bonn by Professor Jens Franke in the winter term (Wintersemester) 2017/18. 

Please report bugs, typos etc. through the \emph{Issues} feature of github.

\tableofcontents

\chapter*{Introduction}
\addcontentsline{toc}{chapter}{Introduction}
\pagenumbering{arabic}
After a slight delay due to the Professor being confused by the large attendance to his lecture, Franke briefly recaps the contents of his lecture course Algebra I. Our notes to this lecture can be found \href{https://github.com/Nicholas42/AlgebraFranke/tree/master/AlgebraI}{here}\cite{alg1}. He mentions specifically
\begin{itemize}
 \item Hilbert's Basissatz and Nullstellensatz,
 \item the Noether Normalization Theorem,
 \item the Zariski-topology on $k^n$,
 \item irreducible topological spaces and their correspondence to the prime ideals of $k[X_1, \ldots, X_n]$,
 \item Noetherian topological spaces and their unique decomposition into irreducible subsets,
 \item the dimension of topological spaces and codimension of their irreducible subsets,
 \item catenary topological spaces,
 \item the fact that $k^n$ is catenary and $\dim(k^n) = n$,
 \item quasi-affine varieties,
 \item structure sheaves,
 \item the fact that quasi-affine varieties $X$ are catenary and $\dim(X) = \deg\tr(K(X)/k)$, where $K(X)$ is the quotient field of $\Oo(X)$. By the way, there is a nice alternative characterization as a direct limit (or colimit)
 \begin{align*}
 	K(X)=\colimit[\substack{\emptyset\not=U\subseteq X\\U\text{ open}}]\Oo(U)\;.
 \end{align*}
 \item going up and going down for integral ring extensions,
 \item localizations.
\end{itemize}
Exercises will be held on Wednesday from 16 to 18 and Friday from 12 to 14 in Room 0.008. It is necessary to have achieved at least half the points on the exercise sheets in order to attend the exams.

\section*{Introduction to Krull dimension and all that}
\addcontentsline{toc}{section}{Introduction to Krull dimension and all that}
Professor Franke recapitulated on some topics of his previous lecture, Algebra I (of which detailed lecture notes may be found in \cite{alg1}). Note that although the numbering of theorems in the following might seem messy, it is \emph{intentionally} messy at least.
\begin{defi}[{\cite[Definition~2.1.2]{alg1}}]
 A topological space $X$ is called \defemph{quasi-compact} if every open cover $X = \bigcup_{\lambda\in\Lambda} U_\lambda$ admits a finite subcover.
 
 $X$ is \defemph{Noetherian} if it satisifies the following equivalent conditions:
 \begin{alphanumerate}
    \item Every open subset is quasi-compact.
    \item There is no infinite properly descending chain of closed subsets.
    \item Every set of closed subsets of $X$ has a $\subseteq$-minimal element.
 \end{alphanumerate}
\end{defi}

\begin{defi}[{\cite[Definition~2.1.3]{alg1}}]
 A topological space $X\not=\emptyset$ is \defemph{irreducible} if it satisifies the following equivalent conditions:
 \begin{alphanumerate}
    \item If $X = X_1\cup X_2$ where $X_1$ and $X_2$ are closed subsets, then $X=X_1$ or $X=X_2$. Also, $X\neq\emptyset$.
    \item Any two non-empty open subsets of $X$ have non-empty intersection.
    \item Every non-empty open subset of $X$ is dense.
 \end{alphanumerate}
\end{defi}
\begin{rem*}
	Condition (a) implies, by induction, the following more general property: 
	For any finite cover $X= \bigcup_{i=1}^n X_i$ by closed subsets, there is $1\leq i\leq n$ such that $X=X_i$.
\end{rem*}
\begin{prop}
 \begin{alphanumerate}
    \item 
        Any subset of a Noetherian topological space is Noetherian with it's induced subspace topology (cf. \cite[Remark~2.2.1]{alg1}).
    \item 
        If $X$ is Noetherian, there is a unique (that is, up to permutation of the $X_i$) decomposition $X = \bigcup_{i=1}^n X_i$ into irreducible closed subsets $X_i\subseteq X$ such that $X_i\not\subseteq X_j$ for $i\neq j$ (cf. \cite[Proposition~2.1.1]{alg1}).
 \end{alphanumerate}

\end{prop}
\begin{defi}[{\cite[Definition~2.1.4]{alg1}}]
 Let $X$ be a topological space, $Z\subseteq X$ irreducible. We put 
 \begin{align*}
    \codim(Z,X) &= \sup\left\{\ell\st 
    \begin{array}{c}
	    \text{there is a strictly descending chain }\\
	    Z_0 \subsetneq Z_1\subsetneq \ldots \subsetneq Z_\ell\subseteq X\text{ of irreducible }Z_i\subseteq X
    \end{array}\right\}\\
    \dim(X) &= \sup\left\{\codim(Z,X)\st Z\subseteq X \text{ irreducible}\right\}
 \end{align*}
\end{defi}
\begin{example}[{\cite[Section~1.7 and 2.1]{alg1}}]\lbl{ex:vanishingSetOfIdeal}
  Let $k =\overline{k}$ be an algebraically closed field. For an ideal $I\subseteq R = k[X_1,\ldots,X_n]$ let 
  \begin{align*}
  	V(I) = \left\{x\in k^n\st f(x)=0\ \forall f\in I\right\}
  \end{align*}
  be the set of zeroes of $I$. By the Hilbert Nullstellensatz, $V(I) \neq \emptyset$ when $I\subsetneq R$. Moreover 
  \begin{align*}
    V(I)& = V\left(\sqrt{I}\right)\\
    V(I\cdot J ) &= V(I) \cup V(J)\\
    V\bigg(\sum_{\lambda\in\Lambda} I_\lambda\bigg) &=\bigcap_{\lambda\in\Lambda} V(I_\lambda)\;.
 \end{align*}
 It follows that there is a topology (called the \emph{Zariski topology}) on $k^n$ containing precisely the subsets of the form $V(I)$ as closed subsets. A version of the Nullstellensatz (\cite[Proposition~1.7.1]{alg1}) says
 \begin{align*}
  \left\{f\in R\st f(x) = 0\forall f\in I\right\} = \left\{f\in R\st V(f)\supseteq V(I)\right\} = \sqrt{I}\;.
 \end{align*}
 This means that there is strictly antimonotonic bijective correspondence between the ideals $I$ of $R$ with $I=\sqrt{I}$ and the Zariski-closed subsets $A\subseteq k^n$ via
 \begin{align*}
	 \left\{\text{ideals }I\subseteq R\text{ such that }I=\sqrt I\right\}&\isomorphism\left\{\text{Zariski-closed subsets }A\subseteq k^n\right\}\\
    \left\{f\in R\st V(f)\supseteq A\right\} &\longmapsfrom A\\
    I&\longmapsto V(I)\;.
 \end{align*}
 (cf. \cite[Remark~2.1.1]{alg1}). As $R$ is Noetherian, any strictly ascending chain of ideals in $R$ terminates, implying that $k^n$ is a Noetherian topological space. Under the above correspondence prime ideals correspond to irreducible subsets and vice versa (cf. \cite[Proposition~2.1.2]{alg1}).  
\end{example}
\begin{rem}[{\cite[Remark~2.1.3]{alg1}}]
 In general for $A\subseteq B\subseteq C\subseteq X$
 \begin{align}\lbl{eq:codimIneq}
    \codim(A,B) +\codim(B,C) &\leq \codim(A,C) \\ 
    \codim(A,X)+\dim A &\leq \dim X.\lbl{eq:codimIneq2}
 \end{align}
 may be strict. A Noetherian topological space is called \emph{catenary} if \eqreff{eq:codimIneq} is an equality whenever $A$, $B$ and $C$ are irreducible.

\end{rem}
\setcounter{thm}{4}
\begin{thm}[{\cite[Theorem~5]{alg1}}]
 The space $X=k^n$ is catenary and in this case equality always occurs in \eqreff{eq:codimIneq2}.
\end{thm}
\begin{example}
 For $n=1$, the closed subsets of $k$ are $k$ itself and the finite subsets. Since $k$ is infinite, the points and $k$ are the irreducible subsets, implying $\dim(k) = 1 $ and the other assertions for $n=1$.
\end{example}
\begin{example}
 The irreducible subsets of $k^2$ are $k^2$ itself, single points, and $V(f)$ where $f\in k[X,Y]$ is a prime element.
\end{example}








\chapter{Krull's Principal Ideal Theorm}\lbl{ch:krullPrincipalIdealThm}
\section{Formulation}
\setcounter{thm}{10}
\begin{thm}\lbl{thm:krullPrincipalIdeal}
    %Should be Theorem 11, because we continue from Algebra I, Ferdi, that's your job
    Let $R$ be Noetherian, $f\in R$, $\pp\in \Spec R$ minimal among all prime ideals containing $f$. Then $\hoehe(\pp) \leq 1$. In other words, $\pp$ is a minimal prime ideal (if $\hoehe(\pp) =0$) or all prime ideals strictly contained in $\pp$ are minimal.
\end{thm}
\begin{rem*}
    \begin{alphanumerate}
    \item
        The \emph{height} of a prime ideal is defined as
        \begin{align*}
            \hoehe(\pp) = \sup\left\{\ell\st
            \begin{array}{c}
	            \text{there is a strictly descending chain}\\
	            \pp=\pp_0\supsetneq\pp_1\supsetneq\ldots\supsetneq\pp_\ell\text{ of prime ideals }\pp_i\in\Spec R
            \end{array}\right\}\;.
        \end{align*}
    \item
        Recall the \emph{Zariski topology} on $\Spec R$: For any ideal $I\subseteq R$, let 
        \begin{align*}
        	V(I) = \left\{\pp\in\Spec R\st I\subseteq \pp\right\}\;.
        \end{align*}
        We have
        \begin{align*}
            V(I)& = V\left(\sqrt{I}\right)\\
            V(I\cdot J ) &= V(I) \cup V(J)\\
            V\bigg(\sum_{\lambda\in\Lambda} I_\lambda\bigg) &= \bigcap_{\lambda\in\Lambda} V(I_\lambda).
        \end{align*}
        This implies (together with $V(0) = \Spec R$ and $V(R) = \emptyset$) that $\Spec R$ can be equipped with a topology in which the closed subsets are precisely the subsets of them for $V(I)$ where $I$ is some ideal in $R$. This topology is Noetherian when $R$ is, hence any closed subset can be decomposed into irreducible components. For $V(f) = V(f\cdot R)$, they are precisely those $V(\pp)$ for which $\pp$ is minimal among all prime ideals containing $f$. Theorem~\reff{thm:krullPrincipalIdeal} thus states that all irreducible components of $V(f)$ have codimension smaller or equal to 1 in $\Spec R$.
    \end{alphanumerate}
\end{rem*}

\printbibliography
\end{document}          

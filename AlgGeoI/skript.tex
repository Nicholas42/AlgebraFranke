\documentclass[a4paper,parskip=full,numbers=enddot]{scrreprt}
\usepackage[utf8]{inputenc}

\usepackage{../header}
\usepackage{../frankenumbering}
\usepackage{../shortcuts}

% Title Page
\title{Algebraic Geometry I}
\author{Nicholas Schwab \& Ferdinand Wagner}
\date{Wintersemester 2017/18}

\widowpenalty=10000
\clubpenalty=10000

\begin{document}
\pagenumbering{Alph}
\maketitle
\pagenumbering{roman}
 
This text consists of notes of the lecture Algebraic Geometry taught at the University of Bonn by Professor Jens Franke in the winter term (Wintersemester) 2017/18. 

Please report bugs, typos etc. through the \emph{Issues} feature of github.

\tableofcontents

\chapter*{Introduction}
\addcontentsline{toc}{chapter}{Introduction}
\pagenumbering{arabic}
The lecture will mainly be about the concept of \emph{schemes}. The topics include but are not limited to the category of (pre-)schemes, properties of schemes, morphisms of schemes, sheaves of $\Oo_X$-modules and cohomology of quasi-coherent sheaves.

Professor Franke said the lecture requires a firm knowledge of commutative algebra and affine and projective varieties. If you are not familiar with this terms you may want to think again about visiting this lecture. If you want to brush up your knowledge about these topics the following literature is recommended
\begin{itemize}
 \item Matsumura, H.: Commutative Ring Theory,
 \item Hartshorne, R.: Algebraic Geometry,
 \item Mumford, D.: The Red Book of Varieties and Schemes,
 \item Schwab, N. \& Wagner, F.: \href{https://github.com/Nicholas42/AlgebraFranke/tree/master/AlgebraI}{Algebra I by Jens Franke}\footnote{\url{https://github.com/Nicholas42/AlgebraFranke/tree/master/AlgebraI}}.
\end{itemize}
Let it be said that the first three recommendations are from Professor Franke while the last one is from the (not so) humble authors of these notes.

\chapter{Varieties and Schemes}\lbl{ch:varietiesAndSchemes}
\begin{defi}[Sheaf and Presheaf]\lbl{def:preSheaf}
    A \emph{presheaf} $\Ff_X$ (the $X$ is often omitted, if it is clear from the context) of rings on a topological space $X$ associates to any open subset $U\subseteq X$ a ring $\Ff_X(U)$ called the ring of sections of $\Ff_X$ on $U$ and to any inclusion of open subsets $V\subseteq U$ a ring homomorphism 
    \begin{align*}
        |_V: \Ff_X(U) &\longto \Ff_X(V)
    \end{align*}
    such that $f|_V = f$ for all $f\in\Ff_X(V)$ and $(f|_V)|_W = f|_W$ for any inclusion $W\subseteq V\subseteq U$ of open subsets. Note that while this notation reminds of the restriction of functions, behaves similarly and often the restriction is indeed used for this homomorphism, the elements of the rings are not always functions. 
    
    A so defined presheaf is furthermore are \emph{sheaf} if additionally, the following condition called \emph{sheaf axiom} holds:
    \begin{quote}
     For every open covering $U = \bigcup_{\lambda\in \Lambda} U_i$ of any open subset $U\subseteq X$ the function
        \begin{align*}\lbl{eq:sheafAxiom}
			\Ff_X(U) &\longto \left\{(f_\lambda)\in \prod_{\lambda\in\Lambda} \Ff_X (U_\lambda)\st f_\lambda|_{U_\lambda\cap U_\vartheta} = f_\vartheta|_{U_\lambda\cap U_\vartheta} \text{ for }\lambda,\vartheta\in \Lambda\right\}\\
			f &\longmapsto (f|_{U_\lambda})_{\lambda\in\Lambda}
        \end{align*}
        is bijective.
    \end{quote}
\begin{rem}
    When $U=\emptyset$ one can take $I=\emptyset$ and obtains $\Ff_X(\emptyset) = \{0\}$.
\end{rem}
\begin{rem}
    Sheaves of groups, sets, etc. are defined in a similar way. A sheaf of rings $\Rr_X$ on $X$ defines two sheaves of groups on $X$: $U\mapsto (\Rr_X(U), +)$ and $U\mapsto ((\Rr_X(U))^\times, \cdot)$
\end{rem}
\begin{rem}
    Elements of $\Rr_X(U)$ are called sections, elements of $\Rr_X(X)$ are called global sections.
\end{rem}
\begin{example}
    Let $R$ be a ring. The sheaf $\Oo_X$ of $R$-valued functions on $X$ associates to any open subset $U\subseteq X$ the ring of $R$-valued functions $f: U \to R$ with the inclusion morphism being the restriction of functions to subsets.
\end{example}

\begin{rem*}
  If $\Gg$ is any (pre)sheaf on $X$ and $U\subseteq X$ an open subset. We get a a sheaf $\Gg_U$ on $U$ by $\Gg_U(V) = \Gg_X(V)$ for the open subsets $V\subseteq U$ equipped with the same restriction morphisms.
\end{rem*}
\begin{defi}[Algebraic (Pre)Varieties] \lbl{def:preVarieties}
    Let $\kk$ be an algebraic closed field. An \emph{algebraic (pre)variety} over $\kk$ is a pair $(X,\Oo_X)$, where $X$ is an irreducible Noetherian topological space together with a sheaf $\Oo_X$ of rings on $X$ such that the following property is satisfied.
    \begin{quote}
        Any $x\in X$ has an open neighbourhood $U$ such that there is a homeomorphism $U\isomorphism[\phi] V$ where $V\subseteq k^n$ is a Zariski-closed (and since $X$ is irreducible, an irreducible) subset such that $\phi$ identifies $\Oo_X|_U$ with the structure sheaf $\Oo_V$. I.e. if $W \subseteq V$ is open and $f: W\to \kk$ any $k$-valued function it is regular if and only if 
        \begin{align*}
            g: \phi^{-1}(W) &\longto \kk\\
            x&\longmapsto f(\phi(x))
        \end{align*}
        is an element of $\Oo_X(\phi^{-1}(W))$. 
    \end{quote}
    A morphism of varieties $(X,\Oo_X) \to (Y,\Oo_Y)$ is a continous map $X\morphism[\phi] Y$ such that $U\subseteq Y$ and $\lambda \in \Oo_Y(U)$ implies $\phi^\ast \lambda \in \Oo(\phi^{-1}(U))$. Here $\phi^\ast\lambda$ is defined as $(\phi^\ast\lambda)(x) = \lambda(\phi(x))$.

\end{defi}
\begin{rem*}
    \begin{alphanumerate}
     \item Let $V\subseteq k^n$ be Zariski-closed, $W\subseteq V$ open. The ring $\Oo_V(W)$ of regular functions on $W$ is the ring of functions $\lambda: W\to k$ such that for any $x\in W$ there is an open neighbourhood $\Omega$ of $x$ and polynomials $p,q\in R= k[X_1,\ldots,X_n]$ such that $q$ does not vanish on $\Omega\cap W$ and such that we have $\lambda(y) = \frac{p(y)}{q(y)}$ for every $y\in \Omega\cap W$. The sheaf $\Oo_V$ defined by $W\mapsto \Oo_V(W)$ is called the \emph{structure sheaf} on $V$. If $W=V$ it can be shown that any $f\in \Oo(W)$ can be written as $f=p|_V$ where $p\in R$.
     \item Let $\IP(V)$, where $V$ is a $k$-vector space, be the set of one-dimensional subspaces of $V$. Let $\IP^n(k) = \IP(k^n+1)$. If $(x_0,\ldots,x_n)\in k^n\setminus\{0\}$, let $[x_0,\ldots,x_n]$ denote the subspace generated by $(x_1,\ldots, x_n)$.
     
     Recall that an ideal $I\subseteq R$ is called homogenous if it is generated by homogenous elements (i.e. polynomials where every monomial has the same total degree). Let $I$ be homogenous, let $V(I)\subseteq \IP^n(k)$ be the set of all $[x_0, \ldots, x_n]\in\IP^n(k)$ such that $f(x_0,\ldots, x_n)$ vanishes for all $f\in I$. Let a subset $A$ of $\IP^n(k)$ be called Zariski-closed if there is a homogenous ideal $I$ such that $A= V(I)$. This turns $\IP^n(k)$ into an irreducible, $n$-dimensional, Noetherian topological space.
     
     Let $V\subset \IP^n(k)$ be closed, $W\subseteq V$ open and $\lambda: W\to k$ any function. We call $\lambda$ regular on $W$ or $\lambda \in \Oo_V(W)$, if any $x\in W$ has an open neighbourhood $\Omega$ such that there are two polynomials $p,q\in k[X_1,\ldots,X_{n+1}]$ being homogenous of the same degree such that $q(y_0,\ldots,y_n) \neq 0$ and $\lambda([y_0,\ldots,y_n]) = \frac{p(y_0,\ldots,y_n}{q(y_0,\ldots,y_n)}$ for all $[y_0,\ldots,y_n]\in W\cap \Omega$.
    \end{alphanumerate}

\end{rem*}






    

\end{defi}

\end{document}          

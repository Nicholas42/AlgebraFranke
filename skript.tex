\documentclass[DIV=14,parskip=half]{scrartcl}
\usepackage[utf8]{inputenc}
\setkomafont{sectioning}{\rmfamily\bfseries}

\usepackage{amsmath,amsfonts,amssymb,mathtools}
\usepackage{amsthm}
\usepackage{bm}
\usepackage{tikz}
\usepackage[shortlabels]{enumitem}

\usepackage[english]{babel}
\usetikzlibrary{calc}
\usepackage{subcaption}

\renewcommand{\phi}{\varphi}
\title{Algebra I}
\author{Nicholas Schwab}
\date{Sommersemester 2017}

\newtheorem{lem}{Lemma}[subsection]
\newtheorem{thm}{Theorem}[subsection]
\newtheorem{sat}{Satz}[subsection]
\newtheorem{exc}{Exercise}[subsection]
\newtheorem{cor}{Corollary}[subsection]
\newtheorem{prop}{Proposition}[subsection]
\theoremstyle{definition}
\newtheorem{defi}{Definition}[subsection]
\newtheorem*{example}{Example}%[subsection]
\newtheorem*{rem}{Remark}%[subsection]

\newcommand{\N}{\mathbb{N}}
\newcommand{\Z}{\mathbb{Z}}
\newcommand{\Q}{\mathbb{Q}}
\newcommand{\R}{\mathbb{R}}
\newcommand{\C}{\mathbb{C}}
\newcommand{\Hom}{\operatorname{Hom}}
\newcommand{\multiline}[1]{#1}
\newcommand{\longto}{\longrightarrow}
\newcommand{\longot}{\longleftarrow}

\renewcommand{\phi}{\varphi}
\renewcommand{\epsilon}{\varepsilon}
\begin{document}

\maketitle
%20.04.2017
%organisatorischer Spam
\section{The Hilbert Basis- and Nullstellensatz}
\subsection{Noetherian Rings}
\begin{defi}\label{def:generatedIdeal}
 Let $R$ be a ring, and $f_1,\dots, f_n\in R$ , then 
 \begin{align*}\langle f_1,\dots,  f_n\rangle_R = \left\{\sum_{i=1}^n \lambda_i f_i \middle| \lambda_i \in R\right\} = \bigcap_{\substack{I\subseteq R,\\ I \text{ ideal },\\ f_i\in I\forall i}} I.
 \end{align*}
This is called the \emph{ideal} generated by the $f_i$ and the $f_i$ are called a \emph{basis} or \emph{generators} of $I$. 
\end{defi}
\begin{rem}
 If $I$ is not neccessarily finite, 
 \begin{align*}
\langle f_i\mid i\in I\rangle_R = \left\{\sum_{i\in I} \lambda_i f_i \middle| \lambda_i = 0 \text{ for all but finitely many } i\right\} = \bigcap_{\substack{I\subseteq R,\\ I \text{ ideal },\\ f_i\in I\forall i}} I.
\end{align*}
\end{rem}
\begin{defi}\label{def:zeroOfIdeal}
 Let $k$ be a field, $I\subseteq k[T_1,\dots, T_n]$ an ideal, $l$ a field extension of $k$. $x\in l^n$ is a zero of $I$ iff $f(x_1,\dots,x_n) = 0$ for all $f\in I$. 
\end{defi}
\begin{rem}
 $x$ is a common zero of the $f_i\in k[X_1,\dots,X_n]$ iff is a zero of the ideal generated by the $f_i$.
\end{rem}
\begin{prop}\label{prop:Noetherian}
 For a ring $R$ the following conditions are equivalent:
 \begin{enumerate}[a)]
  \item Every ideal has a finite set of generators (i.e. is finitely generated).
  \item Every ascending chain $I_0 \subseteq I_1 \subseteq \dots$ of ideals in $R$ terminates after finitely many steps, i.e. there is some $n\in\N$ such that $I_k=I_n$ for all $k\geq n$.
  \item Every non-empty set $\mathcal{M}$ of ideals in $R$ has an $\subseteq$-maximal element $I$. 
 \end{enumerate}

\end{prop}


\begin{defi}\label{def:Noetherian}
 A ring with these properties is called \emph{Noetherian}.
\end{defi}
\begin{example}
 Fields and principal ideal domains are Noetherian. 
\end{example}
\begin{thm}[Hilbert's Basissatz]\label{thm:Basissatz}
 If $R$ is Noetherian, $R[T_1,\dots,T_n]$ (with finite $n$!) is Noetherian.
\end{thm}
\begin{proof}
 The proof is recapitulated later on.
\end{proof}
\begin{cor}[of the Basissatz]
 Every polynomial system of equations in finitely many variables over a field has finite subsystem with the same set of solutions.
\end{cor}
\begin{thm}[Hilbert's Nullstellensatz] \label{thm:Nullstellensatz}
 Let $k$ be a algebraically closed field and $I\subsetneq k[X_1,\dots,X_n]$ a proper ideal. Then $I$ has a zero $x\in k^n$.
\end{thm}
\begin{proof}
 This will be proofed in a few days.
\end{proof}
\subsection{Modules over rings}
\begin{defi}\label{def:module}
 An $R$-Module (where $R$ is a ring) is an abelian group $(M,+)$ with an operation
 \begin{align*}
  \cdot: R\times M &\longto M\\
  (r,m) &\longmapsto r\cdot m
 \end{align*}
 such that
 \begin{align*}
  r\cdot(s\cdot m) &= (r\cdot s)\cdot m \\
  (r+s)\cdot m &= r\cdot m + s\cdot m\\
  r\cdot(m+n) &= r\cdot m +r\cdot n\\
  1\cdot m &= m.
 \end{align*}
A morphism of $R$-Modules is a map $M \overset{f}{\longto} N$ which is a homomorphism of abelian groups compatible with $\cdot$.
A submodule of $M$ is a subgroup $X\subseteq M$ of $(M,+)$ such that $R\cdot X \subseteq X$. 
\end{defi}
\begin{example} The $R$-submodules of $R$ are the ideals in $R$.
\end{example}
\begin{prop} If $N\subseteq M$ is a $R$-submodule of the $R$-module $M$ the quotient group $M/N$ has a unique structure of an $R$-submodule such that the projection $M\overset{\pi}{\longto} M/N$ is a morphism of $R$-modules, and for arbitrary $R$-modules $T$ the map  
\begin{align*}
 \Hom_R(M/N, T) &\longto \{\tau\in \Hom_R(M,T)| \tau|_N = 0\}\\
 t &\longmapsto \tau = t \circ \pi
\end{align*}
is bijective, where $t$ is surjective iff $\tau$ is and $t$ is injective iff $\ker(\tau)$ equals $N$.
\end{prop}
\begin{rem}
 Two important corollaries are:
 \begin{align*}
  (M/L)/(N/L) \overset{\simeq}{\longleftarrow} M/N
 \end{align*}
for $M\supseteq N \supseteq L$ and, for submodules $N$ and $L$ of $M$
\begin{align*}
 (N+L)/N \overset{\simeq}{\longot} L/(N\cap L)
\end{align*}
where $N+L$ denotes the submodule $\{l+n|l\in L, n\in N\}$ of $M$.
\end{rem}
\begin{defi}
 If $M$ and $N$ are $R$-modules, $M\oplus N = \{(m,n),|m\in M, n\in N\} = M\times N$ equipped with component-by-component addition and scalar multiplication. This can be generalized to finitely many summands.
\end{defi}
\begin{example} $R^n = \{(r_i)_{i=1}^n |r_i\in R\}$ is an $R$-module.
\end{example}
\begin{defi}\label{def:generatedModule}
 If $M$ is an $R$-module and $m_1,\dots,m_k\in M$, then the submodule generated by $\{m_i|1\leq i\leq k\}$ is
 \begin{align*}
  \left\{\sum_{i=1}^k r_i\cdot m_i \middle| r_i\in R\right\} = \bigcap_{\substack{X\subseteq M\\ X \text{ module}\\ \text{all } m_i\in X}} X
 \end{align*}
As was the case for Definition \ref{def:generatedIdeal}, this can be generalized to infinitely many generators. $M$ is finitely generated iff there are $(m_i)_{i=1}^k$, $k\in \N$, $m_i\in M$ such that the submodules of $M$ generated by the $m_i$ equals $M$.
\end{defi}
\begin{prop}
 Let $N\subseteq M$ be an $R$-submodule
 \begin{enumerate}[a)]
  \item If $M$ is finitely generated, $M/N$ is finitely generated.
  \item If $N$ and $M/N$ are finitely generated, $M$ is finitely generated.
 \end{enumerate}
 \end{prop}
 
\begin{cor}
 $M\oplus N$ is finitely generated iff $M$ and $N$ are. (Note that: $M\simeq M\oplus\{0\}$ and $(M\oplus N)/M \simeq N$)
\end{cor}
\begin{prop}\label{prop:NoetherianModule}
Let $M$ be an $R$-module. The following properties are equivalent:
\begin{enumerate}[a)]
 \item Every submodule $N\subseteq M$ of $M$ is finitely generated.
 \item Every ascending sequence $N_0\subseteq N_1\subseteq \dots$ of submodules of $N$ terminates.
 \item Every non-empty set $\mathcal{M}$ of $R$-submodules of $M$ has a $\subseteq$-maximal element.
\end{enumerate}
\end{prop}
\begin{proof}
$\mathbf{a)\to b)}$ Let $N_\infty = \bigcup_{i=0}^\infty N_i$, then this is a submodule, hence finitely generated by a). Let $n_1,\dots, n_k$, $k\in\N$, generate $N_\infty$ and let $j_i$, for $1\leq i \leq k$, be chosen such that $n_i\in N_{j_i}$ and let $l = \max\{j_i|1\leq i \leq k\}$, then $n_l = N_\infty$.

 
\end{proof}

\end{document}

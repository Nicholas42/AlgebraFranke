\documentclass[a4paper,parskip=half,numbers=enddot, DIV=12]{scrreprt}
%\usepackage[utf8]{inputenc}

\usepackage{../header}
\usepackage{../frankenumbering2}
\usepackage{../shortcuts2}

\usepackage{blindtext}

\usepackage{eurosym}
\usetikzlibrary{fadings}
\usepackage{tikzsymbols}

\usepackage{csquotes}
%\usepackage{tikz-cd}%I cannot draw diagrams without it - Felix. %well, I can - Ferdinand
\usepackage[backend=biber,style=alphabetic]{biblatex}
\setcounter{biburlnumpenalty}{7000}
\setcounter{biburllcpenalty}{7000}
\setcounter{biburlucpenalty}{8000}
\addbibresource{../literatur.bib}

% Title Page
\title{Jacobians of Curves}
\author{\textsc{Lecturer:} Jens Franke\\
	\textsc{Notes:} Ferdinand Wagner}
\date{Wintersemester 2018/19}

\displaywidowpenalty=8000
%\postdisplaypenalty=8000
\widowpenalty=8000
\clubpenalty=8000

\newcommand{\vC}{\v{C}}
\renewcommand{\geq}{\geqslant}
\renewcommand{\leq}{\leqslant}

\DeclareSymbolFont{cyrletters}{OT2}{wncyr}{m}{n}
\DeclareMathSymbol{\Sha}{\mathalpha}{cyrletters}{"58}
\usetikzlibrary{backgrounds}
\newcommand{\tikzentry}[3][]{\tikz[remember picture, baseline =(#2.base)]\node[#1, inner sep=0pt, outer sep=0pt] (#2) {$#3$};}

\makeatletter
\renewcommand{\@pnumwidth}{3em} 
\renewcommand{\@tocrmarg}{4em}
\makeatother

\begin{document}
\pagenumbering{Alph}
\maketitle
\pagenumbering{roman}

\thispagestyle{plain}
This text consists of notes on the lecture Jacobians of Curves, taught at the University of Bonn by Professor Jens Franke in the winter term (Wintersemester) 2018/19.

Please report bugs, typos etc. through the \emph{Issues} feature of github.

\tableofcontents

%\addchap{Introduction}
%Nothing here yet. 

\chapter{Introduction and preparations}
\pagenumbering{arabic}

\section{A note on limits and their derived functors}
	Let $X_\bullet\colon \ldots\xrightarrow{p_{i+1}}X_i\morphism[p_i]\ldots\morphism[p_2]X_1\morphism[p_1]X_0$ be a diagram of abelian groups or $R$-modules. As usual, we may view $X_\bullet$ as a functor $X_\bullet\colon (\IN,\geq)\morphism \cat{Ab}$ or $\cat{Mod}(R)$, where the category $(\IN,\geq)$ has the nonnegative integers as objects and an arrow $j\morphism i$ iff $j\geq i$. Let 
	\begin{align*}
		d\colon \prod_{i=0}^{\infty}X_i\morphism\prod_{i=0}^\infty X_i\;,\quad d\left(x_i\right)_{i=0}^\infty=\left(p_{i+1}(x_{i+1})-x_i\right)_{i=0}^\infty\;.
	\end{align*}
	Then we put
	\begin{align*}
		\limit[i\in\IN]X_i=\ker d\quad\text{and}\quad\limit[i\in\IN][^1]X_i=\coker d\;.
	\end{align*}
\begin{rem}
	It is easy to see that $\limit X_i$ equals the usual category-theoretical limit (that's how you construct it). It can also be shown that $\limit[][^1]$ is the first right-derived functor of $\limit$, and that its higher derived functors vanish.
\end{rem}
\begin{fact}\lbl{fact:6termLimitSequence}
	Let $0\morphism X'_\bullet\morphism X_\bullet\morphism X''_\bullet\morphism 0$ be a short exact sequence of diagrams of the above type. Then there is a canonical exact sequence
	\begin{align*}
		0\morphism \limit[i\in\IN] X'_i\morphism \limit[i\in\IN] X_i\morphism \limit[i\in\IN] X''_i\morphism \limit[i\in\IN][^1] X'_i\morphism \limit[i\in\IN][^1]X_i\morphism \limit[i\in\IN][^1] X''_i\morphism 0\;.
	\end{align*}
\end{fact}
\begin{proof}
	Since products preserve exact sequences in $\cat{Ab}$ or $\cat{Mod}(R)$, we get a diagram
	\begin{diagram*}
		\object{0,2}{$0\vphantom{\displaystyle\prod_{i=0}^\infty }$}[0o];
		\object{2,2}{$\displaystyle\prod_{i=0}^\infty X'_i$}[1o];
		\object{4,2}{$\displaystyle\prod_{i=0}^\infty X_i$}[2o];
		\object{6,2}{$\displaystyle\prod_{i=0}^\infty X''_i$}[3o];
		\object{8,2}{$0\vphantom{\displaystyle\prod_{i=0}^\infty }$}[4o];
		\object{0,0}{$0\vphantom{\displaystyle\prod_{i=0}^\infty }$}[0u];
		\object{2,0}{$\displaystyle\prod_{i=0}^\infty X'_i$}[1u];
		\object{4,0}{$\displaystyle\prod_{i=0}^\infty X_i$}[2u];
		\object{6,0}{$\displaystyle\prod_{i=0}^\infty X''_i$}[3u];
		\object{8,0}{$0\vphantom{\displaystyle\prod_{i=0}^\infty }$}[4u];
		\scriptsize
		\arrow{0o}{1o};
		\arrow{1o}{2o};
		\arrow{2o}{3o};
		\arrow{3o}{4o};
		\arrow{0u}{1u};
		\arrow{1u}{2u};
		\arrow{2u}{3u};
		\arrow{3u}{4u};
		\arrow{1o}{1u}[left][$d'$];
		\arrow{2o}{2u}[left][$d$];
		\arrow{3o}{3u}[left][$d''$];
	\end{diagram*}
	with exact rows. Then the snake lemma finishes the job.
\end{proof}
\begin{fact}\lbl{fact:limVanishing}
	Let $X_\bullet$ have the property that for every $i\in\IN$ there is some $j\geq i$ such that the composition $p_{j,i}\colon X_j\morphism[p_j]X_{j-1}\xrightarrow{p_{j-1}}\ldots\xrightarrow{p_{i+1}}X_i$ vanishes. Then
	\begin{align*}
		\limit[i\in\IN]X_i=\limit[i\in\IN][^1]X_i=0\;.
	\end{align*}
\end{fact}
\begin{proof}
	If $x=\left(x_i\right)_{i=0}^\infty\in\limit X_i$, then $x_i=p_{j,i}(x_j)$ for all $j\geq i$ by construction, hence $x_i=0$ for all $i\in\IN$. Moreover, let
	\begin{align*}
		s\colon \prod_{i=0}^{\infty}X_i\morphism\prod_{i=0}^\infty X_i\;,\quad s(x)_i=\sum_{j\geq i}p_{j,i}(x_j)\;.
	\end{align*}
	By assumption $s$ is well-defined. Then
	\begin{align*}
		d(s(x))_i=p_{i+1}\bigg(\sum_{j\geq i+1}p_{j,i+1}(x_j)\bigg)-\sum_{j\geq i}p_{j,i}(x_j)=-p_{i,i}(x_i)=-x_i\;.
	\end{align*}
	Hence $-s$ is a right-inverse of $d$, so $\limit[][^1]X_i=\coker d$ vanishes as well.
\end{proof}
\begin{fact}\lbl{fact:MittagLeffler}
	Let $X_\bullet$ have the \defemph{Mittag-Leffler property} that for every $i\in\IN$ there is a $j\geq i$ such that for all $k\geq j$ the images of $p_{j,i}$ and $p_{k,i}$ in $X_i$ coincide. Then $\limit[][^1]X_i=0$.
\end{fact}
\begin{proof}
	Let's first deal with the special case that each $p_i\colon X_i\morphism X_{i-1}$ is surjective. Let $x=\left(x_i\right)_{i=0}^\infty\in\prod_{i=0}^\infty X_i$. For every $i\in\IN$ we may select $x_j^{(i)}\in X_j$ for all $j\geq i$ in such a way that $x_i^{(i)}=x_i$ and $p_{j+1}\big(x_{j+1}^{(i)}\big)=x_j^{(i)}$. Then $s(x)$ defined by
	\begin{align*}
		s(x)_i=\sum_{k=0}^{i-1}x_i^{(k)}
	\end{align*}
	is a preimage of $x$ under $d$, so $\limit[][^1]X_i=\coker d=0$ in this case.
	
	Now let $X_\bullet$ be arbitrary with the Mittag-Leffler property. Let $Y_i=\bigcap_{j\geq i}p_{j,i}(X_j)\subseteq X_i$. Then $\limit[][^1]Y_i=0$ by the special case we just treated, and $\limit[][^1]X_i/Y_i=0$ by Fact~\reff{fact:limVanishing}. Since $\limit[][^1]X_i$ is sandwiched between these two in the exact sequence from Fact~\reff{fact:6termLimitSequence}, this shows $\limit[][\smash{^1}]X_i=0$, as required.
\end{proof}
\section{The theorem about formal functions}
Let $f\colon X\morphism Y=\Spec A$ be a  morphism of quasi-compact schemes. Let $I\subseteq A$ be any ideal. For every $n\geq 0$ consider the morphism
\begin{align*}
	i_n\colon X_n=X\times_Y\Spec (A/I^n)\morphism X\;,
\end{align*}
which is a base change of the closed immersion $Y_n=\Spec (A/I^n)\monomorphism \Spec A$, hence a closed immersion itself. Also, if $f$ is proper, then so is $X_n\morphism Y_n$ because properness is another \emph{proper}ty (tee-hee) that is stable under base change (by \cite[Remark~2.4.1]{alggeo2}).

Let $\Ff$ be a quasi-coherent sheaf of $\Oo_X$-modules ($\Ff$ will eventually be assumed coherent, but for now this suffices) and $\Ff|_{X_n}=i_n ^*\Ff$ its restriction to $X_n$ (this notation is slightly abusive, but convenient).  Since $i_n$ is a closed immersion and thus affine, we have an isomorphism $H^p(X,i_{n,*}\Ff|_{X_n})\cong H^p(X_n,\Ff|_{X_n})$ for all $p\geq 0$ by \cite[Corollary~1.6.1]{alggeo2}. But it's also easy to check (e.g.\ affine-locally) that $i_{n,*}\Ff|_{X_n}\cong \Ff/I^n\Ff$. Together with the projection $ \Ff/I^{n+1}\Ff\morphism \Ff/I^n\Ff$ this gives canonical morphisms 
\begin{align}\lbl{eq:HpXn+1toHpXn}
	H^p(X_{n+1},\Ff|_{X_{n+1}})\cong H^p(X,\Ff/I^{n+1}\Ff)\morphism H^p(X,\Ff/I^n\Ff)\cong H^p(X_n,\Ff|_{X_n})\;.
\end{align}
We have another canonical projection $\Ff\morphism \Ff/I^n\Ff$, which induces a morphism
\begin{align}\lbl{eq:Hmorphism}
	H^p(X,\Ff)\morphism H^p(X,\Ff/I^n\Ff)\cong H^p(X_n,\Ff|_{X_n})
\end{align}
for all $p\geq 0$. This is a morphism of $A$-modules, but $H^p(X_n,\Ff|_{X_n})$ is actually an $A/I^n$-module, so \eqreff{eq:Hmorphism} factors over
\begin{align}\lbl{eq:Hmorphism2}
	H^p(X,\Ff)/I^nH^p(X,\Ff)\morphism H^p(X_n,\Ff|_{X_n})\;.
\end{align}
This is compatible with the canonical morphisms $H^p(X_{n+1},\Ff|_{X_{n+1}})\morphism H^p(X_n,\Ff|_{X_n})$ from equation \eqreff{eq:HpXn+1toHpXn} -- you can just check that on affine \v Cech covers. Taking the limit over \eqreff{eq:Hmorphism2} gives a morphism
\begin{align}\lbl{eq:formalFunctions}
	H^p(X,\Ff)^\complete\morphism\limit[n\geq 1]H^p(X_n,\Ff|_{X_n})\;,
\end{align}
where $\roof{\phantom{x}}$ denotes the $I$-adic completion.
\begin{thm}[Grothendieck--Zariski]\lbl{thm:FormalFunctions}
	When $f\colon X\morphism Y=\Spec A$ is proper (in which case $X$ is automatically a quasi-compact scheme), $A$ is noetherian and $\Ff$ is a coherent sheaf of $\Oo_X$-modules, then \eqreff{eq:formalFunctions} is an isomorphism
	\begin{align*}
		H^p(X,\Ff)^\complete\isomorphism \limit[n\geq 1]H^p(X_n,\Ff|_{X_n})\;.
	\end{align*}
\end{thm}

\begin{proof}
	The following proof is essentially the one from \cite[Ch.\:III (4.1.7)]{egaIII}. Professor Franke also pointed out that the idea is pretty similar to the proof of the Artin--Rees lemma. Let $I\subseteq A$ be the ideal under consideration and let $R=\bigoplus_{n\geq 0}I^n$ be the Rees algebra associated to $I$. Then
	\begin{align*}
		K^p=\bigoplus_{n\geq 0}H^p(X,I^n\Ff)
	\end{align*}
	is a module over $R$ as $i\in I^m$ (considered as the $m\ordinalth$ homogeneous component of $R$) maps $I^n\Ff$ to $I^{n+m}\Ff$.
	\begin{claim}\lbl{claim:FFclaim1}
		$K^p$ is a finitely generated $R$-module for all $p\geq 0$.
	\end{claim}
	Assuming this for the moment, recall that $H^p(X,\Ff/I^n\Ff)\cong H^p(X_n,\Ff|_{X_n})$, so the long exact cohomology sequence associated to $0\morphism I^n\Ff\morphism \Ff\morphism \Ff_n\morphism 0$ appears as
	\begin{align*}%\lbl{eq:FFSeq1}
		\ldots\morphism H^p(X,I^n\Ff)\morphism H^p(X,\Ff)\morphism H^p(X_n,\Ff|_{X_n})\morphism H^{p+1}(X,I^n\Ff)\morphism\ldots \;.
	\end{align*}
	By \eqreff{eq:Hmorphism2}, we can turn the above long exact sequence into four-term exact sequences
	\begin{align}\lbl{eq:FFSeq2}
		0\morphism U_n\morphism H^p(X,\Ff)/I^nH^p(X,\Ff)\morphism H^p(X_n,\Ff|_{X_n})\morphism V_n\morphism 0\;,
	\end{align}
	where $U_n$ is a suitable quotient of $H^p(X,I^n\Ff)$ and $V_n\subseteq H^{p+1}(X,I^n\Ff)$ some submodule. This makes $U=\bigoplus_{n\geq 0}U_n$ a quotient of $K^p$ and $V=\bigoplus_{n\geq 0}V_n$ an $R$-submodule of $K^{p+1}$.
	\begin{claim}\lbl{claim:FFclaim2}
		We have $\limit U_n=\limit[][^1]U_n=0$ and $\limit V_n=\limit[][^1]V_n=0$.
	\end{claim}
	Before we prove this (and Claim~\reff{claim:FFclaim1}), let's see how Theorem~\reff{thm:FormalFunctions} follows from it. Let $W_n$ be the image of $H^p(X,\Ff)/I^nH^p(X,\Ff)\morphism H^p(X_n,\Ff|_{X_n})$. We may split \eqreff{eq:FFSeq2} into two short exact sequences
	\begin{gather*}
		0\morphism U_n\morphism H^p(X,\Ff)/I^nH^p(X,\Ff)\morphism W_n\morphism 0\\
		\hphantom{\;.}0\morphism W_n\morphism H^p(X_,\Ff|_{X_n})\morphism V_n\morphism 0\;.
	\end{gather*}
	Applying Fact~\reff{fact:6termLimitSequence} to the first one gives $H^p(X,\Ff)^\complete\cong \limit[][^1]W_n$. Then the six-term exact sequence associated to the second proves $\limit W_n\cong \limit H^p(X_n\Ff|_{X_n})$ and we are done.
	
	It remains to show the two claims. Note that the Rees algebra $R$ is noetherian. Indeed, $I$ is finitely generated as an ideal in the noetherian ring $A$, hence $R$ is of finite type over $A$. Let's also make the following convention: Whenever we write $I^kU_n$ or $I^kV_n$ in the following, this means multiplication as $A$-modules and the result is contained in $U_n$ resp.\ $V_n$ again, whereas $R_kU_n$ or $R_kV_n$ means multiplication by the $k\ordinalth$ homogeneous component of $R$ (which equals $I^k$ as well, but multiplication in a graded module has a shifting effect, which is what we want to stress by this distinction), so the result is contained on $U_{k+n}$ resp.\ $V_{k+n}$.
	
	\emph{Proof of Claim~\reff{claim:FFclaim2}.} Note that $U$ is finitely generated over $R$, since it is a quotient of the finitely generated $R$-module $K^p$. Fix a finite set of generators and let $d_0$ the maximal non-zero homogeneous components occurring in this set. Then $U_n=R_nU_0+R_{n-1}U_1+\ldots+R_{n-d_0}U_{d_0}$ for all $n\geq d_0$. In particular, $U_{k+n}=R_kU_n$ for all $n\geq d_0$. Thus, for every $n\geq d_0$ the image of $U_{2n}=R_nU_n$ in $U_n$ is contained in $I^nU_n$. But $U_n\subseteq H^p(X,\Ff)/I^nH^p(X,\Ff)$, so $I^nU_n$ vanishes. Therefore, the property from Fact~\reff{fact:limVanishing} is fulfilled for all $n\geq d_0$. But then it clearly holds for all $n\geq 0$ as well, so Fact~\reff{fact:limVanishing} is applicable.
	
	Similarly, $V$ is finitely generated as a submodule of $K^{p+1}$, which is finitely generated over the noetherian ring $R$ by Claim~\reff{claim:FFclaim1}. By the same argument as above we find a $d_1$ such that $V_n=R_nV_0+R_{n-1}V_1+\ldots+R_{n-d_1}V_{d_1}$ for all $n\geqslant d_1$. In particular, we have $V_{k+n}=R_kV_n$ for all $n\geq d_1$. Thus, for $n\geq d_1$ the image of $V_{2n}$ in $V_n$ is contained in $I^nV_n$. But $I^nV_n$ vanishes again, since $V_n$ is the image of $H^p(X_n,\Ff|_{X_n})$, which is a $A/I^n$-module. As above, we can apply Fact~\reff{fact:limVanishing}. This shows Claim~\reff{claim:FFclaim2}.
	
	\emph{Proof of Claim~\reff{claim:FFclaim1}.} Let $\upsilon\colon\snake{Y}=\Spec R\morphism Y$ correspond to $A\monomorphism R$ and let $\xi\colon \snake{X}=X\times_Y\snake{Y}\morphism X$ be its base change by $f$. Note that $\xi$ is affine as a base change of the affine morphism $\upsilon$ (we use \cite[Corollary~2.5.1]{alggeo1} here). We claim
	\begin{align*}
		\xi_*\xi^*\Ff\cong\bigoplus_{n\geq 0}I^n\Ff\;.
	\end{align*}
	Indeed, this is easily checked affine-locally (where $\xi^*$ is given by tensoring with $R$); we leave the details to the reader. Also $H^p(\snake{X},\xi^*\Ff)\cong H^p(X,\xi_*\xi^*\Ff)$ as $\xi$ is affine. This shows
	\begin{align*}
		H^p(\snake{X},\xi^*\Ff)\cong H^p(X,\xi_*\xi^*\Ff)\cong H^p\bigg(X,\bigoplus_{n\geq 0}I^n\Ff\bigg)\cong \bigoplus_{n\geq 0}H^p(X,I^n\Ff)=K^p\;.
	\end{align*}
	Note that direct sums usually \emph{don't} commute with cohomology, but here they do, because $X$ is quasi-compact and $\bigoplus_{n\geq 0}I^n\Ff$ is quasi-coherent (for which we already need that $X$ is quasi-compact, otherwise direct sums of quasi-coherent sheaves aren't necessarily quasi-coherent again), so we may compute $H^p\left(X,\bigoplus_{n\geq 0}I^n\Ff\right)$ via finite affine \v Cech covers. In this case, the products in the \v Cech complex are all finite, hence commute with the direct sum, which is what we needed.
	
	Now $\snake{f}\colon \snake{X}\morphism\snake{Y}=\Spec R$ is proper (as a base change of the proper morphism $f$), hence the right-hand side is a finitely generated $R$-module by our finiteness results for the cohomology of proper morphisms (cf.\ \cite[Theorem~5]{alggeo2}). We win.
\end{proof}
\begin{rem}
	Note that in the lecture Franke used $\Kk_n\cong \Jj^n\Ff$ instead of $I^n\Ff$, where $\Jj=f^{-1}\snake I$ denotes the inverse image in the sense of the upcoming Definition~\reff{def:f-1}. But since the $I^n$-action on $\Ff$ is given via the algebraic component $\Oo_Y\morphism f_*\Oo_X$ of $f$, it is not very surprising that $I^n\Ff= \Jj^n\Ff$ (actually, this is pretty obvious from the construction of $f^{-1}$ described in the proof of Lemma~\reff{lem:f-1} below). I prefer the notation $I^n\Ff$ -- in particular, this is how Grothendieck denotes it in \cite[Ch.\:III (4.1.7)]{egaIII}, so I believe it's my right to do so as well. Nevertheless, Lemma~\reff{lem:f-1} is \emph{perhaps worthwhile to know} (as Professor Franke would likely phrase it), so we will include it now.
\end{rem}
\begin{defi}\lbl{def:f-1}
	Let $f\colon X\morphism Y$ be any morphism of preschemes and $\Jj\subseteq \Oo_Y$ a sheaf of ideals on $Y$. Then define $f^{-1}\Jj$ to be the image of $f^*\Jj\morphism\Oo_X$ (which is obtained as the composition of the pullback of $\Jj\morphism \Oo_Y$ with the isomorphism $f^*\Oo_Y\cong \Oo_X$).
\end{defi}
\begin{lem}\lbl{lem:f-1}
	Let $f\colon X\morphism Y$ be any morphism of preschemes and $\Jj\subseteq \Oo_Y$ quasi-coherent.
	\begin{alphanumerate}
		\item $f^{-1}\Jj\subseteq \Oo_X$ is a quasi-coherent sheaf of ideals.
		\item Let $Y_0$ and $X_0$ be the closed subpreschemes of $Y$ and $X$ defined by $\Jj$ and $f^{-1}\Jj$ respectively. Then $X_0\cong X\times_YY_0$.
		\item For all $n\geq 0$ we have $f^{-1}(\Jj^n)\cong (f^{-1}\Jj)^n$.
	\end{alphanumerate}
\end{lem}
\begin{proof}[Sketch of a proof]
	The question is easily seen to be local on both $X$ and $Y$. So let's consider the affine situation where $Y=\Spec A$, $X=\Spec B$, and $\Jj=\snake{J}$ for some ideal $J\subseteq A$. Let $\phi\colon A\morphism B$ be the morphism of rings corresponding to $f$. Then $f^{-1}\Jj=\snake{I}$ where $I$ is the image of $B\otimes_AJ\morphism B$ sending $b\otimes j\mapsto b\cdot \phi(j)$. All three assertions are then easily checked.
\end{proof}
\begin{rem}
	\begin{alphanumerate}
		\item \lbl{rem:fibres}Recall that for a morphism $f\colon X\morphism Y$ of preschemes and a point $y\in Y$ the \defemph{fibre} $f^{-1}\{y\}$ of $f$ at $y$ is defined as the prescheme $f^{-1}\{y\}=X\times_Y\Spec \KK(y)$. This makes sense, since $f^{-1}\{y\}$ is indeed -- topologically -- the preimage of $y$, because $\Spec \KK(y)\morphism Y$ is an immersion with image $\{y\}$, so \cite[Corollary~1.3.3]{alggeo1} is applicable.
		\item Moreover, $\Spec \Oo_{Y,y}/\mm_{Y,y}^n\morphism Y$ is immersive for all $n\geq 1$ and has image $\{y\}$ too. So the same argument shows that $X_n=X\times_Y\Spec (\Oo_{Y,y}/\mm_{Y,y}^n)$ has $f^{-1}\{y\}$ as underlying topological space too (but, of course, the prescheme structure differs in general). We may thus think of $X_n$ as the \defemph{$\boldsymbol{(n-1)\ordinalst}$ infinitesimal thickening} of $f^{-1}\{y\}$.
	\end{alphanumerate}
\end{rem}
Using Remark~\reff{rem:fibres}, Theorem~\reff{thm:FormalFunctions} can be restated as follows.
\begin{varthm}{thm}\lbl{thm:FormalFunctions2}
	Let $f\colon X\morphism Y$ be a proper morphism between locally noetherian\footnote{Franke only assumes $Y$ to be locally noetherian, but $f$ being of (locally) finite type implies that $X$ is locally noetherian as well by Hilbert's Basissatz. This happens multiple times throughout the text.} preschemes. Let $\Ff$ be a coherent $\Oo_X$-module. For every $y\in Y$ let $X_n=X\times_Y\Spec (\Oo_{Y,y}/\mm_{Y,y}^n)$ be the $(n-1)\ordinalst$ infinitesimal thickening of $f^{-1}\{y\}$. Then there is an isomorphism
	\begin{align*}
		(R^pf_*\Ff)_y^\complete\isomorphism\limit[n\geq 1] H^p(X_n,\Ff|_{X_n})\;,
	\end{align*}
	where $\roof{\phantom{x}}$ denotes the $\mm_{Y,y}$-adic completion.
\end{varthm}
\begin{proof}
	We may assume that $Y=\Spec A$ is affine, and that $A$ is a noetherian ring. Indeed, replacing $Y$ by an affine neighbourhood $U\cong \Spec A$ and $X$ by $f^{-1}(U)$ doesn't change $(R^pf_*\Ff)_y$ (because the construction of $R^pf_*\Ff$ is base-local by \cite[Proposition~1.5.1]{alggeo2}) and also $X_n$ is preserved since $f^{-1}\{y\}$ is already contained in $f^{-1}(U)$.
	
	In this case, $R^pf_*\Ff=H^p(X,\Ff)^\qcmod$ by \cite[Proposition~1.5.1\itememph{d}]{alggeo2}. Let $\pp\in\Spec A$ be the prime ideal associated to $y$. Then $(R^pf_*\Ff)_y\cong H^p(X,\Ff)_\pp$ and $\Oo_{Y,y}\cong A_\pp$ is flat over $A$. Let $\mm=\pp A_\pp\cong \mm_{Y,y}$ be its maximal ideal. We denote $\pi\colon \Spec A_\pp\morphism \Spec A$. By flat base change, which is unfortunately discussed way in the future in Section~\reff{sec:flatBaseChange} (but we have already seen it in \cite[Fact~4.1.1]{alggeo2}) we get
	\begin{align*}
		H^p(X\times_Y\Spec A_\pp,\pi^*\Ff)\cong H^p(X,\Ff)_\pp\cong (R^pf_*\Ff)_y\;.
	\end{align*}
	Also $(X\times_Y\Spec A_\pp)\times_{\Spec A_\pp}\Spec(A_\pp/\mm^n)\cong X\times_Y\Spec(A_\pp/\mm^n)\cong X_n$ by a bit abstract nonsense. Now Theorem~\reff{thm:FormalFunctions} may be applied to $X\times_Y\Spec A_\pp\morphism \Spec A_\pp$ (the base change of $f$) and the assertion follows.
\end{proof}
\section{Application to Zariski's main theorem}
\subsection{Quasi-finite morphisms and a lot of (not necessarily main) theorems by Zariski}
Out there in the real world, there are multiple \emph{main theorems} of Zariski around, and usually they're only loosely related. Professor Franke recommends Mumford's \emph{The red book of varieties and schemes} for a discussion of various such version.
\begin{cor}\lbl{cor:Rpvanishing}
	Let $f\colon X\morphism Y$ be any proper morphism between locally noetherian preschemes and let $d=\sup_{y\in Y}\dim \left(f^{-1}\{y\}\right)$. If $\Ff$ is a coherent $\Oo_X$-module and $p>d$, then $R^pf_*\Ff=0$.
\end{cor}
\begin{proof}
	Since $R^pf_*\Ff$ is coherent (this is \cite[Theorem~5]{alggeo2}), $(R^pf_*\Ff)_y$ is a finitely generated $\Oo_{Y,y}$-module, hence it vanishes iff it $\mm_{Y,y}$-adic completion vanishes by Fact~\reff{fact:completion101}\itememph{b}. But $X_n=X\times_Y\Spec (\Oo_{Y,y}/\mm_{Y,y}^n)$ has underlying space $f^{-1}\{y\}$ (as explained in Remark~\reff{rem:fibres}\itememph{b}), hence $H^p(X_n,\Ff|_{X_n})=0$ when $p>d$ by Grothendieck's theorem on cohomological dimension (cf.\ \cite[Proposition~1.4.1]{alggeo2}). The assertion now follows from Theorem~\reff{thm:FormalFunctions2}.
\end{proof}
\begin{defi}\lbl{def:quasiFinite}
	A morphism $f\colon X\morphism Y$ of finite type is called \defemph{quasi-finite at $\boldsymbol{x\in X}$} if $x$ is discrete in its fibre, i.e., if $\{x\}$ is an open and closed subset of $f^{-1}\{y\}$ where $y=f(x)$. We call $f$ \defemph{quasi-finite} if it is quasi-finite at every $x\in X$. 
\end{defi}
The following fact wasn't mentioned in the lecture, but it's \emph{definitely} (in particular, not only \emph{perhaps}, if you get what I mean) \emph{worthwhile to know}!
\begin{fact}\lbl{fact:openDiscrete}
	Let $f\colon X\morphism Y$ be a morphism of finite type. Let $x\in X$ be open in its fibre $f^{-1}\{y\}$, where $y=f(x)$. Then $f$ is already quasi-finite at $x$. 
\end{fact}
\begin{proof}
	Choose an affine open neighbourhood $y\in U\cong \Spec A$. Then $f^{-1}\{y\}$ is contained in $f^{-1}(U)$, so we may w.l.o.g.\ assume that $Y=\Spec A$ is affine. Put $k=\KK(y)$. Since $X$ may be covered by affine open subsets $\Spec R$, where $R$ is of finite type over $A$, we may cover the fibre product $f^{-1}\{y\}=X\times_Y\Spec k$ by affine open subsets $\Spec (R\otimes_A k)$, in which $R\otimes_A k$ is a $k$-algebra of finite type, hence a Jacobson ring. This proves that $f^{-1}\{y\}$ is a Jacobson prescheme as in \cite[Definition~2.4.2]{alggeo1}. But then $x$ is a closed point of the open subset $\{x\}\subseteq f^{-1}\{y\}$, hence also a closed point of $f^{-1}\{y\}$ by \cite[Fact~2.4.1\itememph{c}]{alggeo1}.
\end{proof}
\begin{fact}
	\begin{alphanumerate}
		\item \lbl{fact:annoyingQF}Any finite morphism is quasi-finite.
		\item If $k$ is a field, a morphism $f\colon X\morphism\Spec k$ of finite type is quasi-finite iff it is finite.
		\item Let $f\colon X\morphism Y$ and $g\colon Y\morphism Z$ be morphisms of finite type such that $g$ is quasi-finite at $y=f(x)$ for some $x\in X$. Then $gf$ is quasi-finite at $x$ iff $f$ is quasi-finite at $x$.
		\item Any base change of a quasi-finite morphism is quasi-finite again.
	\end{alphanumerate}
\end{fact}
\begin{proof}
	Maybe that's my bad, but the proof of this is actually annoyingly laborious (if you find a shorter one, feel invited to share it and I'll put it in these notes). We begin with part~\itememph{a}. Let $f\colon X\morphism Y$ be a finite morphism, $x\in X$ and $y=f(x)$. Then the morphism
	\begin{align*}
		f^{-1}\{y\}=X\times_Y\Spec \KK(y)\morphism \Spec \KK(y)
	\end{align*}
	is finite again, as a base changes of finite morphisms are finite again (cf.\ \cite[Corollary~2.5.1]{alggeo1}). Letting $k=\KK(y)$ this puts us in the situation from \itememph{b}, so it's sufficient to prove \itememph{b}.
	
	In the case of \itememph{b} we have $f^{-1}\{y\}=X$, so what we need to show is that $X$ carries the discrete topology if $f$ is finite. We know that $X\cong \Spec R$ where $R$ is some finite-dimensional $k$-algebra (using finiteness of $f$). For $x\in X$ let $\pp$ be the corresponding prime ideal of $R$. Then $R/\pp$ is a domain and a finite-dimensional $k$-vector space, hence a finite field extension of $k$. This means that $\pp$ is a maximal ideal of $R$. Consequently, all points of $X$ are closed, so it suffices to show that $X$ has finitely many points. Let $\{x_1,\ldots,x_n\}$ be any finite subset of $X$ and $\{\mm_1,\ldots,\mm_n\}$ the corresponding maximal ideals of $R$. For every $i$, we may choose an element $\alpha_i\in\mm_i$ which is not contained in any $\mm_j$ for $j\neq i$ (e.g.\ by the prime avoidance lemma, cf.\ \cite[Lemma~2.5.1]{alg1}). Put $\beta_i=\prod_{j\neq i}\alpha_j$ (so that $\beta_i\in\mm_j$ for all $j\neq i$ but $\beta_i\notin \mm_i$). We claim that $\beta_1,\ldots,\beta_n$ are $k$-linearly independent. Indeed, if $\lambda_1\beta_1+\ldots+\lambda_n\beta_n=0$ for some coefficients $\lambda_1,\ldots,\lambda_n\in k$, then reducing modulo $\mm_i$ gives $\lambda_i\beta_i=0$ in $R/\mm_i=\KK(\mm_i)$. But $\beta_i\neq 0$ in $\KK(\mm_i)$, so $\lambda_i=0$ for all $i=1,\ldots,n$. This proves $\dim_kR\geq n$. But $R$ is finite-dimensional over $k$, hence $X$ must have finitely many points, as claimed.
	
	Conversely, assume that $f\colon X\morphism\Spec k$ is quasi-finite. Then $X$ is discrete, so it must have finitely many points. Indeed, $f$ being of finite type implies it is quasi-compact (by definition), so $X$ is quasi-compact because $\Spec k$ is, and any discrete quasi-compact space is finite. Let $X=\{x_1,\ldots,x_n\}$. Every point $x_i\in X$ together with the restriction $\Oo_X|_{\{x_i\}}$ of the structure sheaf is a prescheme again, hence affine (because $x_i\in\{x_i\}$ must have an affine neighbourhood). Let $\{x_i\}\cong \Spec R_i$. Then
	\begin{align*}
		X\cong\coprod_{i=1}^n\Spec R_i\cong\Spec\bigg(\bigoplus_{i=1}^nR_i\bigg)
	\end{align*}
	is affine. This shows that $f$ is affine, but finiteness is yet to prove. Clearly, it suffices that each $R_i$ is a finite-dimensional $k$-vector space. Note that $R_i$ has precisely one prime ideal $\mm_i$ (corresponding to $x_i$), which is then automatically maximal. Since $f$ is of finite type, $R_i$ has finite type over $k$. In particular $R_i$ is noetherian and we may choose generators $r_1,\ldots,r_m$ of $\mm_s$. Since $\mm_i$ is the only prime ideal of $R_s$, we have $\mm_i=\nil R_i$. Consquently, there is an $N\in\IN$ such that $r_\ell^N=0$ for all $\ell$. Moreover, $R_i/\mm_i$ is a field extension of finite type over $k$, hence a finite field extension by Hilbert's Nullstellensatz. Let $\beta_1,\ldots,\beta_d\in R_i$ be elements whose images modulo $\mm_i$ form a $k$-basis of $R_i/\mm_i$. Then it is straightforward to check that $R$ is generated as a $k$-vector space by the elements
	\begin{align*}
	\beta_j\cdot r_1^{e_1}r_2^{e_2}\cdots r_n^{e_n}\quad\text{where }0\leq e_\ell<N\text{ for all }\ell\;.
	\end{align*}
	This shows $\dim_kR<\infty$, hence $f$ is finite.
	
	Part \itememph{c}. Since $g$ is quasi-finite at $y$, the subset $\{y\}\subseteq g^{-1}\{g(y)\}$ is open and closed, hence $f^{-1}\{y\}\subseteq (gf)^{-1}\{g(y)\}$ is open and closed. This means that $\{x\}$ is open and closed in the fibre $(gf)^{-1}\{g(y)\}$  iff it is open and closed in $f^{-1}\{y\}$ and we win.
	
	Part~\itememph{d}. Let $f\colon X\morphism S$ be quasi-finite and $g\colon Y\morphism S$ another morphism of preschemes. The base change $\snake{f}\colon \snake{X}=X\times_SY\morphism Y$ is of finite type by \cite[Fact~2.2.2]{alggeo1}. Now let $y\in Y$ and put $s=g(y)$. By \itememph{b} we see that $X\times_S\Spec \KK(s)$ is finite over $\Spec \KK(s)$. Then
	\begin{align*}
		\snake{X}\times_Y\Spec\KK(y)&\cong X\times_S\big(\Spec\KK(s)\times_S\Spec\KK(y)\big)\\
		&\cong \big(X\times_S\Spec \KK(s)\big)\times_{\Spec \KK(s)}\Spec \KK(y)
	\end{align*}
	is finite over $\Spec\KK(y)$, so $\snake{f}$ is indeed quasi-finite by \itememph{b} again.
\end{proof}
\begin{thm}[Grothendieck's version of Zariski's main theorem]
	\begin{alphanumerate}
		\item \lbl{thm:ZariskiMain}Let $f\colon X\morphism Y$ be a quasi-finite proper morphism between locally noetherian preschemes. Then $f$ is finite.
		\item Let $f\colon X\morphism Y$ be a quasi-finite and separated morphism between noetherian preschemes. Then there exists a factorization $X\monomorphism[j]\ov{X}\morphism[g]Y$ of $f$ where $j$ is an open immersion and $g$ is finite.
		\item If $f\colon X\morphism Y$ is any morphism of finite type between locally noetherian preschemes, then 
		\begin{align*}
			U=\left\{x\in X\st f\text{ is quasi-finite at }x\right\}
		\end{align*}
		is open in $X$, and the restriction $f|_U$ is quasi-finite (by definition).
	\end{alphanumerate}
\end{thm}
\begin{rem}
	If $\Pp$ is a property of morphisms of preschemes, then a \emph{quasi-$\Pp$} morphism usually means an open immersion followed by a morphism which has property $\Pp$. So Theorem~\reff{thm:ZariskiMain}\itememph{b} essentially says that -- under the right circumstances -- quasi-finiteness as in Definition~\reff{def:quasiFinite} coincides with our expectations.
\end{rem}
\begin{proof}[Proof of Theorem~\reff{thm:ZariskiMain}\itememph{a}]
	For \itememph{a} we may assume that $Y=\Spec A$ is affine as all involved properties are base-local. Let $\Jj\subseteq \Oo_X$ be a sheaf of ideals, then $\Jj$ is coherent as $X$ is locally noetherian. Since $f$ is quasi-finite, all fibres carry the discrete topology. In particular, they are zero-dimensional and Corollary~\reff{cor:Rpvanishing} shows that $R^1f_*\Jj=0$. Then also $0=\Gamma(Y,R^1f_*\Jj)=H^1(X,\Jj)$ (using \cite[Proposition~1.5.1\itememph{d}]{alggeo2}), hence $X$ is affine by Serre's affinity criterion. This shows that $f$ is affine. Moreover, $f_*\Oo_X$ is a coherent $\Oo_Y$-module by \cite[Theorem~5]{alggeo2}, hence $f$ is finite.
\end{proof}
 The proof of \itememph{b} is hard and occupies most of Subsection~\reff{subsec:discussionOfThm3}. Part \itememph{c} is easier, but postponed as well until we have the required theory available. To warm up for Theorem~\reff{thm:ZariskiMain}\itememph{b} and \itememph{c}, we're going to prove more of Zariski's theorems!
\begin{thm}[Zariski's connectedness theorem]\lbl{thm:ZariskiConnectedness}
	Let $f\colon X\morphism Y$ be a proper morphism between locally noetherian schemes, whose algebraic component $f^*\colon \Oo_Y\morphism f_*\Oo_X$ is an isomorphism.
	\begin{alphanumerate}
		\item The fibres $f^{-1}\{y\}$ are connected for all $y\in Y$.
		\item The set
		\begin{align*}
			U=\left\{x\in X\st \{x\}=f^{-1}\{f(x)\}\right\}=\left\{x\in X\st f\text{ is quasi-finite at }x\right\}
		\end{align*}
		is open in $X$, and the restriction $f|_U$ is quasi-finite (by definition).
	\end{alphanumerate}
\end{thm}
\begin{proof}
	Part \itememph{a}. Assume $f^{-1}\{y\}$ is disconnected, say, $f^{-1}\{y\}=U_1\cup U_2$ for disjoint non-empty open subsets $U_1,U_2\subseteq f^{-1}\{y\}$. Since all infinitesimal thickenings $X_n=X\times_Y\Spec(\Oo_{Y,y}/\mm_{Y,y}^n)$ have underlying topological space $f^{-1}\{y\}$, we see that the $X_n$ are disconnected too. Hence for all $n\geq 1$ there is a unique $\epsilon_n\in\Oo_{X_n}(X_n)=H^0(X_n,\Oo_{X_n})$ such that $\epsilon_n|_{U_1}=0$ and $\epsilon_n|_{U_2}=1$. The sequence $(\epsilon_n)_{n\geq 1}$ clearly defines an element $\epsilon$ of
	\begin{align*}
		\limit[n\geq 1]H^0(X_n,\Oo_{X_n})\cong (f_*\Oo_X)_y^\complete\cong \roof{\Oo}_{Y,y}\;.
	\end{align*}
	The left isomorphism here is due to Theorem~\reff{thm:FormalFunctions2} and the fact that $\Oo_{X_n}=\Oo_X|_{X_n}$, and the right one holds by assumption. Hence $\roof{\Oo}_{Y,y}$ is a local ring (by Corollary~\reff{cor:completionLocal}) with an idempotent $\epsilon\neq 0,1$. Then $1-\epsilon\neq 0,1$ is another non-trivial idempotent. Both $\epsilon$ and $1-\epsilon$ can't be units in $\roof{\Oo}_{Y,y}$, otherwise $\epsilon^2=\epsilon$ implies $\epsilon=1$ (and similar for $1-\epsilon$). But then they are elements of the maximal ideal $\mm$, so $\epsilon+(1-\epsilon)=1$ is an element of $\mm$ as well, contradiction!
	
	Part \itememph{b}. By \itememph{a}, any point $x\in X$ is open and closed in its fibre iff $f^{-1}\{f(x)\}=\{x\}$. Therefore the two definitions of $U$ indeed coincide. 
	
	We must show that $U$ is open. This is a local question with respect to $Y$, hence we may assume that $Y=\Spec A$ is affine. Let $x\in U$ and $V\subseteq X$ an affine open neighbourhood of $x$. Put $Z=X\setminus V$. Then $Z\subseteq X$ is closed and disjoint from $f^{-1}\{f(x)\}=\{x\}$. As $f$ is proper, $Z'=f(Z)\subseteq Y$ is closed, and $y=f(x)\notin Z'$. There's an $\alpha\in A$ such that $y\notin V(\alpha)$ and $V(\alpha)\supseteq Z'$. Let $Y_1=Y\setminus V(\alpha)$. Note that $Y_1\cong \Spec A_\alpha$ is affine and $x\in X_1=f^{-1}(Y_1)\subseteq V$. Then $X_1=X\setminus V(f^*\alpha)=V\setminus V(f^*\alpha)$ is affine as well, so the restriction $f|_{X_1}\colon X_1\morphism Y_1$ of $f$ is affine and proper. But every affine proper morphism is finite (because $f|_{X_1,*}\Oo_{X_1}$ is a coherent $\Oo_{Y_1}$-module by \cite[Theorem~5]{alggeo2}), so $f|_{X_1}$ is, in particular, quasi-finite by Fact~\reff{fact:annoyingQF} and $U\cap X_1=X_1$. This proves that $U$ is open.
\end{proof}
\begin{rem}
	On first glance, the argument from Theorem~\reff{thm:ZariskiConnectedness}\itememph{b} might look like it proves that every proper morphism is affine, but what it actually shows is the following: If $f\colon X\morphism Y$ is a proper morphism such that for each $x\in X$ the fibre $f^{-1}\{f(x)\}$ is contained in some affine subset $V\subseteq X$, then $f$ is already affine (and hence finite). 	
\end{rem}
\begin{rem}
	Recall that a prescheme $X$ is called \defemph{normal} if it is integral and all local rings $\Oo_{X,x}$ (which are domains if $X$ is integral) are normal (cf.\ \cite[Definition~2.4.5]{alggeo1}). This is the case iff $\Oo_X(U)$ is a normal domain for all affine $U\subseteq X$, cf.\ the discussion in \cite[Remark~2.5.1]{alggeo1}.
\end{rem}
\begin{cor}[Zariski's birationality theorem]\lbl{cor:ZariskiBirationality}
	Let $f\colon X\morphism Y$ be a proper morphism between locally noetherian preschemes, where $Y$ is normal. Suppose that $f$ is \defemph{birational} in the sense that there is a dense open subset $U\subseteq Y$ such that the restriction $f|_{f^{-1}(U)}\colon f^{-1}(U)\isomorphism U$ is an isomorphism and $f^{-1}(U)$ is dense in $X$. Then all assertions from Theorem~\reff{thm:ZariskiConnectedness} apply to $f$. In particular, $f$ has connected fibres.
\end{cor}
\begin{proof}
	First note that $U$ is irreducible as an open subset of the irreducible space $Y$ (irreducibility of $Y$ is implied by $Y$ being normal). Hence $X$ is irreducible because it has the dense irreducible subset $f^{-1}(U)\cong U$. Let $\Spec A\cong V\subseteq Y$ be an affine open subset, where $A$ is a domain. Then $f^{-1}(V)$ is open in $X$, hence dense in $X$ and thus irreducible. Since $U$ is dense in $Y$, the intersection $U\cap V$ is non-empty, hence $f^{-1}(U\cap V)\subseteq f^{-1}(V)$ is a non-empty open subset and thereby dense again. This shows that we can actually reduce to the case $Y=\Spec A$ (all the other involved properties are clearly base-local). Moreover, we may assume that $X$ is integral. Indeed, the assertions from Theorem~\reff{thm:ZariskiConnectedness} are purely topological, so we may replace $X$ by its reduction $X^\red=V(\nil(\Oo_X))$ to obtain an $X$ which is irreducible and reduced (hence integral) and has the same underlying topological space as the original one.
	\begin{claim}\lbl{claim:quotientField}
		The ring $B=\Oo_X(X)$ is a domain in the above situation, and $A$ and $B$ have the same field of quotients $K$. Moreover, we have $A\subseteq B$ as subrings of $K$.
	\end{claim}
	Believing this for the moment, the proof can be finished as follows. Since $B$ is finitely generated as an $A$-module (because $f_*\Oo_X=\snake{B}$ is coherent by \cite[Theorem~5]{alggeo2}), it is integral over $A$. But $A$ is integrally closed in $K$, hence $A\subseteq B$ implies $A=B$. We conclude $f_*\Oo_X\cong \Oo_Y$, as needed.
	
	Unfortunately, the proof of Claim~\reff{claim:quotientField} wasn't discussed in the lecture, but I think it should have been. Since $X$ and $Y$ are irreducible, they have unique generic points $\eta_X$ and $\eta_Y$. As $\eta_Y$ is dense in $Y$, we have $\eta_Y\in U$ and similarly $\eta_X\in f^{-1}(U)$. Hence $f(\eta_X)=\eta_Y$ and the induced morphism $\Oo_{Y,\eta_Y}\isomorphism\Oo_{X,\eta_X}$ is an isomorphism by the birationality assumption. Moreover, $\eta_Y$ corresponds to $0\in\Spec A$, hence $\Oo_{Y,\eta_Y}\cong K$ is the quotient field of $A$. So we should prove that $\Oo_{X,\eta_X}$ is the quotient field of $B=\Oo_X(X)$ as well.
	
	It's clear that $B$ is a domain because $X$ is integral (cf. \cite[Proposition~2.1.4\itememph{b}]{alggeo1}). Since $U\subseteq \Spec A$ is open, we find an affine open subset $V=\Spec A\setminus V(\alpha)\subseteq U$. Then 
	\begin{align*}
		f^{-1}(V)=X\setminus V(f^*\alpha)=f^{-1}(U)\setminus V(f^*\alpha)\cong V
	\end{align*}
	is affine again by birationality of $f$. We know that $X$ is quasi-compact and separated since so are $f$ and $\Spec A$. In particular, \cite[Proposition~1.5.1\itememph{c}]{alggeo1} is applicable to $\Oo_X$ and gives 
	\begin{align*}
		\Oo_X(f^{-1}(V))=\Oo_X\big(X\setminus V(f^*\alpha)\big)\cong\Oo_X(X)_{f^*\alpha}\;,
	\end{align*}
	so $B$ and $\Oo_X(f^{-1}(V))$ have the same quotient field. But $\Oo_X(f^{-1}(V))\cong \Oo_Y(V)\cong A_\alpha$ has quotient field $K$, so we win.
	
	The fact that $A\subseteq B$ as subrings of $K$ follows from the commutative diagram
	\begin{diagram*}
		\object{2,0}{$\Oo_{X,\eta_X}$}[a];
		\object{0,0}{$\Oo_{Y,\eta_Y}$}[b];
		\object{2,1.5}{$B$}[c];
		\object{0,1.5}{$A$}[d];
		\scriptsize
		\arrow dc;
		\arrow[right hook->]ca;
		\arrow[right hook->]db;
		\isoarrow ba;
	\end{diagram*}
	in which every arrow except the top one is injective, hence $A\morphism B$ is injective as well.
\end{proof}
\begin{fact}\lbl{fact:steinFactorization}
	Every proper morphism $f\colon X\morphism Y$ between locally noetherian preschemes can be factorized as
	\begin{align*}
		f\colon X\morphism[\snake{f}]\SPEC(f_*\Oo_X)\morphism[g]Y\;.
	\end{align*}
	In this composition, $g$ is finite and the assumptions of Theorem~\reff{thm:ZariskiConnectedness} hold for $\snake{f}$.
\end{fact}
\begin{proof}[Sketch of a proof]
	It's pretty obvious that this factorization exists (to construct $\snake{f}$, use the adjunction from \cite[Proposition~1.6.2\itememph{b}]{alggeo2}). To show that $\snake{f}$ and $g$ have the required properties, we look at things locally and assume that $Y=\Spec A$ is affine (and $A$ is noetherian). Then the facorization looks like
	\begin{align*}
		f\colon X\morphism[\snake{f}]\Spec \Oo_X(X)\morphism[g]\Spec A\;,
	\end{align*}
	so $g$ is affine. Moreover, $\Oo_X(X)$ is a finitely generated $A$-module because $f_*\Oo_X$ is a coherent $\Oo_Y$-module (by \cite[Theorem~5]{alggeo2}, as usual), so $g$ is actually finite. Also proving that $\snake{f}_*\Oo_X=\Oo_{\Spec \Oo_X(X)}$ is straightforward, so it remains to show that $\snake{f}$ is proper. But $g$ is finite, hence separated, and $g\snake{f}=f$ is proper, so $\snake{f}$ is proper as well by the two-out-of-three property from \cite[Proposition~2.4.1]{alggeo2}.
\end{proof}
\begin{rem}
	If $f\colon X\morphism Y$ is a quasi-compact and quasi-separated morphism of preschemes (we need these two conditions for $f_*\Oo_X$ to be quasi-coherent, cf.\ \cite[Proposition~1.5.2\itememph{b}]{alggeo1}). Then there always is a factorization
	\begin{align*}
		f\colon X\morphism\SPEC(f_*\Oo_X)\morphism Y
	\end{align*}
	as in Fact~\reff{fact:steinFactorization} (but it usually doesn't have as convenient properties as above). This is called the \defemph{Stein factorization} of $f$.
\end{rem}
\subsection{Proof of Zariski's main theorem}
\label{subsec:discussionOfThm3}
\begin{proof}[Proof of Theorem~\reff{thm:ZariskiMain}\itememph{c}]
	Note that Theorem~\reff{thm:ZariskiMain}\itememph{c} is actually a generalization of Theorem~\reff{thm:ZariskiConnectedness}\itememph{b}. We will reduce it to this special case.	Let's first assume that $f\colon X\morphism Y$ factors over
	\begin{align}\lbl{eq:properCompactification}
		f\colon X\monomorphism[j]\ov{X}\morphism[\smash{\ov{f}}]Y\;,
	\end{align}
	where $j$ is an open immersion and $\ov{f}$ is proper. Then
	\begin{align}\lbl{eq:fandfbar}
		\left\{x\in X\st f\text{ is quasi-finite at }x\right\}=X\cap \left\{x\in \ov{X}\st \ov{f}\text{ is quasi-finite at }x\right\}\;.
	\end{align}
	Indeed, a point $x\in X$ is open in $\ov{f}^{-1}\{y\}$ (where $y=f(x)$) iff it is open in the open subset $f^{-1}\{y\}=X\cap \ov{f}^{-1}\{y\}\subseteq \ov{f}^{-1}\{y\}$. In view of Fact~\reff{fact:openDiscrete} this shows \eqreff{eq:fandfbar}. We thus have reduced \itememph{c} (under the assumption that $\ov{f}$ exists) to the case of proper morphisms. 
	
	If $f\colon X\morphism Y$ is proper, then consider its Stein factorization $f=g\circ \snake{f}$. Since $g$ is finite, it's quasi-finite as well by Fact~\reff{fact:annoyingQF}\itememph{a}. So Fact~\reff{fact:annoyingQF}\itememph{c} shows that
	\begin{align*}
		\left\{x\in X\st f\text{ is quasi-finite at }x\right\}=\left\{x\in X\st \snake{f}\text{ is quasi-finite at }x\right\}\;.
	\end{align*}
	But the right-hand side is open in $X$ by Fact~\reff{fact:steinFactorization} and Theorem~\reff{thm:ZariskiConnectedness}\itememph{b} and we're happy!
	
	Note that such an $\ov{f}$ as in \eqreff{eq:properCompactification} always exists when $X$ and $Y$ are affine. Indeed, if $X$ has finite type over $Y$ and both are affine, we get a closed embedding $X\monomorphism\IA_Y^n$ for some $n\in \IN$. Together with the open embedding $\IA_Y^n\monomorphism\IP_Y^n$ this makes $X$ a closed subprescheme of an open subprescheme of $\IP_Y^n$. But then $X$ is also an open subprescheme of some closed subprescheme $\ov{X}\subseteq \IP_Y^n$ (here we use \cite[\stackstag{01R8}]{stacks-project}). This gives a factorization
	\begin{align*}
		f\colon X\monomorphism[j]\ov{X}\morphism[\smash{\ov{f}}]Y
	\end{align*}
	in which $\ov{f}\colon \ov{X}\monomorphism \IP_Y^n\morphism Y$ is (strongly) projective, hence proper by \cite[Proposition~2.4.2]{alggeo2}. But \itememph{c} is completely local on both $X$ and $Y$ (thanks to Fact~\reff{fact:openDiscrete}), so by checking the affine case we have actually covered all of \itememph{c}. 
\end{proof}
	We will now sketch a proof of Theorem~\reff{thm:ZariskiMain}\itememph{b}, which occupies the rest of the subsection. The proof is split into three parts. In Part~I, we will introduce the notion of quasi-affine morphisms and reduce Zariski's main theorem to the question whether quasi-finite separated morphisms are quasi-affine. In Part~II, we show that Zariski's main theorem holds when $Y$ is the spectrum of a complete noetherian local ring. Finally, Part~III shows how the general case can be reduced to Part~II.
	
	So much for the battle plan, now let's get into action!
	
	\paragraph{Part~I -- proving quasi-affinity is sufficient.} We begin with the observation that things become much easier when we are proper.
	\begin{lem}\lbl{lem:ZMTLemma1}
		Let $f\colon X\morphism Y$ be a quasi-finite separated morphism between noetherian preschemes, and assume that $f$ factors as in \eqreff{eq:properCompactification}. Then Zariski's main theorem holds for $f$.
	\end{lem}
\begin{rem}
	 It can be shown (but this is hard) that such a factorization \eqreff{eq:properCompactification} always exists for morphisms of finite type between noetherian preschemes, for which Professor Franke refers to notes of \emph{Brian Conrad} or \emph{Paul Vojta}, although he isn't sure whether using their results to prove Theorem~\reff{thm:ZariskiMain}\itememph{b} doesn't involve any circular reasoning.
\end{rem}
\begin{proof}[Proof of Lemma~\reff{lem:ZMTLemma1}]
	Let $\ov{f}=g\circ \snake{f}$ be the Stein factorization of $\ov{f}$. Put $\ov{Y}=\SPEC(\ov{f}_*\Oo_{\ov{X}})$ for convenience. Since $g\colon \ov{Y}\morphism Y$ is finite by Fact~\reff{fact:steinFactorization} (for which we need properness of $\ov{f}$), we're done if we show that the composition $\snake{f}|_X\colon X\monomorphism \ov{X}\morphism \ov{Y}$ is an open embedding. Since $f=g\snake{f}|_X$ and $g$ are quasi-finite ($g$ is even finite), Fact~\reff{fact:annoyingQF}\itememph{c} shows that $\snake{f}|_X$ is quasi-finite as well.
	
	In particular $\snake{f}|_X$ is injective, and for all $x\in X$ we have $\snake{f}^{-1}\{\snake{f}(x)\}=\{x\}$, because $\Oo_{\ov{Y}}\cong \snake{f}_*\Oo_{\ov{X}}$ (by Fact~\reff{fact:steinFactorization}), so the argument from Theorem~\reff{thm:ZariskiConnectedness}\itememph{b} can be applied. If $V$ is any open neighbourhood of $x\in X$, then $\snake{f}(\ov{X}\setminus V)\subseteq \ov{Y}$ is closed because $\snake{f}$ is proper (by Fact~\reff{fact:steinFactorization}), hence closed. Moreover, $\snake{f}(\ov{X}\setminus V)$ doesn't contain $\snake{f}(x)$ as $\snake{f}^{-1}\{\snake{f}(x)\}=\{x\}$. Thus, the complement $U$ of $\snake{f}(\ov{X}\setminus V)$ in $\ov{Y}$ is open and $\snake{f}^{-1}(U)\subseteq V$ is an open neighbourhood of $x$. This shows that $\snake{f}|_X$ is actually an open map! Together with $\Oo_{\ov{Y}}\cong \snake{f}_*\Oo_{\ov{X}}$ we see that $\snake{f}|_X$ is an open embedding, as claimed.
\end{proof}
\begin{defi}\lbl{def:quasiAffine}
	A morphism $f\colon X\morphism Y$ of preschemes is called \defemph{quasi-affine}, if it satisfies the following equivalent conditions:
	\begin{alphanumerate}
		\item For all affine open $U\subseteq Y$, $f^{-1}(U)$ is a quasi-affine scheme (that is, a quasi-compact open subscheme of an affine scheme).
		\item $Y$ can be covered by affine open $U$ such that $f^{-1}(U)$ is quasi-affine.
		\item $f$ factors as $X\monomorphism[j]\ov{X}\morphism[\smash{\ov{f}}]Y$, where $j$ is an open embedding and $\ov{f}$ is affine.
		\item $f$ is quasi-compact and quasi-separated and the canonical morphism $X\morphism\SPEC(f_*\Oo_X)$ (which also appears in the Stein factorization, but exists in general) is an open immersion.
	\end{alphanumerate}
\end{defi}
For a proof of equivalence, check out \cite[\stackstag{01SJ}]{stacks-project}. Note that Franke apparently had a different proof (for the case of quasi-affine morphisms of finite type) in mind than the Stacks Project guys, but I have no idea how this was going to work (whereas the Stacks Project proof is pretty clear to me and works without restrictions on $f$). Nevertheless, we are going to need the following lemma he suggested.
\begin{lem}\lbl{lem:coherentExtension}
	Let $X$ be a noetherian prescheme, $\Mm$ a quasi-coherent $\Oo_X$-module, and $U\subseteq X$ an open subset. Let $\Nn\subseteq \Mm|_U$ be a coherent submodule. Then there is a coherent $\snake{\Nn}\subseteq \Mm$ such that $\snake{\Nn}|_U=\Nn$. In particular, $\Mm$ is the union of its coherent submodules.
\end{lem}
\begin{proof}
	We proceed by noetherian induction. Because $Y$ is noetherian, there is a $\subseteq$-maximal open subset $U$ to which $\Nn$ extends (and, with slight abuse of notation, we denote some fixed extension to $U$ by $\Nn$ as well). Assuming $U\neq X$, we will derive a contradiction. Pick $x\in X\setminus U$ and let $V\cong \Spec A$ be an affine open neighbourhood of $x$. 
	
	We claim that there is a coherent $\Qq\subseteq \Mm|_V$ such that $\Qq|_{U\cap V}\supseteq \Nn|_{U\cap V}$. Indeed, $V\cong \Spec A$ is a noetherian topological space (because $Y$ is noetherian), so the open subset $U\cap V$ is quasi-compact. This means that we can cover it by finitely many affine open subsets $V_i\cong \Spec A_{f_i}$ for $i=1,\ldots,n$. For every $i$ we know that $\Nn(V_i)\subseteq \Mm(V_i)\cong \Mm(V)_{f_i}$ is a finitely generated $A_{f_i}$-module (because $\Nn$ is coherent), so we can choose finitely many $\mu_{i,j}\in \Mm(V)$, $j=1,\ldots,k_i$, whose images in $\Nn(V_i)$ form a set of generators. Let $\Qq\subseteq \Mm|_V$ be the subsheaf generated by $\left\{\mu_{i,j}\st i\leq n,\ j\leq k_i\right\}\in\Mm(V)$. Then $\Qq$ has the required property, proving the claim.
	
	Let $j\colon U\cap V\monomorphism V$ be the obvious inclusion which is quasi-compact (as $V$ is noetherian) and separated. Hence $\ov{\Nn}=\Qq\cap j_*\Nn|_{U\cap V}$ is quasi-coherent, using \cite[Proposition~1.5.2\itememph{b}]{alggeo1}. Then $\ov{\Nn}$ is even coherent, being a subsheaf of $\Qq$, and satisfies $\ov{\Nn}|_{U\cap V}=\Nn|_{U\cap V}$. But then $\Nn$ can be extended to a coherent sheaf $\snake{\Nn}$ on $U\cup V$ via $\snake{\Nn}|_U=\Nn$ and $\snake{\Nn}|_V=\ov{\Nn}$. This contradicts maximality of $U$.
\end{proof}
\begin{lem}\lbl{lem:ZMTLemma2}
	If $f\colon X\morphism Y$ is quasi-affine and morphism of finite type between noetherian preschemes, then $f$ factors as in \eqreff{eq:properCompactification}. In particular,  Theorem~\reff{thm:ZariskiMain}\itememph{b} holds for quasi-affine and quasi-finite morphisms.
\end{lem}
\begin{proof}
	The proof I put here before is \emph{horribly wrong}! I hope I can fix it in the following. If you are interested in the more general theory, you may want to have a look at \cite[\S5]{egaII} or \cite[\stackstag{01VV}]{stacks-project} (but I'm not sure if the latter actually proves what we need).
	
	Let $W\cong \Spec A$ be an affine open subset of $Y$. Then $U=f^{-1}(W)$ is an open subset of some affine scheme $\Spec B$, hence $U$ can be covered by affine open subsets of the form $U\setminus V(\beta_i)\cong \Spec B_{\beta_i}$ for $\beta_i\in B$. We can choose a finite cover of this form, using $\beta_1,\ldots,\beta_n$ say, since $U$ is quasi-compact (because $V$ is quasi-compact, and the morphism $f$ is 
	of finite type, hence quasi-compact). Also note that $U$ is separated (being an open subprescheme of the separated prescheme $\Spec B$), so \cite[Proposition~1.5.1\itememph{c}]{alggeo1} shows $B_{\beta_i}\cong \Oo_X(U\setminus V(\beta_i))\cong \Oo_X(U)_{\beta_i}$.
	
	Since $f$ is quasi-finite, it has finite type, hence the $B_{\beta_i}\cong \Oo_X(U)_{\beta_i}$ have finite type over $A$. Then we may choose a finite generating set $\{b_{i,j}\}_j$ of $\Oo_X(U)_{\beta_i}$ over $A$. Without loss of generality we may assume the $b_{i,j}$ are already contained in $\Oo_X(U)$. Let $E$ be the $A$-submodule of $\Oo_X(U)$ generated by the $\{b_{i,j}\}_{i,j}$ for all $i,j$ and $\{\beta_1,\ldots,\beta_n\}$. By Lemma~\reff{lem:coherentExtension} we may extend $E$ to a coherent $\Oo_Y$-submodule $\Ee^{(W)}\subseteq f_*\Oo_X$.
	
	Now since $Y$ is quasi-compact, we may cover it by finitely many affine open subsets $W$. Let $\Ee\subseteq f_*\Oo_X$ be the sum of all $\Ee^{(W)}$. Then $\Ee$ is coherent again. Now consider the graded $\Oo_Y$-algebra
	\begin{align*}
		\Sym_{\Oo_Y}^*(\Ee)=\bigoplus_{d\geq 0}\Sym_{\Oo_Y}^d(\Ee)\;.
	\end{align*}
	That is, $\Sym_{\Oo_Y}^*(\Ee)$ is the symmetric algebra associated to $\Ee$ -- which is, loosely speaking, given by $\bigoplus_{d\geq 0}\Ee^{\otimes d}$ modulo the minimal equivalence relation that turns it into a \emph{commutative} (!) graded $\Oo_Y$-algebra. We put $\IA(\Ee)=\SPEC(\Sym_{\Oo_Y}^*(\Ee))$. Moreover, let $\Rr=(\Sym_{\Oo_Y}^*(\Ee))[T]$ be the polynomial algebra over $\Sym_{\Oo_Y}^*(\Ee)$. We equip $\Rr$ with the grading inherited from $\Sym_{\Oo_Y}^*(\Ee)$ plus the polynomial variable $T$ gets homogeneous degree $1$. Now put $\IP(\Ee)=\PROJ(\Rr)$.
	
	Then $\IA(\Ee)$ is isomorphic to the open subprescheme $\iota\colon \IP(\Ee)\setminus V(T)\monomorphism \IP(\Ee)$. Moreover, we get a canonical morphism $\mu^*\colon \Sym_{\Oo_Y}^*(\Ee)\morphism f_*\Oo_X$ induced by multiplication. This induces a morphism $\mu\colon \SPEC(f_*\Oo_X)\morphism\IA(\Ee)$. Note that $\mu$ need not be a closed immersion since $\mu^*\colon \Gamma(W,\Sym_{\Oo_Y}^*(\Ee))\morphism\Gamma(W,f_*\Oo_X)\cong \Oo_X(U)$ is not necessarily surjective for affine open subsets $W\subseteq Y$ and $U=f^{-1}(U)$ as above. However, it will certainly become surjective after localizing at some $\beta_i$ as above. This shows that $\mu$ is actually a locally closed immersion!
	
	Hence the composition $\mu\circ\iota\colon \SPEC(f_*\Oo_X)\morphism \IP(\Ee)$ is a locally closed immersion too. Since $\mu$ is clearly affine, hence quasi-compact, and $\iota\colon \IA(\Ee)\monomorphism \IP(\Ee)$ is an open immersion of noetherian preschemes, hence quasi-compact too.  Thus we see that the scheme theoretic image of $\mu\circ\iota$ is well-behaved (by \cite[\stackstag{01R8}]{stacks-project}), so $\mu\circ\iota$ may be factored as $\SPEC(f_*\Oo_X)\monomorphism \ov{X}\monomorphism\IP(\Ee)$, in which the first arrow is an open immersion and the second is a closed one. So all in all, $f$ can be factored as 
	\begin{align*}
		f\colon X\monomorphism\ov{X}\morphism Y\;,
	\end{align*}
	in which the first arrow is an open immersion. It remains to show that $\ov{X}$ is proper over $Y$. But $\IP(\Ee)$ is clearly proper over $Y$ by \cite[Proposition~2.4.2]{alggeo2} (actually, $\IP(\Ee)$ is even projective over $Y$), so the closed subprescheme $\ov{X}$ is proper over $Y$ as well. This gives a factorization as in \eqreff{eq:properCompactification}.
	
	To show that Theorem~\reff{thm:ZariskiMain}\itememph{b} applies when $f$ is quasi-finite in addition to being of finite type, first note that quasi-affine morphisms are separated (just use any characterization from Definition~\reff{def:quasiAffine}). Then Lemma~\reff{lem:ZMTLemma1} does the rest.
\end{proof}
So we see that to prove Zariski's main theorem it is sufficient to show that any quasi-finite and separated morphism between noetherian preschemes is quasi-affine.% (which is a weaker assertion than the original one).% (note that this is actually \emph{weaker} than Theorem~\reff{thm:ZariskiMain}\itememph{b}).

\paragraph{Part II -- the case of complete noetherian local rings.} In this part Professor Franke follows \cite[Exposé~VIII.6]{sga1}. Assume $Y=\Spec A$, where $A$ is a noetherian complete local ring (with respect to its maximal ideal $\mm$) and that Zariski's main theorem is true for morphisms $f'\colon X'\morphism Y'$, where in addition to the other assumptions we have $\dim Y'<\dim Y$. Let $s$ be the unique closed point of $A$ (given by $\mm$).

We start with a technical lemma.
\begin{lem}\lbl{lem:connectedComponents}
	Assume we are in the above situation. If $B$ is any finite $A$-algebra, then $\Spec B$ has a decomposition $\Spec B=\bigcup_{i=1}^nU_i$, where the $U_i$ are disjoint open subsets such that the only closed point $s$ of $\Spec A$ has precisely one preimage in each $U_i$.
\end{lem}
\begin{proof}
	By Corollary~\reff{cor:finiteAlgebras}, $B$ has finitely many maximal ideals $\qq_1,\ldots,\qq_n$ and these are precisely the prime ideals over $\mm$. For every $\qq_i$ we will construct an idempotent $e_i\in B$ such that $e_i\in \qq_i$ but $e_i\notin\qq_j$ for $j\neq i$. Then $\qq_i\in V(e_i)$ but $\qq_j\neq V(e_i)$ for $j\neq i$, and $V(e_i)$ is an open and closed subset of $B$ (because $e_i$ is an idempotent), so we see that the $\qq_i$ are contained in distinct connected components of $\Spec B$. But $B$ is noetherian, so $\Spec B$ has finitely many connected components (as pointed out in \cite[Lemma~2.4.2]{alggeo1}). Then every connected component is open and we can construct the required $U_i$ as unions of suitable connected components of $\Spec B$.
	
	Let $\ov{\qq}_i=\qq_i/\mm$ be the prime ideals of $B/\mm B$ (which are automatically maximal). Then the intersection $\bigcap_{j=1}^n\qq_j$ is the nilradical $\nil(B/\mm B)$, so by the Chinese remainder theorem we find an element $\ov{e}_i\in (B/\mm B)/\nil(B/\mm B)$ such that $\ov{e}_i\equiv 0\bmod \ov{\qq}_i$ and $\ov{e}_j\equiv 1\bmod\ov{\qq}_j$ for $j\neq i$. Note that $\ov{e}_i^2=\ov{e}_i$. Since $B/\mm B$ is noetherian, there is an $N\in\IN$ such that $\nil(B/\mm B)^N=0$. Hence $B/\mm B$ is $\nil(B/\mm B)$-adically complete, which means we can lift $\ov{e}_i$ to an idempotent $\snake{e}_i\in B/\mm B$ by Hensel's lemma (cf.\ Proposition~\reff{prop:HenselsLemma}). Now $B$ is $\mm B$-adically complete by Proposition~\reff{prop:modulesComplete}, so using Hensel's lemma once again we can lift $\snake{e}_i$ to an idempotent $e_i\in B$ with the required properties. We're done.
\end{proof}

Now let $f\colon X\morphism Y=\Spec A$ be quasi-finite and separated, and let $x\in f^{-1}\{s\}$. Because the fibre $f^{-1}\{s\}$ is discrete, there is an affine open $U\subseteq X$ such that $U\cap f^{-1}\{s\}=\{x\}$. Zariski's main theorem applies to the affine morphism $f|_U\colon U\morphism \Spec A$, which therefore factors as
\begin{align*}
	U\monomorphism[j]\Spec B\morphism Y
\end{align*}
where $B$ is as above and $j$ is an open embedding. By Lemma~\reff{lem:connectedComponents} (and shrinking $U$ if necessary) we may assume that $B$ has only one closed point $j(x)$, i.e., is local itself. But then the only open subset of $\Spec B$ containing the closed point $j(x)$ is $\Spec B$ itself, hence $j$ is an isomorphism. We have thus found an affine open neighbourhood $U\cong \Spec B$ of $x$ which is finite over $Y$. Then $U\morphism X$ is proper (as $U\morphism Y$ is finite, hence proper, and $f$ is separated, so \cite[Proposition~2.4.1]{alggeo2} applies), so $U$ is also closed. Putting $U=U_1$ and $X\setminus U=X_1$, what we proved is that $X=X_1\amalg U_1$ is the disjoint union of its open subsets $X_1$ and $U_1$.

Iterating this for the remaining preimages\footnote{There are finitely many of them. Indeed, $f^{-1}\{s\}\subseteq X$ is closed (as the preimage of the closed point $s$), hence quasi-compact (because $X$ is noetherian, hence quasi-compact). But the fibre $f^{-1}\{s\}$ is supposed to be discrete, so it must be finite.} of $s$ in $X$ provides a decomposition 
\begin{align*}
	X=X'\amalg \coprod_{i=1}^nU_i
\end{align*}
into disjoint open subsets, where $\coprod_{i=1}^nU_i\cong \coprod_{i=1}^n\Spec \Oo_X(U_i)\cong \Spec\big(\bigoplus_{i=1}^n\Oo_X(U_i)\big)$ is finite over $Y$ and $f'=f|_{X'}\colon X'\morphism Y$ has image in $Y'=Y\setminus \{s\}$, hence the induction assumption applies to $f'\colon X'\morphism Y'$. This means that $f'$ can be written as $f'\colon X'\monomorphism \ov{X}'\morphism Y'$, where $X'$ is an open subprescheme of $\ov{X}'$, which in turn is finite over $Y'$. Still we aren't done yet, as we need something finite over $Y$ rather than $Y'$.
\begin{lem}
	Let $A$ be a noetherian ring and $S\subseteq \Spec A$ an open subprescheme. If $f\colon X\morphism S$ is a finite morphism, then there is a finite $A$-algebra $B$ such that $X$ is an open subprescheme of $\Spec B$ and the diagram
	\begin{diagram*}
		\object{0,1.5}{$\Spec B$}[a];
		\object{2.5,1.5}{$\Spec A$}[b];
		\object{0,0}{$X$}[c];
		\object{2.5,0}{$S$}[d];
		\scriptsize
		\arrow ab;
		\arrow[right hook->] ca;
		\arrow[right hook->] db;
		\arrow cd[above][f];
	\end{diagram*}
	commutes.
\end{lem}
\begin{proof}
	By Lemma~\reff{lem:ZMTLemma2} it is enough to show that $X\morphism\Spec A$ is quasi-affine (which removes the condition that $B$ is finite over $A$ -- this wouldn't be too hard to show by hand, but it's more convenient this way). By Definition~\reff{def:quasiAffine}\itememph{d} it suffices to show that $X\morphism\Spec\Oo_X(X)$ is an open embedding.
	
	Since $\Spec A$ is a noetherian space, its open subset $S$ is quasi-compact. We thus find a finite cover $S=\bigcup_{i=1}^n S_i$, where $S_i\cong \Spec A_{\alpha_i}$ (for some $\alpha_i\in A$) are affine open subschemes. Putting $X_i=f^{-1}(S_i)$ we get an affine open cover $X=\bigcup_{i=1}^nX_i$ of $X$. %, where $X_i\cong \Spec B_i$ for some finite $A_{\alpha_i}$-algebra $B_i$ by finiteness of $f$.
	Note that $X$ is quasi-compact because $S$ and $f$ are. Moreover, $S$ is separated as an open subprescheme of a separated prescheme, hence $X$ is separated as well, as finite morphisms are separated. If we denote the image of $\alpha_i$ under the restriction $A\morphism \Oo_S(S)$ by $\alpha_i$ as well, we see that $S_i=S\setminus V(\alpha_i)$, so $A_{\alpha_i}\cong \Oo_S(S\setminus V(\alpha_i))\cong \Oo_S(S)_{\alpha_i}$. This works because $S$ is quasi-compact and separated, so \cite[Proposition~1.5.1\itememph{c}]{alggeo1} applies. By the same argument applied to $X$, we get $\Oo_X(X_i)=\Oo_X(X\setminus V(f^*\alpha_i))\cong \Oo_X(X)_{f^*\alpha_i}$, so $X_i\cong \Spec \Oo_X(X)_{f^*\alpha_i}$ because the $X_i$ are affine.
	
	Now $X=\bigcup_{i=1}^nX_i\cong \bigcup_{i=1}^n\Spec \Oo_X(X)_{f^*\alpha_i}$ is a union of open subschemes of $\Spec \Oo_X(X)$, hence an open subscheme itself, and we are done.
	%Note that $\Oo_X(X)$ becomes an $A$-algebra via $A\morphism \Oo_S(S)\morphism \Oo_X(X)$. For every $i$ let $x_{i,j}\in B_i$, $j=1,\ldots,k_i$ be generators of $B_i$ as an $A_{\alpha_i}$-module. Multiplying by a suitable power of $f^*\alpha_i$ we can assume that the $x_{i,j}$ are from $\Oo_X(X)$ and integral over $A$ (rather than just integral over $A_{\alpha_i}$). Now let $\Lambda$ be the image of $A$ in $\Oo_X(X)$ and put
	%\begin{align*}
	%	B=\Lambda\left[x_{i,j}\st i\leq n,\ j\leq k_i\right]\;.
	%\end{align*}
	%Then $B$ is finite over $A$ and contains the $f^*\alpha_i$ (because these guys are already in $\Lambda$). Moreover, we have $B_{f^*\alpha_i}=\Oo_X(X)_{f^*\alpha_i}\cong B_i$, because $B$ already contains a set of generators of $B_i$ over $A_{\alpha_i}$. So $X=\bigcup_{i=1}^nX_i$ is a union of open subpreschemes of $\Spec B$, hence an open subprescheme itself and $\Spec B\morphism \Spec A$ clearly has the required properties.
\end{proof}

This finishes the case of complete noetherian local rings.

\paragraph{Part III -- reduction to complete noetherian local rings.} We will show that Zariski's main theorem is valid for (quasi-finite separated) $f\colon X\morphism Y$ with target $Y$ (which is locally noetherian) if for all $y\in Y$ it is valid with $\Spec \roof{\Oo}_{Y,y}$ as target. But before that, some technical preparations need to be done.

We'll give two proofs of the following Lemma~\reff{lem:OpenEmbedding}. The first is probably the one Professor Franke had in mind and uses Chevalley's theorem (Proposition~\reff{prop:Chevalley}, page~\pageref{prop:GenericFreeness}) as well as a result from Section~\reff{sec:TopologicalProperties}. The second proof is the one I came up with and uses only methods we have available so far, but at the cost of not being elegant at all.
\begin{lem}\lbl{lem:OpenEmbedding}
	Let $f\colon X\morphism Y$ be a morphism of locally finite type between locally noetherian preschemes such that $f$ induces an injective map of topological spaces and isomorphisms $f^*\colon \Oo_{Y,f(x)}\isomorphism \Oo_{X,x}$ on stalks. Then $f$ is an open embedding.
\end{lem}
\begin{proof}[First proof]
	It is sufficient to show that $X$ carries the induced topology and is open in $Y$. All we need to do for this is to prove that $f$ is an open morphism. Note that the assumption $\Oo_{X,x}\cong \Oo_{Y,f(x)}$ clearly implies that $f$ is \emph{flat} in the sense of Definition~\reff{def:flat}. Hence $f$ is open by Proposition~\reff{prop:flatGoingDown}.
\end{proof}
\begin{proof}[Second proof]
	 As above, we only need to show that $f$ is open. This will be done by several reduction steps until we arrive at a nice enough situation.
	
	The first of these reductions is that $f$ being an open morphism is a local question both on source and target, so we may assume $X=\Spec B$ and $Y=\Spec A$ to be affine, where $A$ and $B$ are noetherian rings with $B$ of finite type over $A$. Let $Z_1,\ldots,Z_n$ be the irreducible components of $Y$ and let $\pp_i\subseteq A$ be the prime ideals such that $Z_i=V(\pp_i)$. Note that a subset $V\subseteq Y$ is open iff its intersections $V\cap Z_i$ are all open (indeed, in this case $Y\setminus V$ is closed as a finite union of the closed subsets $Z_i\setminus V$). Therefore, it suffices to show that every $x\in X$ has an open neighbourhood $U_i$ such that for all open neighbourhoods $U\subseteq U_i$ of $x$ the intersection $f(U)\cap Z_i$ is open (because then $f(U)\subseteq X$ is an open neighbourhood of $f(x)$ for $U=U_1\cap \ldots\cap U_n$). For those irreducible components $Z_i$ such that $x\notin Z_i$ this is easy -- just take $U_i=f^{-1}(Y\setminus Z_i)$. Now suppose $x\in Z_i$. Since $f(U)\cap Z_i=f\left(U\cap f^{-1}(Z_i)\right)$ it suffices to find such a neighbourhood $U_i$ of $x$ in $f^{-1}(Z_i)$ rather than in $X$. That is, we may replace $A$ by $A/\pp_i$ and $B$ by $B/\pp_i B$ (clearly we still get isomorphisms on stalks), so we may henceforth assume that $A$ is a domain.
	
	Note that $X=\Spec B$ has finitely many connected components (by \cite[Lemma~2.4.2]{alggeo1}). Let $C$ be one of them. Then $C$ is open in $X$, connected, and the local rings $\Oo_{X,x}\cong \Oo_{Y,f(x)}$ are domains for all $x\in X$ (the $\Oo_{Y,y}$ are domains for all $y\in Y$ because $A$ is a domain), so $C$ is an integral prescheme by \cite[Proposition~2.1.4\itememph{d}]{alggeo1}. If we restrict to an affine open subset of $C$ (we can do this because everything is local on $X$) we have reduced the situation to the case where $B$ is a domain as well.
	
	Note that $A\morphism B$ is injective. Indeed, suppose that $a\in A$ is contained in the kernel. Choose any prime ideal $\qq\in\Spec B$ and let $\pp\in\Spec A$ be its preimage in $A$. Then $a$ is contained in the kernel of $A_\pp\morphism B_\qq$ as well, so $a=0$ in $A_\pp$ because $f$ induces isomorphisms on stalks. But $A\morphism A_\pp$ is injective when $A$ is a domain, proving that $a=0$ holds in $A$ as well. Thus, we may regard $A$ as a subring of $B$. Note that both have the same quotient field $K$ because $f$ induces an isomorphism between the stalks at the generic points. Now it's time to remember that $f$ is of locally finite type, so $B$ has finite type over $A$. If $x_1,\ldots,x_m\in B$ are generators over $A$, then we can write $x_i=\frac{a_i}{s_i}$ for $a_i,s_i\in A$, because the $x_i$ are elements of $K$. Putting $s=s_1\cdots s_n$ we deduce that $B=A[s^{-1}]$ is the localization of $A$ at $s$. Then $\Spec B\morphism\Spec A$ is an open embedding, which means we're done, finally.
	%Let $x\in X$. We need to show that some neighbourhood of $f(x)$ is contained in the image of $f$. Otherwise, apply the previous Proposition~\reff{prop:noetherianAffine} (with some care) to show that there is a closed subset of $Y$ containing the image of $f$ and not containing any neighbourhood of $f(x)$. There is $\lambda\in A$ such that $f(X)\subseteq V(\lambda)$. Then $\emptyset =f^{-1}(Y\setminus V(\lambda))=X\setminus V(f^*\lambda)$ implies that $f^*\lambda\in B$ is nilpotent. But then the image of $\lambda$ in $\Oo_{Y,f(x)}$ is nilpotent as well, which means that $V(\lambda)=\Spec A$ contains an open neighbourhood of $f(x)$.
\end{proof}
\begin{proof}[Proof of Theorem~\reff{thm:ZariskiMain}\itememph{b} \textsc{(finally)}]
Now let $f\colon X\morphism Y$ be a quasi-finite separated morphism between noetherian preschemes such that Zariski's main theorem holds with $Y$ replaced by $\Spec \roof{\Oo}_{Y,y}$. Note that $\upsilon\colon \Spec \roof{\Oo}_{Y,y}\morphism Y$ is flat (in the sense of the upcoming Definition~\reff{def:flat}) because of Corollary~\reff{cor:completionFaithfullyFlat}\itememph{b}. Therefore, in the pullback diagram
\begin{diagram}[baseline=0.75cm-0.5ex][\lbl{diag:completionBaseChange}]
	\object{2.5,1.5}{$\Spec \roof{\Oo}_{Y,y}$}[a];
	\object{0,1.5}{$\roof{X}$}[b];
	\object{2.5,0}{$Y$}[c];
	\object{0,0}{$X$}[d];
	\pullback{1.25,0.75};
	\scriptsize
	\arrow ba[above][$\roof{f}$];
	\arrow dc[above][$f$];
	\arrow bd;
	\arrow ac[right][$\upsilon$];
\end{diagram}
we obtain
\begin{align}\lbl{eq:froof*}
(f_*\Oo_X)_y\otimes_{\Oo_{Y,y}}\roof{\Oo}_{Y,y}\cong \Gamma\big(\Spec \roof{\Oo}_{Y,y},\roof{f}_*\Oo_{\roof{X}}\big)\;.
\end{align}
Indeed, this is how the base change isomorphism \eqreff{eq:baseChangeMorphism} from Proposition~\reff{prop:baseChangeMorphism} looks like when applied to $\Ff=\Oo_X$ and $p=0$, and also after taking global sections.% and using the fact that $\Gamma(U,f_*\Oo_X)\otimes_{\Oo_Y(U)}\roof{\Oo}_{Y,y}\cong (f_*\Oo_X)_y\otimes_{\Oo_{Y,y}}\roof{\Oo}_{Y,y}$ when $U\subseteq Y$ is an affine open neighbourhood of $y$.

Note that the local rings $\roof{\Oo}_{Y,y}$ and $\Oo_{Y,y}$ have the same residue field $\KK(y)$. Hence $\Spec \KK(y)\morphism Y$ can be factored through $\upsilon\colon \Spec \roof{\Oo}_{Y,y}\morphism Y$ and together with the Stein factorizations of $f$ and $\roof{f}$ this gives a diagram
\begin{diagram}[baseline=1.5cm-0.5ex][\lbl{diag:PullbacksEverywhere}]
	\object{9,1.5}{$\Spec \roof{\Oo}_{Y,y}$}[Ycomplete];
	\object{0,1.5}{$\roof{X}$}[Xcomplete];
	\object{9,0}{$Y$}[Y];
	\object{0,0}{$X$}[X];
	\object{9,3}{$\Spec \KK(y)$}[Speck];
	\object{0,3}{$X\times_Y\Spec \KK(y)$}[XxSpeck];
	\object{4.5,0}{$\SPEC(f_*\Oo_X)$}[SPEC];
	\object{4.5,1.5}{$\SPEC(\roof{f}_*\Oo_{\roof{X}})$}[SPECcomplete];
	\object{4.5,3}{$\Spec \left((f_*\Oo_X)_y\otimes_{\Oo_{Y,y}}\KK(y)\right)$}[Specfk];
	\pullback{2.25,0.75};
	\pullback{2.25,2.25};
	\pullback{6.75,0.75};
	\pullback{6.75,2.25};
	\scriptsize
	\arrow{X}{SPEC}[above][$j$];
	\arrow{SPEC}{Y}[above][$g$];
	\arrow{Xcomplete}{SPECcomplete}[above][$\roof{\jmath}$];
	\arrow{SPECcomplete}{Ycomplete}[above][$\roof{g}$];
	\arrow{XxSpeck}{Specfk};
	\arrow{Specfk}{Speck};
	\arrow{Xcomplete}{X};
	\arrow{XxSpeck}{Xcomplete};
	\arrow{Ycomplete}{Y}[right][$\upsilon$];
	\arrow{Speck}{Ycomplete};
	\arrow{Specfk}{SPECcomplete};
	\arrow{SPECcomplete}{SPEC};
\end{diagram}
\begin{claim}\lbl{claim:PullbacksEverywhere}
	In the diagram \eqreff{diag:PullbacksEverywhere} every subrectangle (not only those indicated) is a pullback.
\end{claim}
Indeed, it is clear that the whole diagram as well as its bottom half (both can be viewed as subrectangles) are pullbacks, since that's how $X\times_Y\Spec \KK(y)$ and $\roof{X}$ are defined. Moreover, we have the relation $\SPEC \big(\upsilon^*(f_*\Oo_X)\big)\cong \SPEC (f_*\Oo_X)\times_Y\Spec \roof{\Oo}_{Y,y}$ by \cite[Corollary~1.6.2]{alggeo2}, which together with \eqreff{eq:froof*} implies $\SPEC (\roof{f}_*\Oo_{\roof{X}})\cong \SPEC (f_*\Oo_X)\times_Y\Spec \roof{\Oo}_{Y,y}$. This shows that the bottom right rectangle in \eqreff{diag:PullbacksEverywhere} is a pullback. By a similar argument the right half rectangle is a pullback as well. Now it's a matter of abstract nonsense to show that the remaining subrectangles are pullbacks too.

Because quasi-finiteness and separatedness are preserved under base change, $\roof{f}$ satisfies all assumptions from Theorem~\reff{thm:ZariskiMain}\itememph{b} and has target $\Spec \roof{\Oo}_{Y,y}$. Hence Zariski's main theorem may be applied to $\roof{f}$. In particular, this means that $\roof{f}$ is quasi-affine, so $\roof \jmath \colon\roof{X}\morphism\SPEC (\roof{f}_*\Oo_{\roof{X}})$ is an open embedding  by Definition~\reff{def:quasiAffine}\itememph{d}. 

Note that Claim~\reff{claim:PullbacksEverywhere} implies that the fibres $f^{-1}\{y\}$ and $\roof{f}^{-1}\{y\}$ coincide, as do $g^{-1}\{y\}$ and $\roof{g}^{-1}\{y\}$. Moreover we know $\roof\jmath\colon \roof{X}\morphism\SPEC(\roof{f}_*\Oo_{\roof{X}})$ is injective on the fibre $\roof{f}^{-1}\{y\}$ (in fact, it is injective everywhere because we just showed it's an open embedding). By diagram~\eqreff{diag:PullbacksEverywhere} we conclude that
\begin{align*}
	f^{-1}\{y\}=\roof{f}^{-1}\{y\}=X\times_Y\Spec \KK(y)\morphism \Spec \left((f_*\Oo_X)_y\otimes_{\Oo_{Y,y}}\KK(y)\right)=\roof{g}^{-1}\{y\}=g^{-1}\{y\}
\end{align*}
is injective on points. But this means that $j\colon X\morphism \SPEC(f_*\Oo_X)$ is injective on $f^{-1}\{y\}$ -- and since $y$ was chosen arbitrarily, we see that $j$ is actually injective on points! So the only thing left to do is to show that $j$ induces isomorphisms on stalks, then Lemma~\reff{lem:OpenEmbedding} does the rest.

So how do we prove that $j$ induces isomorphisms on stalks? Intuitively it is pretty clear what to do: Choose $x\in X$ such that $y=f(x)$, and let $\roof{x}\in\roof{X}$ be a preimage of $x$. We put
\begin{align*}
	\Oo=\Oo_{\SPEC(f_*\Oo_X),j(x)}\quad\text{and}\quad\roof{\Oo}=\Oo_{\SPEC(\roof{f}_*\Oo_{\roof{X}}),\roof{\jmath}(\roof{x})}
\end{align*}
for convenience. Then $\roof{\jmath}^*\colon\roof{\Oo} \morphism \Oo_{\roof{X},\roof{x}}$ is an isomorphism since $\roof{\jmath}$ is an open embedding. Intuitively, $\roof{\jmath}^*$ should be given by tensoring $j^*\colon \Oo\morphism \Oo_{X,x}$ with $\roof{\Oo}_{Y,y}$. Now faithful flatness of $\roof{\Oo}_{Y,y}$ over $\Oo_{Y,y}$ shows that $j^*$ must already be an isomorphism.

However, the real life is more complicated, since the stalk $\roof{\Oo}$ is usually \emph{only a localization} of $\Oo\otimes_{\Oo_{Y,y}}\roof{\Oo}_{Y,y}$, and the same goes for $\Oo_{\roof{X},\roof{x}}$ and $\Oo_{X,x}\otimes_{\Oo_{Y,y}}\roof{\Oo}_{Y,y}$, cf.\ the construction in \cite[Proposition~1.3.2]{alggeo1}. We first argue why the latter problem is no problem at all in our situation.
\begin{claim}\lbl{claim:Oxxhat}
	Let $x\in X$ be such that $y=f(x)$. Then $x$ has a unique preimage $\roof{x}\in\roof{X}$ and the local ring $\Oo_{\roof{X},\roof{x}}$ is given by $\Oo_{X,x}\otimes_{\Oo_{Y,y}}\roof{\Oo}_{Y,y}$.
\end{claim}
The easiest way to see that $\roof{x}$ is unique is probably that the fibres $f^{-1}\{y\}$ and $\roof{f}^{-1}\{y\}$ coincide, but it also follows from the fact that $\mm_{X,x}\otimes_{\Oo_{Y,y}}\roof{\Oo}_{Y,y}$ is a maximal ideal of $\Oo_{X,x}\otimes_{\Oo_{Y,y}}\roof{\Oo}_{Y,y}$ (by Proposition~\reff{prop:technicalAF}\itememph{a}) and our explicit description of fibre products in \cite[Proposition~1.3.2]{alggeo1}. By Proposition~\reff{prop:technicalAF}\itememph{b}, the ring $\Oo_{X,x}\otimes_{\Oo_{Y,y}}\roof{\Oo}_{Y,y}$ is already local, hence it is isomorphic to $\Oo_{\roof{X},\roof{x}}$ (again by the explicit construction of fibre products).

Still this doesn't solve our problem with $\roof{\Oo}$ and $\Oo\otimes_{\Oo_{Y,y}}\roof{\Oo}_{Y,y}$, since I can't see why Proposition~\reff{prop:technicalAF}\itememph{b} should be applicable here. So we use a workaround. Since the bottom left square in \eqreff{diag:PullbacksEverywhere} is a pullback square, we see that $\Oo_{\roof{X},\roof{x}}$ is a localization of $\Oo_{X,x}\otimes_\Oo\roof{\Oo}$, in which $\roof{\Oo}$ in turn is a localization of $\Oo\otimes_{\Oo_{Y,y}}\roof{\Oo}_{Y,y}$. But $\Oo_{\roof{X},\roof{x}}\cong \Oo_{X,x}\otimes_{\Oo_{Y,y}}\roof{\Oo}_{Y,y}$ by Claim~\reff{claim:Oxxhat}, so an easy argument shows that already $\Oo_{\roof{X},\roof{x}}\cong \Oo_{X,x}\otimes_\Oo\roof{\Oo}$, without needing to localize.

Therefore, $\roof{\jmath}^*\colon \roof{\Oo}\morphism\Oo_{\roof{X},\roof{x}}$ is given by applying $-\otimes_\Oo\roof{\Oo}$ to $j^*\colon \Oo\morphism\Oo_{X,x}$. Thus we can do the argument, which we planned to do with $\roof{\Oo}_{Y,y}$, with $\roof{\Oo}$ instead, provided we are able to show that $\roof{\Oo}$ is faithfully flat over $\Oo$.

But this is not hard. Since $\Oo\morphism\roof{\Oo}$ is a local morphism between local rings, it will be automatically faithfully flat if it is only flat (this follows from Proposition~\reff{prop:faithfullyFlatModule}\itememph{d}, \itememph{a} in the future). Now $\Oo\otimes_{\Oo_{Y,y}}\roof{\Oo}_{Y,y}$ is flat over $\Oo$ because $\roof{\Oo}_{Y,y}$ is flat over $\Oo_{Y,y}$ by Corollary~\reff{cor:completionFaithfullyFlat}. Hence any localization of $\Oo\otimes_{\Oo_{Y,y}}\roof{\Oo}_{Y,y}$ is flat over $\Oo$ as well, which includes $\roof{\Oo}$.

This shows that $j^*\colon \Oo\morphism\Oo_{X,x}$ is an isomorphism. By Lemma~\reff{lem:OpenEmbedding}, $j\colon X\morphism\SPEC(f_*\Oo_X)$ is an open embedding, hence $f$ is quasi-affine by Definition~\reff{def:quasiAffine}\itememph{d}.  Then we're done by Lemma~\reff{lem:ZMTLemma1}.
\end{proof}


\chapter{Flat morphisms}
\begin{defi}\lbl{def:flat}
	Let $f\colon X\morphism Y$ be a morphism of preschemes. A quasi-coherent $\Oo_X$-module $\Ff$ is called \defemph{flat} over $\Oo_Y$ iff it has the following equivalent properties.
	\begin{alphanumerate}
		\item If $U\subseteq X$ and $V\subseteq Y$ are affine opens such that $f(U)\subseteq V$, then $\Ff(U)$ is flat as an $\Oo_Y(V)$-module.
		\item It is possible to cover $X$ by affine open subsets for which such $V$ may be found.
		\item For any $x\in X$ the stalk $\Ff_x$ is a flat $\Oo_{Y,y}$-module, where $y=f(x)$.
	\end{alphanumerate}
	We call $f$ a \defemph{flat morphism} if $\Oo_X$ is flat over $\Oo_Y$, and \defemph{faithfully flat} if $f$ is flat and surjective.
\end{defi}
\begin{proof}[Sketch of a proof of equivalence]
	The implication \itememph{a} $\Rightarrow$ \itememph{b} is trivial. If you think about it, showing \itememph{b} $\Rightarrow$ \itememph{c} $\Rightarrow$ \itememph{a} comes down to the following fact from commutative algebra: Let $B$ be an $A$-algebra and $F$ a $B$-module. Then $F$ is flat over $A$ and only if $F_\qq$ is flat over $A_\pp$ algebra for all primes $\pp\in\Spec A$ and $\qq\in\Spec B$ above $\pp$.
	
	Indeed, if $F$ is flat over $A$, then so is $F_\qq$ (by exactness of localization), hence $F_\qq$ is also flat over $A_\pp$. Conversely, if $F_\qq$ is flat over $A_\pp$, then also over $A$ because $A_\pp$ is flat over $A$ (exactness of localization again). Now let $M'\morphism M\morphism M''$ be an exact sequence of $A$-modules. Then the sequence of $B$-modules $M'\otimes_AF\morphism M\otimes_AF\morphism M''\otimes_AF$ is exact iff its localization at any prime $\qq\in \Spec B$ is exact. But $(M\otimes_AF)_\qq\cong M\otimes_AF_\qq$ (and same for $M'$, $M''$), so the localizations are indeed exact by flatness of the $F_\qq$, proving that $F$ is flat itself.
\end{proof}
\section{Flat base change and cohomology}\lbl{sec:flatBaseChange}
\subsection{The base change morphism}
Let $A$ be a ring, $B$ an $A$-algebra and $f\colon X\morphism \Spec A$ a quasi-compact and separated morphism. Let 
\begin{align*}
	\snake{f}\colon\snake{X}=X\times_{\Spec A}\Spec B\morphism\Spec B%.
\end{align*}
its base change along $\Spec B\morphism\Spec A$. Also let $\xi\colon \snake{X}\morphism X$ be projection to the other fibre product factor. We want to investigate the relation between the cohomology of $X$ and $\snake{X}$. For a quasi-coherent $\Oo_X$-module $\Ff$ and an affine \v Cech cover $\Uu\colon X=\bigcup_{i\in I}U_i$ of $X$, the \v Cech complex $\check{C}^\bullet(\Uu,\Ff)$ calculates $H^\bullet(X,\Ff)$. Pulling back $\Uu$ gives a \v Cech cover
\begin{align*}
	\xi^{-1}\Uu\colon \snake{X}=\bigcup_{i\in I}U_i\times_{\Spec A}\Spec B\;,
\end{align*}
of $\snake{X}$, whose components $U_i\times_{\Spec A}\Spec B$ are affine again (as fibre products of affine schemes), hence $\xi^{-1}\Uu$ may be used to compute $H^\bullet(\snake{X},\xi^*\Ff)$. Explicitly, we obtain
\begin{align}\lbl{eq:tensoredCechComplex}
	\check{C}^\bullet\big(\xi^{-1}\Uu,\xi^*\Ff\big)\cong \check{C}^\bullet(\Uu,\Ff)\otimes_AB\;.
\end{align}
This gives a canonical morphism
\begin{align}\lbl{eq:tensoredCohomology}
	H^\bullet(X,\Ff)\otimes_AB\morphism H^\bullet(\snake{X},\xi^*\Ff)\;,
\end{align}
which clearly is an isomorphism if $B$ is flat over $A$.
\begin{prop}\lbl{prop:baseChangeMorphism}
	Consider a cartesian diagram
	\begin{diagram*}
		\object{2,1.5}{$\snake{Y}$}[a];
		\object{0,1.5}{$\snake{X}$}[b];
		\object{2,0}{$Y$}[c];
		\object{0,0}{$X$}[d];
		\pullback{1,0.75};
		\scriptsize
		\arrow ba[above][$\snake{f}$];
		\arrow dc[above][$f$];
		\arrow bd[left][$\xi$];
		\arrow ac[right][$\upsilon$];
	\end{diagram*}
	in which $f$ is quasi-compact and separated, and let $\Ff$ be a quasi-coherent $\Oo_X$-module. Then we get a morphism of $\Oo_{\snake{Y}}$-modules, called the \defemph{base change morphism},
	\begin{align}\lbl{eq:baseChangeMorphism}
		\upsilon^*R^pf_*\Ff\morphism R^p\snake{f}_*(\xi^*\Ff)
	\end{align}
	which is an isomorphism when $\upsilon$ is flat in the sense of Definition~\reff{def:flat}.
\end{prop}
\begin{proof}
	We may check \eqreff{eq:baseChangeMorphism} locally, where it is just \eqreff{eq:tensoredCohomology}.
\end{proof}
\subsection{The base change spectral sequence}
\begin{varthm}{prop}\lbl{prop:baseChangeMorphismA}
	Suppose that $Y=\Spec A$ and $\snake{Y}=\Spec B$ are affine in the situation from Proposition~\reff{prop:baseChangeMorphism}. If $\Ff$ is flat over $\Oo_Y$ in the sense of Definition~\reff{def:flat}, then the base change morphism \eqreff{eq:baseChangeMorphism} is part of a spectral sequence
	\begin{align*}
		E_2^{p,q}=\Tor_{-p}^A\big(B,H^q(X,\Ff)\big)\converge H^{p+q}\left(\snake{X},\xi^*\Ff\right)\;.
	\end{align*}
\end{varthm}
\begin{rem}
The minus sign in $\Tor_{-p}^A$ is not a typo. This spectral sequence is supported in the $2\ordinalnd$ quadrant and we interpret cohomology or $\Tor$ in negative degrees to be zero.
\end{rem}
\begin{proof}[Proof of Proposition~\reff{prop:baseChangeMorphismA}]
	Let $J^{\bullet,\bullet}$ be a Cartan--Eilenberg resolution of the alternating \v Cech complex $\check{C}_\alt^\bullet(\Uu,\Ff)$, where $\Uu$ is an arbitrary finite affine open cover of $X$ (which is quasi-compact, so this is fine). That is, $J^{\bullet,\bullet}$ is a $2\ordinalnd$ quadrant double complex of $A$-modules such that
	\begin{itemize}
		\item $J^{\bullet,q}\morphism\check{C}_\alt^q(\Uu,\Ff)$ is a projective resolution for all $q\geq 0$ (note that $\bullet$ ranges through all \emph{non-positive} integers here, because we want $J^{\bullet,q}$ to be a cochain complex).
		\item The vertical morphisms $d_\ver^{\bullet,q}\colon J^{\bullet,q}\morphism J^{\bullet,q+1}$ are \emph{split} in the sense that there is a decomposition $J^{\bullet,q}\cong B^{\bullet,q}\oplus H^{\bullet,q}\oplus B^{\bullet,q+1}$ and $d_\ver^{\bullet,q}$ corresponds to the projection onto the summand $B^{\bullet,q+1}$.
		\item The induced cochain complex $H_\ver^q(J^{\bullet,\bullet})\cong H^{\bullet,q}\morphism \check{H}_\alt^q(\Uu,\Ff)\cong H^q(X,\Ff)$ is a projective resolution as well.
	\end{itemize}
It's well-known that Cartan--Eilenberg resolutions exist in the category $\cat{Mod}(A)$ (in fact, in any abelian category with sufficiently many projectives). Moreover, since $\check{C}^q_\alt(\Uu,\Ff)=0$ when $q$ is sufficiently large, say $q\geq q_0$, we may choose $J ^{\bullet,\bullet}$ such that $J^{\bullet,q}=0$ when $q\geq q_0$. Then $J^{\bullet,\bullet}$ is bounded from above in both directions (that is, a $3\ordinalrd$ quadrant double complex, except it's supported completely in the $2\ordinalnd$ quadrant -- do I sound like an idiot? You know what I mean!).

The same is true for the double complex $J^{\bullet,\bullet}\otimes_AB$, hence both its horizontal and its vertical spectral sequence converge -- and even to the same limit. Note that $H_\ver^q(J^{\bullet,\bullet}\otimes_AB)\cong H^{\bullet,q}\otimes_AB$ because of the nice splitness condition for $d_\ver^{\bullet,q}$. We thus have 
\begin{align*}
	\vphantom{E}^\ver E_2^{p,q}=H^p_\hor\big(H^q_\ver(J^{\bullet,\bullet}\otimes_AB)\big)\cong H^p(H^{\bullet,q}\otimes_AB)\cong \Tor_{-p}^A\big(B,H^q(X,\Ff)\big)\;,
\end{align*}
since $H^{\bullet,q}$ is a (negatively indexed) projective resolution of $H^q(X,\Ff)$. Similarly
\begin{align*}
	\vphantom{E}^\hor E_2^{p,q}=H^q_\ver\big(H_\hor^p(J^{\bullet,\bullet}\otimes_AB)\big)=H^q\left(\Tor_{-p}^A\big(B,\check{C}_\alt^\bullet(\Uu,\Ff)\big)\right)
\end{align*}
since $J^{\bullet,q}$ is a projective resolution of $\check{C}^q(\Uu,\Ff)$. But $\check{C}_\alt^q(\Uu,\Ff)=\prod_{i_0<\ldots<i_q}\Ff(U_{i_0,\ldots,i_q})$ is a product of flat $A$-modules because $\Ff$ is flat over $\Oo_Y$, hence $\check{C}_\alt^q(\Uu,\Ff)$ is a flat $A$-module itself (by \cite[Example~1.2.6]{homalg} -- actually, the products occuring here are finite, so it's a lot easier to see that flatness is preserved). Hence
\begin{align*}
	\Tor_{-p}^A\big(B,\check{C}_\alt^q(\Uu,\Ff)\big)\cong
	\begin{cases}
	0 & \text{if }p<0\\
	\check{C}_\alt^q(\Uu,\Ff)\otimes_AB & \text{if }p=0
	\end{cases}\;.
\end{align*}
By \eqreff{eq:tensoredCechComplex} we get $\vphantom{E}^\hor E^{0,q}\cong H^q(\snake{X},\xi^*\Ff)$ and $\vphantom{E}^\hor E^{p,q}=0$ if $p<0$. Therefore the horizontal spectral sequence immediately degenerates, and its limit is $H^{p+q}(\snake{X},\xi^*\Ff)$. That's precisely what we need.
\end{proof}
\section{The Grauert--Grothendieck theorem}
Before we state the theorem, let us once and for all fix some pretty convenient (abuse of) notation.
\begin{con}\lbl{con:AON}
	Let $f\colon X\morphism Y$ be a morphism of preschemes. Let's make the following conventions.
	\begin{alphanumerate}
		\item If $y\in Y$, then $X_y$ denotes the fibre $f^{-1}\{y\}=X\times_Y\Spec\KK(y)$.
		\item We keep the abuse of notation that $\Ff|_{X_y}=i_y^*\Ff$ when $\Ff$ is a quasi-coherent sheaf of $\Oo_X$-modules. Here $i_y\colon X_y=X\times _Y\Spec \KK(y)\morphism X$ is the canonical projection to the first fibre product factor.
		\item If $\Gg$ is a (not necessarily quasi-coherent) sheaf of $\Oo_Y$-modules and $y\in Y$, then we put $\Gg(y)=\Gg_y\otimes_{\Oo_{Y,y}}\KK(y)$.
	\end{alphanumerate}
\end{con}
\begin{thm}\lbl{thm:GrauertGrothendieck}
	Let $f\colon X\morphism Y$ be a proper morphism between locally noetherian preschemes and let $\Ff$ be a coherent $\Oo_X$-module which is flat over $\Oo_Y$.
	\begin{alphanumerate}
		\item If $Y=\Spec A$ is affine, there exists a complex
		\begin{align*}
			P^\bullet\colon 0\morphism P^0\morphism P^1\morphism \ldots \morphism P^d\morphism 0
		\end{align*}
		of finitely generated projective $A$-modules with the following property. If $B$ is an $A$-algebra and $\xi\colon \snake{X}=X\times_Y\Spec B\morphism X$ is the projection of the fibre product to its first factor, then
		\begin{align*}
			H^p\left(\snake{X},\xi^*\Ff\right)\cong H^p(P^\bullet\otimes_AB)\;.
		\end{align*}
		It is possible to choose $d$ to be the largest number with $H^d(X,\Ff)\neq 0$, or $P^\bullet=0$ if $H^\bullet(X,\Ff)=0$. Moreover, if $d'\geq 0$ is such that $H^p(X_y,\Ff|_{X_y})=0$ for all $y\in Y$ when $p<d'$, then it is possible to choose $P^\bullet$ such that $P^i=0$ unless $d'\leq i\leq d$.
		\item The function $h^p(-,\Ff)\colon Y\morphism \IN$ given by
		\begin{align*}
		h^p(y,\Ff)=\dim_{\KK(y)}H^p(X_y,\Ff|_{X_y})
		\end{align*}
		is upper-semicontinuous in the sense that $\left\{y\in Y\st h^p(x,\Ff)\leq c\right\}$ is an open subset of $Y$ for all $c\in \IR$. Moreover, the \defemph{Euler--Poincaré characteristic} $\chi(-,\Ff)\colon Y\morphism \IN$ of the fibres, which is given by
		\begin{align*}
			\chi(y,\Ff)=\sum_{p\geq 0}(-1)^ph^p(y,\Ff)\;,
		\end{align*}
		is locally constant on $Y$.
		\item If $Y$ is reduced and $h^p(-,\Ff)$ is locally constant on $Y$, then the base change morphism
		\begin{align*}
			\upsilon^*R^pf_*\Ff\xrightarrow{\text{\eqreff{eq:baseChangeMorphism}}} R^p\snake{f}_*\left(\xi^*\Ff\right)
		\end{align*}
		(with notation from Proposition~\reff{prop:baseChangeMorphism}) is an isomorphism. In particular, for $\snake Y=\Spec \KK(y)$ we get
		\begin{align}\lbl{eq:RpfFy}
			(R^pf_*\Ff)(y)\isomorphism H^p\left(X_y,\Ff|_{X_y}\right)\;,
		\end{align}
		where $(R^pf_*\Ff)(y)$ is to be interpreted as in Convention~\reff{con:AON}\itememph{c}.
		\item Without the assumptions of \itememph{c} -- i.e., without $Y$ reduced and $h^p(-,\Ff)$ locally constant -- the canonical morphism \eqreff{eq:RpfFy} is bijective (for given $p$ and $y$) if and only if it is surjective. Moreover, for any $p\geq 0$ the set $U_p$ of all $y\in Y$ for which this is the case is open in $Y$.
		\item Let $y\in U_p$, then the following assertions are equivalent.
		\begin{itemize}
			\item[\itememph{\alpha}] We also have $y\in U_{p-1}$.
			\item[\itememph{\beta}] $R^pf_*\Ff$ is locally free in some neighbourhood of $y$.
		\end{itemize}
	\end{alphanumerate}
\end{thm}
\begin{rem}
	\begin{alphanumerate}
		\item Note that in Theorem~\reff{thm:GrauertGrothendieck}\itememph{a}, $H^p(X,\Ff)=0$ for $p$ sufficiently large, because $X$ is quasicompact, hence we can calculate its cohomology from the antisymmetric \v Cech complex of a finite affine open cover (even though $X$ could have infinite dimension).
		\item For instance, the assumptions of Theorem~\reff{thm:GrauertGrothendieck} are fulfilled when $f$ is a proper and flat morphism and $\Ff$ a vector bundle on $X$.
		\item From Theorem~\reff{thm:GrauertGrothendieck}\itememph{d} it follows that $R^pf_*\Ff=0$ when $H^p(X_y,\Ff|_{X_y})=0$ for all $y\in Y$.
		\item Note that $B$ in Theorem~\reff{thm:GrauertGrothendieck}\itememph{a} is completely arbitrary. In particular, it needn't be noetherian!
	\end{alphanumerate}
\end{rem}
	The strategy for Theorem~\reff{thm:GrauertGrothendieck}\itememph{a} is to modify the anti-symmetric \v Cech complex $\check{C}^\bullet=\check{C}_\alt^\bullet(\Uu,\Ff)$, where $\Uu$ is some fixed affine open cover of $X$. As $\Ff$ is flat over $\Oo_Y$, the $\Ff(U_{i_0,\ldots,i_n})$ are flat $A$-modules, hence $\check{C}^\bullet$ is a complex of flat $A$-modules, but the finite generation assertions will typically fail, as will the vanishing assertions. However, from \eqreff{eq:tensoredCechComplex} we know that
	\begin{align}\lbl{eq:baseChangeProperty}
		H^p(\snake{X},\xi^*\Ff)\cong H^p(\check{C}^\bullet\otimes_AB)
	\end{align}
	when $B$ and $\snake{X}$ are as in \itememph{a}. We will construct cochain complexes $P^\bullet$ and $Q^\bullet$ with morphisms $c\colon Q^\bullet\morphism\check{C}^\bullet$ and $p\colon Q^\bullet\morphism P^\bullet$ such that $P^\bullet$ has all the required properties and $c$, $p$ induce isomorphisms on $H^\bullet(-\otimes_AM)$ for arbitrary $A$-modules $M$.
	
	The rest of the proof of \itememph{a} doesn't concern any $\check{C}^\bullet$-specific properties except flatness and the base change property. However, in the upcoming sequence of commutative algebra lemmas we will rather freely use facts from homological algebra (most should be known at this level, and usually you can look them up in \cite{homalg} or any textbook).
	\begin{lem}\lbl{lem:TensorQuasiIsomorphism}
		Let $A$ be a (not necessarily noetherian) ring and $\phi\colon C^\bullet\morphism D^\bullet$ a quasi-isomorphism\footnote{That is, $\phi$ induces isomorphisms on cohomology} between bounded-from-above cochain complexes of flat $A$-modules. Then, for any $A$-module $M$,
		\begin{align*}
			\phi\otimes\id_M\colon C^\bullet\otimes_AM\morphism D^\bullet\otimes_AM
		\end{align*}
		is a quasi-isomorphism again.
	\end{lem}
\begin{proof}
	Using the short exact cone sequence $0\morphism D^\bullet \morphism\Cone(\phi)\morphism C[1]^\bullet\morphism 0$ (cf.\ \cite[Definition~2.1.1]{alggeo2}), we see that $\phi$ is a quasi-isomorphism iff $\Cone(\phi)$ has vanishing cohomology. But $\Cone(\phi\otimes\id_M)\cong \Cone(\phi)\otimes_AM$, so it's sufficient to prove that $\Cone(\phi)\otimes_AM$ is acyclic. It's clear from the construction that $\Cone(\phi)$ consists of flat $A$-modules again and is bounded from above as well. But then $\Cone(\phi)$ gives a flat resolution of the zero $A$-module, so the cohomology of $\Cone(\phi)\otimes_AM$ computes $\Tor_p^A(0,M)=0$ for all $p\geq 0$. We're done.
\end{proof}
\begin{rem}
	\begin{alphanumerate}
		\item\lbl{rem:TorFishy?} If you're uncomfortable with the $\Tor$ shortcut argument at the end of the proof above, Professor Franke suggests an alternative proof as follows. If $Z^p$ is the kernel of $\Cone(\phi)^p\morphism\Cone(\phi)^{p+1}$, then one can show that all $Z^p$ are are flat (using boundedness from above and downward induction on $p$). Then the short exact sequences
		\begin{align*}
			0\morphism Z^p\morphism \Cone(\phi)^p\morphism Z^{p+1}\morphism 0
		\end{align*}
		stay exact after tensoring with $M$, and splicing together the tensored sequences gives $\Cone(\phi)\otimes_AM$.
		\item When $A$ is regular, the boundedness assumption may be dropped. Indeed, since $\Cone(\phi)\otimes_AM$ is exact iff its localizations at any $\pp\in\Spec A$ are exact, we may assume $A$ to be regular and local. But then $\Tor_p^A(-,-)$ vanishes when $p>\dim A$ (cf.\ \cite[Theorem~1]{homalg}), so the $\Tor$ argument from the proof of Lemma~\reff{lem:TensorQuasiIsomorphism} works even if we're not resolving the zero module.
		
		However, boundedness can't be dropped in general. For example, consider $A=\IZ/p^2\IZ$ for some prime $p$, and let
		\begin{align*}
			C^\bullet\colon \ldots\morphism[ p\cdot]A\morphism[ p\cdot]A\morphism[ p\cdot]A\morphism[ p\cdot]\ldots\;,
		\end{align*}
		$D^\bullet=0$, and $M=\IZ/p\IZ$. Then $D^\bullet\otimes_AM=0$, but $C^\bullet\otimes_AM$ has cohomology $\IZ/p\IZ$ in every degree.
	\end{alphanumerate}
\end{rem}
Recall that if $C^\bullet$ is a cochain complex, one can form its \emph{soft truncation from above}
\begin{align*}
	\tau_{\leq d}C^\bullet\colon \ldots\morphism C^n\morphism C^{n+1}\morphism \ldots \morphism C^{d-1}\morphism Z^d\morphism 0\morphism 0\morphism \ldots\;,
\end{align*} 
where $Z^d=\ker\left(C^{d}\morphism C^{d+1}\right)$. We have $H^p(\tau_{\leq d}C^\bullet)=H^p(C^\bullet)$ if $p\leq d$ and $H^p(\tau_{\leq d}C^\bullet)=0$ otherwise.
\begin{lem}\lbl{lem:TruncQuasiIso}
	Let $F^\bullet$ be a bounded-from-above cochain complex of flat modules over an arbitrary ring $A$. If $F^\bullet$ is acyclic in degrees $>d$, then $\tau_{\leq d}F^\bullet$ is still flat over $A$ and $\tau_{\leq d}F^\bullet\monomorphism F^\bullet$ is a quasi-isomorphism.
\end{lem}
\begin{proof}
	Flatness of $\tau_{\geq d}F^\bullet$ is only critical in degree $d$. Since $F^\bullet$ is bounded from above, we may do downward induction on $d$, so let's assume that $\tau_{\leq d+1}F^\bullet$ is flat. Then $F^\bullet$ can be replaced by $\tau_{\leq d+1}F^\bullet$. Together with the fact that $F^\bullet$ is acyclic in degrees $>d$, we may thus assume that $0\morphism Z^d\morphism F^d\morphism F^{d+1}\morphism 0$ is exact. Since $F^d$, $F^{d+1}$ are flat by assumption, the long exact $\Tor$ sequence shows that $\Tor_1^A(Z^d,M)$ vanishes for any $A$-module $M$, hence $Z^d$ is flat as well. This proves that $\tau_{\leq d}F^\bullet$ is flat over $A$. The other assertion about $\tau_{\leq d}F^\bullet\monomorphism F^\bullet$ is trivial.
\end{proof}
\begin{rem}
	\begin{alphanumerate}
		\item In particular, Lemma~\reff{lem:TensorQuasiIsomorphism} applies to $\tau_{\leq d}F^\bullet\monomorphism F^\bullet$.
		\item Without boundedness in Lemma~\reff{lem:TruncQuasiIso}, the counterexample from Remark~\reff{rem:TorFishy?} still applies.
	\end{alphanumerate}
\end{rem}
\begin{lem}\lbl{lem:FiniteApproximation}
	Let $A$ be a noetherian ring and $F^\bullet$ a cochain complex of (not necessarily flat) $A$-modules which is concentrated in degrees $< d$ and such that the cohomology groups $H^p(F^\bullet)$ are finitely generated over $A$. Then there is a cochain complex $Q^\bullet$, consisting of finitely generated free $A$-modules and concentrated in degrees $<d$, together with a quasi-isomorphism $\rho^\bullet\colon Q^\bullet\morphism F^\bullet$.
\end{lem}
\begin{proof}
	Denote the differentials of $F^\bullet$ and $Q^\bullet$ by $d_F^\bullet$, $d_Q^\bullet$ (the latter of which is yet to be constructed). We will construct the $Q^i$ by downward induction. We start by putting $Q^p=0$ when $p\geq d$. Now let $k\in\IZ$ and suppose we have already constructed $Q^p$ and $d_Q^p$ for $p\geq k$ together with $\rho^p\colon Q^p\morphism F^p$ (compatible with the differentials) such that $H^p(Q^\bullet)\isomorphism H^p(F^\bullet)$ is an isomorphism when $p>k$ and $\ker(d_Q^k)\morphism H^k(F^\bullet)$ is surjective.
	
Now it's about constructing suitable $Q^{k-1}$, $d_Q^{k-1}$ such that the above conditions remain valid for $k-1$. Let $\beta_1,\ldots,\beta_\ell$ be generators of $\left\{x\in Q^k\st \rho^k(x)\in\Im\big(d_F^{k-1}\big)\right\}$
(that's where we really need that $A$ is noetherian) and let $\phi_i\in F^{k-1}$ such that $d_F^{k-1}(\phi_i)=\rho^k(\beta_i)$ for all $i=1,\ldots,\ell$. Also choose $\psi_1,\ldots,\psi_m\in\ker\big(d_F^{k-1}\big)$ whose images in $H^{k-1}(F^\bullet)$ generate that $A$-module (here we need that the cohomology of $F^\bullet$ is finitely generated). Put $Q^{k-1}=A^{\ell+m}$ and let $e_i$ be the $i\ordinalth$ standard basis vector. Now define $d_Q^{k-1}\colon Q^{k-1}\morphism Q^k$ and $\rho^{k-1}\colon Q^{k-1}\morphism F^{k-1}$ by
\begin{align*}
	d_Q^{k-1}(e_i)=
	\begin{cases}
		\beta_i &\text{if }i\leq \ell\\
		0 & \text{if }\ell <i\leq \ell+m
	\end{cases}\quad\text{and}\quad\rho^{k-1}(e_i)=
	\begin{cases}
		\phi_i & \text{if }i\leq \ell\\
		\psi_{i-\ell} & \text{if }\ell <i\leq \ell+m
	\end{cases}\;.
\end{align*}
This does it, as is easily checked.
\end{proof}
Let $C^\bullet$ be any cochain complex. Similar to the soft truncation from above (from above), there's a \defemph{soft truncation from below} given by
\begin{align*}
\tau_{\geq d}C^\bullet\colon\ldots\morphism 0\morphism 0\morphism C^d/J^d\morphism C^{d+1}\morphism C^{d+2}\morphism \ldots\;,
\end{align*}
in which $J^d=\Im\left(C^{d-1}\morphism C^d\right)$. It has the property that $H^p(\tau_{\geq d}C^\bullet)=H^p(C^\bullet)$ if $p\geq d$ and $H^p(\tau_{\geq d}C^\bullet)=0$ otherwise.
\begin{lem}\lbl{lem:TruncQuasiIso2}
	Let $A$ be a noetherian ring and $Q^\bullet$ be a cochain complex of finitely generated projective $A$-modules. Suppose there's $d'\in\IZ$ such that $H^p(Q^\bullet)=0$ for all $p<d'$ and $H^p(Q^\bullet\otimes_A\KK(\mm))=0$ for all $p<d'$ and all maximal ideals $\mm\subseteq A$. Then $\tau_{\geq d'}Q^\bullet$ is a cochain complex of projective $A$-modules. Also $Q^\bullet\morphism \tau_{\geq d'}Q^\bullet$ is a quasi-isomorphism if $H^p(Q^\bullet)=0$ for all $p< d'$.
\end{lem}
\begin{proof}
	W.l.o.g.\ let $d'=0$ and $J^0=\Im\left(Q^{-1}\morphism Q^0\right)$. Since $Q^\bullet$ is acyclic in degrees $<d'$ by assumption, we see that
	\begin{align*}
		\ldots \morphism Q^{-2}\morphism Q^{-1}\morphism Q^0\morphism Q^0/J^0\morphism 0
	\end{align*}
	is a projective resolution of $Q^0/J^0$. We use this resolution to compute $\Tor_p^A(Q_0/J_0,-)$. By our assumption we get $\Tor_p^A(Q^0/J^0,\KK(\mm))=H^{-p}(Q^\bullet\otimes_A\KK(\mm))=0$ for all $\mm\in\mSpec A$ and all $p>0$. Since $Q^0/J^0$ is finitely generated over the noetherian ring $A$, this implies that $Q^0/J^0$ is projective (this is essentially \cite[Propositions~1.3.1 and~1.3.2]{homalg}, but you should combine them with \cite[Fact~1.2.4]{homalg} first). The other assertion about $Q^\bullet\morphism \tau_{\geq d'}Q^\bullet$ being a quasi-isomorphism is trivial again.
\end{proof}
\begin{exc}
	Under the assumptions of Theorem~\reff{thm:GrauertGrothendieck}, derive Theorem~\reff{thm:FormalFunctions} for $f\colon X\morphism Y$ and $\Ff$ from what has been shown until Lemma~\reff{lem:FiniteApproximation}.
\end{exc}
\begin{lem}\lbl{lem:TechnicalVanishingAssertion}
	If some $p\geq 0$ has the property that $H^p\left(X_y,\Ff|_{X_y}\right)=0$ for all $y\in Y$, then also $H^p(X,\Ff)=0$. In particular,
	if $d'$ is as in Theorem~\reff{thm:GrauertGrothendieck}\itememph{a}, i.e., $H^p\left(X_y,\Ff|_{X_y}\right)=0$ holds for all $y\in Y$ and $p<d'$, then also $H^p(X,\Ff)=0$ when $p<d'$.
\end{lem}
\begin{proof}
	Professor Franke outsourced a part of this proof to an external fact -- however, this fact looked rather fishy to me (if not to say \emph{wrong} unless $f$ is flat). I think the following captures what he actually had in mind.
	
	By Corollary~\reff{cor:completionFaithfullyFlat2} it's sufficient to show that the $\mm$-adic completions $H^p(X,\Ff)^\complete$ vanish for all maximal ideals $\mm\subseteq A$ and all $p<d'$. Let $y\in Y$ be the point corresponding to $\mm$. For all $n\leq 0$ let
	\begin{align*}
		X_y^{(n)}=X\times_Y\Spec A/\mm^{n+1}
	\end{align*}
	be the $n\ordinalth$ infinitesimal thickening of the fibre $X_y$, and let $\kappa_n\colon X_y^{(n)}\morphism X$ be the projection to the first fibre product factor. We will show $H^p\Big(X_y^{(n)},\Ff|_{X_y^{(n)}}\Big)=0$ for all $n\geq 1$. Note that
	\begin{align*}
		H^p\Big(X_y^{(n)},\Ff|_{X_y^{(n)}}\Big)\cong H^p\left(X,\kappa_{n,*}\kappa_n^*\Ff\right)
	\end{align*}
	and $\kappa_{n,*}\kappa_n^*\Ff\cong \Ff/\mm^{n+1}\Ff$ (we have already seen this in the discussion before Theorem~\reff{thm:FormalFunctions}). Now $H^p(X,\Ff/\mm^{n+1}\Ff)=0$ follows by induction on $n$ as follows. The case $n=0$ is just the assumption that $H^p(X_y,\Ff|_{X_y})=0$. Now let $n\geq 1$ and suppose that $H^p(X,\Ff/\mm^n\Ff)=0$ has already been shown. Consider the short exact sequence
	\begin{align*}
		0\morphism \mm^n\Ff/\mm^{n+1}\Ff\morphism \Ff/\mm^{n+1}\Ff\morphism \Ff/\mm^n\Ff\morphism 0\;.
	\end{align*}
	If we show that $H^p\left(X,\mm^n\Ff/\mm^{n+1}\Ff\right)=0$, then $H^p\left(X,\Ff/\mm^{n+1}\Ff\right)=0$ follows from the long exact cohomology sequence and the induction hypothesis. Note that $\mm^n\otimes_A\Ff\cong \mm^n\Ff$ (and same for $n+1$). Indeed, this follows from \cite[Proposition~1.2.3\itememph{d}]{homalg} because $\Ff$ is flat over $\Oo_Y$. Hence
	\begin{align*}
		\mm^n\Ff/\mm^{n+1}\cong \mm^n/\mm^{n+1}\otimes_A\Ff\cong \mm^n/\mm^{n+1}\otimes_{\KK(\mm)}\Ff/\mm\Ff
	\end{align*}
	is isomorphic to a direct sum of $\dim_{\KK(\mm)}\left(\mm^n/\mm^{n+1}\right)$ copies of $\Ff/\mm \Ff$, because $\mm^n/\mm^{n+1}$ is a finite-dimensional $\KK(\mm)$-vector space. By the $n=0$ case, this shows $H^p\left(X,\mm^n\Ff/\mm^{n+1}\Ff\right)=0$ and we're done.
\end{proof}
\begin{rem}
	As the above proof shows, it's actually sufficient to have $H^p\left(X_y,\Ff|_{X_y}\right)=0$ for all \emph{closed} $y\in Y$.
\end{rem}
We now have all the technicalities together to prove Theorem~\reff{thm:GrauertGrothendieck} (ok, actually we there are still three commutative algebra lemmas to follow, but we can start now anyway).
\begin{proof}[Proof of Theorem~\reff{thm:GrauertGrothendieck}]
	The first step will be to prove \itememph{a}, from which the other assertions will be deduced. In \itememph{a}, $Y$ is affine, and the assertions \itememph{b} to \itememph{e} are all base-local, hence without losing generality we may assume $Y=\Spec A$ for the remainder of this proof.
	
	Fix a finite affine open cover $\Uu$ of $X$ and let $\check{C}^\bullet=\check{C}_\alt^\bullet(\Uu,\Ff)$. Then $\check{C}^\bullet$ is bounded from above (because we take the \emph{alternating} complex of a finite cover) and consists of flat $A$-modules (since $\Ff$ is flat over $\Oo_Y$). Using Lemma~\reff{lem:TruncQuasiIso} we see that $\tau_{\leq d}\check{C}^\bullet$ consists of flat $A$-modules again. Then we choose a finitely generated free approximation $Q^\bullet\morphism\tau_{\leq d}\check{C}^\bullet$ as in Lemma~\reff{lem:FiniteApproximation} (here we crucially use that the $H^p(\tau_{\leq d}\check{C}^\bullet)\cong H^p(X,\Ff)$ are finitely generated over $A$ by \cite[Theorem~5]{alggeo2}). By Lemma~\reff{lem:TensorQuasiIsomorphism} we get
	\begin{align*}
		H^p\left(Q^\bullet\otimes_A\KK(\mm)\right)\cong H^p\left(\tau_{\geq d}\check{C}^\bullet\otimes_A\KK(\mm)\right)\cong H^p\left(\check{C}^\bullet\otimes_A\KK(\mm)\right)\overset{\text{\eqreff{eq:baseChangeProperty}}}{\cong} H^p\left(X_y,\Ff|_{X_y}\right)\;,
	\end{align*}
	where $y\in Y$ on the right-hand side is the point corresponding to the maximal ideal $\mm$, so that we have $X_y\cong X\times_Y\Spec\KK(\mm)$. In particular, we see $H^p\left(Q^\bullet\otimes_A\KK(\mm)\right)=0$ if $p<d'$ and also $H^p(Q^\bullet)\cong H^p(X,\Ff)=0$ if $p<d'$ by Lemma~\reff{lem:TechnicalVanishingAssertion}, hence Lemma~\reff{lem:TruncQuasiIso2} may be applied. Therefore, $P^\bullet=\tau_{\geq d'}Q^\bullet$ is complex of finitely generated projective $A$-modules with is supported only between $d'$ and $d$, and $P^\bullet$ has the required base change property because of \eqreff{eq:baseChangeProperty} and Lemma~\reff{lem:TensorQuasiIsomorphism}. In other words, $P^\bullet$ is what we want. This proves \itememph{a}.
	
	\lbl{page:Pfree}Note that for the rest of the proof we may even assume that $P^\bullet$ is a complex of \emph{free} modules. Indeed, our construction of $P^\bullet$ provided free modules except in degree $d'$, where $P^{d'}$ might only be projective. However, assertions \itememph{b} to \itememph{e} are base-local, so we may replace $\Spec A$ by $\Spec A_f$ such that $P_f^{d'}$ is free over $A_f$.%Dunno if that's needed.
	
	Let $C^p=\coker\left(P^{p-1}\morphism P^p\right)$. Then $\coker\left(P^{p-1}\otimes_AB\morphism P^p\otimes_AB\right)\cong C^p\otimes_AB$. If we put $\delta^p\colon C^p\morphism P^{p+1}$, then $H^p(P^\bullet\otimes_AB)\cong \ker\left(\delta^p\otimes\id_B\right)$ for all $A$-algebras $B$ (actually, we don't even need $B$ to be an algebra).
	
	Now for part \itememph{b}. We prove that $\chi(-,\Ff)$ is locally constant on $Y$ first. Indeed, since the $P^i$ are free over $A$ (or rather we may assume them to be, as explained above) we have
	\begin{align*}
		\chi(y,\Ff)=\sum_{p\geq 0}(-1)^p\dim_{\KK(y)}H^p(X_y,\Ff|_{X_y})&=\sum_{p\geq 0}(-1)^p\dim_{\KK(y)}H^p(P^\bullet\otimes_A\KK(y))\\
		&=\sum_{p\geq 0}(-1)^p\dim_{\KK(y)}\left(P^p\otimes_A\KK(y)\right)\\
		&=\sum_{p\geq 0}(-1)^p\rank P^p
	\end{align*}
	(the transition from the first to the second line is a well-known fact from homological algebra). The last line is clearly constant on $Y$ and we're done.
	
	To prove the rest of \itememph{b} as well as \itememph{c}, we invoke another lemma.
	\begin{lem}\lbl{lem:BasicallyThm4bc}
		Let $A$ be a noetherian ring and $\delta\colon M\morphism F$ a morphism of finitely generated $A$-modules, of which $F$ is free of rank $n$. 
		\begin{alphanumerate}
			\item The function $g\colon \Spec A\morphism\IN$ given by
			\begin{align*}
				g(\pp)=\dim_{\KK(\pp)}\ker\left(\delta_\pp=\delta\otimes\id_{\KK(\pp)}\colon M\otimes_A\KK(\pp)\morphism F\otimes_A\KK(\pp)\right)
			\end{align*}
			is upper-semicontinuous on $\Spec A$.
			\item If in addition $g$ is locally constant and $A$ is reduced, then $M$, $\ker\delta$, and $\coker\delta$ are all projective $A$-modules, and the canonical morphism
			\begin{align*}
				\ker\delta\otimes_AN\morphism\ker\left(\delta\otimes\id_N\right)
			\end{align*}
			is an isomorphism for all $A$-modules $N$.
		\end{alphanumerate}
	\end{lem}
	\begin{proof}
		Part \itememph{a}. Let $\mu_1,\ldots,\mu_s\in M$ be elements whose images $\ov{\mu}_1,\ldots,\ov{\mu}_s$ in $M\otimes_A\KK(\pp)$ form a basis of that $\KK(\pp)$-vector space.\footnote{Why can we do this? Start with any basis of $M\otimes_A\KK(\pp)\cong M_\pp/\pp M_\pp$ and lift it to $M_\pp$. Then the numerators (which are elements of $M$) will do. We will use this argument multiple times (and without further mentioning) in the proofs to come.} Moreover, we may choose them in such a way that their images $\delta_\pp(\ov{\mu}_1),\ldots,\delta_\pp(\ov{\mu}_r)$ (for some $r\leq s$) in $F\otimes_A\KK(\pp)$ vanish while the images $\delta_\pp(\ov{\mu}_{r+1}),\ldots,\delta_\pp(\ov{\mu}_s)$ form a basis of $\Im\delta_\pp$. 
		
		Then we have $r=g(\pp)$. Since the assertions are all local near any given $\pp\in\Spec A$, we are free to replace $A$ by $A_f$ for $f\notin\pp$.
		\begin{claim}\lbl{claim:MatrixShit}
			Choosing $f$ appropriately we can assure that $\mu_1,\ldots,\mu_s$ generate $M$ as an $A$-module, and also that $\delta_\qq(\mu_{r+1}),\ldots,\delta_\qq(\mu_s)\in F\otimes_A\KK(\qq)$ are $\KK(\qq)$-linearly independent for all $\qq\in\Spec A$. 
		\end{claim}
		The first of these two assertions is a Nakayama-style argument that you can find in  \cite[Lemma~1.5.1]{alg2}. The latter, however, needs a bit more care, and we'll explain how this works now. Choose a basis of $F$ (which is free of rank $n$ by assumption) and represent $\delta(\mu_{r+1}),\ldots,\delta(\mu_s)$ as linear combinations of that basis. The resulting coefficient matrix $C\in A^{(s-r)\times n}$ has a $(s-r)\times (s-r)$-minor $D\in A^{(s-r)\times (s-r)}$ whose image in $\KK(\pp)^{(s-r)\times (s-r)}$ is invertible, because $\delta_\pp(\ov{\mu}_{r+1}),\ldots,\delta_\pp(\ov{\mu}_s)$ are $\KK(\pp)$-linearly independent. In particular, we must have $\det D\not\equiv 0\bmod \pp$, so we may localize $A$ at $\det D$ to make $D$ invertible over $A$. Then the images of $\mu_{r+1},\ldots,\mu_s$ under $\delta_\qq\colon M\otimes_A\KK(\qq)\morphism F\otimes_A\KK(\qq)$ are linearly independent for all $\qq\in\Spec A$. This justifies the above Claim~\reff{claim:MatrixShit}.
		
		Now we have $g(\qq)=\dim_{\KK(\qq)}\ker\delta_\qq=\dim_{\KK(\qq)}(M\otimes_A\KK(\qq))-\dim_{\KK(\qq)}\Im\delta_\qq$, in which the inequalities
		\begin{align}\lbl{eq:dimIneqs}
			\dim_{\KK(\qq)}(M\otimes_A\KK(\qq))\leq s\quad\text{and}\quad\dim_{\KK(\qq)}\Im\delta_\qq\geq s-r
		\end{align}
		hold true because of Claim~\reff{claim:MatrixShit}. Hence $g(\qq)\leq r$ for all $\qq\in\Spec A$, and this proves upper-semicontinuity. We're done with \itememph{a}.
		
		Part \itememph{b}. Replacing $A$ by some localization again, we may assume that $g(\qq)=r$ for all $\qq\in\Spec A$. Then equality must hold in both inequalities from \eqreff{eq:dimIneqs}. In particular, $\dim_{\KK(\qq)}(M\otimes_A\KK(\qq))=s$ for all $\qq\in\Spec A$, which implies that $M$ is locally free (of rank $s$) by the upcoming Lemma~\reff{lem:ReducedLocallyFree}. Similarly, we have
		\begin{align*}
			\dim_{\KK(\qq)}\left(\coker\delta\otimes_A\KK(\qq)\right)=\dim_{\KK(\qq)}\coker\delta_\qq&=\dim_{\KK(\qq)}\left(F\otimes_A\KK(\qq)\right)-\dim_{\KK(\qq)}\Im\delta_\qq\\
			&=n-s+r\;,
		\end{align*}
		so $\coker\delta$ is locally free by Lemma~\reff{lem:ReducedLocallyFree} again. In particular, $\coker\delta$ is projective, so $F\epimorphism\coker\delta$ splits and we get $F\cong \Im\delta\oplus\coker\delta$ (\emph{non}-canonically). Then $\Im\delta$ is projective as well, so $M\epimorphism\Im\delta$ splits as well, hence $M\cong \ker\delta\oplus\Im\delta$ and $\ker\delta$ lines up with the other projective modules. Therefore, $\delta$ has the form $\delta\colon \ker\delta\oplus\Im\delta\morphism\Im\delta\oplus\coker\delta$, and this nice splitting behaviour is preserved after tensoring with some $A$-module $N$. This shows \itememph{b}.
	\end{proof}
	Die-hard Franke fans might remember the next lemma from Franke's Algebra II lecture, where it appeared as a remark though; see \cite[25]{alg2}. We'll prove it here again.
	\begin{lem}\lbl{lem:ReducedLocallyFree}
		Let $A$ be a reduced noetherian ring and $M$ a finitely generated $A$-module, such that the function
		\begin{align*}
			\pp\longmapsto \dim_{\KK(\pp)}M\otimes_A\KK(\pp)
		\end{align*}
		is locally constant on $\Spec A$. Then $M$ is locally free.
	\end{lem}
	\begin{proof}
		Again, the question is local on $\Spec A$, so after localizing $A$ appropriately we may assume that $\dim_{\KK(\pp)}(M\otimes_A\KK(\pp))=\ell$ for all $\pp\in\Spec A$ and some $\ell\in\IN$. Choose some $\pp\in\Spec A$ and let $\mu_1,\ldots,\mu_\ell\in M$ be elements whose images form a basis of $M\otimes_\KK(\pp)$. Without losing generality (or rather after another localization, using \cite[Lemma~1.5.1]{alg2}) the $\mu_i$ generate $M$ as an $A$-module. We claim they do so freely. Indeed, if $\alpha_1\mu_1+\ldots+\alpha_\ell\mu_\ell=0$ for some $\alpha_i\in A$ is any relation in $M$, then $\alpha_i\in\qq$ for any $\qq\in\Spec A$ as $\dim_{\KK(\qq)}(M\otimes_A\KK(\qq))<\ell$ otherwise. Hence $\alpha_i\in\bigcap_{\qq\in\Spec A}\qq=\nil A=0$ and we're done.
	\end{proof}
	We come back to the proof of Theorem~\reff{thm:GrauertGrothendieck}\itememph{b} and \itememph{c}. The upper-semicontinuity part of \itememph{b} follows immediately from Lemma~\reff{lem:BasicallyThm4bc}\itememph{a} applied to $M=C^p$ und $F=P^{p+1}$. For part \itememph{c}, we may assume the base change was done along $\upsilon\colon \snake{Y}=\Spec B\morphism Y=\Spec A$ because the question is local on $Y$ and $\snake{Y}$. Then $R^p\snake{f}_*(\xi^*\Ff)$ is a quasi-coherent $\Oo_{\snake{Y}}$-module given by
	\begin{align*}
		H^p(\snake{X},\xi^*\Ff)\cong H^p(P^\bullet\otimes_AB)\cong \ker\left(\delta^p\otimes\id_B\colon C^p\otimes_AB\morphism P^{p+1}\otimes_AB\right)\;,
	\end{align*}
	and $\upsilon^*R^pf_*\Ff$ is given by
	\begin{align*}
		H^p(X,\Ff)\otimes_AB\cong H(P^\bullet)\otimes_AB\cong\ker\left(\delta^p\colon C^p\morphism P^{p+1}\right)\otimes_AB\;.
	\end{align*}
	That these two agree is precisely what Lemma~\reff{lem:BasicallyThm4bc}\itememph{c} says.
	
	For \itememph{d} and \itememph{e} we need yet another lemma, but I promise this will be the last technical commutative algebra lemma required to prove Theorem~\reff{thm:GrauertGrothendieck}.
	\begin{lem}\lbl{lem:BasicallyThm4de}
		Let $\pp\in \Spec A$, and $P^\bullet$, $C^\bullet$ be as above. The following are equivalent.
		\begin{alphanumerate}
			\item The canonical morphism
			\begin{align*}
				(\ker\delta^p)\otimes_A\KK(\pp)\morphism\ker\left(\delta^p_\pp=\delta^p\otimes \id_{\KK(\pp)}\colon C^p\otimes_A\KK(\pp)\morphism P^{p+1}\otimes_A\KK(\pp)\right)
			\end{align*}
			is surjective.
			\item Same as \itememph{a}, but the morphism is even bijective.
			\item The localization $C^{p+1}_\pp$ is a free $A_\pp$-module.
		\end{alphanumerate}
		Moreover, if the above conditions are satisfied, then $(\Im\delta^p)_\pp$ is free over $A_\pp$ as well.
	\end{lem}
\begin{proof}
	We use similar arguments as in Lemma~\reff{lem:BasicallyThm4bc}. Let's do the implication \itememph{a} $\Rightarrow$ \itememph{c} first. If assertion \itememph{a} holds, then we may choose $\mu_1,\ldots,\mu_r\in\ker \delta^p$ whose images $\ov{\mu}_i$ in $\ker\big(\delta^p\otimes\id_{\KK(\pp)}\big)$ form a basis. We may extend $\mu_1,\ldots,\mu_r$ by some $\mu_{r+1},\ldots,\mu_s\in C^p$ such that $\ov{\mu}_1,\ldots,\ov{\mu}_s$ form a basis of $C^p\otimes_A\KK(\pp)$. By the same argument as in Claim~\reff{claim:MatrixShit} we may replace $A$ by some localization to achieve that $\mu_1,\ldots,\mu_s$ generate $C^p$. Then $\Im \delta^p$ is generated by $\delta^p(\mu_{r+1}),\ldots,\delta^p(\mu_s)$, since the other generators are contained in $\ker \delta^p$. We claim that $\delta^p(\mu_{r+1}),\ldots,\delta^p(\mu_s)$ are (up to localizing $A$) actually \emph{free} generators of $\Im\delta^p$!
	
	Indeed, the images $\delta_\pp^p(\ov{\mu}_{r+1}),\ldots,\delta_\pp^p(\ov{\mu}_s)\in P^{p+1}\otimes_A\KK(\pp)$ are $\KK(\pp)$-linearly independent, so we may choose $\pi_1,\ldots,\pi_n\in P^{p+1}$ whose images in $P^{p+1}\otimes_A\KK(\pp)$ extend $\delta_\pp^p(\ov{\mu}_{r+1}),\ldots,\delta_\pp^p(\ov{\mu}_s)$ to a basis.
	By the same argument as in Claim~\reff{claim:MatrixShit}, we can replace $A$ by some localization so that $\delta^p(\mu_{r+1}),\ldots,\delta^p(\mu_s)$ and $\pi_1,\ldots,\pi_n$ generate $P^{p+1}$ -- and we may even assume they do so \emph{freely}! Indeed, write $\delta^p(\mu_{r+1}),\ldots,\delta^p(\mu_s)$ and $\pi_1,\ldots,\pi_n$ as linear combinations of some basis of $P^{p+1}$. Then the coefficient matrix $C$ becomes invertible when reduced to $\KK(\pp)$, hence after localizing $A$ at $\det C$ we may assume that $C$ is already invertible over $A$. Therefore, after replacing $A$ by a suitable localization, we see that $\Im\delta^p$ is a direct summand of $P^{p+1}$. Then $C^{p+1}=\coker\delta^p$ is a direct summand of $P^{p+1}\cong \Im\delta^p\oplus\coker\delta^p$ as well. This proves that $C^{p+1}$ is projective, hence $C^{p+1}_\pp$ is free over $A_\pp$. Thereby the implication \itememph{a} $\Rightarrow$ \itememph{c} has been shown.
	
	Now for \itememph{c} $\Rightarrow$ \itememph{b}. By \cite[Corollary~1.5.1]{alg2}, we may replace $A$ (and $C^{p+1}$) by some localization such that $C^{p+1}$ is already free over $A$. Then $P^{p+1}\epimorphism C^{p+1}=\coker\delta^p$ is split, hence $P^{p+1}\cong \Im\delta^p\oplus \coker\delta^p$ (non-canonically). Then $\Im\delta^p$ is projective as well, so $C^p\cong \ker\delta^p\oplus\Im\delta^p$ (non-canonically). That is, the morphism $\delta^p$ splits as $\delta^p\colon \ker\delta^p\oplus\Im\delta^p\morphism\Im\delta^p\oplus\coker\delta^p$, and this nice splitting behaviour is preserved after tensoring with $\KK(\pp)$. This shows that $\ker\delta^p\otimes_A\KK(\pp)\cong \ker\big(\delta^p\otimes\id_{\KK(\pp)}\big)$. In other words, \itememph{b} holds.
	
	Finally \itememph{b} $\Rightarrow$ \itememph{a} is trivial, and also we've already seen that $(\Im \delta^p)_\pp$ is free over $A_\pp$ in the proof of \itememph{a} $\Rightarrow$ \itememph{c}. We're done.
\end{proof}

	Now there isn't really much left to prove \itememph{d} and \itememph{e}. Since $Y=\Spec A$ is affine (or rather we reduced it to this case), $R^pf_*\Ff$ is the quasi-coherent $\Oo_Y$-module given by the $A$-module $H^p(X,\Ff)\cong H^p(P^\bullet)\cong \ker\delta^p$. Then Lemma~\reff{lem:BasicallyThm4de}\itememph{a}, \itememph{b} show that $(R^pf_*\Ff)(y)\morphism H^p(X_y,\Ff|_{X_y})$ is bijective iff it is surjective. Moreover, the set of $\pp\in\Spec A$, for which $C_\pp^{p+1}$ is free, is open by \cite[Corollary~1.5.1]{alg2}. Using Lemma~\reff{lem:BasicallyThm4de}\itememph{c} this implies that $U_p\subseteq Y$ is open, proving \itememph{d}.
	
	Finally, part \itememph{e}. If $y\in U_p$ corresponds to $\pp\in\Spec A$, then $C_\pp^{p+1}$ as well as $(\Im \delta^p)_\pp$ are free over $A_\pp$ by Lemma~\reff{lem:BasicallyThm4de}\itememph{c}. Therefore the short exact sequence
	\begin{align*}
		0\morphism H^p(X,\Ff)_\pp\morphism C_\pp^p\morphism (\Im\delta^p)_\pp\morphism 0
	\end{align*}
	is split. This shows that $C_\pp^p$ is free over $A_\pp$ iff so is $H^p(X,\Ff)_\pp$. Using Lemma~\reff{lem:BasicallyThm4de}\itememph{c} we thus see that $y\in U_{p-1}$ iff $H^p(X,\Ff)$ is locally free at $\pp$ -- and that's precisely what we want.
\end{proof}
\begin{cor}
	In the situation of Theorem~\reff{thm:GrauertGrothendieck}, if $h^p(y,\Ff)=0$, then $R^pf_*\Ff$ vanishes in some neighbourhood of $y$.
\end{cor}
\begin{proof}
	By Theorem~\reff{thm:GrauertGrothendieck}\itememph{d} we have $(R^pf_*\Ff)(y)=0$, hence $R^pf_*\Ff=0$ near $y$ by Nakayama-style arguments (to be precise, this follows from \cite[Corollary~1.5.1]{alg2}).
\end{proof}
\begin{cor}
	If $\upsilon\colon \snake{Y}\morphism Y$ is a morphism with image contained in $U_p$, and $\xi\colon \snake{X}=X\times_Y\snake{Y}\morphism X$ its base change along $f$, then the base change morphism is an isomorphism
	\begin{align*}
		\upsilon^*R^pf_*\Ff\isomorphism R^p\snake{f}_*(\xi^*\Ff)\;.
	\end{align*}
\end{cor}
\begin{proof}
	The assertion is local on $Y$ and $\snake{Y}$, so let's assume $Y=\Spec A$ and $\snake{Y}=\Spec B$. We may assume that $\upsilon$ factors over $\Spec A_\alpha\monomorphism \Spec A$ for some $\alpha\in A$ such that $\Spec A\setminus V(\alpha)\subseteq U_p$. Then
	\begin{align*}
		H^p(\snake{X},\xi^*\Ff)\cong H^p\left(P^\bullet\otimes_AB\right)\cong H^p\left(P_\alpha^\bullet\otimes_{A_\alpha}B\right)
	\end{align*}
	and
	\begin{align*}
		H^p(X,\Ff)\otimes_AB\cong H^p(P^\bullet)\otimes_AB\cong H^p(P_\alpha^\bullet)\otimes_{A_\alpha}B\;.
	\end{align*}
	Since Lemma~\reff{lem:BasicallyThm4de} is applicable to all $\pp\in \Spec A_\alpha$, we see that $Q_\alpha^{p+1}$ is projective. As in the proof of Lemma~\reff{lem:BasicallyThm4de}, this implies that the localization $\delta_\alpha^p=\delta^p\otimes_A\id_{A_\alpha}$ of $\delta^p$ splits as $\delta_\alpha^p\colon \ker\delta_\alpha^p\oplus \Im \delta_\alpha^p\morphism \Im\delta_\alpha^p\oplus\coker\delta_\alpha^p$, hence it behaves well under tensoring with the $A_\alpha$-algebra $B$. Thus $H^p(X,\Ff)\otimes_AB\morphism H^p(\snake{X},\xi^*\Ff)$ is an isomorphism, which is precisely what we need.
\end{proof}
\section{Application to the group structure of elliptic curves}
\begin{defi}\lbl{def:flatFamily}
	A \defemph{flat family} of curves of genus $g$ is a (faithfully -- this will be automatic since the fibres are non-empty) flat proper morphism $f\colon C\morphism S$ whose \defemph{geometric fibres}
	\begin{align*}
		C_{\ov{s}}=C\times_S\Spec\ov{\KK(s)}
	\end{align*}
	are regular connected curves of genus $g$ over the algebraic closure $\ov{\KK(s)}$ for all $s\in S$. If $g=1$ and in addition a section  $\sigma\colon S\morphism C$  of $f$ is given, we call $(f,\sigma)$ an \defemph{elliptic curve} over $S$.
\end{defi}
We will typically assume $S$ to be locally noetherian, although Professor Franke says one could generalize the situation to allow arbitrary $S$ if $f$ is of \emph{finite presentation}.
\begin{prop}\lbl{prop:RpLl}
	Let $f\colon C\morphism S$ be a flat family of curves of genus $g$, where $S$ is locally noetherian. Moreover, let $\Ll$ be a line bundle on $C$ whose pullbacks to the geometric fibres of $f$ have degree $d$.
	\begin{alphanumerate}
		\item If $d>2g-2$, then $f_*\Ll$ is a vector bundle of dimension $d+1-g$ on $S$, and $R^1f_*\Ll=0$.
		\item If $d<0$, then $f_*\Ll$ vanishes, and $R^1f_*\Ll$ is a vector bundle of dimension $g-1-d$.
		\item If $\Ll=\Oo_C$ (hence $d=0$), then $f^*\colon \Oo_S\morphism f_*\Oo_C$ is an isomorphism and $R^1f_*\Oo_C$ is a vector bundle of dimension $g$.
	\end{alphanumerate}
	Moreover, if we take a base change
	\begin{diagram*}
		\object{0,0}{$C$}[a];
		\object{0,1.5}{$\snake{C}$}[b];
		\object{2,0}{$S$}[c];
		\object{2,1.5}{$\snake{S}$}[d];
		\pullback{1,0.75};
		\scriptsize
		\arrow ba[left][$\gamma$];
		\arrow ac[above][$f$];
		\arrow dc[right][$\varsigma$];
		\arrow bd[above][$\snake{f}$];
	\end{diagram*}
	then the base change morphism $\varsigma^*R^pf_*\Ll\morphism R^p\snake{f}_*(\gamma^*\Ll)$ is an isomorphism for $p\in\{0,1\}$ in each of the above cases.
\end{prop}
\begin{rem}
	\begin{alphanumerate}
		\item \lbl{rem:GeometricFibres}First of all, if $\Ff$ is any quasi-coherent $\Oo_C$-module, then
		\begin{align}\lbl{eq:GeometricFibreIso}
		H^p\left(C_{\ov{s}},\Ff|_{C_{\ov{s}}}\right)\cong H^p\left(C_s,\Ff|_{C_s}\right)\otimes_{\KK(s)}\ov{\KK(s)}
		\end{align}
		for all $s\in S$. Indeed, $\ov{\KK(s)}/\KK(s)$ is a (pretty lame) flat ring extension, so \eqreff{eq:GeometricFibreIso} follows immediately from \eqreff{eq:tensoredCohomology}.
		\item Note that the degree of the pullbacks of $\Ll$ to the geometric fibres of $f$ is locally constant on $S$. Indeed, by Riemann--Roch (in the form of \cite[Theorem~8]{alggeo2}) and \itememph{a} we have
		\begin{align*}
			\deg(\Ll|_{C_{\ov{s}}})=\chi\left(C_{\ov{s}},\Ll|_{C_{\ov{s}}}\right)+g-1=\chi\left(s,\Ll\right)+g-1\;,
		\end{align*}
		where $\chi(s,\Ll)$ on the right-hand side is the Euler--Poincaré characteristic from Theorem~\reff{thm:GrauertGrothendieck}\itememph{c}. But $\chi(-,\Ll)$ is locally constant on $S$. So the assumption of Proposition~\reff{prop:RpLl} is not really restrictive at all.
	\end{alphanumerate}
\end{rem}
\begin{proof}[Proof of Proposition~\reff{prop:RpLl}]
	Throughout the proof, we may assume $S=\Spec A$ for some noetherian ring $A$ because all assertions are local. Also Theorem~\reff{thm:GrauertGrothendieck} is applicable since $f$ is flat, so every line bundle on $C$ is flat over $\Oo_S$. Let $P^\bullet$ be the complex from Theorem~\reff{thm:GrauertGrothendieck}\itememph{a}, concentrated within minimum and maximum of the set
	\begin{align*}
		\DD&=\left\{p\in\IN\st H^p\left(C_s,\Ll|_{C_s}\right)\neq 0\text{ for some }s\in S\right\}\\
		&=\left\{p\in\IN\st H^p\left(C_{\ov{s}},\Ll|_{C_{\ov{s}}}\right)\neq 0\text{ for some }s\in S\right\}
	\end{align*}
	(where we use Lemma~\reff{lem:TechnicalVanishingAssertion} in combination with \eqreff{eq:GeometricFibreIso}). Note that $\DD=\{0\}$ in the case of \itememph{a} and $\DD=\{1\}$ in the case of \itememph{b}. Indeed, in both cases $\DD$ is contained in $\{0,1\}$ by Grothendieck's theorem on cohomological dimension (cf.\ \cite[Proposition~1.4.1]{alggeo2}), and then Lemma~\reff{lem:Hpvanishing} determines $\DD$ (ok, actually it doesn't, but we'll see immediately why $\DD=\emptyset$ is impossible). Then $P^\bullet$ consists of at most one non-zero term, which defines a vector bundle on $S=\Spec A$. The assertions about dimension follow immediately from Riemann--Roch in the form of \cite[equation~(3.1.3)]{alggeo2}.
	
	For \itememph{c}, let $U_p\subseteq S$ be as in Theorem~\reff{thm:GrauertGrothendieck}\itememph{d} and \itememph{e}. %By Serre duality -- or more precisely, \cite[Theorem~7\itememph{b}, \itememph{c}]{alggeo2} -- we see that
	Since $C_{\ov{s}}$ is proper and integral over $\ov{\KK(s)}$, we see that $H^0\left(C_{\ov{s}},\Oo_{C_{\ov{s}}}\right)$ is a domain and a $\ov{\KK}(s)$-vector space of finite dimension. Since $\ov{\KK(s)}$ is algebraically closed, we deduce that
	\begin{align*}
		H^0\left(C_{\ov{s}},\Oo_{C_{\ov{s}}}\right)\cong \ov{\KK(s)}
	\end{align*}
	is a one-dimensional vector space with the constant function $1$ as a basis, and the same holds thus for $H^0(C_s,\Oo_{C_s})$ by \eqreff{eq:GeometricFibreIso}. Therefore, the canonical morphism
	\begin{align}\lbl{eq:H0OC}
		(f_*\Oo_C)(s)\cong\big(R^0f_*\Oo_C\big)(s)\morphism H^0(C_s,\Oo_{C_s})
	\end{align}
	is an surjective for all $s\in S$ (because the constant function $1$ is always hit), so $U_0=S$ by Theorem~\reff{thm:GrauertGrothendieck}\itememph{d}. But $U_{-1}=S$ for trivial reasons, hence $f_*\Oo_C$ is a vector bundle by Theorem~\reff{thm:GrauertGrothendieck}\itememph{e}. Moreover, from \eqreff{eq:H0OC} (which is actually an isomorphism, as we now know from Theorem~\reff{thm:GrauertGrothendieck}\itememph{d}) we see that $(f_*\Oo_C)(s)\cong (f_*\Oo_C)_s\otimes_{\Oo_{S,s}}\KK(s)$ is a one-dimensional $\KK(s)$-vector space with $1\otimes 1$ as a basis for all $s\in S$. Using \cite[Lemma~1.5.1]{alg2}, this implies that $f_*\Oo_C$ is generated by $1\in\Oo_C(C)$ as an $\Oo_S$-algebra. But $f_*\Oo_C$ is a vector bundle, so $f^*\colon \Oo_S\morphism f_*\Oo_C$ must be indeed an isomorphism.
	
	To show the second part of \itememph{c}, note that $H^2(C_s,\Oo_{C_s})=0$ for all $s\in S$ (by Grothendieck's dimension theorem), so $U_2=S$. Then $(R^2f_*\Oo_C)(s)=0$ for all $s\in S$, which proves $R^2f_*\Oo_C=0$ (this uses \cite[Lemma~1.5.1]{alg2} again). In particular, $R^2f_*\Oo_C$ is a vector bundle, and Theorem~\reff{thm:GrauertGrothendieck}\itememph{e} gives $U_1=S$. Since $U_0=S$ as well, reversing the argument shows that $R^1f_*\Oo_C$ is a vector bundle too. The fact that it has dimension $g$ can be seen as follows. We have
	\begin{align*}
		\dim R^1f_*\Oo_C=\dim_{\KK(s)}(R^1f_*\Oo_C)(s)=\dim_{\ov{\KK(s)}}H^1\left(C_{\ov{s}},\Oo_{C_{\ov{s}}}\right)
	\end{align*}
	by \eqreff{eq:GeometricFibreIso} and Theorem~\reff{thm:GrauertGrothendieck}\itememph{d}. Now Serre duality gives
	\begin{align*}
		\dim_{\ov{\KK(s)}}H^1\left(C_{\ov{s}},\Oo_{C_{\ov{s}}}\right)=\dim_{\ov{\KK(s)}}H^0\left(C_{\ov{s}},\Omega_{C_{\ov{s}}}\right)=g\;,
	\end{align*}
	in which the right-most equality holds simply by definition (indeed, $g$ is supposed to be the $\ov{\KK(s)}$-dimension of $\Gamma(C_{\ov{s}},\Omega_{C_{\ov{s}}})\cong H^0(C_{\ov{c}},\Omega_{C_{\ov{s}}})$). This proves \itememph{c}.
	
	Finally, let's see why the additional base change assertion holds. For \itememph{a} and \itememph{b}, we've seen that $P^\bullet$ has only one non-vanishing term, so the required isomorphism is immediate. However, I wasn't able to find a equally down-to-the-earth argument in the case of \itememph{c}. Instead, consider the special case $S=\Spec A$ and $\snake{S}=\Spec B$ (the assertion is local, so that's fine). Then the base change spectral sequence (cf.\ Proposition~\reff{prop:baseChangeMorphismA})
	\begin{align*}
		E_2^{p,q}=\Tor_{-p}^A\big(H^q(C,\Oo_C),B\big)\converge H^{p+q}(\snake{C},\Oo_{\snake{C}})\;.
	\end{align*}
	immediately degenerates because we just showed that $f_*\Oo_C$ and $R^1f_*\Oo_C$ are vector bundles (and the higher derived images vanish anyway), so the base change morphism is an isomorphism in this case as well.
\end{proof}
\begin{rem}
	When $g=d=1$ we have a line bundle $\Ll_0=f_*\Ll$ on $S$ with a canonical morphism $f^*\Ll_0=f^*f_*\Ll\morphism\Ll$ as $f^*$ is left-adjoint to $f_*$. It will turn out (in Proposition~\reff{prop:canonicalSection}) that there is a unique section $\sigma\colon S\morphism C$ such that said canonical morphism constitutes an isomorphism $f^*\Ll_0\isomorphism\Ii\Ll$, where $\Ii$ is the sheaf of ideals defined by $\sigma\colon S\monomorphism C$, which can be regarded as a closed subprescheme of $C$.
	
	Wait \ldots\ why is $\sigma$ a closed embedding? This is because $f$ is proper, hence separated, so any section of it is a closed embedding by \cite[Proposition~1.5.5]{alggeo1}.
\end{rem}
\begin{lem}\lbl{lem:I(s)LineBundle}
	Let $f\colon C\morphism S$ be a flat family of curves and let $\sigma\colon S\morphism C$ be a section of $f$. Let $\Ii=\Ii(\sigma)\subseteq \Oo_C$ denote the sheaf of ideals defining the closed subprescheme $\sigma(S)$ of $C$. Then $\Ii$ is a line bundle.
\end{lem}
\begin{proof}
	The assertion is local, hence it is sufficient to prove it near any $c\in C$. When $c\neq \sigma(f(c))$ (in other words, when $c\in C\setminus \sigma(S)$) this is trivial since $\Ii$ then equals $\Oo_C$ in some neighbourhood of $c$. Thus let $c=\sigma(s)$ where $s=f(c)$. We have a short exact sequence
	\begin{align}\lbl{eq:s(S)closedSES}
		0\morphism \Ii_c\morphism \Oo_{C,c}\morphism[\sigma^*]\Oo_{S,s}\morphism 0
	\end{align}
	(because $S\cong \sigma(S)$ is the closed subprescheme cut out by $\Ii$) of $\Oo_{S,s}$-modules. Since $\Oo_{C,c}$ is flat over $\Oo_{S,s}$ by assumption, this already shows that $\Ii_c$ is a flat module over $\Oo_{S,s}$ (with the $\Oo_{S,s}$-action given by $f^*$) via the long exact $\Tor$ sequence. Also we see that $\Oo_{C,c}$ and $\Oo_{S,s}\cong \Oo_{C,c}/\Ii_c$ have the same residue field $k$.
	
	To show local freeness of $\Ii$, we wish to show that $\Tor_1^{\Oo_{C,c}}(k,\Ii_c)=0$. Since the algebraic closure $\ov{k}$ is (as an $\Oo_{C,c}$-module) isomorphic to a direct sum of copies of $k$, we may equivalently show that $\Tor_1^{\Oo_{C,c}}(\ov{k},\Ii_c)=0$. To do so, consider the (homological) spectral sequence\footnote{If you think using spectral sequences is a little bit overkill, have a look at the proof of Lemma~\reff{lem:CartierDivisor}.}
	\begin{align}\lbl{eq:TorSS1}
		E_{p,q}^2=\Tor_p^{\Oo_{C_{\ov{s}},c}}\left(\ov{k},\Tor_q^{\Oo_{C,c}}(\Ii_c,\Oo_{C_{\ov{s}},c})\right)\converge\Tor_{p+q}^{\Oo_{C,c}}(\ov{k},\Ii_c)\;,
	\end{align}
	where $c$ is used to denote the unique preimage of itself in $C_{\ov{s}}$ as well (which makes sense by Lemma~\reff{lem:stalkOfGeometricFibres}).	One obtains \eqreff{eq:TorSS1} as a special case of the general spectral sequence
	\begin{align}\lbl{eq:TorGrothendieckSS}
		E_{p,q}^2=\Tor_p^B\left(N,\Tor_q^A(M,B)\right)\converge\Tor_{p+q}^A(M,N)\;,
	\end{align}
	in which $A$ may be any ring, $B$ an $A$-algebra, $M$ an $A$-module and $N$ a $B$-module. The sequence \eqreff{eq:TorGrothendieckSS} is in turn a special case of the Grothendieck spectral sequence. 
	
	Another special case of \eqreff{eq:TorGrothendieckSS} using $A=\Oo_{S,s}$, $B=\Oo_{C,c}$ (which becomes an $A$-module via $f^*$), $M=\ov{k}$, and $N$ arbitrary (for now) is
	\begin{align}\lbl{eq:TorSS2}
		E_{p,q}^2=\Tor_p^{\Oo_{C,c}}\left(N,\Tor_q^{\Oo_{S,s}}(\ov{k},\Oo_{C,c})\right)\converge\Tor_{p+q}^{\Oo_{S,s}}(\ov{k},N)\;.
	\end{align}
	But since $\Oo_{C,c}$ is flat over $\Oo_{S,s}$, we have $E_{p,q}^2=0$ unless $q=0$, so \eqreff{eq:TorSS2} collapses to an isomorphism
	\begin{align}\lbl{eq:TorSS3}
	 	\Tor_p^{\Oo_{C,c}}(N,\Oo_{C_{\ov{s}},c})\cong\Tor_p^{\Oo_{C,c}}(N,\Oo_{C,c}\otimes_{\Oo_{S,s}}\ov{k})\cong\Tor_p^{\Oo_{S,s}}(\ov{k},N)
	\end{align}
	(the isomorphism on the left-hand side is due to Lemma~\reff{lem:stalkOfGeometricFibres}). Plugging in $N=\Ii_c$, which is a flat $\Oo_{S,s}$-module, we obtain $\Tor_p^{\Oo_{C,c}}(N,\Oo_{C_{\ov{s}},c})=0$ when $p\neq 0$. Thus \eqreff{eq:TorSS1} immediately degenerates to an isomorphism
	\begin{align*}
		\Tor_p^{\Oo_{C_{\ov{s}},c}}(\ov{k},\Ii_c\otimes_{\Oo_{C,c}}\Oo_{C_{\ov{s}},c})\cong\Tor_p^{\Oo_{C,c}}(\ov{k},\Ii_c)\;.
	\end{align*}
	Applying \eqreff{eq:TorSS3} to $N=\Ii_c$ and $p=0$ gives $\Ii_c\otimes_{\Oo_{C,c}}\Oo_{C_{\ov{s}},c}\cong \Ii_c\otimes_{\Oo_{S,s}}\ov{k}$. Since $\Ii_c$ is flat over $\Oo_{S,s}$, \eqreff{eq:s(S)closedSES} gives an exact sequence
	\begin{align*}
		0=\Tor_1^{\Oo_{S,s}}(\ov{k},\Ii_c)\morphism \Ii_c\otimes_{\Oo_{S,s}}\ov{k}\morphism\Oo_{C,c}\otimes_{\Oo_{S,s}}\ov{k}=\Oo_{C_{\ov{s}},c}\;,
	\end{align*}
	hence $\Ii_c\otimes_{\Oo_{S,s}}\ov{k}$ is isomorphic to some ideal of $\Oo_{C_{\ov{s}},c}$. But the latter is a DVR (because we assume our fibres to be regular curves), so all ideals are projective, which finally shows $\Tor_p^{\Oo_{C,c}}(\ov{k},\Ii_c)=0$, so $\Tor_p^{\Oo_{C,c}}(k,\Ii_c)=0$. Therefore, $\Ii_c$ is a free $\Oo_{C,c}$-module by \cite[Proposition~1.3.1]{homalg}, which shows that $\Ii$ is locally free near $c$ (for which we need noetherianness of $C$ to have something like \cite[Corollary~1.5.1]{alg2} available).
	
	It remains to show that $\rank_{\Oo_C} \Ii=1$. If $c\notin \sigma(S)$, then $\Ii_c=\Oo_{C,c}$, so $\rank_{\Oo_{C,c}}\Ii_c=1$ is trivial. If $c\in s(S)$ and $\Ii_c\cong \Oo_{C,c}^{\oplus n}$ for some $n\geq 0$, then $\Ii_c\otimes_{\Oo_{C,c}}\Oo_{C_{\ov{s}},c}\cong \Oo_{C_{\ov{s}},c}^{\oplus n}$. But we've seen above that this is isomorphic to some ideal in the DVR $\Oo_{C_{\ov{s}},c}$, hence $n=1$.
\end{proof}
\begin{rem}
	As was already mentioned, equation \eqreff{eq:TorSS2} is a special case of the \emph{Grothendieck spectral sequence}
	\begin{align*}
		E_{p,q}^2=L_pG\big(L_qF(X)\big)\converge L_{p+q}(GF)(X)\;.
	\end{align*}
	This requires $F\colon \cat A\morphism \cat B$ and $G\colon \cat B\morphism\cat C$ to be right-exact functors between abelian categories, of which $\cat A$ and $\cat B$ have enough projectives (for the left-derived functors to exist) and $F$ takes projective objects in $\cat A$ to $G$-acyclic objects in $\cat B$; $X$ may be any object of $\cat A$.
\end{rem}
\begin{rem}
	If the geometric fibres $C_{\ov{s}}$ are regular curves, then so are the ordinary fibres $C_s$ (Franke didn't prove this in the lecture, but we do in Proposition~\reff{prop:GeometricFibres101}). In the proof of Lemma~\reff{lem:I(s)LineBundle} we could have worked with $C_s$ instead of $C_{\ov{s}}$ as well, so the assertion still holds if $f\colon C\morphism S$ is only assumed flat with one-dimensional regular fibres.
\end{rem}
\begin{lem}\lbl{lem:I(s)degree-1}
	The ideal $\Ii=\Ii(\sigma)$ in Lemma~\reff{lem:I(s)LineBundle} has automatically (geometric and ordinary) fibre-wise degree $-1$ -- that is, its pullbacks to the geometric and ordinary fibres have degree $-1$.
\end{lem}
\begin{proof}
	Indeed, let $i_s\colon C_s\morphism C$ and $i_{\ov{s}}\colon C_{\ov{s}}\morphism C$ be the inclusion of the fibres. If $D=\sigma(S)$ is the closed subprescheme cut out by $\Ii$, then $D\cap C_s=\{\sigma(s)\}$ has precisely one point for all $s\in S$. Put $d=\sigma(s)$ and $k=\KK(s)$. Since $\Ii=\Oo_C$ outside $D$, it follows that $\Ii_c\otimes_{\Oo_{S,s}}k\cong \Oo_{C_s,c}$ for all $c\neq d$. Moreover, $\Ii_d\otimes_{\Oo_{S,s}}k$ is a maximal ideal of $\Oo_{C_s,d}\cong \Oo_{C,c}\otimes_{\Oo_{S,s}}k$ since
	\begin{align*}
		(\Oo_{C,d}\otimes_{\Oo_{S,s}}k)/(\Ii_d\otimes_{\Oo_{S,s}}k)\cong (\Oo_{C,d}/\Ii_d)\otimes_{\Oo_{S,s}}k\cong \Oo_{S,s}\otimes_{\Oo_{S,s}}k\cong k\;.
	\end{align*}
	Thus $v_c(i_s^*\Ii)=0$ if $c\neq d$ and $v_d(i_s^*\Ii)=1$, which means that $i_s^*\Ii$ has degree $-1$.
	
	For the geometric fibre, we can use the exact same argument since $C_{\ov{s}}$ has a unique point over $d$ (which we denote $d$ too, as before) by Lemma~\reff{lem:stalkOfGeometricFibres}, and $\Ii_d\otimes_{\Oo_{C,d}}\Oo_{C_{\ov{s}},d}\cong \Ii_d\otimes_{\Oo_{S,s}}\ov{k}$ is a maximal ideal of $\Oo_{C_{\ov{s}},d}\cong \Oo_{C,d}\otimes_{\Oo_{S,s}}\ov{k}$ by the same reason as above, the only difference being that the quotient is $\ov{k}$ instead of $k$ this time.
\end{proof}
\begin{defi}\lbl{def:OC(s)}
	In the situation where $\sigma\colon S\monomorphism C$ defines an invertible sheaf of ideals $\Ii(\sigma)$, we put $\Oo_C(\sigma)=\Ii(\sigma)^{-1}$.
\end{defi}
\begin{rem}
	We will eventually generalize this to $\Oo_C(D)$, where $D\subseteq C$ is finite and flat over $S$. The letter $D$ is purposefully chosen: You should think of $D$ as an \emph{effective relative Cartier divisor} (see Definition~\reff{def:CartierDivisor}). Professor Franke suggests you read the first chapter of Katz/Mazur, \emph{Arithmetic Moduli of Elliptic Curves} \cite{katzmazur} -- in particular, the definition of effective relative Cartier divisors and the group structure on elliptic curves (which we are going to sketch now) are explained there.
\end{rem}
\begin{prop}\lbl{prop:canonicalSection}
	Let $f\colon C\morphism S$ (with $S$ locally noetherian) be a flat family of curves of genus $g=1$ and $\Ll$ a line bundle of geometric fibre-wise degree $1$ on $C$. Then $\Ll_0=f_*\Ll$ is a line bundle on $S$, and there is a unique section $\sigma\colon S\morphism C$ of $f$ such that the canonical morphism $f^*\Ll_0=f^*f_*\Ll\morphism\Ll$ (corresponding to $\id_{f_*\Ll}$ under the $f^*$-$f_*$ adjunction) is a monomorphism with image $\Ii(\sigma)\Ll$. In particular, we have a canonical isomorphism
	\begin{align*}
		f^*\Ll_0\otimes_{\Oo_C}\Oo_C(\sigma)\isomorphism\Ll\;,
	\end{align*}
	using that $\Ii(\sigma)\otimes_{\Oo_C}\Oo_C(\sigma)\cong \Oo_C$ by Definition~\reff{def:OC(s)}.
\end{prop}
\begin{proof}
	The assertion is local\footnote{\textbf{Attention!} This is not a priori obvious, but can be seen as follows: By the uniqueness assertion we see that local sections of $f$ with the required property glue well to a section $\sigma$ defined on all of $S$ (and $\sigma$ is automatically unique). Also the isomorphism $f^*\Ll_0\otimes_{\Oo_C}\Oo_C(\sigma)\cong\Ll$ is going to be canonical, hence gluing shouldn't be a problem.} with respect to $S$, hence we may assume that $S=\Spec A$ is affine. Also the claim that $\Ll_0=f_*\Ll$ is a line bundle is just Proposition~\reff{prop:RpLl}\itememph{a}, so without losing generality $\Ll_0$ is trivial with $\lambda\in\Ll_0(S)$ a free generator. Then $\lambda$ is also an element of $f_*\Ll(S)=\Ll(C)$ and it's easy to check that $f^*f_*\Ll$ is isomorphic to the subbundle of $\Ll$ generated by $\lambda$. Let $i\colon D\monomorphism C$ be the closed subscheme defined by the vanishing set of $\lambda$, or in other words, by the short exact sequence
	\begin{align}\lbl{eq:SESforD}
		0\morphism\Ll^{\otimes -1}\xrightarrow{\lambda\otimes-}\Oo_C\morphism i_*\Oo_D\morphism 0\;.
	\end{align}
	\begin{claim}\lbl{claim:DcongS}
		The composition $q=fi\colon D\morphism S$ is an isomorphism.
	\end{claim}
	Believing this for the moment, the proof can be finished as follows. Note that \eqreff{eq:SESforD} induces an isomorphism $\Ll^{\otimes -1}\isomorphism\Jj$ onto some quasi-coherent sheaf of ideals $\Jj\subseteq \Oo_C$. One readily checks that $\Jj$ is the uniquely defined ideal such that $\Jj\Ll$ is the subbundle generated by $\lambda$, i.e.\ the image of $f^*f_*\Ll$. So $\sigma$ -- if it exists -- necessarily identifies $S$ with $D$ via the composition $q=fi$. This settles the uniqueness part, but also existence is immediate: Since we know that $q$ is an isomorphism, we \emph{actually can} choose $\sigma=iq^{-1}\colon S\isomorphism D\monomorphism C$, which gives $\Ii(\sigma)=\Jj$ whence we're done.
	
	To show that $q$ is an isomorphism, first apply the long exact cohomology sequence for $R^\bullet f_*$ to \eqreff{eq:SESforD} to obtain a six-term exact sequence
	\begin{align*}%\lbl{eq:sixtermRpSequence}
		0\morphism f_*\Ll^{\otimes -1}\morphism f_*\Oo_C\morphism f_*i_*\Oo_D\morphism R^1f_*\Ll^{\otimes -1}\morphism R^1f_*\Oo_C\morphism R^1f_*(i_*\Oo_D)\morphism 0\;.
	\end{align*}
	The $0$ on the right end comes from the fact that $R^2f_*\Ll^{\otimes -1}=0$ (we've seen this for $\Ll^{\otimes -1}=\Oo_C$ in the proof of Proposition~\reff{prop:RpLl}\itememph{c}; it's just the same for arbitary $\Ll^{\otimes -1}$). Also note that $f_*\Ll^{\otimes -1}=0$ and $f_*\Oo_C\cong\Oo_S$ by Proposition~\reff{prop:RpLl}\itememph{b} and \itememph{c}.
	\begin{claim}\lbl{claim:DCxfiniteIntersection}
		For every $s\in S$, the intersection $D\cap C_s=q^{-1}\{s\}$ is finite.
	\end{claim}
	To prove Claim~\reff{claim:DCxfiniteIntersection}, we first remark that $H^0(C_s,\Ll|_{C_s})$ is a one-dimensional $\KK(s)$-vector space. Indeed, to show this, it suffices to prove the same for $H^0(C_{\ov{s}},\Ll|_{C_{\ov{s}}})$ by Remark~\reff{rem:GeometricFibres}\itememph{a}. Using Riemann--Roch and $\deg(\Ll|_{C_{\ov{s}}})=1$, this is equivalent to $H^1(C_{\ov{s}},\Ll|_{C_{\ov{s}}})=0$. But we have $\deg(\Ll|_{C_{\ov{s}}})=1>0=2g-2$, so $H^1(C_{\ov{s}},\Ll|_{C_{\ov{s}}})=0$ follows from Lemma~\reff{lem:Hpvanishing}.
	
	Now $\lambda|_{C_s}$ is a generator of the one-dimensional $\KK(s)$-vector space $H^0(C_s,\Ll|_{C_s})$ by the base change assertion from Proposition~\reff{prop:RpLl}. In particular, $\lambda|_{C_s}\neq 0$. Since $C_s$ is a regular connected curve by Corollary~\reff{cor:FibresAreCurvesToo}, thus reduced, irreducible, and one-dimensional, this shows that $D\cap C_s\subsetneq C_s$, hence $D\cap C_s$ must be zero-dimensional. Then it is already finite. Indeed, $D\cap C_s$ is quasi-compact (as a closed subscheme of the irreducible $C_s$), hence can be covered by finitely many $U_i=\Spec R_i$ where $R_i$ is some zero-dimensional noetherian ring. But such a ring has finitely many prime ideals, as follows e.g.\ from Step~2 in the proof of \cite[Proposition~3.1.1]{alg2}. This settles Claim~\reff{claim:DCxfiniteIntersection}.
	
	In particular, we see that $q$ is quasi-finite. It is also proper because $q=fi$, wherein $f$ is proper and $i$ a closed embedding, hence $q$ is even finite by Theorem~\reff{thm:ZariskiMain}\itememph{a}. As $i$ is affine, being a closed embedding, we see that $R^1q_*\Oo_D\cong R^1f_*(i_*\Oo_D)$ (this follows e.g.\ from Leray's spectral sequence $R^pf_*R^ri_*\Oo_D\converge R^{p+r}(fi)_*\Oo_D$, which degenerates as $R^ri_*\Oo_D=0$, $i$ being affine). But $q$ is affine as well, so $R^1f_*(i_*\Oo_D)=0$. Thus the above six-term sequence reduces to 
	\begin{align}
		0\morphism \Oo_S\morphism q_*\Oo_D\morphism R^1f_*\Ll^{\otimes -1}\morphism R^1f_*\Oo_C\morphism 0\;.
	\end{align}
	By Proposition~\reff{prop:RpLl}\itememph{b} and \itememph{c}, we see that $R^1f_*\Ll^{\otimes -1}$ and $R^1f_*\Oo_C$ are line bundles. But every epimorphism of line bundles must be an isomorphism (which is easily checked affine-locally), so $\Oo_S\morphism q_*\Oo_D$ is an isomorphism of $\Oo_S$-modules. But then it is an isomorphism of $\Oo_S$-algebras. Using that $q$ is affine, we thus obtain $D\cong \SPEC (q_*\Oo_D)\cong \SPEC\Oo_S\cong S$. This shows Claim~\reff{claim:DcongS} and we're done.
\end{proof}
\begin{rem}
	Note that despite Proposition~\reff{prop:canonicalSection}, a flat family $f\colon C\morphism S$ of genus $1$ curves may fail to have a section (think of an elliptic curve without a rational point over $\IQ$). An elliptic curve $E$ over $S$ may always be obtained from $C$, but there may be several non-isomorphic $C$ giving the same $E$. When $S=\Spec K$ where $K/\IQ$ is a number field, then all these $C$ are isomorphic over $\ov{\IQ}$ and their isomorphism classes (over $K$) form a group, the \defemph{Tate--Shafarevich group} $\Sha(E/K)$.
\end{rem}
	\paragraph{The group structure on elliptic curves.}\lbl{par:EllipticCurves}Let $f\colon E\morphism S$ with a section $\epsilon\colon S\morphism E$ be an elliptic curve over $S$. Then Proposition~\reff{prop:canonicalSection} may be used to obtain a bijection
	\begin{align*}
		\Pic^0(E/S)\coloneqq\left\{\begin{tabular}{c}
			isomorphism classes of line bundles\\
			$\Ll$ over $E$ of geometric fibre-wise de-\\
			gree 0 with a trivialization $\epsilon^*\Ll\cong\Oo_S$
		\end{tabular}\right\}\isomorphism\left\{\text{sections }\sigma\colon S\to E\text{ of }f\right\}\;.
	\end{align*}
	Note that \cite[pp.~63--67]{katzmazur} define $\Pic^0(E/S)$ as the quotient of the group $\Pic^0(E)$ of line bundles with geometric fibre-wise degree $0$ by the subgroup of line bundles of the form $f^*\Ll_0$ where $\Ll_0$ is a line bundle on $S$. These definitions are equivalent by sending $\Ll\mapsto \Ll\otimes_{\Oo_E}f^*\big((\epsilon^*\Ll)^{\otimes -1}\big)$. We'll work with the Katz/Mazur definition, since this is somewhat easier.
	
	The map $\Pic^0(E/S)\morphism\{\text{sections }\sigma\text{ of }f\}$ sends $\Ll$ to the canonical section $\sigma$ associated to $\Ll\otimes_{\Oo_E}\Oo_E(\epsilon)$ by Proposition~\reff{prop:canonicalSection} (note that this makes sense because $\Ll$ has fibre-wise degree $0$ and $\Ii(\epsilon)$ has fibre-wise degree $-1$ by Lemma~\reff{lem:I(s)degree-1}, hence $\Ll\otimes_{\Oo_E}\Oo_E(\epsilon)$ has fibre-wise degree $1$). The inverse map sends a section $\sigma$ to $\Ii(\epsilon)\otimes_{\Oo_E}\Oo_E(\sigma)$.
	
	To show that these guys are indeed inverse, let $\Ll\in\Pic^0(E/S)$ and $\sigma$ the section of $f$ associated to $\Ll\otimes_{\Oo_E}\Oo_E(\epsilon)$. By Proposition~\reff{prop:canonicalSection}, we have
	\begin{align*}
		f^*f_*\left(\Ll\otimes_{\Oo_E}\Oo_E(\epsilon)\right)\otimes_{\Oo_E}\Oo_E(\sigma)\cong\Ll\otimes_{\Oo_E}\Oo_E(\epsilon)\;.
	\end{align*}
	But $f^*f_*\left(\Ll\otimes_{\Oo_E}\Oo_E(\epsilon)\right)$ vanishes in $\Pic^0(E/S)$ (this is clear by definition -- at least if we work with the Katz/Mazur definition of $\Pic^0(E/S)$), so $\Ll=\Ii(\epsilon)\otimes_{\Oo_E}\Oo_E(\sigma)$ in $\Pic^0(E/S)$. This shows that the composition $\Pic^0(E/S)\morphism\{\text{sections }\sigma\text{ of }f\}\morphism\Pic^0(E/S)$ is the identity. The other direction can be shown in quite the same way.
	
	Let $\cat C$ be a full subcategory of $\cat{PSch}/S$ containing $E$, closed under fibre products over $S$, and such that all its objects are locally noetherian (we could save ourselves these technicalities by working with $f$ of finite presentation). For any $S$-prescheme $T$ let $E(T)$ denote the set of morphisms $T\morphism E$ in $\cat{PSch}/S$. Then $E(T)$ is in canonical bijection with the set of sections of the base change $f_T\colon E_T=T\times_SE\morphism T$ of $f$ and $E(-)=\Hom_{\cat{PSch}/S}(-,E)$ is a contravariant functor $(\cat{PSch}/S)^\op\morphism\cat{Set}$. The set $E(T)$ should be thought of and is called the set of \defemph{$\boldsymbol{T}$-valued points} of $E$.
	
	%This functor can be introduced for any $S$-prescheme $E$ (and not only those that happen to be elliptic curves over $S$). We have, e.g., $(E\times_SF)(Y)=E(Y)\times F(Y)$ by the universal property of fibre products.
	If $T$ is in $\cat C$, then we get canonical isomorphisms
	\begin{align*}
		E(T)\cong\left\{\text{sections of }f_T\colon E_T\to T\right\}\cong\Pic^0(E_T/T)\;.
	\end{align*}
	The right-hand side forms an abelian group with $-\otimes_{\Oo_E}-$. Therefore the functor (or rather its restriction to $\cat C$) $E(-)\colon \cat C^\op\morphism\cat{Set}$ factors over $\cat{Ab}$. That is, $E(-)$ is an \emph{abelian group object} in the functor category $\Funct(\cat{C}^\op,\cat{Set})$. Since $\cat C^\op$ is a full subcategory of $\Funct(\cat{C}^\op,\cat{Set})$ via the Yoneda embedding, this means that $E$ is already an \emph{abelian group object} in $\cat C$.
	
	\lbl{par:groupScheme}Explicitly, this means the following: There are morphisms $m\colon E\times_SE\morphism E$ and $\iota\colon E\morphism E$ of $S$-preschemes (the \emph{multiplication} and the \emph{inversion}) such that the following diagrams commute (which is an abstract way of proposing that the ``group axioms'' hold for $E$):
	\begin{center}
		\begin{minipage}[b]{0.4\textwidth}
			\centering
			\begin{diagram*}
				\object{0,1.5}{$E\times_SE\times_SE$}[EEE];
				\object{0,0}{$E\times_SE$}[EE];
				\object{3.5,1.5}{$E\times_SE$}[ExE];
				\object{3.5,0}{$E$}[E];
				\scriptsize
				\arrow{EEE}{ExE}[above][$(\id_E,m)$];
				\arrow{EEE}{EE}[left][$(m,\id_E)$];
				\arrow{EE}{E}[above][$m$];
				\arrow{ExE}{E}[right][$m$];
			\end{diagram*}
			(associativity)
		\end{minipage}
		\begin{minipage}[b]{0.5\textwidth}
			\centering
			\begin{diagram*}
				\object{0,1.5}{$S\times_SE$}[SE];
				\object{2.5,1.5}{$E$}[E1];
				\object{5,1.5}{$E\times_SS$}[ES];
				\object{0,0}{$E\times_SE$}[EE1];
				\object{2.5,0}{$E$}[E2];
				\object{5,0}{$E\times_SE$}[EE2];
				\scriptsize
				\isoarrow{E1}{SE};
				\isoarrow{E1}{ES};
				\arrow{SE}{EE1}[left][$(\epsilon,\id_E)$];
				\arrow{ES}{EE2}[right][$(\id_E,\epsilon)$];
				\arrow{EE1}{E2}[above][$m$];
				\arrow{EE2}{E2}[above][$m$];
				\draw[transform canvas={xshift=1pt}] (E1) -- (E2);
				\draw[transform canvas={xshift=-1pt}] (E1) -- (E2);
			\end{diagram*}
			($e$ is a left- and right-neutral element)
		\end{minipage}
			\begin{minipage}[b]{0.4\textwidth}
			\centering
			\begin{diagram*}
				\object{0,1.5}{$E\times_SE$}[EE1];
				\object{3.5,1.5}{$E\times_SE$}[EE2];
				\object{1.75,0}{$E$}[E];
				\scriptsize
				\draw[->] (EE1) -- (EE2) node[pos=0.5,above] {exchange} node[pos=0.5,below] {factors};
				\arrow{EE1}{E}[below left][$m$];
				\arrow{EE2}{E}[below right][$m$];
			\end{diagram*}
			(commutativity)
		\end{minipage}
		\begin{minipage}[b]{0.5\textwidth}
			\centering
			\begin{diagram*}
				\object{0,2}{$E\times_SE$}[SE];
				\object{2.5,2}{$E\times_SE$}[E1];
				\object{5,2}{$E\times_SE$}[ES];
				\object{2.5,1}{$S$}[S];
				\object{2.5,0}{$E$}[E2];
				\scriptsize
				\arrow{E1}{SE}[above][$(\iota,\id_E)$];
				\arrow{E1}{ES}[above][$(\id_E,\iota)$];
				\arrow{E1}{S};
				\arrow{S}{E2}[right][$\epsilon$];
				\arrow{SE}{E2}[below left][$m$];
				\arrow{ES}{E2}[below right][$m$];
			\end{diagram*}
			($\iota$ is a left- and right-inverse)
		\end{minipage}
	\end{center}
	For the definition of a group object we don't need $E$ to be an elliptic curve (actually, this works in any category $\cat C$ with finite products and a final object). A group object in some prescheme category is called a \defemph{group prescheme}.
	
\paragraph{The Jacobian problem in general.}\lbl{par:Jacobian}
Let $(f,\epsilon)$ be an elliptic curve, where $f\colon E\morphism S$ is a morphism and $\epsilon$ is a section of $f$. We have seen above that the functor $\Pic_{E/S}^0\colon \cat{C}^\op\morphism\cat{Ab}$, given by $\Pic_{E/S}^0(T)=\Pic(E_T/T)$, is representable. In fact, $E$ itself is a representing object! If we drop the fibre-wise degree $0$ condition, we get a functor $\Pic_{E/S}\colon \cat{C}^\op\morphism\cat{Ab}$ given by $\Pic_{E/S}(T)=\Pic(E_T/T)=\Pic(E_T)/f_T^*\Pic(T)$. It is not hard to see that $\Pic_{E/S}$ is representable again, and a representing object is given by $\coprod_{d\in\IZ}E$. Indeed, $\coprod_{d\in\IZ}E$ has one component for every fibre-wise degree a line bundle can have; and since we have line bundles of fibre-wise degree $1$ (such as $\Oo_E(\epsilon)$, using Lemma~\reff{lem:I(s)degree-1}), we can shift between different degrees.

Now let's drop all our precious assumptions and consider any morphism $f\colon X\morphism S$ of preschemes. Then we can still consider the \emph{relative Picard functor} $\Pic_{X/S}\colon \cat{C}^\op \morphism\cat{Ab}$ defined by
\begin{align}\lbl{eq:PicCS}
	\Pic_{X/S}(T)=\Pic(X_T/T)=\Pic(X_T)/f_T^*\Pic(T)\;.
\end{align}
Then the question whether $\Pic_{X/S}$ is representable is the \emph{Jacobian problem}, and a representing object $\PIC_{X/S}$ -- if it exists -- is called the \defemph{Jacobian scheme} or \defemph{Picard scheme}\footnote{To be precise, the \emph{Jacobian scheme} denotes the degree-$0$ component $\PIC_{X/S}^0\subseteq \PIC_{X/S}$, which corresponds to $\Pic_{X/S}^0$. But Professor Franke called the Picard scheme \emph{Jacobian} as well, and I think that's perfectly fine.} of $X$ over $S$.

So why should $\Pic_{X/S}$ be a likely candidate for a representable functor -- or, to put it the other way around, why shouldn't we try to find a representing object $\PIC_X$ for the \emph{absolute Picard functor} $\Pic_C\colon\cat{C}^\op\morphism\cat{Ab}$ instead, which is defined by $\Pic_C(T)=\Pic(C_T)$ (i.e., we drop the quotient in \eqreff{eq:PicCS})? The reason is that $\Pic_X$ is almost never representable! Indeed, suppose on the contrary there was a representing object $\PIC_X$. If $\Ll$ is a line bundle on $X_T$ and $\Ll_0$ a line bundle on $T$, then $\Ll$ and $\Ll\otimes_{\Oo_{X_T}}f_T^*\Ll_0$ are usually not isomorphic. Yet locally with respect to $T$ they are, because $\Ll_0$ is locally trivial on $T$. Hence if $T=\bigcup_{i\in I}U_i$ is a sufficiently fine open cover, then the morphisms $T\morphism\PIC_X$ defined by (the isomorphism classes of) $\Ll$ and $\Ll\otimes_{\Oo_{X_T}}f_T^*\Ll_0$ coincide when restricted to $f_T^{-1}(U_i)$ for all $i\in I$. But then they coincide on all of $T$, which implies that $\Ll$ and $\Ll\otimes_{\Oo_{X_T}}f_T^*\Ll_0$ are isomorphic, which is (usually) contradiction!

Thus $\Pic_X$ is a hopeless candidate for a representable functor! Moreover, if we revisit the above argument, then we see that the main problem is basically the lack of ``sheaf-like'' properties of $\Pic_X$. So the natural thing to ask is whether $\Pic_X$ can be ``sheafified'' in an appropriate way. What makes things complicated here is that the Zariski topology might not be the right topology the sheafify $\Pic_X$ in (at least, when no section $\sigma\colon S\morphism X$ exists). That's where the \emph{étale}, the \emph{fpqc}, and the \emph{fppf topology} come in to save the day. We'll briefly introduce these notions in Section~\reff{sec:Descent} and Remark~\reff{rem:etaleTopology}\itememph{b}.

Now we can also explain why $\Pic_{X/S}$ should be worth our interest: The functor $\Pic_{X/S}$ can be viewed as a ``sheafified'' version of $\Pic_X$ (at least in a nice enough situation). Let's give some informal motivation for this! First note that $\Pic_X$ and $\Pic_{X/S}$ should have the same ``sheafification''. Indeed, the ``sheafification'' of $\Pic_{X/S}$ should be the ``sheaf quotient'' of the ``sheafifications'' of $\Pic_X$ and of $f_{(-)}^*\Pic(-)\colon(\cat{PSch}/S)^\op\morphism\cat{Ab}$. But $f_T^*\Pic(T)$ is locally trivial on the base $T$ for every $S$-prescheme $T$, so $f_{(-)}^*\Pic(-)$ should have trivial ``sheafification''. Then Example~\reff{ex:sheafifyingPic} shows that $\Pic_{X/S}$ is indeed a ``sheaf'' (hence the ``sheafification'' of $\Pic_X$), provided we are in a nice enough situation -- and these ``nice enough'' situations include the case where $f\colon C\morphism S$ is a flat family of curves of genus $g$ which admits a section $\sigma\colon S\morphism C$. That is, we'll be able to construct Jacobians of curves with a section -- which explains the title of this lecture!

To finish this overview of the goal of this lecture, Professor Franke gives a bit of foreshadowing on the construction of $\PIC_{C/S}$ in the case where $f\colon C\morphism S$ is a flat family $f\colon C\morphism S$ of curves of genus $g$ with a section $\sigma\colon S\morphism C$.\footnote{Actually, $\PIC_{C/S}$ still exists if we don't need to assume we have a section $\sigma$. But then the functor $\Pic_{C/S}$ will not be the same as in \eqreff{eq:PicCS}, since the latter is in general not a ``sheaf'' if there is no sections $\sigma$ (see Example~\reff{ex:sheafifyingPic}). Rather, $\Pic_{C/S}$ will be the sheafification of the functor from \eqreff{eq:PicCS} (and thus also the sheafification of $\Pic_C$, as explained above).} It will turn out we only need to construct the degree-$g$ component $\PIC_{C/S}^g\subseteq\PIC_{C/S}$ parametrizing the line bundles of fibre-wise degree $g$. Indeed, since we have a section $\sigma$, we can use $\Oo_C(\sigma)$ to shift between degrees, hence $\PIC_{C/S}$ can be constructed as $\coprod_{d\in\IZ}\PIC_{C/S}^g$. 

Instead of constructing $\PIC_{C/S}^g$, we will construct the open subprescheme $\boldsymbol{U}_{C/S}\subseteq \PIC_{C/S}^g$ parametrizing those line bundles $\Ll\in\Pic(C_T)$ for which $H^0(C_{\ov{t}},\Ll|_{C_{\ov{t}}})$ is a one-dimensional $\ov{\KK(t)}$-vector space for all $t\in T$ (note that by Riemann--Roch these vector spaces always have dimension at least $1$). Then we can recover $\PIC_{C/S}^g$ by gluing ``shifted versions'' of $\boldsymbol{U}_{C/S}$.

If $f\colon C\morphism S$ is a flat family of curves of genus $g$, and $\Ll$ is a line bundle of fibre-wise degree $g$ on $C$ such that $f_*\Ll$ happens to be a a line bundle on $S$, then, in generalization of Proposition~\reff{prop:canonicalSection}, we will obtain a unique subscheme $D\subseteq C$ which is finite and flat of degree $g$ over $S$. These $D$ (we already dropped the name \emph{effective relative Cartier divisor}) turn out to be representable by the \emph{symmetric power} $C^{(g)}\coloneqq C^{g}/\SS_g$, where $C^{g}$ is taken in the category of $S$-schemes (i.e., it is the $g$-fold fibre product of $C$ with itself over $S$) and the symmetric group $\SS_g$ acts by permuting factors. Moreover, we will see that $\boldsymbol{U}_{C/S}$ can be identified with a suitable open subprescheme of $C^{(g)}$. This will finally enable us to construct the Jacobian!

\lbl{page:battlePlan}So all in all our battle plan looks as follows:
\begin{enumerate}
	\item Construct $C^{(g)}$ -- in particular, we need to talk about quotients of group actions on preschemes.
	\item Show that $C^{(g)}$ represents the functor $\Div_{C/S}^g$ of  effective relative Cartier divisors.
	\item Use this to represent an open subprescheme $\boldsymbol{U}_{C/S}\subseteq \PIC_{C/S}^g$.
	\item Glue these open subpreschemes to obtain $\PIC_{C/S}^g$, and then put $\PIC_{C/S}=\coprod_{d\in\IZ}\PIC_{C/S}^g$.
\end{enumerate}
This lecture wouldn't be a Franke lecture if we didn't wait until the very end to realize this plan -- and do a lot of interesting stuff in between! The final Chapter~\reff{sec:Jac} follows the above steps. Step~1 is done in Proposition~\reff{prop:symmetricPowers}. Step~2 is treated in Theorem~\reff{thm:final}\itememph{f}. Finally Step~3 and 4 follow from the combined efforts of Section~\reff{sec:Construction}.
\section{The seesaw theorem and the theorem of the cube I}
As literature for this section Professor Franke recommends \cite{mumford1974abelian}, \cite{cornell1986arithmetic}, and \cite{kleiman}, as well as Grothendieck's FGA.

Note that even without knowing whether a representing object $\PIC_{C/S}$ of $\Pic_{C/S}$ exists, we can still decide whether it should be separated over $S$, since this property can be read off the functor $\Hom_{\cat{PSch}/S}(-,\PIC_{C/S})\cong\Pic_{C/S}$. From that point of view, the seesaw theorem simply states that under certain (essential) assumptions on $f\colon C\morphism S$, the morphism $\PIC_{C/S}\morphism S$ is a \emph{separated} morphism. Before we prove this we'll need some preparations.
\begin{prop}\lbl{prop:trivialLineBundlesOnProperX/k}
	Let $k$ be an algebraically closed field and $X$ an integral proper $k$-scheme.
	\begin{alphanumerate}
		\item We have $\Oo_X(X)\cong k$.
		\item A line bundle $\Ll$ on $X$ is trivial iff $\Ll(X)$ and $\Ll^{\otimes -1}(X)$ are both non-zero (and in this case they're both one-dimensional).
	\end{alphanumerate}
\end{prop}
\begin{proof}
	Part \itememph{a}. Since $X$ is integral, $\Oo_X(X)$ is a domain. It is also a finite-dimensional $k$-algebra by \cite[Theorem~5]{alggeo2} as $X/k$ is proper. Therefore $\Oo_X(X)$ is a finite field extension of $k$, hence isomorphic to $k$ itself.
	
	Part \itememph{b}. If $s\in\Ll(X)$, $t\in\Ll^{\otimes -1}(X)$ are non-vanishing global sections, then $U=X\setminus(V(s)\cup V(t))$ is an open dense subset of $X$ (where we use that $X$ is irreducible). The global section $s\otimes t$ of $\Ll\otimes_{\Oo_X}\Ll^{\otimes -1}\cong \Oo_X$ doesn't vanish on $U$, hence is given by some $\kappa\in k^\times$ using \itememph{a} (and in particular, $s\otimes t$ doesn't vanish at all). Then
	\begin{align*}
		s\otimes-\colon \Oo_X\morphism \Ll\quad\text{and}\quad-\otimes\kappa^{-1}t\colon \Ll\morphism\Oo_X
	\end{align*}
	are mutually inverse morphisms, proving $\Ll\cong \Oo_X$.
\end{proof}
Now let $f\colon X\morphism S$ be a proper between locally noetherian preschemes and $\Ff$ a coherent $\Oo_X$-module which is flat over $\Oo_S$. Recall that
\begin{align*}
	U_p&=\left\{s\in S\st(R^pf_*\Ff)(s)\to H^p(X_s,\Ff|_{X_s})\text{ is surjective}\right\}\\
	&=\left\{s\in S\st(R^pf_*\Ff)(s)\to H^p(X_s,\Ff|_{X_s})\text{ is bijective}\right\}
\end{align*}
is open in $S$ by Theorem~\reff{thm:GrauertGrothendieck}\itememph{d}. Also we defined
\begin{align*}
	h^p(s,\Ff)=\dim_{\KK(s)}H^p\left(X_s,\Ff|_{X_s}\right)=\dim_{\ov{\KK(s)}}H^p\left(X_{\ov{s}},\Ff|_{X_{\ov{s}}}\right)\;,
\end{align*}
where the right-hand equality uses Remark~\reff{rem:GeometricFibres}\itememph{a}.
\begin{prop}\lbl{prop:LineBundlesFromTheBase}
	Let $f\colon X\morphism S$ be a proper flat morphism between locally noetherian preschemes such that the geometric fibres $X_{\ov{s}}$ are integral for all $s\in S$ (in particular, the $X_{\ov{s}}$ are non-empty). Let $\Ll$ be a line bundle on $X$. Then the following are equivalent.
	\begin{alphanumerate}
		\item $\Ll\cong f^*\Ll_0$ for some $\Ll_0\in\Pic(S)$ -- that is, $\Ll$ is trivial in $\Pic(X/S)$.
		\item For both $\Ll$ and $\Ll^{\otimes -1}$, the functions $h^0\left(-,\Ll^{\otimes \pm 1}\right)$ don't vanish on $S$ and $U_0=S$.
	\end{alphanumerate}
	If this is the case, $h^0\left(s,\Ll^{\otimes\pm1}\right)=1$ for all $s\in S$ and the canonical morphism $f^*f_*\Ll\morphism\Ll$ is an isomorphism. Also $\Ll_0\cong f_*\Ll$ in this case.
\end{prop}
\begin{proof}
	Assume for the moment that \itememph{a} and \itememph{b} are equivalent and that $f^*f_*\Ll\morphism \Ll$ is an isomorphism. Under these assumptions we prove $\Ll_0\cong f_*\Ll$ for all $\Ll_0$ as in \itememph{a}. We get a canonical morphism $\Ll_0\morphism f_*\Ll$ corresponding to the isomorphism $f^*\Ll_0\isomorphism \Ll$ under the $f^*$-$f_*$ adjunction. Since $f^*f_*\Ll\morphism\Ll$ is assumed to be an isomorphism too, we see that $\Ll_0\morphism f_*\Ll$ becomes an isomorphism after applying $f^*$. Now the following claim does it.
	\begin{claim}
		Let $\phi\colon \Ff\morphism\Ff'$ be any morphism of quasi-coherent $\Oo_S$-modules such that the pullback $f^*(\phi)\colon f^*\Ff\morphism  f^*\Ff'$ becomes an isomorphism of $\Oo_X$-modules. Then $\phi$ is an isomorphism.
	\end{claim}
	Indeed, note that $f$ is faithfully flat (it is flat and its geometric fibres are non-empty, hence the ordinary fibres are non-empty too), hence for every $s\in S$ we may choose an $x\in X$ such that $s=f(x)$. Then $\Oo_{X,x}$ is faithfully flat over $\Oo_{S,s}$ (e.g.\ by Proposition~\reff{prop:faithfullyFlatModule}\itememph{d}), so $\Ff_s\otimes_{\Oo_{S,s}}\Oo_{X,x}\morphism\Ff'_s\otimes_{\Oo_{S,s}}\Oo_{X,x}$ being an isomorphism implies that $\Ff_s \morphism\Ff'_s$ is one as well. This shows that $\phi$ is an isomorphism, as claimed.
	
	We prove \itememph{a} $\Rightarrow$ \itememph{b} first. Since the assertion is local on $S$, we may assume $\Ll_0\cong\Oo_S$, so that $\Ll\cong f^*\Oo_S\cong \Oo_X$. Then
	\begin{align*}
		h^0\big(s,\Ll^{\otimes \pm 1}\big)=\dim_{\ov{\KK(s)}}H^0\left(X_{\ov{s}},\Oo_{X_{\ov{s}}}\right)=1\quad\text{for all}\quad s\in S 
	\end{align*}
	by Proposition~\reff{prop:trivialLineBundlesOnProperX/k}. Then $H^0(X_s,\Oo_{X_s})$ is a one-dimensional $\KK(s)$-vector space as well the constant function $1$ is a generator. But $1$ is clearly in the image of $(f_*\Oo_X)(s)\morphism H^0(X_s,\Oo_{X_s})$, so $U_0=S$ by Theorem~\reff{thm:GrauertGrothendieck}\itememph{d}.
	
	Now for \itememph{b} $\Rightarrow$ \itememph{a}. Since $U_0=S$ (by assumption) and $U_{-1}=S$ (for trivial reasons), we see that $f_*\Ll$ is locally free by Theorem~\reff{thm:GrauertGrothendieck}\itememph{e}. To see that $f_*\Ll$ is one-dimensional, it suffices to show that $(f_*\Ll)(s)\cong H^0(X_s,\Ll|_{X_s})$ are one-dimensional $\KK(s)$-vector spaces for all $s\in S$. But condition \itememph{b} together with Proposition~\reff{prop:trivialLineBundlesOnProperX/k} (which is needed for both the middle and the right equality) implies that 
	\begin{align*}
		H^0\left(X_s,\Ll|_{X_s}\right)\otimes_{\KK(s)}\ov{\KK(s)}= \dim_{\ov{\KK(s)}}H^0\left(X_{\ov{s}},\Ll|_{X_{\ov{s}}}\right)=\dim_{\ov{\KK(s)}}H^0\left(X_{\ov{s}},\Oo_{X_{\ov{s}}}\right)=1\;,
	\end{align*}
	so $f_*\Ll$ is indeed a line bundle. 
	
	Having established this, all we need to do is proving that $f^*f_*\Ll\morphism\Ll$ is an isomorphism. This is a local question on $S$ again, so we may assume that $f_*\Ll$ is trivial. Let $\lambda\in\Ll(X)=f_*\Ll(S)$ 
	be a generator. Let $x\in X$ and $s=f(x)$. As $(f_*\Ll)(s)\cong H^0(X_s,\Ll|_{X_s})$ (because $U_0=X$) and the left-hand side is generated by the image of $\lambda$, we see that $\lambda|_{X_s}$ is a non-vanishing global section of $\Ll|_{X_s}$. But $\Ll|_{X_s}(X_s)\cong \KK(s)$ (as seen above), so in particular $\lambda|_{X_s}$ generates the stalk of $\Ll|_{X_s}$ at $x$, which is given by $\Ll_x\otimes_{\Oo_{S,s}}\KK(s)\cong \Ll_x/\mm_{S,s}\Ll_x$. Then $\lambda$ can't vanish in $\Ll_x/\mm_{X,x}\Ll$, so $x\notin V(\lambda)$. Therefore,
	\begin{align*}
		\lambda\otimes-\colon \Oo_X\morphism\Ll
	\end{align*}
	is an isomorphism at $x$. Since $x$ was chosen arbitrarily, this shows $f^*f_*\Ll\isomorphism\Ll$ and \itememph{a} follows. We're done.
\end{proof}
\begin{thm}[Seesaw theorem]\lbl{thm:seesaw}
	Let $f\colon X\morphism S$ be a flat proper morphism with integral (in particular, non-empty) geometric fibres $X_{\ov{s}}$ for $s\in S$, and $S$ is locally noetherian. If $\Ll$ is a line bundle on $X$, then there's a closed subprescheme $Z\subseteq S$ having the following universal property: For any locally noetherian $S$-prescheme $\tau\colon T\morphism S$, the pullback $\tau_X^*\Ll$ of $\Ll$ under
	\begin{align*}
		\tau_X\colon X_T=X\times_ST\morphism X
	\end{align*}
	satisfies the equivalent conditions of Proposition~\reff{prop:LineBundlesFromTheBase} (applied to $f_T\colon X_T\morphism T$ instead of $f\colon X\morphism S$) iff $\tau$ factors over $Z\monomorphism S$.
\end{thm}
\begin{rem}
	\begin{alphanumerate}
		\item A related assertion of the general form \emph{You cannot express a non-trivial line bundle as a limit of trivial ones} was shown by A.\ Weil. See \cite{cornell1986arithmetic}, \cite{mumford1974abelian}, or the Wikipedia article for a historical discussion.
		\item It's clear that the base change $f_T\colon X_T\morphism T$ is flat and proper again. It also has integral geometric fibres by \cite[\stackstag{035U} and \stackstag{0364}]{stacks-project}, so Proposition~\reff{prop:LineBundlesFromTheBase} applies (as long as $T$ is locally noetherian, or we prove some generalizations for finitely presented morphisms).
	\end{alphanumerate}
\end{rem}
\begin{rem}	 \lbl{rem:PICscheme}
		Let's see why Theorem~\reff{thm:seesaw} implies that $\pi\colon \PIC_{X/S}\morphism S$ is separated (assuming it exists at all). Let $\tau_1,\tau_2\colon T\morphism\PIC_{C/S}$ be two morphisms such that $\pi\tau_1=\pi\tau_2$ -- or equivalently, let $T$ be an $S$-prescheme and assume that $\tau_1$ and $\tau_2$ are morphisms of $S$-preschemes. We need to show that the equalizer $\Eq(\tau_1,\tau_2)$ is closed. Using the group prescheme structure on $\PIC_{C/S}$, we may equivalently show that $\Eq(\tau,0)$ is closed for $\tau=\tau_1-\tau_2$ and $0$ the neutral element in the group $\Hom_{\cat{PSch}/S}(T,\PIC_{X/S})$.
		
		To see that $\Eq(\tau,0)$ is closed, let $\Ll$ be a line bundle on $X_T=X\times_ST$ such that $\Ll$ is a representative of the element of $\Pic(X_T/T)$ defined by $\tau$. Then a morphism $\tau'\colon T'\morphism T$ factors over $\Eq(\tau,0)$ iff $\tau\tau'=0\tau'$, which holds iff the pullback $p_2^*\Ll$ of $\Ll$ under $p_2\colon X_{T'}=X_T\times_TT'\morphism T'$ is trivial in $\Pic(X_{T'}/T')$. Hence $\Eq(\tau,0)$ has the universal property from Theorem~\reff{thm:seesaw} (applied to $f_T\colon X_T\morphism T$ instead of $f\colon X\morphism S$), hence it is indeed closed. This shows that $\pi$ is indeed separated \ldots
		
		\ldots\ well, up to the following flaw in the argument: We can only do this when $T$ is locally noetherian (or Theorem~\reff{thm:seesaw} wouldn't apply). But this is no problem. It will turn out (from the construction) that $\pi$ has finite type. So the case where $T$ is locally noetherian is enough to deduce that the diagonal $\Delta\colon\PIC_{X/S}\morphism\PIC_{X/S}\times_S\PIC_{X/S}$ is a closed embedding, which suffices for separatedness.
		%Let us assume that $f$ has a section $\sigma\colon S\morphism X$ (so that it doesn't matter in which topology whe sheafify the Picard functor). Assume that $\Pic_{X/S}$ can be represented by an $S$-(group-)prescheme $\pi\colon\PIC_{X/S}\morphism S$. Let $\epsilon\colon S\morphism\PIC_{X/S}$ be the neutral section and denote by $\lambda\colon S\morphism\PIC_{X/S}$ the morphism corresponding to (the isomorphism class of) $\Ll\in\Pic_{X/S}(S)=\Pic(X/S)$.
		
		%If $f_T^*\colon \Oo_T\isomorphism f_{T,*}\Oo_{X_T}$ is an isomorphism for all $S$-preschemes $T$ (i.e.\ $f_*\Oo_X\cong \Oo_S$ \emph{holds universally} -- for example, this is the case in the situation of Proposition~\reff{prop:RpLl}), then $\tau_X^*\Ll$ is trivial in $\Pic_{X/S}(T)=\Pic(X_T/T)$ iff the compositions
		%\begin{align*}
		%	T\morphism[\tau]S\doublemorphism[\lambda][\epsilon]\PIC_{X/S}
		%\end{align*}
		%coincide (since that's how the functor isomorphism $\Pic_{X/S}\cong \Hom_{\cat{PSch}/S}(-,\PIC_{X/S})$ works), which is the case iff $\tau$ factors over $\Eq(\lambda,\epsilon)$. This shows that $Z=\Eq(\lambda,\epsilon)$ in Theorem~\reff{thm:seesaw}. In particular, the equalizer $\Eq(\lambda,\epsilon)$ is \emph{closed}! If you play around with the definitions a bit (you should use that $\Hom_{\cat{PSch}/S}(S,\PIC_{X/S})$ is an abelian group) this shows that $\pi\colon \PIC_{X/S}\morphism S$ is separated. This justifies the interpretation of the seesaw theorem from the beginning of the current section.
\end{rem}
\begin{proof}[Proof of Theorem~\reff{thm:seesaw}]
	\emph{Step 1.} We first define $Z$ as a set. Put
	\begin{align*}
		Z_+=\left\{s\in S\st h^0\big(s,\Ll\big)\text{ doesn't vanish}\right\}\quad\text{and}\quad Z_-=\left\{s\in S\st h^0\big(s,\Ll^{\otimes -1}\big)\text{ doesn't vanish}\right\}
	\end{align*}
	Put $Z=Z_+\cap Z_-$. Then from 
	Proposition~\reff{prop:trivialLineBundlesOnProperX/k}\itememph{b} we see 
	\begin{align*}
		Z=\left\{s\in S\st h^0\big(s,\Ll^{\otimes \pm 1}\big)=1\right\}\;.
	\end{align*}
	By Theorem~\reff{thm:GrauertGrothendieck}\itememph{b}, $Z_\pm$ and thus also $Z$ are closed subsets of $S$. Note that this preliminary definition of $Z$ doesn't cover the second condition from Proposition~\reff{prop:LineBundlesFromTheBase}\itememph{b}, i.e., that $U_0$ contains all of the base. This will be taken care of once we define the prescheme structure on $Z$. The proof now proceeds in three more steps.
	
	 \emph{Step 2.} Let $\tau\colon T\morphism S$ be an $S$-prescheme such that $\tau_X^*\Ll$ is the trivial element in $\Pic_{X/S}(T)=\Pic(X_T/T)$. We show that the image of $\tau$ (as a map between topological spaces) is contained in $Z$.  Let $t\in T$ and $s=\tau(t)$. Then
	\begin{align*}
		X_t=X_T\times_T\Spec\KK(t)\cong X\times_S\Spec\KK(t)\cong X_s\times_{\Spec\KK(s)}\Spec\KK(t)
	\end{align*}
	by some abstract nonsense, hence, using \eqreff{eq:tensoredCohomology} and the fact that $\KK(t)$ is a flat ring extension of $\KK(s)$ (as is any field extension), we obtain
	\begin{align*}
		H^\bullet\left(X_t,\tau_X^*\Ll^{\otimes \pm 1}|_{X_t}\right)\cong H^\bullet\left(X_s,\Ll^{\otimes \pm 1}|_{X_s}\right)\otimes_{\KK(s)}\KK(t)\;.
	\end{align*}
	 Therefore $h\left(t,\tau_X^*\Ll^{\otimes \pm 1}\right)=1$ holds if and only if $h\left(s,\Ll^{\otimes \pm 1}\right)=1$. Since $h\left(t,\tau_X^*\Ll^{\otimes \pm 1}\right)=1$ for all $t\in T$ by Proposition~\reff{prop:LineBundlesFromTheBase} and the assumption on $\tau_X^*\Ll$, this shows $s\in Z$, as claimed.
	
	Conversely, if $\tau\colon T\morphism S$ is any $S$-prescheme such that the image of  $\tau$ is contained in $Z$, then the above argument shows $h\left(t,\tau_X^*\Ll^{\otimes \pm 1}\right)=1$ for all $t\in T$.
	
	\emph{Step 3.} We describe the prescheme structure on $Z$. This task is local on $S$ for the following reason: $Z$ (as a prescheme, not just a topological space) is uniquely determined by the universal property we imposed, and if $U\subseteq S$ is any open subset, then $Z\cap U$ has the corresponding universal property for $U$. So once we proved that the universal property is fulfilled locally (which we'll do in Step~4), everything will glue together nicely. 
	
	Therefore, we may choose an $s\in Z$ and restrict ourselves to an affine open neighbourhood $U\cong \Spec A$ of $s\in S$, where $A$ is noetherian. Moreover, we may replace $A$ by some localization whenever we feel like it, as long as still $s\in\Spec A$. Moreover, we'll relax the assumption $s\in Z$ to only $s\in Z_+$ to describe a prescheme structure on $Z_+$ instead of $Z$. Since $Z_-$ can then be equipped with an analogous prescheme structure, this will define the required prescheme structure on $Z=Z_+\cap Z_-$.
	
	Now let's get to business. Let $P^\bullet$ be the cochain complex of $A$-modules from Theorem~\reff{thm:GrauertGrothendieck}\itememph{a}. As described on page~\pageref{page:Pfree}, we may assume that all $P^i$ are free (after replacing $A$ by some localization). Since $s\in Z_+$, we know that
	\begin{align*}
		H^0\left(X_s,\Ll|_{X_s}\right)\cong\ker\left(\delta_s^0=\delta^0\otimes \id_{\KK(s)}\colon P^0\otimes_A\KK(s)\morphism P^1\otimes_A\KK(s)\right)
	\end{align*}
	has dimension $1$ over $\KK(s)$. Let $\ov{e}_0, \ldots,\ov{e}_n$ be a basis of $P^0\otimes_A\KK(s)$ such that $\ov{e}_0$ is a basis of $\ker \delta_s^0$. Choose an arbitrary basis of $\ov{f}_0,\ldots,\ov{f}_m$ of $P^1\otimes_A\KK(s)$, where $m\geq n-1$. Then the matrix representation $\ov{D}$ of $\delta_s^0$ looks like
		\begin{align*}
	\begin{tikzpicture}[remember picture]
	\node at (0,0) {$\ov{D}=
		\begin{pmatrix}
		 0 & \tikzentry{a}{\phantom{0}} \\
		 \vdots & & \phantom{\ov{D}_1}\\
		 \vdots & & & \tikzentry{b}{\phantom{0}} &\\
		 \vdots & \tikzentry{c}{\phantom{0}} \\
		 0 &  & & \tikzentry{d}{\phantom{0}}
		\end{pmatrix}
		$};
		\draw[rounded corners=0.75] ($(a.north west) +(-2pt,3pt)$) rectangle ($(b.south east) +(2pt,-3pt)$) node[pos=0.5] {$\ov{D}_1$};
		\draw[rounded corners=0.75] ($(c.north west) +(-2pt,3pt)$) rectangle ($(d.south east) +(2pt,-3pt)$) node[pos=0.5] {$\ov{D}_2$};
	\end{tikzpicture}
	\end{align*}
	in which $\ov{D}_1$ is an $n\times n$ matrix and $\ov{D}_2$ is $(m-n+1)\times n$ (I don't know why Franke needed six blocks in the lecture; it seemed rather unnecessarily complicated). Since $\ov{D}$ has rank $n$ by assumption, it must contain an invertible $n\times n$ minor. So permuting $\ov{f}_0,\ldots,\ov{f}_m$, we may assume that $\ov{D}_1$ is invertible.
	
	By Nakayama-style arguments like \cite[Lemma~1.5.1]{alg2} we may replace $A$ by some localization such that the $\ov{e}_i$ and $\ov{f}_j$ possess lifts $e_i\in P^0$ and $f_j\in P^1$ such that $e_0,\ldots,e_n$ form a basis of $P^0$ and $f_0,\ldots,f_m$ a basis of $P^1$. Then the matrix representation $D$ of $\delta^0$ looks like
	\begin{align*}
	\begin{tikzpicture}[remember picture]
	\node at (0,0) {$D=
		\begin{pmatrix}
		 & \tikzentry{e}{\vphantom{0}\hphantom{d_1}}\vphantom{0} & \tikzentry{a}{\phantom{0}} \\
		 &  \vphantom{\vdots} & & \phantom{\ov{D}_1}\\
		 & \tikzentry{f}{\vphantom{0}\hphantom{d_1}}\vphantom{\vdots} & & & \tikzentry{b}{\phantom{0}} &\\
		 & \tikzentry{g}{\vphantom{0}\hphantom{d_2}}\vphantom{\vdots} & \tikzentry{c}{\phantom{0}} \\
		 & \tikzentry{h}{\vphantom{0}\hphantom{d_2}}\vphantom{0} &  & & \tikzentry{d}{\phantom{0}}
		\end{pmatrix}
		$};
	\draw[rounded corners=0.75] ($(a.north west) +(-2pt,3pt)$) rectangle ($(b.south east) +(2pt,-3pt)$) node[pos=0.5] {$D_1$};
	\draw[rounded corners=0.75] ($(c.north west) +(-2pt,3pt)$) rectangle ($(d.south east) +(2pt,-3pt)$) node[pos=0.5] {$D_2$};
	\draw[rounded corners=0.75] ($(e.north west) +(-2pt,3pt)$) rectangle ($(f.south east) +(2pt,-3pt)$) node[pos=0.5] {$d_1$};
	\draw[rounded corners=0.75] ($(g.north west) +(-2pt,3pt)$) rectangle ($(h.south east) +(2pt,-3pt)$) node[pos=0.5] {$d_2$};
	\end{tikzpicture}
	\end{align*}
	in which $d_1$ and $d_2$ are column vectors whose images in $\KK(s)$ vanish, and the images of $D_1$ and $D_2$ in $\KK(s)$ are $\ov{D}_1$ and $\ov{D}_2$ respectively. Since $\det\ov{D}_1$ doesn't vanish in $\KK(s)$, we may replace $A$ by some localization such that $D_1\in A^{n\times n}$ becomes invertible (cf.\ the argument in the proof of Lemma~\reff{lem:BasicallyThm4bc}). But then $d_1$ as well as the rows of $D_2$ may be represented as linear combinations of the columns and rows of $D_1$ respectively. So by performing column and row transformations (that is, replacing $e_0$ and $f_n,\ldots,f_m$ appropriately) we may assume that $d_1=0$ and $D_2=0$, leaving only 
		\begin{align*}
	\begin{tikzpicture}[remember picture]
	\node at (0,0) {$D=
		\begin{pmatrix}
		& 0 & \tikzentry{a}{\phantom{0}} \\
		&  \vdots & & \phantom{\ov{D}_1}\\
		& 0\vphantom{\vdots} & & & \tikzentry{b}{\phantom{0}} &\\
		& \tikzentry{g}{\vphantom{0}\hphantom{d_2}}\vphantom{\vdots} & \tikzentry{c}{\phantom{0}} \\
		& \tikzentry{h}{\vphantom{0}\hphantom{d_2}}\vphantom{0} &  & & \tikzentry{d}{\phantom{0}}
		\end{pmatrix}
		$};
	\draw[rounded corners=0.75] ($(a.north west) +(-2pt,3pt)$) rectangle ($(b.south east) +(2pt,-3pt)$) node[pos=0.5] {$D_1$};
	\path (c) -- (d) node[pos=0.5] {$0$};
	\draw[rounded corners=0.75] ($(g.north west) +(-2pt,3pt)$) rectangle ($(h.south east) +(2pt,-3pt)$) node[pos=0.5] {$d_2$};
	\end{tikzpicture}
	\end{align*}
	(that's the last blocky matrix, I promise). Now note that $Z_+=V(d_2)$ (the ideal $(d_2)\subseteq A$ generated by the vector $d_2$ is defined in the obvious way). Indeed, since $D_1$ will be invertible in $\KK(\pp)$ for all $\pp\in\Spec A$, the kernel of $\delta_\pp^0=\delta^0\otimes\id_{\KK(\pp)}$ is one-dimensional iff $d_2$ vanishes in $\KK(\pp)$. Therefore, it seems sensible to equip $Z_+$ with the prescheme structure of $\Spec A/(d_2)$, and that's what we'll do.
	
	\emph{Step 4.} We show that locally $Z=Z_+\cap Z_-$ has the required universal property. We already know from Step~2 that an $S$-prescheme $\tau\colon T\morphism S$ satisfies $h\left(t,\tau_X^*\Ll^{\otimes \pm 1}\right)=1$ for all $t\in T$ iff the image of $\tau$ is contained in $Z$. Now we prove that $U_0=T$ for such $\tau_X^*\Ll$ if and only if $\tau$ factors over $Z_+$ (and likewise with $\tau_X^*\Ll^{\otimes -1}$ and $Z_-$), which suffices to finish the proof.
	
	Again, we may do this locally, so let's assume $S=\Spec A$ and $Z_+=V(d_2)$. By definition, we have $U_0=T$ iff the base change morphism
	\begin{align*}
		\big(f_{T,*}(\tau_X^*\Ll)\big)(t)\morphism H^0\left(X_t,\tau_X^*\Ll|_{X_t}\right)
	\end{align*}
	is surjective for all $t\in T$. We know 
	\begin{align*}
		H^0\big(X_t,\tau_X^*\Ll|_{X_t}\big)=\ker\left(\delta_t^0=\delta^0\otimes\id_{\KK(t)}\colon P^0\otimes_A\KK(t)\morphism P^1\otimes_A\KK(t)\right)
	\end{align*}
	by Theorem~\reff{thm:GrauertGrothendieck}\itememph{a}, so $\ker \delta^0_t$ is one-dimensional (since the image of $\tau$ lies in $Z_+$ by assumption) and thus generated by $e_0\otimes1$. Now let $V\cong \Spec B$ be a small affine open neighbourhood of $t\in T$. By Theorem~\reff{thm:GrauertGrothendieck}\itememph{a} again, $f_{T,*}(\tau_X^*\Ll)(V)$ is given by
	\begin{align*}
		K=\ker\left(\delta^0\otimes\id_B\colon P^0\otimes_AB\morphism P^1\otimes_AB\right)\;.
	\end{align*}
	So the question is whether $K\otimes_B\KK(t)\morphism \ker\delta_t^0$ is surjective. Since the image of $D_1$ in $B^{n\times n}$ stays invertible, an element $\sum_{i=0}^n\beta_i(e_i\otimes1)$ of $P^0\otimes_AB$ (where $\beta_i\in B$ are some coefficients) is in $K$ iff $\beta_1=\ldots=\beta_n=0$ and $\beta_0d_2=0$. Thus $e_0\otimes1\in\ker\delta_t^0$ is hit by $K\otimes_B\KK(t)\morphism \ker\delta_t^0$ if and only if $\beta d_2=0$ in $\Oo_{T,t}$ for some $\beta\in B$ that doesn't vanish in $\KK(t)$. But such $\beta$ is a unit in $\Oo_{T,t}$ (since $\beta\notin\mm_{T,t}$), which shows $d_2=0$ in $\Oo_{T,t}$. Since this is true for all $t\in T$, we see that $\tau^*d_2=0$.
	
	Thus $U_0=T$ for $\tau_X ^*\Ll$ iff $\tau$ factors over $Z_+$ as a prescheme. The same is true for $\tau_X^*\Ll^{\otimes -1}$ and $Z_-$, so $Z=Z_+\cap Z_-$ has the required universal property.
\end{proof}
\begin{rem}
	If the fibres of $f\colon X\morphism S$ are no longer assumed irreducible, then $\pi\colon \PIC_{X/S}\morphism S$ may fail to be separated (while $\PIC_{X/S}$ still exists), even in fairly simple cases where all other assumptions of Theorem~\reff{thm:seesaw} remain valid. Here is one: Let $S=\Spec A$, where $A$ is a DVR with a uniformizer $\pi$. Let $s=(\pi)\in S$ be the closed point and $\eta\in S$ the generic point and let $X=V(xy-\pi z^2)\subseteq \IP_S^2$. Then
	\begin{align*}
		\IP_{S_\eta}^1&\isomorphism X_\eta\\
		[x,y]&\longmapsto [x^2,\pi y^2,xy]
	\end{align*}
	is an isomorphism, while $X_s=V(xy)\subseteq \IP_{S_s}^2$ is the union of two irreducible components $X_s^{(x)}=V(x)$ and $X_s^{(y)}=V(y)$ glued on $\xi=[0,0,1]\in\IP_{S_s}^2$. Line bundles on $X_\eta\cong \IP_{S_\eta}^1$ are parametrized by their degree (we proved this, e.g., in \cite[Corollary~A.2.1]{alggeo2}), while line bundles on $X_s$ are given by triples $\big(\Ll^{(x)},\Ll^{(y)},\iota\big)$ (up to isomorphism), where $\Ll^{(*)}$ are line bundles on $X_s^{(*)}$ for $*\in\{x,y\}$, and $\iota\colon \Ll^{(x)}(\xi)\morphism\Ll^{(y)}(\xi)$ is an isomorphism. Since $X_s^{(*)}\cong \IP_{S_s}^1$ for $*\in\{x,y\}$, we see that line bundles on $X_s$ are characterized by $\big(d^{(x)},d^{(y)}\big)\in\IZ^2$, where $d^{(*)}=\deg\Ll^{(*)}$.
	
	Following these observations, one finds after some work that $\PIC_{X/S}$ is given by 
	\begin{align*}
		\bigcup_{(m,n)\in\IZ^2}S_{m,n}\Bigm/\sim\;,
	\end{align*}
	where each $S_{m,n}$ is a copy of $S$. The equivalence relation $\sim$ identifies $S_{m,n}\setminus\{s_{m,n}\}$ and $S_{i,j}\setminus\{s_{i,j}\}$ whenever $m+n=i+j$ (and $s_{m,n}$ denotes the closed point of $S_{m,n}\cong S$).
\end{rem}
\begin{thm}[Theorem of the cube I]\lbl{thm:cube}
	Consider a pullback diagram
	\begin{diagram*}
		\object{0,1.5}{$X\times_SY$}[F];
		\object{0,0}{$X$}[X];
		\object{2.5,1.5}{$Y$}[Y];
		\object{2.5,0}{$S$}[S];
		\pullback{1.25,0.75};
		\scriptsize
		\arrow{F}{X}[left][$\upsilon_X$];
		\arrow{F}{Y}[above][$\xi_Y$];
		\arrow{X}{S}[above][$\xi$];
		\arrow{Y}{S}[left][$\upsilon$];
		\arrow[dotted,->,bend left]{S}{X}[below][$\sigma_X$];
		\arrow[dotted,->,bend right]{S}{Y}[right][$\sigma_Y$];
	\end{diagram*}
	 such that Theorem~\reff{thm:seesaw} is applicable to $\xi\colon X\morphism S$ and $\upsilon\colon Y\morphism S$. Also let $\sigma_X\colon S\morphism X$ and $\sigma_Y\colon S\morphism Y$ be sections of $\xi$ and $\upsilon$ respectively. Suppose that $\Ll$ is a line bundle on $X\times_SY$ for which there is a line bundle $\Ll_0$ on $S$ such that
	 \begin{align*}
	 	\Ll|_X\coloneqq(\id_X,\sigma_Y\xi)^*\Ll\cong \xi^*\Ll_0\text{ on }X\quad\text{and}\quad\Ll|_Y\coloneqq(\sigma_X\upsilon,\id_Y)^*\Ll\cong \upsilon^*\Ll_0\text{ on }Y\;.
	 \end{align*}
	 Then the closed subprescheme $Z\subseteq S$ obtained from Theorem~\reff{thm:seesaw} applied to $X\times_SY\morphism S$ is also open (not only as a set, but as a subprescheme too).
\end{thm}
Before proving Theorem~\reff{thm:cube} (which starts on page~\pageref{proof:cube}), Professor Franke decided to go through a sequence of remarks.
\begin{rem}\lbl{rem:productsOfVarieties}
	It's perhaps worthwhile to remark that Theorem~\reff{thm:seesaw} \emph{does} apply to $X\times_SY\morphism S$. The only subtle part is whether the geometric fibres $(X\times_SY)_{\ov{s}}$ remain integral. To see this, put $k=\ov{\KK(s)}$ and note that
	\begin{align*}
		(X\times_SY)_{\ov{s}}\cong X_{\ov{s}}\times_kY_{\ov{s}}\;,
	\end{align*}
	(the second fibre product should be taken over $\Spec k$ actually -- we'll keep this abuse of notation for convenience) in which $X_{\ov{s}}$ and $Y_{\ov{s}}$ are integral by assumption. If $U\cong \Spec B$ and $V\cong \Spec C$ are affine open subsets of $X_{\ov{s}}$ and $Y_{\ov{s}}$ respectively, then $B$ and $C$ are domains of finite type over $k$, hence $U\times_{\Spec k}V\cong \Spec (B\otimes_kC)$ is integral again -- we proved this (for algebraic varieties, but this makes no difference) in \cite[Proposition~2.2.6]{alg1}.
	
	If $U'\subseteq X_{\ov{s}}$ and $V'\subseteq Y_{\ov{s}}$ is another pair of affine open subsets, then $U\cap U'$ and $V\cap V'$ are non-empty because $X_{\ov{s}}$ and $Y_{\ov{s}}$ are irreducible, so every non-empty open subset is dense. Hence $U\times_kV$ and $U'\times_kV'\neq \emptyset$ intersect. This means $X_{\ov{s}}\times_kY_{\ov{s}}$ can be covered by finitely many (using quasi-compactness) integral open subpreschemes, mutually intersecting each other in non-empty (thus dense) subsets. This shows that $X_{\ov{s}}\times_kY_{\ov{s}}$ is integral itself.
\end{rem}
\begin{rem}
	\begin{alphanumerate}
		\item When $S$ is connected and $Z\neq\emptyset$ one has $Z=S$. That is, the line bundle $\Ll$ is trivial in $\Pic(X\times_SY/S)$.
		\item If $\Ll_0$ and $\Ll_0'$ satisfy $\Ll|_X\cong \xi^*\Ll_0$ and $\Ll|_Y\cong \upsilon^*\Ll_0'$, then already $\Ll_0\cong \Ll_0'$ (so Theorem~\reff{thm:cube} can be applied). Indeed, in this case both $\Ll_0$ and $\Ll_0'$ equal the pullback of $\Ll$ onto $S$.
		\item Theorem~\reff{thm:cube} is typically applied in the following situation (which we will make precise later in Theorem~\reff{thm:cube2}): Assume $X\cong A^m$ and $Y\cong A^n$ (and thus also $X\times_SY\cong A^{m+n}$) are powers of some abelian (group-)prescheme $A$ over $S$. Let morphisms
		\begin{align*}
			f_i\colon X\times_SY\morphism A\quad\text{for }i=1,\ldots,N
		\end{align*}
		be given by $f_i=\sum_{j=1}^{m+n}g_{i,j}$, where $g_{i,j}\colon A\morphism A$ is either $0_A$ or $\id_A$ (both $0_A$ as well as the above sum should be interpreted as taken in the group $\Hom_{\cat{PSch}/S}(A,A)$). If $\Ll_0$ is a line bundle on $A$ and exponents $\epsilon_i\in\{\pm 1\}$ are suitably chosen such that ``cancellations occur on the coordinate axes'' (that is, after pulling back to $X$ or $Y$), then Theorem~\reff{thm:cube} can be applied to a line bundle of the form 
		\begin{align*}
			\Ll=\bigotimes_{i=1}^N(f_i^*\Ll_0)^{\otimes\epsilon_i}\;.
		\end{align*}
	\end{alphanumerate}
\end{rem}
\begin{rem}
	In this remark Professor Franke sketches how an analogue of Theorem~\reff{thm:cube} can be obtained in the complex analytic case. This will need a lot of complex analytic theory, which we'll just cite without a proof (for this is only a motivational example).
	
	Let $X$ be a smooth projective variety over $\IC$. Its set of complex points (i.e., the closed points) endowed with the complex topology form a \emph{complex analytic space} $X^\an$ with a structure sheaf $\Oo_{X^\an}$; and to every sheaf $\Ff$ on $X$ one can assign a sheaf $\Ff^\an$ on $X^\an$. By Serre's famous \emph{GAGA theorem} we have $H^\bullet(X^\an,\Ff^\an)\cong H^\bullet(X,\Ff)$ whenever $\Ff$ is a coherent $\Oo_X$-module. For the rest of this remark, we'll drop the  $^\an$noying superscripts and assume everything to be understood in the complex analytic setting.
	
	Put $\IZ_X(n)\coloneqq (2\pi\mathrm{i})^n\IZ_X$. On $X$ we obtain a short exact sequence
	\begin{align}\lbl{eq:ComplexSeq}
		0\morphism \IZ_X(1)\morphism\Oo_X\morphism[\exp]\Oo_X^\times\morphism 0
	\end{align}
	We have $\Pic(X)\cong H^1(X,\Oo_X^\times)$ (compare this to the classification of $\Oo_X^\times$-torsors in \cite[Section~1.7]{alggeo2} -- this also works in the complex analytic case). To compute $\Pic(X)$, we thus need to investigate the long exact cohomology sequence associated to \eqreff{eq:ComplexSeq}.
	
	In degree $0$, this sequence looks like
	\begin{diagram*}
		\node[ob] (0o) at (0,1.5) {$0$};
		\node[ob]  (H0J) [right=0.5cm of 0o] {$H^0(X,\IZ_X(1))$};
		\node[ob]  (H0O) [right=0.5cm of H0J] {$H^0(X,\Oo_X)$};
		\node[ob]  (H0OX) [right=0.5cm of H0O] {$H^0(X,\Oo_X^\times)$};
		\node[ob]  (H1J) [right=0.5cm of H0OX]  {$H^1(X,\IZ_X(1))$};
		\node[ob]  (0u) at (0,0) {$0$};
		\node[ob]  (p) at (0u -| H0J) {$2\pi\mathrm{i}\IZ$};
		\node[ob]  (kX) at (0u -| H0O) {$\IC$};
		\node[ob]  (Ol) at (0u -| H0OX) {$\IC^\times$};
		\scriptsize
		\draw[->] (0o) -- (H0J);
		\draw[->] (H0J) -- (H0O);
		\draw[->] (H0O) -- (H0OX);
		\draw[->] (H0OX) -- (H1J);
		\draw[->] (0u) -- (p);
		\draw[->] (p) -- (kX);
		\draw[->] (kX) -- (Ol) node[pos=0.5, above] {$\exp$};
		\draw[->] (H0J) -- (p) node[pos=0.5, sloped, above=-0.25ex] {$\sim$};
		\draw[->] (H0O) -- (kX) node[pos=0.5, sloped, above=-0.25ex] {$\sim$};
		\draw[->] (H0OX) -- (Ol) node[pos=0.5, sloped, above=-0.25ex] {$\sim$};
	\end{diagram*}
(using that $X$ is connected). But $\exp\colon \IC\morphism\IC^\times$ in the bottom line is surjective, so the morphism $H^0(X,\Oo_X^\times)\morphism H^1(X,\IZ_X(1))$ actually vanishes. Therefore, we'd expect $H^1(X,\Oo_X^\times)$ to be a product of $H^1(X,\Oo_X)/H^1(X,\IZ_X(1))$ (which is a complex torus, because $H^1(X,\IZ_X(1))$ turns out to be a complete lattice in the $\IC$-vector space $H^1(X,\Oo_X)$) and a subgroup of $H^2(X,\IZ_X(1))$, which should be discrete. But $H^1(X\times Y,\Oo_{X\times Y})\cong H^1(X,\Oo_X)\oplus H^1(Y,\Oo_Y)$ by a Künneth-type result. This makes it plausible that the non-discrete (or connected) part of $\Pic(X)$ is compatible with products.

Professor Franke also mentions that the image of $c_1\colon \Pic(X)\morphism H^2(X,\IZ_X(1))$ is the kernel of the composition
\begin{align*}
	H^2(X,\IZ_X(1))\morphism H^2(X,\IC_X)\cong\bigoplus_{p+q=2}H^p(X,\Omega_X^q)\morphism H^2(X,\Oo_X)\oplus H^0(X,\Omega_X^2)\;,
\end{align*}
in which the isomorphism in the middle is the \emph{Hodge decomposition} (and indeed this makes some sense, since I'd expect the first Chern class to have target $H^1(X,\Omega_X^1)$).
\end{rem}
In general there is no exponential map available in the algebraic setting. However, if $x^2=0$, then ``$\exp(x)=1+x$'' is a good candidate. The following lemma exploits this idea.
\begin{lem}\lbl{lem:kerPicH1}
	Let $X$ be a quasi-compact scheme and $X_0=V(\Jj)$ the closed subscheme cut out by a quasi-coherent sheaf of ideals $\Jj\subseteq \Oo_X$ such that $\Jj^2=0$. Assume that the map of global sections $\Oo_X(X)\morphism \Oo_{X_0}(X_0)$ is surjective. Then there is an isomorphism
	\begin{align*}
		\ker\big(\Pic(X)\morphism\Pic(X_0)\big)\isomorphism H^1(X,\Jj)
	\end{align*}
	which is functorial in the pair $(X,X_0)$, i.e., compatible with pullbacks $Y_0=X_0\times_XY$ along arbitrary morphisms $f\colon Y\morphism X$.
\end{lem}
\begin{proof}
	The proof in the lecture was pretty technical and followed the idea to identify the kernel in the above isomorphism with $\left\{\text{isomorphism classes of }\Jj\text{-torsors on }X\right\}\cong H^1(X,\Jj)$, using \cite[Proposition~1.7.1\itememph{b}]{alggeo2}. However, there is a much easier and conceptionally better proof if we allow ourselves to use sheaf cohomology as a derived functor (which we didn't define though, but I think it should be known anyway at this level) rather than \v Cech cohomology.
	
	Let $i\colon X_0\morphism X$ denote the closed embedding of $X_0$. Then there's a short exact sequence $0\morphism\Jj\morphism \Oo_X\morphism i_*\Oo_{X_0}\morphism 0$. We claim that it gives rise to a short exact sequence
	\begin{align}\lbl{eq:multiplicativeSES}
		0\morphism(1+\Jj)\morphism \Oo_X^\times\morphism i_*\Oo_{X_0}^\times\morphism 0
	\end{align}
	of multiplicative sheaves. Indeed, exactness can be tested on stalks, so let's consider the stalks at $x\in X$. Since $\Jj_x^2=0$, $\Oo_{X,x}$ is $\Jj_x$-adically complete. Also $\Oo_{X_0,x}\cong \Oo_{X,x}/\Jj_x$, hence Corollary~\reff{cor:HenselApplications}\itememph{a} can be applied to show that $\Oo_{X,x}^\times\morphism \Oo_{X_0,x}^\times$ is surjective, with kernel clearly given by $1+\Jj_x$. This shows exactness of \eqreff{eq:multiplicativeSES}.
	
	Since $\Oo_X(X)\morphism \Oo_{X_0}(X_0)$ is surjective by assumption, we may apply the same argument applied to $\Oo_X(X)$ and $\Oo_{X_0}(X_0)\cong \Oo_X(X)/\Jj(X)$ to see that $\Oo_X(X)^\times\morphism\Oo_{X_0}(X_0)^\times$ is surjective. Moreover, the additive sheaf $\Jj$ is canonically isomorphic to $1+\Jj$ with multiplicative group structure, since $(1+j_1)(1+j_2)=1+j_1+j_2$ as $j_1j_2=0$. Therefore, the long exact cohomology sequence of \eqreff{eq:multiplicativeSES} appears as
	\begin{align}\lbl{eq:multiplicativeCohoSeq}
		\Oo_X(X)^\times\morphism\Oo_{X_0}(X_0)^\times \morphism H^1(X,\Jj)\morphism H^1(X,\Oo_X^\times)\morphism H^1(X,i_*\Oo_{X_0}^\times)\;.
	\end{align}
	Actually, it doesn't matter if we use sheaf or \v Cech cohomology here, since it is a well-known fact that both coincide in degrees $\leq 1$ (as can be seen from the \v Cech-to-derived spectral sequence). In this sequence, we can identify $H^1(X,\Oo_X^\times)\cong \Pic(X)$ by \cite[Section~1.7]{alggeo2}. Similarly, $H^1(X,i_*\Oo_{X_0}^\times)\cong H^1(X_0,\Oo_{X_0}^\times)\cong \Pic(X_0)$ since $X$ and $X_0$ are literally the same topological space (and $i_*\Oo_{X_0}^\times$ and $\Oo_{X_0}^\times$ are the same sheaf), only their prescheme structure differs, which is something neither sheaf nor \v Cech cohomology cares about. As explained above, $\Oo_X(X)^\times\morphism\Oo_{X_0}(X_0)^\times$ is surjective, so \eqreff{eq:multiplicativeCohoSeq} indeed provides the desired isomorphism $H^1(X,\Jj)\cong\ker\big(\Pic(X)\morphism\Pic(X_0)\big)$.
	
	Finally, let's leave some remarks on functoriality. That is, if $f\colon Y\morphism X$ is any morphism (such that $Y$ is a quasi-compact scheme, or we need to work with the more general construction of sheaf cohomology instead of \v Cech cohomology) and $f_0\colon Y_0=X_0\times_XY\morphism X_0$, then the diagram
	\begin{diagram}[baseline=0.75cm-0.5ex][\lbl{diag:H1Pic}]
		\object{0,1.5}{$H^1(Y,f^{-1}\Jj)$}[H1Y];
		\object{0,0}{$H^1(X,\Jj)$}[H1X];
		\object{3,1.5}{$\Pic(Y)$}[PicY];
		\object{6,1.5}{$\Pic(Y_0)$}[PicY0];
		\object{3,0}{$\Pic(X)$}[PicX];
		\object{6,0}{$\Pic(X_0)$}[PicX0];
		\scriptsize
		\arrow{H1X}{H1Y}[left][$f^*$];
		\arrow{PicX}{PicY}[left][$f^*$];
		\arrow{PicX0}{PicY0}[right][$f_0^*$];
		\arrow{H1Y}{PicY};
		\arrow{PicY}{PicY0};
		\arrow{H1X}{PicX};
		\arrow{PicX}{PicX0};
	\end{diagram}
	(in which $f^{-1}\Jj$ denotes the sheaf inverse image as in Definition~\reff{def:f-1} -- in particular, it's easy to check that $Y_0=V(f^{-1}\Jj)$ holds) commutes and has exact rows. To see why this works, we should probably explain the pullback map $f^*\colon H^1(X,\Jj)\morphism H^1(Y,f^{-1}\Jj)$ first (the other two guys should be clear). Let $\Uu\colon X=\bigcup_{i\in I}U_i$ be an affine open cover of $X$ and let $\Vv\colon Y=\bigcup_{j\in J} V_j$ be any affine refinement of $f^{-1}(\Uu)$. For any $j\in J$ let $v(j)\in I$ such that $V_j\subseteq f^{-1}(U_{v(j)})$. Then $(f^*\psi)_{j_0,\ldots,j_n}=\psi_{v(j_0),\ldots,v(j_n)}$ (this makes sense since we're working with $f^{-1}\Jj$ rather than $f^*\Jj$) defines a morphism of \v Cech complexes
	\begin{align*}
		f^*\colon \check{C}^\bullet(\Uu,\Jj)\morphism\check{C}^\bullet(\Vv,f^{-1}\Jj)\;.
	\end{align*}
	By choosing common affine refinements and using \cite[Lemma~1.2.1]{alggeo2}, it's not hard to show that the induced morphism $f^*\colon H^\bullet(X,\Jj)\morphism H^\bullet(Y,f^{-1}\Jj)$ in fact doesn't depend on the choice of $\Uu$ and $\Vv$. Also that $f^*$ defined in this way indeed makes \eqreff{diag:H1Pic} commute is just a matter of unwinding definitions.
\end{proof}
\begin{rem}
	Let $k$ be an algebraically closed field. Put $S=\Spec k$ and $S^{(1)}=\Spec k[T]/(T^2)$. For any \emph{$k$-variety} $X$ (that is, $X$ is an integral prescheme of finite type and separated over $k$) we have bijections
	\begin{align}\lbl{eq:ZariskiTangentSpace}
		\begin{split}
			\Hom_{\cat{PSch}/k}(S,X)&\cong\left\{\text{closed points of }X\right\}\\
			\Hom_{\cat{PSch}/k}\big(S^{(1)},X\big)&\cong\left\{(x,v)\st x\text{ closed point, }v\in\Omega_{X/k}(x)^*\right\}\;.
		\end{split}
	\end{align}
	The first of these bijections is quite clear. For the second one, note that a local  morphism of $k$-algebras $\phi\colon \Oo_{X,x}\morphism k[T]/(T^2)$ is uniquely determined by $\Oo_{X,x}\epimorphism\Oo_{X,x}/\mm_{X,x}\cong k$ (this isomorphism uses Hilbert's Nullstellensatz) and the map $d\colon \Oo_{X,x}\morphism k$ that sends $a\in \Oo_{X,x}$ to the linear coefficient of $\phi(a)$. Then $d$ is a $k$-linear derivation, hence uniquely determined by an $\Oo_{X,x}$-linear morphism $\Omega_{\Oo_{X,x}/k}\morphism k$. Now
	\begin{align*}
		\Hom_{\Oo_{X,x}}(\Omega_{\Oo_{X,x}/k},k)\cong \Hom_k(\Omega_{\Oo_{X,x}/k}\otimes_{\Oo_{X,x}}k,k)\cong \Omega_{X/k}(x)^*\;, 
	\end{align*}
	so the second bijection in \eqreff{eq:ZariskiTangentSpace} makes sense as well.
	
	As usual, $\Omega_{X/k}(x)^*\cong \Hom_k(\mm_{X,x}/\mm_{X,x}^2,k)$ (this isomorphism follows from \cite[Lemma~1.6.1]{alg2} and $\Omega_{k/k}=0$) can be interpreted as the \emph{tangent space} of $X$ at $x$. Hence, if $\PIC_{X/k}$ can be constructed\footnote{Actually it should be $\PIC_{X/S}$, (and also $\cat{PSch}/S$ rather than $\cat{PSch}/k$) but this abuse of notation is commonly agreed on.} we would expect its \emph{tangent space at the origin} to be
	\begin{align}\lbl{eq:TangentSpaceAtTheOrigin}
		\ker\left(\Pic_{X/k}\big(S^{(1)}\big)\morphism\Pic_{X/k}(S)\right)\;.%\cong\ker\big(\Pic(X\times_SS^{(1)})\morphism\Pic(X)\big)\cong H^1(X,\Oo_X)\;.
	\end{align}
	Why does this make sense and what does \emph{origin} even mean? Well, we have $\Pic_{X/k}\big(S^{(1)}\big)\cong \Hom_{\cat{PSch}/k}\big(S^{(1)},\PIC_{X/k}\big)$ and similarly $\Pic_{X/k}(S)\cong \Hom_{\cat{PSch}/k}(S,\PIC_{X/k})$ since $\PIC_{X/k}$ is supposed to represent the functor $\Pic_{X/k}$. In view of \eqreff{eq:ZariskiTangentSpace}, the kernel in \eqreff{eq:TangentSpaceAtTheOrigin} consists of those pairs $(x_0,v)$ (where $x_0\in X$ is closed and $v\in\Omega_{X/k}(x_0)^*$) such that $x_0$ corresponds to the morphism $\epsilon\colon S\morphism X$ that in turn corresponds to the trivial element in the group $\Pic_{X/k}(S)$. That is, $\epsilon$ is the \emph{zero section}, which plays the same role as the $\epsilon$ from page~\pageref{par:EllipticCurves} in the case of an elliptic curve. Then $x_0\in X$ is what we call the \emph{origin} of $X$. Therefore, we identified the kernel $\ker\big(\Pic_{X/k}\big(S^{(1)}\big)\morphism\Pic_{X/k}(S)\big)$ with $\{x_0\}\times\Omega_{X/k}(x_0)^*$, so it makes sense to call this kernel the \emph{tangent space at the origin}.
	
	Moreover, we have $\Pic_{X/k}\big(S^{(1)}\big)\cong \Pic\big(X^{(1)}\big)$, where $X^{(1)}=X\times_SS^{(1)}$. Indeed, $S=\Spec k$ consists of a single point, so every line bundle on $S$ is trivial. Then also every pullback of a line bundle on $S$ to $X^{(1)}$ is trivial, so $\Pic_{X/k}\big(S^{(1)}\big)$ is obtained from $\Pic\big(X^{(1)}\big)$ by quotienting out the trivial subgroup -- in other words, they're isomorphic, as claimed. The same reasoning applies, of course, to $\Pic_{X/k}(S)\cong \Pic(X)$.
	
	Let $i\colon X\monomorphism X^{(1)}$ the inclusion of $X$ as a closed subprescheme of its $1\ordinalst$ infinitesimal thickening. We denote by $\Jj=T\Oo_{X^{(1)}}\subseteq \Oo_{X^{(1)}}$ the sheaf of ideals defined by $X$. Then $\Jj^2=0$ and also $\Jj\cong i_*\Oo_X$. The latter isomorphism is a rather weird one -- locally, it corresponds to $A\cong TA[T]/T^2A[T]$ (given by multiplication with $T$) when $A$ is any ring and $T$ a formal variable. Thus we get $H^1\big(X^{(1)},\Jj\big)\cong H^1\big(X^{(1)},i_*\Oo_X\big)\cong H^1(X,\Oo_X)$, using that $i$ is affine, so \cite[Corollary~1.6.1]{alggeo2} can be applied. Also $\Oo_{X^{(1)}}\big(X^{(1)}\big)\cong \Oo_X(X)[T]/(T^2)\morphism\Oo_X(X)$ is surjective. Therefore Lemma~\reff{lem:kerPicH1} can be applied. Combining this with our previous considerations gives
	\begin{align*}
		\Omega_{X/k}(x_0)^*\cong \ker\left(\Pic_{X/k}\big(S^{(1)}\big)\morphism\Pic_{X/k}(S)\right)&\cong\ker\left(\Pic\big(X^{(1)}\big)\morphism\Pic(X)\right)\\
		&\cong H^1\big(X^{(1)},\Jj\big)\\
		&\cong H^1(X,\Oo_X)
	\end{align*}
	So one would expect $\PIC_{X/k}$ to have dimension $\dim_kH^1(X,\Oo_X)$ if it exists and its connected components are given by varieties. For instance, if $X$ is a regular curve of genus $g$ over $k$, then \cite[Proposition~1.6.3]{alg2} suggests that
	\begin{align*}
		\dim(\PIC_{X/k})=\dim_k\Omega_{X/k}(x_0)=\dim_kH^1(X,\Oo_X)=\dim_kH^0(X,\Omega_{X/k})=g
	\end{align*}
	(where we used Serre duality in the last step of the calculation).
\end{rem}
\begin{rem}
	%The following part of a Künneth-type formula for $H^1(X,\Oo_X)$ then is a further indication that
	By the following Künneth-type result in Lemma~\reff{lem:Künneth}, the functor 
	\begin{align*}
		X\longmapsto \text{tangent space of }\PIC_{X/k}\text{ at the origin}\cong H^1(X,\Oo_X)
	\end{align*}
	is compatible with products. This is further indication that a line bundle on $X\times_SY$ whose pullbacks to $X$ and $Y$ are trivial in $\Pic(X/S)$ and $\Pic(Y/S)$ cannot be ``deformed''.
\end{rem}
\begin{lem}\lbl{lem:Künneth}
	Let $X$ and $Y$ be proper integral schemes over $\Spec k$, where $k$ is algebraically closed. Denote by $f\colon X\times_kY\morphism X$ and $g\colon X\times_kY\morphism Y$ the projections to the fibre product factors. Then the pullback morphisms (we sketched the construction in the proof of Lemma~\reff{lem:kerPicH1}) $f^*\colon H^1(X,\Oo_X)\morphism H^1(X\times_k Y,\Oo_{X\times_kY})$ and $g^*\colon H^1(Y,\Oo_Y)\morphism H^1(X\times_k Y,\Oo_{X\times_kY})$ give an isomorphism
	\begin{align*}
		(f^*,g^*)\colon H^1(X,\Oo_X)\oplus H^1(Y,\Oo_Y)\isomorphism H^1(X\times_k Y,\Oo_{X\times_kY})\;.
	\end{align*}
\end{lem}
\begin{proof}[First proof (sketch)]
	Let $\Ff$ and $\Gg$ be quasi-coherent modules over $\Oo_X$ and $\Oo_Y$ and let $\Uu\colon X=\bigcup_{i\in I}U_i$, $\Vv\colon Y=\bigcup_{j\in J}V_j$ be affine open covers of $X$ and $Y$ respectively. Then the cohomology of $\Ff$ and $\Gg$ can be computed from the \v Cech complexes $\check{C}_X^\bullet=\check{C}^\bullet(\Uu,\Ff)$ and $\check{C}_Y^\bullet=\check{C}^\bullet(\Vv,\Gg)$. Moreover, $f^{-1}(\Uu)\cap g^{-1}(\Vv)\colon X\times_kY=\bigcup_{(i,j)\in I\times J}U_i\times_kV_j$ is an affine open cover of $X\times_kY$, so the \v Cech complex $\check{C}_{X\times_kY}^\bullet=\check{C}^\bullet(f^{-1}(\Uu)\cap g^{-1}(\Vv),f^*\Ff\otimes g^*\Gg)$ computes the cohomology of $f^*\Ff\otimes g^*\Gg$ (the tensor product is taken over $\Oo_{X\times_kY}$, but this would become utterly unreadable).
	
	The ``right'' way to think about these \v Cech complexes is not as some cochain complexes, but as \emph{cosimplicial objects} in their respective categories, as indicated in \cite[Remark~1.2.1]{alggeo2}. In this much more general context it can be shown that $\check{C}_{X\times_kY}^\bullet$ is homotopy equivalent to the tensor product $\check{C}_X^\bullet\otimes_k\check{C}_Y^\bullet$.\footnote{This is a tensor product of chain complexes. That is, $\check{C}_X^\bullet\otimes_k\check{C}_Y^\bullet$ is the total complex $\Tot^\bullet(\check{C})$ of the double complex $\check{C}^{\bullet,\bullet}$ defined by $\check{C}^{p,q}=\check{C}_X^p\otimes_k\check{C}_Y^q$.} In fact, there is an \emph{Alexander--Whitney map} $\Delta\colon \check{C}_{X\times_kY}^\bullet\morphism\check{C}_X^\bullet\otimes_k\check{C}_Y^\bullet$ and a \emph{shuffle map} (or \emph{Eilenber--Zilber map}) $\nabla\colon\check{C}_X^\bullet\otimes_k\check{C}_Y^\bullet\morphism\check{C}_{X\times_kY}^\bullet$ which are homotopy inverses of each other. For a quick description of these maps you may wish to consult the \href{https://ncatlab.org/nlab/show/Alexander-Whitney+map}{$n$Lab}.
	
	Therefore we see that
	\begin{align}\lbl{eq:Künneth}
		H^n\left(X\times_kY,f^*\Ff\otimes g^*\Gg\right)\cong H^n\left(\check{C}_X^\bullet\otimes_k\check{C}_Y^\bullet\right)\cong \bigoplus_{p+q=n}H^p(X,\Ff)\otimes_kH^q(Y,\Gg)\;,
	\end{align}
	from which the assertion follows as a special case since $H^0(X,\Oo_X)\cong k$ (and same for $Y$) by Proposition~\reff{prop:trivialLineBundlesOnProperX/k}\itememph{a} (and strictly speaking we still need to show that $(f^*,g^*)$ is actually the isomorphism which we obtain, but we won't do this here). The second isomorphism in \eqreff{eq:Künneth} follows from any of the two spectral sequences associated to the double complex $\check{C}^{\bullet,\bullet}$. Indeed, both spectral sequences converge to the cohomology of $\Tot^\bullet(\check{C})\cong \check{C}_X^\bullet\otimes_k\check{C}_Y^\bullet$, and since we're taking tensor products over a field, they degenerate soon enough.
\end{proof}
\begin{proof}[Second proof (sketch)]
	This proof is more down-to-the-earth and doesn't need the heavy theory of the first one -- however, it only shows the special case we're interested in. Note that $X\morphism \Spec k$ is automatically flat since $k$ is a field. Hence the base change morphism from Proposition~\reff{prop:baseChangeMorphism} becomes an isomorphism 
	\begin{align}\lbl{eq:baseChangeXxY}
		H^p(Y,\Oo_Y)\otimes_k\Oo_X\isomorphism R^pf_*\Oo_{X\times_kY}\;.
	\end{align}
	Hence the Leray spectral sequence $E_2^{p,q}=H^p(X,R^qf_*\Oo_{X\times_kY})\converge H^{p+q}(X\times_kY,\Oo_{X\times_kY})$ gives rise to a commutative diagram
	\begin{diagram*}
		\node[ob] (0o) at (0,1.5) {$0$};
		\node[ob]  (H0J) [right=0.35cm of 0o] {$H^1(X,f_*\Oo_{X\times_kY})$};
		\node[ob]  (H0O) [right=0.35cm of H0J] {$H^1(X\times_kY,\Oo_{X\times_kY})$};
		\node[ob]  (H0OX) [right=0.35cm of H0O] {$H^0(X,R^1f_*\Oo_{X\times_kY})$};
		\node[ob]  (H1J) [right=0.35cm of H0OX]  {$H^2(X,f_*\Oo_{X\times_kY})$};
		\node[ob]  (0u) at (0,0) {$0$};
		\node[ob]  (p) at (0u -| H0J) {$H^1(X,\Oo_X)$};
		\node[ob]  (kX) at (0u -| H0O) {$H^1(X\times_kY,\Oo_{X\times_kY})$};
		\node[ob]  (Ol) at (0u -| H0OX) {$H^1(Y,\Oo_Y)$};
		\scriptsize
		\draw[->] (0o) -- (H0J);
		\draw[->] (H0J) -- (H0O);
		\draw[->] (H0O) -- (H0OX);
		\draw[->] (H0OX) -- (H1J);
		\draw[->] (0u) -- (p);
		\draw[->] (p) -- (kX) node[pos=0.5, above] {$f^*$};
		\draw[->] (kX) -- (Ol);
		\draw[->, bend left,dotted] (Ol) to node[pos=0.5, below] {$g^*$} (kX);
		\draw[->] (H0J) -- (p) node[pos=0.5, sloped, above=-0.25ex] {$\sim$};
		\draw[transform canvas={xshift=1pt}] (H0O) -- (kX);
		\draw[transform canvas={xshift=-1pt}] (H0O) -- (kX);
		\draw[->] (H0OX) -- (Ol) node[pos=0.5, sloped, above=-0.25ex] {$\sim$};
	\end{diagram*}
	with exact rows. The top row is just the usual low-degree exact sequence of a spectral sequence. The vertical isomorphisms use $H^0(Y,\Oo_Y)\cong k\cong \Oo_X(X)$ and the base change isomorphism \eqreff{eq:baseChangeXxY}. Then it can be shown that $g^*$ gives a split of the rightmost map in the bottom row, which shows that $(f^*,g^*)\colon H^1(X,\Oo_X)\oplus H^1(Y,\Oo_Y)\isomorphism H^1(X\times_k Y,\Oo_{X\times_kY})$ is indeed an isomorphism. We're done.
\end{proof}
\begin{proof}[Proof of Theorem~\reff{thm:cube} \textsc{(Finally)}.]\lbl{proof:cube}
	Recall the shortcut (abuse of) notations $\Ll|_X=(\sigma_Y\xi,\id_X)^*\Ll$ and $\Ll|_Y=(\sigma_X\upsilon,\id_Y)^*\Ll$ introduced in the formulation of the theorem. Throughout the proof we'll denote pullbacks of $\Ll$ in that fashion.
	
	As usual, the assertion is local on $S$, whence we may assume that $S\cong \Spec A$ is affine and $\Ll_0$ is trivial on $S$, so that $\Ll|_X\cong \xi^*\Ll_0$ and $\Ll|_Y\cong \upsilon^*\Ll_0$ trivialize as well. The proof proceeds now in two steps.
	
	\emph{Step 1.} We first deal with the case that $A$ is an artinian local ring with maximal ideal $\mm$. That is, $\Spec A$ consists of a single point $s=\{\mm\}$ since $A$ is zero-dimensional (cf.~\cite[Proposition~3.1.1]{alg2}). In this case we can do induction on $\length_A(\mm)$. If $\length_A(\mm)=0$, then $\mm=0$, so $A$ is a field and any closed subprescheme of $\Spec A$ is open as well. Now assume the assertion is valid for all $S'=\Spec A'$ with $\length_{A'}(\mm')<\length_A(\mm)$. Choosing any $x\in \mm\setminus \{0\}$ such that $x\mm =0$ (such $x$ exists when $\mm\neq 0$ -- indeed, since $\mm$ is the only prime ideal of $A$ and $\Ass_A(A)\neq \emptyset$ by \cite[Corollary~A.3.1]{homalg}, $\mm$ is necessarily an associated prime of $A$) and applying the induction assumption to $A'=A/xA$ gives an open-closed subprescheme $Z'\subseteq S'$. Then $Z'=\emptyset$ or $Z'=S'$, using that $S'$ consists of a single point too. 
	
	In the first case, $Z'=\emptyset$ implies $Z=\emptyset$ (and we're done). Indeed, if $\tau\colon T\morphism S$ has the property that $\tau_{X\times_SY}^*\Ll$ is the trivial element in $\Pic_{X\times_SY/S}(T)$, then the base change $\tau'\colon T'=T\times_SS'\morphism S'$ a fortiori has the same property. But $Z'=\emptyset$, so $T'$ and thus also $T$ must be the empty scheme.
	
	So let's assume $Z'=S'$. Then $\Ll|_{X\times_SY\times_SS'}$ is trivial (in the sense that it is the pullback of some line bundle on $S'$ -- but $S'$ consists of a single point, so all line bundles are trivial, whence $\Ll|_{X\times_SY\times_SS'}$ is even in trivial in the sense that it is isomorphic to $\Oo_{X\times_SY\times_SS'}$), i.e., we have
	\begin{align*}
		\Ll\in\ker\left(\Pic(X\times_SY)\morphism\Pic(X\times_SY\times_SS')\right)\;.
	\end{align*}
	This looks like Lemma~\reff{lem:kerPicH1} can be applied, but to do this, we need to check that all conditions are satisfied. Let $\Jj\subseteq \Oo_{X\times_SY}$ be the sheaf of ideals defined by the closed subprescheme $X\times_SY\times_SS'\subseteq X\times_SY$. Since $x\mm=0$ and $x\in\mm$ we get $x^2=0$. Then also $\Jj^2=0$. To check surjectivity on global sections, we  apply Lemma~\reff{lem:flatProperWithIntegralFibres} to $X\times_SY$ and $X\times_SY\times_SS'$ to obtain that the morphism on global sections is $A\epimorphism A/xA$, hence surjective. Therefore all requirements are met and Lemma~\reff{lem:kerPicH1} shows that $\Ll$ corresponds to an element of $H^1(X\times_SY,\Jj)$.
	
	Now put $k=\KK(s)$ and let $i\colon X_s\times_kY_s\cong (X\times_SY)\times_S\Spec k\monomorphism X\times_SY$ denote the closed embedding of the fibre. Since $x\mm=0$, the multiplication with $x$ map $x\cdot\colon A\epimorphism xA$ factors over an isomorphism $k\cong xA$. This induces an isomorphism $i_*\Oo_{X_s\times_kY_s}\cong \Jj$. Hence
	\begin{align*}
		H^1(X\times_SY,\Jj)\cong H^1\left(X\times_SY,i_*\Oo_{X_s\times_kY_s}\right)\cong H^1\left(X_s\times_kY_s,\Oo_{X_s\times_kY_s}\right)\;.
	\end{align*}
	Tensoring with $k\monomorphism \ov{k}$ (which is faithfully flat) and going over to geometric fibres we finally see that $\Ll$ corresponds to an element of
	\begin{align*}
		H^1\left(X_s\times_kY_s,\Oo_{X_s\times_kY_s}\right)\otimes_k\ov{k}\cong H^1\left(X_{\ov{s}}\times_{\ov{k}}Y_{\ov{s}},\Oo_{X_{\ov{s}}\times_{\ov{k}}Y_{\ov{s}}}\right)\;.
	\end{align*}
	In the same way, $\Ll|_X$ and $\Ll|_Y$ correspond to elements of $H^1\big(X_{\ov{s}},\Oo_{X_{\ov{s}}}\big)$ and $H^1\big(Y_{\ov{s}},\Oo_{Y_{\ov{s}}}\big)$ respectively -- and actually we know they correspond to the trivial elements since we have $\Ll|_X\cong\xi^*\Ll_s$ and $\Ll|_Y\cong \upsilon^*\Ll_s$. But then the Künneth isomorphism from Lemma~\reff{lem:Künneth} shows that $\Ll$ represents the trivial element in
	\begin{align*}
		H^1\left(X_{\ov{s}},\Oo_{X_{\ov{s}}}\right)\oplus H^1\left(Y_{\ov{s}},\Oo_{Y_{\ov{s}}}\right)\cong H^1\left(X_{\ov{s}}\times_{\ov{k}}Y_{\ov{s}},\Oo_{X_{\ov{s}}\times_{\ov{k}}Y_{\ov{s}}}\right)\;,
	\end{align*}
	thus also in $H^1(X\times_SY,\Jj)$. Therefore $\Ll$ is already trivial and we obtain $Z=S$. This settles the special case of artinian local rings.
	
	\emph{Step 2.} We reduce to the case of artinian local rings. To do so, we check the condition for the upcoming Lemma~\reff{lem:ZClopen}. Let $S'\morphism S$ be any morphism with $S'\cong \Spec A$, where $A$ is artinian local with residue field $k$. Let $Z'\subseteq S'$ be the closed subprescheme associated with $\Ll|_{X\times_SY\times_SS'}$. By Step~1 we know that $Z'$ is also open, hence $Z'=\emptyset$ or $Z'=S'$. But if the restriction $\Spec k\morphism S$ of $j$ to $\Spec k$ factors over $Z$, then this shows that $Z'$ can't be empty. Hence $Z'=S'$ by Step~1, so $\Ll|_{X\times_SY\times_SS'}$ is already trivial in $\Pic_{X\times_SY/S}(S')$. But then $j\colon S'\morphism S$ factors over $Z$ -- that is, Lemma~\reff{lem:ZClopen} can indeed be applied, whence we are done.
\end{proof}
Last but not least we prove the two lemmas that have been outsourced during the above proof of Theorem~\reff{thm:cube}. The first is pretty useful and will be occasionally encountered during the rest of the text.
\begin{lem}\lbl{lem:flatProperWithIntegralFibres}
	Let $f\colon X\morphism S$ be a flat proper morphism with integral geometric fibres $X_{\ov{s}}$. Then the canonical morphism $\Oo_S\morphism f_*\Oo_X$ is an isomorphism.
\end{lem}
\begin{proof}
	Let $s\in S$ and put $k=\KK(s)$. From Proposition~\reff{prop:trivialLineBundlesOnProperX/k} we get $H^0(X_{\ov{s}},\Oo_{X_{\ov{s}}})\cong \ov k$. Since $H^0(X_{\ov{s}},\Oo_{X_{\ov{s}}})\cong H^0(X_s,\Oo_{X_s})\otimes_k\ov{k}$, this shows that $H^0(X_s,\Oo_{X_s})$ is a one-dimensional $k$-vector space as well, hence generated by $1$. Since $1$ is in the image of $(f_*\Oo_X)(s)\morphism H^0(X_s,\Oo_{X_s})$, this morphism is surjective for all $s\in S$ (and thus bijective by Theorem~\reff{thm:GrauertGrothendieck}\itememph{d}). Hence $U_0=S$. But $U_{-1}=X$ (our usual trick), hence $f_*\Oo_X$ is a vector bundle by Theorem~\reff{thm:GrauertGrothendieck}\itememph{e}. As noticed above, the $(f_*\Oo_X)(s)$ are generated by $1$, so $f_*\Oo_X$ must be a line bundle generated by $1$ -- that is, $\Oo_S\cong f_*\Oo_X$.
\end{proof}
\begin{lem}\lbl{lem:ZClopen}
	Let $S$ be a locally noetherian prescheme, $Z\subseteq S$ a closed subprescheme with the property that for any Artinian local ring $A$ with residue field $k$, a morphism $\Spec A\morphism S$ factors over $Z$ iff its restriction to $\Spec k$ does. Then $Z$ is also open.
\end{lem}
\begin{proof}
	It suffices to show that for any element $s\in S$ that lies in (the underlying set of) $Z$, there is an open neighbourhood $U\ni s$ contained in $Z$. We choose a preliminary $U$ to be an affine open neighbourhood $U\cong \Spec R$ of $s$, with $s$ corresponding to $\pp\in\Spec R$. Then $Z\cap U$ is given by an ideal $I\subseteq R$ and since $s\in Z$, the morphism $\Spec\KK(\pp)\morphism S$ factors over $Z$, so the same holds for $\Spec(R_\pp/\pp^nR_\pp)\morphism S$ for any $n\in \IN$ by our assumption. It follows that $I_\pp\subseteq \pp^nR_\pp$ for all $n\in \IN$. But $\bigcap_{n\geq 1}\pp^nR_\pp=0$ by the Krull intersection theorem (cf.\ \cite[Corollary~3.4.8]{alg2}). Then $I_\pp=0$, hence also $I_f=0$ for some $f\in R\setminus \pp$. Hence $Z$ contains the open neighbourhood $\Spec R\setminus V(f)$ of $s$.
\end{proof}
\section{Topological properties}\lbl{sec:TopologicalProperties}
The following proposition shouldn't be in a section titled \emph{Topological properties} at all, but we need to do this technical stuff at some point.
\begin{prop}\lbl{prop:faithfullyFlatModule}
	Let $A$ be a ring, $M$ a flat $A$-module, then the following conditions are equivalent.
	\begin{alphanumerate}
		\item $M\otimes_AT\neq 0$ for any $A$-module $T\neq 0$.
		\item For any maximal ideal $\mm\subseteq A$ we have $\mm M\subsetneq M$.
		\item For any proper ideal $I\subseteq A$ we have $IM\subsetneq M$.
	\end{alphanumerate}
	When $M=B$ is an $A$-algebra, this is moreover equivalent to
	\begin{alphanumerate}\setcounter{enumi}{3}
		\item Any maximal ideal is in the image of $\Spec B\morphism\Spec A$.
		\item $\Spec B\morphism\Spec A$ is surjective.
	\end{alphanumerate}
\end{prop}
\begin{proof}
	To see \itememph{a} $\Rightarrow$ \itememph{b} take $T=A/\mm$. For \itememph{b} $\Rightarrow$ \itememph{c}, take $\mm$ to be any maximal ideal containg $I$. Now let's prove \itememph{c} $\Rightarrow$ \itememph{a}. For any $t\in T\setminus\{0\}$ , the annullator $\Ann_A(t)$ is some proper ideal $I\subseteq A$, and multiplication by $t$ defines an injective morphism $A/I\monomorphism T$. Hence
	\begin{align*}
		M/IM\cong M\otimes_AA/I\monomorphism M\otimes_AT
	\end{align*}
	(injectivity of the right-most map is the only thing we need flatness of $M$ for), showing $M\otimes_AT\neq 0$ since $M/IM\neq 0$ by \itememph{c}.
	
	Now let $M=B$ be as above and let $f\colon \Spec B\morphism\Spec A$. It's well-known (and we used this several times before) that 
	\begin{align*}
		f^{-1}\{\pp\}\cong \Spec (B\otimes_A\KK(\pp))\quad\text{for all primes }\pp\in\Spec A\;,
	\end{align*}
	hence \itememph{a} $\Rightarrow$ \itememph{e} is immediate. Also \itememph{e} $\Rightarrow$ \itememph{d} for trivial reasons. Now if $\mm$ is a maximal ideal of $A$, then $\mm$ has a preimage in $\Spec B$ iff $0\neq \Spec (B\otimes_A\KK(\mm))\cong \Spec B/\mm B$. Hence \itememph{b} and \itememph{d} are equivalent and we're done.
\end{proof}
\begin{defi}\lbl{def:faithfullyFlat}
	In this case, $M$ (resp.\ $B$) is called \defemph{faithfully flat} over $A$.
\end{defi}
\begin{prop}\lbl{prop:flatGoingDown}
	Let $A$ be a ring and $B$ a flat $A$-algebra. Then the going-down theorem holds for the ring extension $B/A$. If $A$ is noetherian and $B$ of finite type over $R$, this implies that $\Spec B\morphism\Spec A$ is an open map.
\end{prop}
\begin{proof}
	Let $\pp'\subseteq \pp$ be primes in $\Spec A$ and let $\qq\in\Spec B$ such that $\qq\cap A=\pp$ (this is a lazy notation -- or actually \emph{abuse} of such -- for the preimage of $\qq$ in $\Spec A$). Then $B_\qq$ is a flat $A_\pp$-algebra to which Proposition~\reff{prop:faithfullyFlatModule} may be applied. As $\pp A_\pp$ is the only maximal ideal of $A_\pp$ and it is the preimage of $\qq B_\qq$, we see that $B_\qq/\pp B_\qq\neq 0$, hence $B_\qq$ is faithfully flat over $A_\pp$. Then there is $\snake{\qq}'\in\Spec B_\qq$ such that $\snake{\qq}'\cap A_\pp=\pp'A_\pp$. Then $\qq'=\snake{\qq}'\cap B$ does it.
	
	If $B$ is of finite type over $A$ and $A$ is noetherian, then to show that $\Spec B\morphism \Spec A$ is an open map it suffices to show that the image of $\Spec B_f$ in $\Spec A$ is open for all $f\in B$. Since $B_f$ is again flat and of finite type over $A$, we may as well replace $B$ by $B_f$. Then the image of $\Spec B$ is \emph{constructible} by Chevalley's theorem (Proposition~\reff{prop:Chevalley}, Definition~\reff{def:constructible}). It is also \emph{stable under generalization} in the sense of Definition~\reff{def:specialization}\itememph{b} since this is precisely what going-down says. By Proposition~\reff{prop:constructible2}\itememph{b} this implies that the image of $\Spec B$ open, as required.
\end{proof}
\begin{cor}\lbl{cor:flatMorphismOpen}
	A flat morphism of locally finite type $f\colon X\morphism Y$ between locally noetherian preschemes defines an open map of the underlying topological spaces.
\end{cor}
\begin{proof}
	Affine-locally on $X$ and $Y$ this is just Proposition~\reff{prop:flatGoingDown}.
\end{proof}
\begin{rem}\lbl{rem:nonNoetherianChevalley}
	One can generalize Corollary~\reff{cor:flatMorphismOpen} to all flat morphisms of locally finite presentation (which is automatic if $X$ and $Y$ are locally noetherian and $f$ has finite type). The strategy in general is quite the same as in the noetherian case: First one has to tweak our notion of constructibility so that it behaves well in the non-noetherian case (see \cite[\stackstag{04ZC} and  \stackstag{00I0}]{stacks-project}). Then one proves a generalization of Chevalley's theorem to arbitrary morphisms of locally finite presentation. The key idea in the proof is an ingenious trick that reduces everything to the noetherian case. You can find a very nice exposition of this in Akhil Mathews blog, see \url{https://amathew.wordpress.com/2010/12/26/}.
\end{rem}
\begin{prop}\lbl{prop:toplogicalQuotient}
	If $f\colon X\morphism Y$ is a quasi-compact and faithfully flat morphism of arbitrary preschemes, then the underlying topological space $Y_\Top$ is the quotient of $X_\Top$ by the equivalence relation $X_\Top\times_{Y_\Top}X_{\Top}\subseteq X_\Top\times X_\Top$, which is the image of $(X\times_YX)_\Top$ in $X_\Top\times X_\Top$. In less fancy words, the equivalence relation under consideration is given by $x\sim x'\Leftrightarrow f(x)=f(x')$ (i.e., the one we would expect).
\end{prop}
\begin{proof}
	From the general fact that $(X\times_YX)_\Top\morphism X_\Top\times_{Y_\Top}X_\Top$ is surjective (cf.\ \cite[Corollary~1.3.1\itememph{a}]{alggeo1}), all set-theoretic properties follow. It remains to show the topological assertion, which is equivalent to the condition that for of any open subset $U\subseteq X$ which is closed under the equivalence relation the image $f(U)\subseteq Y$ is open. When $Y$ is noetherian and $X$ of locally finite type over it, this follows from Corollary~\reff{cor:flatMorphismOpen}. In general we may equivalently show that if $A\subseteq X$ is a closed subset and closed under the equivalence relation, then $f(A)$ is closed in $Y$. If $A$ has this property, then applying Lemma~\reff{lem:closurePreimage} to $Z=f(A)$ (so that $A=f^{-1}(Z)$) gives $f^{-1}(\ov{Z})=\ov{A}=A=f^{-1}(Z)$. Hence $\ov{Z}=Z$ as $f$ is surjective, so $Z$ is closed in $Y$.
\end{proof}
\begin{lem}[{\cite[Exposé VII Théorème 4.1]{sga1}}\footnote{Actually, what the SGA$_1$ table of contents says about Exposé VII is \emph{n'existe pas}, which even my non-existent French knowledge translates to \emph{doesn't exist}. It should be Exposé VIII (and I find it quite remarkable that Professor Franke managed to hit the only existing non-existing exposé in SGA$_1$ \Tongey[1.2]).}]\lbl{lem:closurePreimage}
	Let $f\colon X\morphism Y$ be a flat morphism and $Z\subseteq Y$ a subset which is the image of a quasi-compact morphism $\tau\colon T\morphism Y$. Then 
	\begin{align*}
		f^{-1}(\ov{Z})=\ov{f^{-1}(Z)}\;.
	\end{align*}
\end{lem}
\begin{proof}
	The assertion is local with respect to $Y$, hence we may assume $Y\cong\Spec A$ to be affine. Then $T$ is quasi-compact, hence a finite union of affine open subsets. Clearly $T$ may be replaced by the disjoint union of these finitely many affine open subset, hence without losing generality we may assume that $T$ itself is affine, say, $T\cong\Spec R$ for some $A$-algebra $R$. Moreover, the assertion is also local with respect to $X$, hence we may assume $X=\Spec B$ where $B$ is a flat $A$-algebra.
	
	Let $I=\ker(\rho\colon A\morphism R)$ so that $\ov{Z}=V(I)$. Indeed, this is a general fact that just hasn't shown up until now. To see why this is true, note $\ov{Z}=V\left(\bigcap_{\pp\in Z}\pp\right)$. Now 
	\begin{align*}
		\bigcap_{\pp\in Z}\pp=\bigcap_{\qq\in \Spec R}\rho^{-1}(\qq)=\rho^{-1}\bigg(\bigcap_{\qq\in \Spec R}\qq\bigg)=\rho^{-1}(\nil R)=\sqrt{I}\;,
	\end{align*}
	whence $\ov{Z}=V(\sqrt{I})=V(I)$ as claimed. Also $f^{-1}(Z)$ is the image of $X\times_YT\morphism X$, as is easily checked. Since $X\times_YT\cong \Spec(B\otimes_AR)$, the same argument as for $\ov{Z}$ provides $\ov{f^{-1}(Z)}=V(J)$ where $J=\ker(B\morphism B\otimes_AR)$. But $B\cong B\otimes_AA$, hence $J=I\otimes_AB=IB$ by flatness of $B$. Therefore
	\begin{align*}
		\ov{f^{-1}(Z)}=V(IB)=f^{-1}(V(I))=f^{-1}(\ov{Z})\;,
	\end{align*}
	as required. We're done.
\end{proof}
\begin{rem}
	The quasi-compactness of $f\colon X\morphism Y$ is needed unless $Y$ is locally noetherian and $f$ locally of finite type.
\end{rem}
\section{Faithfully flat descent and the fpqc topology (and related ones)}\lbl{sec:Descent}
Let $f\colon X\morphism Y$ be a morphism of preschemes and $\Ff$ a quasi-coherent $\Oo_Y$-module. We would like to reconstruct $\Ff$ from its inverse image $f^*\Ff$. This will need some assumptions (in fact, faithful flatness and quasi-compactness, as in Proposition~\reff{prop:toplogicalQuotient}). For example, assume that $Y=\Spec K$ and $X=\Spec L$ where $L/K$ is some field extension. Then $\Ff=\snake{V}$ for some $K$-vector space $V$ and $f^*\Ff\cong (V\otimes_KL)^\qcmod$. Not every endomorphism of $f^*\Ff$ will thus come from an endomorphism of $\Ff$ and we must take care of some additional structure on $f^*\Ff$. For instance, if $L/K$ is Galois in the above situation, we will have $\Gal(L/K)$ acting on $V\otimes_KL$.

\lbl{page:ZariskiDescent}As another example of what we're up to, let $\Uu\colon Y=\bigcup_{i\in I}U_i$ be an open cover of $Y$ and let $X=\coprod_{i\in I}U_i\morphism Y$ be the obvious morphism. We ask which quasi-coherent $\Oo_X$-modules $\Gg$ can be represented as $\Gg\cong f^*\Ff$ for some quasi-coherent $\Oo_Y$-module $\Ff$. If $\Gg_i$ is the restriction of $\Gg$ to the copy of $U_i$ in $X$, then clearly we must have an isomorphism $\Gg_i\cong \Ff|_{U_i}$. So the the problem in this case becomes whether the $\Gg_i$ can be glued together to form a sheaf over all of $Y$. This requires some auxiliary structure on the $\Gg_i$. First, we have isomorphisms $(\Ff|_{U_i})|_{U_{i,j}}\cong (\Ff|_{U_j})|_{U_{i,j}}$, where $U_{i,j}=U_i\cap U_j$ as usual. Hence the $\Gg_i$ must come with isomorphisms $\mu_{i,j}\colon \Gg_i|_{U_{i,j}}\isomorphism \Gg_j|_{U_{i,j}}$. Moreover, these $\mu_{i,j}$ must be compatible in the sense that
\begin{diagram}[baseline=0.625cm-0.5pt][\lbl{diag:muij}]
	\object{0,1.25}{$\Gg_i|_{U_{i,j,k}}$}[i];
	\object{1.5,0}{$\Gg_k|_{U_{i,j,k}}$}[k];
	\object{3,1.25}{$\Gg_j|_{U_{i,j,k}}$}[j];
	\scriptsize
	\arrow{i}{j}[above][$\mu_{i,j}|_{U_{i,j,k}}$];
	\arrow{i}{k}[below left][$\mu_{i,k}|_{U_{i,j,k}}$];
	\arrow{j}{k}[below right][$\mu_{j,k}|_{U_{i,j,k}}$];
\end{diagram}
commutes (since this is also the case for the $\Ff|_{U_i}$). On the other hand, if $(\Gg_i)_{i\in I}$ and $(\mu_{i,j})_{(i,j)\in I^2}$ are given such that \eqreff{diag:muij} is commutative, then it's easy to see that these data define a sheaf on $Y$.

 Note that $U_i\cap U_j$ can also be written as $U_i\times_YU_j$. Also the disjoint union $\coprod_{(i,j)\in I^2}U_{i,j}$ can be written as $X\times_YX$ in the case where $X=\bigcup_{i\in I}U_i$. Therefore, providing a family of isomorphisms $\mu_{i,j}\colon \Gg_i|_{U_{i,j}}\isomorphism \Gg_j|_{U_{i,j}}$ is equivalent to giving a single isomorphism $p_1^*\Gg\cong p_2^*\Gg$, where $p_1,p_2\colon X\times_YX\morphism X$ are the projections to the two factors. Also the diagrams \eqreff{diag:muij} can be reformulated as a single large diagram. We'll immediately make this precise. 
 
Let's fix the following notation: Let $p_1,p_2\colon X\times_YX\morphism X$ denote the projections to the two factors as above. For better distinction, we denote by $\pi_i$ and $\pi_{i,j}$ the projections from $X\times_YX\times_YX$ to the $i\ordinalth$ factor resp.\ to the $i\ordinalth$ and $j\ordinalth$ factor.
\begin{defi}\lbl{def:descentData}
	Let $f\colon X\morphism Y$ be a morphism of preschemes. A \defemph{descent datum} for quasi-coherent sheaves of modules along $f$ is a pair $(\Gg,\nu)$, where $\Gg$ is a quasi-coherent $\Oo_X$-module and $\nu\colon  p_1^*\Gg\isomorphism p_2^*\Gg$ an isomorphism, such that the following \sout{monster diagram} moderately large diagram commutes.
	\begin{diagram*}
		\object{60:3.5}{$ \pi_{1,2}^* p_2^*\Gg$}[a];
		\object{120:3.5}{$ \pi_{1,2}^* p_1^*\Gg$}[b];
		\object{180:3.5}{$ \pi_{1,3}^* p_1^*\Gg$}[c];
		\object{240:3.5}{$ \pi_{1,3}^* p_2^*\Gg$}[d];
		\object{300:3.5}{$ \pi_{2,3}^* p_2^*\Gg$}[e];
		\object{0:3.5}{$ \pi_{2,3}^* p_1^*\Gg$}[f];
		\object{30:3.5}{$\pi_2^*\Gg$}[1];
		\object{150:3.5}{$\pi_1^*\Gg$}[2];
		\object{270:3.5}{$\pi_3^*\Gg$}[3];
		\scriptsize
		\isoarrow{2}{b};
		\isoarrow2c;
		\isoarrow d3;
		\isoarrow e3;
		\isoarrow a1;
		\isoarrow 1f;
		\isoarrow ba[below][$ \pi_{1,2}^*(\nu)$];
		\isoarrow cd[below left][$ \pi_{1,3}^*(\nu)$];
		\isoarrow fe[below right][$ \pi_{2,3}^*(\nu)$];
	\end{diagram*}
	Commutativity of this diagram is often called the \emph{cocycle condition}.
	
	A \defemph{morphism of descent data} $\phi\colon (\Gg,\nu)\morphism(\Gg',\nu')$ is consists of a morphism $\phi\colon \Gg\morphism\Gg'$ of $\Oo_X$-modules such that
	\begin{diagram*}
		\object{0,1.5}{$ p_1^*\Gg$}[a];
		\object{2.5,1.5}{$ p_2^*\Gg$}[b];
		\object{0,0}{$ p_1^*\Gg'$}[c];
		\object{2.5,0}{$ p_2^*\Gg'$}[d];
		\scriptsize
		\isoarrow{a}{b}[below][$\nu$];
		\isoarrow{c}{d}[below][$\nu'$];
		\arrow ac[left][$ p_1^*(\phi)$];
		\arrow bd[right][$ p_2^*(\phi)$];
	\end{diagram*}
	commutes. We thus obtain a category $\cat{Desc}_f=\cat{Desc}_{X/Y}$ of descent data for quasi-coherent sheaves of modules along $f$.
\end{defi}
\begin{rem}
	It's easy to see that in the case where $X$ results from a Zariski-open cover of $Y$ as above, the definition is equivalent to the gluing data that was briefly sketched there.
\end{rem}
\begin{rem}\lbl{rem:equivalentDefOfDescentData}
	An equivalent (and somewhat easier) approach is to require a family of isomorphisms $\nu_{g,h}\colon g^*\Gg\isomorphism h^*\Gg$ for arbitrary morphisms $g,h\colon T\morphism X$ such that $fg=fh$, rather than a single $\nu$. These $\nu_{g,h}$ should satisfy the following relations.
		\begin{alphanumerate}
			\item If $\tau\colon \snake{T}\morphism T$ is any morphism, then $\nu_{g\tau,h\tau}=\tau^*(\nu_{g,h})$. More precisely (that is, without surpressing canonical isomorphisms in our notation), the following diagram is supposed to commute.
			\begin{diagram*}
				\object{0,1.5}{$\tau^*g^*\Gg$}[a];
				\object{2.5,1.5}{$\tau^*h^*\Gg$}[b];
				\object{0,0}{$(g\tau)^*\Gg$}[c];
				\object{2.5,0}{$(h\tau)^*\Gg$}[d];
				\scriptsize
				\isoarrow ac;
				\isoarrow bd;
				\arrow{a}{b}[above][$\tau^*(\nu_{g,h})$];
				\arrow{c}{d}[above][$\nu_{g\tau,h\tau}$];
			\end{diagram*}
			\item If $g,h,k\colon T\morphism X$ is a triple of morphisms such that $fg=fh=fk$, then the diagram
			\begin{diagram*}
				\object{0,1.25}{$g^*\Gg$}[i];
				\object{1.25,0}{$h^*\Gg$}[k];
				\object{2.5,1.25}{$k^*\Gg$}[j];
				\scriptsize
				\arrow{i}{j}[above][$\nu_{g,h}$];
				\arrow{i}{k}[below left][$\nu_{g,k}$];
				\arrow{j}{k}[below right][$\nu_{h,k}$];
			\end{diagram*}
			commutes.
		\end{alphanumerate}
		Such a family is obtained from $\nu$ by $\nu_{g,h}=(g,h)^*(\nu)$, where $(g,h)\colon T\morphism X\times_YX$ is defined in the obvious way via the universal property of fibre products. Conversely, $\nu$ can be recovered from such a family by $\nu=\nu_{ p_1, p_2}$.
\end{rem}
\begin{rem}
	If $\Ff$ is a quasi-coherent $\Oo_Y$-module, then $\Gg=f^*\Ff$ together with the canonical isomorphism
	\begin{align*}
		\nu\colon  p_1^*\Gg= p_1^*f_*\Ff\isomorphism (f p_1)^*\Ff=(f p_2)^*\Ff\isomorphism  p_2^*f^*\Ff= p_2^*\Gg
	\end{align*}
	is an object of the category $\cat{Desc}_{X/Y}$. Moreover, for any morphism $\phi\colon \Ff\morphism \Ff'$ the morphism $f^*(\phi)\colon \Gg=f^*\Ff\morphism f^*\Ff'=\Gg'$ is a morphism in $\cat{Desc}_{X/Y}$ (this is straightforward to check). One thus obtains a functor $f^*\colon\cat{QCoh}(Y)\morphism \cat{Desc}_{X/Y}$.
\end{rem}
\begin{thm}\lbl{thm:faithfullyFlatDescent}
	If $f\colon X\morphism Y$ is a faithfully flat and quasi-compact morphism, then the above functor $f^*\colon\cat{QCoh}(Y)\morphism\cat{Desc}_{X/Y}$ is an equivalence of categories.
\end{thm}
\begin{proof}[Sketch of a proof]
	If we tried to prove Theorem~\reff{thm:faithfullyFlatDescent} directly, we would get into trouble, because our \emph{ad-hoc} construction of cohomology in \cite[Section~1.2]{alggeo2} via the \v Cech complex only works for quasi-compact schemes (which $X$ and $Y$ are not, in general). However, even if we define sheaf cohomology the Grothendieck way as a derived functor, some of the arguments below (to be precise, those in Step~2) wouldn't work, e.g.\ since the direct image under $f$ of a quasi-coherent module is, in general, not quasi-coherent again, unless $f$ is quasi-separated in addition to being quasi-compact (cf.\ \cite[Proposition~1.5.2]{alggeo1}). That's why we first have to reduce to a sufficiently nice situation.
	
	Also note that Step~$-1$ and 0 weren't discussed in the lecture, so this is again me filling in the blanks Professor Franke (intentionally) let.
	
	\emph{Step $-\text{1}$.} We reduce the situation to the case where $Y$ is affine. Suppose this special case has been settled and choose an affine open cover $Y=\bigcup_{j\in J}V_j$. Then each $\cat{QCoh}(V_j)\morphism\cat{Desc}_{f^{-1}(V_j)/V_j}$ is an equivalence of categories. Hence given an object $(\Gg,\nu)$ of $\cat{Desc}_{X/Y}$, we find quasi-coherent sheaves $\Ff_j$ on $V_j$ such that $\Gg|_{f^{-1}(V_j)}$ is the pullback of $\Ff_j$ along $f$. We need to show that the $\Ff_j$ can be glued together. If we construct a morphism $\mu_{i,j}\colon \Ff_i|_{V_{i,j}}\morphism\Ff_j|_{V_{i,j}}$ in a ``sufficiently canonical'' way, then $\mu_{i,j}$ will be an isomorphism since the pullback of $\mu_{i,j}$ along $f$ will become an isomorphism (roughly speaking the identity on $\Gg|_{f^{-1}(V_{i,j})}$), and $f$ is faithfully flat, so $\mu_{i,j}$ must have already been an isomorphism.
	
	To construct $\mu_{i,j}$, we cover $V_{i,j}=\bigcup_{k\in K}W_k$ by affine open $W_k$. Then we get canonical morphisms $\mu_k\colon \Ff_i|_{W_k}\morphism\Ff_j|_{W_k}$ since $\cat{QCoh}(W_k)\morphism\cat{Desc}_{f^{-1}(W_k)/W_k}$ is fully faithful. Also $\mu_k$ and $\mu_\ell$ coincide on $W_k\cap W_\ell$, which can be seen by covering $W_k\cap W_\ell$ with affine opens and using full faithfulness in the affine case once again. Thus the $\mu_k$ glue together to some $\mu_{i,j}$, and these guys have property \eqreff{diag:muij}, as is easily checked.
	
	\emph{Step 0.} We reduce to the case where $X$ is affine too (hence $X$ and $Y$ are quasi-compact schemes, which suffices for the rest of the proof). Let $X=\bigcup_{i=1}^nU_i$ be an affine open cover and put $X'=\coprod_{i=1}^nU_i$ (finiteness of $n$ is fine because $f$ and thus $X$ are quasi-compact). It's easy to see that descent data $(\Gg,\nu)$ along $f\colon X\morphism Y$ can be pulled back to descent data $(\Gg',\nu')$ along $f'\colon X'\morphism Y$. Then $X'$ is affine, so assuming that the affine case can be shown, there is a quasi-coherent $\Oo_Y$-module $\Ff$ such that $\Gg'\cong f'^*\Ff$. That is, $f^*\Ff$ pulls back to $\Gg'$, as does $\Gg$, hence $f^*\Ff\cong \Gg$ by ``Zariski descent'' (i.e., the special case of gluing sheaves that was discussed on page~\pageref{page:ZariskiDescent}; I also admit I'm really handwavy here).
	
	\emph{Step 1.} Assume that $f\colon X\morphism Y$ has a section $\sigma\colon Y\morphism X$. Then $\sigma^*\colon\cat{Desc}_{X/Y}\morphism\cat{QCoh}(Y)$ given by $(\Gg,\nu)\mapsto \sigma^*\Gg$ is an inverse to $f^*$ up to canonical isomorphism. If $\Gg=f^*\Ff$, said isomorphisms are $\sigma^*\Gg=\sigma^*f^*\Ff\cong (f\sigma)^*\Ff=\id_Y^*\Ff\cong \Ff$. And if $(\Gg,\nu)$ is a descent datum and $\Ff=\sigma^*\Gg$, then
	\begin{align*}
		f^*\Ff=f^*\sigma^*\Gg\cong(\sigma f)^*\Gg\isomorphism[\nu_{\sigma f,\id}]\id_X^*\Gg\cong \Gg\;,
	\end{align*}
	where $\nu_{\sigma f,\id}$ is as in Remark~\reff{rem:equivalentDefOfDescentData} (and for this we use $f\sigma f=\id_Yf=f=f\id_X$). The fact that this is an isomorphism of descent data is left as an exercise. 
	
	\emph{Step 2.} We construct an adjoint functor $R\colon \cat{Desc}_{X/Y}\morphism\cat{QCoh}(Y)$. Put
	\begin{align*}
		R(\Gg,\nu)=\Eq\left(\begin{tikzpicture}[line cap=round, line join=round,baseline=0.625cm-0.5ex]
			\object{0,0}{$f_*\Gg$}[a];
			\object{1.75,1.25}{$(f p_1)_* p_1^*\Gg$}[b];
			\object{3.5,0}{$(f p_2)_* p_2^*\Gg$}[c];
			\scriptsize
			\arrow ac;
			\arrow ab;
			\draw[->] (b) to node[pos=0.5,above right]{$(f p_1)_*(\nu)$} node[pos=0.5,sloped,below=-0.25ex]{$\sim$} (c);
		\end{tikzpicture}\right)\;,
	\end{align*}
	where the left and the bottom arrows are induced by the canonical morphisms $\Gg\morphism p_{i,*} p_i^*\Gg$ for $i\in\{1,2\}$ and the right arrow uses $f p_1=f p_2$. Note that we really need $f$ to be quasi-compact and quasi-separated here, otherwise $f_*\Gg$ wouldn't necessarily be a quasi-coherent $\Oo_Y$-module (cf.\ \cite[Proposition~1.5.2]{alggeo1}). It's straightforward to check that this $R$ is indeed a right-adjoint of $f^*$. Also $R$ is compatible with base change by flat morphisms using Proposition~\reff{prop:baseChangeMorphism}.
	
	\emph{Step~3}. To show that $R$ is an equivalence of categories, all we need to do is to show that the canonical morphisms $\Ff\morphism Rf^*\Ff$ and $f^*R\Gg\morphism \Gg$ are isomorphisms. It's sufficient to do this after some faithfully flat base change $\upsilon\colon\snake{Y}\morphism Y$ (or in fancy words, we may check this \emph{locally in the fpqc topology}). We apply this to $\upsilon=f\colon\snake{Y}=X\morphism Y$. Then the base change $ p_2\colon\snake{X}=X\times_YX\morphism X$ of $f$ has a section given by the diagonal $\Delta$, hence $ p_2^*$ is an equivalence of categories by Step~1. Since $R$ is compatible with flat base change, we get $\upsilon^*Rf^*\Ff\cong R p_2^*(\upsilon^*\Ff)$. Hence showing that $\Ff\cong Rf^*\Ff$ becomes equivalent to $\upsilon^*\Ff\morphism R p_2^*(\upsilon^*\Ff)$ being an isomorphism. Since $ p_2^*$ is an equivalence of categories and $R$ a right-adjoint of it, this follows from some general abstract nonsense.\footnote{Hint: Show $\Hom_{\Oo_Y}(-,\upsilon^*\Ff)\cong \Hom_{\Oo_Y}(-,Rp_2^*(\upsilon^*\Ff))$ and use Yoneda.} The same argument goes for $f^*R\Gg\cong \Gg$ as well, whence we are done.
\end{proof}
\begin{rem}
	As pointed out several times during the proof, quasi-compactness cannot be omitted. Here is an explicit counterexample. Let $R$ be a non-semilocal Dedekind domain (e.g.\ $R=\IZ$, the point is we need infinitely many primes) and put $Y=\Spec R$, $X=\coprod_{\pp\neq 0}\Spec R_\pp$. Take $f\colon X\morphism Y$ to be the obvious morphism. Consider the morphism
	\begin{align}\lbl{eq:nonSplitButSplit}
		\bigoplus_{\pp\neq 0}R/\pp\monomorphism\prod_{\pp\neq 0}R/\pp\;.\tag{$*$}
	\end{align}
	of $R$-modules. We'll show that \eqreff{eq:nonSplitButSplit} is non-split over $R$, hence also the corresponding morphism of $\Oo_Y$-modules is non-split, but its pullback to $X$ becomes split. Then $f^*$ can't be an equivalence of categories (because this fails on the level of morphisms).
	
	\emph{Step 1.} Let's first show that the pullback of \eqreff{eq:nonSplitButSplit} to $X$ becomes split. This can be tested on each $\Spec R_\qq$ component separately, so it suffices to show that \eqreff{eq:nonSplitButSplit} splits (uniquely) after tensoring with $-\otimes_RR_\qq$. Let $O=\bigoplus_{\pp\neq \qq,0}R/\pp$ and $P=\prod_{\pp\neq \qq,0}R/\pp$. Then $O\otimes_RR_\qq=0$ and $P\otimes_RR_\qq=V$ is a vector space over the quotient field $K$ of $R$ (to see this, we only need to check that multiplication with a uniformizer $\pi\in R_\qq$ is bijective, which is easy to check). Then
	\begin{align*}
		\bigg(\bigoplus_{\pp\neq 0}R/\pp\bigg)\otimes_RR_\qq\cong R/\qq\oplus 0\monomorphism R/\qq\oplus V\cong \bigg(\prod_{\pp\neq 0}R/\pp\bigg)\otimes_RR_\qq
	\end{align*}
	has a unique split since the only morphism of $R_\qq$-modules $V\morphism R/\qq$ is $0$ (since multiplication with $\pi$ is bijective on the $K$-vector space $V$ but $0$ on $R/\qq$). Using
	\begin{align*}
		X\times_YX\cong \coprod_{\pp,\pp'\neq 0}\Spec(R_\pp\otimes_RR_{\pp'})\cong \coprod_{\substack{\pp,\pp'\neq 0\\\pp\neq \pp'}}\Spec K\amalg \coprod_{\pp\neq 0}\Spec R_\pp\;,
	\end{align*}
	it's easy to see that the obtained split of the pullback of \eqreff{eq:nonSplitButSplit} is indeed a morphism of descent data.
	
	\emph{Step 2.} We show that \eqreff{eq:nonSplitButSplit} is not split. Any morphism $\prod_{\pp\neq 0}R/\pp\morphism R/\qq$ must be of the form $(r_\pp)_{\pp\neq 0}\mapsto \lambda r_\qq$ for some $\lambda\in R$ (which is unique modulo $\qq$). Indeed, this follows from
	\begin{align*}
		\Hom_R\bigg(\prod_{\pp\neq\qq,0}R/\pp,R/\qq\bigg)=0\;.
	\end{align*}
	For $R=\IZ$, the above equation is easy to see since multiplication by $q$ (which is chosen such that $\qq=(q)$) is bijective on the source, but $0$ on the target. For general Dedekind domains some variation of this argument still works. In order for $\prod_{\pp\neq 0}R/\pp\morphism R/\qq$ to be part of a split of \eqreff{eq:nonSplitButSplit}, we must have 
	$\lambda\equiv1\mod \qq$. But then there's no way the image of $(r_\pp)_{\pp\neq 0}$ has only finitely many non-zero entries if $(r_\pp)_{\pp\neq 0}$ doesn't already have this property -- contradiction!
\end{rem}
\begin{rem}
	However, there are situations where quasi-compactness can be dropped. For example, when $f\colon X\morphism Y$ has a split $\sigma$ (as follows from the proof of Theorem~\reff{thm:faithfullyFlatDescent}). Also, ``Zariski descent'' -- the case of gluing sheaves defined on an open cover -- works regardless of quasi-compactness.% (which in this case is equivalent to the open cover being locally finite).
\end{rem}
Often we would like to descend more general objects than just quasi-coherent sheaves on $Y$. The usual setting for descent theory are \defemph{fibred categories} over $\cat{PSch}/S$ (or another category). This roughly means the following: To every $S$-prescheme\footnote{This $S$ is really just there for technical reasons; it wasn't involved at all in Theorem~\reff{thm:faithfullyFlatDescent}. We could take $S=\Spec \IZ$ to drop $S$ for good.} $T$ we assign a category $\cat C_T$ of objects we would like to descend (like $\cat C_T=\cat{QCoh}(T)$ in Theorem~\reff{thm:faithfullyFlatDescent}) and to every morphism $g\colon T'\morphism T$ we require a pullback functor $g^*\colon \cat{C}_T\morphism\cat C_{T'}$. Then we probably need some more technical conditions (like $g^*h^*$ and $(hg)^*$ being naturally isomorphic), but we didn't get into details in the lecture.

If $\cat C=(\cat C_T)_{T\in\Ob(\cat{PSch}/S)}$ is such a fibred category over $\cat{PSch}/S$ and $f\colon X\morphism Y$ a morphism of $S$-preschemes, we can define \defemph{$\cat C$-valued descent data} along $f$ as pairs $(\Gg,\nu)$, where $\Gg\in\Ob(\cat C_X)$ and $\nu\colon  p_1^*\Gg\isomorphism p_2^*\Gg$ is an isomorphism satisfying the cocycle conditions, i.e., the moderately large diagram from Definition~\reff{def:descentData} commutes. This gives a category $\cat{Desc}_{X/Y}^{\cat{C}}$, and, like in the situation before, there is a functor
\begin{align*}
	f^*\colon \cat C_Y\morphism\cat{Desc}_{X/Y}^{\cat{C}}
\end{align*}
into it (it's intentional that $f^*$ uses the same notation as the pullback functors that are part of the structure of $\cat C$). Then the general question of descent is whether $f^*$ is an equivalence of categories.
\begin{cor}\lbl{cor:stuff2Bdescended}
	If $f\colon X\morphism Y$ is a faithfully flat quasi-compact morphism of preschemes, then $f^*$ is an equivalence of categories for the following choices of a fibred category $\cat C$ over $\cat{PSch}/S$.
	\begin{alphanumerate}
		\item $\cat C_T=\left\{\text{affine morphisms }V\morphism T\right\}$ for all $S$-preschemes $T$ and the pullback functors are defined by $g^*V=V\times_TT'$.
		\item[\itememph{a'}] $\cat C_T=\left\{\text{quasi-coherent }\Oo_T\text{-algebras}\right\}$, and $g^*$ is the usual pullback for sheaves of modules.
		\item $\cat C_T=\left\{\text{closed subpreschemes }V\subseteq T\right\}$, regarded as a full subcategory of \itememph{a}.
		\item Here we put
		\begin{align*}
			\cat C_T=\left\{(P,\Ll)\st\begin{tabular}{c}
				$\pi\colon P\morphism T$ is a proper $T$-scheme, $\Ll$ is an ample line\\
				bundle on $P$ defining a closed embedding $P\monomorphism \IP(\pi_*\Ll)$
			\end{tabular}\right\}\;,
		\end{align*}
		and morphisms $(P,\Ll)\morphism (P',\Ll')$ in $\cat C_T$ are given by pairs $(h,\lambda)$ such that $h\colon P\morphism P'$ is a morphism over $T$ and $\lambda\colon h^*\Ll'\isomorphism \Ll$.\footnote{This probably needs \emph{much} more explanation. By $\IP(\pi_*\Ll)$ we denote $\PROJ_T\left(\Sym(\pi_*\Ll)\right)$, where $\Sym(\pi_*\Ll)$ is the symmetric algebra as in \cite[p.~67]{alggeo2}. Then the canonical morphism $\pi^*\pi_*\Ll\morphism\Ll$ defines a morphism $\pi^*\Sym(\pi_*\Ll)\morphism \bigoplus_{n\geq 0}\Ll^{\otimes n}$, which again leads to a morphism $P\morphism\IP(\pi_*\Ll)$ via \cite[Corollary~2.3.1]{alggeo2}}
		\item[\itememph{c'}] We may also use
		\begin{align*}
			\cat C_T=\left\{\begin{tabular}{c}
				quasi-coherent $\IN$-graded $\Oo_T$-algebras $\Rr$ of lo-\\
				cally finite type that are generated by $\Rr_0$ and $\Rr_1$
			\end{tabular}\right\}\;,
		\end{align*}
		and $g^*$ is the usual pullback of sheaves of modules.
		\item[\itememph{c''}] $\cat C_T=\left\{(P,\Ll)\text{ as in \itememph{c}, but }\Ll\text{ is only assumed ample}\right\}$.
	\end{alphanumerate}
\end{cor}
\begin{proof}[Not nearly a proof]
	The cases \itememph{a'} and \itememph{c'} can be reduced to descent of quasi-coherent $\Oo_X$-modules if one proves that the special properties of being an $\Oo_X$-algebras resp.\ an $\IN$-graded one are preserved under descent. Since \itememph{a} and \itememph{a'} are pretty much equivalent, this also shows \itememph{a}. Similarly, it should be possible to reduce \itememph{b} to \itememph{a} as well as \itememph{c} and \itememph{c''} to \itememph{c'}, but we omit how to do this.
\end{proof}
Let us introduce the shortcut \emph{fpqc} for ``faithfully flat and quasi-compact''. In case you wondered, \emph{fpqc} is an abbreviation of the original French phrase ``fidélement plat et quasi-compact''. Similarly, \emph{fppf} denotes ``faithfully flat and of finite presentation (or ``fidélement plat de présentation finie'' in French).
\begin{cor}\lbl{cor:fpqcSheaf}
	Let $f\colon X\morphism Y$ be an fpqc morphism of $S$-preschemes and let $T$ be another $S$-prescheme. Then there is a bijection
	\begin{align*}
		\Hom_{\cat{PSch}/S}(Y,T)&\isomorphism \left\{h\in\Hom_{\cat{PSch}/S}(X,T)\st h p_1=h p_2\text{ in }\Hom_{\cat{PSch}/S}(X\times_YX,T)\right\}\\
		g&\longmapsto h=gf\;.
	\end{align*}
\end{cor}
\begin{proof}[Sketch of a proof]
	On topological level this follows from the fact that $Y$ carries the quotient topology obtained from $X$ (by Proposition~\reff{prop:toplogicalQuotient}). So we only need to show that this bijection is fine with topological components. 
	
	Put $p=f p_1=f p_2$. Let $R$ be the right-adjoint functor from the proof of Theorem~\reff{thm:faithfullyFlatDescent}. Then we proved that there is an isomorphism $\Oo_Y\isomorphism Rf^*\Oo_Y\cong R\Oo_X$. Unraveling what this actually means, we obtain for all open subsets $U\subseteq Y$ an isomorphism
	\begin{align}\lbl{eq:OYcongROX}
		\begin{split}
			\Oo_Y(U)&\isomorphism \left\{\phi\in\Oo_X\big(f^{-1}(U)\big)\st  p_1^*\phi= p_2^*\phi\text{ in }\Oo_{X\times_YX}\big(p^{-1}(U)\big)\right\}\\
			\lambda&\longmapsto \phi=f^*\lambda
		\end{split}		
	\end{align}
	(note that $f^*$, $ p_1^*$, and $ p_2^*$ here denote the \emph{algebraic components} of the respective morphism and \emph{not} some sheaf pullback thingy!). As $f^*\colon\Oo_Y\morphism f_*\Oo_X$ therefore is a monomorphism, the algebraic component $g^*$ of a morphism $g\colon Y\morphism T$ is uniquely determined by $h=gf$. This shows injectivity of the bijection-in-question.
	
	But also surjectivity is easy to see from \eqreff{eq:OYcongROX}. Indeed, suppose $h\colon X\morphism T$ has the property that $h p_1=h p_2$. If $W\subseteq T$ is open and $\vartheta\in\Oo_T(W)$, then $h^*\vartheta$ clearly satisfies 
	\begin{align*}
		 p_1^*h^*\vartheta=(h p_1)^*\vartheta=(h p_2)^*\vartheta= p_2^*h^*\vartheta\;.
	\end{align*}
	That is, $h^*\vartheta$ is contained in the right-hand side of \eqreff{eq:OYcongROX}. It requires some more (trivial, technical) checks, but this basically shows that a $g^*\colon \Oo_T\morphism g_*\Oo_Y$ can be reconstructed from $h^*$.
\end{proof}
Corollary~\reff{cor:fpqcSheaf} can be viewed as a version of the \emph{sheaf axiom}. Indeed, in the special case $X=\coprod_{i\in I}U_i$ where $Y=\bigcup_{i\in I} U_i$ is an open cover, we obtain
\begin{align*}
	X\times_YX=\bigg(\coprod_{i\in I}U_i\bigg)\times_Y\bigg(\coprod_{j\in I}U_j\bigg)\cong\coprod_{(i,j)\in I^2}U_{i,j}\;,
\end{align*}
and the property that the ``pullbacks along $ p_1$ and $ p_2$'' of some object coincide becomes the usual intersection property. This finally brings us to \emph{Grothendieck topologies}.
\begin{defi}
	Let $\cat C$ be any category and let $X\in\Ob(\cat C)$ be an object. A \defemph{sieve} over $X$ is a class $\Ss$ of morphisms $U\morphism X$, such that $(V\morphism U\morphism X)\in \Ss$ whenever $(U\morphism X)\in \Ss$ and $V\morphism U$ is an arbitrary morphism in $\cat C$.
\end{defi}
\begin{defi}\lbl{def:GrothendieckTopology}
	A \defemph{Grothendieck topology} $J$ on a category $\cat C$ is given by specifying a class $J_X$ of sieves over $X$ (the so-called \defemph{covering sieves}) for every object $X\in\Ob(\cat C)$, such that the following conditions hold.
	\begin{alphanumerate}
		\item If $\Ss\in J_X$ and $\Ss'\supseteq \Ss$, then $\Ss'\in J_X$.
		\item If $\Ss\in J_X$ and $\upsilon\colon Y\morphism X$ is any morphism, then
		\begin{align*}
			\upsilon^*\Ss\coloneqq\left\{\iota\colon U\morphism Y\st \upsilon\iota\in\Ss\right\}
		\end{align*}
		is an element of $ J_Y$.
		\item The set (or class) of all morphisms $U\morphism X$ (the \emph{all-sieve}) is an element of $ J_X$.
		\item[\itememph{c'}] We have $ J_X\neq\emptyset$.
		\item Let $\Ss\in J_X$ and $\Ss'$ be another sieve over $X$ such that for all $(\iota\colon U\morphism X)\in\Ss$ it is true that $\iota^*\Ss'\in J_U$. Then $\Ss'\in J_X$.
	\end{alphanumerate}
	A category $\cat C$ together with a Grothendieck topology $J$ on it is called a \defemph{site}.
\end{defi}
\begin{rem}
	These axioms are redundant. We have \itememph{c} $\Rightarrow$ \itememph{c'} assuming \itememph{a} holds. Also \itememph{d} and \itememph{c} imply \itememph{a}. Hence \itememph{a}, \itememph{b}, \itememph{c'}, \itememph{d}, or \itememph{b}, \itememph{c}, \itememph{d} would suffice.
\end{rem}
\begin{rem}\lbl{rem:coveringFamily}
	Definition~\reff{def:GrothendieckTopology} is one of two common ways to define Grothendieck topologies. The other approach via \emph{covering families} (see e.g.\ \cite[\stackstag{00VH}]{stacks-project}) retains more of the feeling of an ordinary topology and is generally more intuitive (which is why most people use this today), but has the disadvantage that different choices of covering families may lead to the same topology (which is why Franke doesn't).
	
	If we have a Grothendieck topology defined by covering sieves, we can get one via covering families out of it as follows: We say a sieve $\Ss$ over $X$ is \emph{generated} by a class $\left\{U_i\morphism X\right\}_{i\in I}$ if every $(U\morphism X)\in \Ss$ factors over some $U_i$. Then we can take all generating classes of covering sieves to be our covering families.
\end{rem}
Now we're going to define some of the most famous (and most useful) Grothendieck topologies in Algebraic Geometry!
\begin{example}
	\begin{alphanumerate}
		\item \lbl{ex:bigZariski}Let $\cat C=\cat{PSch}/S$ be the category of all $S$-preschemes. For every $S$-prescheme $X$, let a sieve $\Ss$ over $X$ be a covering sieve iff there exists an open cover $X=\bigcup_{i\in I}U_i$ such that every $U_i\monomorphism X$ lies in $\Ss$. Then this defines a Grothendieck topology on $\cat C=\cat{PSch}/S$, and the resulting site is called the \emph{big Zariski site} $(\cat{PSch}/S)_\Zar$.
		\item Same as \itememph{a}, but $S$ is assumed locally noetherian and we restrict $\cat C$ to the category of locally noetherian $S$-preschemes.
	\end{alphanumerate}
\end{example}
\begin{example}\lbl{ex:bigfpqc}
	Let $\cat C=\cat{PSch}/S$ and call a sieve $\Ss$ over an $S$-prescheme $X$ a covering sieve iff the following equivalent conditions hold.
	\begin{alphanumerate}
		\item It is possible to cover $X$ by open subsets $X=\bigcup_{i\in I}U_i$, for each of which there are finitely many flat quasi-compact morphisms $V_{i,j}\morphism U_i$, $j=1,\ldots,n_i$ such that $\coprod_{j=1}^{n_i}V_{i,j}\morphism U_i$ is fpqc and $(V_{i,j}\morphism U_i\monomorphism X)\in\Ss$.
		\item Same as \itememph{a}, but the $V_{i,j}$ and $U_i$ are all affine.
	\end{alphanumerate}
	This gives the \emph{(big) fpqc site} $(\cat{PSch}/S)_\fpqc$. In contrast to Example~\reff{ex:bigZariski}\itememph{b}, it is usually \emph{not} possible to restrict this to the category of locally noetherian $S$-preschemes (we'll see below where precisely this fails).
\end{example}
\begin{example}\lbl{ex:bigfppf}
	Let $\cat C=\cat{PSch}/S$ and define covering sieves $\Ss$ over an $S$-prescheme $X$ by the following equivalent conditions.
	\begin{alphanumerate}
		\item As in Example~\reff{ex:bigfpqc}\itememph{a}, but the $V_{i,j}\morphism U_i$ are flat and of finite presentation and $\coprod_{j=1}^{n_i}V_{i,j}\morphism U_i$ is fppf.
		\item As in \itememph{a}, but all $V_{i,j}$ and $U_i$ are affine.
		\item As in \itememph{b}, but the $V_{i,j}\morphism U_i$ are quasi-finite in addition to being affine.
		\item It is possible to cover $X$ by open subsets $X=\bigcup_{i\in I}U_i$ for which there are fppf morphisms $V_i\morphism U_i$ such that $(V_i\morphism U_i)\in \Ss$. That is, the condition is the same as in \itememph{a}, but $n_i=1$ for all $i$.
	\end{alphanumerate}
	The resulting site is called the \emph{(big) fppf site} $(\cat{PSch}/S)_\fppf$. As in Example~\reff{ex:bigZariski}\itememph{b}, it is actually possible to restrict this to locally noetherian $S$-preschemes if $S$ is locally noetherian itself.
\end{example}
\begin{proof}[Sketch of a proof of Example~\reff{ex:bigfpqc} and \reff{ex:bigfppf}]
	Before we show that the fpqc and fppf topologies are indeed Grothendieck topologies, let's discuss why the involved conditions are indeed equivalent. It's easy to see that \itememph{a} $\Leftrightarrow$ \itememph{b} holds both in Example~\reff{ex:bigfpqc} and Example~\reff{ex:bigfppf} (basically this is just the observation that quasi-compact preschemes have finite affine open covers).
	
	So the interesting ones are Example~\reff{ex:bigfppf}\itememph{c} and \itememph{d}. The proof of \itememph{c} $\Leftrightarrow$ \itememph{a} uses Cohen--Macaulay properties and can be found in \cite[\stackstag{056X}]{stacks-project}. For \itememph{d} $\Leftrightarrow$ \itememph{a}, the crucial thing to notice is that every flat morphism of locally finite presentation is an open map\footnote{Indeed, in the case of locally noetherian preschemes we proved this in Corollary~\reff{cor:flatMorphismOpen}, and immediately after that we sketched the proof of the general case in Remark~\reff{rem:nonNoetherianChevalley}.}, so that for every $V_{i,j}$ we may replace $U_i$ by the image $U_{i,j}$ of $V_{i,j}\morphism U_i$. Then $X=\bigcup_{i,j}U_{i,j}$ is an open cover which satisfies condition \itememph{c}.
	
	Now let's prove that the fpqc and fppf topologies satisfy the Grothendieck topology axioms. The fact that the all-sieve is covering is trivial both for fpqc and fppf, as is the fact that any sieve containing a covering sieve is covering itself.
	
	Let's check the condition from Definition~\reff{def:GrothendieckTopology}\itememph{d}. Let $\Ss\in J_X$ and choose $U_i$ and $V_{i,j}$ as in Example~\reff{ex:bigfpqc}\itememph{b} (for fpqc) or Example~\reff{ex:bigfppf}\itememph{b} (for fppf), so that the $U_i$ and the $V_{i,j}$ are affine. Now let $\Ss'$ be another sieve over $X$ satisfying the condition from Definition~\reff{def:GrothendieckTopology}\itememph{d}. This means that for all $(i,j)$ there's an indexing set $K_{i,j}$ and an open cover $V_{i,j}=\bigcup_{k\in K_{i,j}}U'_{i,j,k}$, such that for all $(i,j,k)$ there is a finite number of flat and quasi-compact/finitely presented morphisms $V'_{i,j,k,\ell}\morphism U'_{i,j,k}$, where $\ell=1,\ldots,n'_{i,j,k}$, such that
	\begin{align*}
		\coprod_{\ell=1}^{n'_{i,j,k}}V'_{i,j,k,\ell}\morphism U'_{i,j,k}
	\end{align*}
	is fpqc/fppf and we have $(V'_{i,j,k,\ell}\morphism V_{i,j}\morphism X)\in \Ss$ for all $(i,j,k,\ell)$ (this is horribly confusing, but all we actually did was unraveling definitions). Since the $V_{i,j}$ are (affine and thus) quasi-compact, we may assume that the $K_{i,j}$ are finite indexing sets, and also we may choose the $U'_{i,j,k}$ to be affine. Then each $V'_{i,j,k,\ell}\morphism U_i$ is flat and quasi-compact/finitely presented since it factors as the composition $V'_{i,j,k,\ell}\morphism U'_{i,j,k}\monomorphism V_{i,j}\morphism U_i$ in which the outer arrows are flat and quasi-compact/finitely presented and the middle arrow is an open immersion of affine schemes (hence flat, quasi-compact, and even finitely presented -- that's why we need $V_{i,j}$ affine and not only quasi-compact as in the lecture). For similar reasons,
	\begin{align*}
		\coprod_{j=1}^{n_i}\coprod_{k\in K_{i,j}}\coprod_{\ell=1}^{n'_{i,j,k}}V'_{i,j,k,\ell}\morphism U_i
	\end{align*}
	is fpqc/fppf. Also each $V'_{i,j,k,\ell}\morphism X$ is contained in $\Ss$ (as mentioned above), which shows that $\Ss'$ is an fpqc-/fppf-covering sieve as well.
	
	Finally, let's check the pullback condition, i.e., Definition~\reff{def:GrothendieckTopology}\itememph{b}. Let $\Ss\in J_X$ be an fpqc- or fppf-covering sieve and $\upsilon\colon Y\morphism X$ any morphism. We need to show that $v^*\Ss\in J_Y$. Let $U_i$ and $V_{i,j}$ be as in Example~\reff{ex:bigfpqc}\itememph{a} (in the case of fpqc) or Example~\reff{ex:bigfppf}\itememph{a} (in the case of fppf). Put $\snake{U}_i=\upsilon^{-1}(U_i)$ and $\snake{V}_{i,j}=V_{i,j}\times_{U_i}\snake{U}_{i,j}$. Then $\snake{V}_{i,j}\morphism\snake{U}_i$ is flat and quasi-compact or finitely presented, since these properties are preserved under base change. For the same reason the canonical morphims
	\begin{align*}
		\coprod_{j=1}^{n_i}\snake{V}_{i,j}\cong \bigg(\coprod_{j=1}^{n_i}V_{i,j}\bigg)\times_{U_i}\snake{U}_i\morphism \snake{U}_i
	\end{align*}
	are fpqc/fppf. Also $(\snake{V}_{i,j}\morphism \snake{U}_i)\in\upsilon^*\Ss$ since $\snake{V}_{i,j}\morphism U_i$ factors over $V_{i,j}$ (by definition of the fibre product), hence it is an element of $\Ss$. Therefore, since $Y=\bigcup_{i\in I}\snake{U}_i$ is an open cover, we obtain $\upsilon^*\Ss\in J_Y$.
\end{proof}
\begin{rem}
	\defemph{Caveat!} The pullback condition is where the trouble starts when one tries to equip the category of locally noetherian $S$-preschemes with the fpqc topology. The problem is that (local) noetherianness is usually not inherited by a fibre product $X\times_SY$, even when both factors and the base are (locally) noetherian. For example, $\IC\otimes_\IQ\IC$ is non-noetherian since the kernel $I$ of the multiplication map $\IC\otimes_\IQ\IC\morphism\IC$ has $\dim_\IC I/I^2=\dim_\IC\Omega_{\IC/\IQ}$ (by Lemma~\reff{lem:smartKählerDifferentials}) and the latter has the cardinality of the continuum.
\end{rem}
\begin{example}
	Let $p\in\IZ$ be a prime. Then the sieve generated by $\Spec \IZ_p\morphism\Spec\IZ$ and $\Spec\IZ[p^{-1}]\morphism\Spec \IZ$ is an fpqc-covering of $\Spec \IZ$.
\end{example}
Grothendieck topologies are designed to make sheaf theory work in a more general context. In the rest of the section we'll introduce sheaves on arbitrary sites and prove some properties of fpqc-/fppf-sheaves.
\begin{defi}
	Let $\cat C$ be a category and $\cat A$ whatever target category we want to have (like sets, (abelian) groups, rings, \ldots).
	\begin{alphanumerate}
		\item An \defemph{$\cat A$-valued presheaf on $\cat C$} is a contravariant functor $\Ff\colon \cat C^\op\morphism\cat A$. 
		\item If $\cat C$ comes equipped with a Grothendieck topology $J$, then a presheaf $\Ff$ is called a \defemph{sheaf} if for all objects $X$ and all $\Ss\in J_X$ the following condition holds: The morphisms $\Ff(f)\colon\Ff(X)\morphism\Ff(Y)$ for all $(f\colon Y\morphism X)\in\Ss$ induce an isomorphism
		\begin{align}\lbl{eq:sheafAxiom}
		\Ff(X)\isomorphism \limit[f\in\Ss]\Ff(Y)
		\end{align}%=\left\{(\phi_f)_{f\in\Ss}\in\prod_{f\in\Ss}\st \phi_{fg}=\Ff(g)\phi_f\text{ for all morphisms }g\colon Z\to Y\right\}
		(in particular, the limit on the right-hand side is supposed to exist in $\cat A$).
		
	\end{alphanumerate}
	\end{defi}
\begin{rem}
	\begin{alphanumerate}
		\item \lbl{rem:sheafAxiom}If $\cat A$ is one of the usual target categories (like sets, (abelian) groups, or rings), the limit on the right-hand side of \eqreff{eq:sheafAxiom} can be written as
		\begin{align*}
			\limit[f\in\Ss]\Ff(Y)=\left\{(\phi_f)_{f\in\Ss}\in\prod_{f\in\Ss}\Ff(Y)\st
			\begin{tabular}{c}
				$\phi_{fg}=\Ff(g)\phi_f$ holds for\\
				all morphisms $g\colon Z\to Y$
			\end{tabular} \right\}\;,
		\end{align*}
		and $\Ff(X)$ maps into it via $\phi\mapsto (\Ff(f)\phi)_{f\in\Ss}$ for all $\phi\in\Ff(X)$ (indeed, this is how you form limits in these target categories $\cat A$).
		\item If $\Ss$ is generated by a covering family $\{U_i\morphism X\}_{i\in I}$ (i.e., we use the covering family formalism as in Remark~\reff{rem:coveringFamily}), then every $\Ff(f)\colon \Ff(X)\morphism \Ff(Y)$ factors over some $\Ff(U_i)$. It's easy to check that the limit taken over the diagram of all $\Ff(Y)$ for $(f\colon Y\morphism X)\in\Ss$ equals the limit taken over the subdiagram consisting only of $\Ff(U_i)\morphism\Ff(U_i\times_XU_j)$ and $\Ff(U_j)\morphism\Ff(U_i\times_XU_j)$ for all $(i,j)\in I^2$. The latter limit is clearly given by
		\begin{align*}
			\Eq\bigg(\prod_{i\in I}\Ff(U_i)\doublemorphism[ p_1^*][ p_2^*]\prod_{(i,j)\in I^2}\Ff(U_i\times_XU_j)\bigg)
		\end{align*}
		(assuming products and equalizers exist, which they usually do for reasonable target categories $\cat A$, such as sets, groups, \ldots), where $ p_1^*$ and $ p_2^*$ correspond to the two classes of morphisms $ p_1\colon U_i\times_XU_j\morphism U_ i$ and $ p_2\colon U_i\times_XU_j\morphism U_j$.
		
		Hence another way of stating the sheaf axiom using covering families is that for every covering family $\{U_i\morphism X\}_{i\in I}$ the diagram
		\begin{align*}
			\Ff(X)\morphism\prod_{i\in I}\Ff(U_i)\doublemorphism[ p_1^*][ p_2^*]\prod_{(i,j)\in I^2}\Ff(U_i\times_XU_j)
		\end{align*}
		is an equalizer diagram. Now this \emph{really} looks like the sheaf axiom we used to know.
	\end{alphanumerate}
\end{rem}
\begin{rem}
	One can define the \emph{sheafification} of a presheaf $\Ff$ as follows. Put
	\begin{align}\lbl{eq:+construction}
		\Ff^+(X)=\colimit[\Ss\in J_X]\limit[f\in S]\Ff(Y)\;,
	\end{align}
	where the colimit on the right-hand side is taken over all covering sieves ordered by \emph{reverse inclusion}. That is, if $\Ss\subseteq \Ss'$ are covering sieves over $X$, then there are canonical projections
	\begin{align*}
		\limit[f'\in \Ss']\Ff(Y')\morphism \Ff(Y)\quad\text{for all}\quad(f\colon Y\to X)\in \Ss\;,
	\end{align*}
	since every such $f$ is an element of $\Ss'$ as well (using $\Ss\subseteq \Ss'$). These projections induce a canonical morphism 
	\begin{align*}
		\limit[f'\in\Ss']\Ff(Y')\morphism\limit[f\in \Ss]\Ff(Y)\;,
	\end{align*}
	and the colimit in \eqreff{eq:+construction} is taken over the ensuing diagram. It can be shown that $(-)^+\colon \cat{PSh}(\cat C)\morphism\cat{PSh}(\cat C)$ is a functor from the category of presheaves on $\cat C$ to itself, and moreover that applying it twice always gives a sheaf $\Ff^\sh=\Ff^{++}$.\footnote{In particular, $\Ff^+$ is \emph{not yet} the sheafification of $\Ff$, which  I believe -- correct me if I'm wrong -- Professor Franke falsely claimed in the heat of the lecture.} Then $(-)^\sh$ is the required sheafification functor and it has all reasonable properties we could hope for (e.g.\ it is right-adjoint to the forgetful functor $\cat {Sh}(\cat C)\morphism\cat{PSh}(C)$). Relevant proofs can be found in \cite[\stackstag{00W1}]{stacks-project}.
	
	Also note that \eqreff{eq:+construction} assumes that our target category has the required (co)limits, as well as that one is able to deal with the involved set-theoretic difficulties (in particular when fpqc is used), such as taking (co)limits over potentially proper classes.\footnote{Professor Franke remarked that \eqreff{eq:+construction} probably already contains half a dozen violations of von Neumann--Bernays--Gödel set theory.}
\end{rem}
\begin{example}\lbl{ex:ZariskiSheaf}
	$\Ff$ is a sheaf on the big Zariski site site iff for every %(locally noetherian)
	$S$-prescheme $X$ the map $U\mapsto \Ff(U)$ on Zariski-open subsets of $X$ defines a Zariski sheaf in the ordinary sense. This is still true if we restrict to locally noetherian $S$-preschemes and $S$ is locally noetherian itself.
\end{example}
\begin{prop}\lbl{prop:fpqcSheaf}
	A functor $\Ff$ on the category of $S$-preschemes $X$ is a fpqc sheaf iff it is a Zariski sheaf in the sense of Example~\reff{ex:ZariskiSheaf} and for every fpqc morphism $f\colon X\morphism Y$ over $S$, the map
	\begin{align*}
		\Ff(Y)\morphism\left\{f\in\Ff(X)\st \Ff( p_1)f=\Ff( p_2)f\text{ in }\Ff(X\times_YX)\right\}
	\end{align*}
	is bijective. In particular, if $T$ is an $S$-prescheme, then $\Hom_{\cat{PSch}/S}(-,T)$ is an fpqc sheaf on $\cat{PSch}/S$.
	
	The same is true if we equip $\cat{PSch}/S$ with the fppf topology instead (and consider $f\colon X\morphism Y$ which are fppf). In this case, we may moreover restrict ourselves to locally noetherian $S$-preschemes if $S$ itself is locally noetherian.
\end{prop}
\begin{proof}
	It is clear that every fpqc sheaf has the asserted properties, so we only need to prove the converse. Let $\Ff$ satisfy the above properties. To check the fpqc sheaf axiom for $\Ff$, we may use the version with covering families from Remark~\reff{rem:sheafAxiom}\itememph{b}. Let $\left\{V_{i,j}\morphism U_i\monomorphism X\st i\in I, j=1,\ldots, n_i\right\}$ be such a covering family. Then what we need to show is that any element
	\begin{align}\lbl{eq:phiEq1}
		\phi=(\phi_{i,j})\in\Eq\bigg(\prod_{i,j}\Ff(V_{i,j})\doublemorphism[ p_1^*][ p_2^*]\prod_{i,j}\prod_{k,\ell}V_{i,j}\times_XV_{k,\ell}\bigg)
	\end{align}
	has a unique preimage in $\Ff(X)$. Since $\coprod_{j=1}^{n_i}V_{i,j}\morphism U_i$ is fpqc, the assumption shows that 
	\begin{align}\lbl{eq:phiEq2}
		\Ff(U_i)\morphism\Ff\bigg(\coprod_{j=1}^{n_i}V_{i,j}\bigg)\doublemorphism[ p_1^*][ p_2^*]\Ff\bigg(\coprod_{j=1}^{n_i}\coprod_{\ell=1}^{n_i}V_{i,j}\times_{U_i}V_{i,\ell}\bigg)
	\end{align}
	is an equalizer diagram (where $ p_1^*$ and $ p_2^*$ correspond to $\Ff( p_1)$ and $\Ff( p_2)$ in this case). But $\Ff$ is also a Zariski sheaf, so the ordinary sheaf axiom yields $\Ff\left(\coprod_{j=1}^{n_i}V_{i,j}\right)\cong\prod_{j=1}^{n_i}\Ff(V_{i,j})$, and likewise for the term on the right of the above diagram. By \eqreff{eq:phiEq1} we see that for every fixed $i$ the sequence $(\phi_{i,j})_{j=1,\ldots,n_i}$ is an element of the equalizer in equation \eqreff{eq:phiEq2}. That is, the $(\phi_{i,j})_{j=1,\ldots,n_i}$ correspond to a unique element $\phi_i\in \Ff(U_i)$.
	
	If we can show that the $(\phi_i)_{i\in I}$ satisfy the ordinary Zariski sheaf axiom, then we're done since $\Ff$ is a Zariski sheaf. To show that $\phi_i$ and $\phi_k$ have the same image in $\Ff(U_i\cap U_k)\cong \Ff(U_i\times_XU_k)$ for all $i,k\in I$, it's sufficient to note that the images of $\phi_i$ and $\phi_k$ in $\Ff(V_{i,j}\times_XV_{k,\ell})$ coincide for all $(j,\ell)$ (and indeed they do because of \eqreff{eq:phiEq1}), because $\coprod_{j,\ell}V_{i,j}\times_XV_{k,\ell}\morphism U_i\times_XU_k$ is fpqc, so the fpqc property of $\Ff$ can be applied.
	
	If $T$ is an $S$-prescheme, then $\Hom_{\cat{PSch}/S}(-,T)$ is clearly a Zariski sheaf, and Corollary~\reff{cor:fpqcSheaf} shows that the fpqc property is fulfilled, hence $\Hom_{\cat{PSch}/S}(-,T)$ is an fpqc sheaf as well by what we just proved. Also the fppf case follows by the exact same arguments.
\end{proof}
\begin{example}\lbl{ex:sheafifyingPic}
	Let $f\colon X\morphism S$ be a morphism of preschemes. For any $S$-prescheme $T$ we denote $X_T=X\times_ST$. On the big Zariski site $(\cat{PSch}/S)_\Zar$ we have a presheaf $\Pic_X$ given by $\Pic_X(T)=\Pic(X_T)$. Let $\Pic_{X/S}$ denote its sheafification with respect to the Zariski topology. We wish to describe $\Pic_{X/S}$, provided $f$ is ``sufficiently nice''.
	
	To this end, assume that $f$ is fppf, has a section $\sigma\colon S\morphism X$, and the canonical morphism $\Oo_T\morphism f_{T,*}\Oo_{X_T}$ is an isomorphism for all $S$-preschemes $T$.\footnote{The last condition was missing in the lecture, but it is needed in the proof of \cite[\stackstag{0B9N}]{stacks-project}.} We claim that
	\begin{align}\lbl{eq:PicX/SInNiceCases}
		\Pic_{X/S}(T)=\ker\big(\sigma_T^*\colon \Pic(X_T)\morphism\Pic(T)\big)
	\end{align}
	in this case. To prove this, let temporarily $K(T)$ denote the right-hand side of \eqreff{eq:PicX/SInNiceCases}. Note that $\sigma_T^*$ is a section of $f_T^*\colon \Pic(T)\morphism\Pic(X_T)$ making $\Pic(T)$ into a direct summand of $\Pic(X_T)$, so $K(T)\cong \Pic(X_T)/f^*\Pic(T)=\Pic(X_T/T)$ is what we would $\Pic_{X/S}(T)$ expect to be after the discussion starting on page~\pageref{par:Jacobian}. 
	
	Also, pretending for the moment we didn't know that $K$ is supposed to be a sheaf, then still $K$ and $\Pic_X$ necessarily have the same sheafification. Indeed, sheafification is exact (this is still true for sites, cf.\ \cite[\stackstag{00WJ}]{stacks-project}), hence the sheafification of $K$ will be the kernel of the morphism from $\Pic_{X/S}$ to the sheafification of $\Pic$. But $\Pic$ has trivial sheafification (because line bundles are locally trivial in the Zariski topology), hence said kernel is all of $\Pic_{X/S}$.
	
	Therefore, if we show that $K$ is a Zariski sheaf, then it must be equal to $\Pic_{X/S}$. To show the Zariski sheaf axiom, we follow the proof from \cite[\stackstag{0B9N}]{stacks-project} (note that the Stacks Project guys actually show the fppf-sheaf axiom). Let $T=\bigcup_{i\in I}U_i$ be an open cover. We use the usual convention $U_{i,j}\eqqcolon U_i\cap U_j$. Let us also introduce the shortcut $X_i=X_{U_i}$ (and likewise for more indices) to save us some notational pain. Now suppose we're given $\Ll_i\in K(U_i)$ for $i\in I$ such that $\Ll_i$ and $\Ll_j$ have the same image in $K(U_{i,j})$. Hence we may fix isomorphisms $\phi_{i,j}\colon \Ll_i|_{X_{i,j}}\isomorphism\Ll_j|_{X_{i,j}}$, and also isomorphisms $\alpha_i\colon \Oo_{U_i}\isomorphism \sigma_{U_i}^*\Ll_i$ (by definition of $K(U_i)$). Then the composition
	\begin{align*}
		(\alpha_j|_{U_{i,j}})^{-1}\circ\sigma_{U_{i,j}}^*(\phi_{i,j})\circ\alpha_i|_{U_{i,j}}\colon \Oo_{U_{i,j}}\isomorphism \Oo_{U_{i,j}}
	\end{align*}
	is an isomorphism as well, hence given by multiplication with some invertible scalar $u_{i,j}$. Rescaling $\phi_{i,j}$ by $u_{i,j}^{-1}$ (or rather its image under $\Oo_{U_{i,j}}\morphism f_{U_{i,j},*}\Oo_{X_{i,j}}$), we may assume that $u_{i,j}=1$ (rescaling is fine because there's no cocycle condition yet that we could destroy).
	
	Our goal now is to show the cocycle condition for the $\phi_{i,j}$, i.e., that
	\begin{diagram}[baseline=0.75cm-0.5ex][\lbl{diag:phiijCocycle}]
		\object{0,1.5}{$\Ll_i|_{\smash{X_{i,j,k}}}$}[i];
		\object{0,1.5}{$\phantom{\Ll_i|_{X_{i,j,k}}}$}[ii];
		\object{1.5,0}{$\Ll_k|_{X_{i,j,k}}$}[k];
		\object{3.5,1.5}{$\Ll_j|_{\smash{X_{i,j,k}}}$}[j];
		\scriptsize
		\arrow{i}{j}[above][$\phi_{i,j}|_{X_{i,j,k}}$];
		\arrow{ii}{k}[below left][$\phi_{i,k}|_{X_{i,j,k}}$];
		\arrow{j}{k}[below right][$\phi_{j,k}|_{X_{i,j,k}}$];
	\end{diagram}
	commutes. Since all arrows are isomorphisms, we know at least that \eqreff{diag:phiijCocycle} commutes up to some scalar $u_{i,j,k}$. But if we apply $\sigma_{U_{i,j,k}}^*$ to everything, then the diagram will surely become commutative since all the $u_{i,j}$ are equal to $1$. Hence the image of $u_{i,j,k}$ under the canonical morphism 
	\begin{align*}
		f_*(\sigma^*)\colon f_*\Oo_{X_{i,j,k}}\morphism f_*\sigma_*\Oo_{U_{i,j,k}}
	\end{align*}
	equals $1$ (since this will be the scalar we get after applying $\sigma^*$). Note that I dropped the base change indices $U_{i,j,k}$ of $f_{U_{i,j,k}}$ and $\sigma_{U_{i,j,k}}$ because this is better for my mental health. But the composition
	\begin{align*}
		\Oo_{U_{i,j,k}}\morphism[f^*] f_*\Oo_{X_{i,j,k}}\xrightarrow{f_*(\sigma^*)} f_*\sigma_*\Oo_{U_{i,j,k}}
	\end{align*}
	is an isomorphism since $f_*\sigma_*=(f\sigma)_*=\id_*$, and also the first arrow $f^*$ is an isomorphism by our assumption that $\Oo_T\cong f_{T,*}\Oo_{X_T}$ for every $S$-prescheme $T$. Hence $f_*(\sigma^*)$ is an isomorphism as well. This shows $u_{i,j,k}=1$, so \eqreff{diag:phiijCocycle} commutes indeed. But then the $\Ll_i$ can be glued together to form a line bundle $\Ll$ on $X_T$. We have $\Ll\in K(T)$ since the $\alpha_i$ can be glued together to an isomorphism $\alpha\colon \Oo_T\isomorphism \sigma_T^*\Ll$ (since all the $u_{i,j}$ are equal to $1$).
	
	This shows the surjectivity part of the sheaf axiom, but we still need to argue why $\Ll$ is unique (up to isomorphism). This will be a bit handwavy. Note that since we rescaled the $\phi_{i,j}$ afterwards, their choice wasn't arbitrary (and rescaling was necessary to obtain $\Ll$ with $\sigma_T^*\Ll$ trivial), so the only real choice we had was how to choose the $\alpha_i$. If we replace each $\alpha_i$ by $\alpha_i'=u_i\alpha_i$ for some unit $u_i$, then this leads to rescalation of each $\phi_{i,j}$ by $u_j^{-1}u_i$. Let $\phi_{i,j}'$ be the new gluing maps and $\Ll'$ the new line bundle glued together via the $\phi_{i,j}'$. By \cite[Section~1.7]{alggeo2}, what we need to show is that $\Ll$ and $\Ll'$ have the same image in $\check{H}^1(X,\Oo_X^\times)$. But $[\Ll]$ and $[\Ll']$ differ by the image of $(u_j^{-1}u_i)_{i,j\in I}=\hacek{d}^0(u_i)_{i\in I}$, which is a coboundary, hence they coincide.
	
	This shows that $K$ is a sheaf on the big Zariski site, so $K=\Pic_{X/S}$ as explained above. But we can show even more: $\Pic_{X/S}$ is even an fpqc sheaf! To show this, it suffices (by Proposition~\reff{prop:fpqcSheaf}) to show that for any fpqc morphism $U\morphism T$,
	\begin{align}\lbl{eq:PicfpqcSheaf}
		\Pic_{X/S}(T)\morphism \Pic_{X/S}(U)\doublemorphism[ p_1^*][ p_2^*]\Pic_{X/S}(U\times_TU)
	\end{align}
	is an equalizer diagram. Let $\Ll\in\Pic(X_U)$ be a line bundle such that $ p_1^*\Ll= p_2^*\Ll$ holds in $\Pic(X_{U\times_TU})\cong\Pic(X_U\times_{X_T}X_U)$. Choose an isomorphism $\alpha\colon \Oo_U\isomorphism \sigma_U^*\Ll$ and an isomorphism $\phi\colon  p_1^*\Ll\isomorphism  p_2^*\Ll$. As in the Zariski case, we may rescale $\phi$ and use the $\Oo_T\cong f_{T,*}\Oo_{X_T}$ assumption to show that $(\Ll,\phi)$ satisfies the cocycle condition (i.e., the moderately large diagram from Definition~\reff{def:descentData}). Hence we may descend $\Ll$ to a line bundle $\Ll_0$ on $X_T$ by Theorem~\reff{thm:faithfullyFlatDescent} (and the descent is a line bundle again by Lemma~\reff{lem:descentIsALineBundleAgain}). Also the pullback of $\sigma_T^*\Ll_0$ to $U$ is trivial (since $\Oo_U\cong \sigma_U^*\Ll$), hence $\sigma_T^*\Ll_0$ must already be trivial by the uniqueness part in Theorem~\reff{thm:faithfullyFlatDescent}. This shows that $\Pic_{X/S}(T)$ surjects onto $\Eq( p_1^*, p_2^*)$ in \eqreff{eq:PicfpqcSheaf}. Injectivity can be done as in the Zariski case, so $\Pic_{X/S}$ is indeed an fpqc sheaf.
	
	Now let's drop the condition that $f\colon X\morphism S$ has a section (but keep the others). Note that fppf-locally on $S$ one always has a section (e.g.\ the diagonal $\Delta\colon X\morphism X\times_SX$ after base-change by $f$ itself). Hence in absence of a section of $f$, we can sheafify $\Pic_X$ with respect to the fppf topology (little or no set-theoretic difficulties) to obtain $\Pic_{X/S}$, which is automatically an fpqc sheaf (because this can be checked fppf-locally, where sections exist, so the above discussion does it). This is a way to work around the set-theoretic difficulties that would arise if one had tried to sheafify $\Pic_X$ directly in the fpqc topology.
\end{example}
\begin{example}[I'm not sure what exactly this illustrates]
	For a division algebra $D$ over a field $k$ such that the center of $D$ is $k$, one has a \emph{Brauer--Severi manifold} $X$ with the property that
	\begin{align*}
		\Hom_{\cat{PSch}/k}(\Spec A,X)\cong \left\{k\text{-isomorphisms }D\otimes_kA\isomorphism A^{d\times d}\right\}\;,
	\end{align*}
	where $\dim_kD=d^2$. Then the base change $X_{\ov{k}}$ to an algebraic closure of $k$ is isomorphic to $\IP_{\ov{k}}^{d-1}$, and for any algebraic extension $\ell/k$ one has a line bundle on $X_\ell$ whose pullback to $X_{\ov{k}}$ is $\Oo(1)$ iff $D\otimes_k\ell\cong\ell^{d\times d}$.
\end{example}
\section{Étale and smooth morphisms}
\begin{prop}\lbl{prop:unramified}
	Let $f\colon X\morphism S$ be a morphism of locally finite type between locally noetherian preschemes. Then the following conditions are equivalent.
	\begin{alphanumerate}
		\item We have $\Omega_{X/S}=0$.
		\item The diagonal $\Delta_{X/S}\colon X\morphism X\times_SX$ is an open embedding.
	\end{alphanumerate}
\end{prop}
To prove this, we first recall the \emph{smart} contruction of Kähler differentials (in contrast to the \emph{brute force} approach as in \cite[Proposition~1.4.1]{alg2}).
\begin{lem}\lbl{lem:smartKählerDifferentials}
	Let $B$ be an algebra over $A$. Then $\Omega_{B/A}\cong I/I^2$, where $I$ is the kernel of the multiplication map $B\otimes_AB\morphism B$.
\end{lem}
\begin{proof}[Sketch of a proof]
	In fact, for any $B$-module $M$ we obtain a canonical bijection
	\begin{align*}
		\Hom_B(I/I^2,M)&\isomorphism \Der_A(B,M)\\
		\phi(b_1\otimes b_2)=b_1d(b_2)&\longmapsfrom d\\
		\phi&\longmapsto d(b)=\phi(1\otimes b-b\otimes 1)
	\end{align*}
	(lots of things are to check here actually, but we leave it like that since this is also a pretty well-known fact).
\end{proof}
\begin{proof}[Proof of Proposition~\reff{prop:unramified}]
	The diagonal in $X\times_SX$ (but not all of $X\times_SX$ though!) can be covered by open subsets of the form $U\times_WU$, where $U\subseteq X$ and $W\subseteq S$ are affine such that $f(U)\subseteq W$. Since both \itememph{a} and \itememph{b} are local with respect to such a cover, we may assume that $X=\Spec B$ and $S=\Spec A$ are affine, with $B$ of finite type over $A$, for the rest of the proof.
	
	Let's first assume $\Omega_{X/S}=0$. Then, with notation as above, $0=\Omega_{B/A}\cong I/I^2$ implies $I_\pp=0$ for all prime ideals of $B\otimes_AB$ in which $I$ is contained, using the Nakayama lemma. Hence $V(I)$ consists precisely of those prime ideals for which $I_\pp=0$. In particular, we see that $\Delta_{X/S}$ induces isomorphisms on stalks, so it only remains to show that the image of $\Delta_{X/S}$ is open. Since $I$ is finitely generated, we may apply Nakayama-style arguments (like \cite[Lemma~1.5.1]{alg2}) to see that $I_\pp=0$ implies $I_\lambda=0$ for some $\lambda\notin\pp$, hence $V(I)$ is also open. But since $(B\otimes_AB)/I\cong B$, $V(I)$ is the image of $X=\Spec B$ under the diagonal $\Delta_{X/S}$, thus we have finished the proof of \itememph{a} $\Rightarrow$ \itememph{b}.
	
	Conversely, if $\Delta_{X/S}$ is an open immersion in addition to being closed (note that we only have closedness since $X$ and $S$ are affine -- in general, this fails if $f$ isn't separated, cf.\ \cite[Fact~1.5.7]{alggeo1}), then it induces isomorphisms on stalks. Hence $I_\pp=0$ for all prime ideals of $B\otimes_AB$ in the image of $\Delta_{X/S}$ (i.e., in $V(I)$) and $I_\pp=(B\otimes_AB)_\pp$ else. In either case, $I_\pp/I_\pp^2=0$, hence also $I/I^2=0$. This shows \itememph{b} $\Rightarrow$ \itememph{a}. 
\end{proof}
\begin{rem}\lbl{rem:stuffWorkWithoutNoetherian}
	This proof works without any noetherianness or finite presentation assumptions. All we need for our Nakayama arguments is that $I$ is finitely generated, which is always the case when $B$ is of finite type over $A$. Indeed, if $b_1,\ldots,b_n\in B$ are $A$-algebra generators, then it can be shown that $I$ is generated by the $1\otimes b_i-b_i\otimes 1$. In particular, $\Omega_{B/A}\cong I/I^2$ is a finitely generated $B$-module in this case.
\end{rem}
\begin{defi}\lbl{def:etale}
	A morphism $f\colon X\morphism S$ with the equivalent properties of Proposition~\reff{prop:unramified} is called \defemph{unramified}. If $f$ is unramified and flat, it is called \defemph{étale}.
\end{defi}
If you are familiar with Algebraic Number Theory, you probably came across \emph{unramified extensions} already. This is actually a special case of Definition~\reff{def:etale}. The relation between ramification and Kähler differentials for number fields is discussed in \cite[Ch.\:III \S2]{NEUKIRCH} -- from this, it's actually easy to deduce a proof that the number-theoretic notion of unramifiedness is a special case of the geometric one. Another proof is given in the following lemma (which didn't appear in the lecture).
\begin{lem}\lbl{lem:unramified2}
	Let $f\colon X\morphism S$ be a morphism of locally finite type. Let $x\in X$ and $s=f(x)$. Then $\Omega_{X/S,x}=0$ (i.e., $f$ is \defemph{unramified at $\boldsymbol{x}$}) iff $\mm_{S,s}\Oo_{X,x}=\mm_{X,x}$ and $\KK(x)/\KK(s)$ is a finite separable extension.
\end{lem}
\begin{proof}
	We have $\Omega_{X/S,x}\cong\Omega_{\Oo_{X,x}/\Oo_{S,s}}$. Put $k=\KK(s)$ and let $R=\Oo_{X,x}/\mm_{S,s}\Oo_{X,x}$. Then $R$ is local with maximal ideal $\mm=\mm_{X,x}/\mm_{S,s}\Oo_{X,x}$ and we have $\Omega_{R/k}\cong \Omega_{\Oo_{X,x}/\Oo_{S,s}}\otimes_{\Oo_{S,s}}k$ by well-known base change properties of Kähler differentials.
	
	Let's first assume $\Omega_{X/S,x}=0$. By the above, we also have $\Omega_{R/k}=0$. From the conormal sequence
	\begin{align}\lbl{eq:conormalSequence}
		\mm/\mm^2\morphism\Omega_{R/k}\otimes_R\KK(\mm)\morphism\Omega_{\KK(\mm)/k}\morphism 0
	\end{align}
	we see that $\Omega_{\KK(\mm)/k}=0$, hence $\KK(\mm)=\KK(x)$ is an algebraic separable extension by \cite[Corollary~16.17]{eisenbudCommAlg}. That $\KK(x)/k$ is be finite follows basically from $f$ having locally finite type. To be more precise, since $X$ has locally finite type over $S$, so has the fibre $X_s=f^{-1}\{s\}$ over $k$. Choose an affine neighbourhood $\Spec B\subseteq X_s$ of $x$ such that $B$ has finite type over $k$. If $x$ corresponds to $\pp\in\Spec B$, then $\KK(x)$ is the quotient field of $B/\pp$. But $B/\pp$ has finite type over $k$, hence $\KK(x)$ is a finitely generated algebraic field extension of $k$, thus finite.
	
	To see why $\mm_{S,s}\Oo_{X,x}=\mm_{X,x}$, what we need to show is that $\mm=0$. Since $\KK(\mm)$ is separable over $k$, \cite[Lemma~1.6.1]{alg2} shows that the conormal sequence \eqreff{eq:conormalSequence} can be extended by $0$ on the left. Hence $\mm/\mm^2=0$. But $R$ is noetherian -- indeed, the fibre $X_s$ is locally noetherian since it has locally finite type over $k$, and $R\cong \Oo_{X,x}\otimes_{\Oo_{Y,y}}k\cong \Oo_{X_s,x}$ is its local ring at $x$ -- so $\mm=0$ follows from Nakayama.
	
	Conversely, let $\mm_{S,s}\Oo_{X,x}=\mm_{X,x}$ and $\KK(x)/k$ be finite separable. Then $R=\KK(x)$, hence \cite[Corollary~16.17]{eisenbudCommAlg} again implies $0=\Omega_{R/k}=\Omega_{X/S,x}\otimes_{\Oo_{Y,y}}k$. Then also $0=\Omega_{X/S,x}\otimes_{\Oo_{X,x}}\KK(x)$. Since $\Omega_{X/S,x}$ is finitely generated over $\Oo_{X,x}$ (by Remark~\reff{rem:stuffWorkWithoutNoetherian}), we may apply Nakayama's lemma to obtain $\Omega_{X/S,x}=0$.
\end{proof}
\begin{prop}\lbl{prop:finiteEtale}
	A finite morphism $f\colon X\cong \SPEC_S\Bb\morphism S$ with $S$ locally noetherian is étale iff $\Bb$ is a locally free $\Oo_S$-module and the trace pairing
	\begin{align*}
		(a,b)\longmapsto \Tr_{\Bb/\Oo_S}(ab)
	\end{align*}
	defines a non-degenerate self-duality on $\Bb$.
\end{prop}
\begin{proof}
	Since $S$ is locally noetherian and $\Bb$ a coherent $\Oo_S$-module, $\Bb_s$ is free over $\Oo_{S,s}$ iff it is flat, see \cite[Proposition~1.3.1]{homalg} (without noetherianness, we would need $f$ to be of locally finite presentation). Hence $\Bb$ is locally free iff $f$ is flat, and it remains to show that $f$ is unramified iff the trace pairing is non-degenerate.
	
	Take a small affine open $U\subseteq S$, say, $U\cong\Spec A$, such that the restriction of $f$ to $f^{-1}(U)$ has the form $\Spec B\morphism \Spec A$ with $B$ free of rank $n$ over $A$. By definition, the trace pairing is non-degenerate iff the induced morphism $\Tr_{B/A}\colon B\morphism \Hom_A(B,A)$ is an isomorphism. After choosing a basis of $B$, we get a (non-canonical) isomorphism $\Hom_A(B,A)\cong B$. Using the upcoming Lemma~\reff{lem:tensorTrick} (plus some easy arguments), we see that $\Tr_{B/A}\colon B\morphism B$ is an isomorphism iff $\Tr_{B\otimes_A\KK(s)/\KK(s)}\colon B\otimes_A\KK(s)\morphism B\otimes_A\KK(s)$ is an isomorphism for all $s\in U$, i.e., iff $\Tr_{B\otimes_A\KK(s)/\KK(s)}$ induces a non-degenerate pairing.
	
	Put $k=\KK(s)$ and $R=B\otimes_Ak$ for convenience. Since $R$ finite over $k$, it is an Artinian local ring, and as such it is the product $R\cong \prod_iR_{\mm_i}$ of the localizations at its finitely many prime (hence maximal) ideals $\mm_i$ (see \cite[Corollary~2.16]{eisenbudCommAlg}). Note that the $R_{\mm_i}$ correspond to the $\Oo_{X,x}/\mm_{S,s}\Oo_{X,x}$ for $x\in X$ a point over $s$. Hence, by Lemma~\reff{lem:unramified2}, what we need show is that $\Tr_{R/k}$ is non-degenerate iff every $R_{\mm_i}$ is a separable field extension of $k$.
	
	The \emph{if} part is clear, since it is well-known that the trace form is non-degenerate for separable field extensions, so every $\Tr_{R_{\mm_i}/k}$ is degenerate, hence also $\Tr_{R/k}$ as $R=\prod_iR_{\mm_i}$. For the \emph{only if} part, assume that $\Tr_{R/k}$ and thus any $\Tr_{R_{\mm_i}/k}$ is non-degenerate. If $x\in\mm_iR_{\mm_i}$, then $x$ is nilpotent, hence $r\mapsto\Tr_{R_{\mm_i}/k}(rx)$ is identically zero for $r\in R$ since nilpotent maps have trace $0$. Hence $\mm_iR_{\mm_i}$ is in the kernel of $R_{\mm_i}\morphism\Hom_k(R_{\mm_i},k)$, so $\mm_iR_{\mm_i}=0$ and $R_{\mm_i}$ is a finite extension of $k$. Using once again the well-known fact that a finite field extension is separable iff the trace form is non-degenerate, we're done.
\end{proof}
\begin{rem}
	\begin{alphanumerate}
		\item \lbl{rem:etaleTopology}Such \emph{étale coverings} are used to define  $\pi_1^\et(X)$ in \cite{sga1}.
		\item Assuming the $V_{i,j}\morphism U_i$ in the above definition of the fppf topology (Example~\reff{ex:bigfppf}\itememph{a}) to be unramified in addition to the other conditions, one obtains the famous \emph{étale topology}.
	\end{alphanumerate}
\end{rem}
\begin{defi}\lbl{def:smooth}
	Let $f\colon X\morphism S$ be a morphism of locally finite type between locally noetherian preschemes.
	\begin{alphanumerate}
		\item We call $f$ \defemph{smooth at $\boldsymbol{x\in X}$} iff there is an étale morphism $U\morphism \IA_S^n$ for $S$-preschemes for some open neighbourhood $U$ of $x\in X$. If $f$ is smooth at every $x\in X$, then it is just called \defemph{smooth}.
		\item We call $f$ a \defemph{regular embedding} if it is a closed embedding and the corresponding sheaf of ideals can be locally generated by a regular sequence.
		\item We call $f$ a \defemph{local complete intersection} if (locally) one (and thus any) closed embedding of $X$ into a smooth $S$-prescheme is regular.
	\end{alphanumerate}
\end{defi}
\begin{rem}
	Similar to the étale topology, one obtains the \emph{syntomic topology} if one assumes the $V_{i,j}\morphism U_i$ in Example~\reff{ex:bigfppf}\itememph{a} to  be flat and local complete intersections.
\end{rem}
The rest of the section mostly wasn't in the lecture, but rather it is an attempt to settle my confusion that arose from the plethora of different characterizations of smoothness. I hope it might be helpful to you as well.

Definition~\reff{def:smooth} is by far not the only way to define smooth morphisms -- and it captures only the noetherian case (we discuss the non-noetherian situation briefly in Subsection~\reff{subsec:nonnoetherian}). Another approach is via \emph{formally smooth} respectively \emph{formally étale} morphisms (cf.\ \cite[\stackstag{02GZ} and \stackstag{02HF}]{stacks-project}). However, it is not at all trivial that these approaches lead to the same notion of smooth- and étaleness. The Stacks Project actually proves this, but be aware that they don't define smooth and ètale morphisms in the way we do, but via rather strong splitting properties of the conomormal sequence (cf.\ \cite[\stackstag{00T1} and \stackstag{00U0}]{stacks-project}) -- which are much easier to work with, but a lot harder to establish in typical situations. Also, albeit their definitions surely \emph{are} equivalent to ours, this is a nightmare to prove.

If at this point confusion and frustration have driven you desperate enough, then having a look at \cite[Ch.\:IV \S17]{egaIV4} is definitely worth a try -- Grothendieck mercilessly proves all of the above equivalences.

For the reader's convenience (and to clear my own confusion), I will prove equivalence of the three most important characterizations of smoothness (except the one via formal smoothness) in Proposition~\reff{prop:smoothnessCriteria} below. Let's start with a definition.
\begin{defi}\lbl{def:relativeDimension}
	Let $f\colon X\morphism S$ be a morphism of preschemes. Let $x\in X$, $s=f(x)$ and $X_s=X\times_S\Spec \KK(s)$ denote the fibre over $S$. Then the \defemph{relative dimension} at $x$ of $X$ over $S$ is defined as
	\begin{align*}
		\dim_x(X_s)=\dim\Oo_{X_s,x}+\trdeg(\KK(x)/\KK(s))\;.
	\end{align*} 
\end{defi}

\begin{rem}
	If $f$ is of locally finite type, then $X_s$ is locally noetherian, so the relative dimension of $X$ over $S$ is finite at every $x\in X$. It should be thought of as the dimension of the fibre $X_s$ near the point $x$. Indeed, by \cite[Theorem~10]{alg1} this makes a lot of sense if $X_s$ is integral. The general case can be reduced to this since algebras of finite type over a field are catenary, so modding out a minimal prime ideal contained in $x$ shouldn't be any problem.
\end{rem}
\begin{prop}\lbl{prop:smoothnessCriteria}
	Let $f\colon X\morphism S$ be a morphism of locally finite type between locally noetherian preschemes. Then the following are equivalent.
	\begin{alphanumerate}
		\item $f$ is smooth in the sense of Definition~\reff{def:smooth}\itememph{a}.
		\item $f$ is flat and the sheaf of Kähler differentials $\Omega_{X/S}$ is locally free of rank the relative dimension of $X$ over $S$.
		\item $f$ is flat and its geometric fibres $X_{\ov{s}}=X\times_S\Spec\ov{\KK(s)}$ for $s\in S$ are regular.
	\end{alphanumerate}
\end{prop}
\begin{cor}\lbl{cor:smoothRegular}
	Let $f\colon X\morphism S$ be a smooth morphism. Then the fibres $X_s=f^{-1}\{s\}$ are regular for all $s\in S$.
\end{cor}
\begin{proof}
	By Proposition~\reff{prop:smoothnessCriteria}\itememph{c}, the geometric fibres $X_{\ov{s}}$ are regular, hence so are the ordinary ones by Proposition~\reff{prop:GeometricFibres101}\itememph{b}.
\end{proof}
In particular, Proposition~\reff{prop:smoothnessCriteria} implies that the relative dimension is locally constant on $X$. The proof will be quite lengthy (but the upside is it won't contain any spamming of the Jacobi criterion). We start with a lemma (taken from \cite[Ch.\:0 (19.1.12)]{egaIV1}), which will be the key to reduce conditions for $X$ to fibrewise conditions.
\begin{lem}\lbl{lem:tensorTrick}
	Let $A$ be a ring and $u\colon M\morphism P$ a morphism between finitely generated $A$-modules of which $P$ is projective. Then for a prime ideal $\pp\in\Spec A$ the following are equivalent.
\begin{alphanumerate}
	\item The morphism $u_\pp\colon M_\pp\morphism P_\pp$ is split injective.
	\item There is an $f\in A\setminus \pp$ such that $u_f\colon M_f\morphism P_f$ is split injective.
	\item The morphism $u\otimes\id_{\KK(\pp)}\colon M\otimes_A\KK(\pp)\morphism P\otimes_A\KK(\pp)$.
\end{alphanumerate}
\end{lem}
\begin{proof}
	Clearly \itememph{b} $\Rightarrow$ \itememph{a} $\Rightarrow$ \itememph{c}, so it suffices to show \itememph{c} $\Rightarrow$ \itememph{b}. Let $\mu_1,\ldots,\mu_n\in M$ be elements whose images form a basis of $M\otimes_A\KK(\pp)$. Let $\beta_i=u(\mu_i)$. Since $u\otimes\id_{\KK(\pp)}$ is injective, the images of $\beta_1,\ldots,\beta_n$ in $P\otimes_A\KK(\pp)$ can be extended to a basis, say, by the images of $\gamma_1,\ldots,\gamma_m\in P$. Apply \cite[Lemma~1.5.1]{alg2} to find $f\in A\setminus \pp$ such that $M_f$ is generated by $\mu_1,\ldots,\mu_n$ and $P_f$ is free with basis $\beta_1,\ldots,\beta_n,\gamma_1,\ldots,\gamma_m$. Then a split of $u$ is given by sending each $\beta_i$ to the corresponding $\mu_i$ and each $\gamma_j$ to $0$.
\end{proof}
A typical application of Lemma~\reff{lem:tensorTrick}, which is quite useful in its own right, is the following.
\begin{cor}\lbl{cor:tensorTrick}
	Let $f\colon X\morphism S$ be a morphism of schemes and $u\colon \Ff\morphism\Vv$ a morphism $\Oo_X$-modules, where $\Ff$ is quasi-coherent and locally finitely generated and $\Vv$ is a vector bundle. Then the following are equivalent:
	\begin{alphanumerate}
		\item $u$ is a locally split monomorphism.
		\item The pullback $u_s\colon \Ff|_{X_s}\morphism \Vv|_{X_s}$ is a locally split monomorphism for all $s\in S$.
		\item The pullback $u_{\ov{s}}\colon \Ff|_{X_{\ov{s}}}\morphism \Vv|_{X_{\ov{s}}}$ is a locally split monomorphism for all $s\in S$.
	\end{alphanumerate}
\end{cor}
\begin{proof}
	It's clear that \itememph{a} $\Rightarrow$ \itememph{b} $\Rightarrow$ \itememph{c}. Now let \itememph{c} be given. We may assume that $X$ and $Y$ affine, say, $X\cong \Spec B$ and $Y\cong \Spec A$. Then $\Ff$ and $\Vv$ are given by a finitely generated $B$-module $M$ and a projective $B$-module $P$, and shrinking $X$ if neccessary we may assume that $u_{\ov{s}}\colon M\otimes_A\ov{\KK(s)}\morphism P\otimes_A\ov{\KK(s)}$ is a split injection for some $s\in S$. Since $\ov{\KK(s)}$ is a direct sum of copies of $\KK(s)$, we see that any split of $u_{\ov{s}}$ restricts to a split of $u_s$, which shows \itememph{c} $\Rightarrow$ \itememph{b}.
	
	For \itememph{b} $\Rightarrow$ \itememph{a}, note that if $M\otimes_A\KK(s)\morphism P\otimes_A\KK(s)$ is a split monomorphism, then so is $M\otimes_B\KK(x)\morphism P\otimes_B\KK(x)$ for all $x\in X$ such that $s=f(x)$ (this uses that $\mm_{S,s}\Oo_{X,x}\subseteq \mm_{X,x}$). Then Lemma~\reff{lem:tensorTrick} shows that $u$ is a locally split monomorphism. We're done.
\end{proof}
\begin{proof}[Proof of Proposition~\reff{prop:smoothnessCriteria}]
	\emph{Proof of \itememph{a} $\Rightarrow$ \itememph{b}}. Since the property from Definition~\reff{def:smooth}\itememph{a} is pretty local, we may assume that $X\cong \Spec B$ and $S\cong \Spec A$ are affine, with $B$ of finite type over $A$, and $f$ factors over an étale morphism $X\morphism Y=\IA_A^n$. Let $R=A[X_1,\ldots,X_n]$ (so that $Y=\Spec R$). Since $B$ has finite type over $A$, hence also over $R$, we may choose a presentation $B\cong R[t_1,\ldots,t_m]/I$. Consider the conormal sequence
	\begin{align*}
		I/I^2\morphism \Omega_{R[t_1,\ldots,t_m]/R}\otimes_{R[t_1,\ldots,t_m]}B\morphism \Omega_{B/R}\morphism 0\;,
	\end{align*}
	in which $\Omega_{B/R}$ vanishes since $X\to Y$ is étale. Hence $I/I^2$ surjects onto the middle term, which is a free $B$-module generated by $\d t_1,\ldots,\d t_m$.
	
	Now $B\cong R[t_1,\ldots,t_m]/I\cong A[X_1,\ldots,X_n,t_1,\ldots,t_m]/I$ is also a presentation of $B$ as a finite type $A$-algebra, hence we also get a conormal sequence
	\begin{align*}
		I/I^2\morphism \Omega_{R[t_1,\ldots,t_m]/A}\otimes_{R[t_1,\ldots,t_m]}B\morphism \Omega_{B/A}\morphism 0\;.
	\end{align*}
	Since the middle term is a free $B$-module with basis $\d X_1,\ldots,\d X_n,\d t_1,\ldots,\d t_m$ and $I/I^2$ surjects onto the free submodule generated by $\d t_1,\ldots,\d t_m$ as seen above, we obtain an isomorphism $\bigoplus_{i=1}^nB\d X_i\cong \Omega_{B/A}$. 
	
	It remains to show that $n$ equals the relative dimension of $X$ over $S$. Consider $s\in S$, let $y\in Y$ be a point over $s$ and let $x\in X$ be a point over $y$. Then
	\begin{align}\lbl{eq:Anndim}
		n=\dim\Oo_{Y_s,y}+\trdeg(\KK(y)/\KK(s))
	\end{align}
	follows from \cite[Theorem~10]{alg1}. By Lemma~\reff{lem:unramified2}, $X\to Y$ being étale implies that $\KK(x)$ is finite separable over $\KK(y)$, hence $\trdeg(\KK(x)/\KK(s))=\trdeg(\KK(y)/\KK(s))$. Thus, together with \eqreff{eq:Anndim}, it suffices to show $\dim\Oo_{X_s,x}=\dim\Oo_{Y_s,y}$. We'll give two proofs, which finishes \itememph{a} $\Rightarrow$ \itememph{b}.
	
	We start with a quick-and-dirty argument. Since $X\morphism Y$ and thus also its base change $X_s\morphism Y_s$ is étale, hence flat, we deduce that $\Oo_{X_s,x}$ is faithfully flat over $\Oo_{Y_s,y}$. Then going-down (from Proposition~\reff{prop:flatGoingDown}) together with surjectivity of $\Spec \Oo_{X_s,x}\morphism\Spec\Oo_{Y_s,y}$ shows $\dim\Oo_{X_s,x}\geq \dim\Oo_{Y_s,y}$. On the other hand,
	\begin{align*}
		\dim\Oo_{X_s,x}\leq \dim_{\KK(x)}\mm_{X_s,x}/\mm_{X_s,x}^2&=\dim_{\KK(x)}\left(\mm_{Y_s,y}/\mm_{Y_s,y}^2\otimes_{\KK(y)}\KK(x)\right)\\
		&=\dim_{\KK(y)}\mm_{Y_s,y}/\mm_{Y_s,y}^2\\
		&=\dim\Oo_{Y_s,y}\;,
	\end{align*}
	which proves that equality holds. Here we needed $\mm_{Y_s,y}\otimes_{\Oo_{Y_s,y}}\Oo_{X_s,x}\cong \mm_{Y_s,y}\Oo_{X_s,x}=\mm_{X_s,x}$ (using that $X\morphism Y$ is both flat and unramified) and also that $Y_s\cong\IA_{\KK(s)}^n$ is regular.
	
	The morally correct proof, however, goes as follows: Since $\Omega_{X/Y}$ vanishes (for $X\morphism Y$ being étale), so does its base change $\Omega_{X_y/\KK(y)}$ to the fibre over $y$. In particular, $\Omega_{\Oo_{X_y,x}/\KK(y)}=0$. Then \cite[Corollary~16.16]{eisenbudCommAlg} shows that $\Oo_{X_y,x}$ is a finite product of finite separable field extensions of $\KK(y)$. In particular, $\dim\Oo_{X_y,x}=0$, so it suffices to show $\dim\Oo_{X_s,x}=\dim\Oo_{Y_s,y}+\dim\Oo_{X_y,x}$. This follows from \cite[\stackstag{02JS}]{stacks-project}.
	
	\emph{Proof of \itememph{b} $\Rightarrow$ \itememph{c}.} Let $s\in S$ and put $k=\KK(s)$. By base changing $\Omega_{X/S}$ to the fibre $X_s$ over $s\in S$, we see that $\Omega_{X_s/k}$ is locally free at $x$ of rank $\dim_x(X_s)$ for every $x\in X_s$. Hence (by another base change) for every point $\ov{x}$ over $x$ in the geometric fibre, the sheaf of Kähler differentials $\Omega_{X_{\ov{s}}/\smash{\ov{k}}}$ is locally free at $\ov{x}$ of rank $\dim_x(X_s)$. But $\dim_x(X_s)=\dim_{\ov{x}}(X_{\ov{s}})$ follows from Proposition~\reff{prop:GeometricFibres101}\itememph{a}. This  assures that we can apply \cite[Proposition~1.6.3]{alg2} to see that $X_{\ov{s}}$ is indeed regular.
	
	\emph{Proof of \itememph{b} $\Rightarrow$ \itememph{a}.} So much for the straightforward part of the proof, now the actual work begins. Choose affine opens $W\subseteq S$ and $U\subseteq X$, say $W\cong \Spec A$ and $U\cong \Spec B$, such that $f(U)\subseteq W$, $B$ is of finite type over $A$, and $\Omega_{X/S}$ is free on $U$. Since $X$ can be covered by such $U$, it suffices to factor $f|_U\colon \Spec B\morphism\Spec A$ through an étale morphism $\Spec B\morphism \IA_A^n$. So we may w.l.o.g.\ assume $X=\Spec B$ and $S=\Spec A$.
	
	Let $b_1,\ldots,b_n\in B$ be elements whose images $\d_{B/A}b_1,\ldots,\d_{B/A}b_n$ in $\Omega_{B/A}$ form a basis (it's clear that such $b_i$ exist, cf.\ the brute-force construction in \cite[Proposition~1.4.1]{alg2}). Let $R=A[X_1,\ldots,X_n]$ and define an $A$-algebra morphism $R\morphism B$ via $X_i\mapsto b_i$. This gives a morphism $X\morphism Y=\IA_A^n$. We show that it is indeed étale. 
	
	In the relative cotangent sequence
	\begin{align*}
		\Omega_{R/A}\otimes_RB\morphism \Omega_{B/A}\morphism \Omega_{B/R}\morphism 0\;,
	\end{align*}
	the first arrow is an isomorphism since it sends the basis $\d X_1,\ldots,\d X_n$ of the free $B$-module $\Omega_{R/A}\otimes_RB$ to the basis $\d_{B/A}b_1,\ldots,\d_{B/A}b_n$ of $\Omega_{B/A}$. Hence $\Omega_{B/R}=0$, so $X\morphism Y$ is unramified.
	
	For flatness, we will use a sequence of flatness criteria which is set up below. Consider $s\in S$, let $y\in Y$ be a point over $s$ and let $x\in X$ be a point over $y$. We first show that $\Oo_{X_s,x}$ is flat over $\Oo_{Y_s,y}$. Since \itememph{b} $\Rightarrow$ \itememph{c} has already been established, we see that the geometric fibre $X_{\ov{s}}$ is regular, hence so is the ordinary one by Proposition~\reff{prop:GeometricFibres101}. Also $Y_s\cong \IA_{\KK(s)}^n$ is clearly regular. In particular, $\Oo_{X_s,x}$ and $\Oo_{Y_s,y}$ are both regular noetherian local rings. Moreover, since the rank $n$ of $\Omega_{X/S}$ equals the relative dimension of $X/S$, we obtain
	\begin{align*}
		\dim\Oo_{X_s,x}+\trdeg(\KK(x)/\KK(s))=n=\dim\Oo_{Y_s,y}+\trdeg(\KK(y)/\KK(s))\;.
	\end{align*}
	Note that $\KK(x)/\KK(y)$ is finite separable by Lemma~\reff{lem:unramified2} since $X\morphism Y$ is unramified. Hence $\dim\Oo_{X_s,x}=\dim\Oo_{Y_s,y}$. Moreover, as in the \emph{morally correct} proof above, we see that $\Omega_{X/Y}=0$ implies $\Oo_{X_y,x}=\Oo_{X_s,x}/\mm_{Y_s,y}\Oo_{X_s,x}$ is zero-dimensional. Thus
	\begin{align*}
		\dim\Oo_{X_s,x}=\dim\Oo_{Y_s,y}+\dim\Oo_{X_s,x}/\mm_{Y_s,y}\Oo_{X_s,x}\;,
	\end{align*}
	so Lemma~\reff{lem:miracleFlatness} comes to the rescue showing that $\Oo_{X_s,x}$ is indeed flat over $\Oo_{Y_s,y}$.
	
	Now apply Lemma~\reff{lem:platitudeDeFibres} with $M=B'=\Oo_{X,x}$ to deduce that $\Oo_{X,x}$ is flat over $\Oo_{Y,y}$, thus proving that $X\morphism Y$ is indeed flat and hence étale.  Here we used that $\Oo_{X,x}$ is flat over $\Oo_{S,s}$ (since $f\colon X\morphism S$ is flat by assumption) and $\Oo_{X,x}/\mm_{S,s}\Oo_{X,x}\cong \Oo_{X_s,x}$ is flat over $\Oo_{Y,y}/\mm_{S,s}\Oo_{Y,y}\cong \Oo_{Y_s,y}$ (as we just proved).
	
	\emph{Proof of \itememph{c} $\Rightarrow$ \itememph{b}.} The last remaining implication will require the most work of all. Let's first outline our strategy. As usual we may assume $X\cong \Spec B$ and $Y\cong \Spec A$ such that $B$ is of finite type over $A$. Our goal is to show that $\Omega_{B/A}$ is projective of rank the relative dimension of $X$ over $S$. Put $T=A[t_1,\ldots,t_m]$ and choose a presentation $B\cong T/I$. If we can show that the conormal sequence
	\begin{align}\lbl{eq:conormalSeq}
		I/I^2\morphism \Omega_{T/A}\otimes_TB\morphism\Omega_{B/A}\morphism 0
	\end{align}
	is injective on the left and locally split exact, then we're certainly done (up to proving that $\Omega_{B/A}$ has the correct rank), since in this case $\Omega_{B/A}$ becomes locally a direct summand of the free $B$-module $\Omega_{T/A}\otimes_TB\cong B^m$, hence it will be (locally and thus everywhere) projective. We will do this by a sequence of reductions steps.
	
	\emph{Step 1.} The first step is to reduce everything to a question about geometric fibres. By Corollary~\reff{cor:tensorTrick}, $\nabla\colon I/I^2\morphism B^m$ is locally split injective iff the same is true $\nabla_{\ov{s}}\colon I/I^2\otimes_A\ov{\KK(s)}\morphism B^m\otimes_A\ov{\KK(s)}$. We put $k=\ov{\KK(s)}$ for convenience. There is an exact sequence
	\begin{align*}
		\Tor_1^A(k,B)\morphism I\otimes_Ak\morphism T\otimes_Ak\morphism B\otimes_Ak\morphism 0\;,
	\end{align*}
	in which $T\otimes_Ak\cong k[t_1,\ldots,t_m]$ and $B\otimes_Ak\cong k[t_1,\ldots,t_m]/J$, where $J\subseteq k[t_1,\ldots,t_m]$ is the ideal generated by the image of $I$. Hence the vanishing of $\Tor_1^A(k,B)$ (since $B$ is flat over $A$ by assumption) shows that $I\otimes_Ak\cong J$. Hence the map $\nabla_{\ov{s}}$ of which injectivity and local splitness is to be shown becomes part of an ordinary conormal sequence
	\begin{align}\lbl{eq:conormalSeq2}
		J/J^2\morphism \Omega_{T/k}\otimes_TB\morphism \Omega_{B/k}\morphism 0
	\end{align}
	 (where the old $T$ was replaced by $k[t_1,\ldots,t_m]$ and the old $B$ by $B\otimes_A k$). That is, we are in the same situation as before, but with two advantages: The first is that we are over an algebraically closed field now. For the second, let's replace $X$ by $X_{\ov{s}}$ first, so that $X$ is now regular over the algebraically closed field $k$. Then we already know $\Omega_{B/k}$ to be projective. Indeed, \cite[Proposition~1.6.3]{alg2} shows that that the sheaf $\Omega_{X/k}\cong (\Omega_{B/k})^\qcmod$ is locally free of rank the relative dimension of $X$ over $k$.
	 
	 \emph{Step 2.} Now that the reduction to the geometric fibres is done, the next step is to reduce to the case of $f\colon X\morphism \Spec k$ being étale. First note that instead of $B\cong T/J$ we may use any other presentation $B\cong T'/J'$, $T'=k[t_1,\ldots,t_{m'}]$ we like. Indeed, for any two such presentations, their corresponding \emph{cotangent complexes}\footnote{The cotangent complex associated to $B\cong T/J$ is defined as
	 \begin{align*}
	 	\ldots \morphism 0\morphism 0\morphism J/J^2\morphism \Omega_{T/J}\otimes_TB\morphism 0\morphism 0\morphism\ldots\;,
 	\end{align*}
	with the only non-zero entries in degree $0$ and $1$. I admit this is a rather unimpressive complex.} are homotopy equivalent by \cite[\stackstag{00S1}]{stacks-project}. Hence if \eqreff{eq:conormalSeq2} is injective on the left for $T'$ and $J'$, then the same is true for $T$ and $J$ (and splitness is automatic since $\Omega_{B/k}$ is projective), as claimed.

	Since \itememph{b} holds for $X$ and \itememph{b} $\Rightarrow$ \itememph{a} has already been established, we may cover $X$ by affine open subsets $U$ for which $f|_U\colon U\morphism\Spec k$ factors over an étale morphism $U\morphism\IA_k^n$. Since \eqreff{eq:conormalSeq2} being injective on the left is a local question, we may replace $X$ by such a $U$ (and consequently $B$ by $\Oo_X(U)$). Put $R=k[X_1,\ldots,X_n]$ and replace $T$ and $J$ by a presentation $B\cong T/J$, $T=R[t_1,\ldots,t_m]$ (we may do this as argued above). Then $\Omega_{T/R}\otimes_TB$ is isomorphic to the submodule of $\Omega_{T/k}\otimes_TB$ generated by $\d t_1,\ldots,\d t_m$. Hence for \eqreff{eq:conormalSeq2} to be injective on the left it suffices that the conormal sequence $J/J^2\morphism\Omega_{T/R}\otimes_TB\morphism\Omega_{B/R}\morphism 0$ for $R$ instead of $k$ is (locally) split injective on the left (and thus the left arrow is an isomorphism as $\Omega_{B/R}=0$ from $X\morphism \IA_k^n$ being étale). Hence, replacing the original $S$ by $S=\IA_k^n$, we're in the same situation as at the very beginning (that is, before Step~1), but now $X\morphism S$ is étale. Then we may apply the reduction from Step~1 once again to further reduce this to the case of $S=\Spec k$ with $k$ an algebraically closed field.
	
	\emph{Step 3.} Finally let's deal with the étale case. Here comes the last ingredient of the proof: We claim that if the canonical morphism $T/J^2\morphism T/J\cong B$ has a split, then \eqreff{eq:conormalSeq2} is split injective on the left. Indeed, we actually proved a pretty similar result in \cite[Lemma~1.6.1]{alg2}, and the proof there can literally be copied -- except for the part where we construct a split of $R/\mm^2\morphism R/\mm$, since such a split is already given (and that's the only time the proof uses that $\mm$ is maximal).
	
	Now let's construct such a split! Since $B$ has finite type over $k$ and $\Omega_{B/k}=0$ (from $X\morphism\Spec k$ being étale), \cite[Corollary~16.16]{eisenbudCommAlg} shows that $B$ is a finite product of finite separable field extensions of $k$. But $k$ is algebraically closed, so $B\cong k^r$ for some $r\geq 1$. Note that $T/J^2$ is $J$-adically complete (in a pretty lame way). Now since the standard basis vectors $e_1,\ldots,e_r$ in $B\cong k^r$ are idempotents, Corollary~\reff{cor:HenselApplications}\itememph{b} shows that each $e_i$ has a unique idempotent lift $\tau_i\in T/J^2$. These guys satisfy $\tau_i\tau_j=0$ for all $i\neq j$. Indeed, $\tau_i\tau_j$ is a lift of $e_ie_j=0$, hence lies in $J$. But $\tau_i\tau_j$ is also an idempotent, so $\tau_i\tau_j\in J^2=0$. Hence $\tau_1+\ldots+\tau_r$ is an idempotent and a lift of $1=e_1+\ldots+e_r$, so $1=\tau_1+\ldots+\tau_r$ as well by the uniqueness part of Corollary~\reff{cor:HenselApplications}\itememph{b}. This shows that $e_i\mapsto \tau_i$ defines morphism of $k$-algebras $B\morphism T/J^2$ which is a split of $T/J^2\morphism B$. This settles the étale case.
	
	\emph{Step 4.} We have proved by now that $\Omega_{B/A}$ in \eqreff{eq:conormalSeq} is projective, so it remains to check that the rank is correct. But since $\dim_x(X_s)=\dim_{\ov{x}}(X_{\ov{s}})$ for all $\ov{x}$ over $x$ (by Proposition~\reff{prop:GeometricFibres101}\itememph{a}), the rank may be checked after base changing to the geometric fibres, where it is certainly correct by \cite[Proposition~1.6.3]{alg2}. We're done at last \ldots
\end{proof}
\ldots well, modulo proving the asserted flatness criteria, that is. The point where all of this begins is the \emph{local flatness criterion}. It might look like \cite[Proposition~1.3.1\itememph{d}]{homalg} at first glance, but its true power comes from the fact that it has \emph{no finiteness restriction at all} on the morphism $A\morphism B$!! Our presentation here follows \cite[\stackstag{00MD}]{stacks-project}.
\begin{prop}[Local flatness criterion]\lbl{prop:localFlatness}
	Let $A\morphism B$ be a local morphism of noetherian local rings. Let $M$ be a finitely generated $B$-module. Then $M$ is flat over $B$ iff $\Tor_1^A(A/\mm_A,M)=0$.
\end{prop}
\begin{proof}
	Omitted -- but you can find it in \cite[Theorem~6.8]{eisenbudCommAlg} or \cite[\stackstag{00MK}]{stacks-project}.
\end{proof}
A common and useful variant is the following.
\begin{cor}\lbl{cor:localFlatnees}
	Let $A\morphism B$ be a local morphism of noetherian local rings and let $M$ be a finitely generated $B$-module. Suppose for some proper ideal $I\subsetneq A$ the quotient $M/IM$ is flat over $A/I$ and $\Tor_1^A(A/I,M)=0$. Then $M$ is flat over $A$.
\end{cor}
\begin{proof}
	Consider the sequence $0\morphism\mm_A/I\morphism A/I\morphism A/\mm_A\morphism 0$. Tensoring with $M$ gives an exact sequence
	\begin{align*}
		\Tor_1^A(A/I,M)\morphism \Tor_1^A(A/\mm_A,M)\morphism \mm_A/I\otimes_AM\morphism A/I\otimes_AM
	\end{align*}
	in which $\Tor_1^A(A/I,M)$ vanishes by assumption. Hence we're done once we show that the morphism $\mm_A/I\otimes_AM\morphism A/I\otimes_AM$ is injective, since then $\Tor_1^A(A/\mm_A,M)=0$, which shows that $M$ is flat over $A$ by the local flatness criterion (Proposition~\reff{prop:localFlatness}).
	
	To obtain said injectivity, note that $\mm_A/I\otimes_AM\cong \mm_A/I\otimes_{A/I}M/IM$ and $A/I\otimes_AM\cong M/IM$. Now the sequence
	\begin{align*}
		\Tor_1^{A/I}(A/\mm_A,M)\morphism \mm_A/I\otimes_{A/I}M/IM\morphism M/IM
	\end{align*}
	is exact (as part of a long exact $\Tor$ sequence) and $\Tor_1^{A/I}(A/\mm_A,M)$ vanishes since $M/IM$ is flat over $A/I$ by assumption, so $\mm_A/I\otimes_AM\morphism A/I\otimes_AM$ is indeed injective.
\end{proof}
The following lemma uses the notion of depth and Cohen--Macaulay rings. If you don't feel familiar with these, have a look at \cite[Section~2.3 and 2.4]{homalg}.
\begin{lem}[Miracle flatness theorem]\lbl{lem:miracleFlatness}
	Let $A\morphism B$ be a local morphism of noetherian local rings. Assume that $A$ is regular, $B$ is Cohen--Macaulay, and we have
	\begin{align*}
		\dim B=\dim A+\dim B/\mm_AB\;.
	\end{align*}
	Then $B$ is flat over $A$.
\end{lem}
\begin{proof}
	We do induction on $\dim A$. If $\dim A=0$, then $A$ being regular implies that $A$ is a field and the assertion is trivial. Now let $\dim A>0$ and thus also $\dim B>0$ by assumption. Let $\qq_1,\ldots,\qq_n$ be the finitely many minimal prime ideals of $B$. We claim that $\qq_i\nsupseteq \mm_AB$ for all $i$. Indeed, otherwise we would have $\dim B=\dim B/\qq_i\leq \dim B/\mm_AB$, hence $\dim A=0$. Therefore, if $\pp_i=\qq_i\cap A$ denotes the preimage of $\qq_i$ in $A$, then $\pp_i\subsetneq \mm_A$. In particular, by prime avoidance (see \cite[Lemma~2.5.1]{alg1}) it is possible to choose an element $x\in\mm_A\setminus\bigcup_{i=1}^n\pp_i$.
	
	The plan is to replace $B$ and $A$ by $B/xB$ and $A/xA$. Since $B$ is Cohen--Macaulay, all associated primes are minimal (see \cite[Corollary~2.4.2]{stacks-project}), so the $\qq_i$ comprise all the associated primes of $B$. Thus, by choice of $x$ and \cite[Proposition~A.3.3\itememph{a}]{homalg}, we see that multiplication with $x$ is injective on $B$. Hence $(x)$ is a $B$-regular sequence. By \cite[Proposition~2.3.2\itememph{b}]{homalg} we may extend it to a $B$-regular sequence $(x,x_2,\ldots,x_n)$, where $n=\depth_B(B)=\dim B$. Then $(x_2,\ldots,x_n)$ is a $B/xB$-regular sequence, so that $\depth_{B/xB}(B/xB)\geq n-1$. But $\dim B/xB\leq n-1$ (since none of the minimal prime ideals $\qq_i$ contains $x$), so \cite[Fact~2.4.1]{homalg} shows that equality must hold in both inequalities. In particular, $B/xB$ is Cohen--Macaulay again.
	
	Moreover, the proof of \cite[Proposition~2.2.1]{homalg} shows that $\dim A/xA$ is regular again and $\dim A/xA=\dim A-1$. Since $B/\mm_AB$ remains unchanged when replacing $B$ and $A$ by $B/xB$ and $A/xA$, we see that still
	\begin{align*}
		\dim B/xB=\dim A/xA+\dim B/\mm_AB
	\end{align*}
	holds. Hence the inductive hypothesis shows that $B/xB$ is flat over $A/xA$. Moreover, $\Tor_1^A(A/xA,B)$ computes the $x$-torsion part of $B$, which vanishes since $(x)$ is $B$-regular. Thus Corollary~\reff{cor:localFlatnees} is applicable to show that $B$ is flat over $A$.
\end{proof}
\begin{lem}[Critère de platitude par fibres]\lbl{lem:platitudeDeFibres}
	Let $A\morphism B\morphism B'$ be local morphisms of noetherian local rings. Let $M$ be a finite $B'$-module such that $M$ is flat over $A$ and $M/\mm_AM$ is flat over $B/\mm_AB$. Then $M$ is also flat over $B$.
\end{lem}
\begin{proof}
	Put $I=\mm_AB$, so that $M/IM$ is flat over $B/I$ by assumption. Then by Corollary~\reff{cor:localFlatnees} it suffices to show $\Tor_1^B(B/I,M)=0$. Note that $\Tor_1^B(B/I,M)=\ker(I\otimes_BM\morphism M)$. Now the multiplication map $\mm_A\otimes_AB\morphism\mm_AB=I$ is surjective, hence $\mm_A\otimes_AM\morphism I\otimes_BM$ is surjective as well. But the composite $\mm_A\otimes_AM\morphism I\otimes_BM\morphism M$ is injective since $M$ is flat over $A$, which means that so is $I\otimes_BM\morphism M$. Hence we're done.
\end{proof}
The last flatness criterion on the list really doesn't have anything to do with the previous ones, and also we don't need it for the proof of the smoothness criteria in Proposition~\reff{prop:smoothnessCriteria}. However, it will prove useful (even if it is a really silly statement) later in the text, and I decided fits best with the above collection of flatness criteria.
\begin{lem}\lbl{lem:sillyFlatness}
	Assume that
	\begin{diagram*}
		\object{0,0}{$A$}[A];
		\object{2.5,0}{$B$}[B];
		\object{0,1.5}{$A'$}[a];
		\object{2.5,1.5}{$B'$}[b];
		\scriptsize
		\arrow AB;
		\arrow Aa;
		\arrow Bb;
		\arrow ab;
	\end{diagram*}
	is a diagram of rings, in which $B$ and $B'$ are local rings and the morphism $B\morphism B'$ is local as well. Assume $A'$ is flat over $A$, and $B'$ is flat over $A'$ as well as over $A'\otimes_AB$ (in particular, the last assertion is fulfilled if $B'$ is a localization of $A'\otimes_AB$). Then $B$ is flat over $A$.
\end{lem}
\begin{proof}
	First note that since $A'$ is flat over $A$ and flatness is preserved under base change, $A'\otimes_AB$ is flat over $B$. Hence $B'$ is flat over $B$ as well, and since we have a local morphism of local rings here, we see that $B'$ is actually faithfully flat over $B$ by Proposition~\reff{prop:faithfullyFlatModule}\itememph{d}. Hence $B$ is flat over $A$ iff $B'$ is flat over $A$. But the latter is clearly true by assumption.
\end{proof}


\chapter{Abelian varieties}
Recall that a \defemph{group prescheme} over $S$ is an $S$-prescheme $G$ equipped with a \emph{binary operation} $s_G\colon G\times_SG\morphism G$, an \emph{inversion} $\iota\colon G\morphism G$ and a \emph{neutral element} $0_G\colon S\morphism G$, such that the diagrams from page~\pageref{par:groupScheme} (except possibly the one for commutativity) commute.
\begin{defi}\lbl{def:abelianScheme}
	An \defemph{abelian scheme} over $S$ is a proper flat $S$-prescheme $\alpha\colon A\morphism S$ with the structure of a group prescheme over $S$  and such that all geometric fibres $A_{\ov{s}}=A\times_S\Spec\ov{\KK(s)}$ are varieties.\footnote{That is, integral and of finite type (but the latter is automatically implied by properness of $\alpha$). It's about time we define what a variety is, really.}
\end{defi}
The following fact is pretty important for what we're going to do, even though Professor Franke forgot to mention it in the lecture (but explained it to me afterwards).
\begin{fact}\lbl{fact:groupSchemesSmooth}
	Let $\alpha\colon A\morphism S$ be an abelian scheme. If $S$ is locally noetherian (or $\alpha$ is locally of finite presentation), then $A$ is smooth over $S$.
\end{fact}
\begin{proof}
	Let's assume $S$ is locally noetherian. By Proposition~\reff{prop:smoothnessCriteria} it suffices to show that the geometric fibres $A_{\ov{s}}$ are regular varieties. Note that the group prescheme structure of $A$ survives on the geometric fibres. That is, $A_{\ov{s}}$ is an abelian scheme over $\Spec \ov{\KK(s)}$ for all $s\in S$. Hence, replacing $A$ by $A_{\ov{s}}$ we may assume that $S$ is the spectrum of an algebraically closed field $k$ and need to show that $A$ is a regular variety.
	
	Let us first show that the \emph{regular locus} (i.e., the subset of points $a\in A$ such that $\Oo_{A,a}$ is regular) is open and contains a closed point. Let $U\cong \Spec R$ be an affine open subset of $A$, where $R$ is a finite type $k$-algebra. Then $R$ is a domain since $A$ is integral (being a variety). By Grothendieck's generic freeness theorem (Proposition~\reff{prop:GenericFreeness}) there is an $f\in R\setminus \{0\}$ such that $(\Omega_{R/k})_f\cong \Omega_{R_f/k}$ is free over $R_f$. Since $A$ is reduced, we may apply \cite[Proposition~1.6.3]{alg2} with no rank restrictions to see that $R_f$ is regular, hence $U$ contains a regular open subset. Also any maximal ideal of $R_f$ corresponds to a point that is closed in the open subset $\Spec R_f$ of $A$ and thus also closed in $A$, using that $A$ has finite type over $k$, hence is a Jacobson prescheme (here we used all relevant results of \cite[Section~2.4]{alggeo1}).
	
	Now we shall see that $A$ is in fact regular. It suffices to show regularity at every closed point since a localization of a regular local ring at a prime ideal is regular again (cf.\ \cite[Corollary~2.2.1]{homalg}). Let $a\in A$ be a regular closed point (which exists as seen above) and $b\in A$ another closed point. Let $\sigma_a,\sigma_b\colon \Spec k\morphism A$ be the corresponding sections of $\alpha$ (here we use the first bijection from equation~\eqreff{eq:ZariskiTangentSpace}). Consider the element $\sigma=\sigma_b-\sigma_a$ in the group\footnote{Dear Professor Franke, please forgive me for using additive notation in a group I do not (yet) know commutativity of \Winkey[1.2].} $\Hom_{\cat{PSch}/k}(\Spec k,A)$ (which we will soon denote $A(k)$ -- the \emph{$k$-valued points} of $A$). Then we obtain a morphism $t_\sigma\colon A\morphism A$ given by the composition of $(\id_A,\sigma\alpha)\colon A\morphism A\times_kA$ and the group operation $s_A\colon A\times_kA\morphism A$. We call it the \emph{translation by $\sigma$}. Note that $t_\sigma$ is an isomorphism -- an inverse is given by $t_{-\sigma}$, with $-\sigma$ denoting the inverse of $\sigma$ in the group $A(k)$ of $k$-valued points -- and also $t_\sigma$ maps $a$ to $b$. In particular, $t_\sigma$ induces an isomorphism $\Oo_{A,a}\cong \Oo_{A,b}$, which shows that $b$ is a regular point as well. Since $b$ was chosen arbitrarily, we're done.
	
	The case where $S$ is not necessarily noetherian but $\alpha$ is locally of finite presentation can be treated similarly, using the methods from Subsection~\reff{subsec:nonnoetherian}.
\end{proof}
\begin{cor}\lbl{cor:groupSchemesSmooth}
	If $\alpha\colon A\morphism S$ is an abelian scheme with $S$ locally noetherian (or $\alpha$ locally finitely presented), then $\Omega_{A/S}$ is a vector bundle on $A$.
\end{cor}
\begin{proof}
	Follows from Fact~\reff{fact:groupSchemesSmooth} and Proposition~\reff{prop:smoothnessCriteria}.
\end{proof}
\section{Application of a rigidity principle}
\begin{prop}\lbl{prop:rogodoty}
	Let $S$ be locally noetherian and $\xi\colon X\morphism S$ be a proper morphism whose geometric fibres $X_{\ov{s}}=X\times_S\Spec\ov{\KK(s)}$ are varieties and we have an isomorphism $\Oo_S\isomorphism\xi_*\Oo_X$. Let $\upsilon \colon Y\morphism S$ be another $S$-prescheme and $f\colon X\morphism Y$ be any morphism over $S$.
	\begin{alphanumerate}
		\item The set 
		\begin{align*}
			U=\left\{s\in S\st \text{the image of }f_{\ov{s}}\colon X_{\ov{s}}\to Y_{\ov{s}}\text{ consists of  a single point}\right\}
		\end{align*}
		is open in $S$.
		\item For a morphism $\tau\colon T\morphism S$ the base change $f_T\colon X_T\morphism Y_T$ factors over $\xi_T\colon X_T\morphism T$ iff the image of $\tau$ is contained in $U$.
		\item If, in addition, $\xi$ is flat and $\upsilon$ is separated\footnote{The latter condition was missing in the lecture, but I can't see how to finish the proof without it.}, then $U$ is also closed (and the assumption $\Oo_S\cong\xi_*\Oo_X$ is automatically true).
	\end{alphanumerate}
\end{prop}
\begin{proof}
	Part~\itememph{a}. The first step is as usual to replace as much as possible of the occuring schemes by affine ones. Since the assertion is local on $S$, we may assume that $S$ is affine.	Let $s\in U$, then $f(X_s)=\{y\}\subseteq Y_s$ consists of a single point, since this is also true for the geometric fibres. Let $W\subseteq Y$ be an affine open neighbourhood of $y$. As $\xi$ is proper (hence universally closed), the subset $Z=\xi(X\setminus f^{-1}(W))$ is closed in $S$ and moreover $s\notin Z$. Let $\lambda\in\Oo_S(S)$ such that $s\notin V(\lambda)$ and $V(\lambda)\supseteq Z$. Replacing $S$ by $S\setminus V(\lambda)$ and $Y$ by $\upsilon^{-1}(S\setminus V(\lambda))=W\setminus V(\upsilon ^*\lambda)$ (and $X$ by the preimage under $\xi$ of the new $S$) we achieve that both $S$ and $Y$ are affine.
	
	Put $S=\Spec A$ and $Y=\Spec B$. Then $\Hom_{\cat{PSch}/S}(X,Y)\cong \Hom_{\cat{Alg}(A)}(B,\Oo_X(X))$ (by a version of the well-known adjunction). But $\Oo_X(X)\cong A$ by our assumption that $\Oo_S\cong \xi_*\Oo_X$, hence $f$ is given by a section $\rho^*\colon B\morphism A$ of the algebraic component $\upsilon^*\colon A\morphism B$ of $\upsilon$. But then the corresponding morphism $\rho\colon S\morphism Y$ satisfies $f=\rho\xi$ (indeed, by the above adjunction its enough to check this on global sections, where it is true by construction). This clearly implies $U=S$ (since then also $f_{\ov{s}}=\rho_{\ov{s}}\xi_{\ov{s}}$, i.e., $f_{\ov{s}}$ factors over $\Spec\ov{\KK(s)}$, which is a point). In particular, $U$ is open in $S$, as claimed. And also we see that $U$ doesn't change if the geometric fibres in the definition are replaced by ordinary ones, since that's all we needed to construct $\rho$.
	
	Part~\itememph{b}. Note that the restriction of $f$ to $\xi^{-1}(U)$ factors over $\xi$ since the local construction of $\rho$ from part~\itememph{a} is canonical, hence behaves fine under gluing. Hence if the image of $\tau$ is contained in $U$, then also $f_T$ factors over $\xi_T$, as claimed. Conversely, assume that $f_T$ factors over $\xi_T$. Let $t\in T$, $s=\tau(t)$ and put $k=\ov{\KK(s)}$, $K=\ov{\KK(t)}$. Consider the diagram
	\begin{diagram}[baseline=0.75cm-0.5ex][\lbl{diag:geometricFibres}]
		\object{0,1.5}{$X\times_S\Spec K$}[a];
		\object{0,0}{$X\times_S\Spec k$}[b];
		\object{3.5,0}{$Y\times_S\Spec k$}[c];
		\object{3.5,1.5}{$Y\times_S\Spec K$}[d];
		\scriptsize
		\arrow ab;
		\arrow bc;
		\arrow ad;
		\arrow dc;
	\end{diagram}
	The argument from part~\itememph{a} shows that the image of the top horizontal arrow consists of a single point, so it suffices to show that the left vertical arrow is surjective (as a map of topological spaces). Note that $X\times_S\Spec K\cong (X\times_S\Spec k)\times_{\Spec k}\Spec K$. However, the fibre product over $\Spec k$ of $X\times_S\Spec k$ and $\Spec K$ as topological spaces (!) is homeomorphic to $X\times_S\Spec k$ since $\Spec k$ and $\Spec K$ are single points. Hence the surjectivity we need follows from the fact that the scheme-theoretic fibre product surjects onto the topological one (cf.\ \cite[Corollary~1.3.2\itememph{a}]{alggeo1})
	
	Part~\itememph{c}. If $\xi$ is flat, then $\Oo_S\cong \xi_*\Oo_X$ follows from Lemma~\reff{lem:flatProperWithIntegralFibres}, so we don't need this as an extra assumption. To show that $U$ is closed, let's first assume $\xi$ has a section $\sigma\colon S\morphism X$. Consider the set
	\begin{align*}
		E=\Eq\Big(X\doublemorphism[f][f\sigma\xi]Y\Big)\subseteq X\;.
	\end{align*}
	Note that $S\setminus U=\xi(X\setminus E)$. Indeed, a point $s\in S$ is not contained in $U$ iff there is some $x\in X_s$ such that $f(x)\neq f(\sigma(s))$ (here we use that we may work with ordinary fibres instead of geometric ones as pointed out at the end of part~\itememph{a}) -- that is, $s\in S\setminus U$ iff $X_s\setminus E$ is non-empty. Since $\upsilon$ is assumed separated, $E$ is a closed subset of $X$, hence $S\setminus U=\xi(X\setminus E)$ is open since $\xi$ is flat and of finite type, hence an open map by Proposition~\reff{prop:flatGoingDown}. Then $U$ is closed, as claimed.
	
	In the general case we use that the construction of $U$ is compatible with base change. That is, if $\tau\colon T\morphism S$ is any $S$-prescheme, then $\tau^{-1}(U)$ has the corresponding property for $f_T\colon X_T\morphism X_T$ (which can be shown by the same arguments as in part~\itememph{b}). Base changing by $\xi$ itself, we always get a section (the diagonal $\Delta_{X/S}\colon X\morphism X\times_SX$), hence $\xi^{-1}(U)$ is closed by the above special case. Then $U$ is closed itself since $\xi$ is proper and thus a closed map.
\end{proof}
\begin{rem*}
	Without the flatness assumption one could take $X$ to be a blow-up of  a point of $S$ and $f$ to be the identity $\id_X\colon X\morphism X$. Then the closedness assertion fails.
\end{rem*}
Since only $S$-preschemes are considered in the following, we will write $\Omega_X$ as a shortcut for $\Omega_{X/S}$. If $f\colon X\morphism Y$ is a morphism of $S$-preschemes, we obtain a canonical pullback of differentials
\begin{align*}
	f^\oast \colon f^*\Omega_Y\morphism\Omega_X
\end{align*}
(locally this is a part of the relative cotangent sequence, cf.\ \cite[Corollary~1.4.2\itememph{b}]{alg2}). The notation $f^\oast$ is chosen to avoid confusion with pullbacks of sheaves.

If $X$ and $Y$ are $S$-preschemes and $p_1,p_2$ denote the projections from $X\times_SY$ to its factors, then there is a canonical isomorphism
\begin{align}\lbl{eq:p1p2Omega}
	p_1^\oast+p_2^\oast\colon p_1^*\Omega_X\oplus p_2^*\Omega_Y\isomorphism \Omega_{X\times_SY}\;.
\end{align}
As usual, this can be checked locally, where it is rather easy to see and follows, e.g., from \cite[Corollary~16.5]{eisenbudCommAlg}.
\begin{thm}\lbl{thm:abelianSchemes}
	Let $S$ be locally noetherian and $\alpha\colon A\morphism S$ an abelian scheme with binary operation $s_A\colon A\times_SA\morphism A$, inversion $i_A\colon A\morphism A$, and neutral section $0_A\colon S\morphism A$.
	\begin{alphanumerate}
		\item The group law on $A$ is abelian (hence the name).
		\item If $\beta\colon B\morphism S$ is another abelian scheme, then any morphism $f\colon A\morphism B$ over $S$ has the form
		\begin{align*}
			f=\phi+b\alpha\;,
		\end{align*}
		where $\phi\colon A\morphism B$ is a morphism of group preschemes over $S$ (i.e., compatible with the group scheme structure morphisms) and $b\colon S\morphism B$ a section of $\beta$; and ``$+$'' is taken in the group $\Hom_{\cat{PSch}/S}(-,B)$.
		\item Let $p_1,p_2\colon A\times_SA\morphism A$ be the two canonical projections. Then
		\begin{align*}
			s_A^\oast\colon s_A^*\Omega_A\morphism \Omega_{A\times_SA}\cong p_1^*\Omega_A\oplus p_2^*\Omega_A
		\end{align*}
		is the graph of an isomorphism $\gamma\colon p_1^*\Omega_A\isomorphism p_2^*\Omega_A$ (we explain in Claim~\reff{claim:weredone} what this is supposed to mean). Moreover, $(\Omega_A,\gamma)$ is a descent datum with respect to $\alpha$. Thus $\Omega_A\cong \alpha^*\omega_A$ for some vector bundle $\omega_A$ on $S$.
		\item Let $\pi\colon P\morphism S$ be a proper flat morphism whose geometric fibres $P_{\ov{s}}$ are projective spaces (i.e., isomorphic to $\IP_k^n$, where $k=\ov{{\KK(s)}}$ and $n$ may depend on $s$), then any morphism $f\colon P\morphism A$ of $S$-preschemes factors over $\pi$.
	\end{alphanumerate}
\end{thm}
\begin{proof}
	Throughout the proof and beyond, we use $A(T)$ as a shortcut for $\Hom_{\cat{PSch}/S}(T,A)$ and call this the \emph{$T$-valued points} of $A$. Also be warned that the proof will be quite lengthy.
	
	We start with \itememph{b}. In contrast to the lecture, I will use additive notation even though we do not yet know that the group structures are abelian (again, I ask for forgiveness), so we need to take care of the order of operations. Put $\phi=f-(f\circ0_A\circ\alpha)$, with ``$-$'' taken in the group $B(A)=\Hom_{\cat{PSch}/S}(A,B)$. Since $0_B\circ \alpha$ is the neutral element in the group $B(A)$, we get $-(f\circ 0_A\circ \alpha)=(-(f\circ 0_A))\circ \alpha$, where the first ``$-$'' denotes the inverse taken in $B(A)$ and the second the inverse in $B(S)$. Hence $\phi$ satisfies
	\begin{align*}
		\phi\circ 0_A=f\circ 0_A-(f\circ 0_A\circ \alpha)\circ 0_A=f\circ 0_A-(f\circ 0_A)\circ (\alpha\circ 0_A)&=f\circ 0_A-(f\circ 0_A)\circ \id_A\\
		&=0_B\;.
	\end{align*}
	We will show that this suffices for $\phi$ to be a morphism of group preschemes.
	
	Let $\psi\colon A\times_SA\morphism B$ be defined by $\psi=\phi\circ s_A-\phi\circ p_2-\phi\circ p_1$, so that if $a_1,a_2\in A(T)$ for some $S$-prescheme $T$ are $T$-valued points, then
	\begin{align}\lbl{eq:psi}
		\psi\circ(a_1,a_2)=\phi\circ(a_1+a_2)-\phi\circ a_2-\phi\circ a_1\;, 
	\end{align}
	where the ``+'' is taken in $A(T)$ and the two ``$-$'' in $B(T)$. What we need to show in order for $\phi$ to be a morphism of group schemes is that $\psi=0_B\circ p_1\circ\alpha=0_B\circ p_2\circ\alpha$, i.e., $\psi$ is the neutral element in $B(A\times_SA)$ and thus $\psi\circ(a_1,a_2)$ is the neutral element in $B(T)$ for any $a_1,a_2\in A(T)$.
	
	We apply Proposition~\reff{prop:rogodoty}\itememph{c} to $(\psi,p_2)\colon A\times_SA\morphism B\times_SA$, where $A\times_SA$ and $B\times_SA$ are considered $A$-preschemes via projection to the second factor. Also $B\times_SA$ is separated over $A$ (and even proper) since $B$ is proper over $S$, being an abelian scheme. This gives an open and closed subset $U\subseteq A$, describing the location where $\psi$ only depends on the second component. Then $U$ contains the image of $0_A\colon S\morphism A$. Indeed, if $o_A\in A(T)$ is a morphism factoring over $0_A$ -- i.e., $o_A$ is the neutral element in $A(T)$ -- then $o_B=\phi\circ o_A$ is the neutral element in $B(T)$ by our assumption that $\phi\circ 0_A=0_B$. Hence \eqreff{eq:psi} appears as
	\begin{align*}
		\psi\circ(a_1,o_A)=\phi\circ (a_1+o_A)-\phi\circ o_A-\phi\circ a_1=\phi\circ a_1 -o_B-\phi\circ a_1=o_B\;,
	\end{align*}
	which, as claimed, doesn't depend on $a_1$. But the image of $0_A$ (and thus $U$) intersects every fibre of $\alpha\colon A\morphism S$, hence $U$ contains every fibre because $U$ is open-closed and all fibres are connected
	(since so are the geometric ones as follows from Definition~\reff{def:abelianScheme}). Thus $U=A$.
	
	By Proposition~\reff{prop:rogodoty}\itememph{b}, $(\psi,p_2)$ factors over some morphism $\rho\colon A\morphism B\times_SA$. Hence we get $\psi=\rho_1 p_2$ for some morphism $\rho_1=p_1\rho\colon A\morphism B$ of $S$-preschemes. Then \eqreff{eq:psi} becomes
	\begin{align*}
		\rho_1\circ a_2=\phi\circ(a_1+a_2)-\phi\circ a_2-\phi\circ a_1\;,
	\end{align*}
	which independent of $a_1$. Then (with the above notation) we may as well take $a_1=o_A$ to be the neutral element in $A(T)$ to obtain $\rho_1\circ a_2=o_B$ for any $S$-prescheme $T$ and any $T$-valued point $a_2\in A(T)$. Thus $\psi$ is indeed the neutral element in $B(A\times_SA)$ and we're done.
	
	Part~\itememph{a} is an immediate consequence: The inversion $i_A\colon A\morphism A$ clearly satisfies $i_A\circ 0_A=0_A$, hence it is a morphism of group preschemes by the argument in the proof of \itememph{a}. This clearly implies that $A$ is abelian.
	
	Now part~\itememph{c}. We begin with some general considerations. Let $a_1,a_2\in A(S)$ and consider $\sigma=(a_1,a_2)\colon S\morphism A\times_SA$. Applying $\sigma^*$ to the isomorphism \eqreff{eq:p1p2Omega} gives
	\begin{align*}
		\sigma^*(p_1^\oast+p_2^\oast)\colon a_1^*\Omega_A\oplus a_2^*\Omega_A\isomorphism \sigma^*\Omega_{A\times_SA}\;.
	\end{align*}
	\begin{claim}\lbl{claim:technicalAFInverse}
		Consider $\sigma_1=(\id_A, a_2\alpha)\colon A\morphism A\times_SA$ and $\sigma_2=( a_1\alpha,\id_A)\colon A\morphism A\times_SA$. Then  $(\sigma^*\sigma_1^\oast,\sigma^*\sigma_2^\oast)$ is an inverse of the above isomorphism $\sigma^*(p_1^\oast+p_2^\oast)$. 
	\end{claim}
	It suffices to show that $(\sigma_1^\oast,\sigma_2^\oast)$ is an inverse of $p_1^\oast+p_2^\oast$ since this doesn't change when the pullback functor $\sigma^*$ is applied. Moreover, it suffices to check this on sections $\omega$ of $\Omega_{A\times_SA}$ which happen to have the form $\omega=p_1^\oast\vartheta$ or $\omega=p_2^\oast \vartheta$ for some section $\vartheta$ of $\Omega_A$, since $\Omega_{A\times_SA}\cong p_1^\oast\Omega_A\oplus p_2^\oast\Omega_A$ is generated by sections $\omega$ of this form. If $\omega=p_1^\oast\vartheta$, then
	\begin{align*}
		\sigma_1^\oast\omega=\sigma_1^\oast p_1^\oast\vartheta=(p_1 \sigma_1)^\oast\vartheta=\id_A^\oast\vartheta=\vartheta
	\end{align*}
	and
	\begin{align*}
		\sigma_2^\oast\omega=\sigma_2^\oast p_1^\oast=(p_2\sigma_2)^\oast=(a_1\alpha)^\oast \vartheta=\alpha^\oast a_1^\oast\vartheta=0\;,
	\end{align*}
	since $a_1^\oast\vartheta$ is a section of $\Omega_S=0$. Similar calculations apply to the case $\omega=p_2^\oast\vartheta$, thus proving Claim~\reff{claim:technicalAFInverse}. It follows that 
	\begin{align*}
		\sigma^*(s_A^\oast)\colon \sigma^*s_A^*\Omega_A\morphism \sigma^*\Omega_{A\times_SA}\cong a_1^*\Omega_A\oplus a_2 ^*\Omega_A
	\end{align*}
	is given by $(a_1^*t_{a_2}^\oast,a_2^*t_{a_1}^\oast)$, where $t_{a_2}=s_A(\id_A,a_2\alpha)$ and $t_{a_1}=s_A(a_1\alpha,\id_A)$ are the ``translations'' by $a_1$ and $a_2$ (we could have put $t_{a_1}=s_A(\id_A,a_1\alpha)$ as well since $A$ is abelian by \itememph{a}). Note that $s_A\sigma=a_1+a_2$ (with the sum taken in $A(S)$), hence $\sigma^*s_A^*\Omega_A\cong (a_1+a_2)^*\Omega_A$. Also the translations $t_{a_1},t_{a_2}\colon A\morphism A$ are automorphisms (with inverses $t_{-a_1}$ and $t_{-a_2}$ respectively) and satisfy $t_{a_1}a_2=a_1+a_2=t_{a_2}a_1$. Hence both components of $\sigma^*(s_A^\oast)$ are isomorphisms
	\begin{align}\lbl{eq:a1a2}
		a_1^*t_{a_2}^\oast\colon (a_1+a_2)^*\Omega_A\isomorphism a_1^*\Omega_A\quad\text{and}\quad a_2^*t_{a_1}^\oast\colon (a_1+a_2)^*\Omega_A\isomorphism a_2^*\Omega_A\;.
	\end{align}
	\begin{claim}\lbl{claim:weredone}
		This shows that the image $\Gamma_A$ of 
		\begin{align*}
			s_A^\oast\colon s_A^*\Omega_A\morphism \Omega_{A\times_SA}\cong p_1^*\Omega_A\oplus p_2^*\Omega_A
		\end{align*}
		is indeed the graph of an isomorphism $\gamma\colon p_1^*\Omega_A\isomorphism p_2^*\Omega_A$. That is, $\Gamma_A$ consists of all sections of the form $(\omega,\gamma(\omega))$, where $\omega$ is a section of $p_1^*\Omega_A$ and $\gamma$ is the mysterious isomorphism.
	\end{claim}
	To prove Claim~\reff{claim:weredone}, we need to show that the $i\ordinalth$ component of $s_A^\oast$ induces an isomorphism $s_A^*\Omega_A\isomorphism p_i^*\Omega_A$ for $i\in\{1,2\}$ -- which looks pretty similar to \eqreff{eq:a1a2}. Since $\Omega_A$ and thus also $p_i^*\Omega_A$ are vector bundles (by Corollary~\reff{cor:groupSchemesSmooth}), we may check whether $s_A^*\Omega_A\morphism p_i^*\Omega_A$ is an isomorphism after base change to the geometric fibres $A_{\ov{s}}$ for $s\in S$ (by Corollary~\reff{cor:tensorTrick} -- here we also use that a locally split monomorphism between vector bundles of the same rank is automatically an isomorphism). Hence we may assume that $S=\Spec k$ is the spectrum of an algebraically closed field $k$.
	
	Now \eqreff{eq:a1a2} can be effectively used. The closed points of $A\times_kA$ are dense, as $A\times_kA$ is a variety (by Remark~\reff{rem:productsOfVarieties}), hence Jacobson, so \cite[Fact~2.4.1\itememph{b}]{alggeo1} applies. Then Lemma~\reff{lem:tensorTrick} shows that it is sufficient  to show that $s_A^*\Omega_A(x)\morphism p_i^*\Omega_A(x)$ is an isomorphism for every closed point $x\in A\times_kA$ (using the notation from Convention~\reff{con:AON}\itememph{c}). But closed points of $A\times_kA$ correspond to $k$-valued points $(A\times_kA)(k)\cong A(k)\times A(k)$ by \eqreff{eq:ZariskiTangentSpace} and the universal property of fibre products, hence it suffices to show that $\sigma^*s_A^*\Omega_A\morphism \sigma^*p_i^*\Omega_A$ is an isomorphism for all $\sigma=(a_1,a_2)$ with $a_1,a_2\in A(k)$. Since $s_A\sigma=a_1+a_2$ and $p_i\sigma=a_i$, this is nothing else but \eqreff{eq:a1a2}. Thus we proved Claim~\reff{claim:weredone}.
	
	It remains to prove that $(\Omega_A,\gamma)$ is a descent datum. Using the formalism from Remark~\reff{rem:equivalentDefOfDescentData}, we put $\gamma_{a,b}=(a,b)^*\gamma$ when $a,b\in A(T)$ are $T$-valued points for some $S$-prescheme $T$. Then what we need to show is $\gamma_{a_1,a_3}=\gamma_{a_2,a_3}\gamma_{a_1,a_2}$ for every triple of $T$-valued points $a_1,a_2,a_3\in A(T)$. Replacing $A$ by $A\times_ST$ (which is fine since it is easy to see that $\Gamma_{A\times_ST}$ is obtained from $\Gamma_A$ via base change) we may assume $S=T$, so that \eqreff{eq:a1a2} is applicable. Let $\vartheta$ be a section of $(a_1+a_2)^*\Omega_A$. By \eqreff{eq:a1a2}, the isomorphism $\gamma_{a_1,a_2}$ sends $a_1^*t_{a_2}^\oast\vartheta$ to $a_2^*t_{a_1}^\oast\vartheta$. To simplify calculations, put $\eta=t_{a_1+a_2}^\oast\vartheta$, which is now a section of $0_A^*\Omega_A$. Then $a_1^*t_{a_2}^\oast\vartheta=a_1^*t_{-a_1}^\oast\eta$ is sent to $a_2^*t_{-a_2}^\oast\eta$ under $\gamma_{a_1,a_2}$. By symmetry arguments, $\gamma_{a_2,a_3}$ sends $a_2^*t_{-a_2}^\oast\eta$ to $a_3^*t_{-a_3}^\oast\eta$, which is the same as the image of $a_1^*t_{-a_1}^\oast\eta$ under $\gamma_{a_1,a_3}$. Thus $\gamma_{a_1,a_3}=\gamma_{a_2,a_3}\gamma_{a_1,a_2}$, as claimed.
	
	Now $\alpha\colon A\morphism S$ is an fpqc morphism (surjectivity follows from the fact that $0_A$ is always a split of $\alpha$), Theorem~\reff{thm:faithfullyFlatDescent} shows that $\Omega_A$ can be descended to a quasi-coherent $\Oo_S$-module $\omega_A$. By Lemma~\reff{lem:descentIsALineBundleAgain}, $\omega_A$ is a vector bundle of the same rank as $\Omega_A$.
	
	Finally, part~\itememph{d}. By Proposition~\reff{prop:rogodoty}\itememph{b}, we only need to show that every geometric fibre $P_{\ov{s}}$ of $\pi$ is mapped to a single point in $A_{\ov{s}}$. This reduces the question to the case where $S=\Spec k$ is the spectrum of an algebraically closed field and $P\cong\IP_k^n$. Note that $f\colon P\morphism A$ is necessarily proper since $\alpha f=\pi$ is proper and $\alpha$ is separated, so \cite[Proposition~2.4.1]{alggeo2} can be invoked. Hence every closed point of $\IP_k^n$ is sent to a closed point. If we can show that actually \emph{all} closed points of $\IP_k^n$ are sent to the same closed point in $A$, then we're done, since the closed points are dense in $\IP_k^n$. Moreover, any two closed points in $\IP_k^n$ are contained in a projective line $\IP_k^1$, hence in order to show that they are mapped to the same point, it suffices to consider the special case $P=\IP_k^1$.
	
	Hence the global sections of $\Omega_P\cong \Oo(-2)$ vanish by \cite[Theorem~2\itememph{a}]{alggeo2}. In constrast, \itememph{c} shows that $\Omega_A\cong\alpha^*\omega_A\cong \Oo_A^{\oplus\dim A}$, since $\omega_A$ is necessarily trivial (as is any vector bundle on $\Spec k$). Therefore $\Omega_A$ -- and thus $f^*\Omega_A$ as well -- is generated by global sections. This shows that $f^\oast\colon f^*\Omega_A\morphism \Omega_P$ vanishes identically.
	
	In particular, we deduce that $\d_P(f^*\lambda)=0$ whenever $\lambda$ is a section of $\Oo_A$ on some open subset. That is, $f^*\lambda$ -- which can be seen as a rational function in $k(X,Y)$ since it is a section of the structure sheaf $\Oo_P$ -- has vanishing partial derivatives with respect to $X$ and $Y$, since that's how the universal derivation $\d_P\colon \Oo_P\morphism\Omega_P\cong \Oo(-2)$ works (cf.\ \cite[85]{alggeo2}). Let's distinguish two cases.
	
	\emph{Case 1.} If $\cha k=0$, then vanishing of the partial differentials of $f^*\lambda$ implies $f^*\lambda\in k$. It's now an easy argument to see that $f$ factors over $\Spec k$. Indeed, if $U\subseteq A$ and $W\subseteq P$ are affine opens such that $f(W)\subseteq U$, then $f|_W\colon W\morphism U$ factors over $\Spec k$ as can be seen from the adjunction $\Hom_{\cat{PSch}/k}(W,U)\cong \Hom_{\cat{Alg}(k)}(\Oo_P(W),\Oo_A(U))$. Since $P$ can be covered by such $W$, we see that $f$ factors over $k$ as well.
	
	\emph{Case 2.} If $\cha k=p>0$, then $\d_P(f^*\lambda)=0$ only shows that $f^*\lambda$ is a $p\ordinalth$ power. The idea to circumvent this is to prove iteratively that $f^*\lambda$ is a $(p^n)\ordinalth$ power for every $n\geq 1$. If this could be achieved, then $f^*\lambda\in k$ will follow again and the proof can be completed as in Case~1. To carry out the iterative argument, we need to investigate the \emph{Frobenius} as a morphism of preschemes.
	
	Let $X$ be any $\IF_p$-prescheme. Let $\Frob_X\colon X\morphism X$ be the \emph{absolute Frobenius} which acts as identity on points and by $\Frob_X^*(\lambda)=\lambda^p$ on sections of the structure sheaf. However, if $X$ is a prescheme over $k$ where $k$ is of characteristic $p$, then $\Frob_X$ will, in general, not be a morphism of $k$-preschemes (since $k$ isn't fixed by the Frobenius unless $k=\IF_p$). If $k$ is algebraically closed (or just perfect; what we need is that $\Frob_k$ is invertible), we may consider $X^{(p)}$, which is the same prescheme as $X$ but equipped with the structure morphism
	\begin{align*}
		X\morphism[\xi]\Spec k\xrightarrow{\Frob_k^{-1}}\Spec k
	\end{align*}
	(so only the structures of $X$ and $X^{(p)}$ as $k$-preschemes differ). Then $\Frob_X\colon X\morphism X^{(p)}$ is a  morphism of $k$-preschemes.
	
	If $f\colon P\morphism A$ is as before, then the fact that all $f^*\lambda$ are $p\ordinalth$ powers shows that $f=f_1\Frob_P$, where $f_1\colon P^{(p)}\morphism A$ is  a new morphism of $k$-preschemes. However, we still have $P^{(p)}\cong \IP_k^1$, hence the induction assumption may be applied to $f_1$, showing that $f^*\lambda$ is a $(p^{n+1})\ordinalth$ power if $f_1^*\lambda$ is a $(p^n)\ordinalth$. We're done.
\end{proof}
\begin{rem}
	\begin{alphanumerate}
		\item Theorem~\reff{thm:abelianSchemes}\itememph{b} still holds if $B$ is only a group prescheme which is separated over $S$ (and in fact, that's all we needed in the proof).
		\item Milne gives another proof of part \itememph{d} in \cite{cornell1986arithmetic}.
	\end{alphanumerate}
\end{rem}

\section{Line bundles: Applications of the theorem of the cube}
\subsection{The theorem of the cube II (now featuring a real cube!)}
\begin{thm}[Theorem of the cube II]\lbl{thm:cube2}
	Let $\alpha\colon A\morphism S$ be an abelian scheme over $S$ (which is locally noetherian) and let $\Ll$ be a line bundle on $A$. For a subset $I\subseteq\{1,2,3\}$, let
	\begin{align*}
		s_I\colon A\times_SA\times_SA\morphism A
	\end{align*}
	denote the sum $\sum_{i\in I}p_i$ taken in the group $\Hom_{\cat{PSch}/S}(A\times_SA\times_SA,A)$. Then there is a unique isomorphism
	\begin{align}\lbl{eq:desiredIso}
		\bigotimes_{\#I\text{ even}}s_I^*\Ll\isomorphism \bigotimes_{\#I\text{ odd}}s_I^*\Ll
	\end{align}
	whose pullback under $( 0_A, 0_A, 0_A)\colon S\morphism A$ is the identity on $ 0_A^*\Ll^{\otimes 4}$.
\end{thm}
\begin{proof}
	Let $p_1,p_2\colon A\times_SA\morphism A$ be the canonical projections and let $q\colon A\times_SA\morphism S$ denote $\alpha p_1=\alpha p_2$. We write $\pi_x$, $\pi_y$, $\pi_z$ for the three embeddings of $A\times_SA$ into $A\times_SA\times_SA$ as a ``coordinate plane''. That is, 
	\begin{align*}
		\pi_x=( 0_Aq,\id_A p_1,\id_Ap_2)\;,\quad\pi_y=(\id_A p_1, 0_Aq,\id_Ap_2)\;,\quad\text{and}\quad \pi_z=(\id_A p_1,\id_Ap_2, 0_Aq)\;.
	\end{align*}
	\begin{claim}\lbl{claim:Levenodd}
		If $\pi\in\{\pi_x,\pi_y,\pi_z\}$, then $\pi^*\Ll_{\mathrm{ev}}\cong \pi^*\Ll_{\mathrm{odd}}$, where $\Ll_{\mathrm{ev}}$ and $\Ll_{\mathrm{odd}}$ are shortcut notations for the left- resp.\ the right-hand side of the desired isomorphism \eqreff{eq:desiredIso}.
	\end{claim}
	It suffices to prove Claim~\reff{claim:Levenodd} for $\pi=\pi_z$, the other cases being similar. Note that if $I\subseteq\{1,2\}$, then $s_{I\cup\{3\}}\pi_z=s_I\pi_z$. Therefore, if $I\subseteq \{1,2,3\}$ has odd cardinality $\#I$, then the tensor factor $\pi_z^*s_I^*\Ll$ in $\pi_z^*\Ll_{\mathrm{odd}}$ is isomorphic to the tensor factor $\pi_z^*s_J^*\Ll$ in $\pi_z^*\Ll_{\mathrm{ev}}$, where $J=I\setminus \{3\}$ if $3\in I$ resp.\ $J=I\cup\{3\}$ if $3\notin I$. This proves the claim.
	
	In particular, Claim~\reff{claim:Levenodd} shows that the line bundle
	\begin{align*}
		\Mm=\bigotimes_{I\subseteq \{1,2,3\}}s_I^*\Ll^{\otimes (-1)^{\# I}}\cong \Ll_{\mathrm{ev}}\otimes\Ll_{\mathrm{odd}}^{\otimes -1}
	\end{align*}
	(with the tensor product taken over $\Oo_{A\times_SA\times_SA}$) becomes trivial after pulling back along the coordinate plane embeddings $\pi_x$, $\pi_y$, or $\pi_z$.
	
	Now apply Theorem~\reff{thm:cube} to the following pullback diagram
	\begin{diagram*}
		\object{0,1.5}{$A\times_SA\times_SA$}[F];
		\object{0,0}{$A\times_SA$}[X];
		\object{3,1.5}{$A\times_SA$}[Y];
		\object{3,0}{$A$}[S];
		\pullback{1.5,0.75};
		\scriptsize
		\arrow{F}{X}[left][$p_{1,2}$];
		\arrow{F}{Y}[above][$p_{1,3}$];
		\arrow{X}{S}[above][$p_1$];
		\arrow{Y}{S}[left][$p_1$];
		\arrow[dotted,->,bend left]{S}{X}[below][$(\id_A,0_A\alpha)$];
		\arrow[dotted,->,bend right]{S}{Y}[right][$(\id_A,0_A\alpha)$];
		%\arrow[dotted,->,bend left]{X}{F}[left][$\pi_z$];
		%\arrow[dotted,->,bend right]{Y}{F}[above][$\pi_y$];
	\end{diagram*}
	The isomorphism $(A\times_SA)\times_A(A\times_SA)\cong A\times_SA\times_SA$ follows from abstract nonsense. Also the morphisms $(\id_X,\sigma_Y\xi)$ and $(\sigma_X\upsilon,\id_Y)$ from Theorem~\reff{thm:cube} correspond to $\pi_z$ and $\pi_y$ respectively (and that's fine since we just showed that $\pi_z^*\Mm$ and $\pi_y^*\Mm$ are trivial). What we obtain is an open and closed subprescheme $Z\subseteq A$ characterized by the universal property from Theorem~\reff{thm:seesaw}.
	\begin{claim}\lbl{claim:Z=A}
		We actually have $Z=A$.
	\end{claim}
	The trick will be the same as in the proof of Theorem~\reff{thm:abelianSchemes}\itememph{b}: We will show that $Z$ contains the image of $0_A\colon S\morphism A$, so that $Z$ intersects any fibre $A_s=\alpha^{-1}\{s\}$ of $\alpha$. But then $\alpha^{-1}\{s\}\subseteq Z$ for all $s\in S$ since the fibres are connected (because the geometric fibres are irreducible by Definition~\reff{def:abelianScheme}, hence connected) and $Z$ is open-closed.
	
	To check that $0_A$ indeed factors over $Z$, we use the universal property of $Z$. It's easy to check that
	\begin{diagram*}
		\object{0,1.5}{$A\times_SA$}[F];
		\object{0,0}{$A\times_SA\times_SA$}[X];
		\object{3,1.5}{$S$}[Y];
		\object{3,0}{$A$}[S];
		\pullback{1.5,0.75};
		\scriptsize
		\arrow{F}{X}[left][$\pi_x$];
		\arrow{F}{Y}[above][$q$];
		\arrow{X}{S}[above][$p_1$];
		\arrow{Y}{S}[right][$0_A$];
	\end{diagram*}
	is a pullback diagram. Since $\pi_x^*\Mm$ is trivial, the universal property from Theorem~\reff{thm:seesaw} shows that $0_A$ factors over $Z$, thus proving Claim~\reff{claim:Z=A}.
	
	Applying the universal property again, now with our new knowledge that $Z=A$, we obtain $\Mm\cong p_1^*\Mm_0$ for some line bundle $\Mm_0$ on $A$. Consider $t\colon A\morphism A\times_SA\times_SA$ given by $t=(\id_A,0_A\alpha,0_A\alpha)$. Then $p_1t=\id_A$, hence $\Mm_0\cong t^*\Mm$. But $t$ factors over $\pi_z$, hence $t^*\Mm$ is a pullback of the trivial line bundle $\pi_z^*\Mm$. Hence $\Mm_0$ is trivial and thus the same is true for $\Mm\cong p_1^*\Mm_0$.
\end{proof}
\begin{cor}\lbl{cor:cube2}
	Let $f_1,f_2,f_3\colon T\morphism A$ be three morphisms of $S$-preschemes, and for $I\subseteq \{1,2,3\}$ we denote $f_I=\sum_{i\in I}f_i$ (the sum being taken using the group prescheme structure on $A$). Then any line bundle $\Ll$ on $A$ there is a canonical isomorphism
	\begin{align}\lbl{eq:desiredIso2}
		\bigotimes_{\#I\text{ even}}f_I^*\Ll\isomorphism \bigotimes_{\#I\text{ odd}}f_I^*\Ll\;.
	\end{align}
\end{cor}
\begin{proof}
	This is a pull-back of the isomorphism from Theorem~\reff{thm:cube2}.
\end{proof}
\begin{cor}\lbl{cor:nL}
	Let $\Ll$ be a line bundle on an abelian scheme $A$ and let $[n]\colon A\morphism A$ denote multiplication by $n\in \IZ$ on $A$, so that $[1]=\id_A$ and $[n]=[n-1]+\id_A$. Then there are canonical isomorphisms
	\begin{align*}
		[n]^*\Ll\cong \Ll^{\otimes \frac{n^2+n}{2}}\otimes_{\Oo_A}[-1]^*\Ll^{\otimes\frac{n^2-n}{2}}\otimes_{\Oo_A}[0]^*\Ll^{\otimes(1-n^2)}\;.
	\end{align*}
\end{cor}
\begin{proof}[Sketch of a proof]
	For $n\in\{-1,0,1\}$ this is obvious. For arbitrary $n$, taking $f_1=[n-2]$ and $f_2=f_3=1_A=\id_A$ in Corollary~\reff{cor:cube2} shows
	\begin{align*}
		[n]^*\Ll\cong [n-1]^*\Ll^{\otimes 2}\otimes_{\Oo_A}[n-2]^*\Ll^{\otimes-1}\otimes_{\Oo_A} \Ll^{\otimes -2}\otimes_{\Oo_A}[2]^*\Ll\;,
	\end{align*}
	which allows for a proof by induction on $n$ (where the $n=2$ case follows from the above isomorphism and the $n=1$ case).
\end{proof}
\begin{cor}\lbl{cor:thmOfTheSquare}
	Suppose that $\Pic(S)$ is trivial (e.g., $S$ is the spectrum of a field or a PID\footnote{Actually, $\Pic(\Spec A)$ is trivial whenever $A$ is a factorial domain, cf.\ \cite[\stackstag{0BCH}]{stacks-project}.}). For every $S$-valued point $a\in A(S)$ let $t_a=\id_A+a\alpha$ denote the translation by $a$. Then for every $x,y\in A(S)$ there is a (non-canonical) isomorphism
	\begin{align*}
		t_{x+y}^*\Ll\otimes_{\Oo_A}\Ll\cong t_x^*\Ll\otimes_{\Oo_A}t_y^*\Ll\;.
	\end{align*}
\end{cor}
\begin{proof}
	Apply Corollary~\reff{cor:cube2} with $T=A$, $f_1=\id_A$, $f_2=x\alpha$, and $f_3=y\alpha$ to obtain an isomorphism (which, for now, is still canonical)
	\begin{align*}
		 \alpha^*0_A^*\Ll\otimes_{\Oo_A}t_x^*\Ll\otimes_{\Oo_A}t_y^*\Ll\otimes_{\Oo_A}\otimes_{\Oo_A}\alpha^*(x+y)^*\Ll\cong \id_A^*\Ll\otimes_{\Oo_A}\alpha^*x^*\Ll\otimes_{\Oo_A}\alpha^*y^*\Ll\otimes_{\Oo_A}t_{x+y}^*\Ll\;.
	\end{align*}
	Since $\Pic(S)$ vanishes, all factors containing an $\alpha^*$ are trivial. Also $\id_A^*\Ll\cong \Ll$, so the desired (non-canonical) isomorphism follows.
\end{proof}
\subsection{Interlude: A technical lemma on group schemes over fields}
The following lemma wasn't in the lecture, but we will need it to give a full proof of the upcoming Proposition~\reff{prop:ampleFiniteStabilizer}. Also, to avoid technical complications, I'm not going to prove it under the most general assumptions. If you are interested in more general results, have a look at \cite[\stackstag{047J}]{stacks-project}.
\begin{lem}\lbl{lem:interlude}
	Let $\gamma\colon G\morphism\Spec k$ be a group prescheme (not necessarily commutative) of finite type over an algebraically closed field $k$, with group operation $m_G\colon G\times_kG\morphism G$, unit $1_G\colon \Spec k\morphism G$, and inversion $i_G\colon G\morphism G$.
	\begin{alphanumerate}
		\item The connected components of $G$ are precisely its irreducible components.
		\item If $Z\subseteq G$ is the connected component containing the closed point $e\in G$ corresponding to $1_G\colon \Spec k\morphism G$ via \eqreff{eq:ZariskiTangentSpace}, then $Z$ is a subgroup prescheme. Moreover, its reduction $Z^\red\subseteq Z$ is an integral subgroup prescheme.
	\end{alphanumerate}
\end{lem}
\begin{proof}
	Part~\itememph{a}. Since all irreducible components are connected, it suffices to show that if $Z_1\neq Z_2$ are irreducible components, then $Z_1\cap Z_2=\emptyset$. So assume that $Z_1$ and $Z_2$ are irreducible components such that $Z_1\cap Z_2$ is non-empty open. Since $G$ is quasi-compact (having finite type over $k$), we find a closed point $z\in Z_1\cap Z_2$ by \cite[Proposition~2.1.1]{alggeo1}. By \eqreff{eq:ZariskiTangentSpace}, we may regard $z$ as a morphism $z\colon \Spec k\morphism G$. Then the translation by $z$, i.e., $t_z=m(\id_G,z\gamma)$, is an automorphism of $G$ and sends the unit element $e$ to $z$. Hence we may w.l.o.g.\ assume $e=z$.
	
	Equip $Z_1$ and $Z_2$ with their canonical reduced closed subprescheme structures, so that both become integral, hence varieties over $k$. Since $k$ is algebraically closed, $Z_1\times_kZ_2\subseteq G\times_kG$ is a variety again (see Remark~\reff{rem:productsOfVarieties}). In particular, $Z_1\times_kZ_2$ is irreducible. Hence $m_G(Z_1\times_kZ_2)$ is irreducible, as is the image of any irreducible set under a continuous map. Therefore, $m_G(Z_1\times_kZ_2)$ must be contained in an irreducible component of $G$. But since $e\in Z_1\cap Z_2$, we see that $m_G(Z_1\times_kZ_2)$ contains both $Z_1$ and $Z_2$. This only leaves $Z_1=Z_2$, as claimed.
	
	Part~\itememph{b}. By \itememph{a}, $Z$ is an irreducible component as well. Since $G$ is noetherian (being of finite type over $k$), it has only finitely many connected components, hence $Z$ is open-closed. We equip $Z$ with its canonical structure of an open prescheme of $G$. While $Z$ is irreducible, might not be reduced, but its reduction $Z^\red$ surely is, so $Z^\red$ is a variety over $k$. Hence so is $Z^\red\times_k Z^\red$ (see Remark~\reff{rem:productsOfVarieties}). In particular, $Z^\red\times_k Z^\red$ is irreducible, so the same is true for $Z\times_kZ$ (since these guys have the same underlying topological space).
	
	Therefore, $m_G(Z\times_kZ)$ is connected, hence it must be contained in a connected component of $G$. But $m_G(Z\times_kZ)$ intersects $Z$ (as $e$ is contained in both), hence $m_G$ restricts to a map $m_Z=m_G|_{Z\times_kZ}\colon Z\times_kZ\morphism Z$ (of topological spaces). Since $Z$ is an open subprescheme of $G$, we see that $m_Z$ is also a morphism of preschemes. Similar reasoning applies to $i_Z=i_G|_Z$, proving that $Z$ is ineed a subgroup prescheme.
	
	To prove the same for $Z^\red$, note that $m_{Z^\red}=m_Z|_{Z^\red\times_kZ^\red}$ is the same as $m_Z$ as a map of topological spaces. However, $Z^\red\times_k Z^\red$ is already reduced, hence $m_{Z^\red}$ factors over $Z^\red$ as a morphism of preschemes. A similar argument for $i_{Z^\red}=i_Z|_{Z^\red}$ shows that $Z^\red$ is indeed an integral subgroup prescheme of $Z$.
\end{proof}

\subsection{Ample line bundles on abelian varieties}
We recall some basic facts about  ampleness and some relatively basic facts about relative ampleness.

\begin{prop}\lbl{prop:relativelyAmple}
	Let $f\colon X\morphism Y=\Spec A$ be a proper morphism with $A$ a noetherian ring. For a line bundle $\Ll$ on $X$, the following are equivalent.
	\begin{alphanumerate}
		\item There is a positive integer $k$ such that $\Ll^{\otimes k}$ is generated by global sections $\lambda_1,\ldots,\lambda_n\in\Ll(X)$ such that $X\setminus V(\lambda_i)$ is affine for all $i$.
		\item There is a positive integer $k$ such that $\Ll^{\otimes k}$ is generated by $m$ global sections and the resulting morphism $\iota\colon X\morphism \IP_Y^{m-1}$ is affine.
		\item Same as \itememph{b}, but $\iota$ is required to be a closed embedding.
	\end{alphanumerate}
	If $\Ll$ satisfies these conditions $\Mm$ is another line bundle such that  $\Mm^{\otimes n}$ is generated by its global sections for all $n\gge 0$, then $\Ll\otimes_{\Oo_X}\Mm$ also satisfies these conditions.
\end{prop}
\begin{proof}
	I have put the proof where it belongs-- and this place is not here (as Professor Franke openly admits), but in the Algebraic Geometry II lecture notes. Equivalence of \itememph{a}, \itememph{b}, and \itememph{c} then follows from \cite[Theorem~6 and Lemma~2.5.2]{alggeo2}, and all of them are equivalent to $\Ll$ being an ample line bundle on $X$.
	
	To prove the additional assertion, it's customary to use the definition of ampleness from \cite[Definition~2.2.1]{alggeo2}. If $\Ff$ is any locally finitely generated $\Oo_X$-module, then $\Ff\otimes_{\Oo_X}\Ll^{\otimes n}$ is generated by global sections for all $n\gge 0$. If $n$ is large enough, then $\Mm^{\otimes n}$ is generated by global sections as well, hence we see that $\Ff\otimes_{\Oo_X}(\Ll\otimes_{\Oo_X} \Mm)^{\otimes n}\cong \Ff\otimes_{\Oo_X}\Ll^{\otimes n}\otimes_{\Oo_X}\Mm^{\otimes n}$ is generated by global sections too for $n\gge 0$.
\end{proof}
\begin{defi}\lbl{def:relativelyAmple}
	Let $f\colon X\morphism Y$ be a proper morphism between noetherian preschemes. Then a line bundle $\Ll$ is called \defemph{relatively ample} if it satisfies the following equivalent conditions.
	\begin{alphanumerate}
		\item For every affine open $U\subseteq Y$, the restriction $\Ll|_{f^{-1}(U)}$ satisfies the equivalent conditions from Proposition~\reff{prop:relativelyAmple}.
		\item $Y$ may be covered by affine open $U$ for which this is the case.
	\end{alphanumerate}
\end{defi}
\begin{proof}[Proof of equivalence]
	This is actually true under weaker assumptions, cf.\ \cite[\stackstag{01VJ}]{stacks-project}. However, in our situation there is a quick-and-dirty argument: Since \itememph{a} $\Rightarrow$ \itememph{b} is trivial, let's assume \itememph{b}. Then \cite[Theorem~6\itememph{a} and \itememph{d}]{alggeo2} together with $X$ being quasi-compact shows that $R^pf_*(\Ff\otimes_{\Oo_X}\Ll^{\otimes n})=0$ for every coherent $\Oo_X$-module $\Ff$ and all $n\gge 0$. This easily implies \itememph{a} via applying the cited theorem backwards.
\end{proof}
We will now use the language of Weil divisors to deal with line bundles. I have put some basic facts on this topic in Subsection~\reff{subsec:How2Divisors}. However, I'm sure that \cite[\stackstag{01WO}]{stacks-project} or \cite[Section~II.6]{hartshorne} provide way better introductions, so be sure to have a look into them!

If $\alpha\colon A\morphism\Spec k$ is an \defemph{abelian variety} over a field $k$ (i.e., an abelian scheme over $\Spec k$), then the group of $k$-valued points $A(k)$ acts on $A$ via the translations $t_a=\id_A+a\alpha$ for all $a\in A(k)$. This action is clearly free and transitive on the set of $k$-valued points $A(k)$ itself (which may be regarded as a subset of $A$ -- and if $k$ is algebraically closed, $A(k)$ is actually the subset of all closed points by \eqreff{eq:ZariskiTangentSpace}), but this says nothing about points of $A$ which do not belong to $A(k)$.

Note that all translations $t_a$ are automorphisms of $A$ (since $t_{-a}$ defines an inverse). Hence the pullback of a Weil divisor $D=\sum_Zn_ZZ$ defined as 
\begin{align*}
	t_a^*D=\sum_Zn_Zt_a^{-1}(Z)
\end{align*}
makes sense, and we obtain an action of $A(k)$ on $\Div(A)$ via pullbacks $t_a^*$. This action is clearly compatible with the action of $A(k)$ on $\Pic(A)$ via pullbacks $t_a^*$. In particular, if $D$ is a divisor and $\Ll=\Oo_A(D)$, we obtain \emph{stabilizer subgroups}
\begin{align*}
	A(k)_D=\left\{a\in A(k)\st t_a^*D=D\right\}\quad\text{and}\quad A(k)_\Ll=\left\{a\in A(k)\st t_a^*\Ll\cong\Ll\right\}\;,
\end{align*}
which clearly satisfy $A(k)_D\subseteq A(k)_\Ll$.
\begin{prop}\lbl{prop:ampleFiniteStabilizer}
	Let $\alpha\colon A\morphism \Spec k$ be an abelian variety over an algebraically closed\footnote{In a subsequent lecture, Professor Franke incidentally claimed Proposition~\reff{prop:ampleFiniteStabilizer} was true for arbitrary $k$. However, I suspect this requires stronger finiteness conditions in \itememph{a}, \itememph{b}, and \itememph{c}, since in general, $A(k)$ doesn't see too much of the closed points of $A$. Anyway, the algebraically closed case will suffice for our purposes.} field $k$. Let $D=\sum_in_iZ_i$ be an effective divisor on $A$ in the sense of Definition~\reff{def:divisors}\itememph{c} and put $\Ll=\Oo_A(D)$. Then the following conditions are equivalent.
	\begin{alphanumerate}
		\item The stabilizer subgroup $A(k)_D\subseteq A(k)$ is finite.
		\item The intersection $\bigcap_{n_i\neq 0}A(k)_{Z_i}\subseteq A(k)$ is finite.
		\item The stabilizer subgroup $A(k)_\Ll\subseteq A(k)$ is finite.
		\item The line bundle $\Ll^{\otimes 2}\cong \Oo_A(2D)$ is generated by global sections $\lambda_0,\ldots,\lambda_N$, and the induced morphism $\iota\colon A\morphism\IP_k^N$ is finite.
		\item $\Ll$ is ample.
	\end{alphanumerate}
\end{prop}
\begin{proof}
	The implication \itememph{d} $\Rightarrow$ \itememph{e} follows from Proposition~\reff{prop:relativelyAmple}. For \itememph{e} $\Rightarrow$ \itememph{c}, Professor Franke referred to \cite[\S6 Application~1]{mumford1974abelian}, but this didn't quite convince me for the reason that Mumford works with algebraic varieties rather than schemes (and this really makes things easier at some points). So here is a worked-out adaptation of Mumford's proof. Brace yourselves, for this is going to take \emph{long}!
	
	\emph{Proof of \itememph{e} $\Rightarrow$ \itememph{c}.}
	\emph{Step 1.} We consider the projection to the first component $p_1\colon A\times_kA\morphism A$ as an $A$-prescheme. Moreover, $p_1$ is flat and proper, since it is a base change of $\alpha\colon A\morphism\Spec k$ for which this is the case. Finally, we claim that the geometric fibres of $p_1$ are integral. Indeed, if $s\in A$ and $K=\ov{\KK(s)}$, then the geometric fibre over $s$ is given by the base change $A_K=A\times_k\Spec K$. But since $A$ is integral over the algebraically closed field $k$, \cite[\stackstag{020I} and \stackstag{020J}]{stacks-project} show that $A_K$ is integral again.
	
	Therefore, all assumptions of the seesaw theorem (Theorem~\reff{thm:seesaw}) are fulfilled. Applying it to the line bundle $\Mm=s_A^*\Ll\otimes p_2^*\Ll^{\otimes-1}$ on $A\times_kA$ gives some closed subprescheme $Z\subseteq X$.
	
	\emph{Step 2.} We will show that $Z$ is a subgroup prescheme of $A$! To prove this claim, the first step will be to show that $A(k)\cap Z$ is a subgroup of $A(k)$. Let $a\in A(k)$ be any $k$-valued point (or equivalently a closed point of $A$ by \eqreff{eq:ZariskiTangentSpace}), then we get a pullback diagram
	\begin{diagram*}
		\object{0,1.5}{$A$}[F];
		\object{0,0}{$A\times_kA$}[X];
		\object{2.5,1.5}{$\Spec k$}[Y];
		\object{2.5,0}{$A$}[S];
		\pullback{1.25,0.75};
		\scriptsize
		\arrow{F}{X}[left][$(a\alpha,\id_A)$];
		\arrow{F}{Y}[above][$\alpha$];
		\arrow{X}{S}[above][$p_1$];
		\arrow{Y}{S}[right][$a$];
	\end{diagram*}
	in the category $\cat{PSch}/k$ (proving that this is indeed a pullback is just pushing around some universal properties). Using the universal property of $Z$ (see Theorem~\reff{thm:seesaw}), we see that $a$ lies in $Z$ iff $(a\alpha,\id_A)^*\Mm$ is the pullback along $\alpha$ of some line bundle on $\Spec k$ . But $\Pic(\Spec k)$ is trivial, so $a\in Z$ turns out to be equivalent to
	\begin{align*}
		(a\alpha,\id_A)^*\Mm=(a\alpha,\id_A)^*s_A^*\Ll\otimes_{\Oo_A}(a\alpha,\id_A)^*p_2^*\Ll^{\otimes-1}\cong t_a^*\Ll\otimes_{\Oo_A}\Ll^{\otimes -1}
	\end{align*}
	 being a trivial line bundle. That is, $A(k)\cap Z$ is precisely the stabilizer subgroup $A(k)_\Ll$!
	 
	 For $Z$ to be a subgroup prescheme, we need to show that $s_Z=s_A|_{Z\times_kZ}\colon Z\times_kZ\morphism A$ factors over $Z$. By the universal property of $Z$, this is the case iff $S=Z\times_kZ$ and $X=(A\times_kA)\times_AS$ satisfy the conditions from Proposition~\reff{prop:LineBundlesFromTheBase}\itememph{b}.\footnote{Note that Proposition~\reff{prop:LineBundlesFromTheBase}\itememph{b} \emph{is} applicable! Indeed, flatness and properness are stable under base change and for the geometric fibres to be integral we can use the same argument as at the beginning of Step~2.} Let $s\in S$ be a closed point, corresponding to $s\colon \Spec k\morphism Z\times_kZ$. Then $s$ is given by a pair of morphisms (or equivalently closed points) $x,y\colon \Spec k\morphism Z$, and $s_Z\circ s=x+y$ holds by definition. Hence the fibre $X_s$ of $X\morphism S$ over $s$ equals the fibre of $A\times_kA\morphism A$ over $x+y$ (using some abstract nonsense). But this implies
	 \begin{align*}
	 	h^0\big(s,\Mm|_X^{\otimes \pm1}\big)=h^0\big(x+y,\Mm^{\otimes \pm 1}\big)\;.
	 \end{align*}
	 In particular, since $x+y$ is a point in $Z$ (because $A(k)\cap Z$ is a group), the universal property of $Z$ shows that the right-hand side doesn't vanish. Hence also $h^0\big(s,\Mm|_X^{\otimes \pm1}\big)\neq 0$. A similar argument shows $s\in U_0$. Therefore, neither the vanishing set $V$ of $h^0\big(-,\Mm|_X^{\otimes \pm1}\big)$ on $X$ nor the set $S\setminus U_0$ contain any closed point of $S$. But $V$ is open and $S\setminus U_0$ is closed by Theorem~\reff{thm:GrauertGrothendieck}\itememph{b}, \itememph{d}. Since closed points are dense in $Z\times_kZ$ (because it is Jacobson, cf.\ \cite[Definition~2.4.2]{alggeo1}) and any closed subset of $Z\times_kZ$ contains a closed point (because it is quasi-compact, cf.\ \cite[Proposition~2.1.1]{alggeo1}), this shows that $V=\emptyset$ and $S=U_0$. Therefore, the conditions of Proposition~\reff{prop:LineBundlesFromTheBase}\itememph{b} are indeed satisfied.
	 
	 Thus $s_Z$ is indeed a morphism $s_Z\colon Z\times_kZ\morphism Z$. In a similar way one proves that the inversion $i_A\colon A\morphism A$ restricts to $Z$. Hence $Z$ is a subgroup prescheme, as claimed above.
	 
	 \emph{Step~3.} We start setting up a proof by contradiction. Suppose $A(k)_\Ll=A(k)\cap Z$ is not finite. Then $\dim Z>0$. Indeed, if $\dim Z=0$, then $\alpha|_Z\colon Z\morphism\Spec k$ would be a proper morphism (because $Z\monomorphism A$ is a closed embedding and $A\morphism\Spec k$ is proper) with zero-dimensional fibres. But this would imply that $Z$ is finite over $k$ (by Corollary~\reff{cor:Rpvanishing} and the proof of Theorem~\reff{thm:ZariskiMain}\itememph{a}), hence $Z$ would consist of finitely many points, contradicting $\# A(k)_\Ll=\infty$.
	 
	 Now let $Y$ be the connected component of the (closed point associated to the) zero section $0_A\in A(k)$. We equip $Y$ with its reduced prescheme structure. Then $Y$ is a subgroup prescheme of $A$ by Lemma~\reff{lem:interlude}\itememph{b} and $\dim Y>0$ since $\dim Z>0$. Let $s_Y\colon Y\times_kY\morphism Y$ denote the addition on $Y$ and $0_Y\colon \Spec k\morphism Y$ the zero section. Let $\Ll|_Y$ be the pullback of $\Ll$ to $Y$ and consider the line bundle
	 \begin{align*}
	 	 \Nn=s_Y^*\Ll|_Y\otimes p_1^*\Ll|_Y^{\otimes-1}\otimes p_2^*\Ll|_Y^{\otimes-1}
	 \end{align*}
	on $Y\times_kY$. Our preliminary goal is to show that $\Nn$ is trivial. Let $\sigma_1\colon (\id_Y,0_Y\alpha|_Y)\colon Y\morphism Y\times_kY$ and $\sigma_2\colon (0_Y\alpha|_Y,\id_Y)\colon Y\morphism Y\times_kY$ be sections of $p_1$ and $p_2$ as considered in the theorem of the cube (Theorem~\reff{thm:cube}). Then
	\begin{align*}
		\sigma_1^*\Nn&\cong (s_Y\sigma_1)^*\Ll|_Y\otimes(p_1\sigma_1)^*\Ll|_Y^{\otimes -1}\otimes(p_2\sigma_1)^*\Ll|_Y^{\otimes -1}\\
		&\cong \id_Y^*\Ll|_Y\otimes\id_Y^*\Ll|_Y^{\otimes -1}\otimes\alpha|_Y^*0_Y^*\Ll|_Y^{\otimes -1}\;,
	\end{align*}
	in which $\alpha|_Y^*0_Y^*\Ll|_Y^{\otimes -1}$ is trivial since it is the pullback of a line bundle on $\Spec k$. Hence $\sigma_1^*\Nn$ is trivial. Now consider the pullback diagram
	\begin{diagram*}
		\object{0,1.5}{$Y\times_kA$}[F];
		\object{0,0}{$A\times_kA$}[X];
		\object{2.5,1.5}{$Y$}[Y];
		\object{2.5,0}{$A$}[S];
		\pullback{1.25,0.75};
		\scriptsize
		\arrow{F}{X};
		\arrow{F}{Y}[above][$p_1$];
		\arrow{X}{S}[above][$p_1$];
		\arrow{Y}{S};
	\end{diagram*}
	Since $Y\monomorphism A$ factors over $Z$, we obtain $\Mm|_{Y\times_kA}\cong p_1^*\Mm_0$ for some line bundle $\Mm_0$ on $Y$, by the universal property of $Z$. Since $\Nn\cong \Mm|_{Y\times_kY}\otimes p_1^*\Ll|_Y^{\otimes -1}$ (by definition), we get $\Nn\cong p_1^*(\Mm_0\otimes_{\Oo_Y}\Ll|_Y^{\otimes -1})$. Because $\sigma_1$ is a section of $p_1$, this implies
	\begin{align*}
		\Mm_0\otimes_{\Oo_Y}\Ll|_Y^{\otimes -1}\cong \sigma_1^*p_1^*\left(\Mm_0\otimes_{\Oo_Y}\Ll|_Y^{\otimes -1}\right)\cong \sigma_1^*\Nn\;.
	\end{align*}
	Hence $\Mm_0\otimes_{\Oo_Y}\Ll|_Y^{\otimes -1}$ is trivial and thus $\Nn\cong p_1^*(\Mm_0\otimes_{\Oo_Y}\Ll|_Y^{\otimes -1})$ is trivial as well.
	
	Consider the morphism $\psi=(\id_Y,-\id_Y)\colon Y\morphism Y\times_kY$. Then $s_Y\psi=\id_Y-\id_Y=0_Y\alpha|_Y$, hence the pullback of any line bundle on $Y$ along $s_Y\psi$ is trivial. Therefore, the line bundle
	\begin{align*}
		\Ll|_Y\otimes_{\Oo_Y}(-\id_Y)^*\Ll|_Y\cong (s_Y\psi)^*\Ll|_Y^{\otimes -1}\otimes_{\Oo_Y} (p_1\psi)^*\Ll|_Y\otimes_{\Oo_Y} (p_2\psi)^*\Ll|_Y\cong\psi^*\Nn^{\otimes -1} \;.
	\end{align*}
	is trivial since $\Nn$ is trivial. However, $\Ll|_Y$ is ample (indeed, using $\Ll$ is ample on $A$ and $Y\subseteq A$ is a closed subprescheme, this is quite easy to see) and $-\id_Y$ is an automorphism on $Y$, hence $(-\id_Y)^*\Ll|_Y$ is ample as well. Since tensor products of ample line bundles are ample again (this follows, e.g., from the extra assertion in Proposition~\reff{prop:relativelyAmple}), we conclude that $\Ll|_Y\otimes_{\Oo_Y}(-\id_Y)^*\Ll|_Y\cong \Oo_Y$ is ample.
	
	\emph{Step~4.} We finally derive a contradiction! Since $Y$ is a closed subprescheme of the proper $k$-scheme $A$, we see that $Y$ is proper over $k$ as well. Then $Y$ is noetherian, so any sheaf of ideals $\Jj\subseteq \Oo_Y$ is coherent. But $\Oo_Y$ is ample, hence \cite[Theorem~6\itememph{e}]{alggeo2} shows $H^1(Y,\Jj\otimes_{\Oo_Y}\Oo_Y^{\otimes n})=0$ when $n\gge 0$. However, $\Jj$ is isomorphic to $\Jj\otimes_{\Oo_Y}\Oo_Y^{\otimes n}$ for all $n$, hence $H^1(Y,\Jj)=0$. This shows that $Y$ is affine by Serre's affinity criterion (cf.\ \cite[Proposition~1.3.1]{alggeo2}). But then $Y$ is finite over $k$ since $\alpha|_Y^*\Oo_Y$ is a coherent sheaf on $\Spec k$ because $Y$ is proper over $k$. This implies $\dim Y=0$, contradicting our assumption $\dim Y>0$. This finally finishes the proof of \itememph{e} $\Rightarrow$ \itememph{c}.
	
	\emph{Proof of \itememph{c} $\Rightarrow$ \itememph{a} $\Rightarrow$ \itememph{b}.} Since clearly $A(k)_D\subseteq A(k)_\Ll$ and $\bigcap_{n_Z\neq 0}A(k)_Z\subseteq A(k)_D$, these implications are trivial.
	
	\emph{Proof of \itememph{b} $\Rightarrow$ \itememph{d}.} \emph{Step~5.} We show that a global section $\lambda\in\Gamma(A,\Ll)$ may be chosen such that $V(\lambda)=D$ (where the divisor $D=\sum_in_iZ_i$ is identified with the codimension-$1$ subset $\bigcup_iZ_i$). Since $\Ll=\Oo_A(D)$, we use Proposition~\reff{prop:divisors101}\itememph{a} to get an identification
	\begin{align*}
		\Gamma(A,\Ll)\cong\left\{f\in K\st\div(f)+D\geq 0\right\}
	\end{align*}
	(where $K=K(A)$ is the function field of $A$, i.e., the stalk at the generic point). Then we may choose $\lambda=1\in K$ on the right-hand side. To see this, take any point $x\in A$, then $\Oo_{A,x}$ is a UFD by the Auslander--Buchsbaum theorem (as explained on page~\pageref{par:AuslanderBuchsbaum}), hence for all ideals $\Ii_{Z_i}$ from \eqreff{eq:OXD} the stalk $\Ii_{Z_i,x}$ is a principal ideal $(p_i)\subseteq \Oo_{A,x}$ with $p_i\in\Oo_{A,x}$ a prime or a unit (all of this was explained in the appendix subsection on divisors).
	
	Since $D$ is an effective divisor (i.e., all $n_i>0$), we see that $\Ii_D$ is a sheaf of ideals, and it cuts out the closed subset $D=\bigcup_iZ_i$ (equipped with some closed subprescheme structure). Then $\Ii_{D,x}$ is given by $\prod_ip_i^{n_i}\Oo_{A,x}$. Hence $x\in D$ iff some $p_i$ lies in $\mm_{A,x}$, i.e., iff not all of the $p_i$ are units.
	
	Similarly, the stalk $\Ll_x=\Oo_A(D)_x\cong \Ii_{D,x}^{\otimes -1}$ is given by $\prod_ip_i^{-n_i}\Oo_{A,x}$. Hence, using $n_i>0$ once again, the image of $\lambda=1$ is contained in $\mm_{A,x}\Ll_x$ iff some $p_i\in \mm_{A,x}$. This shows $V(\lambda)=D$, as claimed.
	
	\emph{Step~6.} We use this to show that $\Ll^{\otimes 2}$ can be generated by global sections. Using Corollary~\reff{cor:thmOfTheSquare}, we see that for every $k$-valued point $a\in A(k)$ (or equivalently, a closed point by \eqreff{eq:ZariskiTangentSpace}), then $\mu_a=t_a^*\lambda\otimes t_{-a}^*\lambda$ is a global section of $t_a^*\Ll\otimes_{\Oo_A}t_{-a}^*\Ll\cong \Ll^{\otimes 2}$, and by construction of $\lambda$ it has vanishing set 
	\begin{align*}
		V(\mu_a)=V(t_a^*\lambda)\cup V(t_{-a}^*\lambda)=t_a^*D+t_{-a}^*D\;,
	\end{align*}
	where, again, the Weil divisor $t_a^*D+t_{-a}^*D$ is identified with the corresponding closed subset $\bigcup_i(t_{a}^{-1}(Z_i)\cup t_{-a}^{-1}(Z_i))$. If we can show that for all $x\in A$ we can find a $k$-valued (or closed) point $a\in A(k)$ such that $x\notin t_a^*D+t_{-a}^*D$, then $A$ can be covered by open subsets of the form $A\setminus V(\mu_a)$ for global sections $\mu_a\in\Gamma(A,\Ll^{\otimes 2})$, hence $\Ll^{\otimes 2}$ is generated by global sections (and then finitely many of them suffice by quasi-compactness).
	
	Let $W=\bigcup_{a\in A(k)}(A\setminus V(\mu_a))\subseteq A$. If $W=A$ then we're done. Otherwise $A\setminus W$ contains a closed point $z$ by \cite[Proposition~2.1.1]{alggeo1} (and then $z$ is also a $k$-valued point, as we have seen quite a lot of times now). Consider the closed subset $Z=D\cup (2z-D)$, where $2z-D$ denotes the image of $D$ under $t_{2z}-\id_A$ (or equivalently the pullback of $D$ under $t_{-2z}-\id_A$). Since $D$ has codimension $1$, and so has its isomorphic image $2z-D$, we see $A\setminus Z$ is non-empty open. Since closed points are dense in $A$ (because $A$ is Jacobson, cf.\ \cite[Definition~2.4.2]{alggeo1}), we find a closed point $y\in A\setminus Z$. Then $a=y-z$ has the property that neither $z+a=y$ nor $z-a=2z-y$ are contained in $D$. Hence $z\notin t_a^*D+t_{-a}^*D$, contradicting our assumption $z\in A\setminus W$.
	
	Hence we may choose a set of global generators $\lambda_0,\ldots,\lambda_n\in\Gamma(A,\Ll^{\otimes 2})$ of $\Ll^{\otimes 2}$. Note that $\Gamma(A,\Ll^{\otimes 2})$ is a finite-dimensional $k$-vector space since $A$ is proper over $k$, so \cite[Theorem~5]{alggeo2} can be applied. Thus we may additionally assume that $\lambda_0,\ldots,\lambda_N$ generates $\Gamma(A,\Ll^{\otimes 2})$ as a $k$-vector space (this has technical reasons, as we will see later). Then $\lambda_0,\ldots,\lambda_N$ define a morphism $\iota\colon A\morphism\IP_k^N$. We need to show that $\iota$ is finite.
	
	\lbl{par:ClosedPoint}\emph{Step~7.} Suppose $\iota$ is not finite. We show that there is a closed integral subprescheme $C\subseteq A$ such that $\dim C=1$ and $\iota$ maps $C$ to a single point. Indeed, first note that $\iota$ is proper since $\alpha\colon A\morphism \Spec k$ is proper and $\IP_k^N\morphism\Spec k$ is separated so \cite[Proposition~2.4.1]{alggeo2} applies. If for all closed points $s\in \IP_k^n$ the fibre $A_s=\iota^{-1}\{s\}$ was zero-dimensional, then the argument from Corollary~\reff{cor:Rpvanishing} shows $(R^p\iota_*\Ff)_s=0$ for all coherent $\Oo_A$-modules $\Ff$ and all $p>0$. But the closed points $s$ are dense in $\IP_k^N$, hence $R^p\iota_*\Ff=0$, which implies that $\iota$ is finite, using Serre's affinity criterion as in the proof of Theorem~\reff{thm:ZariskiMain}\itememph{a}. This gives a contradiction.
	
	So let $s\in \IP_k^n$ be a closed point such that $\dim (\iota^{-1}\{s\})>0$. Since $s$ is closed, $\iota^{-1}\{s\}$ is a closed subset of $A$. Then $C$ may be chosen as any closed irreducible subset of $\iota^{-1}\{s\}$ which has dimension $1$, and then equip $C$ with its canonical reduced subprescheme structure to have $C$ integral.
	
	\emph{Step~8.} We prove that for every $k$-valued (or closed) point $a\in A(k)$, $C$ is either contained in or disjoint from $t_a^*D+t_{-a}^*D$. Since $t_a^*D+t_{-a}^*D=V(\mu_a)$, it's enough to show that for all global sections $\mu\in\Gamma(A,\Ll^{\otimes 2})$ either $C\subseteq V(\mu)$ or $C\cap V(\mu)=\emptyset$. By construction of $\iota$ we have $\Ll^{\otimes 2}\cong \iota^*\Oo(1)$. Note that the canonical morphism $\Oo(1)\morphism\iota_*\iota^*\Oo(1)\cong \iota_*\Ll$ induces a surjection on global sections because $X_i\in k[X_0,\ldots,X_N]_1\cong \Gamma(\IP_k^N,\Oo(1))$ (this isomorphism is due to \cite[Theorem~2\itememph{a}]{alggeo2}) is sent to $\lambda_i$, and the $\lambda_i$ form a generating set of $\Gamma(A,\Ll^{\otimes 2})$ as a $k$-vector space (that's why we made this assumption!). Hence we may write $\mu=\iota^*\ell$ as the image of some homogeneous linear polynomial $\ell\in k[X_0,\ldots,X_N]_1$.
	
	Then $V(\mu)=V(\iota^*\ell)=\iota^{-1}(V(\ell))$ (indeed, this holds for sections of the structure sheaf, hence also for sections of line bundles as line bundles are locally trivial). In particular, if $V(\mu)$ intersects a fibre $\iota^{-1}\{s\}$, then it contains all of it. This proves that $C$ has the asserted property.
	
	\emph{Step~9.} We finally derive a contradiction. To do so, we consider the set of $k$-valued (or equivalently closed) points $A(k)$ and equip it with the subspace topology from $A$. This makes $A(k)$ into an algebraic variety (with its Zariski topology) -- that is, $A(k)$ is one of these ancient objects from the long-forgotten time before we defined what a prescheme is (if you feel uncomfortable with this, have a look at \cite[Section~2.3]{alggeo1}). Whenever we write something like $C$ or $D$ or $Z_i$ in the following three paragraphs, we always mean its intersection with $A(k)$, i.e., we only consider the closed points (and also note that $C\cap A(k)$ is irreducible and one-dimensional again in the induced subspace topology).
	
	Let $D=\sum_{i=1}^rn_iZ_i$ for distinct irreducible closed subsets $Z_i\subseteq A$ such that $\codim(Z_i,A)=1$ (and from now on we only consider their closed points). Let $z\in Z_1$ and $c_1\in C$. Then $c_1\in t_{z-c_1}^*Z_1$, hence $C$ intersects $t_{z-c_1}^*D+t_{c_1-z}^*D$ and thus it must be contained in it by Step~8. Therefore, for all $c_2\in C$ we find an index $i$ such that $c_2\in t_{z-c_1}^*Z_i$ or $c_2\in t_{c_1-z}^*Z_i$. That is, we can write
	\begin{align}\lbl{eq:CCZdecomposition}
		C\times C\times Z_1=\bigcup_{i=1}^r\left\{(c_1,c_2,z)\st c_2\in t_{z-c_1}^*Z_i\right\}\cup \bigcup_{i=1}^r\left\{(c_1,c_2,z)\st c_2\in t_{c_1-z}^*Z_i\right\}\;.
	\end{align}
	Here $C\times C\times Z_1$ is equipped with its product variety structure (as in \cite[Proposition~2.2.6]{alg1}) -- this coincides with the closed points of $C\times_kC\times_kZ_1$ (as a fibre product of preschemes) considered as an algebraic variety. Since all factors are irreducible, so is $C\times C\times Z_1$. Moreover, the sets on the right-hand side of the above decomposition \eqreff{eq:CCZdecomposition} are closed. Indeed, $\left\{(c_1,c_2,z)\st c_2\in t_{z-c_1}^*Z_i\right\}$ is the preimage of the closed set $Z_1$ under the map $C\times C\times Z_1\morphism A(k)$ sending $(c_1,c_2,z)$ to $c_2+z-c_1\in A(k)$. This map is continuous (since the group structure on $A(k)$ is given by morphisms of varieties), hence the set under consideration is indeed closed. The other case is similar.
	
	In particular, since $C\times C\times Z_1$ is irreducible, it must already be contained in one of sets on the right-hand side on \eqreff{eq:CCZdecomposition}. That is, there is an index $i$ such that $-c_1+c_2+z\in Z_i$ for all $(c_1,c_2,z)\in C\times C\times Z_1$, or there is an $i$ such that $c_2+c_1-z\in Z_i$ for all $(c_1,c_2,z)\in C\times C\times Z_1$. Hence $-c_1+c_2+Z_1\subseteq Z_i$ or $c_2+c_1-Z_1\subseteq Z_i$, and in both cases equality follows easily from comparing dimensions. In the first case, if $-c_1+c_2+Z_1=Z_j$ for all $(c_1,c_2)\in C^2$, then $Z_1$ is invariant under translations by $(-c_1+c_2)-(-c_1'+c_2')$ for all $c_1,c_2,c_1',c_2'\in C$. Similarly, $Z_1$ is invariant under translations by $(c_2+c_1)-(c_2'+c_1')$ in the second case. In either case, we see that $Z_1$ is invariant under translations by $c-c'$ for $c,c'\in C$. Analogously, the same is true for all the other $Z_i$.
	
	Now we restore the original meaning of $C$ and $Z_i$. That is, they are integral preschemes again, and the previous considerations were all about the closed points of $C$ and $Z_i$. Note that since the closed points $Z\cap A(k)$ of $Z_i$ are dense in it (since $Z_i$ is Jacobson, cf.\ \cite[Definition~2.4.2]{alggeo1}), translation by $a\in A(k)$ sends $Z_i$ to itself iff it sends $Z_i\cap A(k)$. So in order to derive a contradiction to \itememph{b}, it suffices to show that $\#\left\{c-c'\st c,c'\in C\cap A(k)\right\}=\infty$. Clearly it suffices to show that $\# (C\cap A(k))=\infty$ (since we can just fix $c'$). But if $C\cap A(k)$ had finitely many points, then it would be a discrete set (as all points are closed), hence zero-dimensional. However, $\dim C=1$, hence the set of its closed points form an algebraic variety which has Krull dimension $1$ as well (for example, by comparing structure sheaves). We are done!
\end{proof}
As Professor Franke points out, the proof in Mumford's book proceeds somewhat differently to derive the same contradiction. This uses the following lemma, which is perhaps worthwhile to know on its own. However, there's something wrong with the proof \ldots
\begin{lem}[see {\cite[61]{mumford1974abelian}}]
	If $E\subseteq A$ is irreducible of codimension $1$ and disjoint from the irreducible closed curve $C\subseteq A$, then $E$ is invariant under all $c_1-c_2$ with $c_1,c_2\in C\cap A(k)$.
\end{lem}
\begin{proof}[No proof at all]
	Since $C\cap E=\emptyset$ by assumption, the line bundle $\Ll=\Oo_A(E)|_C$ is trivial, hence for all $a\in A(k)$, $t_a^*\Ll|_C\cong \Oo_A(t_a^*E)|_C$ always has degree $0$. Indeed, the pullback of $\Ll$ under $s_A\colon C\times_SA\morphism A$ has constant (fibrewise) degree $0$ on the flat family of curves $p_2\colon C\times_SA\morphism A$ (indeed, on every fibre $s_A^*\Ll$ is the pullback of  a line bundle defined on the spectrum of a field, hence $s_A^*\Ll$ is fibrewise trivial). Now $t_a^*\Ll$ is, in turn, a pullback of $s_A^*$, hence it also has degree $0$ on $C$. Thus any global section of $\Oo_A(t_a^*E)$ vanishes on $C$ or has no zeros on $C$, hence $C\subseteq t_a^*E$ or $C\cap t_a^*E=\emptyset$.
	
	Now let $c_1,c_2\in C$ and $y\in E$ be closed points and apply the above to $a=y-c_2$. Then $C$ and $t_{y-c_2}^*E$ meet in $c_2$, hence $C\subseteq t_{y-c_2}^*E$ and thus $y-c_2+c_1\in E$. Since $y\in E$ was an arbitary closed point, we see that $E$ is indeed invariant under $c_1-c_2$.
\end{proof}
\begin{rem}
	So what's the problem with this proof? Well, it only works if the degree of line bundles on $C$ is a well-defined notion -- i.e., if $C$ is regular! But why should that be? So to fix the proof one has to show that $C$ may be replaced by its \emph{normalization} $\snake{C}\morphism C$ -- which we won't do here.
\end{rem}
	
	To finish the chapter on abelian varieties, we will prove a very nice theorem about the morphisms $[n]\colon A\morphism A$ from Corollary~\reff{cor:nL}.
\begin{thm}\lbl{thm:abelianVarieties}
	Let $A$ be an abelian variety of dimension $g=\dim A$ over a field $k$.
	\begin{alphanumerate}
		\item There are ample line bundles on $A$. In particular, $A$ is (strongly) projective.
		\item For $n\neq 0$, the multiplication morphism $[n]\colon A\morphism A$ is finite flat of degree $n^{2g}$.
	\end{alphanumerate}
\end{thm}
\begin{rem}
	Perhaps we should explain what we mean by ``degree'' in Theorem~\reff{thm:abelianVarieties}\itememph{b}. Since $[n]$ is finite flat (as will be shown), $[n]_*\Oo_A$ is a vector bundle, and we may define $\deg{[n]}=\rank {[n]_*\Oo_A}$. In particular, if $\eta_A\in A$ denotes the generic point and $K=\Oo_{A,\eta_A}$ the function field of $A$, then 
	\begin{align*}
		\deg{[n]}=\dim_K([n]_*\Oo_A)_{\eta_A}\;.
	\end{align*}
\end{rem}
Before we start with the proof, we have to show a lemma to make up for the fact that Proposition~\reff{prop:ampleFiniteStabilizer} assumes $k$ to be algebraically closed, but in Theorem~\reff{thm:abelianVarieties}, $k$ may be arbitrary. Note that this lemma was not in the lecture,  (but we didn't prove Theorem~\reff{thm:abelianVarieties}\itememph{a} either).
\begin{lem}\lbl{lem:amplePullback}
	Let $k$ be any field, $X$ a proper $k$-prescheme, and $\ov{X}=X\times_k\Spec \ov{k}$. Let $\pi\colon \ov{X}\morphism X$ denote the canonical projection. Then a line bundle $\Ll$ on $X$ is ample iff its pullback $\ov{\Ll}=\pi^*\Ll$ to $\ov{X}$ is ample.
\end{lem}
\begin{proof}
	First note that $\Ll^{\otimes m}$ is generated by global sections iff $\ov{\Ll}^{\otimes n}$ is. Indeed, for all open subsets $U\subseteq X$ we have
	\begin{align*}
		H^0\left(\pi^{-1}(U),\ov{\Ll}^{\otimes m}\right)\cong H^0(U,\Ll^{\otimes n})\otimes_k\ov{k}
	\end{align*}
	since $\ov{k}$ is flat over $k$. Hence if $\lambda_0,\ldots,\lambda_n\in H^0(X,\Ll^{\otimes m})$ are global generators of $\Ll$, then $\lambda_0\otimes 1,\ldots,\lambda_n\otimes 1$ are global generators of $\ov{\Ll}^{\otimes m}$. Conversely, if $\ov{\lambda}_0,\ldots,\ov{\lambda}_n$ are global generators of $\ov{\Ll}$, we may write $\ov{\lambda}_i=\sum_j\lambda_{i,j}\otimes x_{i,j}$ for $\lambda_{i,j}\in H^0(X,\Ll^{\otimes n})$ and $x_{i,j}\in\ov{k}$. Then the $\lambda_{i,j}$ are global generators of $\Ll^{\otimes n}$.
	
	If one of $\Ll$ or $\ov{\Ll}$ is ample, then the above shows that we may replace $\Ll$ and $\ov{\Ll}$ by some $n\ordinalth$ power, such that the new $\Ll$ and $\ov{\Ll}$ are generated by global sections. Moreover, we may choose global generators $\lambda_0,\ldots,\lambda_n$ and $\ov{\lambda}_0,\ldots,\ov{\lambda}_n$ such that $\ov{\lambda}_i=\lambda_i\otimes 1$. As explained in the proof of \cite[Theorem~6]{alggeo2}, this choice of generators defines a choice of global generators of $\Ll^{\otimes m}$ and $\ov{\Ll}^{\otimes m}$ for all $m>0$, which in turn define the horizontal arrows in the diagram
	\begin{diagram*}
		\object{0,1.5}{$\ov{X}$}[F];
		\object{0,0}{$X$}[X];
		\object{2.5,1.5}{$\IP_{\ov{k}}^{n_m}$}[Y];
		\object{2.5,0}{$\IP_k^{n_m}$}[S];
		\scriptsize
		\arrow{F}{X}[left][$\pi$];
		\arrow{F}{Y}[above][$\ov{\iota}_m$];
		\arrow{X}{S}[above][$\iota_m$];
		\arrow{Y}{S};
	\end{diagram*}
	Since our choices of generators of $\Ll^{\otimes m}$ and $\ov{\Ll}^{\otimes m}$ are compatible by construction, we see that this diagram commutes and is actually a pullback diagram! Now if $\Ll$ is ample, then $\iota_m$ is affine for $m\gge 0$ by \cite[Theorem~6]{alggeo2}.\footnote{Actually, the \cite[Theorem~6]{alggeo2} only says that for \emph{some} choice of generators $\iota_m$ will eventually be affine. However, the proof shows that this is true for any choice.} Hence its base change $\ov{\iota}_m$ is affine as well, proving that $\ov{\Ll}$ is ample by the cited theorem again. Conversely, if $\ov{\Ll}$ is ample, then $\ov{\iota}_m$ will be affine for $m\gge 0$. If $\Ff$ is any quasi-coherent $\Oo_X$-module, then
	\begin{align}\lbl{eq:Rpiota}
		(R^p\iota_{m,*}\Ff)\otimes_k\ov{k}\cong R^p\ov{\iota}_{m,*}(\pi^*\Ff)=0\;,
	\end{align}
	where the isomorphism on the left-hand side comes from the base change isomorphism (Proposition~\reff{prop:baseChangeMorphism}) and the fact that $\ov{k}$ is flat over $k$. The equation on the right-hand follows from $\ov{\iota}_m$ being affine.
	
	However, $\ov{k}$ is faithfully flat over $k$, hence \eqreff{eq:Rpiota} implies $R^p\iota_{m,*}\Ff=0$. Then $\iota_m$ is affine by \cite[Proposition~1.6.1\itememph{c}]{alggeo2}, which shows that $\Ll$ is ample too.
\end{proof}
\begin{proof}[Proof of Theorem~\reff{thm:abelianVarieties}]
	 \emph{Step 1.} For part~\itememph{a}, we first handle the case where $k$ is algebraically closed. By Proposition~\reff{prop:ampleFiniteStabilizer}, it suffices to construct an effective divisor $D=\sum_in_iZ_i$ such that the intersection $\bigcap_{n_i\neq 0} A(k)_{Z_i}$ is finite. The following construction is the one I came up with, and it even achieves that the intersection is trivial.
	
	Let $e\in A$ be the closed point corresponding to $0_A\colon \Spec k\morphism A$. Since $A$ is smooth over $k$ of dimension $d$ (by Fact~\reff{fact:groupSchemesSmooth}), $\Oo_{A,e}$ is a regular local ring of dimension $g$. So let $x_1,\ldots,x_g\in\Oo_{A,e}$ be generators of the maximal ideal $\mm_{A,e}$. Then all the $x_i$ are prime elements. Indeed, for all $i$ the quotient $\Oo_{A,a}/x_i\Oo_{A,e}$ is regular again (as explained, e.g., in the proof of \cite[Proposition~2.2.1]{homalg}), hence a domain, which shows that $x_i$ is indeed prime.
	
	Let $U\cong \Spec R$ be an affine open neighbourhood of $e$ such that the $x_i$ extend to elements of $R$. Let $\pp_i\in \Spec R$ be the preimage of the prime ideal $x_i\Oo_{A,e}$ in $R$, and put $\mm=\mm_{A,e}\cap R$. Using Nakayama's lemma (in a form like \cite[Lemma~1.5.1]{alg2}), we may shrink $U$ to ensure that $\mm$ is generated by $x_1,\ldots,x_g$ and $\pp_i=x_iR$ is a principal ideal with generator $x_i$ for all $i$. In particular, $\pp_i$ has height one by Krull's principal ideal theorem (see \cite[Theorem~11]{alg2}), where we use that $R$ is a domain (since $A$ is integral), so that $\hoehe(\pp_i)=0$ can't happen unless $\pp_i=0$. 
	
	Let $Y_1,\ldots,Y_g$ be the irreducible closed subsets of $A$ having $\pp_1,\ldots,\pp_g$ as generic points. Then all $Y_i$ have codimension $1$ (by construction of the $\pp_i$). Let $Z_1,\ldots,Z_m$ be the irreducible components of the closed subset $A\setminus U$. Then also all $Z_j$ have codimension $1$. This is not obvious at all, but follows from \cite[\stackstag{0BCU}]{stacks-project} (and Mumford completely ignores this problem in \cite[62]{mumford1974abelian}). Let $D=\sum_iY_i+\sum_jZ_j$. We claim that the effective Weil divisor $D$ does it!
	
	Let $a\in A(k)$ such that all $Y_i$ and $Z_j$ are mapped to themselves under $t_a^*$. Since $t_a$ is an isomorphism, it thus maps $U=A\setminus \bigcup_jZ_j$ to itself. Hence $t_a|_U$ is given by a ring automorphism $\phi\colon R\morphism R$. Since $t_a$ maps every $Y_i$ to itself, we must have $\phi(x_i)R=x_iR$ for all $i$, hence $\phi(x_i)=f_ix_i$ for some $f_i\in R^\times$. Then $\phi(\mm)=(f_1x_1,\ldots,f_gx_g)_R=(x_1,\ldots,x_g)_R=\mm$, hence $e$ is mapped to itself too. But $t_a(e)=e$ implies $a=e$ since $e$ is the origin. This shows that $D$ is indeed as required.
	
	\emph{Step~2.} Now we prove \itememph{a} in general. Let $k$ be arbitrary and let $\ov{A}=A\times_k\Spec \ov{k}$. By the special case above, we find a divisor $\ov{D}_0=\sum_i\ov{Z}_i$ in $\Div(\ov{A})$ such that $\bigcap_i\ov{A}(\ov{k})_{\ov{Z}_i}$ is finite. Let $\ov{\eta}_i$ be the generic point of $\ov{Z}_i$ and $\eta_i$ be the image of $\ov{\eta}_i$ in $A$. By Proposition~\reff{prop:GeometricFibres101}\itememph{a}, $\eta_i$ is a codimension-$1$ point again, hence $Z_i=\ov{\{\eta_i\}}\subseteq A$ is an irreducible closed subset of codimension $1$. Thus $D=\sum_iZ_i$ defines a Weil divisor on $A$. We will show that $\Ll=\Oo_A(D)$ is ample. If $\ov{\Ll}$ is the pullback of $\Ll$ to $\ov{A}$ and $\ov{D}$ a divisor on $\ov{A}$ as in \eqreff{eq:divisorPullback} on page~\pageref{par:divisorPullback} (so that $\ov{D}$ satisfies $\ov{\Ll}\cong\Oo_{\ov{A}}(\ov{D})$), then $\ov{D}$ has the finiteness condition from Proposition~\reff{prop:ampleFiniteStabilizer}\itememph{b}, because $\ov{D}$ contains all irreducible components of $\ov{D}_0$, which already has the required finiteness condition (by construction). Hence $\ov{\Ll}$ is ample by Proposition~\reff{prop:ampleFiniteStabilizer}. Now we can apply Lemma~\reff{lem:amplePullback} to see that $\Ll$ is ample as well. By \cite[Theorem~6\itememph{c}]{alggeo2} it follows that $A$ is strongly projective. This finishes the proof of \itememph{a}.
	
	\emph{Step~3.} We prove that $[n]$ is finite. Let $\Ll$ be an ample line bundle on $A$. Since $[-1]$ is an isomorphism, $\Ll\otimes_{\Oo_A}[-1]^*\Ll$ is also ample by Proposition~\reff{prop:relativelyAmple}. We may thus assume $\Ll\cong [-1]^*\Ll$. Then $[n]^*\Ll\cong \Ll^{\otimes n^2}$ by Corollary~\reff{cor:nL} (where we use that $[0]^*\Ll\cong 0_A^*\alpha^*\Ll$ is trivial, since it is a pullback of a line bundle on $\Spec k$). We claim that this implies $[n]$ is finite. 
	
	Suppose not, then there would be an irreducible closed subprescheme $C\subseteq A$  such that $\dim C=1$ and $[n](C)=\{a\}$ is a single closed point $a\in A$ (this is the same argument as in Step~7 of the proof of Proposition~\reff{prop:ampleFiniteStabilizer}). Then $[n]^*\Ll|_C$ is the trivial line bundle, because it is a pullback of a line bundle defined on a single point. However, $[n]^*\Ll\cong \Ll^{\otimes n^2}$ is ample because $\Ll$ is ample. Since pullbacks of ample line bundles along closed immersions are ample again (which is easy to check), we see that $[n]^*\Ll|_C\cong \Oo_C$ must an ample line bundle on $C$. However, this contradicts the fact that $\dim C=1$, as explained in Step~4 of the proof of Proposition~\reff{prop:ampleFiniteStabilizer}.
	
	\emph{Step~4.} We start proving flatness. The first step is to show that $\Oo_A\morphism[] [n]_*\Oo_A$ is a monomorphism of sheaves. Suppose not, and let $\Ii\subseteq \Oo_A$ denote the kernel. Since $[n]_*\Oo_A$ is a quasi-coherent $\Oo_A$-algebra (because $[n]$ is affine), it is easy to see that $\Ii$ is a quasi-coherent sheaf of ideals. Then $[n]\colon A\morphism A$ factors over the closed subprescheme $Y=V(\Ii)\subseteq A$. However, we have $\dim Y<\dim A$ unless $\Ii=0$ because $A$ is integral. Also all fibres of $A\morphism Y$ have dimension $0$ because the same is true for the fibres of $[n]$ (as we have already seen that $[n]$ is finite). Hence $\dim A>\dim_y(A_y)+\dim Y$ for all $y\in Y$, which contradicts basic facts about dimension such as \cite[\stackstag{02JS}]{stacks-project}.
	
	\emph{Step~5.} Now flatness can be shown. It suffices to show that the stalk $([n]_*\Oo_A)_x$ is flat over $\Oo_{A,x}$ for all $x\in A$, and since $\Oo_{A,x}\morphism ([n]_*\Oo_A)_x$ is injective by the previous step, we may apply the upcoming Proposition~\reff{prop:miracleFlatness} using that $A$ is regular everywhere (by Fact~\reff{fact:groupSchemesSmooth}), hence also Cohen--Macaulay (by \cite[Example~2.4.1]{homalg}). In particular, $[n]_*\Oo_A$ is a finite flat $\Oo_A$-algebra, hence a vector bundle (this follows, e.g., from \cite[Proposition~1.3.2]{homalg}).
	
	\emph{Step~6.} It remains to show the $\deg{[n]}=n^{2g}$. Mumford proves this using an intersection theory argument in \cite[63]{mumford1974abelian}. Since we haven't discussed intersection theory yet, Professor Franke explained a workaround via Hilbert polynomials (which were introduced in \cite[Section~3.1]{alg2}). This also appears in Mumford's book some pages ahead.
	
	Let $\Ll$ be an ample line bundle on $A$. We may assume $\Ll\cong [-1]^*\Ll$ (otherwise replace $\Ll$ by $\Ll\otimes_{\Oo_A}[-1]^*\Ll$ as above) and that $\Ll$ is very ample (otherwise replace $\Ll$ by some very ample power). Hence $[n]^*\Ll\cong \Ll^{\otimes n^2}$ (as above) and $\Ll$ is generated by global sections such that the induced morphism $\iota\colon A\morphism\IP_k^r$ (for some $r>0$) is a closed embedding.
	
	Then $\Ll\cong \iota^*\Oo(1)$. If $\Ff$ is any coherent $\Oo_A$-module, then $\iota_*\Ff$ is a coherent module on $\IP_k^m$. Moreover,
	\begin{align*}
		\iota_*\left(\Ff\otimes_{\Oo_A}\Ll^{\otimes m}\right)\cong \iota_*\left(\Ff\otimes_{\Oo_A}\iota^*\Oo(m)\right)\cong \iota_*\Ff\otimes_{\Oo_{\IP^r}}\Oo(m) \eqqcolon\iota_*\Ff(m)
	\end{align*}
	holds for all $m\in \IZ$. For the middle isomorphism we need that $\iota$ is a closed embedding (or just affine, cf.\ the discussion in \cite[Remark~2.2.7\itememph{d}]{alggeo2}); also the right-hand side is the usual notation for Serre twists. Hence
	\begin{align*}
		H^p\left(A, \Ff\otimes_{\Oo_A}\Ll^{\otimes m}\right)\cong H^p\left(\IP_k^r,\iota_*\left(\Ff\otimes_{\Oo_A}\Ll^{\otimes m}\right)\right)\cong H^p\left(\IP_k^r,\iota_*\Ff(m)\right)
	\end{align*}
	holds for all $p>0$. Therefore we may apply \cite[Proposition~2.2.4\itememph{b}]{alggeo2}\footnote{Actually, \cite[Proposition~2.2.4\itememph{b}]{alggeo2} makes an assertion about the Euler--Poincaré characteristic of (twists of) modules on $\IP_k^r$ rather than global sections (of their twists). However, these two coincide if we twist often enough, as is shown in the proof of the cited proposition.} to obtain
	\begin{align*}
		\dim_k H^0\left(A,\Ff\otimes_{\Oo_A}\Ll^{\otimes m}\right)=P_{\Ff}(m)\quad\text{if }m\gge 0\;,
	\end{align*}
	where $P_{\Ff}\in\IQ[T]$ is a some polynomial. More precisely, if $M$ is a graded module over $k[X_0,\ldots,X_r]$ such that $\iota_*\Ff\cong \snake{M}$ (here we use the graded twiddlization from \cite[Remark~2.2.1]{alggeo2} of course\footnote{\ldots and we should check that such an $M$ can always be found. It can be shown that $M=\bigoplus_{m\geq 0}H^0(\IP_k^r,\iota_*\Ff(m))$ does it. But for general quasi-coherent modules on some $\Proj R$ this might fail!}), then $P_\Ff$ coincides with the Hilbert polynomial $P_M$ associated to the graded module $M$. Moreover, if $0\morphism\Ff'\morphism\Ff\morphism\Ff''\morphism 0$ is a short exact sequence, then $P_{\Ff}=P_{\Ff'}+P_{\Ff''}$, because Hilbert polynomials of modules also behave in that way.
	
	From \cite[Proposition~3.1.3]{alg2}\footnote{\emph{Yes}, we are talking about algebraic varieties in this proposition, but as long as only the structure sheaf (or sheaves of modules on it) are concerned, all assertions hold for scheme-theoretic varieties as well.} we see that
	\begin{align*}
		\deg P_{\Ff}\leq g=\dim A
	\end{align*}
	with equality for $\Ff=\Oo_A$; moreover if the support of $\Ff$ is a proper closed subset $Z\subsetneq A$, then the inequality is strict. We denote the $g\ordinalth$ coefficient of $P_{\Ff}$ by $d(\Ff)$. In particular, $d(\Ff)=0$ if the support of $\Ff$ is strictly smaller than $A$, and $d(-)$ behaves additively on short exact sequences since so does $P_{(-)}$.
	
	If $\lambda\in\Gamma(A,\Ll)\setminus \{0\}$ is a global section, then $U=A\setminus V(\lambda)$ is a non-empty open subset and $-\otimes\lambda\colon \Ff\morphism\Ff\otimes_{\Oo_A}\Ll$ is an isomorphism over $U$. Then both the kernel and cokernel of this morphism have support smaller than $A$, so their $g\ordinalth$ coefficients vanish, thus
	\begin{align}\lbl{eq:dLShift}
		d(\Ff)=d(\Ff\otimes_{\Oo_A}\Ll)
	\end{align}
	by additivity on short exact sequences.\footnote{Another way to see \eqreff{eq:dLShift} is that the Hilbert polynomial of $\Ff\otimes_{\Oo_A}\Ll$ equals that of $\Ff$ shifted by $1$} Now choose $N$ large enough such that $\Ff\otimes_{\Oo_A}\Ll^{\otimes N}$ is generated by global sections. We may choose some of these global sections whose images in the stalk $(\Ff\otimes_{\Oo_A}\Ll^{\otimes N})_{\eta_A}$ at the generic point $\eta_A\in A$ form a basis as a vector space over $K=\Oo_{A,\eta_A}$. If $w=\dim_K(\Ff\otimes_{\Oo_A}\Ll^{\otimes N})_{\eta_A}$, then these $w$ global sections define a morphism $\phi\colon \Oo_A^{\oplus w}\morphism \Ff\otimes_{\Oo_A}\Ll^{\otimes N}$ which is an isomorphism on stalks at $\eta_A$, hence (by the usual Nakayama argument) also an isomorphism on some open neighbourhood $U$ of $A$. Thus $\ker\phi$ and $\coker \phi$ have support strictly smaller than $A$, so the same argument as for \eqreff{eq:dLShift} shows
	\begin{align}\lbl{eq:dFormula}
		d(\Ff)=d(\Ff\otimes_{\Oo_A}\Ll^{\otimes N})=d(\Oo_A^{\oplus w})=w\cdot d(\Oo_A)=\dim_K\Ff_{\eta_A}\cdot d(\Oo_A)\;,
	\end{align}
	where on the right-hand side we use that $\Ff_{\eta_A}$ and $(\Ff\otimes_{\Oo_A}\Ll^{\otimes N})_{\eta_A}$ are isomorphic as $K$-vector spaces.
	
	Now since $[n]\colon A\morphism A$ is affine (we showed it is even finite), we get $H^p(A,\Ff)\cong H^p(A,[n]_*\Ff)$, and by \cite[Remark~2.2.7\itememph{d}]{alggeo2},
	\begin{align*}
		[n]_*\Oo_A\otimes_{\Oo_A}\Ll^{\otimes m}\cong[n]_*(\Ff\otimes_{\Oo_A}[n]^*\Ll^{\otimes m})\cong  [n]_*(\Ff\otimes_{\Oo_A}\Ll^{\otimes n^2m})\;,
	\end{align*}
	 using that $[n]^*\Ll\cong \Ll^{\otimes n^2}$. Hence
	\begin{align*}
		P_{[n]_*\Oo_A}(m)&=\dim_kH^0\left(A,[n]_*\Oo_A\otimes_{\Oo_A}\Ll^{\otimes m}\right)\\
		&=\dim_kH^0\left(A,[n]_*\left(\Oo_A\otimes_{\Oo_A}[n]^*\Ll^{\otimes m}\right)\right)\\
		&=\dim_kH^0(A,\Oo_A\otimes_{\Oo_A}\Ll^{\otimes n^2m})\\
		&=P_{\Oo_A}(n^2m)\quad\text{for }m\gge 0\;.
	\end{align*}
	This shows $d([n]_*\Oo_A)=n^{2g}\cdot d(\Oo_A)$. In combination with \eqreff{eq:dFormula} we obtain $[n]_*\Oo_A$ is a vector bundle of rank $n^{2g}$, as desired.
\end{proof}
\begin{rem}
	The construction of ample divisors from Theorem~\reff{thm:abelianVarieties}\itememph{a} above breaks down for more general abelian schemes, and there are problems even for artinian rings of characteristic $0$.
\end{rem}
\begin{rem}\lbl{rem:LK}
	Since $A\morphism[][n]_*\Oo_A$ is injective as shown in Step~4, it's easy to check that $[n]$ maps the generic point $\eta_A$ to itself. Moreover, for all affine opens $U\subseteq A$ we know that $[n]_*\Oo_A=\Oo_A([n]^{-1}(U))$ is a domain because $A$ is integral. Since localizations of domains are domains again, $([n]_*\Oo_A)_{\eta_A}$ is a domain and a finite-dimensional vector space over $K=\Oo_{A,\eta_A}$, hence a finite field extension $L$ of $K$. But then $L$ has only one prime ideal, which shows that no other point than $\eta_A$ itself is mapped to $\eta_A$. Moreover, Theorem~\reff{thm:abelianVarieties}\itememph{b} shows $[L:K]=n^{2g}$.
\end{rem}
Before we derive some interesting corollaries of Theorem~\reff{thm:abelianVarieties}, we have to prove the flatness criterion that was needed in Step~5.
\begin{prop}\lbl{prop:miracleFlatness}
Let $R\subseteq S$ be a finite ring extension\footnote{If $R$ is not a subring of $S$ (this condition was missing in the lecture), the proposition will be wrong! Indeed, if $R$ is any regular local ring and $x\in \mm$ an element of its maximal ideal, then $R/xR$ is regular again and finite over $R$, but clearly not flat over $R$ in general!} where $R$ is a  regular noetherian local ring and all localizations of $S$ at maximal ideals are Cohen--Macaulay. Then $S$ is a flat $R$-algebra.
\end{prop}
\begin{proof}
	It suffices to show that $S_\qq$ is flat over $R$ for all maximal ideals $\qq\in\Spec S$. If $\mm$ is the maximal ideal of $R$, then $S/\mm S$ is finite over the field $R/\mm$, hence a zero-dimensional ring. Then $\dim S_\qq/\mm S_\qq$ is zero-dimensional as well. Moreover, since $R\subseteq S$ is finite, it is also integral, hence the going up theorem (as in \cite[Theorem~7]{alg1}) shows that all maximal ideals $\qq\subseteq S$ have height $\dim R$. Then also $\dim S_\qq=\dim R$. We conclude $\dim S_\qq=\dim R+\dim S_\qq/\mm S_\qq$, so the miracle flatness theorem (Lemma~\reff{lem:miracleFlatness}) may be applied.
\end{proof}
\begin{cor}\lbl{cor:nTorsion}
	When $A$ is an abelian variety of dimension $g=\dim A$ over an algebraically closed field and $n\neq 0$ an integer, then $A(k)$ is $n$-divisible and the number of $n$-torsion points divides $n^{2g}$, with equality iff $n$ is not a multiple of $\cha k$.
\end{cor}
\begin{proof}
	This proof wasn't in the lecture. In the following we use without further mention that $A(k)$ equals the set of closed points of $A$ by \eqreff{eq:ZariskiTangentSpace}, as $k$ is algebraically closed.
	
	The closure of the image of $[n]$ equals $V(\Ii)$, where $\Ii$ is the kernel of $\Oo_A\morphism[][n]_*\Oo_A$ (this was proved for the affine case in  the proof of Lemma~\reff{lem:closurePreimage}). However, we have $\Ii=0$ by Step~4 of the proof of Theorem~\reff{thm:abelianVarieties}, hence the image of $[n]$ is dense in $A$. But $[n]$ is also proper by the two-out-of-three property from \cite[Proposition~2.4.1]{alggeo2}, hence its image is closed, proving that $[n]$ is surjecive. In other words, $A(k)$ is $n$-divisible (because the preimage of any closed point is non-empty and closed, hence contains a closed point by \cite[Proposition~2.1.1]{alggeo2}).
	
	Let $e\in A$ be the \emph{origin}, i.e., the closed point corresponding to $0_A\colon \Spec k\morphism A$. Then $n$-torsion of $A(k)$ points are given by the closed points of the fibre over $e$. That is, they're given by the maximal ideals of $R=([n]_*\Oo_A)_e\otimes_{\Oo_{A,e}}\KK(e)$. Since $\KK(e)=\Oo_{A,e}/\mm_{A,e}\cong k$ and $[n]$ is finite, $R$ is a finite-dimensional $k$-vector space, hence an artinian ring. Thus, by \cite[Corollary~2.16]{eisenbudCommAlg}, we have $R\cong \prod_\mm R_\mm$, where $\mm$ ranges through the finitely many maximal ideals of $R$ (which are actually all prime ideals, since $\dim R=0$). Note that if $\mm_1,\mm_2$ are maximal ideals of $R$ corresponding to closed points $a_1,a_2\in[n]^{-1}\{e\}$, then $R_{\mm_1}\cong R_{\mm_2}$. Indeed, we may send $a_1$ to $a_2$ and vice versa via the translation $t_{a_2-a_1}$ and its inverse $t_{a_1-a_2}$, thus exchanging $\mm_1$ and $\mm_2$. In particular, all the $R_\mm$ have the same dimension $d=\dim_k R_\mm$ as $k$-vector spaces. Hence
	\begin{align*}
		n^{2g}=\dim_k R=\#\{n\text{-torsion points}\}\cdot d\;,
	\end{align*}
	proving that $\#\{n\text{-torsion points}\}$ divides $n^{2g}$.
	
	Moreover, equality holds iff all $R_\mm$ are isomorphic to $k$. Let $\mm_0\subseteq R$ correspond to $e$ (which is clearly mapped to itself). Since all $R_\mm$ are isomorphic to $R_{\mm_0}$ as seen above, it's enough to show $R_{\mm_0}\cong k$. Using that $k$ is algebraically closed and $R_{\mm_0}$ has only one maximal ideal together with \cite[Corollary~16.16]{eisenbudCommAlg}, it is easy to see that $R_{\mm_0}\cong k$ iff $\Omega_{R_{\mm_0}/k}=0$. Let $\Omega$ be the sheaf of Kähler differentials associated to $[n]\colon A\morphism A$ (I don't want to denote this $\Omega_{A/A}$ as usual, since this would be extremely misleading). Then $\Omega_{R_{\mm_0}/k}\cong \Omega_e\otimes_{\Oo_{A,e}}\KK(e)$ (since $\mm_0$ corresponds to $e$), hence Nakayama's lemma shows that $\Omega_{R_{\mm_0}/k}=0$ iff $\Omega_e=0$.
	
	Let $\Omega_A$ denote the sheaf of Kähler differentials for $\alpha\colon A\morphism\Spec k$ and let $\Omega_{A,e}\cong \Omega_{\Oo_{A,e}/k}$ be its stalk at $e$. Since $[n]$ maps $e$ to itself, we obtain an induced morphism $[n]^*\colon \Oo_{A,e}\morphism\Oo_{A,e}$. Writing down the relative cotangent sequence for this morphism gives a short exact sequence
	\begin{align*}
		\Omega_{A,e}\morphism \Omega_{A,e}\morphism \Omega_e\morphism 0\;.
	\end{align*}
	We claim that the left-most arrow $[n]^\oast\colon \Omega_{A,e}\morphism\Omega_{A,e}$ (adapting the notation from Theorem~\reff{thm:abelianSchemes}) is simply given by multiplication with $n$. Indeed, using the $[n]=s_A([n-1],\id_A)$ together with the isomorphism \eqreff{eq:p1p2Omega} and Theorem~\reff{thm:abelianSchemes}\itememph{c}, this is not too hard to show by induction (but you have to unravel quite a lot of stuff). In particular, $[n]^\oast$ is an isomorphism if $\cha k\nmid n$ and the zero morphism if $\cha k\mid n$. Since $\Omega_{A,e}$ doesn't vanish -- indeed, $\Omega_A$ is a vector bundle of rank $\dim A$ since $A$ is smooth over $k$ by Fact~\reff{fact:groupSchemesSmooth} -- this shows $\Omega_e=0$ iff $\cha k\nmid n$, which finishes the proof.
\end{proof}
\begin{rem*}
	Using that $e$ can be sent to any other closed point $a$ using the translation $t_a$, which is an isomorphism, it's easy to see that $\Omega_e=0$ implies $\Omega_a=0$ for all closed points, hence $\Omega=0$ since closed points are dense in $A$ (because $A$ is Jacobson, cf.\ \cite[Definition~2.4.2]{alggeo1}). Therefore, Corollary~\reff{cor:nTorsion} shows that $\#\{n\text{-torsion points}\}=n^{2g}$ iff $[n]$ is étale.
\end{rem*}
\begin{cor}
	If $\alpha\colon A\morphism S$ is an abelian scheme, then $[n]\colon A\morphism A$ is finite flat of degree $n^{2g}$, where $g$ is the relative dimension (as in Definition~\reff{def:relativeDimension}) of $A$ over $S$.
\end{cor}
\begin{proof}
	Since $[n]$ is proper by the two-out-of-three property from \cite[Proposition~2.4.1]{alggeo2}, it suffices by Theorem~\reff{thm:ZariskiMain}\itememph{a} to check that it is quasi-finite and flat. Of course, quasi-finiteness can be checked fibre-wise, and so can $\deg{[n]}=n^{2g}$ once flatness has been established. Moreover, the critère de platitude par fibres (Lemma~\reff{lem:platitudeDeFibres}) shows that flatness too can be checked on fibres. Then all we have to do is to apply Theorem~\reff{thm:abelianVarieties}\itememph{b} to each fibre.
	
	Note that Professor Franke gave an ad-hoc argument for flatness (which actually can be used to derive another proof of the critère de platitude par fibres). His argument uses two spectral sequences and proceeds completely analogous to the proof of Lemma~\reff{lem:I(s)LineBundle}.
\end{proof}
\begin{rem}\lbl{rem:dualVariety}
	Let $\alpha\colon A\morphism S$ be an abelian scheme and $\Ll$ a relatively ample (with respect to $\alpha$) line bundle on $A$. Consider the projection $p_1\colon A\times_SA\morphism A$ as an $A$-prescheme and let $K\subseteq A$ be the closed subprescheme we obtain when we apply the seesaw theorem (Theorem~\reff{thm:seesaw}) to the line bundle $\Mm=s_A^*\Ll\otimes p_2^*\Ll^{\otimes -1}$ on $A\times_SA$. Generalizing Step~2 in the proof of Proposition~\reff{prop:ampleFiniteStabilizer}, we see that $K$ characterizes those morphisms of $S$-preschemes $a\colon T\morphism A$ (we consider them as $T$-valued points $a\in A(T)$) which have the property that $t_a^*\Ll_T\otimes \Ll_T^{\otimes -1}$ comes from the base $T$, where $\Ll_T$ denotes the pullback of $\Ll$ under $A\times_ST\morphism A$. Moreover, this closed subprescheme $K$ can be shown to be a subgroup prescheme.
	
	Then we may ``define'' the \defemph{dual} abelian scheme $A^*=A/K$ (provided we know how to form a quotient of group preschemes). It is the (fibre-wise) connected component of a group scheme representing the Picard functor $\Pic_{A/S}$.
	
	Professor Franke refers to Mumford's book \cite{mumford1974abelian} and Milne's text in \cite{cornell1986arithmetic} for a further introduction to this topic.
\end{rem}


\chapter{Construction of the Jacobian}\lbl{sec:Jac}
%\setcounter{section}{-1}
%\section{Strategy of the construction}
\lbl{par:Strategy}Before we dive into technical constructions, Professor Franke takes the time to review our battle plan from page~\pageref{page:battlePlan}, adding a bit more details and motivation.

Let $D$ be a divisor on a regular curve $C$ of genus $g=\dim_k\Gamma(C,\Omega_C)$ over an algebraically closed field $k$. Recall the notation $\ell(D)=\dim_k\Gamma(C,\Oo_C(D))$. If $\deg D=g$ and $K$ is a divisor such that $\Oo_C(K)\cong \Omega_C$, then Riemann--Roch (in the form of \cite[Corollary~3.1.1]{alggeo2}) shows
\begin{align*}
	\ell(D)-\ell(K-D)=\deg D+1-g=1\;.
\end{align*}
Thus $\ell(D)\geq 1$, so $\Oo_C(D)$ has a non-vanishing global section, and when $\ell(K-D)=0$, then the $k$-vector space of global sections of $\Oo_C(D)$ is one-dimensional. If $g\leq 1$, then $\deg (K-D)=\deg K-\deg D=(2g-2)-g=g-2<0$, hence $\ell(K-D)$ vanishes for arbitrary divisors $D$ of degree $g$ by Lemma~\reff{lem:Hpvanishing}. For general $g$, this depends on $D$, but it is always possible to find an effective divisor $D$ such that $\deg D=g$ and $\ell(K-D)=0$. A possible construction goes as follows.

We start by choosing a closed point (or equivalently a point different from the generic point since $\dim C=1$ and $C$ is irreducible) $p_1\in C$ such that not all of the $g$ elements of some $k$-basis of $\Gamma(C,\Omega_C)$ vanish at $p_1$. If $i\colon \Spec k\morphism C$ corresponds to the closed point $p_1$ via \eqreff{eq:ZariskiTangentSpace}, then there is a short exact sequence
\begin{align*}
	0\morphism \Omega_C(-p_1)\morphism\Omega_C\morphism i_*\Oo_{\Spec k}\morphism 0
\end{align*}
as in the proof of \cite[Lemma~3.2.2]{alggeo2}. By choice of $p_1$, the kernel of the induced morphism $H^0(C,\Omega_C)\morphism H^0(C,i_*\Oo_{\Spec k})$ is non-empty, hence $\ell(K-p_1)<\ell(K)=g$. But $\deg p_1=1$, hence $\ell(p_1)\geq 1$, so Riemann--Roch shows that actually $\ell(K-p_1)=g-1$. Now choose a basis of the $(g-1)$-dimensional $k$-vector space $\Gamma(C,\Omega_C(-p_1))$ and let $p_2$ be a closed point such that not all the basis elements vanish at $p_2$. Analogously to the above, we show that $\ell(K-p_1-p_2)=g-2$. Continuing inductively, we find a sequence $p_1,\ldots,p_g$ such that $D=\sum_{i=1}^gp_i$ is an effective divisor and $\ell(K-D)=0$, as desired.

By the upper semicontinuity assertion in Theorem~\reff{thm:GrauertGrothendieck}\itememph{b} we would expect this to be the ``generic behaviour'', i.e., if we just choose $p_1,\ldots,p_g$ randomly then this will ``almost surely'' give a divisor $D$ with $\ell(K-D)=0$.

One should thus be able to identify an open subscheme of $\PIC_{C/k}^g$ with an open subscheme of the ``scheme of effective divisors of degree $g$'' (assuming that there \emph{is} a scheme parametrizing these guys, which we do not yet know), and then be able to construct the Picard scheme $\PIC_{C/k}$ by gluing appropriate shifts of this open subset.

Moreover, the ``scheme of effective divisors of degree $g$'' shouldn't be able to see the order of summands in $D=\sum_{i=1}^gp_i$. Thus it should be something like $C^{(g)}=C^g/\SS_g$, where $C^g=C\times_S\cdots\times_SC$ ($g$ factors) and the symmetric group $\SS_g$ acts by permuting the factors. 

So the first step will be to construct quotients of schemes (which are also needed when constructing the dual variety as in Remark~\reff{rem:dualVariety}) and then the symmetric powers $C^{(g)}$.

\section{Quotients by actions of finite groups}
Let $X$ be any prescheme on which a finite group $G$ acts (via morphisms of preschemes). Define the quotient $X/G$ (which is a priori only a ringed space but not a prescheme) in the following way: 
\begin{itemize}
	\item \emph{Points.} On point-set level, $X$ consists of the $G$-orbits, and $\pi\colon X\morphism X/G$ maps a point to its orbit.
	\item \emph{Topology}. A subset $U\subseteq X/G$ is open iff $\pi^{-1}(U)\subseteq X$ is open, i.e., we equip $X/G$ with the canonical quotient topology.
	\item \emph{Structure sheaf.} We put $\Oo_{X/G}=(\pi_*\Oo_X)^G$. That is, for every open $U\subseteq X/G$,
	\begin{align*}
		\Oo_{X/G}(U)=\left\{\lambda\in\Oo_X\big(\pi^{-1}(U)\big)\st g\lambda=\lambda\text{ for all }g\in G\right\}\;.
	\end{align*}
	And the algebraic component $\pi^*\colon \Oo_{X/G}\morphism \pi_*\Oo_X$ is just the inclusion of the subsheaf $(\pi_*\Oo_X)^G\subseteq\pi_*\Oo_X$.
\end{itemize}
\begin{thm}\lbl{thm:QuotientByGroup}
	Let $X$ be a prescheme with a finite group $G$ acting on it, and assume that every $G$-orbit is contained in a $G$-invariant affine open subset of $X$.
	\begin{alphanumerate}
		\item The ringed space $X/G$ is a prescheme again and $\pi\colon X\morphism X/G$ is a morphism of preschemes with the following universal property: For every test prescheme $T$, there is a bijection
		\begin{align*}
			\Hom_{\cat{PSch}}(X/G,T)&\isomorphism \left\{t\in\Hom_{\cat{PSch}}(X,T)\st tg=t\text{ for all }g\in G\right\}\\
			\tau&\longmapsto t=\tau\pi\;.
		\end{align*}
		\item Let, additionally, $f\colon X\morphism S$ be an $S$-prescheme of (locally) finite type with $S$ locally noetherian, so that $X$ is locally noetherian too, and assume that $G$ acts by morphisms of $S$-preschemes. Then the canonical induced morphism $\ov{f}\colon X/G\morphism S$ is an $S$-prescheme of (locally) finite type, hence locally noetherian, and $\pi\colon X\morphism X/G$ is a finite morphism. Moreover, let $y\in X/G$ and choose a preimage $x\in X$. Then
		\begin{align*}
			\roof{\Oo}_{X/G,y}\cong (\roof{\Oo}_{X,x})^{G_x}\;,
		\end{align*}
		where $G_x$ is the stabilizer of $x\in X$ and $\roof{\phantom{x}}$ denotes the completion of local rings with respect to their maximal ideals.
	\end{alphanumerate}
\end{thm}
\begin{rem}
	\begin{alphanumerate}
		\item The condition of Theorem~\reff{thm:QuotientByGroup}\itememph{a} is automatically satisfied if any finite subset of $X$ is contained in an affine open subset. To see this, first note that such an $X$ is separated by \cite[Fact~1.5.5]{alggeo1}, hence finite intersections of affine opens are affine again by \cite[Proposition~1.5.4]{alggeo1}. Then every $G$-orbit is a finite set, hence contained in some affine open $U$, and $U$ may be replaced by $\bigcap_{g\in G}gU$ to obtain a $G$-invariant affine open neighbourhood.
		
		For example, this is always the case if $X$ is projective over some affine base. Another case is where $X$ is projective over $S$ and $G$ acts by morphisms in the category of $S$-preschemes.
		\item Without the assertion about the $G$-orbits, $X/G$ (which can still be characterized by the universal property of Theorem~\reff{thm:QuotientByGroup}\itememph{a}) may exist only as an \emph{algebraic space} (this may also be the case for the dual variety). See \emph{Hironaka's example} of  a non-projective proper variety in \cite[Ch.\:III Exercise 5.9]{hartshorne}.
	\end{alphanumerate}
\end{rem}
Before we prove Theorem~\reff{thm:QuotientByGroup}, we study the affine case.
\begin{lem}\lbl{lem:ringInvariants}
	Let $G\subseteq \Aut(B)$ be a finite subgroup of the group of automorphisms of some ring $B$ and let $A=B^G$ be its ring of invariants. Put $X=\Spec B$. Then $\Spec A\cong X/G$ (as ringed spaces), identifying $\pi$ with the morphism induced by $A\monomorphism B$. Moreover, $B$ is integral over $A$.
\end{lem}
\begin{proof}
	We denote the $G$-action on $B$ by exponent notation, i.e., $b^g$ means the application of $g\in G$ to $b\in B$. 
	
	The fact that $B$ is integral over $A$ is easy: For $b\in B$, let $P(T)=\prod_{g\in G}(T-b^g)$, then $P\in A[T]$ is a monic polynomial satisfying $P(b)=0$. In particular, the going-up theorem (as in \cite[Theorem~7\itememph{b}]{alg1}) shows that for all $\pp\in \Spec A$ there are no proper inclusions between the prime ideals of $B$ over $\pp$.
	
	Next we check that $G$ acts transitively on the preimage of any $\pp\in\Spec A$. Suppose $\qq,\qq'\in X$ lie over $\pp$  and $\qq'$ is not contained in the $G$-orbit of $\qq$. Then $\qq'\subsetneq \qq^g$ for all $g\in G$ (by going-up as explained above). Since $G$ is a finite group, we may apply the prime avoidance lemma (cf.\ \cite[Lemma~2.5.1]{alg1}) find an element $b\in\qq'\setminus \bigcup_{g\in G}\qq^g$. Then $a=\prod_{g\in G}b^g$ is contained in $\qq'\setminus \qq$. However, we clearly have $a\in B^G=A$, contradicting $\pp=\qq\cap A$.
	
	The morphism $f\colon X\morphism\Spec A$ (coming from $A=B^G\subseteq B$) factors over a continous map $\ov{f}\colon X/G\morphism \Spec A$ (by the universal property of the quotient topology). Moreover, the previous paragraph shows that $\ov{f}$ is a bijection. It remains to show $\ov{f}$ is open. If $W\subseteq X$ is a $G$-invariant open subset and $\qq\in W$, then by prime avoidance again there is an element $b\in B$ vanishing on $X\setminus W$ but not contained in any $\qq^g$ for $g\in G$. Put $a=\prod_{g\in G}b^g$. Then $a\in A$ and $U=\Spec A\setminus V(a)$ is an open subset with the property that $\qq\in f^{-1}(U)\subseteq V$ and $f^{-1}(U)$ is $G$-invariant. It's easy to see that this implies $\ov{f}$ is indeed open.
	
	Moreover, if $a\in A$, then it's not hard to check that $A_a=(B_a)^G$, hence $\ov{f}$ also identifies the structure sheaves on $X/G$ and $\Spec A$. This finishes the proof.
\end{proof}
\begin{proof}[Proof of Theorem~\reff{thm:QuotientByGroup}]
	Part~\itememph{a}. Since the construction of $X/G$ is local with respect to $G$-invariant open subsets, all of this easily follows from Lemma~\reff{lem:ringInvariants}.
	
	Part~\itememph{b}. The assertions is local with respect to $S$, hence without losing generality let $S=\Spec R$, where $R$ is a noetherian ring. If $X$ is affine, say, $X=\Spec B$, then $B$ is finite over $B^G$ (since it is integral by Lemma~\reff{lem:ringInvariants} and of finite type, as $B$ has already finite type over $R$). Hence $B$ is finite over $B^G$. By the Artin--Tate theorem (see \cite[Proposition~1.4.2]{alg1}), $B^G$ has finite type over $R$ too. This pretty much shows all assertions except the one about completions.
	
	To see this, we may look at things locally. Choose an affine open neighbourhood $U\cong \Spec A$ of $y$, such that $\pi^{-1}(U)\cong \Spec B$ is affine again (since $\pi$ is finite) and $A\cong B^G$. Let $\pp\in\Spec A$ correspond to $y$. We may replace $B$ by $B_\pp$ and $A$ by $A_\pp\cong (B_\pp)^G$ to obtain that $A$ is local with maximal ideal $\pp$.
	
	Let $\qq_1,\ldots,\qq_m\in\Spec B$ be the prime ideals over $\pp$ (we know from Corollary~\reff{cor:finiteAlgebras}\itememph{a} that there are finitely many of them). By going-up (as in\cite[Theorem~7]{alg1}) the $\qq_i$ are precisely the maximal ideals of $B$. Therefore $\qq_i+\qq_j=B$ for $i\neq j$, hence also $\qq_i^n+\qq_j^n=B$ for all $n\geq 1$. Thus $\bigcap_{i=1}^m\qq_i^n=\prod_{i=1}^m\qq_i^n$ for all $n\geq 1$. In particular, since $\bigcap_{i=1}^m\qq_i=\prod_{i=1}^m\qq_i$ equals the Jacobson radical $\rad (B)$, we deduce $\bigcap_{i=1}^m\qq_i^n=\rad (B)^n$ for all $n\geq 1$. Therefore, the Chinese remainder theorem shows
	\begin{align}\lbl{eq:ChineseRemainderTheorem}
		B/\rad (B)^n\cong \prod_{i=1}^mB/\qq_i^n\quad\text{ for all }n\geq 1\;.
	\end{align}
	Since $G$ acts transitively on the $\qq_i$ and the $G$-action is via ring isomorphisms, we see that all the $B/\qq_i^n$ are isomorphic. Let $\qq\in\{\qq_1,\ldots,\qq_m\}$ be the prime ideal over $\pp$ that corresponds to $x$. Then the stabilizer subgroup $G_x=G_\qq$ acts on $B/\qq^n$, whereas an element $g\in G\setminus G_\qq$ sends $B/\qq^n$ to one of the other factors $B/\qq_i^n$. Taking $G$-invariants, we thus get
	\begin{align}\lbl{eq:BradBG}
		\left(B/\rad(B)^n\right)^G\cong (B/\qq^n)^{G_\qq}\cong (B_\qq/\qq^nB_\qq)^{G_\qq}\;.
	\end{align}
	Since $G$ is a finite group, taking $G$-invariants can be expressed as a finite limit (indeed, $M^G=\bigcap_{g\in G}\ker(g-\id_M\colon M\morphism M)$ holds for every $A$-module on which $G$ acts). Hence $(-)^G$ commutes with exact functors and arbitrary limits. Thus, taking limits over \eqreff{eq:BradBG} gives
	\begin{align}\lbl{eq:OXxG}
		\bigg(\limit[n\geq 1]B/\rad(B)^n\bigg)^G\cong \roof{B}_\qq^{G_\qq}\cong (\roof{\Oo}_{X,x})^{G_x}\;,
	\end{align}
	where $\roof{B}_\qq$ and $\roof{\Oo}_{X,x}$ denote the completions of these local rings with respect to their maximal ideals. Also the left-hand side of \eqreff{eq:OXxG} gives the $\rad(B)$-adic completion of $B$. Since $B/\pp B$ is a finite algebra over the field $A/\pp$, it is artinian, hence $\rad(B)/\pp B$ is the nilradical of $B/\pp B$ and thus nilpotent. In particular, there is an $N\geq 1$ such that $\rad(B)^N\subseteq \pp B$, hence the $\rad(B)$-adic and $\pp B$-adic completions of $B$ coincide. We denote either completion by $\roof{B}$. Since $\roof{\phantom{x}}$ is exact (on finite $A$-modules) by Corollary~\reff{cor:completionExact} and $(-)^G$ commutes with exact functors, we obtain $\roof{B}^G\cong \big(B^G\big)^\complete\cong\roof{A}$, and the latter is, by definition, nothing else but $\roof{\Oo}_{X/G,y}$. Combining with \eqreff{eq:OXxG} finishes the proof.
\end{proof}
\begin{rem}\lbl{rem:X/GbaseChange}
	Let $f\colon X\morphism S$ be a $G$-equivariant map with $G$ acting trivially on $S$, and assume that every $G$-orbit of $X$ is contained in a $G$-invariant affine open set (so that $X/G$ is a prescheme again by Theorem~\reff{thm:QuotientByGroup}\itememph{a}). If $S'\morphism S$ is any morphism, then $G$ acts on $X\times_SS'$ via its action on $X$ and the identity on $S'$. It's not hard to check that every $G$-orbit of $X\times_SS'$ is contained in a $G$-invariant affine open subset again, hence $(X\times_SS')/G$ is a prescheme -- but it might be different from $(X/G)\times_SS'$.
	
	However, if $S'\morphism S$ is flat at $V(\# G)\subseteq S'$ (where the cardinality $\#G$ -- which is a natural number -- can be considered as a global section of $\Oo_{S'}$), then we always have a canonical isomorphism
	\begin{align*}
		(X/G)\times_SS'\cong (X\times_SS')/G\;.
	\end{align*}
	Let's sketch how to prove this. Working locally, we may assume $X\cong \Spec B$, $S\cong \Spec A$, and $S'\cong \Spec A'$ to be affine. Then $B^G\otimes_AA'\cong (B\otimes_AA')^G$ needs to be shown. It suffices to verify this after localizing at the prime ideals of $\qq'\in\Spec A'$. Put $n=\# G$. If $\qq'\notin V(n)$, then $(B\otimes_AA'_{\qq'})^G$ is generated by elements of the form $\sum_{g\in G}b^g\otimes n^{-1}a$ and its easy to check that it is isomorphic to $B^G\otimes_AA'_{\qq'}$. If $\qq'\in V(n)$, then $A'_{\qq'}$ is flat over $A$ by assumption, hence $B^G\otimes_AA'_{\qq'}\cong (B\otimes_AA'_{\qq'})^G$ follows from the fact that $(-)^G$ can be represented as a finite limit (as in the proof of Theorem~\reff{thm:QuotientByGroup}\itememph{b}), hence tensoring with a flat algebra commutes with it.
\end{rem}
\begin{example}
	\begin{alphanumerate}
		\item \lbl{ex:symmetricPolynomials}If $R[X_1,\ldots,X_n]$ is a polynomial ring with the symmetric group $\SS_n$ acting by permutations, then $R[X_1,\ldots,X_n]^{\SS_n}\cong R[S_1,\ldots,S_n]$ by sending $S_i$ to the $i\ordinalth$ elementary symmetric polynomial $\sigma_i(X_1,\ldots,X_n)$. 
		\item Actually, $R[X_1,\ldots,X_n]$ is finite free of degree $n!$ over $R[S_1,\ldots,S_n]$, with a basis given by $X_1^{\alpha_1}\cdots X_n^{\alpha_n}$, where $0\leq \alpha_i< i$ for all $i$.
	\end{alphanumerate}
\end{example}
\begin{proof}
	Part~\itememph{a} is a well-known fact and commonly referred to as the \emph{fundamental theorem of symmetric polynomials}. We'll prove the lesser known assertion \itememph{b}. Write $B=R[X_1,\ldots,X_n]$ and $A=R[S_1,\ldots,S_n]$ for convenience.
	
	We first show that $B$ is generated over $A$ by the given monomials. This is done by induction on $n$, the case $n=1$ being trivial. If $P\in B$ is any polynomial, we can write $P=\sum_{i\geq 0}P_iX_n^i$ where the $P_i$ are polynomials in $X_1,\ldots,X_{n-1}$. By the induction hypothesis, each $P_i$ may be written as $P_i=\sum_\alpha Q_\alpha X_1^{\alpha_1}\cdots X_{n-1}^{\alpha_{n-1}}$, where $\alpha\in\IZ^{n-1}$ ranges through all multiindices such that $0\leq \alpha_j<j$ for all $j$, and $Q_\alpha$ are symmetric polynomials in $X_1,\ldots,X_{n-1}$. Therefore, it suffices to show that all polynomials $Q\in B$ which are fixed under $\SS_{n-1}$ (acting by permuting $X_1,\ldots,X_{n-1}$) can be written as $\sum_{j=0}^{n-1}Q_jX_n^j$ with $Q_j\in A$. Indeed, if this can be shown, then we may apply it to all $Q_\alpha X_n^i$ (which are clearly fixed under $\SS_{n-1}$ by construction) to obtain a representation of the desired form for our original $P$.
	
	Let $B'$ be the set of all polynomials of the form $Q=\sum_{j=0}^{n-1}Q_jX_n^j$ with $Q_j\in A$. Then $B'$ is an $A$-algebra since $X_n$ is integral over $A$, as $X_n$ is a root of
	\begin{align*}
		T^n-S_1T^{n-1}+S_2T^{n-2}\mp\ldots +(-1)^nS_n=\prod_{i=1}^n(T-X_i)\;.
	\end{align*}
	In particular, $B'$ contains $X_n$, and thus also the elementary symmetric polynomials in $X_1,\ldots,X_{n-1}$, which follows inductively from $\sigma_i(X_1,\ldots,X_{n-1})=S_i-X_n\sigma_{i-1}(X_1,\ldots,X_{n-1})$. But this already shows $B'=B^{\SS_{n-1}}$, because any $\SS_{n-1}$-invariant polynomial $Q\in B^{\SS_{n-1}}$ can be written as $Q=\sum_{j\geq 0}Q_jX_n^j$ with $Q_j$ a symmetric polynomial in $X_1,\ldots,X_n$ -- i.e., $Q_j$ is a polynomial in the $\sigma_i(X_1,\ldots,X_{n-1})$. This proves the inductive step.
	
	For freeness, it suffices to consider $R=\IZ$ as $R[X_1,\ldots,X_n]^{\SS_n}\cong \IZ[X_1,\ldots,X_n]^{\SS_n}\otimes_ZR$ by \itememph{a}. If there was a $\IZ[S_1,\ldots,S_n]$-linear dependence among the above monomials, then there would also be a $\IQ(S_1,\ldots,S_n)$-linear dependence, hence $\IQ(X_1,\ldots,X_n)$ would have degree less than $n!$ over $\IQ(S_1,\ldots,S_n)=\IQ(X_1,\ldots,X_n)^{\SS_n}$. This contradicts basic Galois theory.
\end{proof}
\section{Basic facts about symmetric powers}
\begin{defi}
	Let $M$ be an $R$-module. The \defemph{$\boldsymbol{n\ordinalth}$ symmetric power} is defined as
	\begin{align*}
		M^{(n)}=\left(M\otimes_R\cdots\otimes_RM\right)^{\SS_n}
	\end{align*}
	($n$ factors), whereupon $\SS_n$ acts by permuting factors.
\end{defi}
\begin{rem}
	\begin{alphanumerate}
		\item This is different (in general) from $\Sym_R^nM=\left\{\SS_n\text{-coinvariants of }M^{\otimes n}\right\}$, which is also commonly referred to as the \emph{$n\ordinalth$ symmetric power}.
		\item The functor $(-)^{(n)}$ commutes with base change along $R\morphism R'$ if $R'$ is flat over $R$ or $n!$ is invertible in $R'$. To prove this, one can use essentially the same argument as in Remark~\reff{rem:X/GbaseChange}.
		\item If $M$ and $N$ are both flat and $M\morphism N$ is injective, then $M^{\otimes n}\morphism N^{\otimes n}$ is injective again. Indeed, using the factorization $M^{\otimes n}\morphism M^{\otimes n-1}\otimes_RN\morphism\ldots\morphism M\otimes_RN^{\otimes n-1}\morphism N^{\otimes n}$ and flatness of $M$ and $N$, this is easily verified. 
	\end{alphanumerate}
\end{rem}
Recall that a \emph{directed set} (or \emph{filtered category} -- I prefer the latter terminology since it is less ambiguous) is a small category $\Lambda$ such that for all objects $x,y\in\Ob(\Lambda)$ there is an object $z$ with arrows $x\morphism z$ and $y\morphism z$; moreover, for any pair of parallel arrows $\alpha,\alpha'\colon x\morphism y$ there is an arrow $\beta\colon y\morphism z$ for some $z\in\Ob(\Lambda)$ such that $\beta\alpha=\beta\alpha'$. A \emph{directed} (or \emph{filtered}) colimit is a colimit taken over a diagram indexed by a directed set (or equivalently, a colimit taken over a functor $\Lambda\morphism\cat A$ into some category $\cat A$).

We will need -- and prove -- Lazard's famous characterization of flat modules.
\begin{prop}[Lazard]\lbl{prop:Lazard}
	An $R$-module is flat iff it may be written as a filtered colimit over finitely generated free $R$-modules.
\end{prop}
We start the proof with a lemma, which essentially says that all linear dependencies in a flat $R$-module come from linear dependencies in $R$.
\begin{lem}\lbl{lem:flatLinearDependence}
	Let $M$ be a flat $R$-module. If $m_1,\ldots, m_k\in M$ and $r_1,\ldots,r_k\in R$ are chosen such that $\sum_{i=1}^kr_im_i=0$, then, for some $\ell$, there is a vector $(\mu_j)\in M^{\ell}$ and a matric $(r_{i,j})\in R^{k\times \ell}$ such that
	\begin{align*}
		m_i=\sum_{j=1}^\ell r_{i,j}\mu_j\quad\text{for }i=1,\ldots,k\quad\text{and}\quad \sum_{i=1}^kr_ir_{i,j}=0\quad\text{for }j=1,\ldots,\ell\;.
	\end{align*}
\end{lem}
\begin{proof}
	If $(r_i)\cdot $ indicates multiplication with the row vector $(r_i)_{i=1}^k$, then
	\begin{align*}
		\ker\Big(M^{\oplus k}\xrightarrow{(r_i)\cdot }M\Big)\cong \ker\Big(R^{\oplus k}\xrightarrow{(r_i)\cdot }R\Big)\otimes_RM\;,
	\end{align*}
	since $M$ is flat.
\end{proof}
\begin{proof}[Proof of Proposition~\reff{prop:Lazard}]
	Since tensor products commute with colimits (because the tensor product is a left-adjoint functor by the tensor-hom adjunction) and filtered colimits of $R$-modules are exact, we see that a filtered colimit of flat $R$-modules is flat again. This shows the \emph{if} part.
	
	For the \emph{only if} part, let $\Lambda$ be the small category whose objects are vectors $m=(m_i)_{i=1}^k\in M^{\oplus k}$ of arbitrary length $k\geq 0$ and whose morphisms $m\morphism m'$ are given by matrices $A\in R^{k\times k'}$ satisfying $Am'=m$ (where $m$ and $m'$ are supposed to have length $k$ and $k'$ respectively). We show that $\Lambda$ is filtered!
	
	It is clear that for all $m_1,m_2\in\Ob(\Lambda)$ there is an $m'\in\Ob(\Lambda)$ with arrows $m_1\morphism m'$ and $m_2\morphism m'$ -- just stack $m_1$ on top of $m_2$ and project to $m_1$ and $m_2$ again.
	
	For the equalizer property, we will need the previous lemma. If $A_1,A_2\colon m\morphism m'$ are parallel morphisms, then $B=A_1-A_2$ satisfies $Bm'=0$. Put $m'_0=m'$ and $B_0=B$. Using Lemma~\reff{lem:flatLinearDependence} inductively, we find morphisms $B_i\colon m_{i-1}'\morphism m_i'$ such that the first $i$ rows of the matrix $B_0B_1\cdots B_i\colon m\morphism m'_i$ vanish. Thus, if $m$ has length $k$, then $m'\morphism m'_k$ equalizes $A_1$ and $A_2$.
	
	Now define a functor $F\colon \Lambda\morphism\cat{Mod}(R)$ as follows. We send objects $m=(m_i)_{i=1}^k$ to $F(m)=R^{\oplus k}$, and morphisms $A\colon m\morphism m'$ are sent to $A\colon R^{\oplus k}\morphism R^{\oplus k'}$, where we interpret the elements of $R^{\oplus k}$ and $R^{\oplus k'}$ as \emph{row} vectors (!) and the matrix $A$ acts by \emph{right} multiplication (!!). For every $m=(m_i)_{i=1}^k\in\Ob(\Lambda)$ we have a morphism $F(m)\morphism A$ sending $(r_i)_{i=1}^k\in R^{\oplus k}$ to $\sum_{i=1}^kr_im_i$. These morphisms are compatible with the $A\colon R^{\otimes k}\morphism R^{\otimes k'}$ since $rm=r(Am')=(rA)m'$ holds for all $r\in R^{\oplus k}$. Therefore, we obtain a canonical morphism
	\begin{align*}
		\colimit[\Lambda]F\morphism M\;,
	\end{align*}
	which is clearly surjective (by construction of $\Lambda$). By Lemma~\reff{lem:flatLinearDependence} it is also injective and we are done.
\end{proof}
\begin{cor}\lbl{cor:HorrorBasis}
	If $M$ is a flat $R$-module, then so is $M^{(n)}$. Moreover, in this case $(-)^{(n)}$ commutes with base change in the sense that
	\begin{align*}
		M^{(n)}\otimes_RR'\cong (M\otimes_RR')^{(n)}\;.
	\end{align*}
\end{cor}
\begin{proof}
	Assume first that $M$ is free with a basis $(m_i)_{i\in I}$. Then $M^{\otimes n}$ is free with a basis given by $m_{i_1}\otimes\cdots \otimes m_{i_n}$ for $(i_1,\ldots,i_n)\in I^n$. From this it is easy to see that $M^{(n)}$ is free as well, with a basis given by elements of the form
	\begin{align}\lbl{eq:HorrorBasis}
		\sum_{\sigma\in\SS_n/\SS_{j_1}\times\cdots\times\SS_{j_\ell}}\Big(\underbrace{m_{i_1}\otimes \cdots\otimes m_{i_1}}_{j_1}\otimes \underbrace{m_{i_2}\otimes\cdots\otimes m_{i_2}}_{j_2}\otimes\cdots \otimes \underbrace{m_{i_\ell}\otimes \cdots \otimes m_{i_\ell}}_{j_\ell}\Big)^\sigma\;,
	\end{align}
	where $j_1+\ldots+j_\ell=n$ and $i_1<i_2<\ldots <i_\ell$ with respect to some chosen well-ordering on $I$. The normal subgroup $\SS_{j_1}\times\cdots\times\SS_{j_\ell}\subseteq \SS_n$ is given by those permutations that leave all sets $\{1,\ldots,j_1\},\{j_1+1,\ldots,j_1+j_2\} ,\ldots$ invariant, and $(-)^\sigma$ denotes the application of a permutation. That is, the sum in \eqreff{eq:HorrorBasis} runs over all \emph{distinct} permutations of $m_{i_1}\otimes \cdots m_{i_1}\otimes\cdots\otimes m_{i_\ell}\otimes \cdots \otimes m_{i_\ell}$.
	
	In particular, $M^{(n)}$ is flat in this case and its formation is compatible with base change (since the basis in \eqreff{eq:HorrorBasis} doesn't depend on $R$). Note that the functor $(-)^{(n)}$ commutes with filtered colimits. Indeed, tensor products have this property, and the $(-)^{\SS_n}$ functor can be written as a finite limit (as explained in the proof of Theorem~\reff{thm:QuotientByGroup}\itememph{b}), hence commutes with filtered colimits as these guys are exact (for $R$-modules). Hence the case of arbitrary flat $R$-modules $M$ follows from Proposition~\reff{prop:Lazard} and the above special case.
\end{proof}
\begin{cor}\lbl{cor:Mnflatsurjective}
	If $M\morphism N$ is a surjective map of $R$-modules and $N$ is flat, then $M^{(n)}\morphism N^{(n)}$ is surjective.
\end{cor}
\begin{proof}
	As noted in the proof of Corollary~\reff{cor:HorrorBasis} above, $(-)^{(n)}$ commutes with filtered colimits. Thus, Proposition~\reff{prop:Lazard} gives 
	\begin{align*}
		N^{(n)}\cong \colimit[\lambda\in\Lambda] F_\lambda^{(n)}\;,
	\end{align*}
	where $\Lambda$ is a filtered category and the $F_\lambda$ are finite free $R$-modules. Hence every $x\in N^{(n)}$ has a preimage $\phi\in F_\lambda^{(n)}$ for some $\lambda$. But since $F_\lambda$ is free and $M\morphism N$ is surjective, $F_\lambda\morphism N$ can be lifted to a morphism $F_\lambda\morphism M$. Then the image of $\phi$ under $F_\lambda^{(n)}\morphism M^{(n)}$ provides a preimage of $x$.
\end{proof}
\begin{con}\lbl{con:sigmaja}
	For any $R$-algebra $A$, we have canonical morphisms $\iota_1,\ldots,\iota_n\colon A\morphism A^{\otimes n}$, where $\iota_i(a)=1\otimes \cdots \otimes a\otimes \cdots \otimes 1$ for all $a\in A$, with $a$ occcuring in the $i\ordinalth$ position. Let $\sigma_j$ denote the $j\ordinalth$ elementary symmetric polynomial. It defines a morphism (which -- by abuse of notation -- we also denote $\sigma_j$)
	\begin{align*}
	\sigma_j\colon A&\morphism A^{\otimes n}\\
	a&\longmapsto \sigma_j(\iota_1(a),\ldots,\iota_n(a))\;.
	\end{align*}
\end{con}
\begin{cor}\lbl{cor:AngeneratedBySigma}
	If $A$ is a flat $R$-algebra, then $A^{(n)}$ is generated (as an $R$-algebra) by the $\sigma_j(a)$ for $a\in A$ and $j=1,\ldots,n$.
\end{cor}
\begin{proof}
	We find a surjection $R\left[X_i\st i\in I\right]\epimorphism A$ from some polynomial ring onto $A$. Since $A$ is flat, the corresponding morphism $R\left[X_i\st i\in I\right]^{(n)}\epimorphism A^{(n)}$ is surjective again by Corollary~\reff{cor:Mnflatsurjective}. Hence it suffices to do the case where $A$ is a  polynomial ring. Moreover, every polynomial ring is a filtered colimit of polynomial rings in finitely many variables, so by compatibility of $(-)^{(n)}$ with filtered colimits we only need to do the case where $A=R[X_1,\ldots,X_\ell]$ for some $\ell\geq 0$.
	
	Then $A^{\otimes n}$ can be identified with $R\left[X_{i,j}\st i=1,\ldots,n,\ j=1,\ldots,k\right]$, where $i$ is considered a row index and $j$ a column index, and $\SS_n$ acts by permuting the rows of $(X_{i,j})$. Note that $A$, as a polynomial ring, is a free $R$-module and the set of all monomials is a basis. Hence $A^{(n)}$ is free again with a basis as in the proof of Corollary~\reff{cor:HorrorBasis}, equation~\eqreff{eq:HorrorBasis}.
	
	Let's spell out what this means explicitly. We choose a well-ordering on the set of monomials (i.e., on $\IN^k$). Without loss of generality we may assume that $\beta\leq 0$ for all $\beta\in \IN^k$ (this has technical reasons, as we will soon see). If $\beta=(\beta_1,\ldots,\beta_k)\in\IN^\ell$ is a monomial exponent, we put $Y_i^\beta=\prod_{j=1}^k X_{i,j}^{\beta_j}$ for short. Then a basis of $A^{(n)}$ as in \eqreff{eq:HorrorBasis} is given by the elements of the form
	\begin{align}\lbl{eq:HorrorBasis2}
		\mu=\sum_{\sym}Y_1^{\beta^1}Y_2^{\beta^2}\cdots Y_n^{\beta^n}\;,
	\end{align}
	where the $\beta^i\in\IN^k$ are monotonic with respect to the chosen well-ordering, i.e., $\beta^1\leq\ldots\leq\beta^n$ (we require ``$\leq$'' rather than ``$<$'' since we didn't specify the blocks $j_1,\ldots,j_\ell$), and $\sum_\sym$ indicates the sum over all \emph{distinct} permutations as in \eqreff{eq:HorrorBasis} (I really don't want to write down again what $\sum_\sym$ looks like explicitly).
	
	It suffices to show that all elements $\mu$ as in \eqreff{eq:HorrorBasis2} are contained in the subalgebra $B$ of $A^{(n)}$ generated by the $\sigma_i(a)$. We do this by induction on the number of $\beta^i\neq 0$ in $\mu$. If all $\beta^i$ vanish, then $\mu=1\in B$. Now assume for some $m>0$ that there are precisely $m$ indices $i$ such that $\beta^i\neq 0$ -- i.e., we have $\beta^{m+1}=\ldots=\beta^n=0$ since the $\beta^i$ are monotonic and $0$ comes last in the well-ordering. Put $\gamma=\beta^m$ and let $r<m$ be such that $\beta^r<\gamma$ but $\beta^{r+1}=\ldots=\beta^n=\gamma$. In analogy to our notation above, we put $Y^\gamma=\prod_{j=1}^kX_j^{\gamma_j}\in A$. Then $\sigma_{k-r}(Y^\gamma)$ is some element of $A^{(n)}$. Let $\nu=\sum_\sym Y_1^{\beta^1}\cdots Y_r^{\beta^r}\in A^{(n)}$. Then $\nu$ has $r<k$ non-vanishing $\beta^i$, hence $\nu\in B$ by the inductive hypothesis. Now every monomial appearing in
	\begin{align*}
		\mu-\nu\cdot \sigma_{k-r}(Y^\gamma)
	\end{align*}
	has less than $k$ non-vanishing exponents. Indeed, the monomials in $\nu\cdot \sigma_{k-r}(Y^\gamma)$ have at most $r+(k-r)=k$ non-vanishing exponents, and those with precisely $k$ cancel out with $\mu$. Hence $\mu-\nu\cdot\sigma_{k-r}(Y^\gamma)\in B$ by the inductive hypothesis. But also $\nu\cdot\sigma_{k-r}(Y^\gamma)\in B$ as seen above, so we finally obtain $\mu\in B$, as desired.
\end{proof}
For an $S$-prescheme $f\colon X\morphism S$ we denote the $n$-fold fibre product of $X$ over $S$ by $(X/S)^n$. If the base is clear from the context, we often just write $X^n$. Connecting Theorem~\reff{thm:QuotientByGroup} with the results of this section, we obtain the following proposition.
\begin{prop}\lbl{prop:symmetricPowers}
	Let $f\colon X\morphism S$ be a morphism\footnote{Professor Franke additionally required that $f$ be separated. However, this follows rather easily from our assumption on the fibres $f^{-1}\{s\}$ (and it might well be he mentioned that too). We explain the argument in the proof of Theorem~\reff{thm:final}\itememph{a}.} of preschemes such that for every finite subset $F$ of a fibre $f^{-1}\{s\}$ there are affine open $V\subseteq X$ and $W\subseteq S$ such that $f(V)\subseteq W$ and $F\subseteq V$.
	\begin{alphanumerate}
		\item The symmetric power $X^{(n)}=(X/S)^{(n)}$ exists and is covered by affine open $V^{(n)}$, where $V$ is as above. Moreover, $V_1^{(n)}\cap V_2^{(n)}=(V_1\cap V_2)^{(n)}$ for all $V_1,V_2$ as above (and assuming their intersection is affine again).
		\item If $f$ is flat, then $X^{(n)}$ is flat over $S$ and its construction commutes with base change. Moreover, $\Oo_{X^{(n)}}(V^{(n)})$ is generated by the $\sigma_i(\phi)$ for $\phi\in\Oo_X(V)$ and $i=1,\ldots,n$.
		\item Assume that $f$ is (locally) of finite type and $S$ is locally noetherian. Then $X^{(n)}$ is (locally) of finite type over $S$ and $X^n\morphism X^{(n)}$ is a finite morphism. Moreover, let $y\in X^{(n)}$ be a point, $x\in X^n$ a preimage, $G_x\subseteq \SS_n$ the stabilizer subgroup of $x$, and $x_1,\ldots,x_n\in X$ the projections of $x$. Then
		\begin{align*}
			\roof{\Oo}_{X^{(n)},y}\cong (\roof{\Oo}_{X^n,x})^{G_x}\cong \left(\bigg(\bigotimes_{i=1}^n\Oo_{X,x_i}\bigg)_\pp^\complete\right)^{G_x}\;,
		\end{align*}
		where $\pp\subseteq \bigotimes_{i=1}^n\Oo_{X,x_i}$ corresponds to $x$ and the tensor product is taken over $\Oo_{S,s}$, with $s$ the image of $x$ in $S$.
	\end{alphanumerate}
\end{prop}
\begin{proof}
	Let $x\in X^n$ and $x_1,\ldots,x_n\in X$ its projections to the $n$ factors. Then all $x_i$ have the same image $s\in S$ (since $X^n$ was taken over $S$), so $x_1,\ldots,x_n\in f^{-1}\{s\}$. Hence, by assumption, we find affine opens $V\subseteq X$ and $W\subseteq S$ such that $f(V)\subseteq W$ and $x_1,\ldots,x_n\in V$. Then $(V/W)^n=V\times_W\cdots\times_WV\subseteq (X/S)^n$ is an affine open neighbourhood of $x$. It contains the $\SS_n$-orbit of $x$ because $(V/W)^n$ contains all points whose projections are $x_1,\ldots,x_n$ in any order. Also $(V/W)^n$ is clearly $\SS_n$-invariant. By Theorem~\reff{thm:QuotientByGroup}\itememph{a}, this shows all of part~\itememph{a} except for the intersection assertion, which is also easy to check.
	
	Part~\itememph{b} follows from Corollary~\reff{cor:HorrorBasis} and Corollary~\reff{cor:AngeneratedBySigma}. The two finiteness assertions in \itememph{c} follow immediately from Theorem~\reff{thm:QuotientByGroup}\itememph{b}, and so does the isomorphism $\roof{\Oo}_{X^{(n)},y}\cong (\roof{\Oo}_{X^n,x})^{G_x}$. The second isomorphism follows from the explicit description of fibre products in \cite[Proposition~1.3.2]{alggeo2}.
\end{proof}
\section{Symmetric powers of a curve and effective relative Cartier divisors}\lbl{sec:curveSymmetricPowers}
Let $f\colon C\morphism S$ be a flat projective morphism such that the geometric fibres $C_{\ov{s}}=C\times_S\Spec \ov{\KK(s)}$ are regular connected curves of some genus $g_s$ for all $s\in S$. When $S$ is locally noetherian and $g_s>1$ for all $s\in S$, then $f$ being projective is no real restriction compared to  $f$ just being proper, since the line bundle $\Omega_{C/S}$ is relatively ample in this case, so $f$ is locally (strongly) projective by \cite[Theorem~6]{alggeo2}.
\begin{rem}\lbl{rem:gLocallyConstant}
	Note that $g_s$ is locally constant on $S$. Indeed, by Riemann--Roch (and an argument similar to Remark~\reff{rem:GeometricFibres}\itememph{b}) we have $\chi(s,\Oo_C)=1-g_s$ and $\chi(-,\Oo_C)$ is locally constant on $S$ by Theorem~\reff{thm:GrauertGrothendieck}\itememph{b}. Therefore it is ok to call $f$ a \emph{flat family of curves}, even though Definition~\reff{def:flatFamily} assumes $g$ to be constant.
\end{rem}
\begin{rem}
	Let us justify why $\Omega_{C/S}$ is indeed relatively ample. This was not in the lecture, and also it is extremely technical, so feel free to skip this remark. Before we start the proof, just note that $\Omega_{C/S}$ is indeed a line bundle by Proposition~\reff{prop:smoothnessCriteria}, since $f$ has regular geometric fibres and the relative dimension of $C/S$ is $1$.
	
	First let's assume that $C$ is a proper regular connected curve over some field $k$. We show that every line bundle of degree $\deg \Ll>0$ is ample. In particular, this applies to $\Omega_{C/k}$ if $C$ has genus $g>1$, since $\deg \Omega_{C/k}=2g-2$ (by \cite[Corollary~3.1.2]{alggeo2}). There is a proof in Hartshorne, \cite[Ch.\:III Corollary~3.3]{hartshorne}, but here is a cohomological argument. By \cite[Theorem~6\itememph{e}]{alggeo2}, it suffices to show $H^p(C,\Jj\otimes_{\Oo_C}\Ll^{\otimes n})=0$ when $p>0$, $n\gge 0$, and $\Jj$ is a sheaf of ideals in $\Oo_C$. By Grothendieck's theorem on cohomological dimension (cf.\ \cite[Proposition~1.4.1]{alggeo2}), this is only interesting for $p=1$. Now $\Jj$ is a line bundle (except for $\Jj=0$) since $C$ is locally Dedekind and every non-zero ideal in a Dedekind domain is invertible. Since $\deg \Ll>0$, $\Jj\otimes_{\Oo_C}\Ll^{\otimes n}$ will be a line bundle of degree $>2g-2$ for $n\gge 0$, and then its $H^1$ vanishes by Lemma~\reff{lem:Hpvanishing}.
	
	Now let $f\colon C\morphism S$ be a flat family family of curves of genus $g_s>1$ and assume $S$ is locally noetherian. To show $\Omega_{C/S}$ is relatively ample is a local question, whence we may assume $S\cong \Spec A$ and $g_s$ takes constant value $g$ on $S$ (by Remark~\reff{rem:gLocallyConstant}), and we are free to shrink $S$ further. We start by showing that $\Omega_{C/S}$ is $S$-locally globally generated (\ldots\ does that make me sound like an idiot?). Choose $s\in S$. Then the fibre $C_s$ is a regular connected curve over $k=\KK(s)$ by Corollary~\reff{cor:FibresAreCurvesToo} and our assumption on $C_{\ov{s}}$, and $\Omega_{C/S}|_{C_s}\cong \Omega_{C_s/k}$ is ample by the above paragraph. Let $n$ be large enough such that $\Omega_{C_s/k}^{\otimes n}$ is generated by global sections and has degree $>2g-2$. Then Proposition~\reff{prop:RpLl}\itememph{a} (or rather its proof) shows $s\in U_1$ and $R^1f_*\Omega_{C/S}^{\otimes n}=0$ near $s$. In particular, $R^1f_*\Omega_{C/S}^{\otimes n}$ is a vector bundle, hence also $s\in U_0$ by Theorem~\reff{thm:GrauertGrothendieck}\itememph{e}. Since $U_0$ and $U_1$ are open, we may assume $S=U_0=U_1$. This shows $f_*\Omega_{C/S}^{\otimes n}(s)\cong H^0(C_s,\Omega_{C_s/k})$. Choose global generators $\ov{\lambda}_0,\ldots,\ov{\lambda}_N$ of $\Omega_{C_s/k}^{\otimes n}$. Shrinking $S$, we can assure that the $\ov{\lambda}_i\in f_*\Omega_{C/S}^{\otimes n}(s)$ can be lifted to sections $\lambda_i\in \Gamma(S,f_*\Omega_{C/S}^{\otimes n})=\Gamma(C,\Omega_{C/S}^{\otimes n})$. Since the $\ov{\lambda}_i$ are global generators of $\Omega_{C_s/k}^{\otimes n}$, the images of the $\lambda_i$ generate the stalk $\Omega_{C_s/k,x}^{\otimes n}\cong \Omega_{C/S,x}^{\otimes n}\otimes_{\Oo_{S,s}}k$ for all $x$ in the fibre over $s$. By Nakayama's lemma (in the usual form of \cite[Lemma~1.5.1]{alg2}), we also find a neighbourhood $W_x\ni x$ such that $\Omega_{C/S}^{\otimes n}|_{W_x}$ is generated by the $\lambda_i$. Let $W=\bigcup_{x\in f^{-1}\{s\}}W_x$. Then $W$ is open and contains the fibre $f^{-1}\{s\}$. Since $f$ is proper, $f(X\setminus W)$ is closed and doesn't contain $s$, hence we may shrink $S$ again such that the new $\Omega_{C/S}^{\otimes n}$ is generated by the $\lambda_i$. This finishes the hardest technical part of the proof.
	
	It suffices to show that $\Omega_{C/S}^{\otimes n}$ defines a finite morphism $\iota\colon C\morphism\IP_S^N$ for $n\gge 0$, and for this, it suffices that $\iota$ is quasi-finite, as $\iota$ is proper (by the two-out-of-three criterion from \cite[Proposition~2.4.1]{alggeo2}), so Theorem~\reff{thm:ZariskiMain}\itememph{a} applies. If the base change $\iota_s\colon C_s\morphism\IP_{\KK(s)}^N$ is a closed embedding for all $s\in S$, then $\iota$ is injective on each fibre $f^{-1}\{s\}= C_s$, hence injective everywhere, and thus quasi-finite. Therefore, it suffices to show that $n\gge 0$ can be chosen large enough so that all the $\iota_s$ simultaneously become closed embeddings (we already know that this is possible for every \emph{single} $s$, since $\Omega_{C_s/\KK(s)}$ is ample). To prove this, we revisit the proof of \cite[Theorem~6]{alggeo2}: The construction of the descending chain $K_1\supseteq K_2\supseteq\ldots$ in Step~1 and 2 works regardless of whether $\Omega_{C/S}^{\otimes n}$ is ample -- we only need it to be generated by global sections. Hence there is an $n_0$ such that this chain stabilizes for $n\geq n_0$. Then also its base change to the fibres stabilizes for $n\geq n_0$, hence the $\iota_s\colon C_s\morphism\IP_{\KK(s)}^N$ are closed embeddings by Step~3 and 4 of the proof of the cited theorem (and now we \emph{do} need ampleness of the $\Omega_{C_s/\KK(s)}$). This finishes the proof.\qed
\end{rem}
\begin{lem}\lbl{lem:CartierDivisor}
	Let $f\colon C\morphism S$ be a flat family of curves, where $S$ is locally noetherian (alternatively, we could add a lot of finite presentation hypotheses). For a closed subprescheme $D\subseteq C$ defined by the sheaf of ideals $\Jj\subseteq \Oo_C$, the following conditions are equivalent.
	\begin{alphanumerate}
		\item $\Jj$ is a line bundle and the intersection of $D$ with any fibre $C_s$ is finite.
		\item $D$ is finite and flat over $S$.
	\end{alphanumerate}
\end{lem}
\begin{proof}
	We start with \itememph{b} $\Rightarrow$ \itememph{a}. We first show that $\Jj$ is flat over $\Oo_C$, which shows that $\Jj$ is a vector bundle since we are in the noetherian case (cf.\ \cite[Proposition~1.3.2]{homalg}). Let $s\in S$ and $x\in C$ a point over $s$. By the critère de platitude par fibres (Lemma~\reff{lem:platitudeDeFibres}, applied to $A=\Oo_{S,s}$ and $B=B'=\Oo_{C,x}$) it suffices to show that $\Jj_x$ is flat over $\Oo_{S,s}$ and $\Jj_x/\mm_{S,s}\Jj_x$ is flat over $\Oo_{C,x}/\mm_{S,s}\Oo_{C,x}$. The first one is easy: We have a short exact sequence $0\morphism \Jj_x\morphism\Oo_{C,x}\morphism\Oo_{D,x}\morphism 0$, in which both $\Oo_{C,x}$ and $\Oo_{D,x}$ are flat over $\Oo_{S,s}$, hence so is $\Jj_x$. 
	
	Tensoring  the above exact sequence with $\KK(s)$, we obtain
	\begin{align*}
		\Tor_1^{\Oo_{S,s}}(\KK(s),\Oo_{D,x})\morphism \Jj_x/\mm_{S,s}\Jj_x\morphism \Oo_{C,x}/\mm_{S,s}\Oo_{C,x}\;,
	\end{align*}
	in which the left-hand term vanishes as $\Oo_{D,x}$ is flat over $\Oo_{S,s}$, hence $\Jj_x/\mm_{S,s}\Jj_x$ is isomorphic to an ideal of $\Oo_{C,x}/\mm_{S,s}\Oo_{C,x}\cong \Oo_{C_s,x}$. However, $\Oo_{C_s,x}$ is a DVR since $C_s$ is a regular curve (by Corollary~\reff{cor:FibresAreCurvesToo}), hence every ideal is free. This proves that $\Jj_x$ is flat over $\Oo_{C,x}$ and thus $\Jj$ is a vector bundle on $C$. Moreover, we have seen that $\Jj_x/\mm_{S,s}\Jj_x$ is isomorphic to an ideal of the DVR $\Oo_{C_s,x}$, hence $\Jj$ has rank at most $1$.
	
	However, $\Jj=0$ is impossible since otherwise $C=D$ would be finite over $S$, and then the relative dimension of $C$ over $S$ couldn't be $1$. Hence $\Jj$ is a line bundle. Also $D\cap C_s$ is a finite prescheme over $\KK(s)$ for all $s\in S$, since being finite is stable under base change. This proves \itememph{b} $\Rightarrow$ \itememph{a}.
	
	Now \itememph{a} $\Rightarrow$ \itememph{b}. Since the intersections $D\cap C_s$ are finite over $\KK(s)$ for all $s\in S$, and the morphism $D\morphism S$ is the composition of the closed embedding $D\monomorphism C$ with the proper morphism $f$, we see that $D\morphism S$ is quasi-finite and proper, hence finite by Theorem~\reff{thm:ZariskiMain}\itememph{a}. Thus, for $s\in S$ and $x\in D$ a point over $s$, we may test for flatness of $\Oo_{D,x}$ over $\Oo_{S,s}$ by checking $\Tor_1^{\Oo_{S,s}}(\KK(s),\Oo_{D,x})=0$ (actually, we don't need finiteness of $D$ over $S$, since we can always apply the local flatness criterion, cf.\ Proposition~\reff{prop:localFlatness}).
	
	Since $\Jj$ is a line bundle on $C$, $\Jj_x$ is generated by some non-zero divisor $\lambda\in\Oo_{C,x}$. Hence $\Oo_{D,x}\cong \Oo_{C,x}/\lambda\Oo_{C,x}$. Since $\Oo_{C,x}$ is flat over $\Oo_{S,s}$ by assumption, it's easy to check that
	\begin{align}\lbl{eq:TorKer}
		\Tor_1^{\Oo_{S,s}}\left(\KK(s),\Oo_{D,x}\right)\cong \ker\Big(\Oo_{C,x}\otimes_{\Oo_{S,s}}\KK(s)\morphism[\lambda\cdot]\Oo_{C,x}\otimes_{\Oo_{S,s}}\KK(s)\Big)\;.
	\end{align}
	Note that $\Oo_{C,x}\otimes_{\Oo_{S,s}}\KK(s)\cong \Oo_{C_s,x}$ is a domain since it is a DVR, as $C_s$ is a regular curve over $\KK(s)$ (by Corollary~\reff{cor:FibresAreCurvesToo}). Also $\lambda$ can't vanish in $\Oo_{C_s,x}$, otherwise $\Jj|_{C_s}$ would vanish on some small open neighbourhood of $x$ in $C_s$ (by Nakayama), and then $D\cap C_s$ would contain that neighbourhood, contradicting it being finite over $\KK(s)$. Hence the kernel in \eqreff{eq:TorKer} indeed, vanishes, proving flatness.
\end{proof}
\begin{defi}\lbl{def:CartierDivisor}
	If the equivalent properties of Lemma~\reff{lem:CartierDivisor} are satisfied, $D$ is called an \defemph{effective relative Cartier divisor}. The set of relative (with respect to $f\colon C\morphism S$) effective Cartier divisors is denoted $\Div_f=\Div_{C/S}$.
\end{defi}
\begin{rem}\lbl{rem:CartierSemigroup}
	If $D$ and $E$ are such divisors given by sheaves of ideals $\Jj$ and $\Kk$, then the multiplication $\Jj\otimes_{\Oo_C}\Kk \morphism\Oo_C$ is a monomorphism with image $\Jj\Kk$, which is a line bundle again and has vanishing set $D\cup E$. Therefore, $\Jj\Kk$ defines an effective relative Cartier divisor again, which we denote $D+E$. Also, if $D\subseteq E$, then using that $\Jj$ and $\Kk$ are locally generated by non-zero divisors in $\Oo_C$ it's easy to see that the multiplication $\Jj^{\otimes -1}\otimes_{\Oo_C}\Kk\morphism \Oo_C$ is well-defined and induces an isomorphism onto some sheaf of ideals in $\Oo_C$. Thus we get an effective relative Cartier divisor $E-D$. Since $\Jj^{\otimes -1}\otimes_{\Oo_C}\Jj\otimes_{\Oo_C}\Kk\cong \Kk$, we get $(D+E)-D=E$.
		
	In this way, $\Div_{C/S}$ forms a semigroup with cancellation. By some standard construction, it can be turned into a group, in which $E-D$ for some effective divisors $D$ and $E$ is effective again iff $D\subseteq E$.
\end{rem}
\begin{rem}\lbl{rem:degree}
	We denote the restriction of $f$ to $D$ by $f|_D$. The \defemph{degree of $D/S$} is defined as the rank of the vector bundle $f|_{D,*}\Oo_D$ -- this is a vector bundle since $D$ is finite and flat over $S$, which is locally noetherian; in the non-noetherian case we would need $f$ (and thus $f|_D$) to be finitely presented. Note that the degree is in general only a locally constant function on $S$, but nevermind.
		
	The degree defines a semigroup homomorphism
	\begin{align*}
		\deg(-/S)\colon \Div_{C/S}\morphism \IN_S\;,
	\end{align*}
	where $\IN_S$ denotes the sheaf of locally constant $\IN$-valued functions on $S$. Indeed, if $D$ and $E$ are effective Cartier divisors corresponding to $\Jj$ and $\Kk$, then we have a short exact sequence $0\morphism \Jj/\Jj\Kk\morphism\Oo_C/\Jj\Kk\morphism\Oo_C/\Jj\morphism 0$. Using that $\Jj$ and $\Kk$ are locally generated by non-zero divisors in $\Oo_C$, it's easy to check that $\Jj/\Jj\Kk\cong \Oo_C/\Kk$. Now let $i_D\colon D\monomorphism D+E$ and $i_E\colon E\monomorphism D+E$ denote the inclusions of $D$ and $E$. Then the above considerations provide a short exact sequence $0\morphism i_{E,*}\Oo_E\morphism\Oo_{D+E}\morphism i_{D,*}\Oo_D\morphism 0$. Since $f|_{D+E}$ is  a finite morphism, the direct image functor $f|_{D+E,*}$ is exact because its higher derived functors vanish by \cite[Proposition~1.6.1\itememph{c}]{alggeo2}. Using $f|_D=f|_{D+E}\circ i_D$ (and the same for $E$) we obtain a short exact sequence
	\begin{align*}
		0\morphism f|_{E,*}\Oo_E\morphism f|_{D+E,*}\Oo_{D+E}\morphism f|_{D,*}\Oo_D\morphism 0
	\end{align*}
	on $S$. Since rank is additive on short exact sequences, this shows that the degree is indeed additive, i.e., a semigroup homomorphism.
\end{rem}
\begin{rem}\lbl{rem:degree=degree}
	If $D$ has degree $d$, then for every point $s\in S$ the pullback $\Jj^{\otimes -1}|_{C_{\ov{s}}}$ to the geometric fibre over $s$ is a line bundle of degree $d$ in the sense of \cite[Theorem~7\itememph{a}]{alggeo2}. Indeed, from Lemma~\reff{lem:CartierDivisor} we know that $D_s$ is finite over $\KK(s)$, hence $D_{\ov{s}}$ is finite over $\ov{\KK(s)}$. In particular, $D_{\ov{s}}$ has only finitely many points (since artinian rings have only finitely many prime ideals). Since we assume that $f$ is projective, we find an affine subset $U\cong \Spec R$ of $C_{\ov{s}}$ containing $D_{\ov{s}}$ by \cite[Proposition~2.2.1\itememph{b}]{alggeo2}. Hence $D\cong \Spec R/J$ for some ideal $J$ (which is given by the pullback of $\Jj$).
	
	 We put $k=\ov{\KK(s)}$. Since $D$ is finite locally free of degree $d$ over $S$-- that's how the degree was defined in Remark~\reff{rem:degree} -- it's easy to check that $d=\dim_k R/J$. But if $J=\pp_1^{e_1}\cdots\pp_n^{e_n}$ is the decomposition of $J$ into prime ideals in the Dedekind domain $R$, then $R/J\cong \prod_{i=1}^nR/\pp_i^{e_i}$ by the Chinese remainder theorem, and $\dim_k R/\pp_i^{e_i}=e_i$ since the residue field extension $\KK(\pp)/k$ must be trivial by Hilbert's Nullstellensatz and the fact that $k$ is algebraically closed. Hence
	 \begin{align*}
	 	d=\sum_{i=1}^ne_i=\sum_\pp v_\pp(J)\;.
	 \end{align*}
	 But this is how the degree of line bundles (or to be precise, of line bundles given by Weil divisors) was defined before.
\end{rem}
\begin{rem}
	\begin{alphanumerate}
		\item \lbl{rem:CartierBaseChange}If $S'\morphism S$ is any morphism, then $D'=D\times_SS'$ is an effective relative Cartier divisor for $C'=C\times_SS'\morphism S'$ since finiteness and flatness are stable under base change.
		\item The base change map $\Div_{C/S}\morphism\Div_{C'/S'}$ is a semigroup homomorphism. Indeed, let $U\cong \Spec A$ be a small affine open subset of $C$ such that $D\cap U$ and $E\cap U$ are defined by $A/\lambda A$ and $A/\mu A$ respectively. Then $(D+E)\cap U$ is given by $A/\lambda\mu A$. If $U'\cong \Spec A'$ is a small affine open subset of $C'$ such that the image of $U'$ under $C'\morphism C$ is contained in $U$, then $D'\cap U$, $E'\cap U$, and $(D'+E')\cap U$ are given by $A'/\lambda A'$, $A'/\mu A'$, and $A'/\lambda\mu A'$ respectively, proving that $D+E$ is mapped to $D'+E'$.
		\item Since $D$ is locally free over $S$ -- because it is finite and flat (and finitely presented in the non-noetherian case) over $S$ -- it's easy to check that the degree is preserved under base change, i.e., $\deg(D'/S')=\deg(D/S)$.
	\end{alphanumerate}
	\end{rem}
\begin{lem}\lbl{lem:fppfLocallySumOfSections}
	If $D\subseteq C$ is an effective relative Cartier divisor of constant degree $d$, then there is an fppf morphism $S'\morphism S$ such that $D'=D\times_SS'\subseteq C\times_SS'=C'$ is given by $\sum_{i=1}^d\sigma_i(S')$, where $\sigma_i\colon S'\morphism C'$ are sections of $f'\colon C'\morphism S'$.
\end{lem}
\begin{proof}
	We do induction on $d=\deg(D/S)$. If $d=0$, then $D=\emptyset$ and the assertion becomes trivial. If $d\geq 1$, then $D$ is faithfully flat over $S$, since $S$ may be covered by affine open subsets $U\cong \Spec R$ such that $f|_D^{-1}(U)\cong \Spec A$ is affine as well and $A$ is finite free of rank $d\geq 1$ over $R$, so that $\Spec A\morphism\Spec R$ is surjective. Hence base changing along $S'=D\morphism S$ is an fppf base change. Then $f'\colon C'=C\times_SS'\morphism S'$ has a section $\sigma=(i_D,\id_D)$, where $i_D\colon D\monomorphism C$ is the inclusion of the closed subprescheme $D$. Clearly $\sigma$ is a closed embedding (as a base change of $i_D$) and defines a degree-$1$  effective relative Cartier divisor for $f'$ since the identity on $S'=D$ is obviously finite flat of rank $1$. Moreover, $\sigma$ factors over $D'=D\times_SS'$. Hence $D'-\sigma(S')$ is an effective relative Cartier divisor of degree $d-1$, to which the induction assumption (together with Remark~\reff{rem:CartierBaseChange}\itememph{c}) may be applied.
\end{proof}
\begin{rem}
If the degree $d=\deg (D/S)$ is only locally constant, then (by the above argument) there is always an fppf-cover $\{S'_i\morphism S\}_{i\in I}$ such that each $D_i'=D\times_SS'_i$ is a sum of sections of $f'_i\colon C'_i=C\times_SS'_i\morphism S'_i$.
\end{rem}
\begin{lem}\lbl{lem:fppfBaseChange}
	Let $S$ be any prescheme, $s\in S$ a point and $\ell$ a finite field extension of $k=\KK(s)$.
	\begin{alphanumerate}
		\item There are an affine open neighbourhood $U\subseteq S$ of  $s$ and an fppf morphism $U'\morphism U$ such that there is a point $s'\in U'$ above $s$ with residue field $\KK(s')=\ell$. Moreover, we may require that $s'$ is closed in its fibre over $s$.
		\item Let $X\morphism S$ be a prescheme of locally finite type over $S$ and $x\in X$ a point over $s$ which is closed in its fibre. Then $\ell=\KK(x)$ is a finite extension of $k=\KK(s)$. If $U'\morphism U$ and $s'\in U'$ are as in \itememph{a}, then any point $x'\in X\times_SU'$, whose images in $X$ and $U'$ are $x$ and $s'$ respectively, has residue field $\KK(x')=\ell$.
	\end{alphanumerate}
	\end{lem}
\begin{proof}
	Part~\itememph{a}. Choose any affine open neighbourhood $U\cong \Spec R$ and presentation
	\begin{align*}
		\ell\cong k[X_1,\ldots,X_n]/\mm
	\end{align*}
	for some maximal ideal $\mm$. Then $U'=\IA_R^n\morphism U$ is an fppf morphism. We get a morphism $\phi\colon R[X_1,\ldots,X_n]\morphism\ell$ sending each $X_i$ to its image in $\ell$. Then $\ker \phi$ is a prime ideal, being the preimage of the prime ideal $(0)\subseteq\ell$. If $s'\in U'$ is the point corresponding to $\ker \phi$, then $s'$ lies over $s$ since the restriction $\phi|_R$ factors as $\phi|_R\colon R\morphism k\monomorphism \ell$, so $\ker\phi\cap R=\ker(R\morphism k)$. Moreover $\KK(s')=\ell$ by construction, hence $U'\morphism U$ and $s'$ have the required properties.
	
	Part~\itememph{b}. This part wasn't in the lecture, but it's worthwhile to mention since we will need it soon. First note that if $x'$ is closed in its fibre $X_s$, which is of locally finite type over $k=\KK(s)$ (since $X$ has locally finite type over $S$), then $x'$ is given by a maximal ideal of some affine open subset of $X_s$, hence $\KK(x')$ has finite type over $k$, hence it is a finite field extension by Hilbert's Nullstellensatz. By \cite[Proposition~1.3.2]{alggeo1}, the points $x'$ over $x$ and $s'$ are given by the prime ideals of $\Oo_{X,x}\otimes_{\Oo_{S,s}}\Oo_{U',s'}$ containing both the preimage of $\mm_{X,x}$ and the preimage of $\mm_{U',s'}$. That is, the $x'$ we are interested in are given by the prime ideals of $\KK(x')\otimes_{\Oo_{S,s}}\KK(s')\cong \ell\otimes_k\ell$.
	
	So we need to show that for finite field extensions $\ell/k$, every prime ideal of $\ell\otimes_k\ell$ has residue field $\ell$. Choose a filtration $k=\ell_0\subseteq \ell_1\subseteq\ldots\subseteq\ell_r=\ell$ such that each $\ell_i/\ell_{i-1}$ is generated by a single element. Without loss of generality let $\ell_1=k^\sep\cap \ell$ be the separable closure of $k$ in $\ell$ and $\ell_i/\ell_{i-1}$ for $i\geq 2$ is generated by an element $x_i$ whose minimal polynomial over $\ell_{i-1}$ is $T^{p^{n_i}}-y_i$ for some $n_i\geq 1$ and $y_i\in\ell_{i-1}$ (this works since $[\ell:\ell_1]$ is a power of $p=\cha k$). Then
	\begin{align*}
		\ell\otimes_ k\ell\cong \ell\otimes_k\ell_1\otimes_{\ell_1}\ell_2\otimes_{\ell_2}\cdots\otimes_{\ell_{r-1}}\ell_r
	\end{align*}
	If $x_1$ with minimal polynomial $f$ generates $\ell_1/k$, then $\ell\otimes_k\ell_1\cong \ell\otimes_kk[T]/(f)\cong \ell[T]/(f)$, which is isomorphic to $\ell^{\oplus[\ell_1:k]}$ by the Chinese remainder theorem, since $f$ splits into distinct linear factors over $\ell$. So it suffices to show that all residue fields of $\ell\otimes_{\ell_1}\ell_2\otimes_{\ell_2}\cdots\otimes_{\ell_{r-1}}\ell_r$ are equal to $\ell$. Using induction on $r$, it's easy to see that 
	\begin{align*}
		\ell\otimes_{\ell_1}\ell_2\otimes_{\ell_2}\cdots\otimes_{\ell_{r-1}}\ell_r\cong \ell[X_2,\ldots,X_r]/\left(X_2^{p^{n_2}}-y_2,\ldots,X_r^{p^{n_r}}-y_r\right)\;.
	\end{align*}
	In particular, $\ell\otimes_{\ell_1}\ell_2\otimes_{\ell_2}\cdots\otimes_{\ell_{r-1}}\ell_r$ has a nilpotent maximal ideal given by $(X_2-x_2,\ldots,X_r-x_r)$, hence it is a zero dimensional local ring with residue field $\ell$, as claimed.
\end{proof}
Now we are going to prove the final theorem of the lecture. Since the statement is quite long, we will break it into parts and alternate between proofs and more parts of the theorem.
\begin{thm}[beginning]\lbl{thm:final}
	Let $f\colon C\morphism S$ be a flat family of curves with $S$ locally noetherian. Let $g\geq 0$. Moreover, let (for all parts of the theorem) $V\subseteq C$ be an affine open subset for which there is an affine open subset $W\subseteq S$ such that $f(V)\subseteq W$.
	\begin{alphanumerate}
		\item $X=C^{(g)}\morphism S$ exists and is proper and flat. Its formation commutes with base change $S'\morphism S$, and its geometric fibres $X_{\ov{s}}=X\times_S\Spec \ov{\KK(s)}$ are regular. Moreover, the $V^{(g)}\subseteq C^{(g)}$ are affine open subsets covering $C^{(g)}$, and $\Oo_{C^{(g)}}(V^{(g)})$ is generated over $\Oo_S(W)$ by elements of the form $\sigma_j(\lambda)$ (following Convention~\reff{con:sigmaja})  for $\lambda\in \Oo_C(V)$.
		\item The morphism $C^g\morphism C^{(g)}$ is finite flat of degree $g!$. 
	\end{alphanumerate}
\end{thm}
Before we dive into the proof, let us discuss a lemma on how to compute the local rings of $C^g$. This was just stated without proof in the lecture, but it really wasn't that obvious to me.
\begin{lem}\lbl{lem:completionOfStalks}
	Let $x\in C^g$ be a point. Let $x_1,\ldots,x_g\in C$ be its projections to the factors and $s\in S$ its image in $S$. If $\KK(x)=\KK(s)$, then
	\begin{align*}
		\roof{\Oo}_{C^g,x}\cong \roof{\Oo}_{S,s}\llbracket X_1,\ldots,X_g\rrbracket\;.
	\end{align*}
\end{lem}
\begin{proof}
	Put $\KK(x)=\KK(s)=k$. Since $C$ is smooth of relative dimension $1$ over $S$ (because its geometric fibres are regular curves, hence Proposition~\reff{prop:smoothnessCriteria} applies), we find an affine open neighbourhood $V\subseteq S$ of $s$, and for every $x_i$ an open neighbourhood $U_i$ such that $f(U_i)\subseteq V$ and $f|_{U_i}$ factors over an étale morphism $\phi_i\colon U_i\morphism\IA_V^1$. Note that $\KK(x_i)=k$ since it is sandwiched between $\KK(x)$ and $\KK(s)$. Let $y_i\in \IA_V^1$ be the image of $x_i$. Then $\Oo_{\IA_V^1,y_i}$ is a localization of $\Oo_{S,s}[X_i]$ at some prime ideal over $\mm_{S,s}$. Hence $y_i$ corresponds to a prime ideal of $\Oo_{S,s}[X_i]/\mm_{S,s}\Oo_{S,s}[X_i]\cong k[X_i]$. But also $\KK(y_i)=k$ (since it is sandwiched between $\KK(x_i)$ and $\KK(s)$), so an easy argument shows that $y_i$ corresponds to a prime ideal of the form $\mm_{S,s}\Oo_{S,s}[X_i]+(X_i-\alpha_i)$ for some $\alpha_i\in\Oo_{S,s}$. Shifting by $\alpha_i$ we may assume $\alpha_i=0$.
	
	By Lemma~\reff{lem:etaleProperties}\itememph{b}, the morphism $\phi\colon U_1\times_V\cdots\times_VU_g\morphism(\IA_V^1)^g\cong \IA_V^g$ is étale again. We claim that $\IA_V^g$ has a unique point with projections $y_1,\ldots,y_g$. Indeed, by \cite[Proposition~1.3.2]{alggeo1} we need to show that $\bigotimes_{i=1}^g\Oo_{\IA_V^1,y_i}$ has a unique prime ideal containing the images of the $\mm_{\IA_V^1,y_i}$. But if we quotient out these images one after another, we obtain 
	\begin{align}\lbl{eq:manyks}
		\KK(y_1)\otimes_{\Oo_{S,s}}\cdots \otimes_{\Oo_{S,s}}\KK(y_g)\cong k\otimes_k\cdots\otimes_kk\cong k
	\end{align}
	hence the ideal generated by the images of the $\mm_{\IA_V^1,y_i}$ is already maximal. In particular, there is a unique point $y$ over $y_1,\ldots,y_n$, and $\KK(y)=k$ again. Then necessarily $\phi(x)=y$, and from
	Lemma~\reff{lem:etaleCompletion} we obtain
	\begin{align}\lbl{eq:completion1}
		\roof{\Oo}_{C^g,x}\cong \roof{\Oo}_{\IA_V^g,y}\;.
	\end{align}
	But each $\mm_{\IA_V^1,y_i}$ corresponds to the maximal ideal $\mm_{S,s}\Oo_{S,s}[X_i]+(X_i)\subseteq \Oo_{S,s}[X_i]$, and their images generate the maximal ideal of $\bigotimes_{i=1}^g\Oo_{\IA_V^1,y_i}$ corresponding to $y$. Thus it's easy to see that
	\begin{align}\lbl{eq:completion2}
		\roof{\Oo}_{\IA_V^g,y}\cong \roof{\Oo}_{S,s}\llbracket X_1,\ldots,X_g\rrbracket\;.
	\end{align}
	Combining \eqreff{eq:completion1} and \eqreff{eq:completion2} proves the assertion.
\end{proof}
\begin{con}\lbl{con:sorted}
	Throughout the proof, let us fix the following notation: Let $x\in C^g$ and $\{x_1,\ldots,x_\ell\}$ the set (!) of projections of $x$ to the $g$ factors (in particular, the sequence of $x_i$ is free of repetition). We call $x$ a \emph{sorted} point if the sequence of projections (now including repetitions) is $(x_1,\ldots,x_1,x_2,\ldots,x_2,\ldots,x_\ell,\ldots,x_\ell)$ in this order.
\end{con}
\begin{proof}[Proof of Theorem~\reff{thm:final}\itememph{a} and \itememph{b}.]
	Part~\itememph{a}. First note that any finite subset of any fibre of $f\colon C\morphism S$ is contained in an affine open subset of $C$, since we assume $f$ (and thus its fibres) to be projective, so \cite[Proposition~2.2.1\itememph{b}]{alggeo2} is applicable. Hence Proposition~\reff{prop:symmetricPowers} may be applied, which shows everything except separatedness and universal closedness of $C^{(g)}\morphism S$ (which we need for properness), and the regularity assertion. 
	
	Let's prove separatedness first. Let $y_1,y_2\in X^{(g)}$ be two points in the same fibre over $S$ and let $x_1,x_2\in X^g$ be preimages. Then the $2g$ projections of $x_1$ and $x_2$ to the $g$ factors are contained in some affine open subset $V\subseteq C$ such that $f(V)\subseteq W$ for some affine open subset $W\subseteq S$. Thus $U=(V/W)^{(g)}$ is an affine open neighbourhood of $y_1, y_2$ in $C^{(g)}$. This shows that $C^{(g)}\times_SC^{(g)}$ can be covered by affine open subsets of the form $V\times_WV$. Indeed, just consider any point $y$ and let $y_1,y_2\in C^{(g)}$ be its projections to the two factors. Then apply the above. By an easy argument, we thus get that the diagonal $\Delta\colon C^{(g)}\morphism C^{(g)}\times_SC^{(g)}$ is a closed embedding, proving separatedness by \cite[Fact~1.5.7\itememph{c}]{alggeo1}. For universal closedness, it suffices to show closedness of $C^{(g)}\morphism S$ since the construction of $C^{(g)}$ is compatible with base change. But $C^{(g)}$ carries the quotient topology from $C^g$, hence closedness follows from closedness of $C^g\morphism S$.
	
	Now we prove the regularity assertion. By compatibility with base change, what we need to show is that $C^{(g)}$ is regular over $k$ in the special case where $S=\Spec k$ with $k$ algebraically closed. Then $C^{(g)}$ has finite type over $k$, hence it is Jacobson (cf.\ \cite[Definition~2.4.2]{alggeo1}), and thus by Serre's characterization of regular rings (or more precisely, by \cite[Corollary~2.2.1]{homalg}) it suffices to show regularity at closed points.
	
	If $y\in C^{(g)}$ is a closed point (i.e., corresponds to some maximal ideal affine-locally), then $\KK(y)$ has finite type over $k$ since $C^{(g)}$ has finite type over $k$ too. But $k$ is algebraically closed, hence $\KK(y)=k$ by Hilbert's Nullstellensatz. Moreover, $C^g$ is finite over $C^{(g)}$ by Theorem~\reff{thm:QuotientByGroup}\itememph{b}. Hence it's easy to see (but not entirely trivial; you need some artinianness stuff) that for all points $x$ over $y$ the residue field $\KK(x)$ is finite over $\KK(y)$, hence $\KK(x)=k$ too. Now choose such an $x$ and let $\{x_1,\ldots,x_\ell\}$ be the (repetition-free) set of projections of $x$ to the $g$ factors. Assume that each $x_i$ occurs $j_i$ times, such that $j_1+\ldots+j_\ell=g$.
	
	Without losing generality we may assume that $x$ is \emph{sorted} (using the terminology from Convention~\reff{con:sorted}). We claim that the stabilizer group of $x$ under $\SS_n$ is given by $G_x=\SS_{j_1}\times\cdots\times\SS_{j_\ell}$. Indeed, note that $\KK(x_i)=k$ since $\KK(x_i)$ is sandwiched between $\KK(x)$ and $k$. Hence, by an argument like \eqreff{eq:manyks}, every permutation of the $x_i$ (with repetitions) determines a unique point in $C^g$, which shows $G_x=\SS_{j_1}\times\cdots\times\SS_{j_\ell}$ as claimed. Therefore, using Theorem~\reff{thm:QuotientByGroup}\itememph{b} and Lemma~\reff{lem:completionOfStalks}, we obtain
	\begin{align}\lbl{eq:OCgx}
		\roof{\Oo}_{C^{(g)},y}\cong (\roof{\Oo}_{C^g,x})^{G_x}\cong k\llbracket X_1,\ldots,X_g\rrbracket^{\SS_{j_1}\times\cdots\times\SS_{j_\ell}}\;,
	\end{align}
	Now let $r_i=j_1+\ldots,j_i$ and put $S_{i,j}=\sigma_j\left(X_{r_{i-1}+1},\ldots,X_{r_i}\right)$, where $\sigma_j$ denotes the $j\ordinalth$ elementary symmetric polynomial, as usual. 
	It's easy to generalize the fundamental theorem of symmetric polynomials (Example~\reff{ex:symmetricPolynomials}\itememph{a}) in such a way that it can be applied to the right-hand side of \eqreff{eq:OCgx}. This gives
	\begin{align}\lbl{eq:OCgx2}
		\roof{\Oo}_{C^{(g)},y}\cong k\llbracket S_{1,1},\ldots,S_{1,j_1},\ldots,S_{\ell,1},\ldots,S_{\ell,j_\ell}\rrbracket\;.
	\end{align}
	In particular, the right-hand side is a power series ring over a field, hence regular (e.g., by \cite[Proposition~2.2.3]{homalg}). This finally shows that $C^{(g)}$ is regular.	
	
	Part~\itememph{b}. Finiteness follows from Proposition~\reff{prop:symmetricPowers}\itememph{c}. For flatness, we will do three reduction steps to arrive at a sufficiently nice situation.
	
	\emph{Step~1.} We claim it is sufficient to show that $\Oo_{C^g,x}$ is flat over $\Oo_{C^{(g)},y}$ whenever $x$ lies over $y$ and $x$ is closed in its fibre over $S$. The first condition here is the usual one from Definition~\reff{def:flat}\itememph{c}. But why the second condition? Note that every maximal ideal of an affine open subset of $C^g$ corresponds to a point which is closed in its fibre, since the fibres are Jacobson (being of finite type over a field), hence closedness may be tested locally by \cite[Fact~2.4.1\itememph{c}]{alggeo1}. And since flatness can be tested at maximal ideals, it is indeed ok to restrict to $x$ which are closed in their fibre over $S$.
	
	\emph{Step~2.} Let $x\in C^g$ have this property, and let $y\in C^{(g)}$ and $s\in S$ be the images of $x$. Since $C^g$ has finite type over $S$ and $x$ is given by a maximal ideal of some open neighbourhood in its fibre, we see that $\KK(x)$ is a field of finite type over $\KK(s)$, hence a finite extension of $\KK(s)$ by Hilbert's Nullstellensatz. Hence, by Lemma~\reff{lem:fppfBaseChange}, we may take an fppf base change to obtain $\KK(x)=\KK(s)\eqqcolon k$ (and also flatness may be tested after fppf base change, using the silly flatness criterion from Lemma~\reff{lem:sillyFlatness}, so that's ok). 
	
	\emph{Step~3.} Now, since completion is faithfully flat by Corollary~\reff{cor:completionFaithfullyFlat}\itememph{c}, it suffices to show that $\roof{\Oo}_{C^g,x}$ is flat over $\roof{\Oo}_{C^{(g)},y}$. Let $\{x_1,\ldots,x_\ell\}$ be the set of projections of $x$ and $j_i$ the multiplicity of $x_i$. Also let's assume without loss of generality that $x$ is a sorted point (as in Convention~\reff{con:sorted}). Since each $\KK(x_i)$ is sandwiched between $\KK(x)$ and $\KK(s)$, we see $\KK(x_i)=k$. From the same argument we get $\KK(y)=k$. Hence we find ourselves in a situation where Lemma~\reff{lem:completionOfStalks} is applicable. Together with arguments analogous to \eqreff{eq:OCgx} and \eqreff{eq:OCgx2} we thus get
	\begin{align}\lbl{eq:OCgx3}
		\roof{\Oo}_{C^g,x}\cong \roof{\Oo}_{S,s}\llbracket X_1,\ldots,X_g\rrbracket\quad\text{and}\quad\roof{\Oo}_{C^{(g)},y}\cong \roof{\Oo}_{S,s}\llbracket S_{1,1},\ldots,S_{1,j_1},\ldots,S_{\ell,1},\ldots,S_{\ell,j_\ell}\rrbracket\;.
	\end{align}
	Adapting Example~\reff{ex:symmetricPolynomials}\itememph{b} to our situation proves shows that $\roof{\Oo}_{C^g,x}$ is free over $\roof{\Oo}_{C^{(g)},y}$ with a basis given by $X_1^{\alpha_1}\cdots X_g^{\alpha_g}$, where $0\leq \alpha_j<j-r_{i-1}$ whenever $r_{i-1}<j\leq r_i$ (and as in the proof of part~\itememph{a} we put $r_i=j_1+\ldots+j_i$). This shows flatness.
	
	We still need to show that the degree is $g!$. Note that the above basis has only $j_1!\cdots j_\ell!$ elements, but this is compensated by the fact that there are $\frac{g!}{j_1!\cdots j_\ell!}$ distinct permutations of the $x_i$, each having multiplicity $j_i$. Let's make this argument precise! First note that the degree of a finite flat morphism is preserved under base change. Hence we may assume that we are in the situation from Step~3 again. Let $U\cong \Spec B$ be a $\SS_g$-invariant affine open neighbourhood of the preimage of $y$ (i.e., the orbit of $x$) and let $V\cong \Spec A$ for $A=B^{\SS_g}$ be the corresponding affine open neighbourhood of $y$. Let $\pp\in \Spec A$ be the prime ideal corresponding to $y$ and $\qq_1,\ldots,\qq_m\in\Spec B$ the prime ideals over $\pp$ (i.e., the orbit of $x$ by the proof of Lemma~\reff{lem:ringInvariants}). Then the proof of Theorem~\reff{thm:QuotientByGroup}\itememph{b} (or more precisely, equation \eqreff{eq:ChineseRemainderTheorem} together with the discussion after \eqreff{eq:OXxG}) shows that
	\begin{align}\lbl{eq:BpBqi}
		\roof{B}_\pp\cong \prod_{i=1}^m\roof{B}_{\qq_i}\;,
	\end{align}
	where $\roof{B}_\pp$ on the left-hand side is the $\pp B_\pp$-adic completion, whereas $\roof{B}_{\qq_i}$ on the right-hand side denotes the $\qq_iB_{\qq_i}$-adic one. Now $m=\#(\SS_g/G_x)=\#(\SS_g/\SS_{j_1}\times\cdots\times\SS_{j_\ell})=\frac{g!}{j_1!\cdots j_\ell!}$ and each $\roof{B}_{\qq_i}$ is free of rank $j_1!\cdots j_\ell!$ over $\roof{A}_\pp$ by Step~3. Therefore, the $\pp B_\pp$-adic completion $\roof{B}_\pp\cong B_\pp\otimes_{A_\pp}\roof{A}_\pp$ is free of rank $g!$ over $\roof{A}_\pp$. Since completion is faithfully flat, this shows that the rank of $B_\pp$ over $A_\pp$ is $g!$ as well, and we are done.
\end{proof}
\setcounter{thm}{11}
\begin{thm}[continuation]
	Let $\pi\colon C^g\morphism C^{(g)}$ be the finite flat projection from \itememph{b}. Let $\varsigma_i\colon C^g\morphism C^g\times_SC$ for $i=1,\ldots,g$ be given $\varsigma_i=(\id_{C^g},p_i)$, where $p_i\colon C^g\morphism C$ denotes the projection to the $i\ordinalth$ factor. Consider the relative (with respect to $C^g\times_SC\morphism C^g$) effective Cartier divisor $\snake{D}\subseteq C^g\times_SC$ of degree $g$ given by
	\begin{align*}
		\snake{D}=\sum_{i=1}^g\varsigma_i(C^g)\;.
	\end{align*}
	\begin{alphanumerate}\setcounter{enumi}{2}
		\item Then $\snake{D}$ descends to a unique effective Cartier divisor $D\subseteq C^{(g)}\times_SC$ of degree $g$.
		\item Let $D$ be as in \itememph{c} and $V\subseteq C$ an affine open subset as in the beginning of Theorem~\reff{thm:final}. Then $D\cap (V^{(g)}\times_SC)\subseteq V^{(g)}\times_SV$.
	\end{alphanumerate}
\end{thm}
\begin{proof}
	Part~\itememph{c}. Since $\pi$ is fpqc (even fppf) by part~\itememph{b}, one is tempted to use the machinery of faithfully flat descent here, but this could get nasty. Instead we use a direct approach.
	
	 Let $U\subseteq C^g\times_SC$ be an $\SS_g$-invariant affine open subset, and let $\Jj$ be the sheaf of ideals giving $\snake{D}$ on  $U$. Let $\Jj'\subseteq \Jj$ be its subsheaf generated by sections which are contained in the image of $\pi^*\colon \Oo_{C^{(g)}\times_SC}(U/\SS_g)\morphism\Oo_{C^g\times_SC}(U)$. If we show $\Jj'=\Jj$, then $\Jj$ is given by $\Jj=\pi^*\Jj_0$ for some sheaf of ideals $\Jj_0$ on $U/\SS_g$. To test whether $D=V(\Jj_0)$ is a  effective relative Cartier divisor for $C^{(g)}\times_SC\morphism C^{(g)}$, we may check the two conditions from Lemma~\reff{lem:CartierDivisor}\itememph{a}. By descent of line bundles (Lemma~\reff{lem:descentIsALineBundleAgain}), $\Jj_0$ is a line bundle again, so the first condition holds. Also the intersection of $V(\Jj_0)$ with each fibre of $C^{(g)}\times_SC\morphism C^{(g)}$ is finite since we know the same is true for $V(\Jj)=\snake{D}$, since $\snake{D}$ is a  effective relative Cartier divisor for $C^g\times_SC\morphism C^g$.
	 
	 To prove $\Jj'=\Jj$, we may check this on stalks, and as in Step~1 of \itememph{b}, it is sufficient to consider points $x$ which are closed in their fibre over $S$. Moreover, $\Jj'_x=\Jj_x$ may be tested after a faithfully flat base change. Hence, as in Step~2 and 3 of \itememph{b} we may take fppf base change to ensure that all appearing residue fields coincide, and then take completions everywhere. Let $\{x_1,\ldots,x_\ell\}$ be the projections of $x\in C^g\times_S C$ to the first $g$ factors, each $x_i$ appearing with multiplicity $j_i$, and let $z$ be the projection to the last factor. As usual, we assume that $x$ is sorted (Convention~\reff{con:sorted}) in its first $g$ coordinates. If $y$ denotes the image of $x$ in $C^{(g)}\times_SC$, then as in \eqreff{eq:OCgx3}, we obtain
	 \begin{align}\lbl{eq:OCgx4}
	 	\roof{\Oo}_{C^g\times_SC,x}\cong \roof{\Oo}_{S,s}\llbracket X_1,\ldots,X_g,Z\rrbracket\quad\text{and}\quad\roof{\Oo}_{C^{(g)}\times_SC,y}\cong \roof{\Oo}_{S,s}\llbracket S_{1,1},\ldots,S_{\ell,j_\ell},Z\rrbracket\;.
	 \end{align}
	 The additional variable $Z$ represents the fact that the $\SS_g$-action only permutes the first $g$ factors.
	 
	 If $z\notin\{x_1,\ldots,x_\ell\}$, then $x$ is not in the image of any $\varsigma_i$. Since the $\varsigma_i$ are closed embeddings (being sections of the proper and thus separated morphism $C^g\times_SC\morphism C^g$, so \cite[Proposition~1.5.5]{alggeo1} applies), we obtain $\Jj_x=\Oo_{C^g\times_SC,x}$ in this case, hence $\Jj_x$ and $\Jj'_x$ are both generated by $1$, and the assertion is true. 
	 
	 Otherwise, assume w.l.o.g.\ $x'=x_1$, so that $x$ is in the image of $\varsigma_1,\ldots,\varsigma_{j_1}$, but not in the image of any $\varsigma_i$ for $i>j_1$. Let's describe $\Jj_x$ in this case! Let $x'$ be the image of $x$ under $C^g\times_SC\morphism C^g$, so that $x=\varsigma_j(x')$ for all $j=1,\ldots,j_1$. It's easy to see\footnote{\lbl{footnote:varsigmai}Indeed, let $V\cong \Spec A$ and $W\cong \Spec R$ be affine open neighbourhoods of $C$ and $S$ respectively, such that $f(V)\subseteq W$. Then the projection $p_i\colon C^g\morphism C$ corresponds to $\iota_i\colon A\morphism A^{\otimes g}$ defined by $\iota_i(a)=1\otimes\cdots\otimes a\otimes\cdots\otimes 1$, with $a$ occuring in the $i\ordinalth$ position. Hence $\varsigma_i^*\colon A^{\otimes g}\otimes_RA\morphism A^{\otimes g}$ is given by
	 \begin{align*}
	 	\varsigma_i^*(a_1\otimes\cdots\otimes a_g\otimes b)=a_1\otimes\cdots\otimes ba_i\otimes\cdots \otimes a_g\;.
	 \end{align*}	
	 This coincides with what we claim.} that the induced morphism on completions of stalks
	 \begin{align*}
	 	\varsigma_j^*\colon \roof{\Oo}_{C^g\times_SC,x}\cong \roof{\Oo}_{S,s}\llbracket X_1,\ldots,X_g,Z\rrbracket\morphism\roof{\Oo}_{S,s}\llbracket X_1,\ldots,X_g\rrbracket\cong \roof{\Oo}_{C^g,x'}
	 \end{align*}
	 is given by $Z\mapsto X_j$ (and everything else stays fixed). Hence $\ker\varsigma_j^*$ is the principal ideal generated by $(Z-X_j)$.
	 
	 Since $\roof{\Jj}_x=\Jj_x\roof{\Oo}_{C^g\times_SC,x}$ is given by $\prod_{j=1}^{j_1}\ker\varsigma_j^*$ (this is clear from the construction in Remark~\reff{rem:CartierSemigroup}), we conclude that $\roof{\Jj}_x$ is the principal ideal generated by
	 \begin{align*}
	 	\prod_{j=1}^{j_1}(Z-X_j)=Z^{j_1}+\sum_{j=1}^{j_1}(-1)^j\sigma_j(X_1,\ldots,X_{j_1})Z^{j_1-j}=Z^{j_1}+\sum_{j=1}^{j_1}(-1)^jS_{1,j}Z^{j_1-j}\;.
	 \end{align*}
	 Hence the generator is an element of $\roof{\Oo}_{C^{(g)}\times_SC,y}$, and moreover it only depends on $y$, but not on the choice of a preimage $x$. This shows $\vphantom{\widehat{\Jj}}\roof{\Jj}'_x=\roof{\Jj}_x$ and we are done, as explained above.
	 
	 Part~\itememph{d}. From the definition of the $\varsigma_i$ it is clear that $\varsigma_i(V^g)\subseteq V^g\times_SV$. Hence we get $\snake{D}\cap (V^g\times_SC)\subseteq V^g\times_SV$, and since $\snake{D}$ is the preimage of $D$ in $C^g\times_SC$, the assertion becomes rather obvious.
\end{proof}
Now let $V$ and $W$ be affine open subsets of $C$ and $S$ respectively, such that $f(V)\subseteq W$. Let $T$ be an affine scheme and $\tau\colon T\morphism W$ a morphism, making $T$ into an $S$-prescheme. For a morphism $t\in \Hom_{\cat{PSch}/S}(T,V^{(g)})$ let $D_t$ be the pullback of the Cartier divisor $D$ from \itememph{c} under the morphism
\begin{align*}
	(t,\id_C)\colon T\times_SC\morphism V^{(g)}\times_SC\monomorphism C^{(g)}\times_SC\;.
\end{align*}
From \itememph{d} we get $D_t\subseteq T\times_SV$. We would like to show that $t$ is uniquely determined by $D_t$. This will be done using the following technical result.
\setcounter{thm}{11}
\begin{thm}[continuation]
	Let $p_2\colon T\times_SV\morphism V$ be the projection to the second factor, and $q\colon D_t\morphism T$ the projection to the first one.
	\begin{alphanumerate}\setcounter{enumi}{4}
		\item Let $\lambda\in\Oo_C(V)$. The pullback $p_2^*\lambda$ of $\lambda$ acts by multiplication on $q_*\Oo_{D_t}$, which is a vector bundle on $T$ since $D_t$ -- being a Cartier divisor -- is finite flat over $T$. Then the characteristic polynomial of $p_2^*\lambda$ over $\Oo_T$ is given by
		\begin{align*}
			\chi=X^g+\sum_{j=1}^g(-1)^jt^*(\sigma_j(\lambda))X^{g-j}\;.
		\end{align*}
		This implies that $t$ is uniquely determined by $D_t$.
	\end{alphanumerate}
\end{thm}
\begin{proof}
	It's easy to see that the characteristic polynomial commutes with arbitrary base change, and so does the above expression for $\chi$. Therefore it's ok to prove the assertion after performing an fppf base change  $t'\colon T'\morphism T$ (since it's easy to check that $\Oo_T\morphism t'_*\Oo_{T'}$ is injective). Thus, using \itememph{b}, we may replace $T$ by $T\times_{V^{(g)}}V^g$. Hence we may assume that $t\colon T\morphism V^{(g)}$ factors over $V^g$. In this case $D_t$ is also a pullback of $\snake{D}$. Since, again, $\chi$ doesn't change under base change, it therefore suffices to consider the case where $T=V^g$, $t=\pi$, and $D_t=\snake{D}$.
	
	 Put $V\cong \Spec A$ and $W\cong \Spec R$. Then the morphisms $\varsigma_i^*\colon A^{\otimes g}\otimes_RA\morphism A^{\otimes g}$ are given by $\varsigma_i^*(a_1\otimes\cdots\otimes a_g\otimes b)=a_1\otimes\cdots \otimes ba_i\otimes\cdots\otimes a_g$, as explained in the footnote on page~\pageref{footnote:varsigmai}. Let $J_i=\ker\varsigma_i^*$ and put $J=\prod_{i=1}^g$.
	 
	 Then $(A^{\otimes g}\otimes_RA)/J$ is the finite locally free $A^{\otimes g}$-module we are interested in, and $p_2^*\lambda=1\otimes\cdots\otimes 1\otimes \lambda$ acts on it by multiplication. We need to show that its characteristic polynomial is given by
	 \begin{align*}
	 	\chi=X^g+\sum_{j=1}^g(-1)^j\sigma_j(\lambda)X^{g-j}=X^g+\sum_{j=1}^g(-1)^j\sigma_j\big(\iota_1(\lambda),\ldots,\iota_g(\lambda)\big)X^{g-j}=\prod_{j=1}^g\big(X-\iota_j(\lambda)\big)\;,
	 \end{align*}
	 wherein the middle equality comes from unraveling the definition of $\sigma_j(\lambda)$ in Covention~\reff{con:sigmaja}. In particular, it suffices to show that each $\iota_i(\lambda)$ is an eigenvalue of $p_2^*\lambda$.
	 
	 To construct such eigenvectors, consider the short exact sequence
	 \begin{align*}
	 	0\morphism J_j/J\morphism (A^{\otimes g}\otimes_RA)/J\morphism (A^{\otimes g}\otimes_RA)/J_j\morphism 0\;.
	 \end{align*}
	 The right-most term can be identified with $A^{\otimes g}$ via $\varsigma_j^*$. In particular, the above sequence is split exact as a sequence of $A^{\otimes g}$-modules. Thus $A^{\otimes g}\otimes_RA)/J_j$ is a direct summand of $(A^{\otimes g}\otimes_RA)/J$. Moreover, by definition of $\varsigma_j^*$ multiplication by $p_2^*\lambda=1\otimes\cdots\otimes 1\otimes \lambda$ on $(A^{\otimes g}\otimes_RA)/J_j$ is the same as multiplication by $\iota_j(\lambda)$ as an $A^{\otimes g}$-module. This gives the desired eigenvectors, proving that $\chi$ has the above form.
	 
	 This implies that $t$ is uniquely determined by $D_t$. Indeed, the characteristic polynomial of $p_2^*\lambda$ acting by multiplication on $q_*\Oo_{D_t}$ over $\Oo_T$ only depends on $D_t$. Hence the $t^*(\sigma_j(\lambda))$ are uniquely determined. But $\Oo_{C^{(g)}}(V^{(g)})$ is generated over $\Oo_S(W)$ by elements of the form $\sigma_j(\lambda)$ for $\lambda\in\Oo_C(V)$. This shows that the morphism of affine schemes $t\colon T\morphism V^{(g)}$ is indeed uniquely determined by $D_t$.
\end{proof}
We are now able to prove the highlight of Theorem~\reff{thm:final}. For every $S$-prescheme $T$ put 
\begin{align*}
	\Div_{C/S}^g(T)=\left\{\text{ effective relative Cartier divisors of degree $g$ for }C_T\to T\right\}\;.
\end{align*}
By Remark~\reff{rem:CartierBaseChange}\itememph{a} and \itememph{c}, we use pullbacks of Cartier divisors to make $\Div_{C/S}^g$ into a functor $\Div_{C/S}^g\colon (\cat{PSch}/S)^\op\morphism\cat{Set}$.
\setcounter{thm}{11}
\begin{thm}[end]
	Let $D\in \Div_{C/S}^g(C^{(g)})$ be the  effective relative Cartier divisor from \itememph{c}.
	\begin{alphanumerate}\setcounter{enumi}{5}
		\item The functor $\Div_{C/S}^g$ is represented by $C^{(g)}$, and a functor isomorphism is given by
		\begin{align*}
			\Hom_{\cat{PSch}/S}\left(-,C^{(g)}\right)&\isomorphism\Div_{C/S}^g\\
			\big(t\colon T\to C^{(g)}\big)&\longmapsto D_t\;.
		\end{align*}
	\end{alphanumerate}
\end{thm}
\begin{proof}
	The key observation is that both sides are actually sheaves in the fppf topology! Indeed, for $\Hom_{\cat{PSch}/S}(-,C^{(g)})$ this was proved in Proposition~\reff{prop:fpqcSheaf}. For $\Div_{C/S}^g$, what we need to prove is that  effective relative Cartier divisors satisfy fppf descent. But this is an easy corollary of faithfully flat descent (Theorem~\reff{thm:faithfullyFlatDescent}) together the fact that descent of line bundles preserves line bundles (Lemma~\reff{lem:descentIsALineBundleAgain}) and descent of sheaves of ideals preserves sheaves of ideals (easy to see from faithful flatness). It's also easy to see that the isomorphism in question is a morphism of fppf-sheaves.
	
	In particular, we have a morphism of Zariski sheaves. Thus by \itememph{e} (and the fact that $C^{(g)}$ can be covered by such $V^{(g)}$ by \itememph{a}) we see that $\Hom_{\cat{PSch}/S}(-,C^{(g)})\morphism\Div_{C/S}^g$ is a monomorphism. Now let $E\in \Div_{C/S}^g(T)$. We need to show that $E$ is in the image. Since we have already shown injectivity (this is crucial here!), this can be checked fppf-locally. Thus, by Lemma~\reff{lem:fppfLocallySumOfSections}, it suffices to consider the case where $E=\sum_{i=1}^g\sigma_i(T)$ is a sum of sections $\sigma_i$ of $C_T\morphism T$. For $i=1,\ldots,g$ let $t_i$ denote the composition $t_i\colon T\morphism[\sigma_i]C_T\morphism C$. Then it is easily verified that
	\begin{align*}
		t\colon T\xrightarrow{(t_1,\ldots,t_g)} C^g\morphism[\pi]C^{(g)}
	\end{align*}
	provides a preimage of $E$. We are done!
\end{proof}

\section{Construction of the Jacobian in a special case}\lbl{sec:Construction}
The proofs in this final section will be pretty sketchy and only cover a special case, due to the lecture running out of time. However, I think there is a good chance Professor Franke might explain this section in a bit more detail at the beginning of his lecture on the Mordell conjecture next semester! He also plans explaining how the construction of $\PIC_{C/S}$ in the general case can be reduced to the special case that is treated below, via faithfully flat descent (for which a canonical ample line bundle on $\PIC_{C/S}$ is needed, which will be given by the \emph{$\theta$-divisor}). So be sure to attend!

To formulate our results appropriately, we need a relative notion of representability of functors (or fppf-sheaves, actually). This covers only the special case we're interested in; a general notion can be found in \cite[\stackstag{0021}]{stacks-project}.
\begin{defi}\lbl{def:relativeRepresentable}
	Let $\phi\colon\Ff\morphism\Gg$ be a morphism of sheaves (of sets, say) on the big fppf site $(\cat{PSch}/S)_\fppf$.
	\begin{alphanumerate}
		\item We call $\phi$ \defemph{representable} if for every $S$-prescheme $T$ and each $\gamma\in\Gg(T)$ the fppf-sheaf $\Ff\times_\Gg T$ is representable by some $T$-prescheme $X_{\gamma,T}\morphism T$. Here $\Ff\times_\Gg T$ denotes the sheaf on $(\cat{PSch}/S)_\fppf$ given by
		\begin{align*}
			(\Ff\times_\Gg T)(T')=\left\{(f,\tau)\st
		\begin{tabular}{c}
			$f\in \Ff(T')$, $\tau\colon T'\morphism T$ is a morphism of $S$-preschemes,\\
			 and $\phi(f)\in\Gg(T')$ equals the restriction of $\gamma$ along $\tau$
		\end{tabular}\right\}\;.
		\end{align*}
		\item Let $\Pp$ be any property of morphisms of preschemes (like being flat, proper, or an open or closed embedding). We say \defemph{$\boldsymbol{\phi}$ has $\boldsymbol{\Pp}$} if all the $X_{\gamma,T}\morphism T$ have $\Pp$.
		\item Let open immersions (in the sense of \itememph{b}) $\{\phi_i\colon \Ff_i\morphism \Gg\}_{i\in I}$ be given. If, for all $T$ and $\gamma$, the $\{X_{i,\gamma,T}\morphism T\}_{i\in I}$ are an open cover of $T$, then the $\Ff_i$ are said to be an \defemph{open cover} of $\Gg$.
	\end{alphanumerate}
\end{defi}
Now let $f\colon C\morphism S$ be as at the beginning of Section~\reff{sec:curveSymmetricPowers}, and assume, additionally, that all geometric fibres $C_{\ov{s}}$ are curves of genus $g>1$. Also let $f$ be finitely presented or $S$ be locally noetherian. Then for all $S$-preschemes $T$ the canonical morphism $\Oo_T\morphism f_{T,*}\Oo_{C_T}$ is an isomorphism by Proposition~\reff{prop:RpLl}.\footnote{Here we \emph{do not} need a strengthened version that works for finitely presented morphisms or restrict to $T$ which are locally noetherian, thanks to the additional assertion about the base change morphism.} If moreover a section $\sigma\colon S\morphism C$ of $f$ exists (which is always the case fppf-locally), then our situation is nice enough that Example~\reff{ex:sheafifyingPic} provides an elegant description of the fppf-sheafified functor $\Pic_{C/S}$, and thus also of $\Pic_{C/S}^g$, the functor characterizing all line bundles of geometric fibre-wise degree $g$.
\begin{defi}\lbl{def:UCS}
	For all $S$-preschemes $T$ let $U_C(T)\subseteq \Pic^g(C_T)$ be given by the line bundles $\Ll\in\Pic(C_T)$ such that $H^1(C_{\ov{t}},\Ll|_{C_{\ov{t}}})=0$ for all $t\in T$. The fppf-sheafification of $U_C$ gives a subsheaf $U_{C/S}\subseteq \Pic_{C/S}^g$.
\end{defi}
\begin{rem}
	\begin{alphanumerate}
		\item $U_{C/S}(T)$ does \emph{not} consist of line bundles, and not even of isomorphism classes of line bundles -- that's simply not how $\Pic_{C/S}(T)$ works! You should think of the above $\Ll$ as a representative of some residue class modulo $f_T^*\Pic(T)$ instead.
		\item It's not hard to deduce from Theorem~\reff{thm:GrauertGrothendieck}\itememph{b} that $U_{C/S}$ is an open subsheaf of $\Pic_{C/S}^g$ in the sense of Definition~\reff{def:relativeRepresentable}\itememph{b}.
	\end{alphanumerate}
\end{rem}
Recall that Theorem~\reff{thm:final}\itememph{f} shows that $\Div_{C/S}^g$ is represented by $C^{(g)}$. Moreover, by Remark~\reff{rem:degree=degree} we have a morphism $\gamma\colon\Div_{C/S}^g\morphism\Pic_{C/S}^g$, given by sending a  effective relative Cartier divisor $E\in\Div_{C/S}^g(T)$, which is cut out by a sheaf $\Jj_E\subseteq \Oo_{C_T}$ of invertible ideals, to $\Jj_E^{\otimes -1}$. It is common to denote $\Jj_E^{\otimes -1}=\Oo_{C_T}(E)$.
\begin{prop}
	\begin{alphanumerate}
		\item \lbl{prop:UCS}The morphism $\gamma\colon\Div_{C/S}^g\morphism\Pic_{C/S}^g$ is proper and an isomorphism over the open subsheaf $U_{C/S}\subseteq \Pic_{C/S}^g$.
		\item If $T\morphism S$ is a morphism with $T=\Spec k$ the spectrum of a field, then $\gamma$ defines a surjection on $k$-valued points.
	\end{alphanumerate}
 \end{prop}
\begin{proof}[Sketch of a proof]
	Part~\itememph{a}. Let $T$ be an $S$-prescheme; we assume that $T$ is locally noetherian for simplicity (since we proved most results, like the seesaw theorem, only in this case). Let $[\Ll]\in\Pic_{C/S}^g(T)$ an element given by a line bundle $\Ll$ on $C_T$. We need to show that $F=\Div_{C/S}^g\times_{\Pic_{C/S}^g}T$ is representable by some $Z$, and that $Z$ is proper over $T$.
	
	Unwinding Definition~\reff{def:relativeRepresentable}\itememph{a}, we see that $F(T')$ consists of pairs $(E,\tau)$ such that $E\in \Div_{C/S}^g(T')$, $\tau\colon T'\morphism T$ is a morphism of $S$-preschemes, and $[\Oo_{C_{T'}}(E)]=[\tau^*\Ll]$ holds in $\Pic_{C/S}^g(T')$. Since
	\begin{align*}
		\Div_{C/S}^g(T')\cong\Div_{C_T/T}^g(T')\cong \Hom_{\cat{PSch}/T}\left(T',C_T^{(g)}\right)
	\end{align*}
	(the right bijections follows from applying Theorem~\reff{thm:final}\itememph{f} to $T$ instead of $S$), we get $E=D_t$ for some $t\colon T'\morphism C_T^{(g)}$, hence $\Oo_{C_T'}(E)=\Jj_{D_t}^{\otimes -1}$. If we are in a sufficiently nice situation such that Example~\reff{ex:sheafifyingPic} is applicable, then $[\Jj_{D_t}^{\otimes -1}]=[\tau^*\Ll]$ is equivalent to
	\begin{align*}
		\Jj_{D_t}\otimes\tau^*\Ll\cong f_{T'}^*\Ll_0
	\end{align*}
	for some line bundle $\Ll_0$ on $T'$. But now the seesaw theorem (Theorem~\reff{thm:seesaw}) can be applied to $p_1\colon C_T^{(g)}\times_TC_T\morphism C_T^{(g)}$ and the line bundle $\Jj_D\otimes p_1^*\Ll$ on $C_T^{(g)}\times_TC_T$, which proves that $\gamma$ is representable by some closed subprescheme $Z\subseteq C_T^{(g)}$. Since $C_T^{(g)}$ (and thus is closed subprescheme $Z$ as well) is proper over $T$ by Theorem~\reff{thm:final}\itememph{a}, we get properness of $\gamma$ for free.
	
	In general, a section of $f\colon C\morphism S$ exists only fppf-locally, so we need to descend the closed subpreschemes $Z$ we obtain fppf-locally. However, fppf descent of closed subpreschemes is innocuous (see Corollary~\reff{cor:stuff2Bdescended}\itememph{b}).
	
	Now let $[\Ll]\in U_{C/S}(T)$, where $\Ll$ is some line bundle on $C_T$. Then $H^1(C_{\ov{t}},\Ll|_{C_{\ov{t}}})=0$ for all $t\in T$, and by the same argument as in the proof of Proposition~\reff{prop:RpLl}\itememph{a} (unfortunately we can't use this proposition straightaway) we see that $f_{T,*}\Ll$ is a line bundle on $T$. We may replace $\Ll$ by $\Ll\otimes f_T^*f_{T,*}\Ll$, which doesn't change $[\Ll]$, to obtain that $f_{T,*}\Ll$ is a trivial line bundle on $T$ (this is not hard to check), hence generated by some global section $\lambda$ of $\Ll$. Now consider
	\begin{align*}
		\lambda\otimes -\colon \Ll^{\otimes -1}\morphism \Oo_{C_T}\;.
	\end{align*}
	It's image is a sheaf of ideals $\Jj\subseteq \Oo_{C_T}$, and it can be checked that $\Jj$ defines an effective relative Cartier divisor $E\in\Div_{C/S}^g(T)$ which is mapped to $[\Ll]$. This construction can be reversed, which shows that $\gamma$ is an isomorphism over $U_{C/S}$.
	
	Part~\itememph{b}. Let $T=\Spec k$ and $\Ll$ be a line bundle on $C_T$. Let $C_{\ov{T}}=C_T\times_T\Spec\ov{k}$. Since $\ov{\Ll}$ has degree $g$, Riemann--Roch shows $\dim_{\ov{k}}H^0(C_{\ov{T}},\Ll|_{C_{\ov{T}}})\geq 1$ (compare the argument on page~\pageref{par:Strategy}). Since $\ov{k}$ is flat over $k$, we get $\dim_kH^0(C_T,\Ll)\geq 1$ as well. Hence we may choose a non-vanishing global section $\lambda$ of $\Ll$ which defines a divisor $E$ mapping to $[\Ll]$ as in \itememph{a}.
\end{proof}
\begin{prop}
	\begin{alphanumerate}
		\item \lbl{prop:covering}Let $\{\sigma_i\}_{i\in I}$ be sections of $C\morphism S$ and $\Gamma\subseteq \Pic(C/S)$ the subgroup generated by the $\sigma_i$. If the translations of $U_{C/S}(T)$ by the elements of $\Gamma$ cover $\Pic_{C/S}(T)$ whenever $T=\Spec k$ is the spectrum of an algebraically closed field, then $\Pic_{C/S}$ is representable, and for every $d\in \IZ$ the representing prescheme $\PIC_{C/S}^d\morphism S$ of line bundles of degree $d$ is proper over $S$.
		\item If $\Gamma$ is any indexing set and $\Gg$ is any sheaf (of sets) on $(\cat{PSch}/S)_\fppf$ which can be covered, in the sense of Definition~\reff{def:relativeRepresentable}\itememph{c}, by representable open subsheaves $\{U_\gamma\}_{\gamma\in\Gamma}$, then $\Gg$ is representable itself.
	\end{alphanumerate}
\end{prop}
\begin{proof}[Sketch of a proof.]
	Part~\itememph{b}. Let $U_\gamma$ be represented by an $S$-prescheme $\boldsymbol{U}_\gamma$. Since $U_\vartheta\monomorphism \Gg$ is an open embedding, we obtain that $U_\gamma\cap U_\vartheta$ is representable by some open subset $\boldsymbol{U}_{\gamma,\vartheta}\subseteq \boldsymbol{U}_{\gamma}$ for all $\gamma,\vartheta\in\Gamma$. But the representing object is unique up to unique isomorphism, hence $\boldsymbol{U}_{\gamma,\vartheta}\cong \boldsymbol{U}_{\vartheta,\gamma}$ canonically. Then it's easy to see that the $\boldsymbol{U}_\gamma$ can glued together to a representing object of $\Gg$.
	
	Part~\itememph{a}. By \itememph{b} and the fact that the covering property can be tested fibre-wise, hence also geometric fibre-wise, we see that $\Pic_{C/S}^g$ is representable by some $S$-prescheme $\PIC_{C/S}^g$. But $f\colon C\morphism S$ has sections, hence we can construct  effective relative Cartier divisors of degree $1$, hence also line bundles $\Ll_1$ of fibre-wise degree $1$ by Remark~\reff{rem:degree=degree}. Such an $\Ll_1$ can be used to shift line bundles of arbitrary fibre-wise degree to line bundles of fibre-wise degree $g$. Hence $\Pic_{C/S}$ is representable by $\PIC_{C/S}=\coprod_{d\in\IZ}\PIC_{C/S}^g$.
	
	For properness, it suffices to consider the case $d=g$ as the above shows $\PIC_{C/S}^d\cong \PIC_{C/S}^g$ for all $d\in\IZ$. But still this needs to be done by hand. Note that separatedness of $\PIC_{C/S}^g\morphism S$ is automatic from the seesaw theorem (as explained in Remark~\reff{rem:PICscheme}\itememph{b}). Finally universal closedness is easy if one shows that $C^{(g)}\morphism\PIC_{C/S}^g$ is surjective on points, since then properness of $C^{(g)}\morphism S$ (by Theorem~\reff{thm:final}\itememph{a}) can be used. But surjectivity is a fibre-wise problem, hence it may be reduced to the case where $S$ is a field, which was dealt with in Proposition~\reff{prop:UCS}\itememph{b}.
\end{proof}
\begin{prop}
	Fppf-locally with respect to $S$, Proposition~\reff{prop:covering} can be applied.
\end{prop}
\begin{proof}
	Omitted.
\end{proof}


\Appendix
\chapter{Appendix}
\section{Some prerequisites about completions}
We briefly recall the most important facts about completions. An excellent introduction to this subject can be found in \cite[Section~10]{atiyahMacdonald}.
\subsection{Definitions and Hensel's lemma}
\begin{defi}
	Let $A$ be a ring (commutative with $1$), $I\subseteq A$ and ideal, and $M$ an $A$-module.
	\begin{alphanumerate}
		\item The \defemph{$I$-adic topology} on $M$ is the unique topology such that $\{I^n\}_{n\in \IN}$ is a fundamental system of neighbourhoods of $0$ and $M$ (with its additive structure) becomes a topological group in this topology.
		\item The \defemph{completion} of $M$ with respect to the $I$-adic topology is
		\begin{align*}
			\roof{M}=\limit[n\in \IN]M/I^nM\;.
		\end{align*}
		Note that $\roof{A}$ is a ring again. We call $M$ \defemph{complete} in the $I$-adic topology if the canonical morphism $M\morphism\roof{M}$ is an isomorphism.
	\end{alphanumerate}
\end{defi}
\begin{rem}
	$M$ with its $I$-adic topology is \emph{pseudo-metrizable} via $d(x,y)=\mathrm{e}^{-\sup\left\{n\st x-y\in I^n\right\}}$. It is easy to check that $\roof{M}$ is also the completion of $M$ in the analytical sense, i.e., the set of Cauchy sequences modulo the zero sequences.
\end{rem}
\begin{example}
	If $I^n=0$ for some $n\in \IN$, then any $A$-module is complete in the $I$-adic topology.
\end{example}
\begin{example}
	If $A=\IZ$ and $I=p\IZ$ for some prime $p$, then $\roof{A}=\IZ_p$ is the ring of $p$-adic integers.
\end{example}
\begin{prop}[Hensel's lemma]\lbl{prop:HenselsLemma}
	Suppose the ring $A$ is complete in the $I$-adic topology. Let $P\in A[T]$ be a polynomial and $a_0\in A$ such that $P(a_0)\equiv 0\bmod I$ and $P'(a_0)$ is a unit in $A/I$. Then there is a unique $a\in A$ such that $a\equiv a_0\bmod I$ and $P(a)=0$.
\end{prop}
\begin{proof}
	\emph{Step 1.} Consider the special case $I^2=0$. For $\delta\in I$ we have $P(a_0+\delta)=P(a_0)+\delta P'(a_0)$ since all terms of order $\delta^2$ or higher vanish in the binomial expansion. Now $P'(a_0)$ being a unit in $A/I$ gives a unique $\delta\in I$ such that $a=a_0+\delta$ satisfies $P(a)=0$.
	
	\emph{Step 2.} Suppose that $I^{2^n}=0$ for some $n\in \IN$. Using induction on $n$ (with the base case being precisely Step~1) we may assume that Hensel's lemma holds for $A/I^{2^{n-1}}$. In particular, there is a unique $a_{n-1}$ such that $P(a_{n-1})\equiv 0\bmod I^{2^{n-1}}$ and $a_{n-1}\equiv a_0\bmod I$. Moreover, $P'(a_{n-1})$ is invertible in $A/I^{2^{n-1}}$. Indeed, this follows from Hensel's lemma applied to $A/I^{2^{n-1}}$ (for which it holds by induction hypothesis) and the polynomial $Q=P'(a_{n-1})T-1$. The derivative $Q'(a_{n-1})$ equals $P'(a_{n-1})$ which is invertible in $A/I$ since $P'(a_{n-1})\equiv P'(a_0)\bmod I$, so Hensel's lemma is indeed applicable. Now replacing $I$ by $I^{2^{n-1}}$ and $a_0$ by $a_{n-1}$ reduces the situation to Step~1, proving the inductive step.
	
	\emph{Step 3.} Now let $I$ be arbitrary. By Step~2 there is for every $n\in\IN$ a unique $a_n\in A/I^{2^n}$ such that $P(a_n)\equiv 0\bmod I^{2^n}$ and $a_n\equiv a_0\bmod I$. Then $a_n\equiv a_{n-1}\bmod I^{2^{n-1}}$ is forced by uniqueness. Hence $a=(a_n)_{n\in \IN}$ defines an element of
	\begin{align*}
		\limit[n\in \IN]A/I^{2^n}=\limit[n\geq 1] A/I^n=\roof{A}\;,
	\end{align*}
	providing the desired element $a\in\roof{A}$.
\end{proof}
\begin{cor}\lbl{cor:HenselApplications}
	Let $A$ be complete in the $I$-adic topology.
	\begin{alphanumerate}
		\item If $a\in A$ becomes a unit in $A/I$, then already $a\in A^\times$.
		\item For every idempotent $e\in A/I$ there is a unique idempotent in $A$ whose image modulo $I$ is $\pi $. Therefore, $\Spec A$ and $\Spec A/I$ have the same connected components.
		\item $I$ is contained in the Jacobson radical $\rad A$.
	\end{alphanumerate}
\end{cor}
\begin{proof}
	Part \itememph{a} follows from Proposition~\ref{prop:HenselsLemma} applied to $P=aT-1$ (whose derivative $a$ is a unit in $A/I$ by assumption, so this is fine). For \itememph{b} we use the polynomial $P=T^2-T$. Again, $P'(e)=2e -1$ is a unit in $A/I$ since $(2e -1)^2=4e ^2-4e +1=1$ in $A/I$. To prove \itememph{c} recall the characterization
	\begin{align*}
		\rad A=\left\{x\in A\st 1-ax\in A^\times\text{ for all }a\in A\right\}\;.
	\end{align*}
	If $x\in I$, then $1-ax$ is a unit in $A/I$, hence also in $A$ by \itememph{a}.
\end{proof}
\begin{prop}
	Let $A$ be noetherian and $N\subseteq M$ finitely generated $A$-modules. Then the $I$-adic topology on $N$ coincides with the induced topology by the $I$-adic topology on $M$.
\end{prop}
\begin{proof}[Sketch of a proof]
	By the Artin--Rees lemma (cf.\ \cite[Proposition~3.4.1]{alg2}) there exists a number $c\in\IN$ such that $N\cap I^{n+c}M\subseteq I^nN$. From this, the assertion is easily deduced.
\end{proof}
\subsection{Flatness properties}
\begin{fact}
	\begin{alphanumerate}
		\item \lbl{fact:completion101}The canonical morphism $\roof{M}=\limit M/I^nM\morphism M/IM$ is surjective.
		\item If $M$ is finitely generated and $I$ is contained in the Jacobson radical of $A$, then $\roof{M}=0$ implies $M=0$.
	\end{alphanumerate}
\end{fact}
\begin{proof}
	For \itememph{a}, note that the composition $M\morphism\roof{M}\morphism M/IM$ equals the projection $M\morphism M/IM$ by definition of the limit. Since the latter is surjective, so is $\roof{M}\morphism M/IM$. 
	
	In particular, part \itememph{a} shows that $\roof{M}=0$ implies $M=IM$. In the situation of \itememph{b} this is equivalent to $M=0$ by Nakayama's lemma (which -- as we all know -- Professor Franke also likes to attribute to Azumaya and Krull, even though he regards Krull as a noob compared to Grothendieck).
\end{proof}
\begin{cor}\lbl{cor:completionExact}
	If $A$ is noetherian, then the functor $M\mapsto \roof{M}$ is exact on the category of finitely generated $A$-modules.
\end{cor}
\begin{proof}
	Let $0\morphism M'\morphism M\morphism M''\morphism 0$ be a short exact sequence of finitely generated $A$-modules. Then $M'+I^nM$ is the kernel of $M\epimorphism M''/I^nM''$. Using $(M'+I^nM)/I^nM\cong M'/(M'\cap I^nM)$ we get short exact sequences
	\begin{align}\lbl{eq:completion}
		0\morphism M'/(M'\cap I^nM)\morphism M/I^nM\morphism M''/I^nM''\morphism 0\tag{$*$}
	\end{align}
	for every $n\in\IN$. Since $M'/(M'\cap I^nM)$ is sandwiched between $M'/I^nM'$ and $M'/I^{n+c}M'$ for some $c\in\IN$ by the Artin--Rees lemma, it's easy to see that 
	\begin{align*}
		\limit[n\in \IN] M'/(M'\cap I^nM)=\limit[n\geq 1]M'/I^{n+c}M'=\limit[n\geq 1]M'/I^nM'=\roof{M}'\;. 
	\end{align*}
	Moreover, each $M'/(M'+I^{n+1}M)\morphism M'/(M'+I^nM)$ is clearly surjective, so Fact~\reff{fact:MittagLeffler} gives 
	\begin{align*}
		\limit[n\geq 1][^1]M'/(M'\cap I^nM)=0\;.
	\end{align*}
	Thus, taking the limit over \eqreff{eq:completion} gives a short exact sequence $0\morphism \roof{M}'\morphism \roof{M}\morphism\roof{M}''\morphism 0$ by Fact~\reff{fact:6termLimitSequence}. We are done.
\end{proof}
Recall the notion of an $A$-module $F$ being \emph{faithfully flat} (cf.\ Definition~\reff{def:faithfullyFlat}). We will eventually see that $\roof{A}$ has this property, but before that let's prove another property of faithfully flat modules (which then also explains the name \emph{faithful}).
\begin{lem}
	An $A$-module $F$ is faithfully flat (in the sense of Definition~\reff{def:faithfullyFlat}) iff the following condition holds: Whenever one of the sequences of $A$-modules
	\begin{gather*}
		M'\morphism[\alpha]M\morphism[\beta]M''\\
		M'\otimes_AF\xrightarrow{\alpha\otimes\id}M\otimes_AF\xrightarrow{\beta\otimes\id}M''\otimes_AF
	\end{gather*}
	is exact, then so is the other.
\end{lem}
\begin{proof}
	If $F$ has the property of (faithfully) preserving exact sequences, then $F$ is flat and for all $A$-modules $T$ we have $F\otimes_AT= 0$ iff $T=0$. Hence $F$ is faithfully flat in the sense of Definition~\reff{def:faithfullyFlat}.
	
	Conversely, let $F$ be faithfully flat and let $M'\otimes_AF\morphism M\otimes_AF\morphism M''\otimes_AF$ be exact (the other direction is trivial since $F$ is flat). We first claim that $\beta\alpha=0$. Indeed, if $B$ is the image of $M'$ in $M''$, then $B\otimes_AF$ is the image of $M'\otimes_AF$ in $M''\otimes_AF$ (using flatness of $B$). Since $B\otimes_AF=0$ by assumption, this implies $B=0$ since $F$ is faithfully flat.
	
	This induces a morphism $M'\morphism\ker\beta$, and we need to show that this is an epimorphism. So let $Q$ be its cokernel. Then $Q\otimes_AF$ is the cokernel of $M'\otimes_AF\morphism\ker\beta\otimes_AF\cong \ker(\beta\otimes\id_F)$ (using flatness of $F$), hence $Q\otimes_AF=0$ by assumption. Then $Q=0$ by faithful flatness again and we're done.
\end{proof}
\begin{cor}\lbl{cor:completionFaithfullyFlat}
	Let $A$ be a noetherian ring and $\roof{\phantom{x}}$ the $I$-completion for some ideal $I\subseteq A$.
	\begin{alphanumerate}
		\item When $M$ is a finitely generated $A$-module, then $\roof{M}\cong M\otimes_A\roof{A}$.
		\item $\roof{A}$ is flat as an $A$-module.
		\item If $I$ is contained in the Jacobson radical of $A$, then $\roof{A}$ is faithfully flat over $A$.
	\end{alphanumerate}
\end{cor}
\begin{proof}
	Part \itememph{a}. Every finitely generated $A$-module is finitely presented as well since $A$ is noetherian. So take a presentation $M\cong \coker(A^m\morphism A^n)$ for some $m,n\in\IN$. It's obvious that $(A^n)^\complete\cong \roof{A}^n\cong A^n\otimes_A\roof{A}$. Since both completion and tensor products commute with cokernels, this shows $\roof{M}\cong M\otimes_A\roof{A}$ as well.
	
	This immediately shows \itememph{b}. Indeed, by Corollary~\reff{cor:completionExact} and \itememph{a}, $-\otimes_A\roof{A}$ is exact on finitely generated $A$-modules -- which is sufficient for flatness by \cite[Proposition~1.2.2]{homalg}.
	
	For \itememph{c}, we need to show that if $M$ is any $A$-module, then $M\otimes_A\roof{A}=0$ implies $M=0$. Indeed, to see this, let $N\monomorphism M$ be any finitely generated submodule. Because $\roof{A}$ is flat over $A$ by \itememph{b}, $N\otimes_A\roof{A}\morphism M\otimes_A\roof{A}$ is still injective, so $0=N\otimes_A\roof{A}=\roof{N}$ by assumption and part~\itememph{a}. Hence $N=0$ by Fact~\reff{fact:completion101}. But $M$ is the union of its finitely generated submodules, so $M=0$ as well.
\end{proof}
\begin{cor}
	Suppose that $I$ is contained in the Jacobson radical of the noetherian ring $A$. If $\mu\colon M\morphism N$ is a morphism of finitely generated $A$-modules whose $I$-adic completion $\roof{\mu}\colon \roof{M}\morphism\roof{N}$ is an isomorphism, then $\mu$ is an isomorphism.
\end{cor}
\begin{proof}
	Follows from Corollary~\reff{cor:completionFaithfullyFlat}\itememph{a} and \itememph{c}.
\end{proof}
\begin{cor}\lbl{cor:completionFaithfullyFlat2}
	Let $A$ be noetherian. A sequence $0\morphism M'\morphism M\morphism M''\morphism 0$ of finitely generated $A$-modules is exact iff for all maximal ideals $\mm\subseteq A$ the sequence $0\morphism\roof{M}'\morphism \roof{M}\morphism \roof{M}''\morphism 0$ of $\mm$-adic completions is exact.
\end{cor}
\begin{proof}
	This sequence is exact iff its localizations at the maximal ideals $\mm\subseteq A$ are all exact. Having localized at $\mm$, we may check exactness after going over to $\mm A_\mm$-adic completions by Corollary~\reff{cor:completionFaithfullyFlat}\itememph{c}. So all that's left to do is to show that $\roof{M}\cong \roof{M}_\mm$ holds for the $\mm$-adic resp.\ $\mm A_\mm$-adic completions. This follows from $M/\mm^nM\cong M_\mm/\mm^nM_\mm$ for all $n\geq 1$. Indeed, $A/\mm^n$ is a local ring with maximal ideal $\mm/\mm^n$, hence multiplication by $a\in A\setminus \mm$ is already bijective on $M/\mm^nM$.
\end{proof}
\begin{cor}\lbl{cor:JM}
	If $J\subseteq A$ is any ideal and $M$ a finitely generated $A$-module, then $(JM)^\complete\morphism\roof{M}$ defines an isomorphism $(JM)^\complete\isomorphism J\roof{M}$.
\end{cor}
\begin{proof}
	We may view $(JM)^\complete$ as a submodule of $\roof{M}$ since completion preserves injectivity of the inclusion $JM\subseteq M$ by Corollary~\reff{cor:completionExact}. It's easy to see that $J\roof{M}$ is contained in $(JM)^\complete$. To prove the converse, take generators $j_1,\ldots,j_n$ of $J$. Then completion preserves surjectivity of $(j_1,\ldots,j_n)\colon M^n\epimorphism JM$ and we are done.
\end{proof}
\begin{cor}\lbl{cor:completionLocal}
	If $A$ is a noetherian local ring with maximal ideal $\mm$, then $\roof{A}$ is local with maximal ideal $\mm \roof{A}$.
\end{cor}
\begin{proof}
	We proved this in \cite[Corollary~2.2.2]{homalg}.
\end{proof}
\subsection{Completions and noetherianness}
\begin{prop}\lbl{prop:completionNoetherian}
	Let $A$ be noetherian and $I\subseteq A$ any ideal, then the $I$-adic completion $\roof{A}$ is noetherian again.
\end{prop}
To prove this, we need to prove the evil twin of Hilbert's Basissatz first.
\begin{lem}\lbl{lem:(not)HilbertBasis}
	If $A$ is noetherian, then so is the power series ring $R=A\llbracket T\rrbracket$.
\end{lem}
\begin{proof}
	We can (and will) basically copy the proof of Hilbert's Basissatz. Let $J\subseteq R$ be any ideal and put $J_n=\left\{a_n\st \sum_{k=n}^\infty a_kT^k\in J\right\}$ for $n\geq 0$. Then $(J_n)_{n\in \IN}$ form an ascending sequence of ideals in $A$. Noetherianness of $A$ tells us that this sequence becomes eventually stationary, say, at $n=s$. So we may choose $a^i=\sum_{k\geq s}a_k^iT^k\in R$ for $i=1,\ldots,N$ such that $a_s^1,\ldots,a_s^N$ generate $J_s$. Then $a^1,\ldots,a^N$ generate $J\cap T^sR$. Indeed, given any $b=\sum_{k\geq s}b_kT^k\in J$ we can inductively choose coefficients $r_k^1,\ldots,r_k^N\in A$ such that $r^i=\sum_{k\geq 0}r_k^iT^k$ satisfy $r^1a^1+\ldots+r^Na^N=b$ up to degree $T^{s+k}$. This works because $J_{k+s}=J_s$ for all $k\geq 0$ is generated by $a_s^1,\ldots,a_s^N$ again.
	
	Now $R/T^sR$ is a finitely generated $A$-module, hence the image of $J$ in it is finitely generated as well, $A$ being noetherian. We thus may choose $a^{N+1},\ldots,a^{N+M}\in J$ whose images modulo $T^sR$ generate the image of $J$ in $R/T^sR$. Then $a^{1},\ldots,a^{N+M}$ generate $J$ and our job's done here.
\end{proof}
\begin{proof}[Proof of Proposition~\reff{prop:completionNoetherian}]
	Let $r_1,\ldots,r_n$ be generators of $I$. Then sending $X_i\mapsto r_i$ defines a surjective morphism $A\llbracket X_1,\ldots,X_n\rrbracket\epimorphism \roof{A}$. Since $A\llbracket X_1,\ldots,X_n\rrbracket$ is noetherian by Lemma~\reff{lem:(not)HilbertBasis} and induction on $n$, its quotient $\roof{A}$ must be noetherian as well.
\end{proof}
\begin{cor}\lbl{cor:localIffCompletionIs}
	Suppose that $A$ is a noetherian local ring and $I\subseteq A$ any (proper) ideal. Then $\dim A=\dim\roof{A}$. In particular, $A$ is regular iff $\roof{A}$ is regular.
\end{cor}
\begin{proof}
	Let $\mm$ be the maximal ideal of $A$. Then $\roof{\mm}=\mm\roof{A}$ (this equality holds because of Corollary~\ref{cor:JM}) is the maximal ideal of the local ring $\roof{A}$ as was shown in the proof of \cite[Corollary~2.2.2]{homalg}. Since $I\subseteq \mm$, the quotients $\mm^i/\mm^{i+1}$ already have $I$-torsion, hence 
	\begin{align*}
		\mm^i/\mm^{i+1}\cong \left(\mm^i/\mm^{i+1}\right)^\complete\cong \roof{\mm}^i/\roof{\mm}^{i+1}
	\end{align*}
	(the last isomorphism follows from exactness of completion). This shows that the associated graded rings $\gr(A,\mm)$ and $\gr(\roof{A},\roof{\mm})$ agree, hence $(A,\mm)$ and $(\roof{A},\roof{\mm})$ have the same Hilbert--Samuel polynomials, which shows $\dim A=\dim\roof{A}$ by \cite[Theorem~20]{alg2}.
	
	Now $A$ and $\roof{A}$ have the same residue field $k$ and $\mm/\mm^2\cong \roof{\mm}/\roof{\mm}^2$ as seen above, so $\dim_k\mm/\mm^2=\dim_k\roof{\mm}/\roof{\mm}^2$. Clearly this implies that $A$ is regular iff $\roof{A}$ is.
\end{proof}
\begin{rem}
	In a similar fashion one can show that a noetherian local ring is Cohen--Macaulay, or Gorenstein, or a complete intersection, iff its $I$-adic completion is one as well. For example, for Cohen--Macaulayness one would need to show $\depth_A(A)=\depth_{\roof{A}}(\roof{A})$, which follows from the isomorphism $\Ext_A^p(k,A)\cong\Ext_{\roof{A}}^p(k,\roof{A})$ that was described in the proof of \cite[Proposition~2.4.2]{homalg}.
\end{rem}
\section{Generic freeness and Chevalley's theorem}
A good reference for this section is \cite[Section~14.2 and 14.3]{eisenbudCommAlg}. 

Professor Franke points out that there are multiple approaches for Generic freeness: A clever \emph{dévissage}-style proof due to Grothendieck (cf.\ \cite[Theorem~14.4]{eisenbudCommAlg}), which Franke says he could not have come up with, or a more natural proof similar to that of Hilbert's Basissatz (cf.\ \cite[\S 24]{matsumuraCRT}). However, both references assume $R$ to be noetherian, which we will circumvent by Gröbner basis theory.\footnote{True Franke fans will remember \ldots}
\begin{prop}[Grothendieck's generic freeness theorem]\lbl{prop:GenericFreeness}
	Let $R$ be a domain, $A$ an $R$-algebra of finite type, and $M$ a finitely generated $A$-module. Then there is $f\in R\setminus\{0\}$ such that $M_f$ is a free $R_f$-module
\end{prop}
\begin{proof}
	Every finitely generated $A$-module $M$ has a filtration $0=M_0\subseteq M_1\subseteq \ldots\subseteq M_n=M$ such that $M_i/M_{i-1}$ has the form $A/I_i$ for some ideal $I_i\subseteq A$. If all $M_i/M_{i-1}$ are free $R$-modules, then each sequence $0\morphism M_{i-1}\morphism M_i\morphism M_i/M_{i-1}\morphism 0$ is split, so $M$ is a free $R$-module as well. This argument shows that we only need to deal with the case $M\cong A/I$ for some ideal $I\subseteq A$. Replacing $A$ by $A/I$ this can be further reduced to $M=A$.
	
	We can present $A$ as $A\cong R[X_1,\ldots,X_n]/I$ for some ideal $I\subseteq R[X_1,\ldots,X_n]$. Let $K$ be the field of quotients of $R$ and $J\subseteq K[X_1,\ldots,X_n]$ the ideal generated by the image of $I$. Let $(\beta_1,\ldots,\beta_r)$ be a Gröbner basis of $J$ (for some fixed monomial order), then $\beta_i=\frac{b_i}{d_i}$ for some $b_i\in I$ and $d_i\neq 0$ and the $\beta_i$ have leading term $1$. Replacing $R$ by the localization $R_{d_1\cdots d_r}$ we may assume $\beta_i\in I$. Using generalized division with remainder (i.e.\ Gröbner basis stuff) one easily derives that the $\beta_i$ generate $I$. Then the monomials $X_1^{\alpha_1}\cdots X_n^{\alpha_n}$ where $\alpha=(\alpha_1,\ldots,\alpha_n)$ has the property that there is no $i$ such that $\alpha$ dominates the leading term of $\beta_i$, form a basis of $A$ as an $R$-module.
\end{proof}
\begin{prop}[Chevalley's theorem]\lbl{prop:Chevalley}
	If $f\colon \Spec B\morphism\Spec A$ is a morphism of finite type between affine noetherian schemes, then the image of $f$ is contained in a proper closed subset of $\Spec A$ or contains a dense open subset.
\end{prop}
\begin{proof}
	What Chevalley's theorem actually says is that the image of $f$ is \emph{constructible}, that is, a finite union of open subsets of closed subsets of $\Spec A$ (cf.\ \cite[Corollary~14.6]{eisenbudCommAlg} and Definition~\reff{def:constructible}). This is what we're going to prove now (and we leave it to the reader to show that this implies the assertion -- which I found quite a pain, actually).
	
	It suffices to prove the assertion for every irreducible component of $\Spec A$ (as these guys are closed and there are finitely many of them). Hence, if $\pp\in\Spec A$ is a minimal prime corresponding to the irreducible component $V(\pp)$ of $\Spec A$, then we may replace $A$ and $B$ by $A/\pp$ and $B/\pp B$. Therefore, we can assume that $A$ is a domain. Moreover, suppose that $A\subseteq B$. Otherwise, $A$ could be replaced by $A/\ker(A\morphism B)$, and since that cuts out a closed subset of $\Spec A$, we're fine with that. 
	
	By noetherian induction, we may assume that the assertion is true for all closed subsets of $\Spec A$. Let $\alpha\in A$ such that $B_\alpha$ is free as an $A_\alpha$-module (we can do this by Proposition~\reff{prop:GenericFreeness}). We claim that any $\pp\in\Spec A\setminus V(\alpha)$ is in the image of $f$. Indeed, such $\pp$ may be regarded as prime ideals $\pp\in\Spec A_\alpha$. Then $B_\pp\cong B\otimes_AA_\pp$ is still free as an $A_\pp$-module, say, $B_\pp\cong \bigoplus_{i\in I}A_\pp$ for some indexing set $I$ (which is non-empty as $A_\pp\subseteq B_\pp$). Hence $B_\pp/\pp B_\pp\cong B_\pp\otimes_{A_\pp}A_\pp/\pp A_\pp\cong \bigoplus_{i\in I}\KK(\pp)$ is not the zero ring because $\KK(\pp)\neq 0$. If $\qq\subseteq B$ is the preimage of any prime ideal of $B_\pp/\pp B_\pp$, then it's easily seen that $f(\qq)=\pp$, proving that $\pp$ is in the image of $f$, as claimed.
	
	By the Noetherian induction hypothesis, the image of $f$ in $V(\alpha)$ -- which equals the image of $\ov{f}\colon\Spec(B/\alpha B)\morphism\Spec (A/\alpha A)$ -- is constructible. We are done.		
\end{proof}
\section{Some additions to the lectures}
Occasionally it happens that Professor Franke uses some technical facts without explanation, which I wish afterwards he had proved, because they seem not at all obvious (at least to me). I usually notice only after the lecture as these facts are subtle and get lost easily in the flow of the presentation. And sometimes he just states such facts without proving them because of the lecture's limited time, which I then often take as an invitation to figure out the proof myself. In this section I collect those facts which I couldn't possibly fit into the lecture notes without bursting their overarching structure.

%\textbf{Warning 1!} I see no point in reading this section if you don't look for a specific result. There's no real coherence between the listed lemmas and propositions.
%
%\textbf{Warning 2!} Also there are some technical nightmares ahead.
\subsection{More technical facts about completions}
\begin{prop}\lbl{prop:modulesComplete}
	Let $A$ be a noetherian ring which is complete in the $I$-adic topology and let $M$ be a finitely generated $A$-module. Then $M$ is $I$-adically complete.
\end{prop}
\begin{proof}
	Note that this is clearly fulfilled if $M=A^n$ is a finitely generated free $A$-module. Now let $M$ be arbitrary. Since $A$ is noetherian, $M$ can be represented as $\coker\left(A^m\morphism A^n\right)$. Because $A^m$, $A^n$ equal their own completions (as we have just seen) and  completion is exact (by Corollary~\reff{cor:completionExact}), $M=\roof{M}$ holds as well.
\end{proof}
\begin{cor}\lbl{cor:finiteAlgebras}
	Let $A$ be a noetherian local ring with maximal ideal $\mm$. Let $B$ be a finite $A$-algebra.
	\begin{alphanumerate}
		\item Then $B$ has only finitely many prime ideals over $\mm$, and all of them are maximal.
		\item If $A$ is, in addition, $\mm$-adically complete, then all maximal ideals of $B$ lie over $\mm$. In particular, $B$ is \defemph{semi-local} (i.e.\ has finitely many maximal ideals). The same is true if $A\subseteq B$.
	\end{alphanumerate}
\end{cor}
\begin{proof}
	Part \itememph{a}. If $\qq\in\Spec B$ is a prime ideal over $\mm$, then $B/\qq$ is a finitely generated domain over the residue field $k=A/\mm$, hence a finite field extension of $k$, so $\qq$ is maximal. Moreover, $B/\mm B$ is a finite-dimensional $k$-algebra, hence it has only finitely many maximal ideals by the argument from Fact~\reff{fact:annoyingQF}\itememph{b}.
	
	Part \itememph{b}. If $A$ is $\mm$-adically complete, then $B$ is $\mm B$-adically complete by Proposition~\reff{prop:modulesComplete}, so $\mm B$ is contained in the Jacobson radical $\rad B$ by Corollary~\reff{cor:HenselApplications}\itememph{c}. Then all maximal ideals of $B$ lie over $\mm$.
	
	Now assume $A\subseteq B$. Let $\qq\in\Spec B$ be a maximal ideal and $\pp=\qq\cap A$. Then $A/\pp\subseteq B/\qq$ is an integral ring extension in which $B/\qq$ is a field, hence so is $A/\pp$, (by \cite[Proposition~1.5.1\itememph{d}]{alg1}) which proves $\pp=\mm$.
\end{proof}
\begin{prop}\lbl{prop:technicalAF}
	Let $A$ be a noetherian ring and $B$ an $A$-algebra. Let $\qq\in\Spec B$ be a prime ideal and $\pp\in\Spec A$ its preimage in $A$. Let $\roof{A}_\pp$ denote the $\pp A_\pp$-adic completion of $A_\pp$.
	\begin{alphanumerate}
		\item The ideal $\qq B_\qq\otimes_{A_\pp}\roof{A}_\pp$ is maximal in $B_\qq\otimes_{A_\pp}\roof{A}_\pp$.
		\item Moreover, assume that $B$ is of finite type over $A$ (so that $B$ is noetherian as well) and that $B\otimes_A\KK(\pp)$ is a finite-dimensional $\KK(\pp)$-vector space. Then $B_\qq\otimes_{A_\pp}\roof{A}_\pp$ equals the $\qq B_\qq$-adic completion $\roof{B}_\qq$ of $B_\qq$. In particular, $B_\qq\otimes_{A_\pp}\roof{A}_\pp$ is a local ring again.
	\end{alphanumerate}
\end{prop}
\begin{rem}
	The conditions from Proposition~\reff{prop:technicalAF}\itememph{b} say precisely that the induced morphism $\Spec B\morphism\Spec A$ of schemes is \emph{quasi-finite} at all primes above $\pp$, cf.\ Definition~\reff{def:quasiFinite} and Fact~\reff{fact:annoyingQF}\itememph{b}.
\end{rem}
\begin{proof}[Proof of Proposition~\reff{prop:technicalAF}]
	Part \itememph{a}. This follows basically from the fact that $\roof{A}_\pp$ is local again with maximal ideal $\roof{\pp}=\pp\roof{A}_\pp$ and has the same residue field $\KK(\pp)$ as $A_\pp$.The details go as follows: We have
	\begin{align*}
	\left(B_\qq\otimes_{A_\pp}\roof{A}_\pp\right)/\left(\qq B_\qq\otimes_{A_\pp}\roof{A}_\pp\right)\cong \KK(\qq)\otimes_{A_\pp}\roof{A}_\pp\;.
	\end{align*}
	Denote $\phi\colon A\morphism B$ the ring morphism that makes $B$ an $A$-algebra. If $x\in \KK(\qq)\otimes_{A_\pp}\roof{\pp}$, then $x$ can be written as $x=\sum_{i=1}^{n}b_i\otimes p_ia_i$ where $b_i\in B_\qq$, $a_i\in A_\pp$, and $p_i\in \pp$. But then $x=\sum_{i=1}^{n}\phi(p_i)b_i\otimes a_i=0$, because the $\phi(p_i)$ are elements of $\qq$ as $\pp=\phi^{-1}(\qq)$ by assumption. Hence
	\begin{align*}
	\KK(\qq)\otimes_{A_\pp}\roof{A}_\pp\cong \left(\KK(\qq)\otimes_{A_\pp}\roof{A}_\pp\right)/\left(\KK(\qq)\otimes_{A_\pp}\roof{\pp}\right)\cong \KK(\qq)\otimes_{A_\pp}\KK(\pp)\cong \KK(\qq)
	\end{align*}
	is a field, which shows that $\qq B_\qq\otimes_{A_\pp}\roof{A}_\pp$ is a maximal ideal.
	
	Part \itememph{b}. Brace yourself, because the proof to come is \emph{horrible}. I'm trying my best to self-contained and cite The Stacks Project as rarely as possible, but I make no promises. We will separate the agonizing process into tiny lemmas.
	\begin{lem}\lbl{lem:ANNOYING1}
		In the situation of Proposition~\reff{prop:technicalAF}\itememph{b}, $B_\qq\otimes_{A_\pp}\KK(\pp)$ is a finite-dimensional $\KK(\pp)$-vector space as well.
	\end{lem}
	\begin{proof}
		We know that $\ov{B}=B\otimes_A\KK(\pp)$ is a finite-dimensional $\KK(\pp)$-algebra. Let $B_\pp=B\otimes_AA_\pp$ be the localization of $B$ at the image of the multiplicative set $A\setminus \pp$. Put $\ov{\qq}=\qq B_\pp/\pp B_\pp$, so that $\ov{B}_{\ov{\qq}}\cong B_\qq\otimes_{A_\pp}\KK(\pp)$. We know from Fact~\reff{fact:annoyingQF}\itememph{b} that $\Spec\ov{B}\morphism\Spec \KK(\pp)$ is quasi-finite, hence $\left\{\ov{\qq}\right\}$ is open and closed in $\Spec\ov{B}$. By \cite[\stackstag{00EE}]{stacks-project} we get an idempotent $e\in \ov{B}$ such that $\left\{\ov{\qq}\right\}\cong \Spec \ov{B}_e$. Then $\ov{\qq}$ is the only prime ideal ov $\ov{B}_e$, hence $\ov{B}_e\cong \ov{B}_{\ov{\qq}}$. Also $\ov{B}_e\cong \ov{B}[e^{-1}]$ is a $\KK(\pp)$-algebra of finite type again and $\Spec \ov{B}_e\morphism \Spec \KK(\pp)$ is clearly quasi-finite in the sense of Definition~\reff{def:quasiFinite}. Using Fact~\reff{fact:annoyingQF}\itememph{b} again, we see that $\ov{B}_e$ is finite over $\KK(\pp)$, which is precisely what we wanted to show.
	\end{proof}
	\begin{lem}\lbl{lem:ANNOYING2}
		The $\pp B_\qq$-adic completion of $B_\qq$ equals its $\qq B_\qq$-adic completion $\roof{B}_\qq$. Moreover, $\roof{B}_\qq$ is finitely generated as an $\roof{A}_\pp$-module, and such generators can be chosen from $B_\qq$.
	\end{lem}
	\begin{proof}
		As we have just seen in Lemma~\reff{lem:ANNOYING1}, $B_\qq\otimes_{A_\pp}\KK(\pp)\cong B_\qq/\pp B_\qq$ is a noetherian local ring with only one prime ideal $\ov{\qq}$. Then $\ov{\qq}$ must be the nilradical of $B_\qq/\pp B_\qq$, hence $\ov{\qq}^N=0$ for some $N\in\IN$ (because we are noetherian). This shows $(\qq B_\qq)^N\subseteq \pp B_\qq$, hence the $\pp B_\qq$-adic and the $\qq B_\qq$-adic completions of $B_\qq$ coincide.
		
		Since $B_\qq/\pp B_\qq\cong \roof{B}_\qq/\pp\roof{B}_\qq$ is finite-dimensional as a vector space over $\KK(\pp)\cong \roof{A}_\pp/\pp\roof{A}_\pp$, the second assertion follows from the more general Lemma~\reff{lem:ANNOYING3}.
	\end{proof}
	\begin{lem}\lbl{lem:ANNOYING3}
		Let $A$ be a noetherian ring which is $I$-adically complete and let $M$ be an $I$-adically complete $A$-module. If $M/IM$ is finitely generated over $A/I$, then $M$ is finitely generated over $A$. In fact, lifting a finite set of generators of $M/IM$ over $A/I$ to $M$ gives a (finite) generating set of $M$
	\end{lem}
	\begin{rem}
		The Stacks Project has a more general version of this, cf.\ \cite[\stackstag{031D}]{stacks-project}.
	\end{rem}
	\begin{proof}[Proof of Lemma~\reff{lem:ANNOYING3}]
		Let $x_1,\ldots,x_s$ be lifts of a finite generating set of $M/IM$ and let $N\subseteq M$ be the submodule generated by the $x_i$. Then $N$ is $I$-adically complete by Proposition~\reff{prop:modulesComplete}.
		
		We claim that also $N/I^nM\morphism M/I^nM$ is surjective for all $n\geq 1$. Indeed, for $n=1$ this is trivial. Using this, it's easy to see that
		\begin{align*}
		\coker\Big(N/I^nN\morphism M/I^nM\Big)=(I/I^n)\cdot\coker\Big(N/I^nN\morphism M/I^nM\Big)\;.
		\end{align*}
		But $(I/I^n)$ is a nilpotent ideal in $A/I^n$, hence $\coker\big(N/I^nN\morphism M/I^nM\big)=0$ (this feels like a dummy version of Nakayama's lemma). We thus get short exact sequences
		\begin{align*}
		0\morphism\left(I^nM\cap N\right)/I^nN\morphism N/I^nN\morphism M/I^nM\morphism 0\;.
		\end{align*}
		Let $x\in I^nM\cap N$. Then $x$ can be written as $x=\sum_i\alpha_i m_i$, where $\alpha_i\in I^n$ and $m_i\in M$. Because $N/IN\morphism M/IM$ is surjective, we may write $m_i=n_i+\mu_i$ where $n_i\in N$ and $\mu_i\in IM$. Then $y=\sum_i\alpha_i\mu_i$ is an element in $I^{n+1}M\cap N$ and has the same image in $(I^nM\cap N)/N$ as $x$. This shows that $\left((I^nM\cap N)/I^nN\right)_{n\geq 1}$ has the Mittag-Leffler property from Fact~\reff{fact:MittagLeffler}. This proves $\limit[][^1](I^nM\cap N)/I^nN=0$, hence 
		\begin{align*}
		N\cong \limit[n\geq 1]N/I^nN\morphism \limit[n\geq 1]M/I^nM\cong M
		\end{align*}
		is surjective by the six-term exact sequence from Fact~\reff{fact:6termLimitSequence}.
	\end{proof}
	Finally, the proof of Proposition~\reff{prop:technicalAF}\itememph{b} can be finished. By \cite[p.~18]{homalg}, $B_\qq\otimes_{A_\pp}\roof{A}_\pp$ is the union (or rather the colimit) over all $M\otimes_{A_\pp}\roof{A}_\pp$ where $M$ ranges through the finitely generated $A_\pp$-submodules of $B_\qq$. But for such $M$ we have $M\otimes_{A_\pp}\roof{A}_\pp\cong \roof{M}\subseteq \roof{B}_\qq$ by Corollary~\reff{cor:completionFaithfullyFlat}\itememph{a}. Moreover, by Lemma~\reff{lem:ANNOYING2} we can find such an $M$ that satisfies $\roof{M}=\roof{B}_\qq$. Hence $B_\qq\otimes_{A_\pp}\roof{A}_\pp\cong \roof{B}_\qq$, and we're finally done.
\end{proof}
\subsection{A quick reminder about curves and divisors}
\begin{lem}\lbl{lem:Hpvanishing}
	Let $k$ be an algebraically closed field, $C\morphism\Spec k$ a proper regular connected curve of genus $g$ over $k$. Let $\Ll$ be a line bundle on $C$.
	\begin{alphanumerate}
		\item If $\deg \Ll<0$, then $H^0(C,\Ll)=0$.
		\item If $\deg \Ll>2g-2$, then $H^1(C,\Ll)=0$.
	\end{alphanumerate}
\end{lem}
\begin{proof}
	Part \itememph{a}. Let $D$ be a divisor such that $\Ll\cong \Oo_C(D)$ (recall that we can do this by \cite[Proposition~3.0.1\itememph{b}]{alggeo2} -- and writing things like $\deg \Ll$ already assumed this implicitly). We need to show that $\Oo_C(D)$ has vanishing global section. Indeed, if $K$ denotes the field of fractions of $C$ (that is, the stalk at the generic point), then
	\begin{align*}
		\Gamma(C,\Oo_C(D))=\left\{f\in K\st \div(f)+D\geq 0\right\}
	\end{align*}
	by definition. But $\sum_{c\in C_1}v_c(f)=0$ for all $f\in K^\times$ (this follows from the fact that $\deg$ is well-defined -- and \emph{no}, that's no circular reasoning if we use the rather odd proof from \cite[p.\:79]{alggeo2}), hence 
	\begin{align*}
		\sum_{c\in C_1}\big(\div(f)_c+D(c)\big)=\deg D<0
	\end{align*}
	for all $f\in K^\times$, which proves that $\Gamma(C,\Oo_C(D))=0$.
	
	Part \itememph{b} is an easy consequence of \itememph{a} and Serre duality. By \cite[Corollary~3.1.2]{alggeo2} we have $\deg\Omega_C=2g-2$. Hence $\Omega_C\otimes_{\Oo_C}\Ll^*$ (where $\Ll^*$ denotes the dual of $\Ll$) has negative degree (which uses \cite[Proposition~3.0.2]{alggeo2}). Also note that $\Omega_C\otimes_{\Oo_C}(\Omega_C\otimes_{\Oo_C}\Ll^*)^*\cong \Ll$. By Serre duality as in \cite[Theorem~7\itememph{c}]{alggeo2} this implies
	\begin{align*}
		H^0\left(C,\Omega_C\otimes_{\Oo_C}\Ll^*\right)\cong \Hom_k\left(H^1(C,\Ll),k\right)\;.
	\end{align*}
	But $H^0\left(C,\Omega_C\otimes_{\Oo_C}\Ll^*\right)=0$ by \itememph{a}, so $H^1(C,\Ll)$ vanishes too.
\end{proof}
\subsection{How to geometric fibres?}
\begin{lem}\lbl{lem:stalkOfGeometricFibres}
	Let $f\colon X\morphism Y$ be a morphism of preschemes and $x\in X$, $y=f(x)$ such that $f^*\colon \KK(y)\morphism\KK(x)$ is an isomorphism. Let $k=\KK(y)$. Then $x$ has precisely one preimage $\ov{x}$ in the geometric fibre
	\begin{align*}
		X_{\ov{y}}=X\times_Y\Spec\ov{k}
	\end{align*}
	and $\Oo_{X_{\ov{y}},\ov{x}}\cong \Oo_{X,x}\otimes_{\Oo_{Y,y}}\ov{k}$. 
\end{lem}
\begin{proof}
	Locally, the question becomes whether $B\otimes_A\ov{k}$ has precisely one prime ideal over $\qq\in\Spec B$, which satisfies $\KK(\pp)\cong \KK(\qq)\cong k$ where $\pp$ is the preimage of $\qq$ in $A$. Let $S=B\setminus \qq$. Certainly, every prime ideal over $\qq$ contains $\qq\otimes_A\ov{k}$, hence these prime ideals survive localizing the multiplicative subset $S\otimes 1$. The localization of $B\otimes_A\ov{k}$ with respect to $S\otimes 1$ equals $B_\qq\otimes_{A_\pp}\ov{k}$ (as can be easily seen by pushing universal properties of tensor products and localizations around). We claim that $\qq B_\qq\otimes_A\ov{k}$ is already a maximal ideal. Indeed, since $B_\qq/\qq B_\qq\cong k$ we have
	\begin{align*}
		\big(B_\qq\otimes_A\ov{k}\big)/\big(\qq B_\qq\otimes_A\ov{k}\big)\cong k\otimes_A\ov{k}\;.
	\end{align*}
	But also $k\cong A_\pp/\pp A_\pp$, so $k\otimes_A\ov{k}\cong \ov{k}$ is a field.
	
	If we show that $\qq B_\qq\otimes_A\ov{k}$ is actually \emph{the only} maximal ideal of $B_\qq\otimes_A\ov{k}$, then both assertions will follow at once (up to noticing that $B\otimes_A\ov{k}$ and $B\otimes_{A_\pp}\ov{k}$ are actually the same). To do this, note that $B_\qq\otimes_A\ov{k}$ is integral over $B\otimes_Ak$. Indeed, if $b=\sum_ib_i\otimes x_i$ (with $b_i\in B$, $x_i\in \ov{k}$) is any element of $B\otimes_A\ov{k}$, then the $x_i$ are contained in some finite extension $\ell/k$, hence $b$ is contained in $B\otimes_A\ell$, which is a finite free module over $B\otimes_Ak$ (because $\ell$ is finite free over $k$). Since $B_\qq\otimes_Ak\cong B_\qq/\pp B_\qq$ is local with maximal ideal $\qq B_\qq\otimes_Ak$, we see that $\qq B_\qq\otimes_A\ov{k}$ is the only maximal ideal over $\qq B_\qq\otimes_Ak$. Then the going-up theorem, or more precisely \cite[Theorem~7\itememph{d}]{alg1}, shows that $\qq B_\qq\otimes_A\ov{k}$ is the only maximal ideal of $B_\qq\otimes_A\ov{k}$ at all, which means we're done.
\end{proof}
\begin{prop}\lbl{prop:GeometricFibres101}
	Let $f\colon X\morphism Y$ be a morphism of preschemes, $y\in Y$ a point and $X_y$, $X_{\ov{y}}$ its ordinary and geometric fibre respectively.
	\begin{alphanumerate}
		\item Let $x$ be a point in the fibre $X_y$ over $y$ and $\ov{x}$ a point in $X_{\ov{y}}$ that lies over $x$. Then $\dim\Oo_{X_y,x}=\dim\Oo_{X_{\ov{y}},\ov{x}}$. Moreover, the residue field $\KK(\ov{x})$ is algebraic over $\KK(x)$.
		\item If $X_y$ and $X_{\ov{y}}$ are locally noetherian (this is e.g.\ the case if $f$ is of locally finite type) and the local rings $\Oo_{X_{\ov{y}},\ov{x}}$ are regular for all $\ov{x}$ above $x$, then $\Oo_{X_y,x}$ is regular itself.
	\end{alphanumerate}
\end{prop}
\begin{proof}
	Let $k=\KK(y)$. Both assertions are local on $X$ and $Y$, so we may work with $B\otimes_Ak$ and $B\otimes_A\ov{k}$ for some ring $A$ and some $A$-algebra $B$. Let $\qq\in\Spec B$ and $\pp\in\Spec A$ be the prime ideals corresponding to $x$ and $y$. Then $\Oo_{X_y,x}\cong B_\qq\otimes_Ak$. As in the proof of Lemma~\reff{lem:stalkOfGeometricFibres} we see that $B\otimes_A\ov{k}$ is integral and free over $B\otimes_Ak$. Hence the same is true after localizing the multiplicative subset $S\otimes 1$ (where $S=B\setminus \qq$), so $B_\qq\otimes_A\ov{k}\cong \Oo_{X_y,x}\otimes_k\ov{k}$ is integral over $B\otimes_Ak\cong \Oo_{X_y,x}$ and contains all relevant prime ideals (again, as in the proof of Lemma~\reff{lem:stalkOfGeometricFibres}).
	
	Since $\Oo_{X_y,x}\otimes_k\ov{k}$ is integral over $\Oo_{X_y,x}$ (as in the proof of Lemma~\reff{lem:stalkOfGeometricFibres}), we may apply the going up theorem (in the form of \cite[Theorem~7]{alg1}) to see
	\begin{align*}
		\dim\Oo_{X_y,x}=\dim\Oo_{X_y,x}\otimes_k\ov{k}\;,
	\end{align*}
	and moreover the prime ideals of $\Oo_{X_y,x}\otimes_k\ov{k}$ which lie over the maximal ideal $\mm_{X_y,x}$ of $\Oo_{X_y,x}$ (i.e., the points $\ov{x}$ over $x$) are precisely the maximal ideals of $\Oo_{X_y,x}\otimes_k\ov{k}$. Hence $\KK(\ov{x})$ is also the quotient of $\KK(x)\otimes_k\ov{k}$ by some maximal ideal. But $\KK(x)\otimes_k\ov{k}$ is integral over $\KK(x)$, thus $\KK(\ov{x})/\KK(x)$ is algebraic. To see $\dim\Oo_{X_y,x}=\dim\Oo_{X_{\ov{y}},\ov{x}}$, note that $\Oo_{X_y,x}\otimes_k\ov{k}$ over also satisfies the going-down property over $\Oo_{X_y,x}$ since $\Oo_{X_y,x}\otimes_k\ov{k}$ is a free $\Oo_{X_y,x}$-module, hence flat, so Proposition~\reff{prop:flatGoingDown} can be applied. Therefore all maximal ideals have height $\dim\Oo_{X_y,x}$, proving $\dim\Oo_{X_y,x}=\dim\Oo_{X_{\ov{y}},\ov{x}}$. This shows \itememph{a}.
	
	To deduce \itememph{b} we use Serre's regularity criterion (cf.\ \cite[Theorem~1]{homalg}). Let $M$ and $N$ be $\Oo_{X_y,x}$-modules and let $P_\bullet\epimorphism M$ be a projective resolution. Then $P_\bullet\otimes_k\ov{k}\epimorphism M\otimes_k\ov{k}$ is a projective resolution over $\Oo_{X_y,x}\otimes_k\ov{k}$. Indeed, $P_i\otimes_k\ov{k}$ is (as an $\Oo_{X_y,x}$-module or an abelian group, this doesn't matter) a direct sum of copies of $P_i$, so exactness is preserved. Also, if $P_i\oplus Q_i$ is free over $\Oo_{X_y,x}$, then $(P_i\otimes_k\ov{k})\oplus (Q_i\otimes_k\ov{k})$ is free over $\Oo_{X_y,x}\otimes_k\ov{k}$, so $P_i\otimes_k\ov{k}$ is still projective. Hence the homology of 
	\begin{align*}
		\left(P_\bullet\otimes_k\ov{k}\right)\otimes_{\Oo_{X_y,x}\otimes_k\ov{k}}\left(M\otimes_k\ov{k}\right)\cong \left(P_\bullet\otimes_{\Oo_{X_y,x}}M\right)\otimes_k\ov{k}
	\end{align*}
	computes $\Tor_\bullet^{\Oo_{X_y,x}\otimes_k\ov{k}}(M\otimes_k\ov{k},N\otimes_k\ov{k})$.
	
	 Note that $d=\dim\Oo_{X_y,x}$ is finite (because $\Oo_{X_y,x}$ is a noetherian local ring), hence $\dim\Oo_{X_{\ov{y}},\ov{x}}=d$ by for all $\ov{x}$ above $x$ by part \itememph{a}. But then $\gdim\Oo_{X_{\ov{y}},\ov{x}}= d$ for all $\ov{x}$ over $x$ by Serre's regularity criterion. Since the $\Oo_{X_{\ov{y}},\ov{x}}$ are precisely the localizations of $\Oo_{X_y,x}\otimes_k\ov{k}$ at its prime ideals, this shows $\gdim\big(\Oo_{X_y,x}\otimes_k\ov{k}\big)\leq d$. But then
	 \begin{align*}
	 	H_p\left(P_\bullet\otimes_{\Oo_{X_y,x}}M\otimes_k\ov{k}\right)=0\quad\text{for all }p\geq d\;.
	 \end{align*}
	 Since $\ov{k}$ is free over $k$, this already shows $H_p\big(P_\bullet\otimes_{\Oo_{X_y,x}}M\big)=0$ for $p\geq d$. In other words, 
	 \begin{align*}
	 	\Tor_p^{\Oo_{X_y,x}}(M,N)=0\quad\text{for }p\geq d\;,
	 \end{align*}
	 hence $\gdim\Oo_{X_y,x}\leq d$. Then $\Oo_{X_y,x}$ is regular by Serre's regularity criterion.
\end{proof}
\begin{rem}
	\begin{alphanumerate}
		\item Some people like to use Kähler differentials together with \cite[Proposition~1.6.3]{alg2} to prove Proposition~\reff{prop:GeometricFibres101}\itememph{b}. However, it seems quite delicate to bypass the separability requirements there, so the proof by Serre's regularity criterion seemed more clean and straightforward to me.
		\item Both Lemma~\reff{lem:stalkOfGeometricFibres} and Proposition~\reff{prop:GeometricFibres101} still hold when $\ov{k}$ is replaced by an arbitrary algebraic extension $\ell/k$ -- in fact, that's all we used in the proofs.
	\end{alphanumerate}
\end{rem}
\begin{cor}\lbl{cor:FibresAreCurvesToo}
	If $f\colon C\morphism X$ is a morphism of preschemes whose geometric fibres are regular connected curves, then so are its ordinary fibres.
\end{cor}
\begin{proof}
	By Proposition~\reff{prop:GeometricFibres101} it's clear that the fibres $C_x$ for $x\in X$ are one-dimensional and regular. To show connectedness, note that $C_{\ov{x}}\morphism C_x$ is surjective as a map of topological spaces. Indeed, locally the question becomes whether $\Spec(B\otimes_A\ov{k})\morphism\Spec (B\otimes_Ak)$ is surjective. But we've seen several times now that $B\otimes_A\ov{k}$ is integral over $B\otimes_Ak$, so surjectivity follows from \cite[Theorem~7\itememph{a}]{alg1}. Therefore, $C_x$ is the surjective image of the connected space $C_{\ov{x}}$, hence connected itself.
\end{proof}
\subsection{A remark about constructible subsets}
We already talked about \emph{constructible subsets} in the proof of Chevalley's theorem (Proposition~\reff{prop:Chevalley}). Here we'll prove a technical property of these guys, which is needed in the proof of Proposition~\reff{prop:flatGoingDown}.
\begin{defi}\lbl{def:constructible}
	Let $X$ be a noetherian topological space. A subset $E\subseteq X$ is called \defemph{constructible} if $E$ is a finite union of locally closed subsets of $X$.
\end{defi}
	In case you wonder where the noetherianness hypothesis comes from: In general, this is \emph{not} the right definition of ``constructible subsets''. A definition for arbitrary spaces (and some useful results in the non-noetherian case) can be found in \cite[\stackstag{04ZC}]{stacks-project}.
\begin{lem}\lbl{lem:constructible1}
	Finite unions and finite intersections of constructible sets are constructible again. If $E\subseteq X$ is constructible, then so is $X\setminus E$.
\end{lem}
\begin{proof}
	The first two assertions are quite trivial. Now if $E$ is constructible, write $E=\bigcup_{i=1}^nC_i$ where $C_i=Z_i\subseteq X$ is locally closed with $Z_i$ being closed and $U_i$ being open. Then the complement $X\setminus C_i=(X\setminus Z_i)\cup (X\setminus U_i)$ is constructible (being a union of a closed and an open subset). Hence $X\setminus E=\bigcap_{i=1}^n(X\setminus C_i)$ is constructible as well.
\end{proof}
\begin{defi}\lbl{def:specialization}
	Let $X$ be any topological space.
	\begin{alphanumerate}
		\item If $x,y\in X$ satisfy $x\in\ov{\{y\}}$, we call $x$ a \defemph{specialization} of $y$ and $y$ a \defemph{generalization} of $x$. A common notation is $y\leadsto x$.
		\item A subset $T\subseteq X$ is called \defemph{stable under specialization} if for all $y\in T$ and all specializations $y\leadsto x$ we have $x\in T$ as well. In the same way $T$ being \defemph{stable under generalization} is defined.
	\end{alphanumerate}
\end{defi}
Using the mysterious generalization of constructibility, one can prove the following proposition for all \emph{spectral spaces} (like $\Spec R$ for any ring $R$) without noetherianness assumptions. See \cite[\stackstag{00I0}]{stacks-project} for proofs.
\begin{prop}\lbl{prop:constructible2}
	Let $X$ be a noetherian sober space and $E\subseteq X$ a constructible subset.
	\begin{alphanumerate}
		\item If $E$ is stable under specialization, then $E$ is closed.
		\item If $E$ is stable under generalization, then $E$ is open.
	\end{alphanumerate}
\end{prop}
\begin{proof}
	It suffices to show \itememph{a}, for $E$ is stable under specialization iff $X\setminus E$ is stable under generalization and we've seen that $X\setminus E$ is constructible as well in Lemma~\reff{lem:constructible1}. So suppose that $E$ is stable under specialization. It suffices to find finitely many $x_1,\ldots,x_n\in E$ such that $E\subseteq \bigcup_{i=1}^n\ov{\{x_i\}}$ since then $E=\bigcup_{i=1}^n\ov{\{x_i\}}$ because $E$ is closed under specialization.
	
	To prove that such $x_i$ exist, we may assume that $E$ is locally closed, i.e., $E=Z\cap U$ where $Z$ is closed and $U$ is open. Since $X$ is noetherian, we can decompose $Z$ into finitely many irreducible components, and since $X$ is sober, each irreducible component of $Z$ has a generic point. Thus $Z=\bigcup_{i=1}^n\ov{\{\eta_i\}}$ for some $\eta_i\in Z$. Moreover, $U$ intersects $\ov{\{\eta_i\}}$ iff $\eta_i\in U$. Hence we can choose $\eta_{i_1},\ldots,\eta_{i_m}\in E$ such that $E=Z\cap U\subseteq \bigcup_{j=1}^m\ov{\{\eta_{i_j}\}}$. We're done.
\end{proof}
\subsection{Descent of vector bundles gives vector bundles again}
Not much to say here -- we're going to prove what is said in the title, since we need it for Example~\reff{ex:sheafifyingPic}. We start with a technical lemma.
\begin{lem}\lbl{lem:descentOfFinGen}
	Let $f\colon X\morphism \Spec A$ be an fpqc morphism and $M$ an $A$-module such that $f^*\snake{M}$ is locally finitely generated. Then $M$ is finitely generated.
\end{lem}
\begin{proof}
	Since $f$ is quasi-compact, so is $X$, hence we may cover it by finitely many affine opens $\Spec B_i$ such that $B_i\otimes_AM$ is finitely generated over $B_i$. For all $i$ let $\mu_{i,j}=\sum_k b_{i,j,k}\otimes m_{i,j,k}$ be finitely many generators of $B_i\otimes_AM$. Let $I=\left\{m_{i,j,k}\right\}$ be the set of all the $m_{i,j,k}$. We claim that $I$ is a generating set of $M$, which would suffice since $I$ is finite.
	
	To show the claim, consider $A^I\morphism M$, sending the standard basis vectors to the elements of $I$. To show this map is surjective, it suffices to prove surjectivity holds after localizing at an arbitrary prime ideal $\pp\in\Spec A$. Let $x\in X$ be a point over $\pp$ and let $i$ be chosen such that $x\in B_i$. Then $\Oo_{X,x}^I\morphism \Oo_{X,x}\otimes_{A_\pp}M_\pp$ hits all the generators $\mu_{i,j}$ of $B_i\otimes_AM$, hence also all generators of $\Oo_{X,x}\otimes_{A_\pp}M_\pp$ and is therefore surjective. But $\Oo_{X,x}$ is faithfully flat over $A_\pp$, hence $A_\pp^I\morphism M_\pp$ must be surjective as well.
\end{proof}
\begin{lem}\lbl{lem:descentIsALineBundleAgain}
	Let $f\colon X\morphism Y$ be an fpqc morphism and let $(\Gg,\nu)$ be a descent datum, where $\Gg$ is a vector bundle on $X$. Then the descent of $\Gg$ is a vector bundle of the same rank again.
\end{lem}
\begin{proof}
	Let $\Vv$ be a quasi-coherent $\Oo_Y$-module such that $f^*\Vv$ is a line bundle. Since proving that $\Vv$ is a vector bundle is a local question on $Y$, we may assume $Y=\Spec A$ and $\Vv=\snake{V}$. It suffices that $V$ is finitely presented and for every $y\in Y$ the stalk $\Vv_y$ is a free $\Oo_{Y,y}$-module of the same rank as $(f^*\Vv)_x$ for all $x$ over $y$, since every finitely presented module whose localizations at all prime ideals are free must be projective (and then the rank also fits).
	
	To show that $V$ is finitely presented, apply Lemma~\reff{lem:descentOfFinGen} to $V$ to obtain a surjection $A^n\epimorphism V$. We need to show that its kernel $K$ is finitely generated as well. This will be another application of Lemma~\reff{lem:descentOfFinGen}. If $\Spec B$ is an affine open in $X$, then $0\morphism B\otimes_AK\morphism B^n\morphism B\otimes_AV\morphism 0$ is exact again because $X$ is flat over $A$. But $B\otimes_AV$ is projective by assumption on $f^*\Vv$, hence this sequence splits, which shows that $B\otimes_AK$ must be finitely generated as well. Then so is $K$ by Lemma~\reff{lem:descentOfFinGen}.
	
	To show $\Vv_y\cong \Oo_{Y,y}$, let $x\in X$ be a point over $y$, so $\Oo_{X,x}$ is faithfully flat over $\Oo_{Y,y}$. Then $\Oo_{X,x}\otimes_{\Oo_{Y,y}}\Vv_y$ is isomorphic to $\Oo_{X,x}^n$ for some $n$, hence we can choose a basis $\lambda_1,\ldots,\lambda_n$. Each $\lambda_i$ has a representation $\lambda_i=\sum_jb_{i,j}\otimes\ell_{i,j}$, where $b_{i,j}\in\Oo_{X,x}$ and $\ell_{i,j}\in \Vv_y$. Then the $1\otimes\ell_{i,j}$ form a generating set of the vector space $\KK(x)\otimes_{\Oo_{Y,y}}\Vv_y$, so we can choose a subset $I\subseteq\{\ell_{i,j}\}$ such that $\#I=n$ and $I$ is a basis of that vector space. By Nakayama's lemma, $I$ is also a generating set of $\Oo_{X,x}\otimes_{\Oo_{Y,y}}\Vv_y$, hence a basis since this free module has rank $n$.
	
	 Now consider $\Oo_{Y,y}^I\morphism V$ given by sending the standard basis vectors to the corresponding elements of $I$. This morphism becomes an isomorphism after tensoring with $\Oo_{X,x}$, hence it must already be an isomomorphism itself by faithful flatness.
\end{proof}

\subsection{How to divisors?}\lbl{subsec:How2Divisors}
We have seen in \cite[Ch.~3]{alggeo2} how to define divisors on curves. Here we give a very brief introduction on how to define them in general (but if you really want to dive into divisors, \cite[\stackstag{01WO}]{stacks-project} or \cite[Section~II.6]{hartshorne} are way better sources).
\begin{defi}\lbl{def:divisors}
	Let $X$ be an  integral noetherian prescheme.
	\begin{alphanumerate}
		\item A \defemph{divisor} on $X$ is a formal linear combination $D=\sum_in_iZ_i$, where the $Z_i$ are irreducible closed subpreschemes of $X$ such that $\codim(Z_i,X)=1$, and $n_i\in\IZ$ such that there are only finitely many $i$ with $n_i\neq 0$.
		\item The group generated by all divisors is called the \defemph{divisor group} $\Div(X)$.
		\item A divisor $D$ is called \defemph{effective} if $n_i\geq 0$ for all $Z$. In this case we also write $D\geq 0$.
	\end{alphanumerate}
\end{defi}
\begin{rem}
	Since every irreducible closed subprescheme has a unique generic point (cf.\ \cite[Fact~2.1.9]{alggeo1}), we could have defined divisors equivalently as formal linear combinations of codimension-$1$ points of $X$. We will switch between these definitions whenever it is customary.
\end{rem}
\lbl{par:AuslanderBuchsbaum}Now assume that $X$ is, additionally, locally factorial: That is, all local rings $\Oo_{X,x}$ are factorial domains. For example, this is the case if $X$ is regular (hence in particular if $X$ is an abelian variety, see Fact~\reff{fact:groupSchemesSmooth}), since a famous theorem of Auslander and Buchsbaum says that every regular local ring is a UFD (we won't prove this here, but you can find it in \cite[Theorem~19.19]{eisenbudCommAlg}).

Let $Z\subseteq X$ be an irreducible closed subprescheme of codimension $1$ and $\Ii_Z\subseteq \Oo_X$ the sheaf of ideals cutting out $Z$. Then the stalks $\Ii_{Z,z}\subseteq \Oo_{X,z}$ for $z\in Z$ are prime ideals of height $1$, hence
$\Ii_{Z,z}=(p_z)$ is a principal ideal generated by a prime element $p_z$. Also if $x\notin Z$, then clearly $\Ii_{Z,x}=\Oo_{X,x}$. In particular, all stalks of $\Ii_Z$ are generated by a single element. Since $X$ is noetherian, we may thus apply the usual Nakayama arguments to see that $\Ii_Z$ \emph{is actually a line bundle!}

Hence if $D=\sum_in_iZ_i$ is a Weil divisor in the above situation, then we may put
\begin{align}\lbl{eq:OXD}
	\Ii_D=\bigotimes_i\Ii_{Z_i}^{\otimes n_i}\quad\text{and}\quad \Oo_X(D)=\Ii_D^{\otimes -1}\;,
\end{align}
and we see that the $\Oo_X(D)$ are line bundles. Actually, $D\mapsto \Oo_X(D)$ is a group homomorphism $\Oo_X(-)\colon \Div(X)\morphism\Pic(X)$.

There is another (but equivalent, as we will see) way to turn a divisor into a line bundle, as is explained in \cite[58]{alggeo2}. Let $K=\Oo_{X,\eta}$ be the stalk of $X$ at its generic point. Then $K$ is a field (since it is a zero-dimensional domain). Let $\Kk_X$ be the constant sheaf on $X$ with values in $K$. Moreover, let $X_1\subseteq X$ denote the set of codimension-$1$ points in $X$. For all $x\in X_1$, $\Oo_{X,x}$ is a one-dimensional regular ring, hence a DVR (by \cite[Theorem~21]{alg2}). So let $v_x$ denote the valuation of $\Oo_{X,x}$; and since $K$ is the quotient field of $\Oo_{X,x}$, we may extend $v_x$ to a valuation $v_x\colon K\morphism\IZ\cup\{\infty\}$ (where $v_x(0)=\infty$ by definition).

\begin{lem}\lbl{lem:divWellDefined}
	Assume we are in the above situation. For any open subset $U\subseteq X$ and all $f\in K^\times$ there are only finitely many $x\in U\cap X_1$ such that $v_x(f)\neq 0$.
\end{lem}
\begin{proof}
	Since $X$ is noetherian, every open subset is quasi-compact (see \cite[Definition~2.1.2\itememph{c}]{alg1}). Hence $U$ may be covered by finitely many affine opens, so it suffices to consider the case where $U$ is affine itself. Since $K$ is the quotient field of the domain $\Oo_X(U)$, we may write $f=\frac gh$ with $g,h\in\Oo_X(U)\setminus\{0\}$. Clearly, if the assertion holds for $g$ and $h$, then it also holds for $f$, whence we may assume $f\in\Oo_X(U)\setminus\{0\}$. Then $v_x(f)\geq 0$ for all $x\in U\cap X_1$ and $v_x(f)>0$ iff $x\in V(f)$. Since $\Oo_X(U)$ is a domain, by Krull's principal ideal theorem (see \cite[Theorem~11]{alg2}) there is a bijective correspondends between points $x\in V(f)$ of codimension $1$ and irreducible components $Z=\ov{\{x\}}$ of $V(f)$. But $V(f)$ has only finitely many irreducible components (by \cite[Proposition~2.1.1]{alg1}). This proves the assertion.
\end{proof}
Therefore, for every open subset $U\subseteq X$ we get a well-defined group homomorphism
\begin{align*}
	\div\colon \Kk_X(U)^\times&\morphism\Div(U)\\
	f&\longmapsto \sum_{x\in U\cap X_1}v_x(f)\ov{\{x\}}\;.
\end{align*}
And as usual, we extend this definition via $\div(0)=\infty$. Since $\Kk_X(X)^\times=K^\times$ (since $X$ is connected and $\Kk_X$ is defined as the constant sheaf with coefficients in $K$), we obtain the following definition.
\begin{defi}
	The quotient group $\Cl(X)=\Div(X)/\div(K^\times)$ is called the \defemph{divisor class group} of $X$.
\end{defi}
\begin{prop}\lbl{prop:divisors101}
	Assume we are in the above situation.
	\begin{alphanumerate}
		\item Let $D\in \Div(X)$ be a Weil divisor and define $\Oo_X(D)$ as in \eqreff{eq:OXD}. Then, for every open subset $U\subseteq X$,
		\begin{align*}
		\Gamma(U,\Oo_X(D))\cong \left\{f\in \Kk_X(U)\st\div(f)+D|_U\geq 0\right\}\;.
		\end{align*}
		\item The group homomorphism $\Oo_X(-)\colon \Div(X)\morphism\Pic(X)$ descends to an isomorphism $\Cl(X)\isomorphism\Pic(X)$.
	\end{alphanumerate}
\end{prop}
\begin{proof}
	Part~\itememph{a}. The right-hand side clearly defines a sheaf $\Ff$. Hence it suffices to show that $\Ff$ and $\Oo_X(D)$ coincide on small enough affine open subsets $U$ such that all line the finitely many bundles $\Ii_{Z_i}$, for which $Z_i$ has a non-zero coefficient $n_i\neq 0$ in $D$, trivialize on $U$. Then $\Ii_{Z_i}|_U$ is given by a principal ideal $(p_i)$ for some $p_i\in\Oo_X(U)$ which may be a prime or a unit. Then $p_i$ has the property that for all $x\in U\cap X_1$ we have $v_x(p_i)=1$ if $x$ is the generic point of $Z_i$ and $v_x(p_i)=0$ else (otherwise -- possibly after shrinking $U$ so that $\Ii_Z|_U$ becomes trivial for $Z=\ov{\{x\}}$ -- $p_i$ would be divisible by another prime, which cannot be). Thereby, for $f\in K$ the condition $\div(f)+D|_U\geq 0$ is equivalent to
	\begin{align}\lbl{eq:vxf}
		 v_x\bigg(f\cdot\prod_ip_i^{n_i}\bigg)\geq 0\quad\text{for all }x\in U\cap X_1\;.
	\end{align}
	However, if $R=\Oo_X(U)$, then $R$ is a normal domain (because all its localizations are factorial by assumption, hence normal). Thus \cite[Theorem~11.5]{matsumuraCRT} shows
	\begin{align}\lbl{eq:RpIntersection}
		R=\bigcap_{\hoehe(\pp)=1}R_\pp\;.
	\end{align}
	Together with \eqreff{eq:vxf} this implies $\div(f)+D|_U\geq 0$ iff $f\cdot\prod_ip_i^{n_i}\in R$. Thus $\Ff(U)=\prod_ip_i^{-n_i}R$. Having this established, it is immediate that $\Ff(U)$ and $\bigotimes_i\Ii_{Z_i}(U)^{\otimes -n_i}$ can be canonically identified.
	
	Part~\itememph{b}. We first show well-definedness. Let $D=\div(\lambda)$ for some $\lambda \in K^\times$. Let $U\cong \Spec R$ be an affine open subset of $X$, where $R$ is a normal domain (as explained in \itememph{a}). Then $\div(f)+D|_U\geq 0$ is equivalent to $\div(\lambda f)\geq 0$, which again holds iff $\lambda f\in R$ by \eqreff{eq:RpIntersection}. Thus $\Gamma(U,\Oo_X(D))\cong \lambda^{-1}R$ (using \itememph{a}), which shows that $\Oo_X(D)$ is a trivial line bundle with generator $\lambda^{-1}$.
	
	Now injectivity. Let $D\in\Div(X)$ be a Weil divisor such that $\Oo_X(D)\cong \Oo_X$. Choose a global generator $\lambda^{-1}\in K^\times$. We claim that $D=\div(\lambda)$. Indeed, if $x\in X_1$ is a codimension-$1$ point corresponding to $Z_i=\ov{\{x\}}$, then \eqreff{eq:OXD} shows $\Oo_X(D)_x\cong \Ii_{Z_i,x}^{\otimes -n_i}$ (as all other tensor factors are trivial at $x$). If we regard $\Ii_{Z_i,x}$ as the maximal ideal $\mm_{X,x}$ of the DVR, we obtain $\Ii_{Z_i,x}^{\otimes -n_i}\cong \mm_{X,x}^{-n_i}$. However, since $\Oo_X(D)$ is generated by $\lambda^{-1}$, this implies $\mm_{X,x}^{-n_i}=\lambda^{-1}\Oo_{X,x}$, hence $v_x(\lambda)=n_i$. This shows $D=\div(\lambda)$, as claimed.
	
	Finally, we prove surjectivity. Let $\Ll$ be a line bundle on $X$ and $\eta\in X$ be the generic point. Then $\Ll_\eta$ is a projective $\Oo_{X,\eta}$-module of rank $1$ -- i.e., a one-dimensional $K$-vector space. Choose an isomorphism $\phi\colon \Ll_\eta\isomorphism K$. For every affine open $U\subseteq X$, $R=\Oo_X(U)$ is a normal domain with fraction field $K$ and the restriction $r_\eta\colon \Ll(U)\morphism\Ll_\eta$ given by tensoring $\Ll(U)$ with $K$ (and sending $\lambda$ to $\lambda\otimes 1$). Since $\Ll(U)$ is a projective $R$-module (and thus torsion-free), $r_\eta$ is injective. Composing with $\phi$ thus gives a monomorphism $\Ll\monomorphism \Kk_X$, hence we may regard $\Ll$ as a sheaf of invertible ideals $\Ii\subseteq \Kk_X$.
	
	Now let $U\subseteq X$ be an affine open such that $\Ii|_U$ trivializes, so it is generated by some $\lambda\in \Kk_X(U)^\times\cong K^\times$. By Lemma~\reff{lem:divWellDefined}, there are only finitely many $x\in U\cap X_1$ such that $v_x(\lambda)\neq 0$. Since $X$ can be covered by finitely many such $U$, we see that there are only finitely many $x\in X_1$ satisfying $v_x(\Ii)\neq 0$. Now put
	\begin{align*}
		D=\sum_{x\in X_1}-v_x(\Ii)\ov{\{x\}}\;.
	\end{align*}
	\begin{claim}\lbl{claim:IOXD}
		We have $\Ii\cong \Oo_X(D)$.
	\end{claim}
	To prove this, it suffices to show $\Gamma(U,\Ii)=\Gamma(U,\Oo_X(D))$ for all affine opens $U\subseteq X$ with the following property (since such $U$ clearly form a topology base on $X$): For all the finitely many $x_i\in X_1$ with $v_{x_i}(\Ii)\neq 0$, the sheaf $\Ii_{Z_i}$ cutting out $Z_i=\ov{\{x_i\}}$ (which is a line bundle as noted above) trivializes on $U$, and so does $\Ii$. Let $U\cong \Spec R$ have this property. Then every such $\Ii_{Z_i}|_U$ is generated by some $p_i\in \Oo_X(U)$ which is a prime or a unit, and $\Ii$ can be generated by $\lambda\in K$. Consider the element
	\begin{align*}
		f=\lambda\cdot\prod_ip_i^{-v_{x_i}(\Ii)}\in K^\times\;.
	\end{align*}
	Since $v_x(\lambda)=v_x(\Ii)$ for all $x\in U\cap X_1$, we obtain $v_x(f)=0$ for all $x\in U\cap X_1$. Applying \eqreff{eq:RpIntersection} to $f$ and $f^{-1}$ we see that $f\in R^\times$. Hence
	\begin{align*}
		\Gamma(\Ii,U)=\lambda R=\prod_ip_i^{v_{x_i}(\Ii)}R\;.
	\end{align*}
	But the proof of \itememph{a} shows that the right-hand side can be canonically identified with $\Gamma(U,\Oo_X(D))$. This shows Claim~\reff{claim:IOXD} and we are done.
\end{proof}
\lbl{par:divisorPullback}We finish the subsection with a remark on divisors and base change. To this end, let $k$ be a field, $X$ a $k$-prescheme, and let $\ov{X}=X\times_k\Spec\ov{k}$ with $\pi\colon \ov{X}\morphism X$ the canonical projection. We assume that both $X$ and $\ov{X}$ are integral noetherian and locally factorial (so that our previous considerations are applicable). Let $Z\subseteq X$ be a closed irreducible subset of codimension $1$, so that the corresponding sheaf of ideals $\Ii_Z$ is a line bundle (as seen before). Then $\pi^*\Ii_Z\cong\Ii_Z\otimes_k\ov{k}$ is a sheaf of ideals in $\Oo_{\ov{X}}$ again since $\ov{k}$ is flat over $k$, and it cuts out the closed subprescheme $\ov{Z}=\pi^{-1}(Z)$. Since $\pi^*\Ii_Z$ is also a line bundle, all irreducible components of $\ov{Z}$ have codimension $1$ by Krull's principal ideal theorem (see \cite[Theorem~11]{alg2}).

Let $\eta\in Z$ be the generic point. It's easy to see (e.g., from arguments as in the proof of Proposition~\reff{prop:GeometricFibres101}\itememph{a}) that the irreducible components of $\ov{Z}$ correspond to the points $\ov{\eta}\in \ov{X}$ lying over $\eta$ (in particular, there are only finitely many points over $\eta$). Then $\pi^*\Ii_Z^{\otimes -1}$ can be represented as $\Oo_{\ov{X}}(\ov{D})$ for a divisor of the form
\begin{align*}
	\ov{D}=\sum_{\ov{\eta}\text{ over }\eta}v_{\ov{\eta}}(\Ii_Z)\ov{\{\ov{\eta}\}}
\end{align*}
in which $v_{\ov{\eta}}(\Ii_Z)>0$ (but not necessarily $v_{\ov{\eta}}(\Ii_Z)=1$ though).

The above can be immediately generalized to arbitrary divisors. If $D=\sum_{i=1}^rn_i\ov{\{\eta_i\}}$ is a Weil divisor on $X$ (for codimension-$1$ points $\eta_i\in X$), then $\pi^*\Oo_X(D)\cong \Oo_{\ov{X}}(\ov{D})$ for a Weil divisor $\ov{D}\in\Div(\ov{X})$ of the form
\begin{align}\lbl{eq:divisorPullback}
	\ov{D}=\sum_{i=1}^r\sum_{\pi(\ov{\eta}_i)=\eta_i}n_{\ov{\eta}_i}\ov{\{\ov{\eta}_i\}}\;,
\end{align}
where $n_{\ov{\eta}_i}$ is a non-zero multiple of $n_i$.

\section{More on étale and smooth morphisms}
\subsection{Some technical properties}
\begin{lem}
	\begin{alphanumerate}
		\item \lbl{lem:etaleProperties}The class of étale morphisms is base-local and closed under composition and base change.
		\item If $X\morphism Y$ and $X'\morphism Y'$ are étale morphisms of $S$-preschemes, then so is the canonically induced morphism $X\times_SX'\morphism Y\times_SY'$.
	\end{alphanumerate}
\end{lem}
\begin{proof}
	For \itememph{a}, note that flat morphisms clearly have these properties. Moreover, from Lemma~\reff{lem:unramified2} it is clear that unramified morphisms are base-local and closed under composition, and closedness under base change is easy to see from Proposition~\reff{prop:unramified}\itememph{a} since the module of Kähler differentials commutes with base change.
	
	For \itememph{b}, note that $X\times_SX'\morphism Y\times_SY'$ can be factored as $X\times_SX'\morphism X\times_SY'\morphism X'\times_SY'$. Both morphisms in this composition are étale since they are base changes of the étale morphisms $X'\morphism Y'$ and $X\morphism Y$ respectively, so the assertion follows from \itememph{a}.
\end{proof}
\begin{lem}\lbl{lem:etaleCompletion}
	Let $f\colon X\morphism Y$ be morphism of preschemes, $x\in X$ a point, and $y=f(x)$. Then $f$ is étale at $x$ with trivial residue field extension $\KK(x)/\KK(y)$ iff we have an isomorphism
	\begin{align*}
		f^*\colon \roof{\Oo}_{Y,y}\isomorphism \roof{\Oo}_{X,x}.
	\end{align*}
	(where $\roof{\phantom{x}}$ denotes the completion of local rings at their maximal ideals).
\end{lem}
\begin{proof}
	If $f^*$ is an isomorphism, then $\KK(x)=\KK(y)$ (and in particular, this is a separable extensiom) since completions preserve the residue fields. Also $\roof{\Oo}_{X,x}$ is flat over $\roof{\Oo}_{Y,y}$, hence over $\Oo_{Y,y}$ (since completion is flat). But then $\Oo_{X,x}$ is flat over $\Oo_{Y,y}$ too since $\roof{\Oo}_{X,x}$ is faithfully flat over $\Oo_{X,x}$ by Corollary~\reff{cor:completionFaithfullyFlat}\itememph{c}. Moreover, the maximal ideal of $\roof{\Oo}_{X,x}$ is given by $\mm_{X,x}\roof{\Oo}_{X,x}\cong \mm_{X,x}\otimes_{\Oo_{X,x}}\roof{\Oo}_{X,x}$ (using flatness), and the same is true for the maximal ideal of $\roof{\Oo}_{Y,y}$. Therefore $\mm_{Y,y}\Oo_{X,x}\morphism\mm_{X,x}$ becomes an isomorphism after tensoring with $\roof{\Oo}_{X,x}$ (as $f^*$ is an isomorphism), hence it must have been an isomorphism before by faithful flatness. By Lemma~\reff{lem:unramified2}, this shows that $f\colon X\morphism Y$ is flat and unramified at $x$, hence étale at $x$.
	
	Conversely, assume that $f$ is étale at $x$. We show by induction on $n$ that the induced morphism $f^*\colon \Oo_{Y,y}/\mm_{Y,y}^n\morphism\Oo_{X,x}/\mm_{X,x}^n$ is an isomorphism. The case $n=1$ is the assumption $\KK(x)=\KK(y)$. For $n\geq 2$, consider the diagram
	\begin{diagram*}
		\object{0,1.5}{$0$}[o1];
		\object{2,1.5}{$\mm_{Y,y}^{n-1}/\mm_{Y,y}^n$}[o2];
		\object{5,1.5}{$\Oo_{Y,y}/\mm_{Y,y}^n$}[o3];
		\object{8,1.5}{$\Oo_{Y,y}/\mm_{Y,y}^{n-1}$}[o4];
		\object{10,1.5}{$0$}[o5];
		\object{0,0}{$0$}[u1];
		\object{2,0}{$\mm_{X,x}^{n-1}/\mm_{X,x}^n$}[u2];
		\object{5,0}{$\Oo_{X,x}/\mm_{X,x}^n$}[u3];
		\object{8,0}{$\Oo_{X,x}/\mm_{X,x}^{n-1}$}[u4];
		\object{10,0}{$0$}[u5];
		\scriptsize
		\arrow{o1}{o2};
		\arrow{o2}{o3};
		\arrow{o3}{o4};
		\arrow{o4}{o5};
		\arrow{u1}{u2};
		\arrow{u2}{u3};
		\arrow{u3}{u4};
		\arrow{u4}{u5};
		\arrow{o2}{u2};
		\arrow{o3}{u3};
		\arrow{o4}{u4};
	\end{diagram*}
	The right-most vertical arrow is an isomorphism by the inductive hypothesis. We claim the left-most is an isomorphism too. If this can be shown, then the middle vertical arrow is an isomorphism as well by the five lemma, concluding the induction step.
	
	Using that $f$ is unramified and flat at $x$, we get $\mm_{X,x}=\mm_{Y,y}\Oo_{X,x}\cong \mm_{Y,y}\otimes_{\Oo_{Y,y}}\Oo_{X,x}$. Hence $\mm_{X,x}^{n-1}/\mm_{X,x}^n\cong \mm_{Y,y}^{n-1}/\mm_{Y,y}^n\otimes_{\Oo_{Y,y}}\Oo_{X,x}$. However, $\mm_{X,x}^{n-1}/\mm_{X,x}^n$ and $\mm_{Y,y}^{n-1}/\mm_{Y,y}^n$ are also vector spaces over $\KK(x)=\KK(y)$, and from this it's easy to check that the left vertical arrow is indeed an isomorphism.
\end{proof}
\subsection{Étale and smooth morphisms in the non-noetherian case}\lbl{subsec:nonnoetherian}
Quite a lot of results for locally noetherian preschemes can be carried over to the non-noetherian case as long as one imposes some kind of \emph{finite presentation} conditions. The trick is to reduce to the noetherian situation using the following rough idea: Suppose $A\morphism B$ is a ring morphism of finite presentation, i.e., $B\cong A[X_1,\ldots,X_n]/(f_1,\ldots,f_m)$. Let $\{a_i\}_{i\in I}$ be the (finite) set of coefficients of the polynomials $f_1,\ldots,f_m$. Let $A'\subseteq A$ be the subring generated by $\IZ$ and $\{a_i\}_{i\in I}$. Then $A\morphism B$ can be viewed as a base change of the morphism
\begin{align*}
A'\morphism A'[X_1,\ldots,X_n]/(f_1,\ldots,f_m)=B'\;,
\end{align*}
in which both $A'$ and $B'$ are noetherian since $A'$ and $B'$ have finite type over $\IZ$. For example, this idea is used in \url{https://amathew.wordpress.com/2010/12/26/} to derive the non-noetherian version of Chevalley's theorem (Proposition~\reff{prop:Chevalley}).

We would like to investigate how this can be applied to étale and smooth morphisms. Note that Proposition~\reff{prop:unramified} and Lemma~\reff{lem:unramified2} do not need any noetherianness assumptions (by Remark~\reff{rem:stuffWorkWithoutNoetherian}), hence the definition of unramified morphisms works in the non-noetherian case as well.
\begin{defi}
	Let $f\colon X\morphism S$ be a morphism of locally finite presentation.
	\begin{alphanumerate}
		\item $f$ is called \defemph{étale} if it is flat and unramified.
		\item $f$ is called \defemph{smooth} if every point $x\in X$ has an open neighbourhood $U$ such that $f|_U$ factors over an étale morphism $U\morphism\IA_S^n$.
	\end{alphanumerate}
\end{defi}
Then Proposition~\reff{prop:finiteEtale} and Proposition~\reff{prop:smoothnessCriteria} still hold if we add the assumption that every morphism be locally of finite presentation. In fact, all the arguments in the proofs still work under the finite presentation hypothesis, except for the following:
\begin{enumerate}
	\item Some fibrewise arguments really need noetherianness. But that's no problem since even if $X$ and $S$ are not locally noetherian, the fibre $X_s$ surely is, since it is of locally finite type over $\Spec \KK(s)$ if $f$ itself is of locally finite type.
	\item The critère de platitude par fibres (Lemma~\reff{lem:platitudeDeFibres}) assumes noetherianness.
\end{enumerate}
So we need to prove a generalization of Lemma~\reff{lem:platitudeDeFibres} that also works in the non-noetherian case. To do so, we follow \cite[\stackstag{00R7}]{stacks-project}.
\begin{lem}\lbl{lem:platitudeDeFibresNonNoetherian}
	Let $A\morphism B\morphism B'$ be local morphisms of local rings such that $B$ is essentially of finite presentation over $A$ (i.e., a localization of a finitely presented $A$-algebra) and $B'$ is essentially of finite presentation over $B$. Let $M$ be a finitely presented $B'$-module such that $M$ is flat over $A$ and $M/\mm_AM$ is flat over $B/\mm_AB$. Then $M$ is also flat over $B$.
\end{lem}
As suggested above, the trick is to reduce to the noetherian situation. The key lemma is the following, which together with its proof is taken from \cite[\stackstag{00R6}]{stacks-project}.
\begin{lem}\lbl{lem:noetherianApproximation}
	Let $(\Lambda,\leq)$ be a filtered (or directed, this means the same) set and let $\{A_\lambda\morphism B_\lambda\}_{\lambda\in\Lambda}$ and $\{M_\lambda\}_{\lambda\in\Lambda}$ be filtered systems such that the following conditions hold:
	\begin{ordnumerate}
		\item Each $A_\lambda\morphism B_\lambda$ is a noetherian local morphism of local rings, such that $B_\lambda$ is essentially of finite type over $A_\lambda$. 
		\item For each $\lambda\leq\mu$, $B_\mu$ is the localization of $B_\lambda\otimes_{A_\lambda}A_\mu$ at some prime ideal.
		\item Each $M_\lambda$ is a finitely presented $B_\lambda$-module.
		\item For each $\lambda\leq \mu$ we have $M_\mu\cong M_\lambda\otimes_{B_\lambda}B_\mu$.
	\end{ordnumerate}
	If $M=\colimit M_\lambda$ is flat over $A=\colimit A_\lambda$, then already some $M_\lambda$ is flat over $A_\lambda$.
\end{lem}
\begin{proof}
	Let $\mm_\lambda\subseteq A_\lambda$ denote the maximal ideal. Pick some $\lambda\in\Lambda$ and consider
	\begin{align*}
	\Tor_1^{A_\lambda}(A_\lambda/\mm_\lambda,M_\lambda)\cong\ker\left(\mm_\lambda\otimes_{A_\lambda}M_\lambda\morphism M_\lambda\right)\;.
	\end{align*}
	In particular, $\Tor_1^{A_\lambda}(A_\lambda/\mm_\lambda,M_\lambda)$ is a finitely generated $B_\lambda$-module (since we are very noetherian here). Pick generators $x_1,\ldots,x_n$. Since $M$ is flat over $A$, we see that
	\begin{align*}
	0=\ker\left(\mm_\lambda A\otimes_AM\morphism M\right)&\cong\colimit[\lambda\leq\mu]\ker\left(\mm_\lambda A_\mu\otimes_{A_\mu}M_\mu\morphism M_\mu\right)\\
	&\cong \colimit[\lambda\leq\mu]\Tor_1^{A_\mu}(A_\mu/\mm_\lambda A_\mu,M_\mu)
	\end{align*}
	(using that tensor products commute with colimits -- in fact, every step in the explicit construction of the tensor products does), hence there is some $\mu\geq \lambda$ such that the image of each $x_i$ in $\Tor_1^{A_\mu}(A_\mu/\mm_\lambda A_\mu,M_\mu)$ vanishes. Then also \begin{align*}
	\Tor_1^{A_\lambda}(A_\lambda/\mm_\lambda,M_\lambda)\otimes_{A_\lambda}A_\mu\morphism \Tor_1^{A_{\mu}}(A_\mu/\mm_\lambda A_\mu,M_\mu)\;.
	\end{align*}
	We wish to show that then $M_\mu$ is already flat over $A_\mu$.
	
	Put $M'=M_\lambda\otimes_{A_\lambda}A_\mu$. By assumptions 2 and 4, $M_\mu$ is a localization of $M'$. Note that $M'/\mm_\lambda M'\cong M/\mm_\lambda M\otimes_{A_\lambda/\mm_\lambda A_\lambda}A_\mu/\mm_\lambda A_\mu$ is flat over $A_\mu/\mm_\lambda A_\mu$, since it is a base change of the (obviously flat) $A_\lambda/\mm_\lambda$-vector space $M_\lambda/\mm_\lambda M_\lambda$. Then also $M_\mu/\mm_\lambda M_\mu$ is flat over $A_\mu/\mm_\lambda A_\mu$ (being a localization of $M'/\mm_\lambda M'$). Hence it suffices to show $\Tor_1^{A_\mu}(A_\mu/\mm_\lambda A_\mu,M_\mu)=0$, since then Corollary~\reff{cor:localFlatnees} of the the local flatness criterion is applicable (with $I=\mm_\lambda A_\mu$, which we'll use as a shortcut in the following).
	
	Now $\Tor_1^{A_\mu}(A_\mu/I,M_\mu)\cong\ker(I\otimes_{A_\mu}M_\mu\morphism M_\mu)$ is a localization (as modules over $B_\lambda\otimes_{A_\lambda}A_\mu$) of $\Tor_1^{A_\mu}(A_\mu/I,M')\cong\ker(I\otimes_{A_\mu}M'\morphism M')$. Therefore it suffices to show that
	\begin{align}\lbl{eq:TorEpi}
	\Tor_1^{A_\lambda}(A_\lambda/\mm_\lambda,M_\lambda)\otimes_{A_\lambda}A_\mu\morphism \Tor_1^{A_{\mu}}(A_\mu/I,M')
	\end{align}
	is surjective, since then $\Tor_1^{A_{\mu}}(A_\mu/I,M')$ is generated (as a $B_\lambda\otimes_{A_\lambda}A_\mu$-module) by the images of the $x_i$, and thus its localization $\Tor_1^{A_\mu}(A_\mu/I,M_\mu)$ is generated by the images of the $x_i$  too (as a $B_\mu$-module this time) -- but these images vanish, hence so does $\Tor_1^{A_\mu}(A_\mu/I,M_\mu)$ and we're done as explained above.
	
	To show surjectivity in \eqreff{eq:TorEpi}, it suffices to show that both arrows in the factorization
	\begin{align*}
	\Tor_1^{A_\lambda}(A_\lambda/\mm_\lambda,M_\lambda)\otimes_{A_\lambda}A_\mu\morphism \Tor_1^{A_\lambda}(A_\mu/I,M_\lambda)\morphism \Tor_1^{A_\mu}(A_\mu/I,M')
	\end{align*}
	are surjective. This follows (in this order) from Lemma~\reff{lem:TorSurjective2} (applied to $A_\lambda\morphism A_\lambda/\mm_\lambda\morphism A_\mu/I$) and Lemma~\reff{lem:TorSurjective} (applied to $A_\lambda\morphism A_\mu$ and $\mm_\lambda$) below.
\end{proof}
The following two technical lemmas have direct proofs in The Stacks Project (cf.\ \cite[\stackstag{00MM} and \stackstag{00MN}]{stacks-project}), but if you use the Grothendieck spectral sequence they become almost trivial.
\begin{lem}\lbl{lem:TorSurjective}
	Let $A\morphism A'$ be a morphism of rings and $M$ an $A$-module. Put $M'=M\otimes_AA'$. If $I\subseteq A$ is an ideal and $I'=IA'$, then the canonical morphism
	\begin{align*}
	\Tor_1^A(A'/I',M)\morphism\Tor_1^{A'}(A'/I',M')
	\end{align*}
	is surjective.
\end{lem}
\begin{proof}
	Since the functor $-\otimes_AM\colon \cat{Mod}(A)\morphism \cat{Mod}(A')$ is isomorphic to the composition of $-\otimes_AA'\colon\cat{Mod}(A)\morphism \cat{Mod}(A')$ and $-\otimes_{A'}M\colon \cat{Mod}(A')\morphism \cat{Mod}(A')$, we obtain a Grothendieck spectral sequence
	\begin{align*}
	E_{p,q}^2=\Tor_p^{A'}\left(A'/I',\Tor_q^A(A/I,M)\right)\converge \Tor_{p+q}^A(A'/I',M)
	\end{align*}
	(this is the same as in \eqreff{eq:TorGrothendieckSS}). In low degrees this gives rise to an exact sequence
	\begin{align*}
	0\morphism E_{0,1}^\infty\morphism\Tor_1^A(A'/I',M)\morphism E_{1,0}^\infty\morphism 0\;,
	\end{align*}
	in which $E_{1,0}^\infty\cong E_{1,0}^2\cong \Tor_1^{A'}(A'/I',M')$, using that on the $E_2$ page (and all higher pages) any differential from and to $E_{1,0}^2$ leaves the support of the spectral sequence. Hence the canonical morphism in question is indeed surjective.
\end{proof}
\begin{lem}\lbl{lem:TorSurjective2}
	Let $A\morphism A'\morphism A''$ be morphisms of rings and $M$ an $A$-module such that $M'=M\otimes_AA'$ is flat over $A'$. Then the canonical morphism
	\begin{align*}
	\Tor_1^A(A,M)\otimes_{A'}A''\morphism \Tor_1^A(A'',M)
	\end{align*}
	is surjective.
\end{lem}
\begin{proof}
	We use the Grothendieck spectral sequence as in Lemma~\reff{lem:TorSurjective} to obtain
	\begin{align*}
	E_{p,q}^2=\Tor_p^{A'}(A'',\Tor_q^A(A',M))\converge \Tor_{p+q}^A(A'',M)\;.
	\end{align*}
	This time we obtain a low-degree exact sequence
	\begin{align*}
	0\morphism E_{0,1}^\infty\morphism\Tor_1^A(A'',M)\morphism E_{1,0}^\infty\morphism 0\;,
	\end{align*}
	in which $E_{1,0}^\infty\cong E_{1,0}^2\cong \Tor_1^{A'}(A'',M')=0$ since we assume $M'$ to be flat over $A'$. Thus $E_{0,1}^\infty\morphism \Tor_1^A(A'',M)$ is an isomorphism. Also $E_{0,1}^2\cong \Tor_1^A(A',M)\otimes_{A'}A''$ surjects onto $E_{0,1}^\infty$ (again by a quick analysis of the differentials), whence we are done.
\end{proof}
Now we finally have enough together to prove the critère de platitude par fibres in the non-noetherian case.
\begin{proof}[Proof of Lemma~\reff{lem:platitudeDeFibresNonNoetherian}]
	Let's first construct noetherian approximations (as in Lemma~\reff{lem:noetherianApproximation}) to $A$, $B$, $B'$, and $M$. To do so, choose presentations $M\cong \coker(\mu\colon B'^r\morphism B'^s)$ as well as
	\begin{align*}
	B'\cong \big(B[Y_1,\ldots,Y_n]/(\beta_1,\ldots,\beta_c)\big)_\qq\quad\text{and}\quad B\cong \big(A[X_1,\ldots,X_m]/(\alpha_1,\ldots,\alpha_d)\big)_\pp
	\end{align*}
	(for some prime ideals $\qq$ and $\pp$ in the respective rings). Let $\{b'_j\}_{j\in J'}\subseteq B'$ be the set of components of $\mu(e_1),\ldots,\mu(e_r)\in B'^s$, where $e_1,\ldots,e_r\in B'^r$ denote the standard basis. Note that $J'$ is finite since $\#J'\leq rs$. Each $b'_j$ can be represented as a fraction of two polynomials in $B[Y_1,\ldots,Y_n]$. Let $\{b_j\}_{j\in J}$ be the set of coefficients occuring in the denominators and numerators of all the $b'_j$ as well as in $\beta_1,\ldots,\beta_c$. Then $J$ is again a finite set and each $b_j$ can be represented as a quotient of two polynomials in $A[X_1,\ldots,X_m]$. Let $\{a_i\}_{i\in I}$ be the set of all coefficients occuring in the denominators and numerators as well as in $\alpha_1,\ldots,\alpha_d$. Then $I$ is again a finite set.
	
	Now let $\{A'_\lambda\}_{\lambda\in \Lambda}$ be the filtered system of subrings $A'_\lambda\subseteq A$ such that each $A'_\lambda$ contains $\{a_i\}_{i\in I}$ and has finite type over $\IZ$. Let $\mm_\lambda\subseteq A'_\lambda$ be preimage of the maximal ideal $\mm_A\subseteq A$ and put $A_\lambda=A_{\mm_\lambda}$ (we'll denote the maximal ideal of the local ring $A_\lambda$ by $\mm_\lambda$ again). Put
	\begin{align*}
	B_\lambda\cong \big(A_\lambda[X_1,\ldots,X_m]/(\alpha_1,\ldots,\alpha_d)\big)_{\pp_\lambda}\quad\text{and}\quad B'_\lambda=\big(B_\lambda[Y_1,\ldots,Y_n]/(\beta_1,\ldots,\beta_c)\big)_{\qq_\lambda}
	\end{align*}
	(where $\pp_\lambda$ and $\qq_\lambda$ denote the preimages of $\pp$ and $\qq$ in the respective polynomial rings) and finally let $M_\lambda=\coker\left(\mu_\lambda\colon {B'}_\lambda^r\morphism {B'}_\lambda^s\right)$, where $\mu_\lambda$ is defined by $\mu_\lambda(e_i)=\mu(e_i)$. Note that all these expressions have meaning since $\{a_i\}_{i\in I}\subseteq A_\lambda$ by construction, hence $\{b_j\}_{j\in J}\subseteq B_\lambda$, thus $\{b'_j\}_{j\in J'}\subseteq B'_\lambda$.
	
	It's easy to see that the $\{A_\lambda\}_{\lambda\in \Lambda}$, $\{B_\lambda\}_{\lambda\in \Lambda}$, $\{B'_\lambda\}_{\lambda\in \Lambda}$, and $\{M_\lambda\}_{\lambda\in \Lambda}$ satisfy the assumptions of Lemma~\reff{lem:noetherianApproximation}, and also that $A=\colimit A_\lambda$, $B=\colimit B_\lambda$, $B'=\colimit B'_\lambda$, and $M=\colimit M_\lambda$. In particular, since $M$ is flat over $A$ we may apply Lemma~\reff{lem:noetherianApproximation} to see that for some $\vartheta\in\Lambda$, $M_\vartheta$ is already flat over $A_\vartheta$, and hence the same is true for all $\mu\geq \vartheta$ since $M_\mu$ is a localization of $M_\lambda\otimes_{A_\vartheta}A_\vartheta$. Moreover, since $M/\mm_AM\cong \colimit M_\lambda/\mm_\lambda M_\lambda$ is flat over $B/\mm_AB\cong \colimit B_\lambda/\mm_\lambda B_\lambda$, Lemma~\reff{lem:noetherianApproximation} shows that $M_\eta/\mm_\eta M_\eta$ is flat over $B_\eta/\mm_\eta B_\eta$ for some $\eta\in \Lambda$ (and thus for all $\mu\geq \eta$ as well). Now choose $\mu\geq \vartheta,\eta$. Then the noetherian case of the critère platitude par fibres (Lemma~\reff{lem:platitudeDeFibres}) shows that $M_\mu$ is flat over $B_\mu$. Hence $M$ is flat over $B$ since it is a localization of $M_\mu\otimes_{B_\mu}B$. We're done.
\end{proof}

\printbibliography

\end{document}          

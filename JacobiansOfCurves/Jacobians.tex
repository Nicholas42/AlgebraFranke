\documentclass[a4paper,parskip=half,numbers=enddot, DIV=12]{scrreprt}
%\usepackage[utf8]{inputenc}

\usepackage{../header}
\usepackage{../frankenumbering2}
\usepackage{../shortcuts2}

\usepackage{blindtext}

\usepackage{eurosym}
\usetikzlibrary{fadings}

\usepackage{csquotes}
%\usepackage{tikz-cd}%I cannot draw diagrams without it - Felix. %well, I can - Ferdinand
\usepackage[backend=biber,style=alphabetic]{biblatex}
\setcounter{biburlnumpenalty}{7000}
\setcounter{biburllcpenalty}{7000}
\setcounter{biburlucpenalty}{8000}
\addbibresource{../literatur.bib}

% Title Page
\title{Jacobians of Curves}
\author{Lecture notes by Ferdinand Wagner}
\date{Wintersemester 2018/19}

\displaywidowpenalty=8000
%\postdisplaypenalty=8000
\widowpenalty=8000
\clubpenalty=8000

\newcommand{\vC}{\v{C}}
\renewcommand{\geq}{\geqslant}
\renewcommand{\leq}{\leqslant}

\begin{document}
\pagenumbering{Alph}
\maketitle
\pagenumbering{roman}

\thispagestyle{plain}
This text consists of notes on the lecture Jacobians of Curves, taught at the University of Bonn by Professor Jens Franke in the winter term (Wintersemester) 2018/19.

Please report bugs, typos etc. through the \emph{Issues} feature of github.

\tableofcontents

%\addchap{Introduction}
\pagenumbering{arabic}
%Nothing here yet. 

\chapter{Introduction and preparations}

\section{A note on limits and their derived functors}
	Let $X_\bullet\colon \ldots\xrightarrow{p_{i+1}}X_i\morphism[p_i]\ldots\morphism[p_2]X_1\morphism[p_1]X_0$ be a diagram of abelian groups or $R$-modules. As usual, we may view $X_\bullet$ as a functor $X_\bullet\colon (\IN,\geq)\morphism \cat{Ab}$ or $\cat{Mod}(R)$, where the category $(\IN,\geq)$ has the nonnegative integers as objects and an arrow $j\morphism i$ iff $j\geq i$. Let 
	\begin{align*}
		d\colon \prod_{i=0}^{\infty}X_i\morphism\prod_{i=0}^\infty X_i\;,\quad d\left(x_i\right)_{i=0}^\infty=\left(p_{i+1}(x_{i+1})-x_i\right)_{i=0}^\infty\;.
	\end{align*}
	Then we put
	\begin{align*}
		\limit[i\in\IN]X_i=\ker d\quad\text{and}\quad\limit[i\in\IN][^1]X_i=\coker d\;.
	\end{align*}
\begin{rem}
	It is easy to see that $\limit X_i$ equals the usual category-theoretical limit (that's how you construct it). It can also be shown that $\limit[][^1]$ is the first right-derived functor of $\limit$, and that its higher derived functors vanish.
\end{rem}
\begin{fact}\lbl{fact:6termLimitSequence}
	Let $0\morphism X'_\bullet\morphism X_\bullet\morphism X''_\bullet\morphism 0$ be a short exact sequence of diagrams of the above type. Then there is a canonical exact sequence
	\begin{align*}
		0\morphism \limit[i\in\IN] X'_i\morphism \limit[i\in\IN] X_i\morphism \limit[i\in\IN] X''_i\morphism \limit[i\in\IN][^1] X'_i\morphism \limit[i\in\IN][^1]X_i\morphism \limit[i\in\IN][^1] X''_i\morphism 0\;.
	\end{align*}
\end{fact}
\begin{proof}
	Since products preserve exact sequences in $\cat{Ab}$ or $\cat{Mod}(R)$, we get a diagram
	\begin{diagram*}
		\object{0,2}{$0\vphantom{\displaystyle\prod_{i=0}^\infty }$}[0o];
		\object{2,2}{$\displaystyle\prod_{i=0}^\infty X'_i$}[1o];
		\object{4,2}{$\displaystyle\prod_{i=0}^\infty X_i$}[2o];
		\object{6,2}{$\displaystyle\prod_{i=0}^\infty X''_i$}[3o];
		\object{8,2}{$0\vphantom{\displaystyle\prod_{i=0}^\infty }$}[4o];
		\object{0,0}{$0\vphantom{\displaystyle\prod_{i=0}^\infty }$}[0u];
		\object{2,0}{$\displaystyle\prod_{i=0}^\infty X'_i$}[1u];
		\object{4,0}{$\displaystyle\prod_{i=0}^\infty X_i$}[2u];
		\object{6,0}{$\displaystyle\prod_{i=0}^\infty X''_i$}[3u];
		\object{8,0}{$0\vphantom{\displaystyle\prod_{i=0}^\infty }$}[4u];
		\scriptsize
		\arrow{0o}{1o};
		\arrow{1o}{2o};
		\arrow{2o}{3o};
		\arrow{3o}{4o};
		\arrow{0u}{1u};
		\arrow{1u}{2u};
		\arrow{2u}{3u};
		\arrow{3u}{4u};
		\arrow{1o}{1u}[left][$d'$];
		\arrow{2o}{2u}[left][$d$];
		\arrow{3o}{3u}[left][$d''$];
	\end{diagram*}
	with exact rows. Then the snake lemma finishes the job.
\end{proof}
\begin{fact}\lbl{fact:limVanishing}
	Let $X_\bullet$ have the property that for every $i\in\IN$ there is a $j\geq i$ such that the composition $p_{j,i}\colon X_j\morphism[p_j]X_{j-1}\xrightarrow{p_{j-1}}\ldots\xrightarrow{p_{i+1}}X_i$ vanishes. Then
	\begin{align*}
		\limit[i\in\IN]X_i=\limit[i\in\IN][^1]X_i=0\;.
	\end{align*}
\end{fact}
\begin{proof}
	If $x=\left(x_i\right)_{i=0}^\infty\in\limit X_i$, then $x_i=p_{j,i}(x_j)$ for all $j\geq i$ by construction, hence $x_i=0$ for all $i\in\IN$. Moreover, let
	\begin{align*}
		s\colon \prod_{i=0}^{\infty}X_i\morphism\prod_{i=0}^\infty X_i\;,\quad s(x)_i=\sum_{j\geq i}p_{j,i}(x_j)\;.
	\end{align*}
	By assumption $s$ is well-defined. Then
	\begin{align*}
		d(s(x))_i=p_{i+1}\bigg(\sum_{j\geq i+1}p_{j,i+1}(x_j)\bigg)-\sum_{j\geq i}p_{j,i}(x_j)=-p_{i,i}(x_i)=-x_i\;.
	\end{align*}
	Hence $-s$ is a right-inverse of $d$, so $\limit[][^1]X_i=\coker d$ vanishes as well.
\end{proof}
\begin{fact}\lbl{fact:MittagLeffler}
	Let $X_\bullet$ have the \defemph{Mittag-Leffler property} that for every $i\in\IN$ there is a $j\geq i$ such that for all $k\geq j$ the images of $p_{j,i}$ and $p_{k,i}$ in $X_i$ coincide. Then $\limit[][^1]X_i=0$.
\end{fact}
\begin{proof}
	Let's first deal with the special case that each $p_i\colon X_i\morphism X_{i-1}$ is surjective. Let $x=\left(x_i\right)_{i=0}^\infty\in\prod_{i=0}^\infty X_i$. For every $i\in\IN$ we may select $x_j^{(i)}\in X_j$ for all $j\geq i$ in such a way that $x_i^{(i)}=x_i$ and $p_{j+1}\big(x_{j+1}^{(i)}\big)=x_j^{(i)}$. Then $s(x)$ defined by
	\begin{align*}
		s(x)_i=\sum_{k=0}^{i-1}x_i^{(k)}
	\end{align*}
	is a preimage of $x$ under $d$, so $\limit[][^1]X_i=\coker d=0$ in this case.
	
	Now let $X_\bullet$ be arbitrary with the Mittag-Leffler property. Let $Y_i=\bigcap_{j\geq i}p_{j,i}(X_j)\subseteq X_i$. Then $\limit[][^1]Y_i=0$ by the special case we just treated, and $\limit[][^1]X_i/Y_i=0$ by Fact~\reff{fact:limVanishing}. Since $\limit[][^1]X_i$ is sandwiched between these two in the exact sequence from Fact~\reff{fact:6termLimitSequence}, this shows $\limit[][\smash{^1}]X_i=0$, as required.
\end{proof}
\section{The theorem about formal functions}
Let $f\colon X\morphism Y=\Spec A$ be a  morphism of quasi-compact schemes. Let $I\subseteq A$ be any ideal. Consider
\begin{align*}
	i_n\colon X_n=X\times_Y\Spec (A/I^n)\morphism X\;,
\end{align*}
which is a base change of the closed immersion $Y_n=\Spec (A/I^n)\monomorphism \Spec A$, hence indeed a closed immersion itself. Also, if $f$ is proper, then so is $X_n\morphism Y_n$ because properness is another \emph{proper}ty (tee-hee) that is stable under base change (by \cite[Remark~2.4.1]{alggeo2}).

Let $\Ff$ be a quasi-coherent sheaf of $\Oo_X$-modules and $\Ff|_{X_n}=i_n ^*\Ff$ its restriction to $X_n$ (this notation is slightly abusive, but convenient). We put $\Ff_n=i_{n,*}\Ff|_{X_n}$. It's easy to check (e.g.\ affine-locally) that $\Ff_n\cong \Ff/I^n\Ff$. Since $i_n$ is a closed immersion and thus affine, we have an isomorphism $H^p(X,\Ff_n)\cong H^p(X_n,\Ff|_{X_n})$ for all $p\geq 0$ by \cite[Corollary~1.5.1]{alggeo2}. Together with the canonical projection $\Ff_{n+1}\cong \Ff/I^{n+1}\Ff\morphism \Ff/I^n\Ff\cong \Ff_n$ this gives canonical morphisms $H^p(X_{n+1},\Ff|_{X_{n+1}})\morphism H^p(X_n,\Ff|_{X_n})$ for all $n\in\IN$.

The canonical morphism $\Ff\morphism i_{n,*}i_n^*\Ff=\Ff_n$ induces a morphism
\begin{align}\lbl{eq:Hmorphism}
	H^p(X,\Ff)\morphism H^p(X,\Ff_n)\cong H^p(X_n,\Ff|_{X_n})
\end{align}
for all $p\geq 0$ (the isomorphism on the right-hand side comes from the fact that $i_n$ is a closed immersion, hence affine, and we can apply \cite[Corollary~1.6.1]{alggeo2}). This is a morphism of $A$-modules, but $H^p(X_n,\Ff|_{X_n})$ is actually an $A/I^n$-module, so \eqreff{eq:Hmorphism} factors over
\begin{align*}
	H^p(X,\Ff)/I^nH^p(X,\Ff)\morphism H^p(X_n,\Ff|_{X_n})\;.
\end{align*}
This is compatible with the canonical morphisms $H^p(X_{n+1},\Ff|_{X_{n+1}})\morphism H^p(X_n,\Ff|_{X_n})$ (you can just check that on an affine \v Cech covers). Passing to the limit gives a morphism
\begin{align}\lbl{eq:formalFunctions}
	H^p(X,\Ff)^\complete\morphism\limit[n\in\IN]H^p(X_n,\Ff|_{X_n})\;,
\end{align}
where $\roof{\phantom{x}}$ denotes the $I$-adic completion.
\begin{thm}[Grothendieck--Zariski]\lbl{thm:FormalFunctions}
	When $f\colon X\morphism Y=\Spec A$ is proper (in which case $X$ is automatically a quasi-compact scheme), $A$ is noetherian and $\Ff$ is a coherent sheaf of $\Oo_X$-modules, then \eqreff{eq:formalFunctions} is an isomorphism
	\begin{align*}
		H^p(X,\Ff)^\complete\isomorphism \limit[n\in\IN]H^p(X_n,\Ff|_{X_n})\;.
	\end{align*}
\end{thm}

\begin{proof}
	The following proof is essentially the one from \cite[(4.1.7)]{egaIII}. Professor Franke also pointed out that the idea is pretty similar to the proof of the Artin--Rees lemma. Let $I\subseteq A$ be the ideal under consideration and let $R=\bigoplus_{n\geq 0}I^n$ be the Rees algebra associated to $I$. Then
	\begin{align*}
		K^p=\bigoplus_{n\geq 0}H^p(X,I^n\Ff)
	\end{align*}
	is a module over $R$ as $i\in I^m$ (considered as the $m\ordinalth$ homogeneous component of $R$) maps $I^n\Ff$ to $I^{n+m}\Ff$.
	\begin{claim}\lbl{claim:FFclaim1}
		$K^p$ is a finitely generated $R$-module for all $p\geq 0$.
	\end{claim}
	Assuming this for the moment, recall that $\Ff_n\cong \Ff/I^n\Ff$ and $H^p(X_,\Ff|_{X_n})\cong H^p(X,\Ff_n)$, so the long exact cohomology sequence associated to $0\morphism I^n\Ff\morphism \Ff\morphism \Ff_n\morphism 0$ appears as
	\begin{align}\lbl{eq:FFSeq1}
		H^p(X,I^n\Ff)\morphism H^p(X,\Ff)\morphism H^p(X_n,\Ff|_{X_n})\morphism H^{p+1}(X,I^n\Ff)\;.
	\end{align}
	As pointed out after \eqreff{eq:Hmorphism}, $H^p(X,\Ff)\morphism H^p(X_n,\Ff|_{X_n})$ factors over $H^p(X,\Ff)/I^nH^p(X,\Ff)$, hence we can turn equation \eqreff{eq:FFSeq1} into an exact sequence
	\begin{align}\lbl{eq:FFSeq2}
		0\morphism U_n\morphism H^p(X,\Ff)/I^nH^p(X,\Ff)\morphism H^p(X_n,\Ff|_{X_n})\morphism V_n\morphism 0\;,
	\end{align}
	where $U_n$ is a suitable quotient of $H^p(X,I^n\Ff)$ and $V_n\subseteq H^{p+1}(X,I^n\Ff)$ some submodule. This makes $U=\bigoplus_{n\geq 0}U_n$ a quotient of $K^p$ and $V=\bigoplus_{n\geq 0}V_n$ an $R$-submodule of $K^{p+1}$.
	\begin{claim}\lbl{claim:FFclaim2}
		We have $\limit U_n=\limit[][^1]U_n=0$ and $\limit V_n=\limit[][^1]V_n=0$.
	\end{claim}
	Before we prove this (and Claim~\reff{claim:FFclaim1}), let's see how Theorem~\reff{thm:FormalFunctions} follows from it. Let $W_n$ be the image of $H^p(X,\Ff)/I^nH^p(X,\Ff)$ in $H^p(X_n,\Ff|_{X_n})$. We may split \eqreff{eq:FFSeq2} into two short exact sequences $0\morphism U_n\morphism H^p(X,\Ff)/I^nH^p(X,\Ff)\morphism W_n\morphism 0$ and $0\morphism W_n\morphism H^p(X_,\Ff|_{X_n})\morphism V_n\morphism 0$. Applying Fact~\reff{fact:6termLimitSequence} to the first one gives $H^p(X,\Ff)^\complete\cong \limit[][^1]W_n$. Then the six-term exact sequence associated to the second proves $\limit W_n\cong \limit H^p(X_n\Ff|_{X_n})$ and we are done.
	
	It remains to show the two claims. Note that the Rees algebra $R$ is noetherian. Indeed, $I$ is finitely generated as an ideal in the noetherian ring $A$, hence $R$ is of finite type over $A$. Let's also make the following convention: Whenever we write $I^kU_n$ or $I^kV_n$ in the following, this means multiplication as $A$-modules and the result is contained in $U_n$ resp.\ $V_n$ again, whereas $R_kU_n$ or $R_kV_n$ means multiplication by the $k\ordinalth$ homogeneous component of $R$ (which equals $I^k$ as well), so the result is contained on $U_{k+n}$ resp.\ $V_{k+n}$.
	
	\emph{Proof of Claim~\reff{claim:FFclaim2}.} Note that $U$ is finitely generated over $R$, since it is a quotient of the finitely generated $R$-module $K^p$. Fix a finite set of generators and let $d_0$ the maximal non-zero homogeneous components occurring in this set. Then $U_n=R_nU_0+R_{n-1}U_1+\ldots+R_{n-d_0}U_{d_0}$ for all $n\geq d_0$. In particular, $U_{k+n}=R_kU_n$ for all $n\geq d_0$. Thus, for every $n\geq d_0$ the image of $U_{2n}=R_nU_n$ in $U_n$ is contained in $I^nU_n$. But $U_n\subseteq H^p(X,\Ff)/I^nH^p(X,\Ff)$, so $I^nU_n$ vanishes. Therefore, the property from Fact~\reff{fact:limVanishing} is fulfilled for all $n\geq d_0$. But then it clearly holds for all $n\geq 0$ as well, so Fact~\reff{fact:limVanishing} is applicable.
	
	Similarly, $V$ is finitely generated as a submodule of $K^{p+1}$, which is finitely generated over the noetherian ring $R$ by Claim~\reff{claim:FFclaim1}. By the same argument as above we find a $d_1$ such that $V_n=R_nV_0+R_{n-1}V_1+\ldots+R_{n-d_1}V_{d_1}$ for all $n\geqslant d_1$. In particular, we have $V_{k+n}=R_kV_n$ for all $n\geq d_1$. Thus, for $n\geq d_1$ the image of $V_{2n}$ in $V_n$ is contained in $I^nV_n$. But $I^nV_n$ vanishes again, since $V_n$ is the image of $H^p(X_n,\Ff|_{X_n})$, which is a $A/I^n$-module. As above, we can apply Fact~\reff{fact:limVanishing}. This shows Claim~\reff{claim:FFclaim2}.
	
	\emph{Proof of Claim~\reff{claim:FFclaim1}.} Let $\upsilon\colon\snake{Y}=\Spec R\morphism Y$ correspond to $A\monomorphism R$ and let $\xi\colon \snake{X}=X\times_Y\snake{Y}\morphism X$ be its base change by $f$. Note that $\xi$ is affine as a base change of the affine morphism $\upsilon$ (we use \cite[Corollary~2.5.1]{alggeo1} here). We claim
	\begin{align*}
		\xi_*\xi^*\Ff\cong\bigoplus_{n\geq 0}I^n\Ff\;.
	\end{align*}
	Indeed, this is easily checked affine-locally (where $\xi^*$ is given by tensoring with $R$); we leave the details to the reader. Also $H^p(\snake{X},\xi^*\Ff)\cong H^p(X,\xi_*\xi^*\Ff)$ as $\xi$ is affine. This shows
	\begin{align*}
		H^p(\snake{X},\xi^*\Ff)\cong H^p(X,\xi_*\xi^*\Ff)\cong H^p\bigg(X,\bigoplus_{n\geq 0}I^n\Ff\bigg)\cong \bigoplus_{n\geq 0}H^p(X,I^n\Ff)=K^p\;.
	\end{align*}
	Note that direct sums usually \emph{don't} commute with cohomology, but here they do, because $X$ is quasi-compact and $\bigoplus_{n\geq 0}I^n\Ff$ is quasi-coherent (for which we need quasi-compactness as well), so we may compute $H^p\left(X,\bigoplus_{n\geq 0}I^n\Ff\right)$ via finite affine \v Cech covers. In this case, the products in the \v Cech complex are all finite, hence commute with the direct sum, which is what we needed.
	
	Now $\snake{f}\colon \snake{X}\morphism\snake{Y}=\Spec R$ is proper (as a base change of the proper morphism $f$), hence the right-hand side is a finitely generated $R$-module by our finiteness results for the cohomology of proper morphisms (cf.\ \cite[Theorem~5]{alggeo2}). We win.
\end{proof}
\begin{rem}
	Note that in the lecture Franke used $\Kk_n\cong \Jj^n\Ff$ instead of $I^n\Ff$, where $\Jj=f^{-1}\Ii$ is the inverse image (in the sense of Definition~\reff{def:f-1}). But $\Jj^n\Ff\cong I^n\Ff$ -- which is not that surprising, since the $I^n$-action on $\Ff$ is given via the algebraic component $\Oo_Y\morphism f_*\Oo_X$ of $f$, so $I^n\Ff=\Jj^n\Ff$ is pretty obvious from the construction of $f^{-1}$ described in the proof of Lemma~\reff{lem:f-1} below. I prefer the notation $I^n\Ff$ -- in particular, this is how Grothendieck denotes it in \cite[(4.1.7)]{egaIII}, so I believe it's my right to do so as well. Nevertheless, Lemma~\reff{lem:f-1} is \emph{perhaps worthwhile to know} (if you get what I mean), so we will include it now.
\end{rem}
\begin{defi}\lbl{def:f-1}
	Let $f\colon X\morphism Y$ be any morphism of preschemes and $\Jj\subseteq \Oo_Y$ a sheaf of ideals on $Y$. Then define $f^{-1}\Jj$ to be the image of $f^*\Jj\morphism\Oo_X$ (which is obtained as the composition of the pull-back of $\Jj\morphism \Oo_Y$ with the isomorphism $f^*\Oo_Y\cong \Oo_X$).
\end{defi}
\begin{lem}\lbl{lem:f-1}
	Let $f\colon X\morphism Y$ be any morphism of preschemes and $\Jj\subseteq \Oo_Y$ quasi-coherent.
	\begin{alphanumerate}
		\item $f^{-1}\Jj\subseteq \Oo_X$ is quasi-coherent.
		\item Let $Y_0$ and $X_0$ be the closed subpreschemes of $Y$ and $X$ defined by $\Jj$, $f^{-1}\Jj$ respectively. Then $X_0\cong X\times_YY_0$.
		\item For all $n\geq 0$ we have $f^{-1}(\Jj^n)\cong (f^{-1}\Jj)^n$.
	\end{alphanumerate}
\end{lem}
\begin{proof}[Sketch of a proof]
	The question is easily seen to be local on both $X$ and $Y$. So let's consider the affine situation where $Y=\Spec A$, $X=\Spec B$, and $\Jj=\snake{J}$ for some ideal $J\subseteq A$. Let $\phi\colon A\morphism B$ be the morphism of rings corresponding to $f$. Then $f^{-1}\Jj=\snake{I}$ where $I$ is the image of $B\otimes_AJ\morphism B$ sending $b\otimes j\mapsto b\cdot \phi(j)$. All three assertions are then easily checked.
\end{proof}
\begin{rem}\lbl{rem:fibres}
	Recall that for a morphism $f\colon X\morphism Y$ of preschemes and a point $y\in Y$ the \defemph{fibre} $f^{-1}\{y\}$ of $f$ at $y$ is defined as the prescheme $f^{-1}\{y\}=X\times_Y\Spec \KK(y)$. This makes sense, since $f^{-1}\{y\}$ is indeed -- topologically -- the preimage of $y$, as proved in \cite[Corollary~1.3.3]{alggeo1}. Moreover, $\Spec \Oo_{Y,y}/\mm_{Y,y}^n\morphism Y$ is immersive for all $n\geq 1$ and has image $\{y\}$ as well. So \cite[Corollary~1.3.3]{alggeo1} is applicable again and shows that $X_n=X\times_Y\Spec (\Oo_{Y,y}/\mm_{Y,y}^n)$ has $f^{-1}\{y\}$ as underlying topological space too (but, of course, the prescheme structure differs in general). We may thus think of $X_n$ as the $n\ordinalth$ infinitesimal thickening of $f^{-1}\{y\}$. 
\end{rem}
Using this, Theorem~\reff{thm:FormalFunctions} can be restated as follows.
\begin{varthm}{thm}\lbl{thm:FormalFunctions2}
	Let $f\colon X\morphism Y$ be a proper morphism between locally noetherian\footnote{Franke only assumes $Y$ to be locally noetherian, but $f$ being of (locally) finite type implies that $X$ is locally noetherian as well by Hilbert's Basissatz. This happens multiple times throughout the text.} preschemes. Let $\Ff$ be a coherent $\Oo_X$-module. For every $y\in Y$ let $X_n=X\times_Y\Spec (\Oo_{Y,y}/\mm_{Y,y}^n)$ be the $n\ordinalth$ infinitesimal thickening of $f^{-1}\{y\}$. Then there is an isomorphism
	\begin{align*}
		(R^pf_*\Ff)_y^\complete\isomorphism\limit[n\in\IN] H^p(X_n,\Ff|_{X_n})\;,
	\end{align*}
	where $\roof{\phantom{x}}$ denotes the $\mm_{Y,y}$-adic completion.
\end{varthm}
\begin{proof}
	We may assume that $Y=\Spec A$ is affine, and that $A$ is a noetherian ring. Indeed, replacing $Y$ by an affine neighbourhood $U\cong \Spec A$ and $X$ by $f^{-1}(U)$ doesn't change $(R^pf_*\Ff)_y$ (because the construction of $R^pf_*\Ff$ is base-local) and also $X_n$ is preserved since $f^{-1}\{y\}$ is already contained in $f^{-1}(U)$ (by \cite[postnote]{alggeo2}).
	
	In this case, $R^pf_*\Ff=H^p(X,\Ff)^\qcmod$ by \cite[Proposition~1.5.1\itememph{d}]{alggeo2}. Let $\pp\in\Spec A$ be the prime ideal associated to $y$. Then $R^pf_*\Ff\cong H^p(X,\Ff)_\pp$ and $\Oo_{Y,y}\cong A_\pp$ is flat over $A$. Let $\mm=\pp A_\pp\cong \mm_{Y,y}$ be its maximal ideal. We denote $\pi\colon \Spec A_\pp\morphism \Spec A$. Applying \cite[Fact~4.1.1]{alggeo2} to $\pi$ gives 
	\begin{align*}
		H^p(X\times_Y\Spec A_\pp,\pi^*\Ff)\cong H^p(X,\Ff)_\pp\cong (R^pf_*\Ff)_y\;.
	\end{align*}
	Also
	\begin{align*}
		\left(X\times_Y\Spec A_\pp\right)\times_{\Spec A_\pp}\Spec(A_\pp/\mm^n)&\cong X\times_Y\left(\Spec A_\pp\times_{\Spec_{A_\pp}}\Spec (A_\pp/\mm^n)\right)\\
		&\cong X\times_Y\Spec(A_\pp/\mm^n)\\
		&\cong X_n
	\end{align*}
	by a bit abstract nonsense. Now Theorem~\reff{thm:FormalFunctions} may be applied to $X\times_Y\Spec A_\pp\morphism \Spec A_\pp$ (the base change of $f$) and the assertion follows.
\end{proof}
\section{Application to Zariski's main theorem}
\subsection{A lot of (not necessarily main) theorems by Zariski}
Out there in the real world, there are multiple \emph{main theorems} of Zariski around, and usually they're only loosely related. Professor Franke recommends Mumford's \emph{The red book of varieties and schemes} for a discussion of various such version.
\begin{cor}\lbl{cor:Rpvanishing}
	Let $f\colon X\morphism Y$ be any proper morphism between locally noetherian preschemes and let $d=\sup_{y\in Y}\dim \left(f^{-1}\{y\}\right)$. If $\Ff$ is a coherent $\Oo_X$-module and $p>d$, then $R^pf_*\Ff=0$.
\end{cor}
\begin{proof}
	Since $R^pf_*\Ff$ is coherent (this is \cite[Theorem~5]{alggeo2}), $(R^pf_*\Ff)_y$ is a finitely generated $\Oo_{Y,y}$-module, hence it vanishes iff it $\mm_{Y,y}$-adic completion vanishes by Fact~\reff{fact:completion101}\itememph{b}. But $X_n=X\times_Y\Spec (\Oo_{Y,y}/\mm_{Y,y}^n)$ has underlying space $f^{-1}\{y\}$ (as explained in Remark~\reff{rem:fibres}), hence $H^p(X_n,\Ff|_{X_n})=0$ when $p>d$ by Grothendieck's theorem on cohomological dimension (cf.\ \cite[Proposition~1.4.1]{alggeo2}). The assertion now follows from Theorem~\reff{thm:FormalFunctions2}.
\end{proof}
\begin{defi}
	A morphism $f\colon X\morphism Y$ of finite type is called \defemph{quasi-finite at $\boldsymbol{x\in X}$} if $x$ is discrete in its fibre, i.e., if $\{x\}$ is an open and closed subset of $f^{-1}\{y\}$ where $y=f(x)$. We call $f$ \defemph{quasi-finite} if it is quasi-finite at every $x\in X$. 
\end{defi}
The following fact wasn't mentioned in the lecture, but it's \emph{definitely} (in particular, not only \emph{perhaps}) \emph{worthwhile to know}!
\begin{fact}\lbl{fact:openDiscrete}
	Let $f\colon X\morphism Y$ be a morphism of finite type. Let $x\in X$ be open in its fibre $f^{-1}\{y\}$, where $y=f(x)$. Then $f$ is already quasi-finite at $x$. 
\end{fact}
\begin{proof}
	Choose an affine open neighbourhood $y\in U\cong \Spec A$. Then $f^{-1}\{y\}$ is contained in $f^{-1}(U)$, so we may w.l.o.g.\ assume that $Y=\Spec A$ is affine. Put $k=\KK(y)$. Since $X$ may be covered by affine open subsets $\Spec R$, where $R$ is of finite type over $A$, we may cover the fibre product $f^{-1}\{y\}=X\times_Y\Spec k$ by affine open subsets $\Spec (R\otimes_A k)$, in which $R\otimes_A k$ is a $k$-algebra of finite type, hence a Jacobson ring. This proves that $f^{-1}\{y\}$ is a Jacobson prescheme as in \cite[Definition~2.4.2\itememph{c}]{alggeo1}. But then $x$ is a closed point of the open subset $\{x\}\subseteq f^{-1}\{y\}$, hence also a closed point of $f^{-1}\{y\}$ by \cite[Fact~2.4.1\itememph{c}]{alggeo1}.
\end{proof}
\begin{fact}
	\begin{alphanumerate}
		\item \lbl{fact:annoyingQF}Any finite morphism is quasi-finite.
		\item If $k$ is a field, a morphism $f\colon X\morphism\Spec k$ of finite type is quasi-finite iff it is finite.
		\item Let $f\colon X\morphism Y$ and $g\colon Y\morphism Z$ be morphisms of finite type such that $g$ is quasi-finite at $y=f(x)$ for some $x\in X$. Then $gf$ is quasi-finite at $x$ iff $f$ is quasi-finite at $x$.
	\end{alphanumerate}
\end{fact}
\begin{proof}
	Maybe that's my bad, but the proof of this is actually annoyingly laborious. We begin with part~\itememph{a}. Let $f\colon X\morphism Y$ be a finite morphism, $x\in X$ and $y=f(x)$. Then the morphism
	\begin{align*}
		f^{-1}\{y\}=X\times_Y\Spec \KK(y)\morphism \Spec \KK(y)
	\end{align*}
	is finite again, as a base changes of finite morphisms are finite again (cf.\ \cite[Corollary~1.5.1]{alggeo1}). Letting $k=\KK(y)$ this puts us in the situation from \itememph{b}, so it's sufficient to prove \itememph{b}.
	
	In the case of \itememph{b} we have $f^{-1}\{y\}=X$, so what we need to show is that $X$ carries the discrete topology if $f$ is finite. We know that $X\cong \Spec R$ where $R$ is some finite-dimensional $k$-algebra (using finiteness of $f$). For $x\in X$ let $\pp$ be the corresponding prime ideal of $R$. Then $R/\pp$ is a domain and a finite-dimensional $k$-vector space, hence a finite field extension of $k$ by Hilbert's Nullstellensatz. This means that $\pp$ is a maximal ideal of $R$. Consequently, all points of $X$ are closed, so it suffices to show that $X$ has finitely many points. Let $\{x_1,\ldots,x_n\}$ be any finite subset of $X$ and $\{\mm_1,\ldots,\mm_n\}$ the corresponding maximal ideals of $R$. For every $i$, we may choose an element $\alpha_i\in\mm_i$ which is not contained in any $\mm_j$ for $j\neq i$ (e.g.\ by the prime avoidance lemma, cf.\ \cite[Lemma~2.5.1]{alg1}). Put $\beta_i=\prod_{j\neq i}\alpha_j$ (so that $\beta_i\in\mm_j$ for all $j\neq i$ but $\beta_i\notin \mm_i$). We claim that $\beta_1,\ldots,\beta_n$ are $k$-linearly independent. Indeed, if $\lambda_1\beta_1+\ldots+\lambda_n\beta_n=0$ for some coefficients $\lambda_1,\ldots,\lambda_n\in k$, then reducing modulo $\mm_i$ gives $\lambda_i\beta_i=0$ in $R/\mm_i=\KK(\mm_i)$. But $\beta_i\neq 0$ in $\KK(\mm_i)$, so $\lambda_i=0$ for all $i=1,\ldots,n$. This proves $\dim_kR\geq n$. But $R$ is finite-dimensional over $k$, hence $X$ must have finitely many points, as claimed.
	
	Conversely, assume that $f\colon X\morphism\Spec k$ is quasi-finite. Then $X$ is discrete, so it must have finitely many points. Indeed, $f$ being of finite type implies it is quasi-compact (by definition), so $X$ is quasi-compact because $\Spec k$ is, and any discrete quasi-compact space is finite. Let $X=\{x_1,\ldots,x_n\}$. Every point $x_i\in X$ together with the restriction $\Oo_X|_{\{x_i\}}$ of the structure sheaf is a prescheme again, hence affine (because $x_i\in\{x_i\}$ must have an affine neighbourhood). Let $\{x_i\}\cong \Spec R_i$. Then
	\begin{align*}
		X\cong\coprod_{i=1}^n\Spec R_i\cong\Spec\bigg(\bigoplus_{i=1}^nR_i\bigg)
	\end{align*}
	is affine. This shows that $f$ is affine, but finiteness is yet to prove. Clearly, it suffices that each $R_i$ is a finite-dimensional $k$-vector space. Note that $R_i$ has precisely one prime ideal $\mm_i$ (corresponding to $x_i$), which is then automatically maximal. Since $f$ is of finite type, $R_i$ has finite type over $k$. In particular $R_i$ is noetherian and we may choose generators $r_1,\ldots,r_m$ of $\mm_s$. Since $\mm_i$ is the only prime ideal of $R_s$, we have $\mm_i=\nil R_i$. Consquently, there is an $N\in\IN$ such that $r_\ell^N=0$ for all $\ell$. Moreover, $R_i/\mm_i$ is a field extension of finite type over $k$, hence a finite field extension by Hilbert's Nullstellensatz. Let $\beta_1,\ldots,\beta_d\in R_i$ be elements whose images modulo $\mm_i$ form a $k$-basis of $R_i/\mm_i$. Then it is straightforward to check that $R$ is generated as a $k$-vector space by the elements
	\begin{align*}
	\beta_j\cdot r_1^{e_1}r_2^{e_2}\cdots r_n^{e_n}\quad\text{where }0\leq e_\ell<N\text{ for all }\ell\;.
	\end{align*}
	This shows $\dim_kR<\infty$, hence $f$ is finite.
	
	Part \itememph{c}. Since $g$ is quasi-finite at $y$, the subset $\{y\}\subseteq g^{-1}\{g(y)\}$ is open and closed, hence $f^{-1}\{y\}\subseteq (gf)^{-1}\{g(y)\}$ is open and closed. This means that $\{x\}$ is open and closed in the fibre $(gf)^{-1}\{g(y)\}$  iff it is open and closed in $f^{-1}\{y\}$ and we win.
\end{proof}
\begin{thm}[Grothendieck's version of Zariski's main theorem]
	\begin{alphanumerate}
		\item \lbl{thm:ZariskiMain}Let $f\colon X\morphism Y$ be a quasi-finite proper morphism between locally noetherian preschemes. Then $f$ is finite.
		\item Let $f\colon X\morphism Y$ be a quasi-finite and separated morphism between noetherian preschemes. Then there exists a factorization $X\monomorphism[j]\ov{X}\morphism[g]Y$ of $f$ where $j$ is an open immersion and $g$ is finite.
		\item If $f\colon X\morphism Y$ is any morphism of finite type between locally noetherian preschemes, then 
		\begin{align*}
			U=\left\{x\in X\st f\text{ is quasi-finite at }x\right\}
		\end{align*}
		is open in $X$, and the restriction $f|_U$ is quasi-finite (by definition).
	\end{alphanumerate}
\end{thm}
\begin{proof}
	Part \itememph{a}. We may assume that $Y=\Spec A$ is affine (indeed, all involved properties are base-local). Let $\Jj\subseteq \Oo_X$ be a sheaf of ideals, then $\Jj$ is coherent as $X$ is locally noetherian. Since $f$ is quasi-finite, all fibres carry the discrete topology. In particular, they are zero-dimensional and Corollary~\reff{cor:Rpvanishing} shows that $R^1f_*\Jj=0$. Then also $0=R^1f_*\Jj(Y)=H^1(X,\Jj)$ (using \cite[Proposition~1.5.1\itememph{d}]{alggeo2}), hence $X$ is affine by Serre's affinity criterion. This shows that $f$ is affine. Moreover, $f_*\Oo_X$ is a coherent $\Oo_Y$-module by \cite[Theorem~5]{alggeo2}, hence $f$ is finite.
	
	Part~\itememph{b} is hard, see the discussion on page~\pageref{proof:discussionOfThm3}. We only prove a special case there, which, however, is sufficient to prove \itememph{c}. But before we can do this, we need to prove some more theorems of Zariski.
\end{proof}
\begin{thm}[Zariski's connectedness theorem]\lbl{thm:ZariskiConnectedness}
	Let $f\colon X\morphism Y$ be a proper morphism between locally noetherian schemes, whose algebraic component $f^*\colon \Oo_Y\morphism f_*\Oo_X$ is an isomorphism.
	\begin{alphanumerate}
		\item The fibres $f^{-1}\{y\}$ are connected for all $y\in Y$.
		\item The set
		\begin{align*}
			U=\left\{x\in X\st \{x\}=f^{-1}\{f(x)\}\right\}=\left\{x\in X\st f\text{ is quasi-finite at }x\right\}
		\end{align*}
		is open in $X$, and the restriction $f|_U$ is quasi-finite (by definition).
	\end{alphanumerate}
\end{thm}
\begin{proof}
	Part \itememph{a}. Assume $f^{-1}\{y\}$ is not connected, say, $f^{-1}\{y\}=U_1\cup U_2$ for disjoint non-empty open subsets $U_1,U_2\subseteq f^{-1}\{y\}$. Since all infinitesimal thickenings $X_n=X\times_Y\Spec(\Oo_{Y,y}/\mm_{Y,y}^n)$ have underlying topological space $f^{-1}\{y\}$, there is a unique $\epsilon_n\in\Oo_{X_n}(X_n)=H^0(X_n,\Oo_{X_n})$ such that $\epsilon_n|_{U_1}=0$ and $\epsilon_n|_{U_2}=1$. The sequence $(\epsilon_n)_{n\in\IN}$ clearly defines an element $\epsilon$ of
	\begin{align*}
		\limit[n\in\IN]H^0(X_n,\Oo_{X_n})\cong (f_*\Oo_X)_y^\complete\cong \roof{\Oo}_{Y,y}\;.
	\end{align*}
	The left isomorphism here is due to Theorem~\reff{thm:FormalFunctions2} and the fact that $\Oo_{X_n}=\Oo_X|_{X_n}$, and the right one holds by assumption. Hence $\roof{\Oo}_{Y,y}$ is a local ring (by Corollary~\reff{completionLocal}) with an idempotent $\epsilon\neq 0,1$. Then $1-\epsilon\neq 0,1$ is another non-trivial idempotent. Both $\epsilon$ and $1-\epsilon$ can't be units in $\roof{\Oo}_{Y,y}$, otherwise $\epsilon^2=\epsilon$ implies $\epsilon=1$ (and similar for $1-\epsilon$). But then they are elements of the maximal ideal $\mm$, so $\epsilon+(1-\epsilon)=1$ is an element of $\mm$ as well, contradiction!
	
	Part \itememph{b}. By \itememph{a}, any point $x\in X$ is open and closed in its fibre iff $f^{-1}\{f(x)\}=\{x\}$. Therefore the two definitions of $U$ indeed coincide. 
	
	We must show that $U$ is open. This is a local question with respect to $Y$, hence we may assume that $Y=\Spec A$ is affine. Let $x\in U$ and $V\subseteq X$ an affine open neighbourhood of $x$. Put $Z=X\setminus V$. Then $Z\subseteq X$ is closed and disjoint from $f^{-1}\{f(x)\}=\{x\}$. As $f$ is proper, $Z'=f(Z)\subseteq Y$ is closed, and $y=f(x)\notin Z'$. There's an $\alpha\in A$ such that $y\notin V(\alpha)$ and $V(\alpha)\supseteq Z'$. Let $Y_1=Y\setminus V(\alpha)$. Note that $Y_1\cong \Spec A_\alpha$ is affine and $x\in X_1=f^{-1}(Y_1)\subseteq V$. Then $X_1=X\setminus V(f^*\alpha)=V\setminus V(f^*\alpha)$ is affine as well, so the restriction $f|_{X_1}\colon X_1\morphism Y_1$ of $f$ is affine and proper. But every affine proper morphism is finite (because $f|_{X_1,*}\Oo_{X_1}$ is a coherent $\Oo_{Y_1}$-module by \cite[Theorem~5]{alggeo2}), so $f|_{X_1}$ is, in particular, quasi-finite by Fact~\reff{fact:annoyingQF} and $U\cap X_1=X_1$. This proves that $U$ is open.
\end{proof}
\begin{rem}
	On first glance, the argument from Theorem~\reff{thm:ZariskiConnectedness}\itememph{b} might look like it proves that every proper morphism is affine, but what it actually shows is the following: If $f\colon X\morphism Y$ is a proper morphism such that for each $x\in X$ the fibre $f^{-1}\{f(x)\}$ is contained in some affine subset $V\subseteq X$, then $f$ is already affine (and hence finite). 	
\end{rem}
\begin{rem}
	Recall that a prescheme $X$ is called \defemph{normal} if it is integral and all local rings $\Oo_{X,x}$ (which are domains if $X$ is integral) are normal (cf.\ \cite[Definition~2.4.5]{alggeo1}). This is the case iff $\Oo_X(U)$ is a normal domain for all affine $U\subseteq X$, cf.\ the discussion in \cite[Remark~2.5.1]{alggeo1}.
\end{rem}
\begin{cor}[Zariski's birationality theorem]\lbl{cor:ZariskiBirationality}
	Let $f\colon X\morphism Y$ be a proper morphism between locally noetherian preschemes, where $Y$ is normal. Suppose that $f$ is \defemph{birational} in the sense that there is a dense open subset $U\subseteq Y$ such that the restriction $f|_{f^{-1}(U)}\colon f^{-1}(U)\isomorphism U$ is an isomorphism and $f^{-1}(U)$ is dense in $X$. Then all assertions from Theorem~\reff{thm:ZariskiConnectedness} apply to $f$. In particular, $f$ has connected fibres.
\end{cor}
\begin{proof}
	First note that $U$ is irreducible as an open subset of the irreducible space $Y$ (irreducibility of $Y$ is implied by $Y$ being normal). Hence $X$ is irreducible because it has the dense irreducible subset $f^{-1}(U)\cong U$. Let $\Spec A\cong V\subseteq Y$ be an affine open subset, where $A$ is a domain. Then $f^{-1}(V)$ is open in $X$, hence dense in $X$ and thus irreducible. Since $U$ is dense in $Y$, the intersection $U\cap V$ is non-empty, hence $f^{-1}(U\cap V)\subseteq f^{-1}(V)$ is a non-empty open subset and thereby dense again. This shows that we can actually reduce to the case $Y=\Spec A$ (all the other involved properties are clearly base-local). Moreover, we may assume that $X$ is integral. Indeed, the assertions from Theorem~\reff{thm:ZariskiConnectedness} are purely topological, so we may replace $X$ by its reduction $X^\red=V(\nil(\Oo_X))$ to obtain an $X$ which is irreducible and reduced (hence integral) and has the same underlying topological space as the original one.
	\begin{claim}\lbl{claim:quotientField}
		The ring $B=\Oo_X(X)$ is a domain in the above situation, and $A$ and $B$ have the same field of quotients $K$. Moreover, we have $A\subseteq B$ as subrings of $K$.
	\end{claim}
	Believing this for the moment, the proof can be finished as follows. Since $B$ is finitely generated as an $A$-module (because $f_*\Oo_X=\snake{B}$ is coherent by \cite[Theorem~5]{alggeo2}), it is integral over $A$. But $A$ is integrally closed in $K$, hence $A\subseteq B$ implies $A=B$. We conclude $f_*\Oo_X\cong \Oo_Y$, as needed.
	
	Unfortunately, the proof of Claim~\reff{claim:quotientField} wasn't discussed in the lecture, but I think it should have been. Since $X$ and $Y$ are irreducible, they have unique generic points $\eta_X$ and $\eta_Y$. As $\eta_Y$ is dense in $Y$, we have $\eta_Y\in U$ and similarly $\eta_X\in f^{-1}(U)$. Hence $f(\eta_X)=\eta_Y$ and the induced morphism $\Oo_{Y,\eta_Y}\isomorphism\Oo_{X,\eta_X}$ is an isomorphism by the birationality assumption. Moreover, $\eta_Y$ corresponds to $0\in\Spec A$, hence $\Oo_{Y,\eta_Y}\cong K$ is the quotient field of $A$. So we should prove that $\Oo_{X,\eta_X}$ is the quotient field of $B=\Oo_X(X)$ as well.
	
	It's clear that $B$ is a domain because $X$ is integral. Since $U\subseteq \Spec A$ is open, we find an affine open subset $V=\Spec A\setminus V(\alpha)\subseteq U$. Then $f^{-1}(V)=X\setminus V(f^*\alpha)=f^{-1}(U)\setminus V(f^*\alpha)\cong V$ is affine again by birationality of $f$. We know that $X$ is quasi-compact and separated since so are $f$ and $\Spec A$. In particular, \cite[Proposition~1.5.1\itememph{c}]{alggeo1} is applicable to $\Oo_X$ and gives $\Oo_X(f^{-1}(V))\cong\Oo_X(X)_{f^*\alpha}$, so these two rings have the same quotient field. But $\Oo_X(f^{-1}(V))\cong \Oo_Y(V)\cong A_\alpha$ has quotient field $K$, so we win.
	
	The fact that $A\subseteq B$ as subrings of $K$ follows from the commutative diagram
	\begin{diagram*}
		\object{2,0}{$\Oo_{X,\eta_X}$}[a];
		\object{0,0}{$\Oo_{Y,\eta_Y}$}[b];
		\object{2,1.5}{$B$}[c];
		\object{0,1.5}{$A$}[d];
		\scriptsize
		\arrow dc;
		\arrow[right hook->]ca;
		\arrow[right hook->]db;
		\isoarrow ba;
	\end{diagram*}
	in which every arrow except the top one is injective, hence $A\morphism B$ is injective as well.
\end{proof}
\begin{fact}\lbl{fact:steinFactorization}
	Every proper morphism $f\colon X\morphism Y$ between locally noetherian preschemes can be factorized via the \defemph{Stein factorization} as
	\begin{align*}
		f\colon X\morphism[\snake{f}]\SPEC_Y(f_*\Oo_X)\morphism[g]Y\;.
	\end{align*}
	In this composition, $g$ is finite and the assumptions of Theorem~\reff{thm:ZariskiConnectedness} hold for $\snake{f}$.
\end{fact}
\begin{proof}[Sketch of a proof]
	It's pretty obvious that this factorization exists (to construct $\snake{f}$, use the adjunction from \cite[Proposition~1.6.2\itememph{b}]{alggeo2}). To show that $\snake{f}$ and $g$ have the required properties, we look at things locally and assume that $Y=\Spec A$ is affine (and $A$ is noetherian). Then the facorization looks like
	\begin{align*}
		f\colon X\morphism[\snake{f}]\Spec \Oo_X(X)\morphism[g]\Spec A\;,
	\end{align*}
	so $g$ is affine. Moreover, $\Oo_X(X)$ is a finitely generated $A$-module because $f_*\Oo_X$ is a coherent $\Oo_Y$-module (by \cite[Theorem~5]{alggeo2}, as usual), so $g$ is actually finite. Also proving that $\snake{f}_*\Oo_X=\Oo_{\Spec \Oo_X(X)}$ is straightforward, so it remains to show that $\snake{f}$ is proper. But $g$ is finite, hence separated, and $g\snake{f}=f$ is proper, so $\snake{f}$ is proper as well by \cite[Proposition~2.4.1]{alggeo2}.
\end{proof}
\subsection[Discussion of Theorem~\getrefnumber{thm:ZariskiMain}\itememph{b} and \itememph{c}]{Discussion of Theorem~\reff{thm:ZariskiMain}\itememph{b} and \itememph{c}}%}
\label{proof:discussionOfThm3}
\begin{proof}[Proof of Theorem~\reff{thm:ZariskiMain}\itememph{c}]
	Let's assume that $f\colon X\morphism Y$ factors over
	\begin{align}\lbl{eq:properCompactification}
		f\colon X\monomorphism[j]\ov{X}\morphism[\smash{\ov{f}}]Y\;,
	\end{align}
	where $j$ is an open immersion and $\ov{f}$ is proper. Then
	\begin{align}\lbl{eq:fandfbar}
		\left\{x\in X\st f\text{ is quasi-finite at }x\right\}=X\cap \left\{x\in \ov{X}\st \ov{f}\text{ is quasi-finite at }x\right\}\;.
	\end{align}
	Indeed, a point $x\in X$ is open in $\ov{f}^{-1}\{y\}$ (where $y=f(x)$) iff it is open in the open subset $f^{-1}\{y\}=X\cap \ov{f}^{-1}\{y\}\subseteq \ov{f}^{-1}\{y\}$. In view of Fact~\reff{fact:openDiscrete} this shows \eqreff{eq:fandfbar}. We thus have reduced \itememph{c} (under the assumption that $\ov{f}$ exists) to the case of proper morphisms. 
	
	If $f\colon X\morphism Y$ is proper, then consider its Stein factorization. Since $g$ is finite, it's quasi-finite as well by Fact~\reff{fact:annoyingQF}\itememph{a}. So Fact~\reff{fact:annoyingQF}\itememph{c} shows that
	\begin{align*}
		\left\{x\in X\st f\text{ is quasi-finite at }x\right\}=\left\{x\in X\st \snake{f}\text{ is quasi-finite at }x\right\}\;.
	\end{align*}
	But the right-hand side is open in $X$ by Fact~\reff{fact:steinFactorization} and Theorem~\reff{thm:ZariskiConnectedness}\itememph{b} and we're happy!
	
	Note that such an $\ov{f}$ always exists when $X$ and $Y$ are affine. Indeed, if $X$ has finite type over $Y$ and both are affine, we get a closed embedding $X\monomorphism\IA_Y^n$ for some $n\in \IN$. Together with the open embedding $\IA_Y^n\monomorphism\IP_Y^n$ this makes $X$ a closed subprescheme of an open subprescheme of $\IP_Y^n$. But then $X$ is also an open subprescheme of some closed subprescheme $\ov{X}\subseteq \IP_Y^n$. This gives a factorization
	\begin{align*}
		f\colon X\monomorphism[j]\ov{X}\morphism[\smash{\ov{f}}]Y
	\end{align*}
	in which $\ov{f}\colon \ov{X}\monomorphism \IP_Y^n\morphism Y$ is (strongly) projective, hence proper by \cite[Proposition~2.4.2]{alggeo2}. But \itememph{c} is completely local on both $X$ and $Y$ (thanks to Fact~\reff{fact:openDiscrete}), so by checking the affine case we have actually covered all of \itememph{c}. 
\end{proof}
	The rest of this subsection will be used to sketch a proof of Theorem~\reff{thm:ZariskiMain}\itememph{b}. We begin with the observation that Theorem~\reff{thm:ZariskiMain}\itememph{b} becomes much easier when things are proper.
	\begin{lem}\lbl{lem:ZMTLemma1}
		Let $f\colon X\morphism Y$ be a quasi-finite separated morphism between noetherian schemes, and assume that $f$ factors as in \eqreff{eq:properCompactification}. Then Theorem~\reff{thm:ZariskiMain}\itememph{b} holds for $f$.
	\end{lem}
\begin{rem}
	 It can be shown that such a factorization \eqreff{eq:properCompactification} always exists for morphisms of finite type between noetherian schemes, for which Professor Franke refers to notes of \emph{Brian Conrad} or \emph{Paul Vojta}, although he isn't sure whether using their results to prove Theorem~\reff{thm:ZariskiMain}\itememph{b} doesn't involve any circular reasoning.
\end{rem}
\begin{proof}[Proof of Lemma~\reff{lem:ZMTLemma1}]
	Let $\ov{f}=g\circ \snake{f}$ be the Stein factorization of $\ov{f}$. Put $\ov{Y}=\SPEC_Y\ov{f}_*\Oo_{\ov{X}}$ for convenience. Since $g\colon \ov{Y}\morphism Y$ is finite by Fact~\reff{fact:steinFactorization} (for which we need properness of $\ov{f}$), we're done if we show that the composition $\snake{f}|_X\colon X\monomorphism \ov{X}\morphism \ov{Y}$ is an open embedding. Since $f=g\snake{f}|_X$ and $g$ are quasi-finite ($g$ is even finite), Fact~\reff{fact:annoyingQF}\itememph{c} shows that $\snake{f}|_X$ is quasi-finite as well.
	
	In particular $\snake{f}|_X$ is injective, and for all $x\in X$ we have $\snake{f}^{-1}\{\snake{f}(x)\}=\{x\}$, because $\Oo_{\ov{Y}}\cong \snake{f}_*\Oo_{\ov{X}}$ (by Fact~\reff{fact:steinFactorization}), so the argument from Theorem~\reff{thm:ZariskiConnectedness}\itememph{b} can be applied. If $V$ is any open neighbourhood of $x\in X$, then $\snake{f}(\ov{X}\setminus V)\subseteq \ov{Y}$ is closed because $\snake{f}$ is proper (by Fact~\reff{fact:steinFactorization}), hence closed. Moreover, $\snake{f}(\ov{X}\setminus V)$ doesn't contain $\snake{f}(x)$ as $\snake{f}^{-1}\{\snake{f}(x)\}=\{x\}$. Thus, the complement $U$ of $\snake{f}(\ov{X}\setminus V)$ in $\ov{Y}$ is open and $\snake{f}^{-1}(U)\subseteq V$ is an open neighbourhood of $x$. This shows that $\snake{f}|_X$ is actually an open map! Together with $\Oo_{\ov{Y}}\cong \snake{f}_*\Oo_{\ov{X}}$ we see that $\snake{f}|_X$ is an open embedding, as claimed.
\end{proof}
\begin{defi}\lbl{def:quasiAffine}
	A morphism $f\colon X\morphism Y$ of preschemes is called \defemph{quasi-affine}, if it satisfies the following equivalent conditions:
	\begin{alphanumerate}
		\item For all affine open $U\subseteq Y$, $f^{-1}(U)$ is a quasi-affine scheme (that is, a quasi-compact open subscheme of an affine scheme).
		\item $Y$ can be covered by affine open $U$ such that $f^{-1}(U)$ is quasi-affine.
		\item $f$ factors as $X\monomorphism[j]\ov{X}\morphism[\smash{\ov{f}}]Y$, where $j$ is an open embedding and $\ov{f}$ is affine.
		\item The canonical morphism $X\morphism\SPEC_Y(f_*\Oo_X)$ in the Stein factorization is an open immersion.
	\end{alphanumerate}
\end{defi}
For a proof of equivalence, check out \cite[\stackstag{01SJ}]{stacks-project}. Note that Franke apparently had a different proof (for the case of quasi-affine morphisms of finite type) in mind than the Stacks Project guys, but I have no idea how this was going to work (whereas the Stacks Project proof is pretty clear to me and works without restrictions on $f$). Nevertheless, the following lemma he suggested is perhaps worthwhile to know.
\begin{lem}
	Let $X$ be a noetherian prescheme, $\Mm$ a quasi-coherent $\Oo_X$-module, and $U\subseteq X$ an open subset. Let $\Nn\subseteq \Mm|_U$ be a coherent submodule. Then there is a coherent $\snake{\Nn}\subseteq \Mm$ such that $\snake{\Nn}|_U=\Nn$. In particular, $\Mm$ is the union of its coherent submodules.
\end{lem}
\begin{proof}
	We proceed by noetherian induction. Because $Y$ is noetherian, there is a $\subseteq$-maximal open subset $U$ to which $\Nn$ extends (and, with slight abuse of notation, we denote some fixed extension to $U$ by $\Nn$ as well). Assuming $U\neq X$, we will derive a contradiction. Pick $x\in X\setminus U$ and let $V\cong \Spec A$ be an affine open neighbourhood of $x$. 
	\begin{claim*}
		There is a coherent $\Qq\subseteq \Mm|_V$ such that $\Qq|_{U\cap V}\supseteq \Nn|_{U\cap V}$.
	\end{claim*}
	Indeed, $V\cong \Spec A$ is a noetherian topological space (because $Y$ is noetherian), so the open subset $U\cap V$ is quasi-compact. This means that we can cover it by finitely many affine open subsets $V_i\cong \Spec A_{f_i}$ for $i=1,\ldots,n$. For every $i$ we know that $\Nn(V_i)\subseteq \Mm(V_i)\cong \Mm(V)_{f_i}$ is a finitely generated $A_{f_i}$-module (because $\Nn$ is coherent), so we can choose finitely many $\mu_{i,j}\in \Mm(V)$, $j=1,\ldots,k_i$, whose images in $\Nn(V_i)$ form a set of generators. Let $\Qq\subseteq \Mm|_V$ be the subsheaf generated by $\left\{\mu_{i,j}\st i\leq n,\ j\leq k_i\right\}\in\Mm(V)$. Then $\Qq$ has the required property, proving the claim.
	
	Let $j\colon U\cap V\monomorphism V$ be the obvious inclusion and put $\ov{\Nn}=\Qq\cap j_*\Nn|_{U\cap V}$. This guy is coherent (because subsheaves of a coherent one are coherent again on locally noetherian preschemes) and satisfies $\ov{\Nn}|_{U\cap V}=\Nn|_{U\cap V}$. But then $\Nn$ can be extended to a coherent sheaf $\snake{\Nn}$ on $U\cup V$ via $\snake{\Nn}|_U=\Nn$ and $\Nn|_V=\Kk$. This contradicts maximality of $U$.
\end{proof}
\begin{lem}\lbl{lem:ZMTLemma2}
	If $f\colon X\morphism Y$ is a quasi-finite and quasi-affine morphism between noetherian preschemes. Then Theorem~\reff{thm:ZariskiMain}\itememph{b} holds for $f$.
\end{lem}
\begin{proof}
	Note that quasi-affine morphisms are automatically separated. We will show that $g\colon \SPEC_Y(f_*\Oo_X)\morphism Y$ is proper, which shows that $f$ factors as in \eqreff{eq:properCompactification}. Then Theorem~\reff{thm:ZariskiMain}\itememph{b} holds by Lemma~\reff{lem:ZMTLemma1}. To show properness of $g$, we define
	\begin{align*}
		\Rr=\bigoplus_{n\geq 0}f_*\Oo_X\;,
	\end{align*}
	viewed as a graded quasi-coherent $\Oo_Y$-algebra. Note that $\SPEC_Y(f_*\Oo_X)\cong \PROJ_Y(\Rr)$ as $Y$-preschemes (indeed, locally this reduces to \cite[Example~2.6.2]{alggeo1}). Moreover, $\Rr$ is locally of finite type over $\Oo_Y$. In fact, the $0\ordinalth$ homogeneous component $\Rr_0\cong f_*\Oo_X$ is locally of finite type over $\Oo_Y$ (because $f$ is of finite type), and $\Rr$ is generated by $\Rr_0$ and the element $1\in\Rr_1(Y)$ (which acts as a shift). Therefore, $\PROJ_Y(\Rr)\morphism Y$ is proper by \cite[Proposition~2.4.2]{alggeo2}, whence we're done.
\end{proof}
So we see that to prove Theorem~\reff{thm:ZariskiMain}\itememph{b} it is sufficient to show that any quasi-finite and separated morphism between noetherian preschemes is quasi-affine. This is a base-local question again (since quasi-affinity is base-local again by Definition~\reff{def:quasiAffine}), so we may assume that $Y=\Spec A$ for some noetherian ring $A$.

Professor Franke sketches a proof following \cite[Exposé~VIII.6]{sga1}, where the technique of \emph{faithfully flat descent} is used to reduce the question to the case of complete local rings.

So we may assume $Y=\Spec A$, where $A$ is a noetherian complete local ring (with respect to its maximal ideal $\mm$) and that Zariski's main theorem is true for morphisms $f'\colon X'\morphism Y'$, where in addition to the other assumptions we have $\dim Y'<\dim Y$. Let $s$ be the unique closed point of $A$ (given by $\mm$).
\begin{lem}\lbl{lem:connectedComponents}
	If $B$ is any finite $A$-algebra, then $\Spec B$ has a decomposition $\Spec B=\bigcup_{i=1}^nU_i$, where the $U_i$ are disjoint open subsets such that the only closed point $s$ of $\Spec A$ has precisely one preimage in each $U_i$.
\end{lem}
\begin{proof}
	By Corollary~\reff{cor:finiteAlgebras}, $B$ has finitely many maximal ideals $\qq_1,\ldots,\qq_n$ and these are precisely the prime ideals over $\mm$. For every $\qq_i$ we will construct an idempotent $e_i\in B$ such that $e_i\in \qq_i$ but $e_i\notin\qq_j$ for $j\neq i$. Then $\qq_i\in V(e_i)$ but $\qq_j\neq V(e_i)$ for $j\neq i$, and $V(e_i)$ is an open and closed subset of $B$ (because $e_i$ is an idempotent), so we see that the $\qq_i$ are contained in distinct connected components of $\Spec B$. But $B$ is noetherian, so $\Spec B$ has finitely many connected components (as pointed out in \cite[Lemma~2.4.2]{alggeo1}). Then every connected component is open and we can construct the required $U_i$ as unions of suitable connected components of $\Spec B$.
	
	Let $\ov{\qq}_i=\qq_i/\mm$ be the prime ideals of $B/\mm B$ (which are automatically maximal). Then the intersection $\bigcap_{j=1}^n\qq_j$ is the nilradical $\nil(B/\mm B)$, so by the Chinese remainder theorem we find an element $\ov{e}_i\in (B/\mm B)/\nil(B/\mm B)$ such that $\ov{e}_i\equiv 0\bmod \ov{\qq}_i$ and $\ov{e}_j\equiv 1\bmod\ov{\qq}_j$ for $j\neq i$. Note that $\ov{e}_i^2=\ov{e}_i$. Since $B/\mm B$ is noetherian, there is an $N\in\IN$ such that $\nil(B/\mm B)^N=0$. Hence $B/\mm B$ is $\nil(B/\mm B)$-adically complete, which means we can lift $\ov{e}_i$ to an idempotent $\snake{e}_i\in B/\mm B$ by Hensel's lemma (cf.\ Proposition~\reff{prop:HenselsLemma}). Now $B$ is $\mm B$-adically complete by Proposition~\reff{prop:modulesComplete}, so using Hensel's lemma once again we can lift $\snake{e}_i$ to an idempotent $e_i\in B$ with the required properties. We're done.
\end{proof}

Let $f\colon X\morphism Y=\Spec A$ be quasi-finite and separated, and let $x\in f^{-1}\{s\}$. Because the fibre $f^{-1}\{s\}$ is discrete, there is an affine open $U\subseteq X$ such that $U\cap f^{-1}\{s\}=\{x\}$. Zariski's main theorem applies to the affine morphism $f|_U\colon U\morphism \Spec A$, which therefore factors as
\begin{align*}
	U\monomorphism[j]\Spec B\morphism Y
\end{align*}
where $B$ is as above and $j$ is an open embedding. By Lemma~\reff{lem:connectedComponents} (and shrinking $U$ if necessary) we may assume that $B$ has only one closed point $j(x)$, i.e.\ is local itself. But then the only open subset of $\Spec B$ containing the closed point $j(x)$ is $\Spec B$ itself, hence $j$ is an isomorphism. We have thus found an affine open neighbourhood $U\cong \Spec B$ of $x$ which is finite over $Y$. Then $U\morphism X$ is proper (as $U\morphism Y$ is finite, hence proper, and $f$ is separated, so \cite[Proposition~2.4.1]{alggeo2} applies), so $U$ is also closed. Putting $U=U_1$ and $X\setminus U=X_1$, what we proved is that $X=X_1\amalg U_1$ is the disjoint union of its open subsets $X_1$ and $U_1$.

Iterating this for the remaining preimages\footnote{There are finitely many of them. Indeed, $f^{-1}\{s\}\subseteq X$ is closed (as the preimage of the closed point $s$), hence quasi-compact (because $X$ is noetherian, hence quasi-compact). But the fibre $f^{-1}\{s\}$ is supposed to be discrete, so it must be finite.} of $s$ in $X$ provides a decomposition 
\begin{align*}
	X=X'\amalg \coprod_{i=1}^nU_i
\end{align*}
into disjoint open subsets, where $\coprod_{i=1}^nU_i\cong \coprod_{i=1}^n\Spec \Oo_X(U_i)\cong \Spec\big(\bigoplus_{i=1}^n\Oo_X(U_i)\big)$ is finite over $Y$ and $f'=f|_{X'}\colon X'\morphism Y$ has image in $Y'=Y\setminus \{s\}$, hence the induction assumption applies to $f'\colon X'\morphism Y'$. This means that $f'$ can be written as $f'\colon X'\monomorphism \ov{X}'\morphism Y'$, where $X'$ is an open subprescheme of $\ov{X}'$, which in turn is finite over $Y'$. Still we aren't done yet, as we need something finite over $Y$ rather than $Y'$.

Here it is time for our final lemma.
\begin{lem}
	Let $A$ be a noetherian ring and $S\subseteq \Spec A$ an open subprescheme. If $f\colon X\morphism S$ is a finite morphism, then there is a finite $A$-algebra $B$ such that $X$ is an open subprescheme of $\Spec B$ and the diagram
	\begin{diagram*}
		\object{0,1.5}{$\Spec B$}[a];
		\object{2,1.5}{$\Spec A$}[b];
		\object{0,0}{$X$}[c];
		\object{2,0}{$S$}[d];
		\scriptsize
		\arrow ab;
		\arrow[right hook->] ca;
		\arrow[right hook->] db;
		\arrow cd[above][f];
	\end{diagram*}
	commutes.
\end{lem}
\begin{proof}
	Since $\Spec A$ is a noetherian space, its open subset $S$ is quasi-compact. We thus find a finite cover $S=\bigcup_{i=1}^n S_i$, where $S_i\cong \Spec A_{\alpha_i}$ (for some $f_i\in A$) are affine open subschemes. Putting $X_i=f^{-1}(S_i)$ we get an affine open cover $X=\bigcup_{i=1}^nX_i$, where $X_i\cong \Spec B_i$ for some finite $A_{\alpha_i}$-algebra $B_i$ by finiteness of $f$.
	
	Note that $X$ is quasi-compact because $S$ and $f$ are. Moreover, $S$ is separated as an open subprescheme of a separated prescheme, hence $X$ is separated as well, as finite morphisms are separated. If we denote the image of $\alpha_i$ under the restriction $A\morphism \Oo_S(S)$ by $\alpha_i$ as well, we see that $S_i=S\setminus V(\alpha_i)$, so $A_{\alpha_i}\cong \Oo_S(S\setminus V(\alpha_i))\cong \Oo_S(S)_{\alpha_i}$. Note that this works because $S$ is quasi-compact and separated, so \cite[Proposition~1.5.1\itememph{c}]{alggeo1} applies. By the same argument applied to $X$, we get $B_i\cong \Oo_X(X\setminus V(f^*\alpha_i))\cong \Oo_X(X)_{f^*\alpha_i}$.
	
	Note that $\Oo_X(X)$ becomes an $A$-algebra via $A\morphism \Oo_S(S)\morphism \Oo_X(X)$. For every $i$ let $x_{i,j}\in B_i$, $j=1,\ldots,k_i$ be generators of $B_i$ as an $A_{\alpha_i}$-module. Multiplying by a suitable power of $f^*\alpha_i$ we can assume that the $x_{i,j}$ are from $\Oo_X(X)$ and integral over $A$ (rather than just integral over $A_{\alpha_i}$). Now let $\Lambda$ be the image of $A$ in $\Oo_X(X)$ and put
	\begin{align*}
		B=\Lambda\left[x_{i,j}\st i\leq n,\ j\leq k_i\right]\;.
	\end{align*}
	Then $B$ is finite over $A$ and contains the $f^*\alpha_i$ (because these guys are already in $\Lambda$). Moreover, we have $B_{f^*\alpha_i}=\Oo_X(X)_{f^*\alpha_i}\cong B_i$, because $B$ already contains a set of generators of $B_i$ over $A_{\alpha_i}$. So $X=\bigcup_{i=1}^nX_i$ is a union of open subpreschemes of $\Spec B$, hence an open subprescheme itself and $\Spec B\morphism \Spec A$ clearly has the required properties.
\end{proof}

Hence Zariski's main theorem may likely be proved if we were able to reduce the case of general $Y$ to the case of noetherian complete local rings.

\chapter{Flat morphisms}
\begin{defi}
	A morphism $f\colon X\morphism Y$ of preschemes is called \defemph{flat} iff it has the following equivalent properties.
	\begin{alphanumerate}
		\item If $U\subseteq X$ and $V\subseteq Y$ are affine opens such that $f(U)\subseteq V$, then $\Oo_X(U)$ is flat as an $\Oo_Y(V)$-module.
		\item It is possible to cover $X$ by affine open subsets for which such $V$ may be found.
		\item For any $x\in X$ the stalk $\Oo_{X,x}$ is a flat $\Oo_{Y,y}$-module, where $y=f(x)$.
	\end{alphanumerate}
	We call $f$ \defemph{faithfully flat} if it is flat and surjective.
\end{defi}
\begin{proof}[Sketch of a proof of equivalence]
	The implication \itememph{a} $\Rightarrow$ \itememph{b} is trivial. If you think about it, showing \itememph{b} $\Rightarrow$ \itememph{c} $\Rightarrow$ \itememph{a} comes down to the following fact from commutative algebra: An $A$-algebra $B$ is flat if and only if $B_\qq$ is a flat $A_\pp$ algebra for all primes $\pp\in\Spec A$ and $\qq\in\Spec B$ above $\pp$.
	
	Indeed, if $B$ is flat over $A$, then so is $B_\qq$ (by exactness of localization), hence $B_\qq$ is also flat over $A_\pp$. Conversely, if $B_\qq$ is flat over $A_\pp$, then also over $A$ because $A_\pp$ is flat over $A$ (exactness of localization again). But then the localizations of $B$ at any prime are flat over $A$, hence so is $B$ itself (cf.\ \cite[Fact~1.2.6\itememph{d}]{homalg}).
\end{proof}
\section{Flat base change and cohomology}
Let $A$ be a ring, $B$ an $A$-algebra and $f\colon X\morphism \Spec A$ a quasi-compact and separated morphism. Let 
\begin{align*}
	\snake{f}\colon\snake{X}=X\times_{\Spec A}\Spec B\morphism\Spec B%.
\end{align*}
its base change along $\Spec B\morphism\Spec A$. Also let $\pi\colon \snake{X}\morphism X$ be projection to the other fibre product factor. We want to investigate the relation between the cohomology of $X$ and $\snake{X}$. For a quasi-coherent $\Oo_X$-module $\Ff$ and an affine \v Cech cover $\Uu\colon X=\bigcup_{i\in I}U_i$ of $X$, the \v Cech complex $\check{C}^\bullet(\Uu,\Ff)$ calculates $H^\bullet(X,\Ff)$. Pulling back $\Uu$ gives a \v Cech cover
\begin{align*}
	\pi^{-1}\Uu\colon \snake{X}=\bigcup_{i\in I}U_i\times_{\Spec A}\Spec B\;,
\end{align*}
of $\snake{X}$, whose components $U_i\times_{\Spec A}\Spec B$ are affine again (as fibre products of affine schemes), hence $\pi^{-1}\Uu$ may be used to compute $H^\bullet(\snake{X},\pi^*\Ff)$. Explicitly, we obtain
\begin{align}\lbl{eq:tensoredCechComplex}
	\check{C}^\bullet\left(\pi^{-1}\Uu,\pi^*\Ff\right)\cong \check{C}^\bullet(\Uu,\Ff)\otimes_AB\;.
\end{align}
This gives a morphism
\begin{align}\lbl{eq:tensoredCohomology}
	H^\bullet(X,\Ff)\otimes_AB\morphism H^\bullet(\snake{X},\pi^*\Ff)\;,
\end{align}
which clearly is an isomorphism if $B$ is flat over $A$.

\Appendix
\chapter{Appendix}
\section{Some prerequisites about completions}
We briefly recall the most important facts about completions. An excellent introduction to this subject can be found in \cite[Section~10]{atiyahMacdonald}.
\begin{defi}
	Let $A$ be a ring (commutative with $1$) and $I$ an ideal in $A$. Let $M$ be an $A$-module.
	\begin{alphanumerate}
		\item The \defemph{$I$-adic topology} on $M$ is the unique topology such that $\{I^n\}_{n\in \IN}$ is a fundamental system of neighbourhoods of $0$ and $M$ (with its additive structure) becomes a topological group in this topology.
		\item The \defemph{completion} of $M$ with respect to the $I$-adic topology is
		\begin{align*}
			\roof{M}=\limit[n\in \IN]M/I^nM\;.
		\end{align*}
		Note that $\roof{A}$ is a ring again. We call $M$ \defemph{complete} in the $I$-adic topology if the canonical morphism $M\morphism\roof{M}$ is an isomorphism.
	\end{alphanumerate}
\end{defi}
\begin{rem}
	$M$ with its $I$-adic topology is \emph{pseudo-metrizable} via $d(x,y)=\mathrm{e}^{-\sup\left\{n\st x-y\in I^n\right\}}$. It is easy to check that $\roof{M}$ is also the completion of $M$ in the analytical sense, i.e.\ the set of Cauchy sequences modulo the zero sequences.
\end{rem}
\begin{example}
	If $I^n=0$ for some $n\in \IN$, then any $A$-module is complete in the $I$-adic topology.
\end{example}
\begin{example}
	If $A=\IZ$ and $I=p\IZ$ for some prime $p$, then $\roof{A}=\IZ_p$ is the ring of $p$-adic integers.
\end{example}
\begin{prop}[Hensel's lemma]\lbl{prop:HenselsLemma}
	Suppose the ring $A$ is complete in the $I$-adic topology. Let $P\in A[T]$ be a polynomial and $a_0\in A$ such that $P(a_0)\equiv 0\bmod I$ and $P'(a_0)$ is a unit in $A/I$. Then there is a unique $a\in A$ such that $a\equiv a_0\bmod I$ and $P(a)=0$.
\end{prop}
\begin{proof}
	\emph{Step 1.} Consider the special case $I^2=0$. For $\delta\in I$ we have $P(a_0+\delta)=P(a_0)+\delta P'(a_0)$ since all terms of order $\delta^2$ or higher vanish in the binomial expansion. Now $P'(a_0)$ being a unit in $A/I$ gives a unique $\delta\in I$ such that $a=a_0+\delta$ satisfies $P(a)=0$.
	
	\emph{Step 2.} Suppose that $I^{2^n}=0$ for some $n\in \IN$. Using induction on $n$ (with the base case being precisely Step~1) we may assume that Hensel's lemma holds for $A/I^{2^{n-1}}$. In particular, there is a unique $a_{n-1}$ such that $P(a_{n-1})\equiv 0\bmod I^{2^{n-1}}$ and $a_{n-1}\equiv a_0\bmod I$. Moreover, $P'(a_{n-1})$ is invertible in $A/I^{2^{n-1}}$. Indeed, this follows from Hensel's lemma applied to $A/I^{2^{n-1}}$ (for which it holds by induction hypothesis) and the polynomial $Q=P'(a_{n-1})T-1$. The derivative $Q'(a_{n-1})$ equals $P'(a_{n-1})$ which is invertible in $A/I$ since $P'(a_{n-1})\equiv P'(a_0)\bmod I$, so Hensel's lemma is indeed applicable. Now replacing $I$ by $I^{2^{n-1}}$ and $a_0$ by $a_{n-1}$ reduces the situation to Step~1, proving the inductive step.
	
	\emph{Step 3.} Now let $I$ be arbitrary. By Step~2 there is for every $n\in\IN$ a unique $a_n\in A/I^{2^n}$ such that $P(a_n)\equiv 0\bmod I^{2^n}$ and $a_n\equiv a_0\bmod I$. Then $a_n\equiv a_{n-1}\bmod I^{2^{n-1}}$ is forced by uniqueness. Hence $a=(a_n)_{n\in \IN}$ defines an element of
	\begin{align*}
		\limit[n\in \IN]A/I^{2^n}=\limit[n\in\IN] A/I^n=\roof{A}\;,
	\end{align*}
	providing the desired element $a\in\roof{A}$.
\end{proof}
\begin{cor}\lbl{cor:HenselApplications}
	Let $A$ be complete in the $I$-adic topology.
	\begin{alphanumerate}
		\item If $a\in A$ becomes a unit in $A/I$, then already $a\in A^\times$.
		\item For every idempotent $\pi\in A/I$ there is a unique idempotent in $A$ whose image modulo $I$ is $\pi $. Therefore, $\Spec A$ and $\Spec A/I$ have the same connected components.
		\item $I$ is contained in the Jacobson radical $\rad A$.
	\end{alphanumerate}
\end{cor}
\begin{proof}
	Part \itememph{a} follows from Proposition~\ref{prop:HenselsLemma} applied to $P=aT-1$ (whose derivative $a$ is a unit in $A/I$ by assumption, so this is fine). For \itememph{b} we use the polynomial $P=T^2-T$. Again, $P'(\pi )=2\pi -1$ is a unit in $A/I$ since $(2\pi -1)^2=4\pi ^2-4\pi +1=1$ in $A/I$. To prove \itememph{c} recall the characterization
	\begin{align*}
		\rad A=\left\{x\in A\st 1-ax\in A^\times\text{ for all }a\in A\right\}\;.
	\end{align*}
	If $x\in I$, then $1-ax$ is a unit in $A/I$, hence also in $A$ by \itememph{a}.
\end{proof}
\begin{prop}
	Let $A$ be noetherian and $N\subseteq M$ finitely generated $A$-modules. Then the $I$-adic topology on $N$ coincides with the induced topology by the $I$-adic topology on $M$.
\end{prop}
\begin{proof}[Sketch of a proof]
	By the Artin--Rees lemma (cf.\ \cite[Proposition~3.4.1]{alg2}) there exists a number $c\in\IN$ such that $N\cap I^{n+c}M\subseteq I^nN$. From this, the assertion is easily deduced.
\end{proof}
\begin{fact}
	\begin{alphanumerate}
		\item \lbl{fact:completion101}The canonical morphism $\roof{M}=\limit M/I^nM\morphism M/IM$ is surjective.
		\item If $M$ is finitely generated and $I$ is contained in the Jacobson radical of $A$, then $\roof{M}=0$ implies $M=0$.
	\end{alphanumerate}
\end{fact}
\begin{proof}
	For \itememph{a}, note that the composition $M\morphism\roof{M}\morphism M/IM$ equals the projection $M\morphism M/IM$ by definition of the limit. Since the latter is surjective, so is $\roof{M}\morphism M/IM$. In particular, part \itememph{a} shows that $\roof{M}=0$ implies $M=IM$. In the situation of \itememph{b} this is equivalent to $M=0$ by Nakayama's lemma (which -- as we all know -- Professor Franke also likes to attribute to Azumaya and Krull, even though he regards Krull as a noob compared to Grothendieck).
\end{proof}
\begin{cor}\lbl{cor:completionExact}
	If $A$ is noetherian, then the functor $M\mapsto \roof{M}$ is exact on the category of finitely generated $A$-modules.
\end{cor}
\begin{proof}
	Let $0\morphism M'\morphism M\morphism M''\morphism 0$ be a short exact sequence of finitely generated $A$-modules. Then $M'+I^nM$ is the kernel of $M\epimorphism M''/I^nM''$. Using $(M'+I^nM)/I^nM\cong M'/(M'\cap I^nM)$ we get short exact sequences
	\begin{align}\lbl{eq:completion}
		0\morphism M'/(M'\cap I^nM)\morphism M/I^nM\morphism M''/I^nM''\morphism 0\tag{$*$}
	\end{align}
	for every $n\in\IN$. Since $M'/(M'\cap I^nM)$ is sandwiched between $M'/I^nM'$ and $M'/I^{n+c}M'$ for some $c\in\IN$ by the Artin--Rees lemma, it's easy to see that 
	\begin{align*}
		\limit[n\in \IN] M'/(M'\cap I^nM)=\limit[n\in\IN]M'/I^{n+c}M'=\limit[n\in\IN]M'/I^nM'=\roof{M}'\;. 
	\end{align*}
	Moreover, each $M'/(M'+I^{n+1}M)\morphism M'/(M'+I^nM)$ is clearly surjective, so Fact~\reff{fact:MittagLeffler} gives 
	\begin{align*}
		\limit[n\in\IN][^1]M'/(M'\cap I^nM)=0\;.
	\end{align*}
	Thus, taking the limit over \eqreff{eq:completion} gives a short exact sequence $0\morphism \roof{M}'\morphism \roof{M}\morphism\roof{M}''\morphism 0$ by Fact~\reff{fact:6termLimitSequence}. We are done.
\end{proof}
\begin{cor}
	Let $A$ be a Noetherian ring.
	\begin{alphanumerate}
		\item When $M$ is a finitely generated $A$-module, then $\roof{M}\cong M\otimes_A\roof{A}$.
		\item $\roof{A}$ is flat as an $A$-module.
		\item Suppose that $I$ is contained in the Jacobson radical of $A$. If $\mu\colon M\morphism N$ is a morphism of finitely generated $A$-modules such that $\roof{\mu}\colon \roof{M}\morphism\roof{N}$ is an isomorphism, then $\mu$ is an isomorphism.
	\end{alphanumerate}
\end{cor}
\begin{proof}
	Part \itememph{a}. Every finitely generated $A$-module is finitely presented as well since $A$ is noetherian. So take a representation $M\cong \coker(A^m\morphism A^n)$ for some $m,n\in\IN$. It's obvious that $(A^n)^\complete\cong \roof{A}^n\cong A^n\otimes_A\roof{A}$. Since both completion and tensor products commute with cokernels, this shows $\roof{M}\cong M\otimes_A\roof{A}$ as well.
	
	Part\itememph{b}. By Corollary~\reff{cor:completionExact} and \itememph{a}, $-\otimes_A\roof{A}$ is exact on finitely generated $A$-modules. By \cite[Proposition~1.2.2]{homalg} this is sufficient for flatness.
	
	Part \itememph{c}. Let $K=\ker\mu$ and $Q=\coker\mu$. Since completion is exact on finitely generated $A$-modules, we get an exact sequence
	\begin{align*}
		0\morphism \roof{K}\morphism\roof{M}\morphism[\roof{\mu}]\roof{N}\morphism\roof{Q}\morphism 0\;.
	\end{align*}
	But $\roof{\mu}$ is an isomorphism, so $\roof{K}=0$ and $\roof{Q}=0$. By Fact~\reff{fact:completion101} this shows $K=0$ and $Q=0$. We are done.
\end{proof}
\begin{cor}\lbl{cor:JM}
	If $J\subseteq A$ is any ideal and $M$ a finitely generated $A$-module, then $(JM)^\complete\morphism\roof{M}$ defines an isomorphism $(JM)^\complete\isomorphism J\roof{M}$.
\end{cor}
\begin{proof}
	We may view $(JM)^\complete$ as a submodule of $\roof{M}$ since completion preserves injectivity of the inclusion $JM\subseteq M$ by Corollary~\reff{cor:completionExact}. It's easy to see that $J\roof{M}$ is contained in $(JM)^\complete$. To prove the converse, take generators $j_1,\ldots,j_n$ of $J$. Then completion preserves surjectivity of $(j_1,\ldots,j_n)\colon M^n\epimorphism JM$ and we are done.
\end{proof}
\begin{cor}\lbl{completionLocal}
	If $A$ is a noetherian local ring with maximal ideal $\mm$, then $\roof{A}$ is local with maximal ideal $\mm \roof{A}$.
\end{cor}
\begin{proof}
	We proved this in \cite[Corollary~2.2.2]{homalg}.
\end{proof}
\begin{prop}\lbl{prop:completionNoetherian}
	Let $A$ be noetherian and $I\subseteq A$ any ideal, then the $I$-adic completion $\roof{A}$ is noetherian again.
\end{prop}
To prove this, we need to prove the evil twin of Hilbert's Basissatz first.
\begin{lem}\lbl{lem:(not)HilbertBasis}
	If $A$ is noetherian, then so is the power series ring $A\llbracket T\rrbracket$.
\end{lem}
\begin{proof}
	We can (and will) basically copy the proof of Hilbert's Basissatz. Let $J\subseteq A\llbracket T\rrbracket$ be any ideal and put $J_n=\left\{a_n\st \sum_{k=n}^\infty a_kT^k\in J\right\}$ for $n\geq 0$. Then $(J_n)_{n\in \IN}$ form an ascending sequence of ideals in $A$. Noetherianness of $A$ tells us that this sequence becomes eventually stationary, say, at $n=s$. So we may choose $a^{(i)}=\sum_{k\geq s}a_kT^k\in A\llbracket T\rrbracket$ for $i=1,\ldots,N$ such that $a_s^{(1)},\ldots,a_s^{(N)}$ generate $J_s$. Then $a^{(1)},\ldots,a^{(N)}$ generate $J\cap T^sA\llbracket T\rrbracket$. Indeed, given any $b=\sum_{k\geq s}b_kT^k\in J$ we can inductively choose coefficients $r_k^{(1)},\ldots,r_k^{(N)}\in A$ such that $r^{(i)}=\sum_{k\geq 0}r_k^{(i)}T^k$ satisfy $r^{(1)}a^{(1)}+\ldots+r^{(N)}a^{(N)}=b$ up to degree $T^{s+k}$. This works because $J_{k+s}=J_s$ for all $k\geq 0$ is generated by $a_s^{(1)},\ldots,a_s^{(N)}$ again.
	
	Now $A\llbracket T\rrbracket/T^sA\llbracket T\rrbracket$ is a finitely generated $A$-module, hence the image of $J$ in it is finitely generated as well, $A$ being noetherian. We thus may choose $a^{(N+1)},\ldots,a^{(N+M)}\in J$ whose images modulo $T^sA\llbracket T\rrbracket$ generate the image of $J$ in $A\llbracket T\rrbracket/T^sA\llbracket T\rrbracket$. Then $a^{(1)},\ldots,a^{(N+M)}$ generate $J$ and our job's done here.
\end{proof}
\begin{proof}[Proof of Proposition~\reff{prop:completionNoetherian}]
	Let $r_1,\ldots,r_n$ be generators of $I$. Then sending $X_i\mapsto r_i$ defines a surjective morphism $A\llbracket X_1,\ldots,X_n\rrbracket\epimorphism \roof{A}$. Since $A\llbracket X_1,\ldots,X_n\rrbracket$ is noetherian by Lemma~\reff{lem:(not)HilbertBasis} and induction on $n$, so is its quotient $\roof{A}$.
\end{proof}
\begin{cor}
	Suppose that $A$ is a noetherian local ring and $I\subseteq A$ any (proper) ideal. Then $\dim A=\dim\roof{A}$. In particular, $A$ is regular iff $\roof{A}$ is regular.
\end{cor}
\begin{proof}
	Let $\mm$ be the maximal ideal of $A$. Then $\roof{\mm}=\mm\roof{A}$ (this equality holds because of Corollary~\ref{cor:JM}) is the maximal ideal of the local ring $\roof{A}$ as was shown in the proof of \cite[Corollary~2.2.2]{homalg}. Since $I\subseteq \mm$, the quotients $\mm^i/\mm^{i+1}$ already have $I$-torsion, hence 
	\begin{align*}
		\mm^i/\mm^{i+1}\cong \left(\mm^i/\mm^{i+1}\right)^\complete\cong \roof{\mm}^i/\roof{\mm}^{i+1}
	\end{align*}
	(the last isomorphism follows from exactness of completion). This shows that the associated graded rings $\gr(A,\mm)$ and $\gr(\roof{A},\roof{\mm})$ agree, hence $(A,\mm)$ and $(\roof{A},\roof{\mm})$ have the same Hilbert--Samuel polynomials, which shows $\dim A=\dim\roof{A}$ by \cite[Theorem~20]{alg2}.
	
	Now $A$ and $\roof{A}$ have the same residue field $k$ and $\mm/\mm^2\cong \roof{\mm}/\roof{\mm}^2$ by the $I$-torsion arguments we have seen several times now, so $\dim_k\mm/\mm^2=\dim_k\roof{\mm}/\roof{\mm}^2$. Clearly this implies that $A$ is regular iff $\roof{A}$ is.
\end{proof}
\begin{rem}
	In a similar fashion one can show that a noetherian local ring is Cohen--Macaulay, or Gorenstein, or a complete intersection, iff its $I$-adic completion is one as well. For example, for Cohen--Macaulayness one would need to show $\depth_A(A)=\depth_{\roof{A}}(\roof{A})$, which follows from the isomorphism $\Ext_A^p(k,A)\cong\Ext_{\roof{A}}^p(k,\roof{A})$ that was described in the proof of \cite[Proposition~2.4.2]{homalg}.
\end{rem}
\begin{prop}\lbl{prop:modulesComplete}
	Let $A$ be a noetherian ring which is complete in the $I$-adic topology and let $M$ be a finitely generated $A$-module. Then $M$ is $I$-adically complete.
\end{prop}
\begin{proof}
	Note that this is clearly fulfilled if $M=A^n$ is a finitely generated free $A$-module. Now let $M$ be arbitrary. Since $A$ is noetherian, $M$ can be represented as $\coker\left(A^m\morphism A^n\right)$. Because $A^m$, $A^n$ equal their own completions (as we have just seen) and  completion is exact (by Corollary~\reff{cor:completionExact}), $M=\roof{M}$ holds as well.
\end{proof}
\begin{cor}\lbl{cor:finiteAlgebras}
	Let $A$ be a noetherian local ring maximal ideal $\mm$. Let $B$ be a finite $A$-algebra.
	\begin{alphanumerate}
		\item Then $B$ has only finitely many prime ideals over $\mm$, and all of them are maximal.
		\item If $A$ is, in addition, $\mm$-adically complete, then all maximal ideals of $B$ lie over $\mm$. In particular, $B$ is \defemph{semi-local} (i.e.\ has finitely many maximal ideals). The same is true if $A\subseteq B$.
	\end{alphanumerate}
\end{cor}
\begin{proof}
	Part \itememph{a}. If $\qq\in\Spec B$ is a prime ideal over $\mm$, then $B/\qq$ is a finitely generated domain over the residue field $k=A/\mm$, hence a finite field extension of $k$, so $\qq$ is maximal. Moreover, $B/\mm B$ is a finite-dimensional $k$-algebra, hence it has only finitely many maximal ideals by the argument from Fact~\reff{fact:annoyingQF}\itememph{b}.
	
	Part \itememph{b}. If $A$ is $\mm$-adically complete, then $B$ is $\mm B$-adically complete by Proposition~\reff{prop:modulesComplete}, so $\mm B$ is contained in the Jacobson radical $\rad B$ by Corollary~\reff{cor:HenselApplications}\itememph{c}. Then all maximal ideals of $B$ lie over $\mm$.
	
	Now assume $A\subseteq B$. Let $\qq\in\Spec B$ be a maximal ideal and $\pp=\qq\cap A$. Then $A/\pp\subseteq B/\qq$ is an integral ring extension in which $B/\qq$ is a field, hence so is $A/\pp$, (by \cite[Proposition~1.5.1\itememph{d}]{alg1}) which proves $\pp=\mm$.
\end{proof}

\printbibliography

\end{document}          

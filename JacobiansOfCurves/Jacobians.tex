\documentclass[a4paper,parskip=half,numbers=enddot, DIV=12]{scrreprt}
%\usepackage[utf8]{inputenc}

\usepackage{../header}
\usepackage{../frankenumbering2}
\usepackage{../shortcuts2}

\usepackage{blindtext}

\usepackage{eurosym}
\usetikzlibrary{fadings}

\usepackage{csquotes}
%\usepackage{tikz-cd}%I cannot draw diagrams without it - Felix. %well, I can - Ferdinand
\usepackage[backend=biber,style=alphabetic]{biblatex}
\setcounter{biburlnumpenalty}{7000}
\setcounter{biburllcpenalty}{7000}
\setcounter{biburlucpenalty}{8000}
\addbibresource{../literatur.bib}

% Title Page
\title{Jacobians of Curves}
\author{\textsc{Lecturer:} Jens Franke\\
	\textsc{Notes:} Ferdinand Wagner}
\date{Wintersemester 2018/19}

\displaywidowpenalty=8000
%\postdisplaypenalty=8000
\widowpenalty=8000
\clubpenalty=8000

\newcommand{\vC}{\v{C}}
\renewcommand{\geq}{\geqslant}
\renewcommand{\leq}{\leqslant}

\DeclareSymbolFont{cyrletters}{OT2}{wncyr}{m}{n}
\DeclareMathSymbol{\Sha}{\mathalpha}{cyrletters}{"58}
\usetikzlibrary{backgrounds}
\newcommand{\tikzentry}[3][]{\tikz[remember picture, baseline =(#2.base)]\node[#1, inner sep=0pt, outer sep=0pt] (#2) {$#3$};}

\begin{document}
\pagenumbering{Alph}
\maketitle
\pagenumbering{roman}

\thispagestyle{plain}
This text consists of notes on the lecture Jacobians of Curves, taught at the University of Bonn by Professor Jens Franke in the winter term (Wintersemester) 2018/19.

Please report bugs, typos etc. through the \emph{Issues} feature of github.

\tableofcontents

%\addchap{Introduction}
\pagenumbering{arabic}
%Nothing here yet. 

\chapter{Introduction and preparations}

\section{A note on limits and their derived functors}
	Let $X_\bullet\colon \ldots\xrightarrow{p_{i+1}}X_i\morphism[p_i]\ldots\morphism[p_2]X_1\morphism[p_1]X_0$ be a diagram of abelian groups or $R$-modules. As usual, we may view $X_\bullet$ as a functor $X_\bullet\colon (\IN,\geq)\morphism \cat{Ab}$ or $\cat{Mod}(R)$, where the category $(\IN,\geq)$ has the nonnegative integers as objects and an arrow $j\morphism i$ iff $j\geq i$. Let 
	\begin{align*}
		d\colon \prod_{i=0}^{\infty}X_i\morphism\prod_{i=0}^\infty X_i\;,\quad d\left(x_i\right)_{i=0}^\infty=\left(p_{i+1}(x_{i+1})-x_i\right)_{i=0}^\infty\;.
	\end{align*}
	Then we put
	\begin{align*}
		\limit[i\in\IN]X_i=\ker d\quad\text{and}\quad\limit[i\in\IN][^1]X_i=\coker d\;.
	\end{align*}
\begin{rem}
	It is easy to see that $\limit X_i$ equals the usual category-theoretical limit (that's how you construct it). It can also be shown that $\limit[][^1]$ is the first right-derived functor of $\limit$, and that its higher derived functors vanish.
\end{rem}
\begin{fact}\lbl{fact:6termLimitSequence}
	Let $0\morphism X'_\bullet\morphism X_\bullet\morphism X''_\bullet\morphism 0$ be a short exact sequence of diagrams of the above type. Then there is a canonical exact sequence
	\begin{align*}
		0\morphism \limit[i\in\IN] X'_i\morphism \limit[i\in\IN] X_i\morphism \limit[i\in\IN] X''_i\morphism \limit[i\in\IN][^1] X'_i\morphism \limit[i\in\IN][^1]X_i\morphism \limit[i\in\IN][^1] X''_i\morphism 0\;.
	\end{align*}
\end{fact}
\begin{proof}
	Since products preserve exact sequences in $\cat{Ab}$ or $\cat{Mod}(R)$, we get a diagram
	\begin{diagram*}
		\object{0,2}{$0\vphantom{\displaystyle\prod_{i=0}^\infty }$}[0o];
		\object{2,2}{$\displaystyle\prod_{i=0}^\infty X'_i$}[1o];
		\object{4,2}{$\displaystyle\prod_{i=0}^\infty X_i$}[2o];
		\object{6,2}{$\displaystyle\prod_{i=0}^\infty X''_i$}[3o];
		\object{8,2}{$0\vphantom{\displaystyle\prod_{i=0}^\infty }$}[4o];
		\object{0,0}{$0\vphantom{\displaystyle\prod_{i=0}^\infty }$}[0u];
		\object{2,0}{$\displaystyle\prod_{i=0}^\infty X'_i$}[1u];
		\object{4,0}{$\displaystyle\prod_{i=0}^\infty X_i$}[2u];
		\object{6,0}{$\displaystyle\prod_{i=0}^\infty X''_i$}[3u];
		\object{8,0}{$0\vphantom{\displaystyle\prod_{i=0}^\infty }$}[4u];
		\scriptsize
		\arrow{0o}{1o};
		\arrow{1o}{2o};
		\arrow{2o}{3o};
		\arrow{3o}{4o};
		\arrow{0u}{1u};
		\arrow{1u}{2u};
		\arrow{2u}{3u};
		\arrow{3u}{4u};
		\arrow{1o}{1u}[left][$d'$];
		\arrow{2o}{2u}[left][$d$];
		\arrow{3o}{3u}[left][$d''$];
	\end{diagram*}
	with exact rows. Then the snake lemma finishes the job.
\end{proof}
\begin{fact}\lbl{fact:limVanishing}
	Let $X_\bullet$ have the property that for every $i\in\IN$ there is a $j\geq i$ such that the composition $p_{j,i}\colon X_j\morphism[p_j]X_{j-1}\xrightarrow{p_{j-1}}\ldots\xrightarrow{p_{i+1}}X_i$ vanishes. Then
	\begin{align*}
		\limit[i\in\IN]X_i=\limit[i\in\IN][^1]X_i=0\;.
	\end{align*}
\end{fact}
\begin{proof}
	If $x=\left(x_i\right)_{i=0}^\infty\in\limit X_i$, then $x_i=p_{j,i}(x_j)$ for all $j\geq i$ by construction, hence $x_i=0$ for all $i\in\IN$. Moreover, let
	\begin{align*}
		s\colon \prod_{i=0}^{\infty}X_i\morphism\prod_{i=0}^\infty X_i\;,\quad s(x)_i=\sum_{j\geq i}p_{j,i}(x_j)\;.
	\end{align*}
	By assumption $s$ is well-defined. Then
	\begin{align*}
		d(s(x))_i=p_{i+1}\bigg(\sum_{j\geq i+1}p_{j,i+1}(x_j)\bigg)-\sum_{j\geq i}p_{j,i}(x_j)=-p_{i,i}(x_i)=-x_i\;.
	\end{align*}
	Hence $-s$ is a right-inverse of $d$, so $\limit[][^1]X_i=\coker d$ vanishes as well.
\end{proof}
\begin{fact}\lbl{fact:MittagLeffler}
	Let $X_\bullet$ have the \defemph{Mittag-Leffler property} that for every $i\in\IN$ there is a $j\geq i$ such that for all $k\geq j$ the images of $p_{j,i}$ and $p_{k,i}$ in $X_i$ coincide. Then $\limit[][^1]X_i=0$.
\end{fact}
\begin{proof}
	Let's first deal with the special case that each $p_i\colon X_i\morphism X_{i-1}$ is surjective. Let $x=\left(x_i\right)_{i=0}^\infty\in\prod_{i=0}^\infty X_i$. For every $i\in\IN$ we may select $x_j^{(i)}\in X_j$ for all $j\geq i$ in such a way that $x_i^{(i)}=x_i$ and $p_{j+1}\big(x_{j+1}^{(i)}\big)=x_j^{(i)}$. Then $s(x)$ defined by
	\begin{align*}
		s(x)_i=\sum_{k=0}^{i-1}x_i^{(k)}
	\end{align*}
	is a preimage of $x$ under $d$, so $\limit[][^1]X_i=\coker d=0$ in this case.
	
	Now let $X_\bullet$ be arbitrary with the Mittag-Leffler property. Let $Y_i=\bigcap_{j\geq i}p_{j,i}(X_j)\subseteq X_i$. Then $\limit[][^1]Y_i=0$ by the special case we just treated, and $\limit[][^1]X_i/Y_i=0$ by Fact~\reff{fact:limVanishing}. Since $\limit[][^1]X_i$ is sandwiched between these two in the exact sequence from Fact~\reff{fact:6termLimitSequence}, this shows $\limit[][\smash{^1}]X_i=0$, as required.
\end{proof}
\section{The theorem about formal functions}
Let $f\colon X\morphism Y=\Spec A$ be a  morphism of quasi-compact schemes. Let $I\subseteq A$ be any ideal. Consider
\begin{align*}
	i_n\colon X_n=X\times_Y\Spec (A/I^n)\morphism X\;,
\end{align*}
which is a base change of the closed immersion $Y_n=\Spec (A/I^n)\monomorphism \Spec A$, hence indeed a closed immersion itself. Also, if $f$ is proper, then so is $X_n\morphism Y_n$ because properness is another \emph{proper}ty (tee-hee) that is stable under base change (by \cite[Remark~2.4.1]{alggeo2}).

Let $\Ff$ be a quasi-coherent sheaf of $\Oo_X$-modules and $\Ff|_{X_n}=i_n ^*\Ff$ its restriction to $X_n$ (this notation is slightly abusive, but convenient). We put $\Ff_n=i_{n,*}\Ff|_{X_n}$. It's easy to check (e.g.\ affine-locally) that $\Ff_n\cong \Ff/I^n\Ff$. Since $i_n$ is a closed immersion and thus affine, we have an isomorphism $H^p(X,\Ff_n)\cong H^p(X_n,\Ff|_{X_n})$ for all $p\geq 0$ by \cite[Corollary~1.5.1]{alggeo2}. Together with the canonical projection $\Ff_{n+1}\cong \Ff/I^{n+1}\Ff\morphism \Ff/I^n\Ff\cong \Ff_n$ this gives canonical morphisms $H^p(X_{n+1},\Ff|_{X_{n+1}})\morphism H^p(X_n,\Ff|_{X_n})$ for all $n\in\IN$.

The canonical morphism $\Ff\morphism i_{n,*}i_n^*\Ff=\Ff_n$ induces a morphism
\begin{align}\lbl{eq:Hmorphism}
	H^p(X,\Ff)\morphism H^p(X,\Ff_n)\cong H^p(X_n,\Ff|_{X_n})
\end{align}
for all $p\geq 0$ (the isomorphism on the right-hand side comes from the fact that $i_n$ is a closed immersion, hence affine, and we can apply \cite[Corollary~1.6.1]{alggeo2}). This is a morphism of $A$-modules, but $H^p(X_n,\Ff|_{X_n})$ is actually an $A/I^n$-module, so \eqreff{eq:Hmorphism} factors over
\begin{align*}
	H^p(X,\Ff)/I^nH^p(X,\Ff)\morphism H^p(X_n,\Ff|_{X_n})\;.
\end{align*}
This is compatible with the canonical morphisms $H^p(X_{n+1},\Ff|_{X_{n+1}})\morphism H^p(X_n,\Ff|_{X_n})$ (you can just check that on an affine \v Cech covers). Passing to the limit gives a morphism
\begin{align}\lbl{eq:formalFunctions}
	H^p(X,\Ff)^\complete\morphism\limit[n\geq 1]H^p(X_n,\Ff|_{X_n})\;,
\end{align}
where $\roof{\phantom{x}}$ denotes the $I$-adic completion.
\begin{thm}[Grothendieck--Zariski]\lbl{thm:FormalFunctions}
	When $f\colon X\morphism Y=\Spec A$ is proper (in which case $X$ is automatically a quasi-compact scheme), $A$ is noetherian and $\Ff$ is a coherent sheaf of $\Oo_X$-modules, then \eqreff{eq:formalFunctions} is an isomorphism
	\begin{align*}
		H^p(X,\Ff)^\complete\isomorphism \limit[n\geq 1]H^p(X_n,\Ff|_{X_n})\;.
	\end{align*}
\end{thm}

\begin{proof}
	The following proof is essentially the one from \cite[(4.1.7)]{egaIII}. Professor Franke also pointed out that the idea is pretty similar to the proof of the Artin--Rees lemma. Let $I\subseteq A$ be the ideal under consideration and let $R=\bigoplus_{n\geq 0}I^n$ be the Rees algebra associated to $I$. Then
	\begin{align*}
		K^p=\bigoplus_{n\geq 0}H^p(X,I^n\Ff)
	\end{align*}
	is a module over $R$ as $i\in I^m$ (considered as the $m\ordinalth$ homogeneous component of $R$) maps $I^n\Ff$ to $I^{n+m}\Ff$.
	\begin{claim}\lbl{claim:FFclaim1}
		$K^p$ is a finitely generated $R$-module for all $p\geq 0$.
	\end{claim}
	Assuming this for the moment, recall that $\Ff_n\cong \Ff/I^n\Ff$ and $H^p(X_,\Ff|_{X_n})\cong H^p(X,\Ff_n)$, so the long exact cohomology sequence associated to $0\morphism I^n\Ff\morphism \Ff\morphism \Ff_n\morphism 0$ appears as
	\begin{align}\lbl{eq:FFSeq1}
		H^p(X,I^n\Ff)\morphism H^p(X,\Ff)\morphism H^p(X_n,\Ff|_{X_n})\morphism H^{p+1}(X,I^n\Ff)\;.
	\end{align}
	As pointed out after \eqreff{eq:Hmorphism}, $H^p(X,\Ff)\morphism H^p(X_n,\Ff|_{X_n})$ factors over $H^p(X,\Ff)/I^nH^p(X,\Ff)$, hence we can turn equation \eqreff{eq:FFSeq1} into an exact sequence
	\begin{align}\lbl{eq:FFSeq2}
		0\morphism U_n\morphism H^p(X,\Ff)/I^nH^p(X,\Ff)\morphism H^p(X_n,\Ff|_{X_n})\morphism V_n\morphism 0\;,
	\end{align}
	where $U_n$ is a suitable quotient of $H^p(X,I^n\Ff)$ and $V_n\subseteq H^{p+1}(X,I^n\Ff)$ some submodule. This makes $U=\bigoplus_{n\geq 0}U_n$ a quotient of $K^p$ and $V=\bigoplus_{n\geq 0}V_n$ an $R$-submodule of $K^{p+1}$.
	\begin{claim}\lbl{claim:FFclaim2}
		We have $\limit U_n=\limit[][^1]U_n=0$ and $\limit V_n=\limit[][^1]V_n=0$.
	\end{claim}
	Before we prove this (and Claim~\reff{claim:FFclaim1}), let's see how Theorem~\reff{thm:FormalFunctions} follows from it. Let $W_n$ be the image of $H^p(X,\Ff)/I^nH^p(X,\Ff)$ in $H^p(X_n,\Ff|_{X_n})$. We may split \eqreff{eq:FFSeq2} into two short exact sequences $0\morphism U_n\morphism H^p(X,\Ff)/I^nH^p(X,\Ff)\morphism W_n\morphism 0$ and $0\morphism W_n\morphism H^p(X_,\Ff|_{X_n})\morphism V_n\morphism 0$. Applying Fact~\reff{fact:6termLimitSequence} to the first one gives $H^p(X,\Ff)^\complete\cong \limit[][^1]W_n$. Then the six-term exact sequence associated to the second proves $\limit W_n\cong \limit H^p(X_n\Ff|_{X_n})$ and we are done.
	
	It remains to show the two claims. Note that the Rees algebra $R$ is noetherian. Indeed, $I$ is finitely generated as an ideal in the noetherian ring $A$, hence $R$ is of finite type over $A$. Let's also make the following convention: Whenever we write $I^kU_n$ or $I^kV_n$ in the following, this means multiplication as $A$-modules and the result is contained in $U_n$ resp.\ $V_n$ again, whereas $R_kU_n$ or $R_kV_n$ means multiplication by the $k\ordinalth$ homogeneous component of $R$ (which equals $I^k$ as well), so the result is contained on $U_{k+n}$ resp.\ $V_{k+n}$.
	
	\emph{Proof of Claim~\reff{claim:FFclaim2}.} Note that $U$ is finitely generated over $R$, since it is a quotient of the finitely generated $R$-module $K^p$. Fix a finite set of generators and let $d_0$ the maximal non-zero homogeneous components occurring in this set. Then $U_n=R_nU_0+R_{n-1}U_1+\ldots+R_{n-d_0}U_{d_0}$ for all $n\geq d_0$. In particular, $U_{k+n}=R_kU_n$ for all $n\geq d_0$. Thus, for every $n\geq d_0$ the image of $U_{2n}=R_nU_n$ in $U_n$ is contained in $I^nU_n$. But $U_n\subseteq H^p(X,\Ff)/I^nH^p(X,\Ff)$, so $I^nU_n$ vanishes. Therefore, the property from Fact~\reff{fact:limVanishing} is fulfilled for all $n\geq d_0$. But then it clearly holds for all $n\geq 0$ as well, so Fact~\reff{fact:limVanishing} is applicable.
	
	Similarly, $V$ is finitely generated as a submodule of $K^{p+1}$, which is finitely generated over the noetherian ring $R$ by Claim~\reff{claim:FFclaim1}. By the same argument as above we find a $d_1$ such that $V_n=R_nV_0+R_{n-1}V_1+\ldots+R_{n-d_1}V_{d_1}$ for all $n\geqslant d_1$. In particular, we have $V_{k+n}=R_kV_n$ for all $n\geq d_1$. Thus, for $n\geq d_1$ the image of $V_{2n}$ in $V_n$ is contained in $I^nV_n$. But $I^nV_n$ vanishes again, since $V_n$ is the image of $H^p(X_n,\Ff|_{X_n})$, which is a $A/I^n$-module. As above, we can apply Fact~\reff{fact:limVanishing}. This shows Claim~\reff{claim:FFclaim2}.
	
	\emph{Proof of Claim~\reff{claim:FFclaim1}.} Let $\upsilon\colon\snake{Y}=\Spec R\morphism Y$ correspond to $A\monomorphism R$ and let $\xi\colon \snake{X}=X\times_Y\snake{Y}\morphism X$ be its base change by $f$. Note that $\xi$ is affine as a base change of the affine morphism $\upsilon$ (we use \cite[Corollary~2.5.1]{alggeo1} here). We claim
	\begin{align*}
		\xi_*\xi^*\Ff\cong\bigoplus_{n\geq 0}I^n\Ff\;.
	\end{align*}
	Indeed, this is easily checked affine-locally (where $\xi^*$ is given by tensoring with $R$); we leave the details to the reader. Also $H^p(\snake{X},\xi^*\Ff)\cong H^p(X,\xi_*\xi^*\Ff)$ as $\xi$ is affine. This shows
	\begin{align*}
		H^p(\snake{X},\xi^*\Ff)\cong H^p(X,\xi_*\xi^*\Ff)\cong H^p\bigg(X,\bigoplus_{n\geq 0}I^n\Ff\bigg)\cong \bigoplus_{n\geq 0}H^p(X,I^n\Ff)=K^p\;.
	\end{align*}
	Note that direct sums usually \emph{don't} commute with cohomology, but here they do, because $X$ is quasi-compact and $\bigoplus_{n\geq 0}I^n\Ff$ is quasi-coherent (for which we need quasi-compactness as well), so we may compute $H^p\left(X,\bigoplus_{n\geq 0}I^n\Ff\right)$ via finite affine \v Cech covers. In this case, the products in the \v Cech complex are all finite, hence commute with the direct sum, which is what we needed.
	
	Now $\snake{f}\colon \snake{X}\morphism\snake{Y}=\Spec R$ is proper (as a base change of the proper morphism $f$), hence the right-hand side is a finitely generated $R$-module by our finiteness results for the cohomology of proper morphisms (cf.\ \cite[Theorem~5]{alggeo2}). We win.
\end{proof}
\begin{rem}
	Note that in the lecture Franke used $\Kk_n\cong \Jj^n\Ff$ instead of $I^n\Ff$, where $\Jj=f^{-1}\Ii$ is the inverse image (in the sense of Definition~\reff{def:f-1}). But $\Jj^n\Ff\cong I^n\Ff$ -- which is not that surprising, since the $I^n$-action on $\Ff$ is given via the algebraic component $\Oo_Y\morphism f_*\Oo_X$ of $f$, so $I^n\Ff=\Jj^n\Ff$ is pretty obvious from the construction of $f^{-1}$ described in the proof of Lemma~\reff{lem:f-1} below. I prefer the notation $I^n\Ff$ -- in particular, this is how Grothendieck denotes it in \cite[(4.1.7)]{egaIII}, so I believe it's my right to do so as well. Nevertheless, Lemma~\reff{lem:f-1} is \emph{perhaps worthwhile to know} (if you get what I mean), so we will include it now.
\end{rem}
\begin{defi}\lbl{def:f-1}
	Let $f\colon X\morphism Y$ be any morphism of preschemes and $\Jj\subseteq \Oo_Y$ a sheaf of ideals on $Y$. Then define $f^{-1}\Jj$ to be the image of $f^*\Jj\morphism\Oo_X$ (which is obtained as the composition of the pullback of $\Jj\morphism \Oo_Y$ with the isomorphism $f^*\Oo_Y\cong \Oo_X$).
\end{defi}
\begin{lem}\lbl{lem:f-1}
	Let $f\colon X\morphism Y$ be any morphism of preschemes and $\Jj\subseteq \Oo_Y$ quasi-coherent.
	\begin{alphanumerate}
		\item $f^{-1}\Jj\subseteq \Oo_X$ is quasi-coherent.
		\item Let $Y_0$ and $X_0$ be the closed subpreschemes of $Y$ and $X$ defined by $\Jj$, $f^{-1}\Jj$ respectively. Then $X_0\cong X\times_YY_0$.
		\item For all $n\geq 0$ we have $f^{-1}(\Jj^n)\cong (f^{-1}\Jj)^n$.
	\end{alphanumerate}
\end{lem}
\begin{proof}[Sketch of a proof]
	The question is easily seen to be local on both $X$ and $Y$. So let's consider the affine situation where $Y=\Spec A$, $X=\Spec B$, and $\Jj=\snake{J}$ for some ideal $J\subseteq A$. Let $\phi\colon A\morphism B$ be the morphism of rings corresponding to $f$. Then $f^{-1}\Jj=\snake{I}$ where $I$ is the image of $B\otimes_AJ\morphism B$ sending $b\otimes j\mapsto b\cdot \phi(j)$. All three assertions are then easily checked.
\end{proof}
\begin{rem}\lbl{rem:fibres}
	Recall that for a morphism $f\colon X\morphism Y$ of preschemes and a point $y\in Y$ the \defemph{fibre} $f^{-1}\{y\}$ of $f$ at $y$ is defined as the prescheme $f^{-1}\{y\}=X\times_Y\Spec \KK(y)$. This makes sense, since $f^{-1}\{y\}$ is indeed -- topologically -- the preimage of $y$, as proved in \cite[Corollary~1.3.3]{alggeo1}. Moreover, $\Spec \Oo_{Y,y}/\mm_{Y,y}^n\morphism Y$ is immersive for all $n\geq 1$ and has image $\{y\}$ as well. So \cite[Corollary~1.3.3]{alggeo1} is applicable again and shows that $X_n=X\times_Y\Spec (\Oo_{Y,y}/\mm_{Y,y}^n)$ has $f^{-1}\{y\}$ as underlying topological space too (but, of course, the prescheme structure differs in general). We may thus think of $X_n$ as the $(n-1)\ordinalst$ infinitesimal thickening of $f^{-1}\{y\}$. 
\end{rem}
Using this, Theorem~\reff{thm:FormalFunctions} can be restated as follows.
\begin{varthm}{thm}\lbl{thm:FormalFunctions2}
	Let $f\colon X\morphism Y$ be a proper morphism between locally noetherian\footnote{Franke only assumes $Y$ to be locally noetherian, but $f$ being of (locally) finite type implies that $X$ is locally noetherian as well by Hilbert's Basissatz. This happens multiple times throughout the text.} preschemes. Let $\Ff$ be a coherent $\Oo_X$-module. For every $y\in Y$ let $X_n=X\times_Y\Spec (\Oo_{Y,y}/\mm_{Y,y}^n)$ be the $(n-1)\ordinalst$ infinitesimal thickening of $f^{-1}\{y\}$. Then there is an isomorphism
	\begin{align*}
		(R^pf_*\Ff)_y^\complete\isomorphism\limit[n\geq 1] H^p(X_n,\Ff|_{X_n})\;,
	\end{align*}
	where $\roof{\phantom{x}}$ denotes the $\mm_{Y,y}$-adic completion.
\end{varthm}
\begin{proof}
	We may assume that $Y=\Spec A$ is affine, and that $A$ is a noetherian ring. Indeed, replacing $Y$ by an affine neighbourhood $U\cong \Spec A$ and $X$ by $f^{-1}(U)$ doesn't change $(R^pf_*\Ff)_y$ (because the construction of $R^pf_*\Ff$ is base-local) and also $X_n$ is preserved since $f^{-1}\{y\}$ is already contained in $f^{-1}(U)$ (by \cite[postnote]{alggeo2}).
	
	In this case, $R^pf_*\Ff=H^p(X,\Ff)^\qcmod$ by \cite[Proposition~1.5.1\itememph{d}]{alggeo2}. Let $\pp\in\Spec A$ be the prime ideal associated to $y$. Then $R^pf_*\Ff\cong H^p(X,\Ff)_\pp$ and $\Oo_{Y,y}\cong A_\pp$ is flat over $A$. Let $\mm=\pp A_\pp\cong \mm_{Y,y}$ be its maximal ideal. We denote $\pi\colon \Spec A_\pp\morphism \Spec A$. Applying \cite[Fact~4.1.1]{alggeo2} to $\pi$ gives 
	\begin{align*}
		H^p(X\times_Y\Spec A_\pp,\pi^*\Ff)\cong H^p(X,\Ff)_\pp\cong (R^pf_*\Ff)_y\;.
	\end{align*}
	Also
	\begin{align*}
		\left(X\times_Y\Spec A_\pp\right)\times_{\Spec A_\pp}\Spec(A_\pp/\mm^n)&\cong X\times_Y\left(\Spec A_\pp\times_{\Spec_{A_\pp}}\Spec (A_\pp/\mm^n)\right)\\
		&\cong X\times_Y\Spec(A_\pp/\mm^n)\\
		&\cong X_n
	\end{align*}
	by a bit abstract nonsense. Now Theorem~\reff{thm:FormalFunctions} may be applied to $X\times_Y\Spec A_\pp\morphism \Spec A_\pp$ (the base change of $f$) and the assertion follows.
\end{proof}
\section{Application to Zariski's main theorem}
\subsection{A lot of (not necessarily main) theorems by Zariski}
Out there in the real world, there are multiple \emph{main theorems} of Zariski around, and usually they're only loosely related. Professor Franke recommends Mumford's \emph{The red book of varieties and schemes} for a discussion of various such version.
\begin{cor}\lbl{cor:Rpvanishing}
	Let $f\colon X\morphism Y$ be any proper morphism between locally noetherian preschemes and let $d=\sup_{y\in Y}\dim \left(f^{-1}\{y\}\right)$. If $\Ff$ is a coherent $\Oo_X$-module and $p>d$, then $R^pf_*\Ff=0$.
\end{cor}
\begin{proof}
	Since $R^pf_*\Ff$ is coherent (this is \cite[Theorem~5]{alggeo2}), $(R^pf_*\Ff)_y$ is a finitely generated $\Oo_{Y,y}$-module, hence it vanishes iff it $\mm_{Y,y}$-adic completion vanishes by Fact~\reff{fact:completion101}\itememph{b}. But $X_n=X\times_Y\Spec (\Oo_{Y,y}/\mm_{Y,y}^n)$ has underlying space $f^{-1}\{y\}$ (as explained in Remark~\reff{rem:fibres}), hence $H^p(X_n,\Ff|_{X_n})=0$ when $p>d$ by Grothendieck's theorem on cohomological dimension (cf.\ \cite[Proposition~1.4.1]{alggeo2}). The assertion now follows from Theorem~\reff{thm:FormalFunctions2}.
\end{proof}
\begin{defi}\lbl{def:quasiFinite}
	A morphism $f\colon X\morphism Y$ of finite type is called \defemph{quasi-finite at $\boldsymbol{x\in X}$} if $x$ is discrete in its fibre, i.e., if $\{x\}$ is an open and closed subset of $f^{-1}\{y\}$ where $y=f(x)$. We call $f$ \defemph{quasi-finite} if it is quasi-finite at every $x\in X$. 
\end{defi}
The following fact wasn't mentioned in the lecture, but it's \emph{definitely} (in particular, not only \emph{perhaps}) \emph{worthwhile to know}!
\begin{fact}\lbl{fact:openDiscrete}
	Let $f\colon X\morphism Y$ be a morphism of finite type. Let $x\in X$ be open in its fibre $f^{-1}\{y\}$, where $y=f(x)$. Then $f$ is already quasi-finite at $x$. 
\end{fact}
\begin{proof}
	Choose an affine open neighbourhood $y\in U\cong \Spec A$. Then $f^{-1}\{y\}$ is contained in $f^{-1}(U)$, so we may w.l.o.g.\ assume that $Y=\Spec A$ is affine. Put $k=\KK(y)$. Since $X$ may be covered by affine open subsets $\Spec R$, where $R$ is of finite type over $A$, we may cover the fibre product $f^{-1}\{y\}=X\times_Y\Spec k$ by affine open subsets $\Spec (R\otimes_A k)$, in which $R\otimes_A k$ is a $k$-algebra of finite type, hence a Jacobson ring. This proves that $f^{-1}\{y\}$ is a Jacobson prescheme as in \cite[Definition~2.4.2\itememph{c}]{alggeo1}. But then $x$ is a closed point of the open subset $\{x\}\subseteq f^{-1}\{y\}$, hence also a closed point of $f^{-1}\{y\}$ by \cite[Fact~2.4.1\itememph{c}]{alggeo1}.
\end{proof}
\begin{fact}
	\begin{alphanumerate}
		\item \lbl{fact:annoyingQF}Any finite morphism is quasi-finite.
		\item If $k$ is a field, a morphism $f\colon X\morphism\Spec k$ of finite type is quasi-finite iff it is finite.
		\item Let $f\colon X\morphism Y$ and $g\colon Y\morphism Z$ be morphisms of finite type such that $g$ is quasi-finite at $y=f(x)$ for some $x\in X$. Then $gf$ is quasi-finite at $x$ iff $f$ is quasi-finite at $x$.
		\item Base changes of a quasi-finite morphism is quasi-finite again.
	\end{alphanumerate}
\end{fact}
\begin{proof}
	Maybe that's my bad, but the proof of this is actually annoyingly laborious. We begin with part~\itememph{a}. Let $f\colon X\morphism Y$ be a finite morphism, $x\in X$ and $y=f(x)$. Then the morphism
	\begin{align*}
		f^{-1}\{y\}=X\times_Y\Spec \KK(y)\morphism \Spec \KK(y)
	\end{align*}
	is finite again, as a base changes of finite morphisms are finite again (cf.\ \cite[Corollary~1.5.1]{alggeo1}). Letting $k=\KK(y)$ this puts us in the situation from \itememph{b}, so it's sufficient to prove \itememph{b}.
	
	In the case of \itememph{b} we have $f^{-1}\{y\}=X$, so what we need to show is that $X$ carries the discrete topology if $f$ is finite. We know that $X\cong \Spec R$ where $R$ is some finite-dimensional $k$-algebra (using finiteness of $f$). For $x\in X$ let $\pp$ be the corresponding prime ideal of $R$. Then $R/\pp$ is a domain and a finite-dimensional $k$-vector space, hence a finite field extension of $k$ by Hilbert's Nullstellensatz. This means that $\pp$ is a maximal ideal of $R$. Consequently, all points of $X$ are closed, so it suffices to show that $X$ has finitely many points. Let $\{x_1,\ldots,x_n\}$ be any finite subset of $X$ and $\{\mm_1,\ldots,\mm_n\}$ the corresponding maximal ideals of $R$. For every $i$, we may choose an element $\alpha_i\in\mm_i$ which is not contained in any $\mm_j$ for $j\neq i$ (e.g.\ by the prime avoidance lemma, cf.\ \cite[Lemma~2.5.1]{alg1}). Put $\beta_i=\prod_{j\neq i}\alpha_j$ (so that $\beta_i\in\mm_j$ for all $j\neq i$ but $\beta_i\notin \mm_i$). We claim that $\beta_1,\ldots,\beta_n$ are $k$-linearly independent. Indeed, if $\lambda_1\beta_1+\ldots+\lambda_n\beta_n=0$ for some coefficients $\lambda_1,\ldots,\lambda_n\in k$, then reducing modulo $\mm_i$ gives $\lambda_i\beta_i=0$ in $R/\mm_i=\KK(\mm_i)$. But $\beta_i\neq 0$ in $\KK(\mm_i)$, so $\lambda_i=0$ for all $i=1,\ldots,n$. This proves $\dim_kR\geq n$. But $R$ is finite-dimensional over $k$, hence $X$ must have finitely many points, as claimed.
	
	Conversely, assume that $f\colon X\morphism\Spec k$ is quasi-finite. Then $X$ is discrete, so it must have finitely many points. Indeed, $f$ being of finite type implies it is quasi-compact (by definition), so $X$ is quasi-compact because $\Spec k$ is, and any discrete quasi-compact space is finite. Let $X=\{x_1,\ldots,x_n\}$. Every point $x_i\in X$ together with the restriction $\Oo_X|_{\{x_i\}}$ of the structure sheaf is a prescheme again, hence affine (because $x_i\in\{x_i\}$ must have an affine neighbourhood). Let $\{x_i\}\cong \Spec R_i$. Then
	\begin{align*}
		X\cong\coprod_{i=1}^n\Spec R_i\cong\Spec\bigg(\bigoplus_{i=1}^nR_i\bigg)
	\end{align*}
	is affine. This shows that $f$ is affine, but finiteness is yet to prove. Clearly, it suffices that each $R_i$ is a finite-dimensional $k$-vector space. Note that $R_i$ has precisely one prime ideal $\mm_i$ (corresponding to $x_i$), which is then automatically maximal. Since $f$ is of finite type, $R_i$ has finite type over $k$. In particular $R_i$ is noetherian and we may choose generators $r_1,\ldots,r_m$ of $\mm_s$. Since $\mm_i$ is the only prime ideal of $R_s$, we have $\mm_i=\nil R_i$. Consquently, there is an $N\in\IN$ such that $r_\ell^N=0$ for all $\ell$. Moreover, $R_i/\mm_i$ is a field extension of finite type over $k$, hence a finite field extension by Hilbert's Nullstellensatz. Let $\beta_1,\ldots,\beta_d\in R_i$ be elements whose images modulo $\mm_i$ form a $k$-basis of $R_i/\mm_i$. Then it is straightforward to check that $R$ is generated as a $k$-vector space by the elements
	\begin{align*}
	\beta_j\cdot r_1^{e_1}r_2^{e_2}\cdots r_n^{e_n}\quad\text{where }0\leq e_\ell<N\text{ for all }\ell\;.
	\end{align*}
	This shows $\dim_kR<\infty$, hence $f$ is finite.
	
	Part \itememph{c}. Since $g$ is quasi-finite at $y$, the subset $\{y\}\subseteq g^{-1}\{g(y)\}$ is open and closed, hence $f^{-1}\{y\}\subseteq (gf)^{-1}\{g(y)\}$ is open and closed. This means that $\{x\}$ is open and closed in the fibre $(gf)^{-1}\{g(y)\}$  iff it is open and closed in $f^{-1}\{y\}$ and we win.
	
	Part~\itememph{d}. Let $f\colon X\morphism S$ be quasi-finite and $g\colon Y\morphism S$ another morphism of preschemes. The base change $\snake{f}\colon \snake{X}=X\times_SY\morphism Y$ is of finite type by \cite[Fact~2.2.2]{alggeo1}. Now let $y\in Y$ and put $s=g(y)$. By \itememph{b} we see that $X\times_S\Spec \KK(s)$ is finite over $\Spec \KK(s)$. Then
	\begin{align*}
		\snake{X}\times_Y\Spec\KK(y)\cong X\times_S\big(\Spec\KK(s)\times_S\Spec\KK(y)\big)\cong \big(X\times_S\Spec \KK(s)\big)\times_{\Spec \KK(s)}\Spec \KK(y)
	\end{align*}
	is finite over $\Spec\KK(y)$, so $\snake{f}$ is indeed quasi-finite by \itememph{b} again.
\end{proof}
\begin{thm}[Grothendieck's version of Zariski's main theorem]
	\begin{alphanumerate}
		\item \lbl{thm:ZariskiMain}Let $f\colon X\morphism Y$ be a quasi-finite proper morphism between locally noetherian preschemes. Then $f$ is finite.
		\item Let $f\colon X\morphism Y$ be a quasi-finite and separated morphism between noetherian preschemes. Then there exists a factorization $X\monomorphism[j]\ov{X}\morphism[g]Y$ of $f$ where $j$ is an open immersion and $g$ is finite.
		\item If $f\colon X\morphism Y$ is any morphism of finite type between locally noetherian preschemes, then 
		\begin{align*}
			U=\left\{x\in X\st f\text{ is quasi-finite at }x\right\}
		\end{align*}
		is open in $X$, and the restriction $f|_U$ is quasi-finite (by definition).
	\end{alphanumerate}
\end{thm}
\begin{proof}
	Part \itememph{a}. We may assume that $Y=\Spec A$ is affine (indeed, all involved properties are base-local). Let $\Jj\subseteq \Oo_X$ be a sheaf of ideals, then $\Jj$ is coherent as $X$ is locally noetherian. Since $f$ is quasi-finite, all fibres carry the discrete topology. In particular, they are zero-dimensional and Corollary~\reff{cor:Rpvanishing} shows that $R^1f_*\Jj=0$. Then also $0=R^1f_*\Jj(Y)=H^1(X,\Jj)$ (using \cite[Proposition~1.5.1\itememph{d}]{alggeo2}), hence $X$ is affine by Serre's affinity criterion. This shows that $f$ is affine. Moreover, $f_*\Oo_X$ is a coherent $\Oo_Y$-module by \cite[Theorem~5]{alggeo2}, hence $f$ is finite.
	
	Part \itememph{b} is hard and occupies most of Subsection~\reff{subsec:discussionOfThm3}. Part \itememph{c} is easier, but postponed as well until we have the required theory available.
\end{proof}
To warm up for Theorem~\reff{thm:ZariskiMain}\itememph{b} and \itememph{c}, we're going to prove more of Zariski's theorems!
\begin{thm}[Zariski's connectedness theorem]\lbl{thm:ZariskiConnectedness}
	Let $f\colon X\morphism Y$ be a proper morphism between locally noetherian schemes, whose algebraic component $f^*\colon \Oo_Y\morphism f_*\Oo_X$ is an isomorphism.
	\begin{alphanumerate}
		\item The fibres $f^{-1}\{y\}$ are connected for all $y\in Y$.
		\item The set
		\begin{align*}
			U=\left\{x\in X\st \{x\}=f^{-1}\{f(x)\}\right\}=\left\{x\in X\st f\text{ is quasi-finite at }x\right\}
		\end{align*}
		is open in $X$, and the restriction $f|_U$ is quasi-finite (by definition).
	\end{alphanumerate}
\end{thm}
\begin{proof}
	Part \itememph{a}. Assume $f^{-1}\{y\}$ is not connected, say, $f^{-1}\{y\}=U_1\cup U_2$ for disjoint non-empty open subsets $U_1,U_2\subseteq f^{-1}\{y\}$. Since all infinitesimal thickenings $X_n=X\times_Y\Spec(\Oo_{Y,y}/\mm_{Y,y}^n)$ have underlying topological space $f^{-1}\{y\}$, there is a unique $\epsilon_n\in\Oo_{X_n}(X_n)=H^0(X_n,\Oo_{X_n})$ such that $\epsilon_n|_{U_1}=0$ and $\epsilon_n|_{U_2}=1$. The sequence $(\epsilon_n)_{n\in\IN}$ clearly defines an element $\epsilon$ of
	\begin{align*}
		\limit[n\geq 1]H^0(X_n,\Oo_{X_n})\cong (f_*\Oo_X)_y^\complete\cong \roof{\Oo}_{Y,y}\;.
	\end{align*}
	The left isomorphism here is due to Theorem~\reff{thm:FormalFunctions2} and the fact that $\Oo_{X_n}=\Oo_X|_{X_n}$, and the right one holds by assumption. Hence $\roof{\Oo}_{Y,y}$ is a local ring (by Corollary~\reff{cor:completionLocal}) with an idempotent $\epsilon\neq 0,1$. Then $1-\epsilon\neq 0,1$ is another non-trivial idempotent. Both $\epsilon$ and $1-\epsilon$ can't be units in $\roof{\Oo}_{Y,y}$, otherwise $\epsilon^2=\epsilon$ implies $\epsilon=1$ (and similar for $1-\epsilon$). But then they are elements of the maximal ideal $\mm$, so $\epsilon+(1-\epsilon)=1$ is an element of $\mm$ as well, contradiction!
	
	Part \itememph{b}. By \itememph{a}, any point $x\in X$ is open and closed in its fibre iff $f^{-1}\{f(x)\}=\{x\}$. Therefore the two definitions of $U$ indeed coincide. 
	
	We must show that $U$ is open. This is a local question with respect to $Y$, hence we may assume that $Y=\Spec A$ is affine. Let $x\in U$ and $V\subseteq X$ an affine open neighbourhood of $x$. Put $Z=X\setminus V$. Then $Z\subseteq X$ is closed and disjoint from $f^{-1}\{f(x)\}=\{x\}$. As $f$ is proper, $Z'=f(Z)\subseteq Y$ is closed, and $y=f(x)\notin Z'$. There's an $\alpha\in A$ such that $y\notin V(\alpha)$ and $V(\alpha)\supseteq Z'$. Let $Y_1=Y\setminus V(\alpha)$. Note that $Y_1\cong \Spec A_\alpha$ is affine and $x\in X_1=f^{-1}(Y_1)\subseteq V$. Then $X_1=X\setminus V(f^*\alpha)=V\setminus V(f^*\alpha)$ is affine as well, so the restriction $f|_{X_1}\colon X_1\morphism Y_1$ of $f$ is affine and proper. But every affine proper morphism is finite (because $f|_{X_1,*}\Oo_{X_1}$ is a coherent $\Oo_{Y_1}$-module by \cite[Theorem~5]{alggeo2}), so $f|_{X_1}$ is, in particular, quasi-finite by Fact~\reff{fact:annoyingQF} and $U\cap X_1=X_1$. This proves that $U$ is open.
\end{proof}
\begin{rem}
	On first glance, the argument from Theorem~\reff{thm:ZariskiConnectedness}\itememph{b} might look like it proves that every proper morphism is affine, but what it actually shows is the following: If $f\colon X\morphism Y$ is a proper morphism such that for each $x\in X$ the fibre $f^{-1}\{f(x)\}$ is contained in some affine subset $V\subseteq X$, then $f$ is already affine (and hence finite). 	
\end{rem}
\begin{rem}
	Recall that a prescheme $X$ is called \defemph{normal} if it is integral and all local rings $\Oo_{X,x}$ (which are domains if $X$ is integral) are normal (cf.\ \cite[Definition~2.4.5]{alggeo1}). This is the case iff $\Oo_X(U)$ is a normal domain for all affine $U\subseteq X$, cf.\ the discussion in \cite[Remark~2.5.1]{alggeo1}.
\end{rem}
\begin{cor}[Zariski's birationality theorem]\lbl{cor:ZariskiBirationality}
	Let $f\colon X\morphism Y$ be a proper morphism between locally noetherian preschemes, where $Y$ is normal. Suppose that $f$ is \defemph{birational} in the sense that there is a dense open subset $U\subseteq Y$ such that the restriction $f|_{f^{-1}(U)}\colon f^{-1}(U)\isomorphism U$ is an isomorphism and $f^{-1}(U)$ is dense in $X$. Then all assertions from Theorem~\reff{thm:ZariskiConnectedness} apply to $f$. In particular, $f$ has connected fibres.
\end{cor}
\begin{proof}
	First note that $U$ is irreducible as an open subset of the irreducible space $Y$ (irreducibility of $Y$ is implied by $Y$ being normal). Hence $X$ is irreducible because it has the dense irreducible subset $f^{-1}(U)\cong U$. Let $\Spec A\cong V\subseteq Y$ be an affine open subset, where $A$ is a domain. Then $f^{-1}(V)$ is open in $X$, hence dense in $X$ and thus irreducible. Since $U$ is dense in $Y$, the intersection $U\cap V$ is non-empty, hence $f^{-1}(U\cap V)\subseteq f^{-1}(V)$ is a non-empty open subset and thereby dense again. This shows that we can actually reduce to the case $Y=\Spec A$ (all the other involved properties are clearly base-local). Moreover, we may assume that $X$ is integral. Indeed, the assertions from Theorem~\reff{thm:ZariskiConnectedness} are purely topological, so we may replace $X$ by its reduction $X^\red=V(\nil(\Oo_X))$ to obtain an $X$ which is irreducible and reduced (hence integral) and has the same underlying topological space as the original one.
	\begin{claim}\lbl{claim:quotientField}
		The ring $B=\Oo_X(X)$ is a domain in the above situation, and $A$ and $B$ have the same field of quotients $K$. Moreover, we have $A\subseteq B$ as subrings of $K$.
	\end{claim}
	Believing this for the moment, the proof can be finished as follows. Since $B$ is finitely generated as an $A$-module (because $f_*\Oo_X=\snake{B}$ is coherent by \cite[Theorem~5]{alggeo2}), it is integral over $A$. But $A$ is integrally closed in $K$, hence $A\subseteq B$ implies $A=B$. We conclude $f_*\Oo_X\cong \Oo_Y$, as needed.
	
	Unfortunately, the proof of Claim~\reff{claim:quotientField} wasn't discussed in the lecture, but I think it should have been. Since $X$ and $Y$ are irreducible, they have unique generic points $\eta_X$ and $\eta_Y$. As $\eta_Y$ is dense in $Y$, we have $\eta_Y\in U$ and similarly $\eta_X\in f^{-1}(U)$. Hence $f(\eta_X)=\eta_Y$ and the induced morphism $\Oo_{Y,\eta_Y}\isomorphism\Oo_{X,\eta_X}$ is an isomorphism by the birationality assumption. Moreover, $\eta_Y$ corresponds to $0\in\Spec A$, hence $\Oo_{Y,\eta_Y}\cong K$ is the quotient field of $A$. So we should prove that $\Oo_{X,\eta_X}$ is the quotient field of $B=\Oo_X(X)$ as well.
	
	It's clear that $B$ is a domain because $X$ is integral (cf. \cite[Proposition~2.1.4\itememph{b}]{alggeo1}). Since $U\subseteq \Spec A$ is open, we find an affine open subset $V=\Spec A\setminus V(\alpha)\subseteq U$. Then 
	\begin{align*}
		f^{-1}(V)=X\setminus V(f^*\alpha)=f^{-1}(U)\setminus V(f^*\alpha)\cong V
	\end{align*}
	is affine again by birationality of $f$. We know that $X$ is quasi-compact and separated since so are $f$ and $\Spec A$. In particular, \cite[Proposition~1.5.1\itememph{c}]{alggeo1} is applicable to $\Oo_X$ and gives 
	\begin{align*}
		\Oo_X(f^{-1}(V))=\Oo_X\big(X\setminus V(f^*\alpha)\big)\cong\Oo_X(X)_{f^*\alpha}\;,
	\end{align*}
	so $B$ and $\Oo_X(f^{-1}(V))$ have the same quotient field. But $\Oo_X(f^{-1}(V))\cong \Oo_Y(V)\cong A_\alpha$ has quotient field $K$, so we win.
	
	The fact that $A\subseteq B$ as subrings of $K$ follows from the commutative diagram
	\begin{diagram*}
		\object{2,0}{$\Oo_{X,\eta_X}$}[a];
		\object{0,0}{$\Oo_{Y,\eta_Y}$}[b];
		\object{2,1.5}{$B$}[c];
		\object{0,1.5}{$A$}[d];
		\scriptsize
		\arrow dc;
		\arrow[right hook->]ca;
		\arrow[right hook->]db;
		\isoarrow ba;
	\end{diagram*}
	in which every arrow except the top one is injective, hence $A\morphism B$ is injective as well.
\end{proof}
\begin{fact}\lbl{fact:steinFactorization}
	Every proper morphism $f\colon X\morphism Y$ between locally noetherian preschemes can be factorized via the \defemph{Stein factorization} as
	\begin{align*}
		f\colon X\morphism[\snake{f}]\SPEC(f_*\Oo_X)\morphism[g]Y\;.
	\end{align*}
	In this composition, $g$ is finite and the assumptions of Theorem~\reff{thm:ZariskiConnectedness} hold for $\snake{f}$.
\end{fact}
\begin{proof}[Sketch of a proof]
	It's pretty obvious that this factorization exists (to construct $\snake{f}$, use the adjunction from \cite[Proposition~1.6.2\itememph{b}]{alggeo2}). To show that $\snake{f}$ and $g$ have the required properties, we look at things locally and assume that $Y=\Spec A$ is affine (and $A$ is noetherian). Then the facorization looks like
	\begin{align*}
		f\colon X\morphism[\snake{f}]\Spec \Oo_X(X)\morphism[g]\Spec A\;,
	\end{align*}
	so $g$ is affine. Moreover, $\Oo_X(X)$ is a finitely generated $A$-module because $f_*\Oo_X$ is a coherent $\Oo_Y$-module (by \cite[Theorem~5]{alggeo2}, as usual), so $g$ is actually finite. Also proving that $\snake{f}_*\Oo_X=\Oo_{\Spec \Oo_X(X)}$ is straightforward, so it remains to show that $\snake{f}$ is proper. But $g$ is finite, hence separated, and $g\snake{f}=f$ is proper, so $\snake{f}$ is proper as well by \cite[Proposition~2.4.1]{alggeo2}.
\end{proof}
\subsection{Proof of Zariski's main theorem}
\label{subsec:discussionOfThm3}
\begin{proof}[Proof of Theorem~\reff{thm:ZariskiMain}\itememph{c}]
	Let's assume that $f\colon X\morphism Y$ factors over
	\begin{align}\lbl{eq:properCompactification}
		f\colon X\monomorphism[j]\ov{X}\morphism[\smash{\ov{f}}]Y\;,
	\end{align}
	where $j$ is an open immersion and $\ov{f}$ is proper. Then
	\begin{align}\lbl{eq:fandfbar}
		\left\{x\in X\st f\text{ is quasi-finite at }x\right\}=X\cap \left\{x\in \ov{X}\st \ov{f}\text{ is quasi-finite at }x\right\}\;.
	\end{align}
	Indeed, a point $x\in X$ is open in $\ov{f}^{-1}\{y\}$ (where $y=f(x)$) iff it is open in the open subset $f^{-1}\{y\}=X\cap \ov{f}^{-1}\{y\}\subseteq \ov{f}^{-1}\{y\}$. In view of Fact~\reff{fact:openDiscrete} this shows \eqreff{eq:fandfbar}. We thus have reduced \itememph{c} (under the assumption that $\ov{f}$ exists) to the case of proper morphisms. 
	
	If $f\colon X\morphism Y$ is proper, then consider its Stein factorization. Since $g$ is finite, it's quasi-finite as well by Fact~\reff{fact:annoyingQF}\itememph{a}. So Fact~\reff{fact:annoyingQF}\itememph{c} shows that
	\begin{align*}
		\left\{x\in X\st f\text{ is quasi-finite at }x\right\}=\left\{x\in X\st \snake{f}\text{ is quasi-finite at }x\right\}\;.
	\end{align*}
	But the right-hand side is open in $X$ by Fact~\reff{fact:steinFactorization} and Theorem~\reff{thm:ZariskiConnectedness}\itememph{b} and we're happy!
	
	Note that such an $\ov{f}$ always exists when $X$ and $Y$ are affine. Indeed, if $X$ has finite type over $Y$ and both are affine, we get a closed embedding $X\monomorphism\IA_Y^n$ for some $n\in \IN$. Together with the open embedding $\IA_Y^n\monomorphism\IP_Y^n$ this makes $X$ a closed subprescheme of an open subprescheme of $\IP_Y^n$. But then $X$ is also an open subprescheme of some closed subprescheme $\ov{X}\subseteq \IP_Y^n$. This gives a factorization
	\begin{align*}
		f\colon X\monomorphism[j]\ov{X}\morphism[\smash{\ov{f}}]Y
	\end{align*}
	in which $\ov{f}\colon \ov{X}\monomorphism \IP_Y^n\morphism Y$ is (strongly) projective, hence proper by \cite[Proposition~2.4.2]{alggeo2}. But \itememph{c} is completely local on both $X$ and $Y$ (thanks to Fact~\reff{fact:openDiscrete}), so by checking the affine case we have actually covered all of \itememph{c}. 
\end{proof}
	We will now sketch a proof of Theorem~\reff{thm:ZariskiMain}\itememph{b}, which occupies the rest of the subsection. The proof is split into three parts. In Part~I, we will introduce the notion of quasi-affine morphisms and reduce Zariski's main theorem to the question whether quasi-finite separated morphisms are quasi-affine. In Part~II, we show that Zariski's main theorem holds when $Y$ is the spectrum of a complete noetherian local ring. Finally, Part~III shows how the general case can be reduced to Part~II.
	
	So much for the battle plan, now let's get into action!
	
	\paragraph{Part~I -- proving quasi-affinity is sufficient.} We begin with the observation that things become much easier when we are proper.
	\begin{lem}\lbl{lem:ZMTLemma1}
		Let $f\colon X\morphism Y$ be a quasi-finite separated morphism between noetherian schemes, and assume that $f$ factors as in \eqreff{eq:properCompactification}. Then Zariski's main theorem holds for $f$.
	\end{lem}
\begin{rem}
	 It can be shown that such a factorization \eqreff{eq:properCompactification} always exists for morphisms of finite type between noetherian schemes, for which Professor Franke refers to notes of \emph{Brian Conrad} or \emph{Paul Vojta}, although he isn't sure whether using their results to prove Theorem~\reff{thm:ZariskiMain}\itememph{b} doesn't involve any circular reasoning.
\end{rem}
\begin{proof}[Proof of Lemma~\reff{lem:ZMTLemma1}]
	Let $\ov{f}=g\circ \snake{f}$ be the Stein factorization of $\ov{f}$. Put $\ov{Y}=\SPEC(\ov{f}_*\Oo_{\ov{X}})$ for convenience. Since $g\colon \ov{Y}\morphism Y$ is finite by Fact~\reff{fact:steinFactorization} (for which we need properness of $\ov{f}$), we're done if we show that the composition $\snake{f}|_X\colon X\monomorphism \ov{X}\morphism \ov{Y}$ is an open embedding. Since $f=g\snake{f}|_X$ and $g$ are quasi-finite ($g$ is even finite), Fact~\reff{fact:annoyingQF}\itememph{c} shows that $\snake{f}|_X$ is quasi-finite as well.
	
	In particular $\snake{f}|_X$ is injective, and for all $x\in X$ we have $\snake{f}^{-1}\{\snake{f}(x)\}=\{x\}$, because $\Oo_{\ov{Y}}\cong \snake{f}_*\Oo_{\ov{X}}$ (by Fact~\reff{fact:steinFactorization}), so the argument from Theorem~\reff{thm:ZariskiConnectedness}\itememph{b} can be applied. If $V$ is any open neighbourhood of $x\in X$, then $\snake{f}(\ov{X}\setminus V)\subseteq \ov{Y}$ is closed because $\snake{f}$ is proper (by Fact~\reff{fact:steinFactorization}), hence closed. Moreover, $\snake{f}(\ov{X}\setminus V)$ doesn't contain $\snake{f}(x)$ as $\snake{f}^{-1}\{\snake{f}(x)\}=\{x\}$. Thus, the complement $U$ of $\snake{f}(\ov{X}\setminus V)$ in $\ov{Y}$ is open and $\snake{f}^{-1}(U)\subseteq V$ is an open neighbourhood of $x$. This shows that $\snake{f}|_X$ is actually an open map! Together with $\Oo_{\ov{Y}}\cong \snake{f}_*\Oo_{\ov{X}}$ we see that $\snake{f}|_X$ is an open embedding, as claimed.
\end{proof}
\begin{defi}\lbl{def:quasiAffine}
	A morphism $f\colon X\morphism Y$ of preschemes is called \defemph{quasi-affine}, if it satisfies the following equivalent conditions:
	\begin{alphanumerate}
		\item For all affine open $U\subseteq Y$, $f^{-1}(U)$ is a quasi-affine scheme (that is, a quasi-compact open subscheme of an affine scheme).
		\item $Y$ can be covered by affine open $U$ such that $f^{-1}(U)$ is quasi-affine.
		\item $f$ factors as $X\monomorphism[j]\ov{X}\morphism[\smash{\ov{f}}]Y$, where $j$ is an open embedding and $\ov{f}$ is affine.
		\item The canonical morphism $X\morphism\SPEC(f_*\Oo_X)$ in the Stein factorization is an open immersion.
	\end{alphanumerate}
\end{defi}
For a proof of equivalence, check out \cite[\stackstag{01SJ}]{stacks-project}. Note that Franke apparently had a different proof (for the case of quasi-affine morphisms of finite type) in mind than the Stacks Project guys, but I have no idea how this was going to work (whereas the Stacks Project proof is pretty clear to me and works without restrictions on $f$). Nevertheless, the following lemma he suggested is perhaps worthwhile to know.
\begin{lem}
	Let $X$ be a noetherian prescheme, $\Mm$ a quasi-coherent $\Oo_X$-module, and $U\subseteq X$ an open subset. Let $\Nn\subseteq \Mm|_U$ be a coherent submodule. Then there is a coherent $\snake{\Nn}\subseteq \Mm$ such that $\snake{\Nn}|_U=\Nn$. In particular, $\Mm$ is the union of its coherent submodules.
\end{lem}
\begin{proof}
	We proceed by noetherian induction. Because $Y$ is noetherian, there is a $\subseteq$-maximal open subset $U$ to which $\Nn$ extends (and, with slight abuse of notation, we denote some fixed extension to $U$ by $\Nn$ as well). Assuming $U\neq X$, we will derive a contradiction. Pick $x\in X\setminus U$ and let $V\cong \Spec A$ be an affine open neighbourhood of $x$. 
	\begin{claim*}
		There is a coherent $\Qq\subseteq \Mm|_V$ such that $\Qq|_{U\cap V}\supseteq \Nn|_{U\cap V}$.
	\end{claim*}
	Indeed, $V\cong \Spec A$ is a noetherian topological space (because $Y$ is noetherian), so the open subset $U\cap V$ is quasi-compact. This means that we can cover it by finitely many affine open subsets $V_i\cong \Spec A_{f_i}$ for $i=1,\ldots,n$. For every $i$ we know that $\Nn(V_i)\subseteq \Mm(V_i)\cong \Mm(V)_{f_i}$ is a finitely generated $A_{f_i}$-module (because $\Nn$ is coherent), so we can choose finitely many $\mu_{i,j}\in \Mm(V)$, $j=1,\ldots,k_i$, whose images in $\Nn(V_i)$ form a set of generators. Let $\Qq\subseteq \Mm|_V$ be the subsheaf generated by $\left\{\mu_{i,j}\st i\leq n,\ j\leq k_i\right\}\in\Mm(V)$. Then $\Qq$ has the required property, proving the claim.
	
	Let $j\colon U\cap V\monomorphism V$ be the obvious inclusion and put $\ov{\Nn}=\Qq\cap j_*\Nn|_{U\cap V}$. This guy is coherent (because subsheaves of a coherent one are coherent again on locally noetherian preschemes) and satisfies $\ov{\Nn}|_{U\cap V}=\Nn|_{U\cap V}$. But then $\Nn$ can be extended to a coherent sheaf $\snake{\Nn}$ on $U\cup V$ via $\snake{\Nn}|_U=\Nn$ and $\Nn|_V=\Kk$. This contradicts maximality of $U$.
\end{proof}
\begin{lem}\lbl{lem:ZMTLemma2}
	If $f\colon X\morphism Y$ is a quasi-finite and quasi-affine morphism between noetherian preschemes, then Theorem~\reff{thm:ZariskiMain}\itememph{b} holds for $f$.
\end{lem}
\begin{proof}
	Note that quasi-affine morphisms are automatically separated. We will show that $g\colon \SPEC (f_*\Oo_X)\morphism Y$ is proper, which shows that $f$ factors as in \eqreff{eq:properCompactification}. Then Theorem~\reff{thm:ZariskiMain}\itememph{b} holds by Lemma~\reff{lem:ZMTLemma1}. To show properness of $g$, we define
	\begin{align*}
		\Rr=\bigoplus_{n\geq 0}f_*\Oo_X\;,
	\end{align*}
	viewed as a graded quasi-coherent $\Oo_Y$-algebra. Note that $\SPEC (f_*\Oo_X)\cong \PROJ (\Rr)$ as $Y$-preschemes (indeed, locally this reduces to \cite[Example~2.6.2]{alggeo1}). Moreover, $\Rr$ is locally of finite type over $\Oo_Y$. In fact, the $0\ordinalth$ homogeneous component $\Rr_0\cong f_*\Oo_X$ is locally of finite type over $\Oo_Y$ (because $f$ is of finite type), and $\Rr$ is generated by $\Rr_0$ and the element $1\in\Rr_1(Y)$ (which acts as a shift). Therefore, $\PROJ (\Rr)\morphism Y$ is proper by \cite[Proposition~2.4.2]{alggeo2}, whence we're done.
\end{proof}
So we see that to prove Zariski's main theorem it is sufficient to show that any quasi-finite and separated morphism between noetherian preschemes is quasi-affine (note that this is actually \emph{weaker} than Theorem~\reff{thm:ZariskiMain}\itememph{b}).

\paragraph{Part II -- the case of complete noetherian local rings.} In this part Professor Franke follows \cite[Exposé~VIII.6]{sga1}. Assume $Y=\Spec A$, where $A$ is a noetherian complete local ring (with respect to its maximal ideal $\mm$) and that Zariski's main theorem is true for morphisms $f'\colon X'\morphism Y'$, where in addition to the other assumptions we have $\dim Y'<\dim Y$. Let $s$ be the unique closed point of $A$ (given by $\mm$).
\begin{lem}\lbl{lem:connectedComponents}
	If $B$ is any finite $A$-algebra, then $\Spec B$ has a decomposition $\Spec B=\bigcup_{i=1}^nU_i$, where the $U_i$ are disjoint open subsets such that the only closed point $s$ of $\Spec A$ has precisely one preimage in each $U_i$.
\end{lem}
\begin{proof}
	By Corollary~\reff{cor:finiteAlgebras}, $B$ has finitely many maximal ideals $\qq_1,\ldots,\qq_n$ and these are precisely the prime ideals over $\mm$. For every $\qq_i$ we will construct an idempotent $e_i\in B$ such that $e_i\in \qq_i$ but $e_i\notin\qq_j$ for $j\neq i$. Then $\qq_i\in V(e_i)$ but $\qq_j\neq V(e_i)$ for $j\neq i$, and $V(e_i)$ is an open and closed subset of $B$ (because $e_i$ is an idempotent), so we see that the $\qq_i$ are contained in distinct connected components of $\Spec B$. But $B$ is noetherian, so $\Spec B$ has finitely many connected components (as pointed out in \cite[Lemma~2.4.2]{alggeo1}). Then every connected component is open and we can construct the required $U_i$ as unions of suitable connected components of $\Spec B$.
	
	Let $\ov{\qq}_i=\qq_i/\mm$ be the prime ideals of $B/\mm B$ (which are automatically maximal). Then the intersection $\bigcap_{j=1}^n\qq_j$ is the nilradical $\nil(B/\mm B)$, so by the Chinese remainder theorem we find an element $\ov{e}_i\in (B/\mm B)/\nil(B/\mm B)$ such that $\ov{e}_i\equiv 0\bmod \ov{\qq}_i$ and $\ov{e}_j\equiv 1\bmod\ov{\qq}_j$ for $j\neq i$. Note that $\ov{e}_i^2=\ov{e}_i$. Since $B/\mm B$ is noetherian, there is an $N\in\IN$ such that $\nil(B/\mm B)^N=0$. Hence $B/\mm B$ is $\nil(B/\mm B)$-adically complete, which means we can lift $\ov{e}_i$ to an idempotent $\snake{e}_i\in B/\mm B$ by Hensel's lemma (cf.\ Proposition~\reff{prop:HenselsLemma}). Now $B$ is $\mm B$-adically complete by Proposition~\reff{prop:modulesComplete}, so using Hensel's lemma once again we can lift $\snake{e}_i$ to an idempotent $e_i\in B$ with the required properties. We're done.
\end{proof}

Let $f\colon X\morphism Y=\Spec A$ be quasi-finite and separated, and let $x\in f^{-1}\{s\}$. Because the fibre $f^{-1}\{s\}$ is discrete, there is an affine open $U\subseteq X$ such that $U\cap f^{-1}\{s\}=\{x\}$. Zariski's main theorem applies to the affine morphism $f|_U\colon U\morphism \Spec A$, which therefore factors as
\begin{align*}
	U\monomorphism[j]\Spec B\morphism Y
\end{align*}
where $B$ is as above and $j$ is an open embedding. By Lemma~\reff{lem:connectedComponents} (and shrinking $U$ if necessary) we may assume that $B$ has only one closed point $j(x)$, i.e.\ is local itself. But then the only open subset of $\Spec B$ containing the closed point $j(x)$ is $\Spec B$ itself, hence $j$ is an isomorphism. We have thus found an affine open neighbourhood $U\cong \Spec B$ of $x$ which is finite over $Y$. Then $U\morphism X$ is proper (as $U\morphism Y$ is finite, hence proper, and $f$ is separated, so \cite[Proposition~2.4.1]{alggeo2} applies), so $U$ is also closed. Putting $U=U_1$ and $X\setminus U=X_1$, what we proved is that $X=X_1\amalg U_1$ is the disjoint union of its open subsets $X_1$ and $U_1$.

Iterating this for the remaining preimages\footnote{There are finitely many of them. Indeed, $f^{-1}\{s\}\subseteq X$ is closed (as the preimage of the closed point $s$), hence quasi-compact (because $X$ is noetherian, hence quasi-compact). But the fibre $f^{-1}\{s\}$ is supposed to be discrete, so it must be finite.} of $s$ in $X$ provides a decomposition 
\begin{align*}
	X=X'\amalg \coprod_{i=1}^nU_i
\end{align*}
into disjoint open subsets, where $\coprod_{i=1}^nU_i\cong \coprod_{i=1}^n\Spec \Oo_X(U_i)\cong \Spec\big(\bigoplus_{i=1}^n\Oo_X(U_i)\big)$ is finite over $Y$ and $f'=f|_{X'}\colon X'\morphism Y$ has image in $Y'=Y\setminus \{s\}$, hence the induction assumption applies to $f'\colon X'\morphism Y'$. This means that $f'$ can be written as $f'\colon X'\monomorphism \ov{X}'\morphism Y'$, where $X'$ is an open subprescheme of $\ov{X}'$, which in turn is finite over $Y'$. Still we aren't done yet, as we need something finite over $Y$ rather than $Y'$.
\begin{lem}
	Let $A$ be a noetherian ring and $S\subseteq \Spec A$ an open subprescheme. If $f\colon X\morphism S$ is a finite morphism, then there is a finite $A$-algebra $B$ such that $X$ is an open subprescheme of $\Spec B$ and the diagram
	\begin{diagram*}
		\object{0,1.5}{$\Spec B$}[a];
		\object{2.5,1.5}{$\Spec A$}[b];
		\object{0,0}{$X$}[c];
		\object{2.5,0}{$S$}[d];
		\scriptsize
		\arrow ab;
		\arrow[right hook->] ca;
		\arrow[right hook->] db;
		\arrow cd[above][f];
	\end{diagram*}
	commutes.
\end{lem}
\begin{proof}
	By Lemma~\reff{lem:ZMTLemma2} it is enough to show that $X\morphism\Spec A$ is quasi-affine (which removes the condition that $B$ is finite over $A$). The Stein factorization of $X\morphism\Spec A$ reads
	\begin{align*}
		X\morphism\Spec \Oo_X(X)\morphism\Spec A\;,
	\end{align*}
	so it is enough to show that the canonical morphism $X\morphism\Spec\Oo_X(X)$ is an open embedding.
	
	Since $\Spec A$ is a noetherian space, its open subset $S$ is quasi-compact. We thus find a finite cover $S=\bigcup_{i=1}^n S_i$, where $S_i\cong \Spec A_{\alpha_i}$ (for some $\alpha_i\in A$) are affine open subschemes. Putting $X_i=f^{-1}(S_i)$ we get an affine open cover $X=\bigcup_{i=1}^nX_i$ of $X$. %, where $X_i\cong \Spec B_i$ for some finite $A_{\alpha_i}$-algebra $B_i$ by finiteness of $f$.
	Note that $X$ is quasi-compact because $S$ and $f$ are. Moreover, $S$ is separated as an open subprescheme of a separated prescheme, hence $X$ is separated as well, as finite morphisms are separated. If we denote the image of $\alpha_i$ under the restriction $A\morphism \Oo_S(S)$ by $\alpha_i$ as well, we see that $S_i=S\setminus V(\alpha_i)$, so $A_{\alpha_i}\cong \Oo_S(S\setminus V(\alpha_i))\cong \Oo_S(S)_{\alpha_i}$. This works because $S$ is quasi-compact and separated, so \cite[Proposition~1.5.1\itememph{c}]{alggeo1} applies. By the same argument applied to $X$, we get $\Oo_X(X_i)=\Oo_X(X\setminus V(f^*\alpha_i))\cong \Oo_X(X)_{f^*\alpha_i}$, so $X_i\cong \Spec \Oo_X(X)_{f^*\alpha_i}$ because the $X_i$ are affine.
	
	Now $X=\bigcup_{i=1}^nX_i\cong \bigcup_{i=1}^n\Spec \Oo_X(X)_{f^*\alpha_i}$ is a union of open subschemes of $\Spec \Oo_X(X)$, hence an open subscheme itself, and we are done.
	%Note that $\Oo_X(X)$ becomes an $A$-algebra via $A\morphism \Oo_S(S)\morphism \Oo_X(X)$. For every $i$ let $x_{i,j}\in B_i$, $j=1,\ldots,k_i$ be generators of $B_i$ as an $A_{\alpha_i}$-module. Multiplying by a suitable power of $f^*\alpha_i$ we can assume that the $x_{i,j}$ are from $\Oo_X(X)$ and integral over $A$ (rather than just integral over $A_{\alpha_i}$). Now let $\Lambda$ be the image of $A$ in $\Oo_X(X)$ and put
	%\begin{align*}
	%	B=\Lambda\left[x_{i,j}\st i\leq n,\ j\leq k_i\right]\;.
	%\end{align*}
	%Then $B$ is finite over $A$ and contains the $f^*\alpha_i$ (because these guys are already in $\Lambda$). Moreover, we have $B_{f^*\alpha_i}=\Oo_X(X)_{f^*\alpha_i}\cong B_i$, because $B$ already contains a set of generators of $B_i$ over $A_{\alpha_i}$. So $X=\bigcup_{i=1}^nX_i$ is a union of open subpreschemes of $\Spec B$, hence an open subprescheme itself and $\Spec B\morphism \Spec A$ clearly has the required properties.
\end{proof}

This finishes the case of complete noetherian local rings.

\paragraph{Part III -- reduction to complete noetherian local rings.} We will show that Zariski's main theorem is valid for (quasi-finite separated) $f\colon X\morphism Y$ with target $Y$ (which is locally noetherian) if for all $y\in Y$ it is valid with $\Spec \roof{\Oo}_{Y,y}$ as target. But before that, some technical preparations need to be done.

Professor Franke intended to use Chevalley's theorem (Proposition~\reff{prop:Chevalley}) to prove Lemma~\reff{lem:OpenEmbedding}, but I don't understand how that's supposed to work. The proof I came up with doesn't need Chevalley's theorem, but it's surely worthwhile to know, so I put it in the appendix (page~\pageref{prop:GenericFreeness}).
\begin{lem}\lbl{lem:OpenEmbedding}
	Let $f\colon X\morphism Y$ be a morphism of locally finite type between locally noetherian preschemes such that $f$ induces an injective map of topological spaces and isomorphisms $f^*\colon \Oo_{Y,f(x)}\isomorphism \Oo_{X,x}$ on stalks. Then $f$ is an open embedding.
\end{lem}
\begin{proof}
	It is sufficient to show that $X$ carries the induced topology and is open in $Y$. All we need to do for this is to prove that $f$ is an open morphism. This will be done by several reduction steps until we arrive at a nice enough situation.
	
	The first of these reductions is that $f$ being an open morphism is a local question both on source and target, so we may assume $X=\Spec B$ and $Y=\Spec A$ to be affine, where $A$ and $B$ are noetherian rings with $B$ of finite type over $A$. Let $Z_1,\ldots,Z_n$ be the irreducible components of $Y$ and let $\pp_i\subseteq A$ be the prime ideals such that $Z_i=V(\pp_i)$. Note that a subset $V\subseteq Y$ is open iff its intersections $V\cap Z_i$ are all open (indeed, in this case $Y\setminus V$ is closed as a finite union of the closed subsets $Z_i\setminus V$). Therefore, it suffices to show that every $x\in X$ has an open neighbourhood $U_i$ such that for all open neighbourhoods $U\subseteq U_i$ of $x$ the intersection $f(U)\cap Z_i$ is open (because then $f(U)\subseteq X$ is an open neighbourhood of $f(x)$ for $U=U_1\cap \ldots\cap U_n$). For those irreducible components $Z_i$ such that $x\notin Z_i$ this is easy -- just take $U_i=f^{-1}(Y\setminus Z_i)$. Now suppose $x\in Z_i$. Since $f(U)\cap Z_i=f\left(U\cap f^{-1}(Z_i)\right)$ it suffices to find such a neighbourhood $U_i$ of $x$ in $f^{-1}(Z_i)$ rather than in $X$. That is, we may replace $A$ by $A/\pp_i$ and $B$ by $B/\pp_i B$ (clearly we still get isomorphisms on stalks), so we may henceforth assume that $A$ is a domain.
	
	Note that $X=\Spec B$ has finitely many connected components (by \cite[Lemma~2.4.2]{alggeo1}). Let $C$ be one of them. Then $C$ is open in $X$, connected, and the local rings $\Oo_{X,x}\cong \Oo_{Y,f(x)}$ are domains for all $x\in X$ (the $\Oo_{Y,y}$ are domains for all $y\in Y$ because $A$ is a domain), so $C$ is an integral prescheme by \cite[Proposition~2.1.4\itememph{d}]{alggeo1}. If we restrict to an affine open subset of $C$ (we can do this because everything is local on $X$) we have reduced the situation to the case where $B$ is a domain as well.
	
	Note that $A\morphism B$ is injective. Indeed, suppose that $a\in A$ is contained in the kernel. Choose any prime ideal $\qq\in\Spec B$ and let $\pp\in\Spec A$ be its preimage in $A$. Then $a$ is contained in the kernel of $A_\pp\morphism B_\qq$ as well, so $a=0$ in $A_\pp$ because $f$ induces isomorphisms on stalks. But $A\morphism A_\pp$ is injective when $A$ is a domain, proving that $a=0$ holds in $A$ as well. Thus, we may regard $A$ as a subring of $B$. Note that both have the same quotient field $K$ because $f$ induces an isomorphism between the stalks at the generic points. Now it's time to remember that $f$ is of locally finite type, so $B$ has finite type over $A$. If $x_1,\ldots,x_m\in B$ are generators over $A$, then we can write $x_i=\frac{a_i}{s_i}$ for $a_i,s_i\in A$, because the $x_i$ are elements of $K$. Putting $s=s_1\cdots s_n$ we deduce that $B=A[s^{-1}]$ is the localization of $A$ at $s$. Then $\Spec B\morphism\Spec A$ is an open embedding, which means we're done, finally.
	%Let $x\in X$. We need to show that some neighbourhood of $f(x)$ is contained in the image of $f$. Otherwise, apply the previous Proposition~\reff{prop:noetherianAffine} (with some care) to show that there is a closed subset of $Y$ containing the image of $f$ and not containing any neighbourhood of $f(x)$. There is $\lambda\in A$ such that $f(X)\subseteq V(\lambda)$. Then $\emptyset =f^{-1}(Y\setminus V(\lambda))=X\setminus V(f^*\lambda)$ implies that $f^*\lambda\in B$ is nilpotent. But then the image of $\lambda$ in $\Oo_{Y,f(x)}$ is nilpotent as well, which means that $V(\lambda)=\Spec A$ contains an open neighbourhood of $f(x)$.
\end{proof}
\begin{proof}[Proof of Theorem~\reff{thm:ZariskiMain}\itememph{b} \textsc{(finally)}]
Now let $f\colon X\morphism Y$ be a quasi-finite separated morphism between noetherian preschemes such that Zariski's main theorem holds with $Y$ replaced by $\Spec \roof{\Oo}_{Y,y}$. Note that $\upsilon\colon \Spec \roof{\Oo}_{Y,y}\morphism Y$ is flat (in the sense of the upcoming Definition~\reff{def:flat}) because of Corollary~\reff{cor:completionFaithfullyFlat}\itememph{b}. Therefore, in the pullback diagram
\begin{diagram}[baseline=0.75cm-0.5ex][\lbl{diag:completionBaseChange}]
	\object{2.5,1.5}{$\Spec \roof{\Oo}_{Y,y}$}[a];
	\object{0,1.5}{$\roof{X}$}[b];
	\object{2.5,0}{$Y$}[c];
	\object{0,0}{$X$}[d];
	\pullback{1.25,0.75};
	\scriptsize
	\arrow ba[above][$\roof{f}$];
	\arrow dc[above][$f$];
	\arrow bd;
	\arrow ac[right][$\upsilon$];
\end{diagram}
we obtain
\begin{align}\lbl{eq:froof*}
(f_*\Oo_X)_y\otimes_{\Oo_{Y,y}}\roof{\Oo}_{Y,y}\cong \roof{f}_*\Oo_{\roof{X}}(\Spec \roof{\Oo}_{Y,y})\;.
\end{align}
Indeed, this is just the base change morphism \eqreff{eq:baseChangeMorphism} from Proposition~\reff{prop:baseChangeMorphism}, applied to $\Ff=\Oo_X$ and $p=0$ -- and also we've taken global sections and we used that $(f_*\Oo_X)(U)\otimes_{\Oo_Y(U)}\roof{\Oo}_{Y,y}\cong (f_*\Oo_X)_y\otimes_{\Oo_{Y,y}}\roof{\Oo}_{Y,y}$ when $U$ is an affine open neighbourhood of $y$ in $Y$.

Note that the local rings $\roof{\Oo}_{Y,y}$ and $\Oo_{Y,y}$ have the same residue field $\KK(y)$. Hence $\Spec \KK(y)\morphism Y$ can be factored through $\upsilon\colon \Spec \roof{\Oo}_{Y,y}\morphism Y$ and together with the Stein factorizations of $f$ and $\roof{f}$ this gives a diagram
\begin{diagram}[baseline=1.5cm-0.5ex][\lbl{diag:PullbacksEverywhere}]
	\object{9,1.5}{$\Spec \roof{\Oo}_{Y,y}$}[Ycomplete];
	\object{0,1.5}{$\roof{X}$}[Xcomplete];
	\object{9,0}{$Y$}[Y];
	\object{0,0}{$X$}[X];
	\object{9,3}{$\Spec \KK(y)$}[Speck];
	\object{0,3}{$X\times_Y\Spec \KK(y)$}[XxSpeck];
	\object{4.5,0}{$\SPEC(f_*\Oo_X)$}[SPEC];
	\object{4.5,1.5}{$\SPEC(\roof{f}_*\Oo_{\roof{X}})$}[SPECcomplete];
	\object{4.5,3}{$\Spec \left((f_*\Oo_X)_y\otimes_{\Oo_{Y,y}}\KK(y)\right)$}[Specfk];
	\pullback{2.25,0.75};
	\pullback{2.25,2.25};
	\pullback{6.75,0.75};
	\pullback{6.75,2.25};
	\scriptsize
	\arrow{X}{SPEC}[above][$j$];
	\arrow{SPEC}{Y}[above][$g$];
	\arrow{Xcomplete}{SPECcomplete}[above][$\roof{\jmath}$];
	\arrow{SPECcomplete}{Ycomplete}[above][$\roof{g}$];
	\arrow{XxSpeck}{Specfk};
	\arrow{Specfk}{Speck};
	\arrow{Xcomplete}{X};
	\arrow{XxSpeck}{Xcomplete};
	\arrow{Ycomplete}{Y}[right][$\upsilon$];
	\arrow{Speck}{Ycomplete};
	\arrow{Specfk}{SPECcomplete};
	\arrow{SPECcomplete}{SPEC};
\end{diagram}
\begin{claim}\lbl{claim:PullbacksEverywhere}
	In the diagram \eqreff{diag:PullbacksEverywhere} every subrectangle (not only those indicated) is a pullback.
\end{claim}
Indeed, it is clear that the whole diagram as well as its bottom half (both can be viewed as subrectangles) are pullbacks, since that's how $X\times_Y\Spec \KK(y)$ and $\roof{X}$ are defined. Moreover, we have the relation $\SPEC \big(\upsilon^*(f_*\Oo_X)\big)\cong \SPEC (f_*\Oo_X)\times_Y\Spec \roof{\Oo}_{Y,y}$ by \cite[Corollary~1.6.2]{alggeo2}, which together with \eqreff{eq:froof*} implies $\SPEC (\roof{f}_*\Oo_{\roof{X}})\cong \SPEC (f_*\Oo_X)\times_Y\Spec \roof{\Oo}_{Y,y}$. This shows that the bottom right rectangle in \eqreff{diag:PullbacksEverywhere} is a pullback. By a similar argument the right half rectangle is a pullback as well. Now it's a matter of abstract nonsense to show that the remaining subrectangles are pullbacks too.

Because quasi-finiteness and separatedness are preserved under base change, $\roof{f}$ satisfies all assumptions from Theorem~\reff{thm:ZariskiMain}\itememph{b} and has target $\Spec \roof{\Oo}_{Y,y}$. Hence Zariski's main theorem may be applied to $\roof{f}$. In particular, this means that $\roof{f}$ is quasi-affine, so $\roof \jmath \colon\roof{X}\morphism\SPEC (\roof{f}_*\Oo_{\roof{X}})$ is an open embedding  by Definition~\reff{def:quasiAffine}\itememph{d}. 

Note that Claim~\reff{claim:PullbacksEverywhere} implies that the fibres $f^{-1}\{y\}$ and $\roof{f}^{-1}\{y\}$ coincide, as do $g^{-1}\{y\}$ and $\roof{g}^{-1}\{y\}$. Moreover we know $\roof\jmath\colon \roof{X}\morphism\SPEC(\roof{f}_*\Oo_{\roof{X}})$ is injective on the fibre $\roof{f}^{-1}\{y\}$ (in fact, it is injective everywhere because we just showed it's an open embedding). By diagram~\eqreff{diag:PullbacksEverywhere} we conclude that
\begin{align*}
	f^{-1}\{y\}=\roof{f}^{-1}\{y\}=X\times_Y\Spec \KK(y)\morphism \Spec \left((f_*\Oo_X)_y\otimes_{\Oo_{Y,y}}\KK(y)\right)=\roof{g}^{-1}\{y\}=g^{-1}\{y\}
\end{align*}
is injective on points. But this means that $j\colon X\morphism \SPEC(f_*\Oo_X)$ is injective on $f^{-1}\{y\}$ -- and since $y$ was chosen arbitrarily, we see that $j$ is actually injective on points! So the only thing left to do is to show that $j$ induces isomorphisms on stalks, then Lemma~\reff{lem:OpenEmbedding} does the rest.
\begin{claim}
	Let $x\in X$ be such that $y=f(x)$. Then $x$ has a unique preimage $\roof{x}\in\roof{X}$ and the local ring $\Oo_{\roof{X},\roof{x}}$ is given by $\Oo_{X,x}\otimes_{\Oo_{Y,y}}\roof{\Oo}_{Y,y}$.
\end{claim}
The easiest way to see that $\roof{x}$ is unique is probably that the fibres $f^{-1}\{y\}$ and $\roof{f}^{-1}\{y\}$ coincide, but it also follows from the fact that $\mm_{X,x}\otimes_{\Oo_{Y,y}}\roof{\Oo}_{Y,y}$ is a maximal ideal of $\Oo_{X,x}\otimes_{\Oo_{Y,y}}\roof{\Oo}_{Y,y}$ (by Proposition~\reff{prop:technicalAF}\itememph{a}) and our explicit description of fibre products in \cite[p.~24]{alggeo1}. By Proposition~\reff{prop:technicalAF}\itememph{b}, the ring $\Oo_{X,x}\otimes_{\Oo_{Y,y}}\roof{\Oo}_{Y,y}$ is already local, hence it is isomorphic to $\Oo_{\roof{X},\roof{x}}$ (again by the explicit construction of fibre products).

In the same way we see that $\Oo_{\SPEC(\roof{f}_*\Oo_{\roof{X}}),\roof{\jmath}(\roof{x})}\cong \Oo_{\SPEC(f_*\Oo_X),j(x)}\otimes_{\Oo_{Y,y}}\roof{\Oo}_{Y,y}$. This means that the morphism on stalks $j^*\colon \Oo_{X,x}\morphism \Oo_{\SPEC(f_*\Oo_X),j(x)}$ becomes an isomorphism after tensoring with $\roof{\Oo}_{Y,y}$, because $\roof{\jmath}$ is an open embedding and induces isomorphisms on stalks. But the completion $\roof{\Oo}_{Y,y}$ is faithfully flat over $\Oo_{Y,y}$ by Corollary~\reff{cor:completionFaithfullyFlat}\itememph{c}, hence $j$ already induces isomorphisms on stalks.

By Lemma~\reff{lem:OpenEmbedding}, $j\colon X\morphism\SPEC(f_*\Oo_X)$ is an open embedding, hence $f$ is quasi-affine by Definition~\reff{def:quasiAffine}\itememph{d}.  Then we're done by Lemma~\reff{lem:ZMTLemma1}.
\end{proof}


\chapter{Flat morphisms}
\begin{defi}\lbl{def:flat}
	Let $f\colon X\morphism Y$ be a morphism of preschemes. A quasi-coherent $\Oo_X$-module $\Ff$ is called \defemph{flat} over $\Oo_Y$ iff it has the following equivalent properties.
	\begin{alphanumerate}
		\item If $U\subseteq X$ and $V\subseteq Y$ are affine opens such that $f(U)\subseteq V$, then $\Ff(U)$ is flat as an $\Oo_Y(V)$-module.
		\item It is possible to cover $X$ by affine open subsets for which such $V$ may be found.
		\item For any $x\in X$ the stalk $\Ff_x$ is a flat $\Oo_{Y,y}$-module, where $y=f(x)$.
	\end{alphanumerate}
	We call $f$ a \defemph{flat morphism} if $\Oo_X$ is flat over $\Oo_Y$, and \defemph{faithfully flat} if $f$ is flat and surjective.
\end{defi}
\begin{proof}[Sketch of a proof of equivalence]
	The implication \itememph{a} $\Rightarrow$ \itememph{b} is trivial. If you think about it, showing \itememph{b} $\Rightarrow$ \itememph{c} $\Rightarrow$ \itememph{a} comes down to the following fact from commutative algebra: Let $B$ be an $A$-algebra and $F$ a $B$-module. Then $F$ is flat over $A$ and only if $F_\qq$ is flat over $A_\pp$ algebra for all primes $\pp\in\Spec A$ and $\qq\in\Spec B$ above $\pp$.
	
	Indeed, if $F$ is flat over $A$, then so is $F_\qq$ (by exactness of localization), hence $F_\qq$ is also flat over $A_\pp$. Conversely, if $F_\qq$ is flat over $A_\pp$, then also over $A$ because $A_\pp$ is flat over $A$ (exactness of localization again). Now let $M'\morphism M\morphism M''$ be an exact sequence of $A$-modules. Then the sequence of $B$-modules $M'\otimes_AF\morphism M\otimes_AF\morphism M''\otimes_AF$ is exact iff its localization at any prime $\qq\in \Spec B$ is exact. But $(M\otimes_AF)_\qq\cong M\otimes_AF_\qq$ (and same for $M'$, $M''$), so the localizations are indeed exact by flatness of the $F_\qq$, proving that $F$ is flat itself.
\end{proof}
\section{Flat base change and cohomology}
\subsection{The base change morphism}
Let $A$ be a ring, $B$ an $A$-algebra and $f\colon X\morphism \Spec A$ a quasi-compact and separated morphism. Let 
\begin{align*}
	\snake{f}\colon\snake{X}=X\times_{\Spec A}\Spec B\morphism\Spec B%.
\end{align*}
its base change along $\Spec B\morphism\Spec A$. Also let $\xi\colon \snake{X}\morphism X$ be projection to the other fibre product factor. We want to investigate the relation between the cohomology of $X$ and $\snake{X}$. For a quasi-coherent $\Oo_X$-module $\Ff$ and an affine \v Cech cover $\Uu\colon X=\bigcup_{i\in I}U_i$ of $X$, the \v Cech complex $\check{C}^\bullet(\Uu,\Ff)$ calculates $H^\bullet(X,\Ff)$. Pulling back $\Uu$ gives a \v Cech cover
\begin{align*}
	\xi^{-1}\Uu\colon \snake{X}=\bigcup_{i\in I}U_i\times_{\Spec A}\Spec B\;,
\end{align*}
of $\snake{X}$, whose components $U_i\times_{\Spec A}\Spec B$ are affine again (as fibre products of affine schemes), hence $\xi^{-1}\Uu$ may be used to compute $H^\bullet(\snake{X},\xi^*\Ff)$. Explicitly, we obtain
\begin{align}\lbl{eq:tensoredCechComplex}
	\check{C}^\bullet\big(\xi^{-1}\Uu,\xi^*\Ff\big)\cong \check{C}^\bullet(\Uu,\Ff)\otimes_AB\;.
\end{align}
This gives a canonical morphism
\begin{align}\lbl{eq:tensoredCohomology}
	H^\bullet(X,\Ff)\otimes_AB\morphism H^\bullet(\snake{X},\xi^*\Ff)\;,
\end{align}
which clearly is an isomorphism if $B$ is flat over $A$.
\begin{prop}\lbl{prop:baseChangeMorphism}
	Consider a cartesian diagram
	\begin{diagram*}
		\object{0,0}{$\snake{Y}$}[a];
		\object{0,1.5}{$\snake{X}$}[b];
		\object{2,0}{$Y$}[c];
		\object{2,1.5}{$X$}[d];
		\pullback{1,0.75};
		\scriptsize
		\arrow ba[left][$\snake{f}$];
		\arrow dc[right][$f$];
		\arrow bd[above][$\xi$];
		\arrow ac[above][$\upsilon$];
	\end{diagram*}
	in which $f$ is quasi-compact and separated, and let $\Ff$ be a quasi-coherent $\Oo_X$-module. Then we get a morphism of $\Oo_{\snake{Y}}$-modules, called the \defemph{base change morphism},
	\begin{align}\lbl{eq:baseChangeMorphism}
		\upsilon^*R^pf_*\Ff\morphism R^p\snake{f}_*(\xi^*\Ff)
	\end{align}
	which is an isomorphism when $\upsilon$ is flat in the sense of Definition~\reff{def:flat}.
\end{prop}
\begin{proof}
	We may check \eqreff{eq:baseChangeMorphism} locally, where it is just \eqreff{eq:tensoredCohomology}.
\end{proof}
\subsection{The base change spectral sequence}
\begin{varthm}{prop}\lbl{prop:baseChangeMorphismA}
	Suppose that $Y=\Spec A$ and $\snake{Y}=\Spec B$ are affine in the situation from Proposition~\reff{prop:baseChangeMorphism}. If $\Ff$ is flat over $\Oo_Y$ in the sense of Definition~\reff{def:flat}, then the base change morphism \eqreff{eq:baseChangeMorphism} is part of a spectral sequence
	\begin{align*}
		E_2^{p,q}=\Tor_{-p}^A\big(B,H^q(X,\Ff)\big)\converge H^{p+q}\left(\snake{X},\xi^*\Ff\right)\;.
	\end{align*}
\end{varthm}
\begin{rem}
The minus sign in $\Tor_{-p}^A$ is not a typo. This spectral sequence is supported in the $2\ordinalnd$ quadrant and we interpret cohomology or $\Tor$ in negative degrees to be zero.
\end{rem}
\begin{proof}[Proof of Proposition~\reff{prop:baseChangeMorphismA}]
	Let $J^{\bullet,\bullet}$ be a Cartan--Eilenberg resolution of the alternating \v Cech complex $\check{C}_\alt^\bullet(\Uu,\Ff)$, where $\Uu$ is an arbitrary finite affine open cover of $X$ (which is quasi-compact, so this is fine). That is, $J^{\bullet,\bullet}$ is a $2\ordinalnd$ quadrant double complex of $A$-modules such that
	\begin{itemize}
		\item $J^{\bullet,q}\morphism\check{C}_\alt^q(\Uu,\Ff)$ is a projective resolution for all $q\geq 0$ (note that $\bullet$ ranges through all \emph{non-positive} integers here, because we want $J^{\bullet,q}$ to be a cochain complex).
		\item The vertical morphisms $d_\ver^{\bullet,q}\colon J^{\bullet,q}\morphism J^{\bullet,q+1}$ are \emph{split} in the sense that there is a decomposition $J^{\bullet,q}\cong B^{\bullet,q}\oplus H^{\bullet,q}\oplus B^{\bullet,q+1}$ and $d_\ver^{\bullet,q}$ corresponds to the projection onto the summand $B^{\bullet,q+1}$.
		\item The induced cochain complex $H_\ver^q(J^{\bullet,\bullet})\cong H^{\bullet,q}\morphism \check{H}_\alt^q(\Uu,\Ff)\cong H^q(X,\Ff)$ is a projective resolution as well.
	\end{itemize}
It's well-known that Cartan--Eilenberg resolutions exist in the category $\cat{Mod}(A)$ (in fact, in any abelian category with sufficiently many projectives). Moreover, since $\check{C}^q_\alt(\Uu,\Ff)=0$ when $q$ is sufficiently large, say $q\geq q_0$, we may choose $J ^{\bullet,\bullet}$ such that $J^{\bullet,q}=0$ when $q\geq q_0$. Then $J^{\bullet,\bullet}$ is bounded from above in both directions (that is, a $3\ordinalrd$ quadrant double complex, except it's supported completely in the $2\ordinalnd$ quadrant -- you know what I mean).

The same is true for the double complex $J^{\bullet,\bullet}\otimes_AB$, hence both its horizontal and its vertical spectral sequence converge -- and even to the same limit. Note that $H_\ver^q(J^{\bullet,\bullet}\otimes_AB)\cong H^{\bullet,q}\otimes_AB$ because of the nice splitness condition for $d_\ver^{\bullet,q}$. We thus have 
\begin{align*}
	\vphantom{E}^\ver E_2^{p,q}=H^p_\hor\big(H^q_\ver(J^{\bullet,\bullet}\otimes_AB)\big)\cong H^p(H^{\bullet,q}\otimes_AB)\cong \Tor_{-p}^A\big(B,H^q(X,\Ff)\big)\;,
\end{align*}
since $H^{\bullet,q}$ is a (negatively indexed) projective resolution of $H^q(X,\Ff)$. Similarly
\begin{align*}
	\vphantom{E}^\hor E_2^{p,q}=H^q_\ver\big(H_\hor^p(J^{\bullet,\bullet}\otimes_AB)\big)=H^q\left(\Tor_{-p}^A\big(B,\check{C}_\alt^\bullet(\Uu,\Ff)\big)\right)
\end{align*}
since $J^{\bullet,q}$ is a projective resolution of $\check{C}^q(\Uu,\Ff)$. But $\check{C}_\alt^q(\Uu,\Ff)=\prod_{i_0<\ldots<i_q}\Ff(U_{i_0,\ldots,i_q})$ is a product of flat $A$-modules because $\Ff$ is flat over $\Oo_Y$, hence $\check{C}_\alt^q(\Uu,\Ff)$ is a flat $A$-module itself (by \cite[Example~1.2.6]{homalg} -- actually, the products occuring here are finite, so it's a lot easier to see that flatness is preserved). Hence
\begin{align*}
	\Tor_{-p}^A\big(B,\check{C}_\alt^q(\Uu,\Ff)\big)\cong
	\begin{cases}
	0 & \text{if }p<0\\
	\check{C}_\alt^q(\Uu,\Ff)\otimes_AB & \text{if }p=0
	\end{cases}\;.
\end{align*}
By \eqreff{eq:tensoredCechComplex} we get $\vphantom{E}^\hor E^{0,q}\cong H^q(\snake{X},\xi^*\Ff)$ and $\vphantom{E}^\hor E^{p,q}=0$ if $p<0$. Therefore the horizontal spectral sequence immediately degenerates, and its limit is $H^{p+q}(\snake{X},\xi^*\Ff)$. That's precisely what we need.
\end{proof}
\section{The Grauert--Grothendieck theorem}
Before we state the theorem, let us once and for all fix some pretty convenient (abuse of) notation.
\begin{con}\lbl{con:AON}
	Let $f\colon X\morphism Y$ be a morphism of preschemes. Let's make the following conventions.
	\begin{alphanumerate}
		\item If $y\in Y$, then $X_y$ denotes the fibre $f^{-1}\{y\}=X\times_Y\Spec\KK(y)$.
		\item We keep the abuse of notation that $\Ff|_{X_y}=i_y^*\Ff$ when $\Ff$ is a quasi-coherent sheaf of $\Oo_X$-modules. Here $i_y\colon X_y=X\times _Y\Spec \KK(y)\morphism X$ is the canonical projection to the first fibre product factor.
		\item If $\Gg$ is a (not necessarily quasi-coherent) sheaf of $\Oo_Y$-modules and $y\in Y$, then we put $\Gg(y)=\Gg_y\otimes_{\Oo_{Y,y}}\KK(y)$.
	\end{alphanumerate}
\end{con}
\begin{thm}\lbl{thm:GrauertGrothendieck}
	Let $f\colon X\morphism Y$ be a proper morphism between locally noetherian preschemes and let $\Ff$ be a coherent $\Oo_X$-module which is flat over $\Oo_Y$.
	\begin{alphanumerate}
		\item If $Y=\Spec A$ is affine, there exists a complex
		\begin{align*}
			P^\bullet\colon 0\morphism P^0\morphism P^1\morphism \ldots \morphism P^d\morphism 0
		\end{align*}
		of finitely generated projective $A$-modules with the following property. If $B$ is an $A$-algebra and $\xi\colon \snake{X}=X\times_Y\Spec B\morphism X$ is the projection of the fibre product to its first factor, then
		\begin{align*}
			H^p\left(\snake{X},\xi^*\Ff\right)\cong H^p(P^\bullet\otimes_AB)\;.
		\end{align*}
		It is possible to choose $d$ to be the largest number with $H^d(X,\Ff)\neq 0$, or $P^\bullet=0$ if $H^\bullet(X,\Ff)=0$. Moreover, if $d'\geq 0$ is such that $H^p(X_y,\Ff|_{X_y})=0$ for all $y\in Y$ when $p<d'$, then it is possible to choose $P^\bullet$ such that $P^i=0$ unless $d'\leq i\leq d$.
		\item The function $h^p(-,\Ff)\colon Y\morphism \IN$ given by
		\begin{align*}
		h^p(y,\Ff)=\dim_{\KK(y)}H^p(X_y,\Ff|_{X_y})
		\end{align*}
		is upper-semicontinuous in the sense that $\left\{y\in Y\st h^p(x,\Ff)\leq c\right\}$ is an open subset of $Y$ for all $c\in \IR$. Moreover, the \defemph{Euler--Poincaré characteristic} $\chi(-,\Ff)\colon Y\morphism \IN$ of the fibres, which is given by
		\begin{align*}
			\chi(y,\Ff)=\sum_{p\geq 0}(-1)^ph^p(y,\Ff)\;,
		\end{align*}
		is locally constant on $Y$.
		\item If $Y$ is reduced and $h^p(-,\Ff)$ is locally constant on $Y$, then the base change morphism
		\begin{align*}
			\upsilon^*R^pf_*\Ff\xrightarrow{\text{\eqreff{eq:baseChangeMorphism}}} R^p\snake{f}_*\left(\xi^*\Ff\right)
		\end{align*}
		(with notation from Proposition~\reff{prop:baseChangeMorphism}) is an isomorphism. In particular, for $\snake Y=\Spec \KK(y)$ we get
		\begin{align}\lbl{eq:RpfFy}
			(R^pf_*\Ff)(y)\isomorphism H^p\left(X_y,\Ff|_{X_y}\right)\;,
		\end{align}
		where $(R^pf_*\Ff)(y)$ is to be interpreted as in Convention~\reff{con:AON}\itememph{c}.
		\item If reducedness wasn't required in \itememph{c}, the canonical morphism \eqreff{eq:RpfFy} is bijective (for given $p$ and $y$) if and only if it is surjective. Moreover, for any $p\geq 0$ the set $U_p$ of all $y\in Y$ for which this is the case is open in $Y$.
		\item Let $y\in U_p$, then the following assertions are equivalent.
		\begin{itemize}
			\item[\itememph{\alpha}] We also have $y\in U_{p-1}$.
			\item[\itememph{\beta}] $R^pf_*\Ff$ is locally free in some neighbourhood of $y$.
		\end{itemize}
	\end{alphanumerate}
\end{thm}
\begin{rem}
	\begin{alphanumerate}
		\item Note that in Theorem~\reff{thm:GrauertGrothendieck}\itememph{a}, $H^p(X,\Ff)=0$ for $p$ sufficiently large, because $X$ is quasicompact, hence we can calculate its cohomology from the antisymmetric \v Cech complex of a finite affine open cover (even though $X$ could have infinite dimension).
		\item For instance, the assumptions of Theorem~\reff{thm:GrauertGrothendieck} are fulfilled when $f$ is a proper and flat morphism and $\Ff$ a vector bundle on $X$.
		\item From Theorem~\reff{thm:GrauertGrothendieck}\itememph{d} it follows that $R^pf_*\Ff=0$ when $H^p(X_y,\Ff|_{X_y})=0$ for all $y\in Y$.
		\item Note that $B$ in Theorem~\reff{thm:GrauertGrothendieck}\itememph{a} is completely arbitrary. In particular, it needn't be noetherian!
	\end{alphanumerate}
\end{rem}
	The strategy for Theorem~\reff{thm:GrauertGrothendieck}\itememph{a} is to modify the anti-symmetric \v Cech complex $\check{C}^\bullet=\check{C}_\alt^\bullet(\Uu,\Ff)$, where $\Uu$ is some fixed affine open cover of $X$. As $\Ff$ is flat over $\Oo_Y$, the $\Ff(U_{i_0,\ldots,i_n})$ are flat $A$-modules, hence $\check{C}^\bullet$ is a complex of flat $A$-modules, but the finite generation assertions will typically fail, as will the vanishing assertions. However, from \eqreff{eq:tensoredCechComplex} we know that
	\begin{align}\lbl{eq:baseChangeProperty}
		H^p(\snake{X},\xi^*\Ff)\cong H^p(\check{C}^\bullet\otimes_AB)
	\end{align}
	when $B$ and $\snake{X}$ are as in \itememph{a}. We will construct cochain complexes $P^\bullet$ and $Q^\bullet$ with morphisms $c\colon Q^\bullet\morphism\check{C}^\bullet$ and $p\colon Q^\bullet\morphism P^\bullet$ such that $P^\bullet$ has all the required properties and $c$, $p$ induce isomorphisms on $H^\bullet(-\otimes_AM)$ for arbitrary $A$-modules $M$.
	
	The rest of the proof of \itememph{a} doesn't concern any $\check{C}^\bullet$-specific properties except flatness and the base change property. However, in the upcoming sequence of commutative algebra lemmas we will rather freely use facts from homological algebra (usually you find them in \cite{homalg}).
	\begin{lem}\lbl{lem:TensorQuasiIsomorphism}
		Let $A$ be a (not necessarily noetherian) ring and $\phi\colon C^\bullet\morphism D^\bullet$ a quasi-isomorphism\footnote{That is, $\phi$ induces isomorphisms on cohomology} between bounded-from-above cochain complexes of flat $A$-modules. Then, for any $A$-module $M$,
		\begin{align*}
			\phi\otimes\id_M\colon C^\bullet\otimes_AM\morphism D^\bullet\otimes_AM
		\end{align*}
		is a quasi-isomorphism again.
	\end{lem}
\begin{proof}
	Using the short exact cone sequence $0\morphism D^\bullet \morphism\Cone(\phi)\morphism C[1]^\bullet\morphism 0$ (cf.\ \cite[Definition~2.1.1]{alggeo2}), we see that $\phi$ is a quasi-isomorphism iff $\Cone(\phi)$ has vanishing cohomology. But $\Cone(\phi\otimes\id_M)\cong \Cone(\phi)\otimes_AM$, so it's sufficient to prove that $\Cone(\phi)\otimes_AM$ is acyclic. It's clear from the construction that $\Cone(\phi)$ consists of flat $A$-modules again and is bounded from above as well. But then $\Cone(\phi)$ gives a flat resolution of the zero $A$-module, so the cohomology of $\Cone(\phi)\otimes_AM$ computes $\Tor_p^A(0,M)=0$ for all $p\geq 0$. We're done.
\end{proof}
\begin{rem}
	\begin{alphanumerate}
		\item\lbl{rem:TorFishy?} If you're uncomfortable with the $\Tor$ shortcut argument at the end of the proof above, Professor Franke suggests an alternative proof as follows. If $Z^p$ is the kernel of $\Cone(\phi)^p\morphism\Cone(\phi)^{p+1}$, then one can show that all $Z^p$ are are flat (using boundedness from above and downward induction on $p$). Then the short exact sequences
		\begin{align*}
			0\morphism Z^p\morphism \Cone(\phi)^p\morphism Z^{p+1}\morphism 0
		\end{align*}
		stay exact after tensoring with $M$, and splicing together the tensored sequences gives $\Cone(\phi)\otimes_AM$.
		\item When $A$ is regular, the boundedness assumption may be dropped. Indeed, since $\Cone(\phi)\otimes_AM$ is exact iff its localizations at any $\pp\in\Spec A$ are exact, we may assume $A$ to be regular and local. But then $\Tor_p^A(-,-)$ vanishes when $p>\dim A$ (cf.\ \cite[Theorem~1]{homalg}), so the $\Tor$ argument from the proof of Lemma~\reff{lem:TensorQuasiIsomorphism} works even if we're not resolving the zero module.
		
		However, boundedness can't be dropped in general. For example, consider $A=\IZ/p^2\IZ$ for some prime $p$, and let
		\begin{align*}
			C^\bullet\colon \ldots\morphism[ p\cdot]A\morphism[ p\cdot]A\morphism[ p\cdot]A\morphism[ p\cdot]\ldots\;,
		\end{align*}
		$D^\bullet=0$, and $M=\IZ/p\IZ$. Then $D^\bullet\otimes_AM=0$, but $C^\bullet\otimes_AM$ has cohomology $\IZ/p\IZ$ in every degree.
	\end{alphanumerate}
\end{rem}
Recall that if $C^\bullet$ is a cochain complex, one can form its \defemph{soft truncation from above}
\begin{align*}
	\tau_{\leq d}C^\bullet\colon \ldots\morphism C^n\morphism C^{n+1}\morphism \ldots \morphism C^{d-1}\morphism Z^d\morphism 0\morphism 0\morphism \ldots\;,
\end{align*} 
where $Z^d=\ker\left(C^{d}\morphism C^{d+1}\right)$. We have $H^p(\tau_{\leq d}C^\bullet)=H^p(C^\bullet)$ if $p\leq d$ and $H^p(\tau_{\leq d}C^\bullet)=0$ otherwise.
\begin{lem}\lbl{lem:TruncQuasiIso}
	Let $F^\bullet$ be a bounded-from-above cochain complex of flat modules over an arbitrary ring $A$. If $F$ is acyclic in degrees $>d$, then $\tau_{\leq d}F^\bullet\monomorphism F^\bullet$ is a quasi-isomorphism and $\tau_{\leq d}F^\bullet$ is still flat over $A$.
\end{lem}
\begin{proof}
	Using induction, we may assume $F^\bullet$ to be concentrated in degrees $\leq d+1$. Then $0\morphism Z^d\morphism C^d\morphism C^{d+1}\morphism 0$ is exact. Let $M$ be any $A$-module. Since $C^d$, $C^{d+1}$ are flat, we get
	\begin{align*}
		0=\Tor_2^A\big(C^{d+1},M\big)\morphism \Tor_1^A\big(Z^d,M\big)\morphism \Tor_1^A\big(C^{d},M\big)=0
	\end{align*}
	as part of the long exact $\Tor$ sequence. Hence $\Tor_1^A\big(Z^d,M\big)=0$ for all $M$, so $Z^d$ is flat as well, proving that $\tau_{\leq d}F^\bullet$ is flat over $A$. The other assertion about $\tau_{\leq d}F^\bullet$ is trivial.
\end{proof}
\begin{rem}
	\begin{alphanumerate}
		\item In particular, Lemma~\reff{lem:TensorQuasiIsomorphism} applies to $\tau_{\leq d}F^\bullet\monomorphism F^\bullet$.
		\item Without boundedness, Lemma~\reff{lem:TruncQuasiIso} counterexample from Remark~\reff{rem:TorFishy?} still applies.
	\end{alphanumerate}
\end{rem}
\begin{lem}\lbl{lem:FiniteApproximation}
	Let $A$ be a noetherian ring and $F^\bullet$ a flat cochain complex of $A$-modules which is concentrated in degrees $< d$. Then there is a cochain complex $Q^\bullet$, consisting of finitely generated free $A$-modules and concentrated in degrees $<d$, together with a quasi-isomorphism $\rho^\bullet\colon Q^\bullet\morphism F^\bullet$.
\end{lem}
\begin{proof}
	Denote the differentials of $F^\bullet$ and $Q^\bullet$ by $d_F^\bullet$, $d_Q^\bullet$ (the latter of which is yet to be constructed). We will construct the $Q^i$ by downward induction. We start by putting $Q^i=0$ when $i\geq d$. Now let $k\in\IZ$ and suppose we have already constructed $Q^i$ and $d_Q^i$ for $i\geq k$ together with $\rho^i\colon Q^i\morphism F^i$ (compatible with the differentials) such that $H^p(Q^\bullet)\isomorphism H^p(F^\bullet)$ is an isomorphism when $p>k$ and $\ker(d_Q^k)\morphism H^k(F^\bullet)$ is surjective.
	
Now it's about constructing suitable $Q^{k-1}$, $d_Q^{k-1}$ such that the above conditions remain valid for $k-1$. Let $\beta_1,\ldots,\beta_\ell$ be generators of $\left\{x\in Q^k\st \rho^k(x)\in\Im\big(d_F^{k-1}\big)\right\}$
(that's where we really need that $A$ is noetherian) and let $\phi_i\in F^{k-1}$ such that $d_F^{k-1}(\phi_i)=\rho^k(x)$ for all $i=1,\ldots,\ell$. Also choose $\psi_1,\ldots,\psi_m\in\ker\big(d_F^{k-1}\big)$ whose images in $H^{k-1}(F^\bullet)$ generate that $A$-module. Put $Q^{k-1}=A^{\ell+m}$ and let $e_i$ be the $i\ordinalth$ standard basis vector. Now define $d_Q^{k-1}\colon Q^{k-1}\morphism Q^k$ and $\rho^{k-1}\colon Q^{k-1}\morphism F^{k-1}$ by
\begin{align*}
	d_Q^{k-1}(e_i)=
	\begin{cases}
		\beta_i &\text{if }i\leq \ell\\
		0 & \text{if }\ell <i\leq \ell+m
	\end{cases}\quad\text{and}\quad\rho^{k-1}(e_i)=
	\begin{cases}
		\phi_i & \text{if }i\leq \ell\\
		\psi_{i-\ell} & \text{if }\ell <i\leq \ell+m
	\end{cases}\;.
\end{align*}
This does it, as is easily checked.
\end{proof}
Let $C^\bullet$ be any cochain complex. Similar to the soft truncation from above (from above), there's a \defemph{soft truncation from below} given by
\begin{align*}
\tau_{\geq d}C^\bullet\colon\ldots\morphism 0\morphism 0\morphism C^d/B^d\morphism C^{d+1}\morphism C^{d+2}\morphism \ldots\;,
\end{align*}
in which $B^d=\Im\left(C^{d-1}\morphism C^d\right)$. It has the property that $H^p(\tau_{\geq d}C^\bullet)=H^p(C^\bullet)$ if $p\geq d$ and $H^p(\tau_{\geq d}C^\bullet)=0$ otherwise.
\begin{lem}\lbl{lem:TruncQuasiIso2}
	Let $A$ be a Noetherian ring and $Q^\bullet$ be a cochain complex of finitely generated projective $A$-modules. Suppose there's $d'\in\IZ$ such that $H^p(Q^\bullet\otimes_A\KK(\mm))=0$ for all $p<d'$ and all maximal ideals $\mm\subseteq A$. Then $\tau_{\geq d'}Q^\bullet$ is a cochain complex of projective $A$-modules. Also $Q^\bullet\morphism \tau_{\geq d'}Q^\bullet$ is a quasi-isomorphism if $H^p(Q^\bullet)=0$ for all $p\leq d'$.
\end{lem}
\begin{proof}
	W.l.o.g.\ let $d'=0$ and $B^0=\Im\left(Q^{-1}\morphism Q^0\right)$. Then
	\begin{align*}
		\ldots \morphism Q^{-2}\morphism Q^{-1}\morphism Q^0\morphism Q^0/B^0\morphism 0
	\end{align*}
	is a projective resolution of $Q^0/B^0$. Hence $\Tor_p^A(Q^0/B^0,\KK(\mm))=H^{-p}(Q^\bullet\otimes_A\KK(\mm))=0$ for all $\mm\in\mSpec A$ and all $p>0$. Since $Q^0/B^0$ is finitely generated over the noetherian ring $A$, this implies that $Q^0/B^0$ is projective (this is essentially \cite[Propositions~1.3.1 and~1.3.2]{homalg}, but you should combine them with \cite[Fact~1.2.4]{homalg} first). The other assertion about $Q^\bullet\morphism \tau_{\geq d'}Q^\bullet$ is trivial again.
\end{proof}
\begin{exc}
	Under the assumptions of Theorem~\reff{thm:GrauertGrothendieck}, derive Theorem~\reff{thm:FormalFunctions} for $f\colon X\morphism Y$ and $\Ff$ from what has been shown until Lemma~\reff{lem:FiniteApproximation}.
\end{exc}
\begin{lem}\lbl{lem:TechnicalVanishingAssertion}
	If $d'$ is as in Theorem~\reff{thm:GrauertGrothendieck}\itememph{a}, i.e., has the property that $H^p\left(X_y,\Ff|_{X_y}\right)=0$ for all $y\in Y$ and $p<d'$, then also $H^p(X,\Ff)=0$ when $p<d'$.
\end{lem}
\begin{proof}
	Professor Franke outsourced a part of this proof to an external fact -- however, this fact looked rather fishy to me (if not to say \emph{wrong} unless $f$ is flat). I think the following captures what he actually had in mind.
	
	By Corollary~\reff{cor:completionFaithfullyFlat2} it's sufficient to show that the $\mm$-adic completions $H^p(X,\Ff)^\complete$ vanish for all maximal ideals $\mm\subseteq A$ and all $p<d'$. Let $y\in Y$ be the point corresponding to $\mm$. For all $n\leq 0$ let
	\begin{align*}
		X_y^{(n)}=X\times_Y\Spec A/\mm^{n+1}
	\end{align*}
	be the $n\ordinalth$ infinitesimal thickening of the fibre $X_y$, and let $\kappa_n\colon X_y^{(n)}\morphism X$ be the projection to the first fibre product factor. We will show $H^p\Big(X_y^{(n)},\Ff|_{X_y^{(n)}}\Big)=0$ for all $n\geq 1$. Note that
	\begin{align*}
		H^p\Big(X_y^{(n)},\Ff|_{X_y^{(n)}}\Big)\cong H^p\left(X,\kappa_{n,*}\kappa_n^*\Ff\right)
	\end{align*}
	and $\kappa_{n,*}\kappa_n^*\Ff\cong \Ff/\mm^{n+1}\Ff$ (we have already seen this in the discussion before Theorem~\reff{thm:FormalFunctions}). Now $H^p(X,\Ff/\mm^{n+1}\Ff)=0$ follows by induction on $n$ as follows. The case $n=0$ is just the assumption that $H^p(X_y,\Ff|_{X_y})=0$. Now let $n\geq 1$ and suppose that $H^p(X,\Ff/\mm^n\Ff)=0$ has already been shown. Consider the short exact sequence
	\begin{align*}
		0\morphism \mm^n\Ff/\mm^{n+1}\Ff\morphism \Ff/\mm^{n+1}\Ff\morphism \Ff/\mm^n\Ff\morphism 0\;.
	\end{align*}
	If we show that $H^p\left(X,\mm^n\Ff/\mm^{n+1}\Ff\right)=0$, then $H^p\left(X,\Ff/\mm^{n+1}\Ff\right)=0$ follows from the long exact cohomology sequence and the induction hypothesis. Note that $\mm^n\otimes_A\Ff\cong \mm^n\Ff$ (and same for $n+1$). Indeed, this follows from \cite[Proposition~1.2.3\itememph{d}]{homalg} because $\Ff$ is flat over $\Oo_Y$. Hence
	\begin{align*}
		\mm^n\Ff/\mm^{n+1}\cong \mm^n/\mm^{n+1}\otimes_A\Ff\cong \mm^n/\mm^{n+1}\otimes_{\KK(\mm)}\Ff/\mm\Ff
	\end{align*}
	is isomorphic to a direct sum of $\dim_{\KK(\mm)}\left(\mm^n/\mm^{n+1}\right)$ copies of $\Ff/\mm \Ff$, because $\mm^n/\mm^{n+1}$ is a finite-dimensional $\KK(\mm)$-vector space. By the $n=0$ case, this shows $H^p\left(X,\mm^n\Ff/\mm^{n+1}\Ff\right)=0$ and we're done.
\end{proof}
\begin{rem}
	As the above proof shows, it's actually sufficient to have $H^p\left(X_y,\Ff|_{X_y}\right)=0$ for all \emph{closed} $y\in Y$.
\end{rem}
We now have all the technicalities together to prove Theorem~\reff{thm:GrauertGrothendieck} (ok, actually we there are still three commutative algebra lemmas to follow, but we can start now anyway).
\begin{proof}[Proof of Theorem~\reff{thm:GrauertGrothendieck}]
	The first step will be to prove \itememph{a}, from which the other assertions will be deduced. In \itememph{a}, $Y$ is affine and the other assertions are all base-local, hence without losing generality we may assume $Y=\Spec A$ for the remainder of this proof.
	
	Fix a finite affine open cover $\Uu$ of $X$ and let $\check{C}^\bullet=\check{C}_\alt^\bullet(\Uu,\Ff)$. Using Lemma~\reff{lem:TruncQuasiIso} we see that $\tau_{\leq d}\check{C}^\bullet$ consists of flat $A$-modules again. Then we choose a finitely generated free approximation $Q^\bullet\morphism\tau_{\leq d}\check{C}^\bullet$ as in Lemma~\reff{lem:FiniteApproximation}. By Lemma~\reff{lem:TensorQuasiIsomorphism} we get
	\begin{align*}
		H^p\left(Q^\bullet\otimes_A\KK(\mm)\right)\cong H^p\left(\tau_{\geq d}\check{C}\otimes_A\KK(\mm)\right)\cong H^p\left(\check{C}\otimes_A\KK(\mm)\right)\overset{\text{\eqreff{eq:baseChangeProperty}}}{\cong} H^p\left(X_y,\Ff|_{X_y}\right)\;,
	\end{align*}
	where $y\in Y$ on the right-hand side is the point corresponding to the maximal ideal $\mm$, so that we have $X_y\cong X\times_Y\Spec\KK(\mm)$. In particular, we see $H^p\left(Q^\bullet\otimes_A\KK(\mm)\right)=0$ if $p<d'$ and also $H^p(Q^\bullet)\cong H^p(X,\Ff)=0$ if $p<d'$ by Lemma~\reff{lem:TechnicalVanishingAssertion}, hence Lemma~\reff{lem:TruncQuasiIso2} may be applied. Therefore, $P^\bullet=\tau_{\geq d'}Q^\bullet$ is complex of finitely generated projective $A$-modules with is supported only between $d'$ and $d$, and $P^\bullet$ has the required base change property because of \eqreff{eq:baseChangeProperty} and Lemma~\reff{lem:TensorQuasiIsomorphism}. In other words, $P^\bullet$ is what we want. This proves \itememph{a}.
	
	\lbl{page:Pfree}Note that for the rest of the proof we may even assume that $P^\bullet$ is a complex of \emph{free} modules. Indeed, our construction of $P^\bullet$ provided free modules except in degree $d'$, where $P^{d'}$ might only be projective. However, assertions \itememph{b} to \itememph{e} are base-local, so we may replace $\Spec A$ by $\Spec A_f$ such that $P_f^{d'}$ is free over $A_f$.%Dunno if that's needed.
	
	Let $Q^p=\coker\left(P^{p-1}\morphism P^p\right)$. Then $\coker\left(P^{p-1}\otimes_AB\morphism P^p\otimes_AB\right)\cong Q^p\otimes_AB$. If we put $\delta^p\colon Q^p\morphism P^{p+1}$, then $H^p(P^\bullet\otimes_AB)\cong \ker\left(\delta^p\otimes\id_B\right)$ for all $A$-algebras $B$ (actually, we don't even need $B$ to be an algebra).
	
	Now for part \itememph{b}. We prove that $\chi(-,\Ff)$ is locally constant on $Y$ first. Indeed, since the $P^i$ are free over $A$ (or rather we may assume them to be, as explained above) we have
	\begin{align*}
		\chi(y,\Ff)=\sum_{p\geq 0}(-1)^p\dim_{\KK(y)}H^p(X_y,\Ff|_{X_y})&=\sum_{p\geq 0}(-1)^p\dim_{\KK(y)}H^p(P^\bullet\otimes_A\KK(y))\\
		&=\sum_{p\geq 0}(-1)^p\dim_{\KK(y)}\left(P^p\otimes_A\KK(y)\right)\\
		&=\sum_{p\geq 0}(-1)^p\rank P^p
	\end{align*}
	(the transition from the first to the second line is a well-known fact from homological algebra). The last line is clearly constant on $Y$ and we're done.
	
	To prove the rest of \itememph{b} as well as \itememph{c}, we invoke another lemma.
	\begin{lem}\lbl{lem:BasicallyThm4bc}
		Let $A$ be a noetherian ring and $\delta\colon M\morphism F$ a morphism of finitely generated $A$-modules, of which $F$ is free of rank $n$. 
		\begin{alphanumerate}
			\item The function $g\colon \Spec A\morphism\IN$ given by
			\begin{align*}
				g(\pp)=\dim_{\KK(\pp)}\ker\left(\delta_\pp=\delta\otimes\id_{\KK(\pp)}\colon M\otimes_A\KK(\pp)\morphism F\otimes_A\KK(\pp)\right)
			\end{align*}
			is upper-semicontinuous on $\Spec A$.
			\item If in addition $g$ is locally constant and $A$ is reduced, then $M$, $\ker\delta$, and $\coker\delta$ are all projective $A$-modules, and the canonical morphism
			\begin{align*}
				\ker\delta\otimes_AN\morphism\ker\left(\delta\otimes\id_N\right)
			\end{align*}
			is an isomorphism for all $A$-modules $N$.
		\end{alphanumerate}
	\end{lem}
	\begin{proof}
		Part \itememph{a}. Let $\mu_1,\ldots,\mu_s\in M$ be elements whose images $\ov{\mu}_1,\ldots,\ov{\mu}_s$ in $M\otimes_A\KK(\pp)$ form a basis of that $\KK(\pp)$-vector space.\footnote{Why can we do this? Start with any basis of $M\otimes_A\KK(\pp)\cong M_\pp/\pp M_\pp$ and lift it to $M_\pp$. Then the numerators (which are elements of $M$) will do. We will use this argument implicitly in the proofs to come.} Moreover, we may choose them in such a way that their images $\delta_\pp(\ov{\mu}_1),\ldots,\delta_\pp(\ov{\mu}_r)$ (for some $r\leq s$) in $F\otimes_A\KK(\pp)$ vanish while the images $\delta_\pp(\ov{\mu}_{r+1}),\ldots,\delta_\pp(\ov{\mu}_s)$ form a basis of $\Im\delta_\pp$. 
		
		Then we have $r=g(\pp)$. Since the assertions are all local near any given $\pp\in\Spec A$, we are free to replace $A$ by $A_f$ for $f\notin\pp$.
		\begin{claim}\lbl{claim:MatrixShit}
			Choosing $f$ appropriately we can assure that $\mu_1,\ldots,\mu_s$ generate $M$ as an $A$-module, and also that their images in $M\otimes_A\KK(\qq)$ stay $\KK(\qq)$-linearly independent for all $\qq\in\Spec A$. 
		\end{claim}
		The first of these two assertions is a Nakayama-style argument that you can find in  \cite[Lemma~1.5.1]{alg2}. The latter, however, needs a bit more care, and we'll explain how this works now. Choose a basis of $F$ (which is free of rank $n$ by assumption) and represent $\delta(\mu_{r+1}),\ldots,\delta(\mu_s)$ as linear combinations of that basis. The resulting coefficient matrix $C\in A^{(s-r)\times n}$ has a $(s-r)\times (s-r)$-minor $D\in A^{(s-r)\times (s-r)}$ whose image in $\KK(\pp)^{(s-r)\times (s-r)}$ is invertible, because $\delta_\pp(\ov{\mu}_{r+1}),\ldots,\delta_\pp(\ov{\mu}_s)$ are $\KK(\pp)$-linearly independent. In particular, we must have $\det D\not\equiv 0\bmod \pp$, so we may localize $A$ at $\det D$ to make $D$ invertible over $A$. Then the images of $\mu_{r+1},\ldots,\mu_s$ in $M\otimes_A\KK(\qq)$ are linearly independent for all $\qq\in\Spec A$, because they are still linearly independent after applying $\delta_\qq$. This justifies the above Claim~\reff{claim:MatrixShit}.
		
		Now we have $g(\qq)=\dim_{\KK(\qq)}\ker\delta_\qq=\dim_{\KK(\qq)}(M\otimes_A\KK(\qq))-\dim_{\KK(\qq)}\Im\delta_\qq$, in which the inequalities
		\begin{align}\lbl{eq:dimIneqs}
			\dim_{\KK(\qq)}(M\otimes_A\KK(\qq))\leq s\quad\text{and}\quad\dim_{\KK(\qq)}\Im\delta_\qq\geq s-r
		\end{align}
		hold true because of Claim~\reff{claim:MatrixShit}. Hence $g(\qq)\leq r$ for all $\qq\in\Spec A$, and this proves upper-semicontinuity. We're done with \itememph{a}.
		
		Part \itememph{b}. Replacing $A$ by some localization again, we may assume that $g(\qq)=r$ for all $\qq\in\Spec A$. Then equality must hold in both inequalities from \eqreff{eq:dimIneqs}. In particular, $\dim_{\KK(\qq)}(M\otimes_A\KK(\qq))=s$ for all $\qq\in\Spec A$, which implies that $M$ is locally free (of rank $s$) by the upcoming Lemma~\reff{lem:ReducedLocallyFree}. Similarly, we have
		\begin{align*}
			\dim_{\KK(\qq)}\left(\coker\delta\otimes_A\KK(\qq)\right)=\dim_{\KK(\qq)}\coker\delta_\qq&=\dim_{\KK(\qq)}\left(F\otimes_A\KK(\qq)\right)-\dim_{\KK(\qq)}\Im\delta_\qq\\
			&=n-s+r
		\end{align*}
		(where $n=\rank F$, remember?), so $\coker\delta$ is locally free by Lemma~\reff{lem:ReducedLocallyFree} again. In particular, $\coker\delta$ is projective, so $F\epimorphism\coker\delta$ splits and we get $F\cong \Im\delta\oplus\coker\delta$ (\emph{non}-canonically). Then $\Im\delta$ is projective as well, so $M\epimorphism\Im\delta$ splits as well, hence $M\cong \ker\delta\oplus\Im\delta$ and $\ker\delta$ lines up with the other projective modules. Therefore, $\delta$ has the form $\delta\colon \ker\delta\oplus\Im\delta\morphism\Im\delta\oplus\coker\delta$, and this nice splitting behaviour is preserved after tensoring with some $A$-module $N$. This shows \itememph{b}.
	\end{proof}
	Die-hard Franke fans might remember the next lemma from Franke's Algebra II lecture, where it appeared as a remark though; see \cite[p.~25]{alg2}. We'll prove it here again.
	\begin{lem}\lbl{lem:ReducedLocallyFree}
		Let $A$ be a reduced noetherian ring and $M$ a finitely generated $A$-module, such that the function
		\begin{align*}
			\pp\longmapsto \dim_{\KK(\pp)}M\otimes_A\KK(\pp)
		\end{align*}
		is locally constant on $\Spec A$. Then $M$ is locally free.
	\end{lem}
	\begin{proof}
		Again, the question is local on $\Spec A$, so after localizing $A$ appropriately we may assume that $\dim_{\KK(\pp)}(M\otimes_A\KK(\pp))=\ell$ for all $\pp\in\Spec A$ and some $\ell\in\IN$. Choose some $\pp\in\Spec A$ and let $\mu_1,\ldots,\mu_\ell\in M$ be elements whose images form a basis of $M\otimes_\KK(\pp)$. Without losing generality (or rather after another localization, using \cite[Lemma~1.5.1]{alg2}) the $\mu_i$ generate $M$ as an $A$-module. We claim they do so freely. Indeed, if $\alpha_1\mu_1+\ldots+\alpha_\ell\mu_\ell=0$ for some $\alpha_i\in A$ is any relation in $M$, then $\alpha_i\in\qq$ for any $\qq\in\Spec A$ as $\dim_{\KK(\qq)}(M\otimes_A\KK(\qq))<\ell$ otherwise. Hence $\alpha_i\in\bigcap_{\qq\in\Spec A}\qq=\nil A=0$ and we're done.
	\end{proof}
	We come back to the proof of Theorem~\reff{thm:GrauertGrothendieck}\itememph{b} and \itememph{c}. The upper-semicontinuity part of \itememph{b} follows immediately from Lemma~\reff{lem:BasicallyThm4bc}\itememph{a} applied to $M=Q^p$ und $F=P^{p+1}$. For part \itememph{c}, we may assume the base change was done along $\upsilon\colon \snake{Y}=\Spec B\morphism Y=\Spec A$ because the question is local on $Y$ and $\snake{Y}$. Then $R^p\snake{f}_*(\xi^*\Ff)$ is a quasi-coherent $\Oo_{\snake{Y}}$-module given by
	\begin{align*}
		H^p(\snake{X},\xi^*\Ff)\cong H^p(P^\bullet\otimes_AB)\cong \ker\left(\delta^p\otimes\id_B\colon Q^p\otimes_AB\morphism P^{p+1}\otimes_AB\right)\;,
	\end{align*}
	and $\upsilon^*R^pf_*\Ff$ is given by
	\begin{align*}
		H^p(X,\Ff)\otimes_AB\cong H(P^\bullet)\otimes_AB\cong\ker\left(\delta^p\colon Q^p\morphism P^{p+1}\right)\otimes_AB\;.
	\end{align*}
	That these two agree is precisely what Lemma~\reff{lem:BasicallyThm4bc}\itememph{c} says.
	
	For \itememph{d} and \itememph{e} we need yet another lemma, but I promise this will be the last technical commutative algebra lemma required to prove Theorem~\reff{thm:GrauertGrothendieck}.
	\begin{lem}\lbl{lem:BasicallyThm4de}
		Let $\pp\in \Spec A$. In the above situation, the following conditions are equivalent.
		\begin{alphanumerate}
			\item The canonical morphism
			\begin{align*}
				\ker\delta^p\otimes_A\KK(\pp)\morphism\ker\left(\delta^p_\pp=\delta^p\otimes \id_{\KK(\pp)}\colon Q^p\otimes_A\KK(\pp)\morphism P^{p+1}\otimes_A\KK(\pp)\right)
			\end{align*}
			is surjective.
			\item Same as \itememph{a}, but the morphism is even bijective.
			\item The localization $Q^{p+1}_\pp$ is a free $A_\pp$-module.
		\end{alphanumerate}
		Moreover, in this case the localization $B^{p+1}_\pp$ of $B^{p+1}=\Im\delta^p$ is free over $A_\pp$ as well.
	\end{lem}
\begin{proof}
	We use similar arguments as in Lemma~\reff{lem:BasicallyThm4bc}. Let's do the implication \itememph{a} $\Rightarrow$ \itememph{c} first. If assertion \itememph{a} holds, then we may choose $\mu_1,\ldots,\mu_r\in\ker \delta^p$ whose images $\ov{\mu}_i$ in $\ker\left(\delta^p\otimes\id_{\KK(\pp)}\right)$ form a basis. We may extend $\mu_1,\ldots,\mu_r$ by some $\mu_{r+1},\ldots,\mu_s\in Q^p$ such that $\ov{\mu}_1,\ldots,\ov{\mu}_s$ form a basis of $Q^p\otimes_A\KK(\pp)$. Then the images $\delta_\pp^p(\ov{\mu}_{r+1}),\ldots,\delta_\pp^p(\ov{\mu}_s)\in P^{p+1}\otimes_A\KK(\pp)$ are $\KK(\pp)$-linearly independent, so we may choose $\pi_1,\ldots,\pi_n\in P^{p+1}$ whose images in $P^{p+1}\otimes_A\KK(\pp)$ extend $\delta_\pp^p(\ov{\mu}_{r+1}),\ldots,\delta_\pp^p(\ov{\mu}_s)$ to a basis.
	
	By the same argument as in Claim~\reff{claim:MatrixShit} from the proof of Lemma~\reff{lem:BasicallyThm4bc}\itememph{a}, we can replace $A$ by some localization so that $\delta^p(\mu_{r+1}),\ldots,\delta^p(\mu_s)$ and $\pi_1,\ldots,\pi_n$ generate $P^{p+1}$ -- and we may even assume they do so \emph{freely}! Indeed, write $\delta^p(\mu_{r+1}),\ldots,\delta^p(\mu_s)$ and $\pi_1,\ldots,\pi_n$ as linear combinations of some basis of $P^{p+1}$. Then the coefficient matrix $C$ becomes invertible when reduced to $\KK(\pp)$, hence after localizing at $\det C$ we may assume that $C$ is already invertible over $A$. Therefore, after suitable localization, we see that $\Im\delta^p$ is a direct summand of $P^{p+1}$. Then $Q^{p+1}=\coker\delta^p$ is a direct summand of $P^{p+1}\cong \Im\delta^p\oplus\coker\delta^p$ as well. This proves that $Q^{p+1}$ is projective, hence $Q^{p+1}_\pp$ is free over $A_\pp$. Thereby the implication \itememph{a} $\Rightarrow$ \itememph{c} has been shown.
	
	Now for \itememph{c} $\Rightarrow$ \itememph{b}. By \cite[Corollary~1.5.1]{alg2}, we may replace $A$ (and $Q^{p+1}$) by some localization such that $Q^{p+1}$ is already free over $A$. Then $P^{p+1}\epimorphism Q^{p+1}=\coker\delta^p$ is split, hence $P^{p+1}\cong \Im\delta^p\oplus \coker\delta^p$ (non-canonically). Then $\Im\delta^p$ is projective as well, so $Q^p\cong \ker\delta^p\oplus\Im\delta^p$ (non-canonically). That is, $\delta^p$ splits as $\delta^p\colon \ker\delta^p\oplus\Im\delta^p\morphism\Im\delta^p\oplus\coker\delta^p$, and this nice splitting behaviour is preserved after tensoring with $\KK(\pp)$. This shows $\ker\delta^p\otimes_A\KK(\pp)\cong \ker\left(\delta^p\otimes\id_{\KK(\pp)}\right)$, i.e., \itememph{b} holds.
	
	Finally \itememph{b} $\Rightarrow$ \itememph{a} is trivial, and also we've already seen the additional assertion in the proof of \itememph{a} $\Rightarrow$ \itememph{c}. We're done.
\end{proof}

	Now there isn't really much left to prove \itememph{d} and \itememph{e}. Since $Y=\Spec A$ is affine (or rather we reduced it to this case), $R^pf_*\Ff$ is the quasi-coherent $\Oo_Y$-module given by the $A$-module $H^p(X,\Ff)\cong H^p(P^\bullet)\cong \ker\delta^p$. Then Lemma~\reff{lem:BasicallyThm4de}\itememph{a} and \itememph{b} shows that $(R^pf_*\Ff)(y)\morphism H^p(X_y,\Ff|_{X_y})$ is bijective iff it is surjective. Moreover, the set of $\pp\in\Spec A$ such that the canonical morphism Lemma~\reff{lem:BasicallyThm4de}\itememph{c} holds is open by \cite[Corollary~1.5.1]{alg2}, which directly translates into $U_p\subseteq Y$ being open. This shows \itememph{d}.
	
	Finally, part \itememph{e}. If $y\in U_p$, then $Q^{p+1}$ as well as $B^{p+1}=\Im \delta^p$ are locally free at the prime $\pp\in\Spec A$ corresponding to $y$ (by Lemma~\reff{lem:BasicallyThm4de}\itememph{c}). Therefore, from the short exact sequence
	\begin{align*}
		0\morphism H^p(X,\Ff)\morphism Q^p\morphism B^{p+1}\morphism 0
	\end{align*}
	we obtain $\Tor_1^A\big(H^p(X,\Ff),\KK(y)\big)\cong \Tor_1^A(Q^p,\KK(y))$. By \cite[Proposition~1.3.1]{homalg}, this shows that $Q^p$ is locally free at $\pp$ iff so is $H^p(X,\Ff)$. Using Lemma~\reff{lem:BasicallyThm4de}\itememph{c} we see that $y\in U_{p-1}$ iff $H^p(X,\Ff)$ is locally free at $\pp$ -- and that's precisely what we want.
\end{proof}
\begin{cor}
	In the situation of Theorem~\reff{thm:GrauertGrothendieck}, if $h^p(y,\Ff)=0$, then $R^pf_*\Ff$ vanishes in some neighbourhood of $x$.
\end{cor}
\begin{proof}
	By Theorem~\reff{thm:GrauertGrothendieck}\itememph{d} we have $(R^pf_*\Ff)(y)=0$, hence $R^pf_*\Ff=0$ near $y$ by Nakayama-style arguments (to be precise, this follows from \cite[Corollary~1.5.1]{alg2}).
\end{proof}
\begin{cor}
	If $\upsilon\colon \snake{Y}\morphism Y$ is a morphism with image contained in $U_p$, and $\xi\colon \snake{X}=X\times_Y\snake{Y}\morphism X$ its base change along $f$, then the base change morphism is an isomorphism
	\begin{align*}
		\upsilon^*R^pf_*\Ff\isomorphism R^p\snake{f}_*(\xi^*\Ff)\;.
	\end{align*}
\end{cor}
\begin{proof}
	The assertion is local on $Y$ and $\snake{Y}$, so let's assume $Y=\Spec A$ and $\snake{Y}=\Spec B$. We may assume that $\upsilon$ factors over $\Spec A_\alpha\monomorphism \Spec A$ for some $\alpha\in A$ such that $\Spec A\setminus V(\alpha)\subseteq U_p$. Then
	\begin{align*}
		H^p(\snake{X},\xi^*\Ff)\cong H^p\left(P^\bullet\otimes_AB\right)\cong H^p\left(P_\alpha^\bullet\otimes_{A_\alpha}B\right)
	\end{align*}
	and
	\begin{align*}
		H^p(X,\Ff)\otimes_AB\cong H^p(P^\bullet)\otimes_AB\cong H^p(P_\alpha^\bullet)\otimes_{A_\alpha}B\;.
	\end{align*}
	Since Lemma~\reff{lem:BasicallyThm4de} is applicable to all $\pp\in \Spec A_\alpha$, we see that $Q_\alpha^{p+1}$ is projective. As in the proof of Lemma~\reff{lem:BasicallyThm4de}, this implies that the localization $\delta_\alpha^p=\delta^p\otimes_A\id_{A_\alpha}$ of $\delta^p$ splits as $\delta_\alpha^p\colon \ker\delta_\alpha^p\oplus \Im \delta_\alpha^p\morphism \Im\delta_\alpha^p\oplus\coker\delta_\alpha^p$, hence it behaves well under tensoring with the $A_\alpha$-algebra $B$. Thus $H^p(X,\Ff)\otimes_AB\morphism H^p(\snake{X},\xi^*\Ff)$ is an isomorphism, which is precisely what we need.
\end{proof}
\section{Application to the group structure of elliptic curves}
\begin{defi}
	A \defemph{flat family} of curves of genus $g$ is a (faithfully) flat proper morphism $f\colon C\morphism S$ whose \defemph{geometric fibres}
	\begin{align*}
		C_{\ov{s}}=C\times_S\Spec\ov{\KK(s)}
	\end{align*}
	are regular connected curves of genus $g$ over the algebraic closure $\ov{\KK(s)}$ for all $s\in S$. If $g=1$ and in addition a section  $\sigma\colon S\morphism C$  of $f$ is given, we call $(f,\sigma)$ an \defemph{elliptic curve} over $S$.
\end{defi}
We will typically assume $S$ to be locally noetherian, although Professor Franke says one could generalize the situation to allow arbitrary $S$ if $f$ is of \emph{finite presentation}.
\begin{prop}\lbl{prop:RpLl}
	Let $f\colon C\morphism S$ be a flat family of curves of genus $g$, where $S$ is locally noetherian. Moreover, let $\Ll$ be a line bundle on $C$ whose pullbacks to the geometric fibres of $f$ have degree $d$.
	\begin{alphanumerate}
		\item If $d>2g-2$, then $f_*\Ll$ is a vector bundle of dimension $d+1-g$ on $S$, and $R^1f_*\Ll=0$.
		\item If $d<0$, then $f_*\Ll$ vanishes, and $R^1f_*\Ll$ is a vector bundle of dimension $g-1-d$.
		\item If $\Ll=\Oo_C$ (hence $d=0$), then $f^*\colon \Oo_S\morphism f_*\Oo_C$ is an isomorphism and $R^1f_*\Oo_C$ is a vector bundle of dimension $g$.
	\end{alphanumerate}
	Moreover, if we take a base change
	\begin{diagram*}
		\object{0,0}{$C$}[a];
		\object{0,1.5}{$\snake{C}$}[b];
		\object{2,0}{$S$}[c];
		\object{2,1.5}{$\snake{S}$}[d];
		\pullback{1,0.75};
		\scriptsize
		\arrow ba[left][$\gamma$];
		\arrow ac[above][$f$];
		\arrow dc[right][$\xi$];
		\arrow bd[above][$\snake{f}$];
	\end{diagram*}
	then the base change morphism $\xi^*R^pf_*\Ll\morphism R^p\snake{f}_*(\gamma^*\Ll)$ is an isomorphism for $p\in\{0,1\}$ in each of the above cases.
\end{prop}
\begin{rem}
	\begin{alphanumerate}
		\item \lbl{rem:GeometricFibres}First of all, if $\Ff$ is any quasi-coherent $\Oo_C$-module, then
		\begin{align}\lbl{eq:GeometricFibreIso}
		H^p\left(C_{\ov{s}},\Ff|_{C_{\ov{s}}}\right)\cong H^p\left(C_s,\Ff|_{C_s}\right)\otimes_{\KK(s)}\ov{\KK(s)}
		\end{align}
		for all $s\in S$. Indeed, $\ov{\KK(s)}/\KK(s)$ is a (pretty lame) flat ring extension, so \eqreff{eq:GeometricFibreIso} follows immediately from \eqreff{eq:tensoredCohomology}.
		\item Note that the degree of the pullbacks of $\Ll$ to the geometric fibres of $f$ is locally constant on $S$. Indeed, by Riemann--Roch (in the form of \cite[Theorem~8]{alggeo2}) and \itememph{a} we have
		\begin{align*}
			\deg(\Ll|_{C_{\ov{s}}})=\chi\left(C_{\ov{s}},\Ll|_{C_{\ov{s}}}\right)+g-1=\chi\left(s,\Ll\right)+g-1\;,
		\end{align*}
		where $\chi(s,\Ll)$ on the right-hand side is the Euler--Poincaré characteristic from Theorem~\reff{thm:GrauertGrothendieck}\itememph{c}. But $\chi(-,\Ll)$ is locally constant on $S$. So the assumption of Proposition~\reff{prop:RpLl} is not really restrictive at all.
	\end{alphanumerate}
\end{rem}
\begin{proof}[Proof of Proposition~\reff{prop:RpLl}]
	Throughout the proof, we may assume $S=\Spec A$ for some noetherian ring $A$ because all assertions are local. Also Theorem~\reff{thm:GrauertGrothendieck} is applicable since $f$ is flat, so every line bundle on $C$ is flat over $\Oo_S$. Let $P^\bullet$ be the complex from Theorem~\reff{thm:GrauertGrothendieck}\itememph{a}, concentrated within minimum and maximum of the set
	\begin{align*}
		D&=\left\{p\in\IN\st H^p\left(C_s,\Ll|_{C_s}\right)\neq 0\text{ for some }s\in S\right\}\\
		&=\left\{p\in\IN\st H^p\left(C_{\ov{s}},\Ll|_{C_{\ov{s}}}\right)\neq 0\text{ for some }s\in S\right\}
	\end{align*}
	(where we use \eqreff{eq:GeometricFibreIso}). Note that $D=\{0\}$ in the case of \itememph{a} and $D=\{1\}$ in the case of \itememph{b}. Indeed, in both cases $D$ is contained in $\{0,1\}$ by Grothendieck's theorem on cohomological dimension (cf.\ \cite[Proposition~1.4.1]{alggeo2}), and then Lemma~\reff{lem:Hpvanishing} determines $D$ (ok, actually it doesn't, but we'll see immediately why $D=\emptyset$ is impossible). Then $P^\bullet$ consists of at most one non-zero term, which defines a vector bundle on $S=\Spec A$. The assertions about dimension follow immediately from Riemann--Roch in the form of \cite[equation~(3.1.3)]{alggeo2}.
	
	For \itememph{c}, let $U_p\subseteq S$ be as in Theorem~\reff{thm:GrauertGrothendieck}\itememph{d} and \itememph{e}. By Serre duality -- or more precisely, \cite[Theorem~7\itememph{b} and \itememph{c}]{alggeo2} -- we see that
	\begin{align*}
		H^0\left(C_{\ov{s}},\Oo_{C_{\ov{s}}}\right)\cong \Hom_{\ov{\KK(s)}}\left(H^1\left(C_{\ov{s}},\Omega_{C_{\ov{c}}}\right),\ov{\KK(s)}\right)\cong\ov{\KK(s)}\cong \KK(s)
	\end{align*}
	is a one-dimensional vector space with the constant function $1$ as a basis, and the same holds thus for $H^0(C_s,\Oo_{C_s})$. Therefore, the canonical morphism
	\begin{align}\lbl{eq:H0OC}
		H^0(C,\Oo_C)\otimes_A\KK(s)\cong\big(R^0f_*\Oo_C\big)(s)\morphism H^0(C_s,\Oo_{C_s})
	\end{align}
	is an surjective for all $s\in S$ (because the constant function $1$ is always hit), so $U_0=S$ by Theorem~\reff{thm:GrauertGrothendieck}\itememph{d}. But $U_{-1}=S$ for trivial reasons, hence $f_*\Oo_C$ is a vector bundle by Theorem~\reff{thm:GrauertGrothendieck}\itememph{e}. Moreover, from \eqreff{eq:H0OC} (which is actually an isomorphism, as we now know) we see that $\Oo_C(C)\otimes_A\KK(s)$ is a one-dimensional $\KK(s)$-vector space with $1\otimes 1$ as a basis for all $s\in S$. Using \cite[Lemma~1.5.1]{alg2}, this implies that $f_*\Oo_C$ is generated by $1\in\Oo_C(C)$ as an $\Oo_S$-algebra. But $f_*\Oo_C$ is a vector bundle, so $f^*\colon \Oo_S\morphism f_*\Oo_C$ must be indeed an isomorphism.
	
	To show the second part of \itememph{c}, note that $H^2(C_s,\Oo_{C_s})=0$ for all $s\in S$ (by Grothendieck's dimension theorem), so $U_2=S$. Then $(R^2f_*\Oo_C)(s)=0$ for all $s\in S$, which proves $R^2f_*\Oo_C=0$ (this uses \cite[Lemma~1.5.1]{alg2} again). In particular, $R^2f_*\Oo_C$ is a vector bundle and Theorem~\reff{thm:GrauertGrothendieck}\itememph{e} gives $U_1=S$. Since $U_0=S$ as well, reversing the argument shows that $R^1f_*\Oo_C$ is a vector bundle too. The fact that it has dimension $g$ can be seen as follows. We have
	\begin{align*}
		\dim R^1f_*\Oo_C=\dim_{\KK(s)}(R^1f_*\Oo_C)(s)=\dim_{\ov{\KK(s)}}H^1\left(C_{\ov{s}},\Oo_{C_{\ov{s}}}\right)
	\end{align*}
	by \eqreff{eq:GeometricFibreIso} and Theorem~\reff{thm:GrauertGrothendieck}\itememph{d}. Now Serre duality gives
	\begin{align*}
		\dim_{\ov{\KK(s)}}H^1\left(C_{\ov{s}},\Oo_{C_{\ov{s}}}\right)=\dim_{\ov{\KK(s)}}H^0\left(C_{\ov{s}},\Omega_{C_{\ov{s}}}\right)=g\;,
	\end{align*}
	in which the right-most equality holds simply by definition (indeed, $g$ is supposed to be the $\ov{\KK(s)}$-dimension of $\Omega_{C_{\ov{s}}}(C_{\ov{s}})\cong H^0(C_{\ov{c}},\Omega_{C_{\ov{s}}})$). This proves \itememph{c}.
	
	Finally, let's see why the additional base change assertion holds. For \itememph{a} and \itememph{b}, we've seen that $P^\bullet$ has only one non-vanishing term, so the required isomorphism is immediate. However, I wasn't able to find a equally down-to-the-earth argument in the case of \itememph{c}. Instead, consider the special case $S=\Spec A$ and $\snake{S}=\Spec B$ (the assertion is local, so that's fine). Then the base change spectral sequence (cf.\ Proposition~\reff{prop:baseChangeMorphismA})
	\begin{align*}
		E_2^{p,q}=\Tor_{-p}^A\big(H^q(C,\Oo_C),B\big)\converge H^{p+q}(\snake{C},\Oo_{\snake{C}})\;.
	\end{align*}
	immediately degenerates because we just showed that $f_*\Oo_C$ and $R^1f_*\Oo_C$ are vector bundles (and the higher derived images vanish anyway), so the base change morphism is an isomorphism in this case as well.
\end{proof}
\begin{rem}
	When $g=d=1$ we have a line bundle $\Ll_0=f_*\Ll$ on $S$ with a canonical morphism $f^*\Ll_0=f^*f_*\Ll\morphism\Ll$ as $f^*$ is left-adjoint to $f_*$. It will turn out (in Proposition~\reff{prop:canonicalSection}) there is a unique section $\sigma\colon S\morphism C$ such that said canonical morphism constitutes an isomorphism $f^*\Ll_0\isomorphism\Ii\Ll$, where $\Ii$ is the sheaf of ideals defined by the closed subprescheme $\sigma\colon S\monomorphism C$ of $C$.
	
	Wait \ldots\ why is $\sigma$ a closed embedding? Note that $f$ is proper, hence separated, so any section of it is a closed embedding by \cite[Proposition~1.5.5]{alggeo1}.
\end{rem}
\begin{lem}\lbl{lem:I(s)LineBundle}
	Let $f\colon C\morphism S$ be a flat family of curves and let $\sigma\colon S\morphism C$ be a section of $f$. Let $\Ii=\Ii(\sigma)\subseteq \Oo_C$ denote the sheaf of ideals defining the closed subprescheme $\sigma(S)$ of $C$. Then $\Ii$ is a line bundle.
\end{lem}
\begin{proof}
	The assertion is local, hence it is sufficient to prove it near any $c\in C$. When $c\neq \sigma(f(c))$ (in other words, when $c\in C\setminus \sigma(S)$) this is trivial since $\Ii$ then equals $\Oo_C$ in some neighbourhood of $c$. Thus let $c=\sigma(s)$ where $s=f(c)$. We have a short exact sequence
	\begin{align}\lbl{eq:s(S)closedSES}
		0\morphism \Ii_c\morphism \Oo_{C,c}\morphism[\sigma^*]\Oo_{S,s}\morphism 0
	\end{align}
	(because $S\cong \sigma(S)$ is the closed subprescheme cut out by $\Ii$) of $\Oo_{S,s}$-modules. Since $\Oo_{C,c}$ is flat over $\Oo_{S,s}$ by assumption, this already shows that $\Ii_c$ is a flat module over $\Oo_{S,s}$ (with the $\Oo_{S,s}$-action given by $f^*$) via the long exact $\Tor$ sequence. Also we see that $\Oo_{C,c}$ and $\Oo_{S,s}\cong \Oo_{C,c}/\Ii_c$ have the same residue field $k$.
	
	To show local freeness of $\Ii$, we wish to show that $\Tor_1^{\Oo_{C,c}}(k,\Ii_c)=0$. Since the algebraic closure $\ov{k}$ is (as an $\Oo_{C,c}$-module) isomorphic to a direct sum of copies of $k$, we may equivalently show that $\Tor_1^{\Oo_{C,c}}(\ov{k},\Ii_c)=0$. To do so, consider the (homological) spectral sequence
	\begin{align}\lbl{eq:TorSS1}
		E_{p,q}^2=\Tor_p^{\Oo_{C_{\ov{s}},c}}\left(\ov{k},\Tor_q^{\Oo_{C,c}}(\Ii_c,\Oo_{C_{\ov{s}},c})\right)\converge\Tor_{p+q}^{\Oo_{C,c}}(\ov{k},\Ii_c)\;,
	\end{align}
	where $c$ is used to denote the unique preimage of itself in $C_{\ov{s}}$ as well (which makes sense by Lemma~\reff{lem:stalkOfGeometricFibres}).	One obtains \eqreff{eq:TorSS1} as a special case of the general spectral sequence
	\begin{align}\lbl{eq:TorGrothendieckSS}
		E_{p,q}^2=\Tor_p^B\left(N,\Tor_q^A(M,B)\right)\converge\Tor_{p+q}^A(M,N)\;,
	\end{align}
	in which $A$ may be any ring, $B$ an $A$-algebra, $M$ an $A$-module and $N$ a $B$-module. The sequence \eqreff{eq:TorGrothendieckSS} is in turn a special case of the Grothendieck spectral sequence. 
	
	Another special case of \eqreff{eq:TorGrothendieckSS} using $A=\Oo_{S,s}$, $B=\Oo_{C,c}$ (which becomes an $A$-module via $f^*$), $M=\ov{k}$, and $N$ arbitrary (for now) is
	\begin{align}\lbl{eq:TorSS2}
		E_{p,q}^2=\Tor_p^{\Oo_{C,c}}\left(N,\Tor_q^{\Oo_{S,s}}(\ov{k},\Oo_{C,c})\right)\converge\Tor_{p+q}^{\Oo_{S,s}}(\ov{k},N)\;.
	\end{align}
	But since $\Oo_{C,c}$ is flat over $\Oo_{S,s}$, we have $E_{p,q}^2=0$ unless $q=0$, so \eqreff{eq:TorSS2} collapses to an isomorphism
	\begin{align}\lbl{eq:TorSS3}
	 	\Tor_p^{\Oo_{C,c}}(N,\Oo_{C_{\ov{s}},c})\cong\Tor_p^{\Oo_{C,c}}(N,\Oo_{C,c}\otimes_{\Oo_{S,s}}\ov{k})\cong\Tor_p^{\Oo_{S,s}}(\ov{k},N)
	\end{align}
	(the isomorphism on the left-hand side is due to Lemma~\reff{lem:stalkOfGeometricFibres}). Plugging in $N=\Ii_c$, which is a flat $\Oo_{S,s}$-module, we obtain $\Tor_p^{\Oo_{C,c}}(N,\Oo_{C_{\ov{s}},c})=0$ when $p\neq 0$. Thus \eqreff{eq:TorSS1} immediately degenerates to an isomorphism
	\begin{align*}
		\Tor_p^{\Oo_{C_{\ov{s}},c}}(\ov{k},\Ii_c\otimes_{\Oo_{C,c}}\Oo_{C_{\ov{s}},c})\cong\Tor_p^{\Oo_{C,c}}(\ov{k},\Ii_c)\;.
	\end{align*}
	Applying \eqreff{eq:TorSS3} to $N=\Ii_c$ and $p=0$ gives $\Ii_c\otimes_{\Oo_{C,c}}\Oo_{C_{\ov{s}},c}\cong \Ii_c\otimes_{\Oo_{S,s}}\ov{k}$. Since $\Ii_c$ is flat over $\Oo_{S,s}$, \eqreff{eq:s(S)closedSES} gives an exact sequence
	\begin{align*}
		0=\Tor_1^{\Oo_{S,s}}(\ov{k},\Ii_c)\morphism \Ii_c\otimes_{\Oo_{S,s}}\ov{k}\morphism\Oo_{C,c}\otimes_{\Oo_{S,s}}\ov{k}=\Oo_{C_{\ov{s}},c}\;,
	\end{align*}
	hence $\Ii_c\otimes_{\Oo_{S,s}}\ov{k}$ is isomorphic to some ideal of $\Oo_{C_{\ov{s}},c}$. But the latter is a DVR (because we assume our fibres to be regular curves), so all ideals are projective, which finally shows $\Tor_p^{\Oo_{C,c}}(\ov{k},\Ii_c)=0$, so $\Tor_p^{\Oo_{C,c}}(k,\Ii_c)=0$. Therefore, $\Ii_c$ is a free $\Oo_{C,c}$-module by \cite[Proposition~1.3.1]{homalg}, which shows that $\Ii$ is locally free near $c$ (for which we need noetherianness of $C$ to have something like \cite[Corollary~1.5.1]{alg2} available).
	
	It remains to show that $\rank_{\Oo_C} \Ii=1$. If $c\notin \sigma(S)$, then $\Ii_c=\Oo_{C,c}$, so $\rank_{\Oo_{C,c}}\Ii_c=1$ is trivial. If $c\in s(S)$ and $\Ii_c\cong \Oo_{C,c}^{\oplus n}$ for some $n\geq 0$, then $\Ii_c\otimes_{\Oo_{C,c}}\Oo_{C_{\ov{s}},c}\cong \Oo_{C_{\ov{s}},c}^{\oplus n}$. But we've seen above that this is isomorphic to some ideal in the DVR $\Oo_{C_{\ov{s}},c}$, hence $n=1$.
\end{proof}
\begin{rem}
	As was already mentioned, equation \eqreff{eq:TorSS2} is a special case of the \emph{Grothendieck spectral sequence}
	\begin{align*}
		E_{p,q}^2=L_pG\big(L_qF(X)\big)\converge L_{p+q}(GF)(X)\;.
	\end{align*}
	This requires $F\colon \cat A\morphism \cat B$ and $G\colon \cat B\morphism\cat C$ to be right-exact functors between abelian categories, of which $\cat A$ and $\cat B$ have enough projectives (for the left-derived functors to exist) and $F$ takes projective objects in $\cat A$ to $G$-acyclic objects in $\cat B$; $X$ may be any object of $\cat A$.
\end{rem}
\begin{rem}
	If the geometric fibres $C_{\ov{s}}$ are regular curves, then so are the ordinary fibres $C_s$ (Franke didn't prove this in the lecture, but we do in Proposition~\reff{prop:GeometricFibres101}). In the proof of Lemma~\reff{lem:I(s)LineBundle} we could have worked with $C_s$ instead of $C_{\ov{s}}$ as well, so the assertion still holds if $f\colon C\morphism S$ is only assumed flat with one-dimensional regular fibres.
\end{rem}
\begin{lem}\lbl{lem:I(s)degree-1}
	The ideal $\Ii=\Ii(\sigma)$ in Lemma~\reff{lem:I(s)LineBundle} has automatically (geometric and ordinary) fibre-wise degree $1$, that is, its pullbacks to the geometric and ordinary fibres have degree $-1$.
\end{lem}
\begin{proof}
	Indeed, let $i_s\colon C_s\morphism C$ and $i_{\ov{s}}\colon C_{\ov{s}}\morphism C$ be the inclusion of the fibres. If $D=\sigma(S)$ is the closed subprescheme cut out by $\Ii$, then $D\cap C_s=\{\sigma(s)\}$ has precisely one point for all $s\in S$. Put $d=\sigma(s)$ and $k=\KK(s)$. Since $\Ii=\Oo_C$ outside $D$, it follows that $\Ii_c\otimes_{\Oo_{S,s}}k\cong \Oo_{C_s,c}$ for all $c\neq d$. Moreover, $\Ii_d\otimes_{\Oo_{S,s}}k$ is a maximal ideal of $\Oo_{C_s,d}\cong \Oo_{C,c}\otimes_{\Oo_{S,s}}k$ since
	\begin{align*}
		(\Oo_{C,d}\otimes_{\Oo_{S,s}}k)/(\Ii_d\otimes_{\Oo_{S,s}}k)\cong (\Oo_{C,d}/\Ii_d)\otimes_{\Oo_{S,s}}k\cong \Oo_{S,s}\otimes_{\Oo_{S,s}}k\cong k\;.
	\end{align*}
	Thus $v_c(i_s^*\Ii)=0$ if $c\neq d$ and $v_d(i_s^*\Ii)=1$, which means that $i_s^*\Ii$ has degree $-1$.
	
	For the geometric fibre, we can use the exact same argument since $C_{\ov{s}}$ has a unique point over $d$ (which we denote $d$ too, as before) by Lemma~\reff{lem:stalkOfGeometricFibres}, and $\Ii_d\otimes_{\Oo_{C,d}}\Oo_{C_{\ov{s}},d}\cong \Ii_d\otimes_{\Oo_{S,s}}\ov{k}$ is a maximal ideal of $\Oo_{C_{\ov{s}},d}\cong \Oo_{C,d}\otimes_{\Oo_{S,s}}\ov{k}$ by the same reason as above, the only difference being that the quotient is $\ov{k}$ instead of $k$ this time.
\end{proof}
\begin{defi}\lbl{def:OC(s)}
	In the situation where $\sigma\colon S\monomorphism C$ defines a an invertible sheaf of ideals $\Ii(\sigma)$, we put $\Oo_C(\sigma)=\Ii(\sigma)^{-1}$.
\end{defi}
\begin{rem}
	We will generalize this to $\Oo_C(D)$, where $D\subseteq C$ is finite and flat over $S$. The letter $D$ is purposely chosen: You should think of $D$ as an \emph{effective relative Cartier divisors}. Professor Franke suggests to read the first chapter of Katz/Mazur, \emph{Arithmetic Moduli of Elliptic Curves} \cite{katzmazur} -- in particular, the definition of effective relative Cartier divisors and the group structure on elliptic curves (which we are going to sketch now) are explained there.
\end{rem}
\begin{prop}\lbl{prop:canonicalSection}
	Let $f\colon C\morphism S$ (with $S$ locally noetherian) be a flat family of curves of genus $g=1$ and $\Ll$ a line bundle of geometric fibre-wise degree $1$ on $C$. Then $\Ll_0=f_*\Ll$ is a line bundle on $S$, and there is a unique section $\sigma\colon S\morphism C$ of $f$ such that the canonical morphism $f^*\Ll_0=f^*f_*\Ll\morphism\Ll$ (corresponding to $\id_{f_*\Ll}$ under the $f^*$-$f_*$ adjunction) is a monomorphism with image $\Ii(\sigma)\Ll$. In particular, we have a canonical isomorphism
	\begin{align*}
		f^*\Ll_0\otimes_{\Oo_C}\Oo_C(\sigma)\isomorphism\Ll\;,
	\end{align*}
	using that $\Ii(\sigma)\otimes_{\Oo_C}\Oo_C(\sigma)\cong \Oo_C$ by Definition~\reff{def:OC(s)}.
\end{prop}
\begin{proof}
	The assertion is local\footnote{\textbf{Attention!} This is not a priori obvious, but can be seen as follows: By the uniqueness assertion we see that local sections of $f$ with the required property glue well to a section $\sigma$ defined on all of $S$ (and $\sigma$ is automatically unique). Also the isomorphism $f^*\Ll_0\otimes_{\Oo_C}\Oo_C(\sigma)\cong\Ll$ is going to be canonical, hence gluing shouldn't be a problem.} with respect to $S$, hence we may assume that $S=\Spec A$ is affine. Also the claim that $\Ll_0=f_*\Ll$ is a line bundle is just Proposition~\reff{prop:RpLl}\itememph{a}, so without losing generality $\Ll_0$ is trivial with $\lambda\in\Ll_0(S)$ a free generator. Then $\lambda$ is also an element of $f_*\Ll_0(S)=\Ll(C)$ and it's easy to check that $f^*f_*\Ll$ is isomorphic to the subbundle of $\Ll$ generated by $\lambda$. Let $i\colon D\monomorphism C$ be the closed subscheme defined by the vanishing set of $\lambda$, or in other words, by the short exact sequence
	\begin{align}\lbl{eq:SESforD}
		0\morphism\Ll^{-1}\xrightarrow{\lambda\otimes-}\Oo_C\morphism i_*\Oo_D\morphism 0\;.
	\end{align}
	\begin{claim}\lbl{claim:DcongS}
		The composition $g=fi\colon D\morphism S$ is an isomorphism.
	\end{claim}
	Believing this for the moment, the proof can be finished as follows. Note that \eqreff{eq:SESforD} induces an isomorphism $\Ll^{-1}\isomorphism\Jj$ onto some quasi-coherent sheaf of ideals $\Jj\subseteq \Oo_C$. One readily checks that $\Jj$ is the uniquely defined ideal such that $\Jj\Ll$ is the subbundle generated by $\lambda$, i.e.\ the image of $f^*f_*\Ll$. So $\sigma$ -- if it exists -- necessarily identifies $S$ with $D$ via the composition $g=fi$. This settles the uniqueness part, but also existence is immediate: Since we know that $g$ is an isomorphism, we \emph{actually can} choose $\sigma=ig^{-1}\colon S\isomorphism D\monomorphism C$, which gives $\Ii(\sigma)=\Jj$ whence we're be done.
	
	To show that $g$ is an isomorphism, first apply the long exact cohomology sequence for $R^\bullet f_*$ to \eqreff{eq:SESforD} to obtain a six-term exact sequence
	\begin{align}\lbl{eq:sixtermRpSequence}
		0\morphism f_*\Ll^{-1}\morphism f_*\Oo_C\morphism f_*i_*\Oo_D\morphism R^1f_*\Ll^{-1}\morphism R^1f_*\Oo_C\morphism R^1f_*(i_*\Oo_D)\morphism 0\;.
	\end{align}
	The $0$ on the right end comes from the fact that $R^2f_*\Ll^{-1}=0$ (we've seen this for $\Ll^{-1}=\Oo_C$ in the proof of Proposition~\reff{prop:RpLl}\itememph{c}; it's just the same for arbitary $\Ll^{-1}$). Also note that $f_*\Ll^{-1}=0$ and $f_*\Oo_C\cong\Oo_S$ by Proposition~\reff{prop:RpLl}\itememph{b} and \itememph{c}.
	\begin{claim}\lbl{claim:DCxfiniteIntersection}
		For every $s\in S$, the intersection $D\cap C_s=g^{-1}\{s\}$ is finite.
	\end{claim}
	To prove Claim~\reff{claim:DCxfiniteIntersection}, we first remark that $\Ll|_{C_s}(C_s)$ is a one-dimensional $\KK(s)$-vector space. To show this, it suffices to prove the same for $H^0(C_{\ov{s}},\Ll|_{C_{\ov{s}}})$ by Remark~\reff{rem:GeometricFibres}\itememph{a}. If we could show $H^1(C_{\ov{s}},\Ll|_{C_{\ov{s}}})=0$, then the required one-dimensionality follows from $\deg(\Ll|_{C_{\ov{s}}})=1$ and Riemann--Roch. 
	
	To see the vanishing of the $1\ordinalst$ cohomology of $\Ll|_{C_{\ov{s}}}$, we use that $H^2(C_{\ov{s}},\Ll|_{C_{\ov{s}}})=0$ for all $s\in S$ by Grothendieck's dimension theorem. Hence $U_2=S$ in Theorem~\reff{thm:GrauertGrothendieck}\itememph{d}. Then Theorem~\reff{thm:GrauertGrothendieck}\itememph{e} shows $U_1=S$ as well, since $R^1f_*\Ll=0$ is a vector bundle by Proposition~\reff{prop:RpLl}\itememph{a}. But then $0=(R^1f_*\Ll^{-1})(s)\cong H^1(C_s,\Ll|_{C_s})$ for all $s\in S$. Together with Remark~\reff{rem:GeometricFibres}\itememph{a} this shows $H^1(C_{\ov{s}},\Ll|_{C_{\ov{s}}})=0$, as required.
	
	Now $\lambda|_{C_s}$ is a generator of the one-dimensional $\KK(s)$-vector space $\Ll|_{C_s}(C_s)$ by the base change assertion from Proposition~\reff{prop:RpLl}. In particular, $\lambda|_{C_s}\neq 0$. Since $C_s$ is a regular connected curve by Corollary~\reff{cor:FibresAreCurvesToo}, thus reduced, irreducible, and one-dimensional, this shows that $D\cap C_s\subsetneq C_s$ must be zero-dimensional. Then it is already finite. Indeed, $D\cap C_s$ is quasi-compact (as a closed subscheme of the irreducible $C_s$), hence can be covered by finitely many $U_i=\Spec R_i$ where $R_i$ is some zero-dimensional noetherian ring. But such a ring has finitely many prime ideals, as follows e.g.\ from Step~2 in the proof of \cite[Proposition~3.1.1]{alg2}. This settles Claim~\reff{claim:DCxfiniteIntersection}.
	
	In particular, we see that $g$ is quasi-finite. It is also proper because $g=fi$, wherein $f$ is proper and $i$ a closed embedding, hence $g$ is even finite by Theorem~\reff{thm:ZariskiMain}\itememph{a}. As $i$ is affine, being a closed embedding, we see that $R^1q_*\Oo_D\cong R^1f_*(i_*\Oo_D)$ (this follows e.g.\ from Leray's spectral sequence $R^pf_*R^qi_*\Oo_D\converge R^{p+q}(fi)_*\Oo_D$, which degenerates as $R^qi_*\Oo_D=0$, $i$ being affine). But $q$ is affine as well, so $R^1f_*(i_*\Oo_D)=0$. Thus \eqreff{eq:sixtermRpSequence} reduces to 
	\begin{align}
		0\morphism \Oo_S\morphism q_*\Oo_D\morphism R^1f_*\Ll^{-1}\morphism R^1f_*\Oo_C\morphism 0\;.
	\end{align}
	By Proposition~\reff{prop:RpLl}\itememph{b} and \itememph{c}, we see that $R^1f_*\Ll^{-1}$ and $R^1f_*\Oo_C$ are line bundles. But every epimorphism of line bundles must be an isomorphism (which is easily checked affine-locally), so $\Oo_S\morphism q_*\Oo_D$ is an isomorphism of $\Oo_S$-modules. But then it is an isomorphism of $\Oo_S$-algebras. Using that $q$ is affine, we thus obtain $D\cong \SPEC (q_*\Oo_D)\cong \SPEC\Oo_S\cong S$. This shows Claim~\reff{claim:DcongS} and we're done.
\end{proof}
\begin{rem}
	Note that despite Proposition~\reff{prop:canonicalSection}, a flat family $f\colon C\morphism S$ of genus $1$ curves may fail to have a section (think of an elliptic curve without a rational point over $\IQ$). An elliptic curve $E$ over $S$ may always be obtained from $C$, but there may be several non-isomorphic $C$ giving the same $E$. When $S=\Spec K$ where $K/\IQ$ is a number field, then all these $C$ are isomorphic over $\ov{\IQ}$ and their isomorphism classes (over $K$) form a group, the \defemph{Tate--Shafarevich group} $\Sha(E/K)$.
\end{rem}
	Let $f\colon E\morphism S$ with a section $\epsilon\colon S\morphism E$ be an elliptic curve over $S$. Then Proposition~\reff{prop:canonicalSection} may be used to obtain a bijection
	\begin{align*}
		\Pic^0(E/S)\coloneqq\left\{\begin{tabular}{c}
			isomorphism classes of line bundles\\
			$\Ll$ over $E$ of geometric fibre-wise de-\\
			gree 0 with a trivialization $\epsilon^*\Ll\cong\Oo_S$
		\end{tabular}\right\}\isomorphism\left\{\text{sections }\sigma\colon S\to E\text{ of }f\right\}\;.
	\end{align*}
	Note that \cite[pp.~63--67]{katzmazur} define $\Pic^0(E/S)$ as the quotient of the group $\Pic^0(E)$ of line bundles with geometric fibre-wise degree $0$ by the subgroup of line bundles of the form $f^*\Ll_0$ where $\Ll_0$ is a line bundle on $S$. These definitions are equivalent by sending $\Ll\mapsto \Ll\otimes_{\Oo_E}f^*\big((\epsilon^*\Ll)^{-1}\big)$. We'll work with the Katz/Mazur definition, since this is somewhat easier.
	
	The map $\Pic^0(E/S)\morphism\{\text{sections }\sigma\text{ of }f\}$ sends $\Ll$ to the canonical section $\sigma$ associated to $\Ll\otimes_{\Oo_E}\Oo_E(\epsilon)$ by Proposition~\reff{prop:canonicalSection} (note that this makes sense because $\Ll$ has fibre-wise degree $0$ and $\Ii(\epsilon)$ has fibre-wise degree $-1$ by Lemma~\reff{lem:I(s)degree-1}, hence $\Ll\otimes_{\Oo_E}\Oo_E(\epsilon)$ has fibre-wise degree $1$). The inverse map sends a section $\sigma$ to $\Ii(\epsilon)\otimes_{\Oo_E}\Oo_E(\sigma)$.
	
	To show that these guys are indeed inverse, let $\Ll\in\Pic^0(E/S)$ and $\sigma$ the section of $f$ associated to $\Ll\otimes_{\Oo_E}\Oo_E(\epsilon)$. By Proposition~\reff{prop:canonicalSection}, we have
	\begin{align*}
		f^*f_*\left(\Ll\otimes_{\Oo_E}\Oo_E(\epsilon)\right)\otimes_{\Oo_E}\Oo_E(\sigma)\cong\Ll\otimes_{\Oo_E}\Oo_E(\epsilon)\;.
	\end{align*}
	But $f^*f_*\left(\Ll\otimes_{\Oo_E}\Oo_E(\epsilon)\right)$ vanishes in $\Pic^0(E/S)$ (by definition), so $\Ll=\Ii(e)\otimes_{\Oo_E}\Oo_E(\sigma)$ in $\Pic^0(E/S)$. This shows that the composition $\Pic^0(E/S)\morphism\{\text{sections }\sigma\text{ of }p\}\morphism\Pic^0(E/S)$ is the identity. The other direction can be shown in quite the same way.
	
	Let $\cat C$ be a full subcategory of $\cat{PSch}/S$ containing $E$, closed under fibre products over $S$, and such that all its objects are locally noetherian (we could save ourselves these technicalities by working with $f$ of finite presentation). For any $S$-prescheme $T$ let $E(T)$ denote the set of morphisms $T\morphism E$ in $\cat{PSch}/S$. Then $E(T)$ is in canonical bijection with the set of sections of the base change $f_T\colon E_T=T\times_SE\morphism T$ of $f$ and $E(-)=\Hom_{\cat{PSch}/S}(-,E)$ is a contravariant functor $(\cat{PSch}/S)^\op\morphism\cat{Set}$. The set $E(T)$ should be thought of and is called the set of \defemph{$T$-valued points} of $E$.
	
	%This functor can be introduced for any $S$-prescheme $E$ (and not only those that happen to be elliptic curves over $S$). We have, e.g., $(E\times_SF)(Y)=E(Y)\times F(Y)$ by the universal property of fibre products.
	If $T$ is in $\cat C$, then we get canonical isomorphisms
	\begin{align*}
		E(T)\cong\left\{\text{sections of }f_T\colon E_T\to T\right\}\cong\Pic^0(E_T/T)\;.
	\end{align*}
	The right-hand side forms an abelian group with $-\otimes_{\Oo_E}-$. Therefore the functor (or rather its restriction to $\cat C$) $E(-)\colon \cat C^\op\morphism\cat{Set}$ factors over $\cat{Ab}$. That is, $E(-)$ is an \emph{abelian group object} in the functor category $\Funct(\cat{C}^\op,\cat{Set})$. Since $\cat C^\op$ is a full subcategory of $\Funct(\cat{C}^\op,\cat{Set})$ via the Yoneda embedding, this means that $E$ is already an \emph{abelian group object} in $\cat C$.
	
	Explicitly, this means the following: There are morphisms $m\colon E\times_SE\morphism E$ and $\iota\colon E\morphism E$ of $S$-preschemes (the \emph{multiplication} and the \emph{inversion}) such that the following diagrams commute (which is an abstract way of proposing that the ``group axioms'' hold for $E$):
	\begin{center}
		\begin{minipage}[b]{0.4\textwidth}
			\centering
			\begin{diagram*}
				\object{0,1.5}{$E\times_SE\times_SE$}[EEE];
				\object{0,0}{$E\times_SE$}[EE];
				\object{3.5,1.5}{$E\times_SE$}[ExE];
				\object{3.5,0}{$E$}[E];
				\scriptsize
				\arrow{EEE}{ExE}[above][$(\id_E,m)$];
				\arrow{EEE}{EE}[left][$(m,\id_E)$];
				\arrow{EE}{E}[above][$m$];
				\arrow{ExE}{E}[right][$m$];
			\end{diagram*}
			(associativity)
		\end{minipage}
		\begin{minipage}[b]{0.5\textwidth}
			\centering
			\begin{diagram*}
				\object{0,1.5}{$S\times_SE$}[SE];
				\object{2.5,1.5}{$E$}[E1];
				\object{5,1.5}{$E\times_SS$}[ES];
				\object{0,0}{$E\times_SE$}[EE1];
				\object{2.5,0}{$E$}[E2];
				\object{5,0}{$E\times_SE$}[EE2];
				\scriptsize
				\isoarrow{E1}{SE};
				\isoarrow{E1}{ES};
				\arrow{SE}{EE1}[left][$(\epsilon,\id_E)$];
				\arrow{ES}{EE2}[right][$(\id_E,\epsilon)$];
				\arrow{EE1}{E2}[above][$m$];
				\arrow{EE2}{E2}[above][$m$];
				\draw[transform canvas={xshift=1pt}] (E1) -- (E2);
				\draw[transform canvas={xshift=-1pt}] (E1) -- (E2);
			\end{diagram*}
			($e$ is a left- and right-neutral element)
		\end{minipage}
			\begin{minipage}[b]{0.4\textwidth}
			\centering
			\begin{diagram*}
				\object{0,1.5}{$E\times_SE$}[EE1];
				\object{3.5,1.5}{$E\times_SE$}[EE2];
				\object{1.75,0}{$E$}[E];
				\scriptsize
				\draw[->] (EE1) -- (EE2) node[pos=0.5,above] {exchange} node[pos=0.5,below] {factors};
				\arrow{EE1}{E}[below left][$m$];
				\arrow{EE2}{E}[below right][$m$];
			\end{diagram*}
			(commutativity)
		\end{minipage}
		\begin{minipage}[b]{0.5\textwidth}
			\centering
			\begin{diagram*}
				\object{0,2}{$E\times_SE$}[SE];
				\object{2.5,2}{$E\times_SE$}[E1];
				\object{5,2}{$E\times_SE$}[ES];
				\object{2.5,1}{$S$}[S];
				\object{2.5,0}{$E$}[E2];
				\scriptsize
				\arrow{E1}{SE}[above][$(\iota,\id_E)$];
				\arrow{E1}{ES}[above][$(\id_E,\iota)$];
				\arrow{E1}{S};
				\arrow{S}{E2}[right][$\epsilon$];
				\arrow{SE}{E2}[below left][$m$];
				\arrow{ES}{E2}[below right][$m$];
			\end{diagram*}
			($\iota$ is a left- and right-inverse)
		\end{minipage}
	\end{center}
	For the definition of a group object we don't need $E$ to be an elliptic curve (actually, this works in any category $\cat C$ with finite products and a final object). A group object in some prescheme category is called a \defemph{group prescheme}.
	
Now consider the more general situation where $f\colon C\morphism S$ is only assumed to be a flat family of curves of genus $g$ together with a section $\sigma\colon S\morphism C$. When $g=1$ (i.e.\ we have an elliptic curve over $S$) we've seen above that $C$ represents the functor $\Pic_{C/S}^{0}\colon \cat{C}^\op\morphism\cat{Ab}$ defined by $\Pic_{C/S}^0(T)=\Pic^0(C_T/T)$ for all locally noetherian $S$-preschemes $T$ (where $C_T$ denotes the base change $T\times_SC$ as above). Dropping the fibre-wise degree $0$ condition, we get a functor $\Pic_{C/S}\colon \cat{C}^\op\morphism\cat{Ab}$, which in the case of an elliptic curve $E=C$ is represented by $\coprod_{n\in\IZ}E$, with one component for every fibre-wise degree a line bundle can have.

If one is only interested in $\Pic_{C/S}$, then the conditions on $C$ (properness, flatness, one-dimensional geometric fibres) could be dropped, and so can the restriction to $\cat{C}$. We also don't need to assume a section $\sigma\colon S\morphism C$ exists if we define $\Pic_{C/S}(T)$ in the clever way (like Katz/Mazur do) as
\begin{align}\lbl{eq:PicCS}
	\Pic_{C/S}(T)=\Pic(C_T)/f_T^*\big(\Pic(T)\big)
\end{align}
rather than in terms of trivializations. This gives a functor $\Pic_{C/S}\colon (\cat{PSch}/S)^\op\morphism\cat{Ab}$.

However, if one moreover tries to get rid of the quotient in \eqreff{eq:PicCS}, one cannot hope to end up with a representable functor $\Pic_C\colon\cat{C}^\op\morphism\cat{Ab}$ defined by $\Pic_C(T)=\Pic(C_T)$. Indeed, suppose on the contrary there was a representing object $\PIC_C$. If $\Ll$ is a line bundle on $C_T$ and $\Ll_0$ a line bundle on $T$, then $\Ll$ and $\Ll\otimes_{\Oo_{C_T}}f_T^*\Ll_0$ are usually not isomorphic. Yet locally with respect to $T$ they are, as $\Ll_0$ is locally trivial. Hence if $T=\bigcup_{i\in I}U_i$ is a sufficiently fine open cover, then the morphisms $T\morphism\PIC_C$ defined by (the isomorphism classes of) $\Ll$ and $\Ll\otimes_{\Oo_{C_T}}f_T^*\Ll_0$ coincide when restricted to $f_T^{-1}(U_i)$ for all $i\in I$. But then they coincide on all of $T$, which implies that $\Ll$ and $\Ll\otimes_{\Oo_{C_T}}f_T^*\Ll_0$ are isomorphic, contradiction!

Thus $\Pic_C$ is a hopeless candidate for a representable functor, and we've seen that the main problem is basically its lack of ``sheaf-like'' properties. So the natural thing to ask is whether $\Pic_C$ can be ``sheafified'' in an appropriate way. What makes things complicated here is that the Zariski topology might not be suited to do this, at least when no section $\sigma\colon S\morphism C$ exists. That's where the \emph{étale} and \emph{fpqc topology} come in to save the day. We'll briefly introduce these notions in the proceeding of the lecture.

Also note that ``sheafifying'' the $\Pic_C$ functor should give the same result as with $\Pic_{C/S}$. Indeed, the ``sheafification'' of $\Pic_{C/S}$ should be the ``sheaf quotient'' of the ``sheafifications'' of $\Pic_C$ and of $f_-^*(\Pic(-))\colon(\cat{PSch}/S)^\op\morphism\cat{Ab}$. But $f_T^*(\Pic(T))$ locally trivial on the base $T$ for every $S$-prescheme $T$, so $f_-(\Pic(-))$ should have trivial ``sheafification''.

The strategy for representing $\Pic_{C/S}$ by a prescheme $\PIC_{C/S}$ for a flat family $f\colon C\morphism S$ of curves of genus $g$ will be to look at an open subset $\PIC_{C/S}^g$ of line bundles for which (in the case where $S$ is a field) $\Ll(C)$ is a one-dimensional vector space (it always has dimension at least $1$). Then the open subpreschemes of $\PIC_{C/S}$ which are constructed in this way must be glued in a suitably.

If $\Ll$ is a line bundle of fibre-wise degree $g$ and $f_*\Ll$ is a line bundle on $S$, then the proof of Proposition~\reff{prop:canonicalSection} will only give us a unique subscheme $D\subseteq C$ which is finite and flat of degree $g$ over $S$. These $D$ (we already dropped the name \emph{effective relative Cartier divisor}) turn out to be representable by the \emph{symmetric power} $C^{(g)}\coloneqq C^{g}/\SS_g$, where $C^{g}$ is taken in the categorie of $S$-schemes (i.e., it is the $g$-fold fibre product of $C$ with itself over $S$) and the symmetric group $\SS_g$ acts by permuting factors. This will be our battle plan:
\begin{enumerate}
	\item Construct $C^{(g)}$.
	\item Show that they represent effective relative Cartier divisors.
	\item Use this to represent an open subgroup of $\PIC_{C/S}$.
	\item Glue these open subgroups.
\end{enumerate}
\section{The seesaw theorem (and potentially the theorem of the cube)}
As literature for this section Professor Franke recommends \cite{mumford1974abelian}, \cite{cornell1986arithmetic}, and \cite{kleiman}, as well as Grothendieck's FGA.

Note that even without knowing whether a representing object $\PIC_{C/S}$ of $\Pic_{C/S}$ exists, we can still decide whether it is separated over $S$, since this property can be read off the functor $\Hom_{\cat{PSch}/S}(-,\PIC_{C/S})\cong\Pic_{C/S}$. From that point of view, the seesaw theorem simply states that under certain (essential) assumptions on $f\colon C\morphism S$, the morphism $\PIC_{C/S}\morphism S$ is a \emph{separated} morphism. Before we prove this we'll need some preparations.
\begin{prop}\lbl{prop:trivialLineBundlesOnProperX/k}
	Let $k$ be an algebraically closed field and $X$ an integral proper $k$-scheme.
	\begin{alphanumerate}
		\item We have $\Oo_X(X)\cong k$.
		\item A line bundle $\Ll$ on $X$ is trivial iff $\Ll(X)$ and $\Ll^{-1}(X)$ are both non-zero (and in this case they're both one-dimensional).
	\end{alphanumerate}
\end{prop}
\begin{proof}
	Part \itememph{a}. Since $X$ is integral, $\Oo_X(X)$ is a domain. It is also a finite-dimensional $k$-algebra by \cite[Theorem~5]{alggeo2} as $X/k$ is proper. Therefore $\Oo_X(X)$ is a finite field extension of $k$, hence isomorphic to $k$ itself.
	
	Part \itememph{b}. If $s\in\Ll(X)$, $t\in\Ll^{-1}(X)$ are non-vanishing global sections, then $U=X\setminus(V(s)\cup V(t))$ is an open dense subset of $X$ (where we use that $X$ is irreducible). The global section $s\otimes t$ of $\Ll\otimes_{\Oo_X}\Ll^{-1}\cong \Oo_X$ doesn't vanish on $U$, hence is given by some $\kappa\in k^\times$ (using \itememph{a}). Then
	\begin{align*}
		s\otimes-\colon \Oo_X\morphism \Ll\quad\text{and}\quad-\otimes\kappa^{-1}t\colon \Ll\morphism\Oo_X
	\end{align*}
	are mutually inverse morphisms, proving $\Ll\cong \Oo_X$.
\end{proof}
Now let $f\colon X\morphism S$ be proper and $\Ff$ a coherent $\Oo_X$-module which is flat over $\Oo_S$. Recall
\begin{align*}
	U_p&=\left\{s\in S\st(R^pf_*\Ff)(s)\to H^p(X_s,\Ff|_{X_s})\text{ is surjective}\right\}\\
	&=\left\{s\in S\st(R^pf_*\Ff)(s)\to H^p(X_s,\Ff|_{X_s})\text{ is injective}\right\}
\end{align*}
is open in $S$ by Theorem~\reff{thm:GrauertGrothendieck}\itememph{d}. Also we defined
\begin{align*}
	h^p(s,\Ff)=\dim_{\KK(s)}H^p\left(X_s,\Ff|_{X_s}\right)=\dim_{\ov{\KK(s)}}H^p\left(X_{\ov{s}},\Ff|_{X_{\ov{s}}}\right)
\end{align*}
(the right equality uses Remark~\reff{rem:GeometricFibres}\itememph{a}).
\begin{prop}\lbl{prop:LineBundlesFromTheBase}
	Let $f\colon X\morphism S$ be a proper flat morphism with integral geometric fibres $X_{\ov{s}}$ (in particular, the geometric fibres are non-empty) and $\Ll$ a line bundle on $X$. Also assume that $S$ is locally noetherian. Then the following are equivalent.
	\begin{alphanumerate}
		\item $\Ll\cong f^*\Ll_0$ for some $\Ll_0\in\Pic(S)$ -- that is, $\Ll$ is trivial in $\Pic(X/S)$.
		\item For both $\Ll$ and $\Ll^{-1}$, the functions $h^0\left(-,\Ll^{\pm 1}\right)$ don't vanish on $S$ and $U_0=S$.
	\end{alphanumerate}
	If this is the case, $h^0\left(s,\Ll^{\pm1}\right)=1$ for all $s\in S$ and the canonical morphism $f^*f_*\Ll\morphism\Ll$ is an isomorphism. Also $\Ll_0\cong f_*\Ll$ in this case.
\end{prop}
\begin{proof}
	Assume for the moment that \itememph{a} and \itememph{b} are equivalent. We'll prove the additional assertion about $\Ll_0$ and $f_*\Ll$, i.e.\ that $f^*\Ll_0\cong \Ll$ implies that the corresponding morphism (under the $f^*$-$f_*$ adjunction) $\Ll_0\morphism f_*\Ll$ is an isomorphism. This is true more generally for a morphism $\Ff\morphism\Ff'$ of quasi-coherent $\Oo_S$-modules which becomes an isomorphism after applying $f^*$ (and this can be applied to $\Ll_0$ and $f_*\Ll$ as we assume \itememph{a} $\Rightarrow$ \itememph{b}). Indeed, note that $f$ is faithfully flat (it is flat and its geometric fibres are non-empty, hence the ordinary fibres are non-empty too), hence for every $s\in S$ we may choose an $x\in X$ such that $s=f(x)$. Then $\Oo_{X,x}$ is faithfully flat over $\Oo_{S,s}$ (e.g.\ by \cite[\stackstag{00HP}]{stacks-project}), so $\Ff_s\otimes_{\Oo_{S,s}}\Oo_{X,x}\morphism\Ff'_s\otimes_{\Oo_{S,s}}\Oo_{X,x}$ being an isomorphism implies that $\Ff_s \morphism\Ff'_s$ is one as well.
	
	We prove \itememph{a} $\Rightarrow$ \itememph{b} first. Since the assertion is local on $S$, we may assume $\Ll_0\cong\Oo_S$, so that $\Ll\cong f^*\Oo_S\cong \Oo_X$. Then
	\begin{align*}
		h^0\big(s,\Ll^{\pm 1}\big)=\dim_{\ov{\KK(s)}}H^0\left(X_{\ov{s}},\Oo_{X_{\ov{s}}}\right)=1\quad\text{for all}\quad s\in S 
	\end{align*}
	by Proposition~\reff{prop:trivialLineBundlesOnProperX/k}. Then $H^0(X_s,\Oo_{X_s})$ is a one-dimensional $\KK(s)$-vector space as well the constant function $1$ is a generator. But $1$ is clearly in the image of $(f_*\Oo_X)(s)\morphism H^0(X_s,\Oo_{X_s})$, so $U_0=S$ by Theorem~\reff{thm:GrauertGrothendieck}\itememph{d}.
	
	Now for \itememph{b} $\Rightarrow$ \itememph{a}. Since $U_0=S$ (by assumption) and $U_{-1}=S$ (for trivial reasons), we see that $f_*\Ll$ is locally free by Theorem~\reff{thm:GrauertGrothendieck}\itememph{e}. To see that $f_*\Ll$ is one-dimensional, it suffices to show that $(f_*\Ll)(s)\cong H^0(X_s,\Ll|_{X_s})$ are one-dimensional $\KK(s)$-vector spaces for all $s\in S$. But condition \itememph{b} together with Proposition~\reff{prop:trivialLineBundlesOnProperX/k} (which is needed for both the middle and the right equality) implies that 
	\begin{align*}
		H^0\left(X_s,\Ll|_{X_s}\right)\otimes_{\KK(s)}\ov{\KK(s)}= \dim_{\ov{\KK(s)}}H^0\left(X_{\ov{s}},\Ll|_{X_{\ov{s}}}\right)=\dim_{\ov{\KK(s)}}H^0\left(X_{\ov{s}},\Oo_{X_{\ov{s}}}\right)=1\;,
	\end{align*}
	so $f_*\Ll$ is indeed a line bundle. 
	
	Having established this, all we need to do is proving that $f^*f_*\Ll\morphism\Ll$ is an isomorphism. This is a local question on $S$ again, so we may assume that $f_*\Ll$ is trivial. Let $\lambda\in\Ll(X)=f_*\Ll(S)$ 
	be a generator. Let $x\in X$ and $s=f(x)$. As $(f_*\Ll)(s)\cong H^0(X_s,\Ll|_{X_s})$ (because $U_0=X$) and the left-hand side is generated by the image of $\lambda$, we see that $\lambda|_{X_s}$ is a non-vanishing global section of $\Ll|_{X_s}$. But $\Ll|_{X_s}(X_s)\cong \KK(s)$ (as seen above), so in particular $\lambda|_{X_s}$ generates the stalk of $\Ll|_{X_s}$ at $x$, which is given by $\Ll_x\otimes_{\Oo_{S,s}}\KK(s)\cong \Ll_x/\mm_{S,s}\Ll_x$. Then $\lambda$ can't vanish in $\Ll_x/\mm_{X,x}\Ll$, so $x\notin V(\lambda)$. Therefore,
	\begin{align*}
		\lambda\otimes-\colon \Oo_X\morphism\Ll
	\end{align*}
	is an isomorphism at $x$. Since $x$ was chosen arbitrarily, this shows $f^*f_*\Ll\isomorphism\Ll$ and \itememph{a} follows. We're done.
\end{proof}
\begin{thm}[Seesaw theorem]\lbl{thm:seesaw}
	Let $f\colon X\morphism S$ be a flat proper morphism with integral (in particular, non-empty) geometric fibres $X_{\ov{s}}$ for $s\in S$, and $S$ is locally noetherian. If $\Ll$ is a line bundle on $X$, then there's a closed subprescheme $Z\subseteq S$ having the following universal property: For any locally noetherian $S$-prescheme $\tau\colon T\morphism S$, the pullback $\tau_X^*\Ll$ of $\Ll$ under
	\begin{align*}
		\tau_X\colon X_T=X\times_ST\morphism T
	\end{align*}
	satisfies the equivalent conditions of Proposition~\reff{prop:LineBundlesFromTheBase} iff $\tau$ factors over $Z\monomorphism S$.
\end{thm}
\begin{rem}
	\begin{alphanumerate}
		\item A related assertion of the general form \emph{You cannot express a non-trivial line bundle as a limit of trivial ones} was shown by A.\ Weil. See \cite{cornell1986arithmetic}, \cite{mumford1974abelian}, or the Wikipedia article for a historical discussion.
		\item Let us assume that $f$ has a section $\sigma\colon S\morphism X$ (so that it doesn't matter in which topology whe sheafify the Picard functor). Assume that $\Pic_{X/S}$ can be represented by an $S$-(group-)prescheme $\pi\colon\PIC_{X/S}\morphism S$. Let $\epsilon\colon S\morphism\PIC_{X/S}$ be the neutral section and denote by $\lambda\colon S\morphism\PIC_{X/S}$ the morphism corresponding to (the isomorphism class of) $\Ll\in\Pic_{X/S}(S)=\Pic(X/S)$.
		
		If $f_T^*\colon \Oo_T\isomorphism f_{T,*}\Oo_{X_T}$ is an isomorphism for all $S$-preschemes $T$ (i.e.\ $f_*\Oo_X\cong \Oo_S$ \emph{holds universally} -- for example, this is the case in the situation of Proposition~\reff{prop:RpLl}), then $\tau_X^*\Ll$ is trivial in $\Pic_{X/S}(T)=\Pic(X_T/T)$ iff the compositions
		\begin{align*}
			T\morphism[\tau]S\doublemorphism[\lambda][\epsilon]\PIC_{X/S}
		\end{align*}
		coincide (since that's how the functor isomorphism $\Pic_{X/S}\cong \Hom_{\cat{PSch}/S}(-,\PIC_{X/S})$ works), which is the case iff $\tau$ factors over $\Eq(\lambda,\epsilon)$. This shows that $Z=\Eq(\lambda,\epsilon)$ in Theorem~\reff{thm:seesaw}.
		
		In particular, the equalizer $\Eq(\lambda,\epsilon)$ is \emph{closed}! If you play around with the definitions a bit (you should use that $\Hom_{\cat{PSch}/S}(S,\PIC_{X/S})$ is an abelian group) this shows that $\pi\colon \PIC_{X/S}\morphism S$ is separated. This justifies the interpretation of the seesaw theorem from the beginning of the current section.
	\end{alphanumerate}
\end{rem}
\begin{proof}[Proof of Theorem~\reff{thm:seesaw}]
	\emph{Step 1.} We first define $Z$ as a set. Put
	\begin{align*}
		Z_+=\left\{s\in S\st h^0\big(s,\Ll\big)\text{ doesn't vanish}\right\}\quad\text{and}\quad Z_-=\left\{s\in S\st h^0\big(s,\Ll^{-1}\big)\text{ doesn't vanish}\right\}
	\end{align*}
	Put $Z=Z_+\cap Z_-$. Then
	\begin{align*}
		Z=\left\{s\in S\st h^0\big(s,\Ll^{\pm 1}\big)=1\right\}
	\end{align*}
	due to Proposition~\reff{prop:trivialLineBundlesOnProperX/k}). By Theorem~\reff{thm:GrauertGrothendieck}\itememph{b}, $Z_\pm$ and thus also $Z$ are closed subsets of $S$. Note that this preliminary definition of $Z$ doesn't cover the second condition from Proposition~\reff{prop:LineBundlesFromTheBase}\itememph{b}, i.e., that $U_0$ contains all of the base. This will be taken care of once we define the prescheme structure on $Z$. The proof now proceeds in three steps.
	
	 \emph{Step 2.} Let $\tau\colon T\morphism S$ be an $S$-prescheme such that $\tau_X^*\Ll$ is the trivial element in $\Pic_{X/S}(T)=\Pic(X_T/T)$. We show that the image of $\tau$ (as a map between topological spaces) is contained in $Z$.  Let $t\in T$ and $s=\tau(t)$. Then
	\begin{align*}
		X_{T,t}=X_T\times_T\Spec\KK(t)\cong X\times_S\Spec\KK(t)\cong X_s\times_{\Spec\KK(s)}\Spec\KK(t)\;,
	\end{align*}
	hence
	\begin{align*}
		H^\bullet\left(X_{T,t},\tau_X^*\Ll^{\pm 1}|_{X_{T,t}}\right)\cong H^\bullet\left(X_s,\Ll^{\pm 1}|_{X_s}\right)\otimes_{\KK(s)}\KK(t)
	\end{align*}
	by \eqreff{eq:tensoredCohomology} as $\KK(t)$ is a flat ring extension of $\KK(s)$ (it is free actually). Therefore $h\left(t,\tau_X^*\Ll^{\pm 1}\right)=1$ holds if and only if $h\left(s,\Ll^{\pm 1}\right)=1$. Since $h\left(t,\tau_X^*\Ll^{\pm 1}\right)=1$ for all $t\in T$ by Proposition~\reff{prop:LineBundlesFromTheBase} and the assumption on $\tau_X^*\Ll$, this shows $s\in Z$, as claimed.
	
	Conversely, if $\tau\colon T\morphism S$ is any $S$-prescheme such that the image of  $\tau$ is contained in $Z$, then the above argument shows $h\left(t,\tau_X^*\Ll^{\pm 1}\right)=1$ for all $t\in T$.
	
	\emph{Step 3.} We describe the prescheme structure on $Z$. This task is local on $S$ for the following reason: $Z$ (as a prescheme, not just a topological space) is uniquely determined by the universal property we imposed, and if $U\subseteq S$ is any open subset, then $Z\cap U$ has the corresponding universal property for $U$. So once once we proved that the universal property is fulfilled locally (which we'll do in Step~4), everything will glue together nicely. 
	
	Therefore, we may choose an $s\in Z$ and restrict ourselves to an affine open neighbourhood $U\cong \Spec A$ of $s\in S$, where $A$ is noetherian. Moreover, we may replace $A$ by some localization whenever we feel like it, as long as still $s\in\Spec A$. Moreover, we'll relax the assumption $s\in Z$ to only $s\in Z_+$ to describe a prescheme structure on $Z_+$ instead of $Z$. Since $Z_-$ can then be equipped with an analogous prescheme structure, this will define the required prescheme structure on $Z=Z_+\cap Z_-$.
	
	Now let's get to business. Let $P^\bullet$ be the cochain complex of $A$-modules from Theorem~\reff{thm:GrauertGrothendieck}\itememph{a}. As described on page~\pageref{page:Pfree}, we may assume that all $P^i$ are free (after replacing $A$ by some localization). Since $s\in Z_+$, we know that
	\begin{align*}
		H^0\left(X_s,\Ll|_{X_s}\right)\cong\ker\left(\delta_s^0=\delta^0\otimes \id_{\KK(s)}\colon P^0\otimes_A\KK(s)\morphism P^1\otimes_A\KK(s)\right)
	\end{align*}
	has dimension $1$ over $\KK(s)$. Let $\ov{e}_0, \ldots,\ov{e}_n$ be a basis of $P^0\otimes_A\KK(s)$ such that $\ov{e}_0$ is a basis of $\ker \delta_s^0$. Choose an arbitrary basis of $\ov{f}_0,\ldots,\ov{f}_m$ of $P^1\otimes_A\KK(s)$, where $m\geq n-1$. Then the matrix representation $\ov{D}$ of $\delta_s^0$ looks like
		\begin{align*}
	\begin{tikzpicture}[remember picture]
	\node at (0,0) {$\ov{D}=
		\begin{pmatrix}
		 0 & \tikzentry{a}{\phantom{0}} \\
		 \vdots & & \phantom{\ov{C}_1}\\
		 \vdots & & & \tikzentry{b}{\phantom{0}} &\\
		 \vdots & \tikzentry{c}{\phantom{0}} \\
		 0 &  & & \tikzentry{d}{\phantom{0}}
		\end{pmatrix}
		$};
		\draw[rounded corners=0.75] ($(a.north west) +(-2pt,3pt)$) rectangle ($(b.south east) +(2pt,-3pt)$) node[pos=0.5] {$\ov{C}_1$};
		\draw[rounded corners=0.75] ($(c.north west) +(-2pt,3pt)$) rectangle ($(d.south east) +(2pt,-3pt)$) node[pos=0.5] {$\ov{C}_2$};
	\end{tikzpicture}
	\end{align*}
	in which $\ov{C}_1$ is an $n\times n$ matrix and $\ov{C}_2$ is $(m-n+1)\times n$. Since $\ov{D}$ has rank $n$ by assumption, it must contain an invertible $n\times n$ minor. So permuting $\ov{f}_0,\ldots,\ov{f}_m$, we may assume that $\ov{C}_1$ is invertible.
	
	By Nakayama-style arguments like \cite[Lemma~1.5.1]{alg2} we may replace $A$ by some localization such that the $\ov{e}_i$ and $\ov{f}_j$ possess lifts $e_i\in P^0$ and $f_j\in P^1$ such that $e_0,\ldots,e_n$ form a basis of $P^0$ and $f_0,\ldots,f_m$ a basis of $P^1$. Then the matrix representation $D$ of $\delta^0$ looks like
	\begin{align*}
	\begin{tikzpicture}[remember picture]
	\node at (0,0) {$D=
		\begin{pmatrix}
		 & \tikzentry{e}{\vphantom{0}\hphantom{c_1}}\vphantom{0} & \tikzentry{a}{\phantom{0}} \\
		 &  \vphantom{\vdots} & & \phantom{\ov{C}_1}\\
		 & \tikzentry{f}{\vphantom{0}\hphantom{c_1}}\vphantom{\vdots} & & & \tikzentry{b}{\phantom{0}} &\\
		 & \tikzentry{g}{\vphantom{0}\hphantom{c_2}}\vphantom{\vdots} & \tikzentry{c}{\phantom{0}} \\
		 & \tikzentry{h}{\vphantom{0}\hphantom{c_2}}\vphantom{0} &  & & \tikzentry{d}{\phantom{0}}
		\end{pmatrix}
		$};
	\draw[rounded corners=0.75] ($(a.north west) +(-2pt,3pt)$) rectangle ($(b.south east) +(2pt,-3pt)$) node[pos=0.5] {$C_1$};
	\draw[rounded corners=0.75] ($(c.north west) +(-2pt,3pt)$) rectangle ($(d.south east) +(2pt,-3pt)$) node[pos=0.5] {$C_2$};
	\draw[rounded corners=0.75] ($(e.north west) +(-2pt,3pt)$) rectangle ($(f.south east) +(2pt,-3pt)$) node[pos=0.5] {$c_1$};
	\draw[rounded corners=0.75] ($(g.north west) +(-2pt,3pt)$) rectangle ($(h.south east) +(2pt,-3pt)$) node[pos=0.5] {$c_2$};
	\end{tikzpicture}
	\end{align*}
	in which $c_1$ and $c_2$ are column vectors whose images in $\KK(s)$ vanish, and the images of $C_1$ and $C_2$ in $\KK(s)$ are $\ov{C}_1$ and $\ov{C_2}$ respectively. Since $\det\ov{C}_1$ doesn't vanish in $\KK(s)$, we may replace $A$ by some localization such that $C_1\in A^{n\times n}$ becomes invertible (cf.\ the argument in the proof of Lemma~\reff{lem:BasicallyThm4bc}). But then $c_1$ as well as the rows of $C_2$ may be represented as linear combinations of the columns and rows of $C_1$ respectively. So by performing column and row transformations (that is, replacing $e_0$ and $f_n,\ldots,f_m$ appropriately) we may assume that $c_1=0$ and $C_2=0$, leaving only 
		\begin{align*}
	\begin{tikzpicture}[remember picture]
	\node at (0,0) {$D=
		\begin{pmatrix}
		& 0 & \tikzentry{a}{\phantom{0}} \\
		&  \vdots & & \phantom{\ov{C}_1}\\
		& 0\vphantom{\vdots} & & & \tikzentry{b}{\phantom{0}} &\\
		& \tikzentry{g}{\vphantom{0}\hphantom{c_2}}\vphantom{\vdots} & \tikzentry{c}{\phantom{0}} \\
		& \tikzentry{h}{\vphantom{0}\hphantom{c_2}}\vphantom{0} &  & & \tikzentry{d}{\phantom{0}}
		\end{pmatrix}
		$};
	\draw[rounded corners=0.75] ($(a.north west) +(-2pt,3pt)$) rectangle ($(b.south east) +(2pt,-3pt)$) node[pos=0.5] {$C_1$};
	\path (c) -- (d) node[pos=0.5] {$0$};
	\draw[rounded corners=0.75] ($(g.north west) +(-2pt,3pt)$) rectangle ($(h.south east) +(2pt,-3pt)$) node[pos=0.5] {$c_2$};
	\end{tikzpicture}
	\end{align*}
	(that's the last blocky matrix, I promise). Now note that $Z_+=V(c_2)$. Indeed, since $C_1$ will be invertible in $\KK(\pp)$ for all $\pp\in\Spec A$, the kernel of $\delta_\pp^0=\delta^0\otimes\id_{\KK(\pp)}$ is one-dimensional iff $c_2$ vanishes in $\KK(\pp)$. Therefore, it seems sensible to equip $Z_+$ with the prescheme structure of $\Spec A/(c_2)$, and that's what we'll do.
	
	\emph{Step 4.} We show that locally $Z=Z_+\cap Z_-$ has the required universal property. We already know from Step~2 that an $S$-prescheme $\tau\colon T\morphism S$ satisfies $h\left(t,\tau_X^*\Ll^{\pm 1}\right)=1$ for all $t\in T$ iff the image of $\tau$ is contained in $Z$. Now we prove that $U_0=T$ for such $\tau_X^*\Ll$ if and only if $\tau$ factors over $Z_+$ (and likewise with $\tau_X^*\Ll^{-1}$ and $Z_-$), which suffices to finish the proof.
	
	Again, we may do this locally, so let's assume $S=\Spec A$ and $Z_+=V(c_2)$. By definition, we have $U_0=T$ iff the base change morphism
	\begin{align*}
		\big(f_{T,*}(\tau_X^*\Ll)\big)(t)\morphism H^0\left(X_{T,t},\tau_X^*\Ll|_{X_{T,t}}\right)
	\end{align*}
	is surjective for all $t\in T$. We know 
	\begin{align*}
		H^0\big(X_{T,t},\tau_X^*\Ll|_{X_{T,t}}\big)=\ker\left(\delta_t^0\delta^0\otimes\id_{\KK(t)}\colon P^0\otimes_A\KK(t)\morphism P^1\otimes_A\KK(t)\right)
	\end{align*}
	by Theorem~\reff{thm:GrauertGrothendieck}\itememph{a}, so $\ker \delta^0_t$ is one-dimensional (since the image of $\tau$ lies in $Z_+$ by assumption) and thus generated by $e_0\otimes1$. Now let $V\cong \Spec B$ be a small affine open neighbourhood of $t\in T$. By Theorem~\reff{thm:GrauertGrothendieck}\itememph{a} again, $f_{T,*}(\tau_X^*\Ll)(V)$ is given by
	\begin{align*}
		K=\ker\left(\delta^0\otimes\id_B\colon P^0\otimes_AB\morphism P^1\otimes_AB\right)\;.
	\end{align*}
	So the question is whether $K\otimes_B\KK(t)\morphism \ker\delta_t^0$ is surjective. Since the image of $C_1$ in $B^{n\times n}$ stays invertible, an element $\sum_{i=0}^n\beta_i(e_i\otimes1)$ of $P^0\otimes_AB$ (where $\beta_i\in B$ are some coefficients) is in $K$ iff $\beta_1=\ldots=\beta_n=0$ and $\beta_0c_2=0$. Thus $e_0\otimes1\in\ker\delta_t^0$ is hit by $K\otimes_B\KK(t)\morphism \ker\delta_t^0$ if and only if $c_2=0$ in $\KK(t)$, that is, iff $t\in V(\tau^*c_2)$.
	
	Thus $U_0=T$ for $\tau_X ^*\Ll$ iff $\tau$ factors over $Z_+$ as a prescheme. The same is true for $\tau_X^*\Ll^{-1}$ and $Z_-$, so $Z=Z_+\cap Z_-$ has the required universal property.
\end{proof}

\Appendix
\chapter{Appendix}
\section{Some prerequisites about completions}
We briefly recall the most important facts about completions. An excellent introduction to this subject can be found in \cite[Section~10]{atiyahMacdonald}.
\subsection{Definitions and Hensel's lemma}
\begin{defi}
	Let $A$ be a ring (commutative with $1$), $I\subseteq A$ and ideal, and $M$ an $A$-module.
	\begin{alphanumerate}
		\item The \defemph{$I$-adic topology} on $M$ is the unique topology such that $\{I^n\}_{n\in \IN}$ is a fundamental system of neighbourhoods of $0$ and $M$ (with its additive structure) becomes a topological group in this topology.
		\item The \defemph{completion} of $M$ with respect to the $I$-adic topology is
		\begin{align*}
			\roof{M}=\limit[n\in \IN]M/I^nM\;.
		\end{align*}
		Note that $\roof{A}$ is a ring again. We call $M$ \defemph{complete} in the $I$-adic topology if the canonical morphism $M\morphism\roof{M}$ is an isomorphism.
	\end{alphanumerate}
\end{defi}
\begin{rem}
	$M$ with its $I$-adic topology is \emph{pseudo-metrizable} via $d(x,y)=\mathrm{e}^{-\sup\left\{n\st x-y\in I^n\right\}}$. It is easy to check that $\roof{M}$ is also the completion of $M$ in the analytical sense, i.e.\ the set of Cauchy sequences modulo the zero sequences.
\end{rem}
\begin{example}
	If $I^n=0$ for some $n\in \IN$, then any $A$-module is complete in the $I$-adic topology.
\end{example}
\begin{example}
	If $A=\IZ$ and $I=p\IZ$ for some prime $p$, then $\roof{A}=\IZ_p$ is the ring of $p$-adic integers.
\end{example}
\begin{prop}[Hensel's lemma]\lbl{prop:HenselsLemma}
	Suppose the ring $A$ is complete in the $I$-adic topology. Let $P\in A[T]$ be a polynomial and $a_0\in A$ such that $P(a_0)\equiv 0\bmod I$ and $P'(a_0)$ is a unit in $A/I$. Then there is a unique $a\in A$ such that $a\equiv a_0\bmod I$ and $P(a)=0$.
\end{prop}
\begin{proof}
	\emph{Step 1.} Consider the special case $I^2=0$. For $\delta\in I$ we have $P(a_0+\delta)=P(a_0)+\delta P'(a_0)$ since all terms of order $\delta^2$ or higher vanish in the binomial expansion. Now $P'(a_0)$ being a unit in $A/I$ gives a unique $\delta\in I$ such that $a=a_0+\delta$ satisfies $P(a)=0$.
	
	\emph{Step 2.} Suppose that $I^{2^n}=0$ for some $n\in \IN$. Using induction on $n$ (with the base case being precisely Step~1) we may assume that Hensel's lemma holds for $A/I^{2^{n-1}}$. In particular, there is a unique $a_{n-1}$ such that $P(a_{n-1})\equiv 0\bmod I^{2^{n-1}}$ and $a_{n-1}\equiv a_0\bmod I$. Moreover, $P'(a_{n-1})$ is invertible in $A/I^{2^{n-1}}$. Indeed, this follows from Hensel's lemma applied to $A/I^{2^{n-1}}$ (for which it holds by induction hypothesis) and the polynomial $Q=P'(a_{n-1})T-1$. The derivative $Q'(a_{n-1})$ equals $P'(a_{n-1})$ which is invertible in $A/I$ since $P'(a_{n-1})\equiv P'(a_0)\bmod I$, so Hensel's lemma is indeed applicable. Now replacing $I$ by $I^{2^{n-1}}$ and $a_0$ by $a_{n-1}$ reduces the situation to Step~1, proving the inductive step.
	
	\emph{Step 3.} Now let $I$ be arbitrary. By Step~2 there is for every $n\in\IN$ a unique $a_n\in A/I^{2^n}$ such that $P(a_n)\equiv 0\bmod I^{2^n}$ and $a_n\equiv a_0\bmod I$. Then $a_n\equiv a_{n-1}\bmod I^{2^{n-1}}$ is forced by uniqueness. Hence $a=(a_n)_{n\in \IN}$ defines an element of
	\begin{align*}
		\limit[n\in \IN]A/I^{2^n}=\limit[n\geq 1] A/I^n=\roof{A}\;,
	\end{align*}
	providing the desired element $a\in\roof{A}$.
\end{proof}
\begin{cor}\lbl{cor:HenselApplications}
	Let $A$ be complete in the $I$-adic topology.
	\begin{alphanumerate}
		\item If $a\in A$ becomes a unit in $A/I$, then already $a\in A^\times$.
		\item For every idempotent $\pi\in A/I$ there is a unique idempotent in $A$ whose image modulo $I$ is $\pi $. Therefore, $\Spec A$ and $\Spec A/I$ have the same connected components.
		\item $I$ is contained in the Jacobson radical $\rad A$.
	\end{alphanumerate}
\end{cor}
\begin{proof}
	Part \itememph{a} follows from Proposition~\ref{prop:HenselsLemma} applied to $P=aT-1$ (whose derivative $a$ is a unit in $A/I$ by assumption, so this is fine). For \itememph{b} we use the polynomial $P=T^2-T$. Again, $P'(\pi )=2\pi -1$ is a unit in $A/I$ since $(2\pi -1)^2=4\pi ^2-4\pi +1=1$ in $A/I$. To prove \itememph{c} recall the characterization
	\begin{align*}
		\rad A=\left\{x\in A\st 1-ax\in A^\times\text{ for all }a\in A\right\}\;.
	\end{align*}
	If $x\in I$, then $1-ax$ is a unit in $A/I$, hence also in $A$ by \itememph{a}.
\end{proof}
\begin{prop}
	Let $A$ be noetherian and $N\subseteq M$ finitely generated $A$-modules. Then the $I$-adic topology on $N$ coincides with the induced topology by the $I$-adic topology on $M$.
\end{prop}
\begin{proof}[Sketch of a proof]
	By the Artin--Rees lemma (cf.\ \cite[Proposition~3.4.1]{alg2}) there exists a number $c\in\IN$ such that $N\cap I^{n+c}M\subseteq I^nN$. From this, the assertion is easily deduced.
\end{proof}
\subsection{Flatness properties}
\begin{fact}
	\begin{alphanumerate}
		\item \lbl{fact:completion101}The canonical morphism $\roof{M}=\limit M/I^nM\morphism M/IM$ is surjective.
		\item If $M$ is finitely generated and $I$ is contained in the Jacobson radical of $A$, then $\roof{M}=0$ implies $M=0$.
	\end{alphanumerate}
\end{fact}
\begin{proof}
	For \itememph{a}, note that the composition $M\morphism\roof{M}\morphism M/IM$ equals the projection $M\morphism M/IM$ by definition of the limit. Since the latter is surjective, so is $\roof{M}\morphism M/IM$. 
	
	In particular, part \itememph{a} shows that $\roof{M}=0$ implies $M=IM$. In the situation of \itememph{b} this is equivalent to $M=0$ by Nakayama's lemma (which -- as we all know -- Professor Franke also likes to attribute to Azumaya and Krull, even though he regards Krull as a noob compared to Grothendieck).
\end{proof}
\begin{cor}\lbl{cor:completionExact}
	If $A$ is noetherian, then the functor $M\mapsto \roof{M}$ is exact on the category of finitely generated $A$-modules.
\end{cor}
\begin{proof}
	Let $0\morphism M'\morphism M\morphism M''\morphism 0$ be a short exact sequence of finitely generated $A$-modules. Then $M'+I^nM$ is the kernel of $M\epimorphism M''/I^nM''$. Using $(M'+I^nM)/I^nM\cong M'/(M'\cap I^nM)$ we get short exact sequences
	\begin{align}\lbl{eq:completion}
		0\morphism M'/(M'\cap I^nM)\morphism M/I^nM\morphism M''/I^nM''\morphism 0\tag{$*$}
	\end{align}
	for every $n\in\IN$. Since $M'/(M'\cap I^nM)$ is sandwiched between $M'/I^nM'$ and $M'/I^{n+c}M'$ for some $c\in\IN$ by the Artin--Rees lemma, it's easy to see that 
	\begin{align*}
		\limit[n\in \IN] M'/(M'\cap I^nM)=\limit[n\geq 1]M'/I^{n+c}M'=\limit[n\geq 1]M'/I^nM'=\roof{M}'\;. 
	\end{align*}
	Moreover, each $M'/(M'+I^{n+1}M)\morphism M'/(M'+I^nM)$ is clearly surjective, so Fact~\reff{fact:MittagLeffler} gives 
	\begin{align*}
		\limit[n\geq 1][^1]M'/(M'\cap I^nM)=0\;.
	\end{align*}
	Thus, taking the limit over \eqreff{eq:completion} gives a short exact sequence $0\morphism \roof{M}'\morphism \roof{M}\morphism\roof{M}''\morphism 0$ by Fact~\reff{fact:6termLimitSequence}. We are done.
\end{proof}
\begin{cor}\lbl{cor:completionFaithfullyFlat}
	Let $A$ be a Noetherian ring and $\roof{\phantom{x}}$ the $I$-completion for some ideal $I\subseteq A$.
	\begin{alphanumerate}
		\item When $M$ is a finitely generated $A$-module, then $\roof{M}\cong M\otimes_A\roof{A}$.
		\item $\roof{A}$ is flat as an $A$-module.
		\item If $I$ is contained in the Jacobson radical of $A$, then $\roof{A}$ is \defemph{faithfully flat} over $A$. That is, if one of the sequences of $A$-modules
		\begin{gather*}
			0\morphism M'\morphism[\alpha] M\morphism[\beta] M''\morphism 0\\
			0\morphism M'\otimes_A\roof{A}\xrightarrow{\alpha\otimes\id} M\otimes_A\roof{A}\xrightarrow{\beta\otimes\id} M''\otimes_A\roof{A}\morphism 0
		\end{gather*}
		is exact, then so is the other. In particular, this holds when $A$ is local and $I=\mm$ the maximal ideal of $A$.%Suppose that $I$ is contained in the Jacobson radical of $A$. If $\mu\colon M\morphism N$ is a morphism of finitely generated $A$-modules such that $\roof{\mu}\colon \roof{M}\morphism\roof{N}$ is an isomorphism, then $\mu$ is an isomorphism.
	\end{alphanumerate}
\end{cor}
\begin{proof}
	Part \itememph{a}. Every finitely generated $A$-module is finitely presented as well since $A$ is noetherian. So take a representation $M\cong \coker(A^m\morphism A^n)$ for some $m,n\in\IN$. It's obvious that $(A^n)^\complete\cong \roof{A}^n\cong A^n\otimes_A\roof{A}$. Since both completion and tensor products commute with cokernels, this shows $\roof{M}\cong M\otimes_A\roof{A}$ as well.
	
	This immediately shows \itememph{b}. Indeed, by Corollary~\reff{cor:completionExact} and \itememph{a}, $-\otimes_A\roof{A}$ is exact on finitely generated $A$-modules -- which is sufficient for flatness by \cite[Proposition~1.2.2]{homalg}.
	
	For \itememph{c}, we first prove an auxiliary statement.
	\begin{claim}\lbl{claim:completionFaithfullyFlat}
		If $M$ is any $A$-module, then $M\otimes_A\roof{A}=0$ implies $M=0$.
	\end{claim}
	Indeed, to see this, let $N\monomorphism M$ be any finitely generated submodule. Because $\roof{A}$ is flat over $A$ by \itememph{b}, $N\otimes_A\roof{A}\morphism M\otimes_A\roof{A}$ is still injective, so $0=N\otimes_A\roof{A}=\roof{N}$ by assumption and part~\itememph{a}. Hence $N=0$ by Fact~\reff{fact:completion101}. But $M$ is the union of its finitely generated submodules, so $M=0$ as well.
	
	Suppose now that $0\morphism M'\otimes_A\roof{A}\morphism M\otimes_A\roof{A}\morphism M''\otimes_A\roof{A}\morphism 0$ is exact (the other direction is clear because we proved in \itememph{b} that $\roof{A}$ is flat over $A$). First note that $\beta\alpha=0$. Indeed, if $B$ is the image of $M'$ in $M''$, then $B\otimes_A\roof{A}$ is the image of $M'\otimes_A\roof{A}$ in $M''\otimes_A\roof{A}$ (using that $\roof{A}$ is flat). Hence $B\otimes_A\roof{A}=0$, so $B=0$ by Claim~\reff{claim:completionFaithfullyFlat}.
	
	This induces a morphism $M'\morphism\ker \beta$. Let $K$ and $Q$ be its kernel and cokernel. Then $\ker\left(\beta\otimes\id_{\roof{A}}\right)\cong \ker \beta \otimes_A\roof{A}$ because $\roof{A}$ is flat, hence the sequence
	\begin{align*}
		0\morphism K\otimes_A\roof{A}\morphism M'\otimes_A\roof{A}\morphism \ker\left(\beta\otimes\id_{\roof{A}}\right)\morphism Q\otimes_A\roof{A}\morphism 0\;.
	\end{align*}
	is exact (using flatness of $\roof{A}$ once again). But $M'\otimes_A\roof{A}\morphism \ker\left(\beta\otimes\id_{\roof{A}}\right)$ must be an isomorphism, so $K\otimes_A\roof{A}=0$ and $Q\otimes_A\roof{A}$. By Claim~\reff{claim:completionFaithfullyFlat} this shows $K=0$ and $Q=0$ and we conclude $M'\cong \ker\beta$. In the same way one can show $M''\cong \coker\alpha$. We are done.
\end{proof}
\begin{cor}
	Suppose that $I$ is contained in the Jacobson radical of the noetherian ring $A$. If $\mu\colon M\morphism N$ is a morphism of finitely generated $A$-modules whose $I$-adic completion $\roof{\mu}\colon \roof{M}\morphism\roof{N}$ is an isomorphism, then $\mu$ is an isomorphism.
\end{cor}
\begin{proof}
	Follows from Corollary~\reff{cor:completionFaithfullyFlat}\itememph{a} and \itememph{c}.
\end{proof}
\begin{cor}\lbl{cor:completionFaithfullyFlat2}
	Let $A$ be noetherian. A sequence $0\morphism M'\morphism M\morphism M''\morphism 0$ of finitely generated $A$-modules is exact iff for all maximal ideals $\mm\subseteq A$ the sequence $0\morphism\roof{M}'\morphism \roof{M}\morphism \roof{M}''\morphism 0$ of $\mm$-adic completions is exact.
\end{cor}
\begin{proof}
	This sequence is exact iff its localizations at the maximal ideals $\mm\subseteq A$ are all exact. Having localized at $\mm$, we may check exactness after going over to $\mm A_\mm$-adic completions by Corollary~\reff{cor:completionFaithfullyFlat}\itememph{c}. So all that's left to do is to show that $\roof{M}\cong \roof{M}_\mm$ holds for the $\mm$-adic resp.\ $\mm A_\mm$-adic completions. This follows from $M/\mm^nM\cong M_\mm/\mm^nM_\mm$ for all $n\geq 1$. Indeed, $A/\mm^n$ is a local ring with maximal ideal $\mm/\mm^n$, hence multiplication by $a\in A\setminus \mm$ is already bijective on $M/\mm^nM$.
\end{proof}
\begin{cor}\lbl{cor:JM}
	If $J\subseteq A$ is any ideal and $M$ a finitely generated $A$-module, then $(JM)^\complete\morphism\roof{M}$ defines an isomorphism $(JM)^\complete\isomorphism J\roof{M}$.
\end{cor}
\begin{proof}
	We may view $(JM)^\complete$ as a submodule of $\roof{M}$ since completion preserves injectivity of the inclusion $JM\subseteq M$ by Corollary~\reff{cor:completionExact}. It's easy to see that $J\roof{M}$ is contained in $(JM)^\complete$. To prove the converse, take generators $j_1,\ldots,j_n$ of $J$. Then completion preserves surjectivity of $(j_1,\ldots,j_n)\colon M^n\epimorphism JM$ and we are done.
\end{proof}
\begin{cor}\lbl{cor:completionLocal}
	If $A$ is a noetherian local ring with maximal ideal $\mm$, then $\roof{A}$ is local with maximal ideal $\mm \roof{A}$.
\end{cor}
\begin{proof}
	We proved this in \cite[Corollary~2.2.2]{homalg}.
\end{proof}
\subsection{Completions and noetherianness}
\begin{prop}\lbl{prop:completionNoetherian}
	Let $A$ be noetherian and $I\subseteq A$ any ideal, then the $I$-adic completion $\roof{A}$ is noetherian again.
\end{prop}
To prove this, we need to prove the evil twin of Hilbert's Basissatz first.
\begin{lem}\lbl{lem:(not)HilbertBasis}
	If $A$ is noetherian, then so is the power series ring $R=A\llbracket T\rrbracket$.
\end{lem}
\begin{proof}
	We can (and will) basically copy the proof of Hilbert's Basissatz. Let $J\subseteq R$ be any ideal and put $J_n=\left\{a_n\st \sum_{k=n}^\infty a_kT^k\in J\right\}$ for $n\geq 0$. Then $(J_n)_{n\in \IN}$ form an ascending sequence of ideals in $A$. Noetherianness of $A$ tells us that this sequence becomes eventually stationary, say, at $n=s$. So we may choose $a^{(i)}=\sum_{k\geq s}a_kT^k\in R$ for $i=1,\ldots,N$ such that $a_s^{(1)},\ldots,a_s^{(N)}$ generate $J_s$. Then $a^{(1)},\ldots,a^{(N)}$ generate $J\cap T^sR$. Indeed, given any $b=\sum_{k\geq s}b_kT^k\in J$ we can inductively choose coefficients $r_k^{(1)},\ldots,r_k^{(N)}\in A$ such that $r^{(i)}=\sum_{k\geq 0}r_k^{(i)}T^k$ satisfy $r^{(1)}a^{(1)}+\ldots+r^{(N)}a^{(N)}=b$ up to degree $T^{s+k}$. This works because $J_{k+s}=J_s$ for all $k\geq 0$ is generated by $a_s^{(1)},\ldots,a_s^{(N)}$ again.
	
	Now $R/T^sR$ is a finitely generated $A$-module, hence the image of $J$ in it is finitely generated as well, $A$ being noetherian. We thus may choose $a^{(N+1)},\ldots,a^{(N+M)}\in J$ whose images modulo $T^sR$ generate the image of $J$ in $R/T^sR$. Then $a^{(1)},\ldots,a^{(N+M)}$ generate $J$ and our job's done here.
\end{proof}
\begin{proof}[Proof of Proposition~\reff{prop:completionNoetherian}]
	Let $r_1,\ldots,r_n$ be generators of $I$. Then sending $X_i\mapsto r_i$ defines a surjective morphism $A\llbracket X_1,\ldots,X_n\rrbracket\epimorphism \roof{A}$. Since $A\llbracket X_1,\ldots,X_n\rrbracket$ is noetherian by Lemma~\reff{lem:(not)HilbertBasis} and induction on $n$, so is its quotient $\roof{A}$.
\end{proof}
\begin{cor}
	Suppose that $A$ is a noetherian local ring and $I\subseteq A$ any (proper) ideal. Then $\dim A=\dim\roof{A}$. In particular, $A$ is regular iff $\roof{A}$ is regular.
\end{cor}
\begin{proof}
	Let $\mm$ be the maximal ideal of $A$. Then $\roof{\mm}=\mm\roof{A}$ (this equality holds because of Corollary~\ref{cor:JM}) is the maximal ideal of the local ring $\roof{A}$ as was shown in the proof of \cite[Corollary~2.2.2]{homalg}. Since $I\subseteq \mm$, the quotients $\mm^i/\mm^{i+1}$ already have $I$-torsion, hence 
	\begin{align*}
		\mm^i/\mm^{i+1}\cong \left(\mm^i/\mm^{i+1}\right)^\complete\cong \roof{\mm}^i/\roof{\mm}^{i+1}
	\end{align*}
	(the last isomorphism follows from exactness of completion). This shows that the associated graded rings $\gr(A,\mm)$ and $\gr(\roof{A},\roof{\mm})$ agree, hence $(A,\mm)$ and $(\roof{A},\roof{\mm})$ have the same Hilbert--Samuel polynomials, which shows $\dim A=\dim\roof{A}$ by \cite[Theorem~20]{alg2}.
	
	Now $A$ and $\roof{A}$ have the same residue field $k$ and $\mm/\mm^2\cong \roof{\mm}/\roof{\mm}^2$ by the $I$-torsion arguments we have seen several times now, so $\dim_k\mm/\mm^2=\dim_k\roof{\mm}/\roof{\mm}^2$. Clearly this implies that $A$ is regular iff $\roof{A}$ is.
\end{proof}
\begin{rem}
	In a similar fashion one can show that a noetherian local ring is Cohen--Macaulay, or Gorenstein, or a complete intersection, iff its $I$-adic completion is one as well. For example, for Cohen--Macaulayness one would need to show $\depth_A(A)=\depth_{\roof{A}}(\roof{A})$, which follows from the isomorphism $\Ext_A^p(k,A)\cong\Ext_{\roof{A}}^p(k,\roof{A})$ that was described in the proof of \cite[Proposition~2.4.2]{homalg}.
\end{rem}
\section{Generic freeness and Chevalley's theorem}
A good reference for this section is \cite[Section~14.2 and 14.3]{eisenbudCommAlg}. 

Professor Franke points out that there are multiple approaches for Generic freeness: a clever \emph{dévissage}-style proof due to Grothendieck (cf.\ \cite[Theorem~14.4]{eisenbudCommAlg}), which Franke says he could not have come up with, or a more natural proof similar to that of Hilbert's Basissatz (cf.\ \cite[\S 24]{matsumuraCRT}). However, both references assume $R$ to be noetherian, which we will circumvent by Gröbner basis theory.\footnote{True Franke fans will remember \ldots}
\begin{prop}[Grothendieck's generic freeness theorem]\lbl{prop:GenericFreeness}
	Let $R$ be a domain, $A$ an $R$-algebra of finite type, and $M$ a finitely generated $A$-module. Then there is $f\in R\setminus\{0\}$ such that $M_f$ is a free $R_f$-module
\end{prop}
\begin{proof}
	Every finitely generated $A$-module $M$ has a filtration $0=M_0\subseteq M_1\subseteq \ldots\subseteq M_n=M$ such that $M_i/M_{i-1}$ has the form $A/I_i$ for some ideal $I_i\subseteq A$. If all $M_i/M_{i-1}$ are free $R$-modules, then each sequence $0\morphism M_{i-1}\morphism M_i\morphism M_i/M_{i-1}\morphism 0$ is split, so $M$ is a free $R$-module as well. This argument shows that we only need to deal with the case $M\cong A/I$ for some ideal $I\subseteq A$. Replacing $A$ by $A/I$ this can be further reduced to $M=A$.
	
	We can represent $A$ as $A\cong R[X_1,\ldots,X_n]/I$ for some ideal $I\subseteq R[X_1,\ldots,X_n]$. Let $K$ be the field of quotients of $R$ and $J\subseteq K[X_1,\ldots,X_n]$ the ideal generated by the image of $I$. Let $(\beta_1,\ldots,\beta_r)$ be a Gröbner basis of $J$ (for some fixed monomial order), then $\beta_i=\frac{b_i}{d_i}$ for some $b_i\in I$ and $d_i\neq 0$ and the $\beta_i$ have leading term $1$. Replacing $R$ by the localization $R_{d_1\cdots d_r}$ we may assume $\beta_i\in I$. Using generalized division with remainder (i.e.\ Gröbner basis stuff) one easily derives that the $\beta_i$ generate $I$. Then the monomials $X_1^{\alpha_1}\cdots X_n^{\alpha_n}$ where $\alpha=(\alpha_1,\ldots,\alpha_n)$ has the property that there is no $i$ such that $\alpha$ dominates the leading term of $\beta_i$, form a basis of $A$ as an $R$-module.
\end{proof}
\begin{prop}[Chevalley's theorem]\lbl{prop:Chevalley}
	If $f\colon \Spec B\morphism\Spec A$ is a morphism of finite type between affine noetherian schemes, then the image of $f$ is contained in a proper closed subset of $\Spec A$ or contains a dense open subset.
\end{prop}
\begin{proof}
	What Chevalley's theorem actually says is that the image of $f$ is \defemph{constructible}, that is, a finite union of open subsets of closed subsets of $\Spec A$ (cf.\ \cite[Corollary~14.6]{eisenbudCommAlg}). This is what we're going to prove now (and we leave it to the reader to show that this implies the assertion -- which I found quite a pain, actually).
	
	It suffices to prove the assertion for every irreducible component of $\Spec A$ (as these guys are closed and there are finitely many of them). Hence, if $\pp\in\Spec A$ is a minimal prime corresponding to the irreducible component $V(\pp)$ of $\Spec A$, then we may replace $A$ and $B$ by $A/\pp$ and $B/\pp B$. Therefore, we can assume that $A$ is a domain. Moreover, suppose that $A\subseteq B$. Otherwise, $A$ could be replaced by $A/\ker(A\morphism B)$, and since that cuts out a closed subset of $\Spec A$, we're fine with that. 
	
	By Noetherian induction, we may assume that the assertion is true for all closed subsets of $\Spec A$. Let $\alpha\in A$ such that $B_\alpha$ is free as an $A_\alpha$-module (we can do this by Proposition~\reff{prop:GenericFreeness}). We claim that any $\pp\in\Spec A\setminus V(\alpha)$ is in the image of $f$. Indeed, such $\pp$ may be regarded as prime ideals $\pp\in\Spec A_\alpha$. Then $B_\pp\cong B\otimes_AA_\pp$ is still free as an $A_\pp$-module, say, $B_\pp\cong \bigoplus_{i\in I}A_\pp$ for some indexing set $I$ (which is non-empty as $A_\pp\subseteq B_\pp$). Hence $B_\pp/\pp B_\pp\cong B_\pp\otimes_{A_\pp}A_\pp/\pp A_\pp\cong \bigoplus_{i\in I}\KK(\pp)$ is not the zero ring because $\KK(\pp)\neq 0$. If $\qq\subseteq B$ is the preimage of any prime ideal of $B_\pp/\pp B_\pp$, then it's easily seen that $f(\qq)=\pp$, proving that $\pp$ is in the image of $f$, as claimed.
	
	By the Noetherian induction hypothesis, the image of $f$ in $V(\alpha)$ -- which equals the image of $\ov{f}\colon\Spec(B/\alpha B)\morphism\Spec (A/\alpha A)$ -- is constructible. We are done.		
\end{proof}
\section{Some additions to the lectures}
Occasionally it happens that Professor Franke without explanation uses some technical facts I wish afterwards he had proved, because they seems not at all obvious (at least to me). I usually notice only after the lecture as these facts are subtle and get lost easily in the flow of the presentation. In this section I collect those facts which I couldn't possibly fit into the lecture notes without bursting their overarching structure.

%\textbf{Warning 1!} I see no point in reading this section if you don't look for a specific result. There's no real coherence between the listed lemmas and propositions.
%
%\textbf{Warning 2!} Also there are some technical nightmares ahead.
\subsection{More technical facts about completions}
\begin{prop}\lbl{prop:modulesComplete}
	Let $A$ be a noetherian ring which is complete in the $I$-adic topology and let $M$ be a finitely generated $A$-module. Then $M$ is $I$-adically complete.
\end{prop}
\begin{proof}
	Note that this is clearly fulfilled if $M=A^n$ is a finitely generated free $A$-module. Now let $M$ be arbitrary. Since $A$ is noetherian, $M$ can be represented as $\coker\left(A^m\morphism A^n\right)$. Because $A^m$, $A^n$ equal their own completions (as we have just seen) and  completion is exact (by Corollary~\reff{cor:completionExact}), $M=\roof{M}$ holds as well.
\end{proof}
\begin{cor}\lbl{cor:finiteAlgebras}
	Let $A$ be a noetherian local ring maximal ideal $\mm$. Let $B$ be a finite $A$-algebra.
	\begin{alphanumerate}
		\item Then $B$ has only finitely many prime ideals over $\mm$, and all of them are maximal.
		\item If $A$ is, in addition, $\mm$-adically complete, then all maximal ideals of $B$ lie over $\mm$. In particular, $B$ is \defemph{semi-local} (i.e.\ has finitely many maximal ideals). The same is true if $A\subseteq B$.
	\end{alphanumerate}
\end{cor}
\begin{proof}
	Part \itememph{a}. If $\qq\in\Spec B$ is a prime ideal over $\mm$, then $B/\qq$ is a finitely generated domain over the residue field $k=A/\mm$, hence a finite field extension of $k$, so $\qq$ is maximal. Moreover, $B/\mm B$ is a finite-dimensional $k$-algebra, hence it has only finitely many maximal ideals by the argument from Fact~\reff{fact:annoyingQF}\itememph{b}.
	
	Part \itememph{b}. If $A$ is $\mm$-adically complete, then $B$ is $\mm B$-adically complete by Proposition~\reff{prop:modulesComplete}, so $\mm B$ is contained in the Jacobson radical $\rad B$ by Corollary~\reff{cor:HenselApplications}\itememph{c}. Then all maximal ideals of $B$ lie over $\mm$.
	
	Now assume $A\subseteq B$. Let $\qq\in\Spec B$ be a maximal ideal and $\pp=\qq\cap A$. Then $A/\pp\subseteq B/\qq$ is an integral ring extension in which $B/\qq$ is a field, hence so is $A/\pp$, (by \cite[Proposition~1.5.1\itememph{d}]{alg1}) which proves $\pp=\mm$.
\end{proof}
\begin{prop}\lbl{prop:technicalAF}
	Let $A$ be a noetherian ring and $B$ an $A$-algebra. Let $\qq\in\Spec B$ be a prime ideal and $\pp\in\Spec A$ its preimage in $A$. Let $\roof{A}_\pp$ denote the $\pp A_\pp$-adic completion of $A_\pp$.
	\begin{alphanumerate}
		\item The ideal $\qq B_\qq\otimes_{A_\pp}\roof{A}_\pp$ is maximal in $B_\qq\otimes_{A_\pp}\roof{A}_\pp$.
		\item Moreover, assume that $B$ is of finite type over $A$ (so that $B$ is noetherian as well) and that $B\otimes_A\KK(\pp)$ is a finite-dimensional $\KK(\pp)$-vector space. Then $B_\qq\otimes_{A_\pp}\roof{A}_\pp$ equals the $\qq B_\qq$-adic completion $\roof{B}_\qq$ of $B_\qq$. In particular, $B_\qq\otimes_{A_\pp}\roof{A}_\pp$ is a local ring again.
	\end{alphanumerate}
\end{prop}
\begin{rem}
	The conditions from Proposition~\reff{prop:technicalAF}\itememph{b} say precisely that the induced morphism $\Spec B\morphism\Spec A$ of schemes is \emph{quasi-finite} at all primes above $\pp$, cf.\ Definition~\reff{def:quasiFinite} and Fact~\reff{fact:annoyingQF}\itememph{b}.
\end{rem}
\begin{proof}[Proof of Proposition~\reff{prop:technicalAF}]
	Part \itememph{a}. This follows basically from the fact that $\roof{A}_\pp$ is local again with maximal ideal $\roof{\pp}=\pp\roof{A}_\pp$ and has the same residue field $\KK(\pp)$ as $A_\pp$.The details go as follows: We have
	\begin{align*}
	\left(B_\qq\otimes_{A_\pp}\roof{A}_\pp\right)/\left(\qq B_\qq\otimes_{A_\pp}\roof{A}_\pp\right)\cong \KK(\qq)\otimes_{A_\pp}\roof{A}_\pp\;.
	\end{align*}
	Denote $\phi\colon A\morphism B$ the ring morphism that makes $B$ an $A$-algebra. If $x\in \KK(\qq)\otimes_{A_\pp}\roof{\pp}$, then $x$ can be written as $x=\sum_{i=1}^{n}b_i\otimes p_ia_i$ where $b_i\in B_\qq$, $a_i\in A_\pp$, and $p_i\in \pp$. But then $x=\sum_{i=1}^{n}\phi(p_i)b_i\otimes a_i=0$, because the $\phi(p_i)$ are elements of $\qq$ as $\pp$ is the preimage of $\qq$ under $\phi$. Hence
	\begin{align*}
	\KK(\qq)\otimes_{A_\pp}\roof{A}_\pp\cong \left(\KK(\qq)\otimes_{A_\pp}\roof{A}_\pp\right)/\left(\KK(\qq)\otimes_{A_\pp}\roof{\pp}\right)\cong \KK(\qq)\otimes_{A_\pp}\KK(\pp)\cong \KK(\qq)
	\end{align*}
	is a field, which shows that $\qq B_\qq\otimes_{A_\pp}\roof{A}_\pp$ is a maximal ideal.
	
	Part \itememph{b}. Brace yourself, because this proof is \emph{horrible}. I'm trying my best to cite The Stacks Project as rarely as possible, but I make no promises. We will separate the agonizing process into tiny lemmas.
	\begin{lem}\lbl{lem:ANNOYING1}
		In the situation of Proposition~\reff{prop:technicalAF}\itememph{b}, $B_\qq\otimes_{A_\pp}\KK(\pp)$ is a finite-dimensional $\KK(\pp)$-vector space as well.
	\end{lem}
	\begin{proof}
		We know that $\ov{B}=B\otimes_A\KK(\pp)$ is a finite-dimensional $\KK(\pp)$-algebra. Let $B_\pp=B\otimes_AA_\pp$ be the localization of $B$ at the image of the multiplicative set $A\setminus \pp$. Put $\ov{\qq}=\qq B_\pp/\pp B_\pp$, so that $\ov{B}_{\ov{\qq}}\cong B_\qq\otimes_{A_\pp}\KK(\pp)$. We know from Fact~\reff{fact:annoyingQF}\itememph{b} that $\Spec\ov{B}\morphism\Spec \KK(\pp)$ is quasi-finite, hence $\left\{\ov{\qq}\right\}$ is open and closed in $\Spec\ov{B}$. By \cite[\stackstag{00EE}]{stacks-project} we get an idempotent $e\in \ov{B}$ such that $\left\{\ov{\qq}\right\}\cong \Spec \ov{B}_e$. Then $\ov{\qq}$ is the only prime ideal ov $\ov{B}_e$, hence $\ov{B}_e\cong \ov{B}_{\ov{\qq}}$. Also $\ov{B}_e\cong \ov{B}[e^{-1}]$ is a $\KK(\pp)$-algebra of finite type again and $\Spec \ov{B}_e\morphism \Spec \KK(\pp)$ is clearly quasi-finite in the sense of Definition~\reff{def:quasiFinite}. Using Fact~\reff{fact:annoyingQF}\itememph{b} again, we see that $\ov{B}_e$ is finite over $\KK(\pp)$, which is precisely what we wanted to show.
	\end{proof}
	\begin{lem}\lbl{lem:ANNOYING2}
		The $\pp B_\qq$-adic completion of $B_\qq$ equals its $\qq B_\qq$-adic completion $\roof{B}_\qq$. Moreover, $\roof{B}_\qq$ is finitely generated as an $\roof{A}_\pp$-module, and such generators can be chosen from $B_\qq$.
	\end{lem}
	\begin{proof}
		As we have just seen in Lemma~\reff{lem:ANNOYING1}, $B_\qq\otimes_{A_\pp}\KK(\pp)\cong B_\qq/\pp B_\qq$ is a noetherian local ring with only one prime ideal $\ov{\qq}$. Then $\ov{\qq}$ must be the nilradical of $B_\qq/\pp B_\qq$, hence $\ov{\qq}^N=0$ for some $N\in\IN$ (because we are noetherian). This shows $(\qq B_\qq)^N\subseteq \pp B_\qq$, hence the $\pp B_\qq$-adic and the $\qq B_\qq$-adic completions of $B_\qq$ coincide.
		
		Since $B_\qq/\pp B_\qq\cong \roof{B}_\qq/\pp\roof{B}_\qq$ is finite-dimensional as a vector space over $\KK(\pp)\cong \roof{A}_\pp/\pp\roof{A}_\pp$, the second assertion follows from the more general Lemma~\reff{lem:ANNOYING3}.
	\end{proof}
	\begin{lem}\lbl{lem:ANNOYING3}
		Let $A$ be a noetherian ring which is $I$-adically complete and let $M$ be an $I$-adically complete $A$-module. If $M/IM$ is finitely generated over $A/I$, then $M$ is finitely generated over $A$. In fact, lifting a finite set of generators of $M/IM$ over $A/I$ to $M$ gives a (finite) generating set of $M$
	\end{lem}
	\begin{rem}
		The Stacks Project has a more general version of this, cf.\ \cite[\stackstag{031D}]{stacks-project}.
	\end{rem}
	\begin{proof}[Proof of Lemma~\reff{lem:ANNOYING3}]
		Let $x_1,\ldots,x_s$ be lifts of a finite generating set of $M/IM$ and let $N\subseteq M$ be the submodule generated by the $x_i$. Then $N$ is $I$-adically complete by Proposition~\reff{prop:modulesComplete}.
		
		We claim that also $N/I^nM\morphism M/I^nM$ is surjective for all $n\geq 1$. Indeed, for $n=1$ this is trivial. Using this, it's easy to see that
		\begin{align*}
		\coker\Big(N/I^nN\morphism M/I^nM\Big)=(I/I^n)\cdot\coker\Big(N/I^nN\morphism M/I^nM\Big)\;.
		\end{align*}
		But $(I/I^n)$ is a nilpotent ideal in $A/I^n$, hence $\coker\big(N/I^nN\morphism M/I^nM\big)=0$ (this feels like a dummy version of Nakayama's lemma). We thus get short exact sequences
		\begin{align*}
		0\morphism\left(I^nM\cap N\right)/I^nN\morphism N/I^nN\morphism M/I^nM\morphism 0\;.
		\end{align*}
		Let $x\in I^nM\cap N$. Then $x$ can be written as $x=\sum_i\alpha_i m_i$, where $\alpha_i\in I^n$ and $m_i\in M$. Because $N/IN\morphism M/IM$ is surjective, we may write $m_i=n_i+\mu_i$ where $n_i\in N$ and $\mu_i\in IM$. Then $y=\sum_i\alpha_i\mu_i$ is an element in $I^{n+1}M\cap N$ and has the same image in $(I^nM\cap N)/N$ as $x$. This shows that $\left((I^nM\cap N)/I^nN\right)_{n\geq 1}$ has the Mittag-Leffler property from Fact~\reff{fact:MittagLeffler}. This proves $\limit[][^1](I^nM\cap N)/I^nN=0$, hence 
		\begin{align*}
		N\cong \limit[n\geq 1]N/I^nN\morphism \limit[n\geq 1]M/I^nM\cong M
		\end{align*}
		is surjective by the six-term exact sequence from Fact~\reff{fact:6termLimitSequence}.
	\end{proof}
	Finally, the proof of Proposition~\reff{prop:technicalAF}\itememph{b} can be finished. By \cite[p.~18]{homalg}, $B_\qq\otimes_{A_\pp}\roof{A}_\pp$ is the union (or rather the colimit) over all $M\otimes_{A_\pp}\roof{A}_\pp$ where $M$ ranges through the finitely generated $A_\pp$-submodules of $B_\qq$. But for such $M$ we have $M\otimes_{A_\pp}\roof{A}_\pp\cong \roof{M}\subseteq \roof{B}_\qq$ by Corollary~\reff{cor:completionFaithfullyFlat}\itememph{a}. Moreover, by Lemma~\reff{lem:ANNOYING2} we can find such an $M$ that satisfies $\roof{M}=\roof{B}_\qq$. Hence $B_\qq\otimes_{A_\pp}\roof{A}_\pp\cong \roof{B}_\qq$, and we're finally done.
\end{proof}
\subsection{A quick reminder about curves and divisors}
\begin{lem}\lbl{lem:Hpvanishing}
	Let $k$ be an algebraically closed field, $C\morphism\Spec k$ a proper regular connected curve of genus $g$ over $k$. Let $\Ll$ be a line bundle on $C$.
	\begin{alphanumerate}
		\item If $\deg \Ll<0$, then $H^0(C,\Ll)=0$.
		\item If $\deg \Ll>2g-2$, then $H^1(C,\Ll)=0$.
	\end{alphanumerate}
\end{lem}
\begin{proof}
	Part \itememph{a}. Let $D$ be a divisor such that $\Ll\cong \Oo_C(D)$ (recall that we can do this by \cite[Proposition~3.0.1\itememph{b}]{alggeo2} -- and writing things like $\deg \Ll$ already assumed this implicitly). We need to show that $\Oo_C(D)$ has vanishing global section. Indeed, if $K$ denotes the field of fractions of $C$ (that is, the stalk at the generic point), then
	\begin{align*}
		\Gamma(C,\Oo_C(D))=\left\{f\in K\st \div(f)+D\geq 0\right\}
	\end{align*}
	by definition. But $\sum_{c\in C_1}v_c(f)=0$ for all $f\in K^\times$ (this follows from the fact that $\deg$ is well-defined -- and \emph{no}, that's no circular reasoning if we use the rather odd proof from \cite[p.\:79]{alggeo2}), hence 
	\begin{align*}
		\sum_{c\in C_1}\big(\div(f)_c+D(c)\big)=\deg D<0
	\end{align*}
	for all $f\in K^\times$, which proves that $\Gamma(C,\Oo_C(D))=0$.
	
	Part \itememph{b} is an easy consequence of \itememph{a} and Serre duality. By \cite[Corollary~3.1.2]{alggeo2} we have $\deg\Omega_C=2g-2$. Hence $\Omega_C\otimes_{\Oo_C}\Ll^*$ (where $\Ll^*$ denotes the dual of $\Ll$) has negative degree (which uses \cite[Proposition~3.0.2]{alggeo2}). Also note that $\Omega_C\otimes_{\Oo_C}(\Omega_C\otimes_{\Oo_C}\Ll^*)^*\cong \Ll$. By Serre duality as in \cite[Theorem~7\itememph{c}]{alggeo2} this implies
	\begin{align*}
		H^0\left(C,\Omega_C\otimes_{\Oo_C}\Ll^*\right)\cong \Hom_k\left(H^1(C,\Ll),k\right)\;.
	\end{align*}
	But $H^0\left(C,\Omega_C\otimes_{\Oo_C}\Ll^*\right)=0$ by \itememph{a}, so $H^1(C,\Ll)$ vanishes too.
\end{proof}
\subsection{How to geometric fibres?}
\begin{lem}\lbl{lem:stalkOfGeometricFibres}
	Let $f\colon X\morphism Y$ be a morphism of preschemes and $x\in X$, $y=f(x)$ such that $f^*\colon \KK(y)\morphism\KK(x)$ is an isomorphism. Let $k=\KK(y)$. Then $x$ has precisely one preimage $\ov{x}$ in the geometric fibre
	\begin{align*}
		X_{\ov{y}}=X\times_Y\Spec\ov{k}
	\end{align*}
	and $\Oo_{X_{\ov{y}},\ov{x}}\cong \Oo_{X,x}\otimes_{\Oo_{Y,y}}\ov{k}$. 
\end{lem}
\begin{proof}
	Locally, the question becomes whether $B\otimes_A\ov{k}$ has precisely one prime ideal over $\qq\in\Spec B$, which satisfies $\KK(\pp)\cong \KK(\qq)\cong k$ where $\pp$ is the preimage of $\qq$ in $A$. Let $S=B\setminus \qq$. Certainly, every prime ideal over $\qq$ contains $\qq\otimes_A\ov{k}$, hence these prime ideals survive localizing the multiplicative subset $S\otimes 1$. The localization of $B\otimes_A\ov{k}$ with respect to $S\otimes 1$ equals $B_\qq\otimes_{A_\pp}\ov{k}$ (as can be easily seen by pushing universal properties of tensor products and localizations around). We claim that $\qq B_\qq\otimes_A\ov{k}$ is already a maximal ideal. Indeed, since $B_\qq/\qq B_\qq\cong k$ we have
	\begin{align*}
		\big(B_\qq\otimes_A\ov{k}\big)/\big(\qq B_\qq\otimes_A\ov{k}\big)\cong k\otimes_A\ov{k}\;.
	\end{align*}
	But also $k\cong A_\pp/\pp A_\pp$, so $k\otimes_A\ov{k}\cong \ov{k}$ is a field.
	
	If we show that $\qq B_\qq\otimes_A\ov{k}$ is actually \emph{the only} maximal ideal of $B_\qq\otimes_A\ov{k}$, then both assertions will follow at once (up to noticing that $B\otimes_A\ov{k}$ and $B\otimes_{A_\pp}\ov{k}$ are actually the same). To do this, note that $B_\qq\otimes_A\ov{k}$ is integral over $B\otimes_Ak$. Indeed, $\ov{k}$ is free over $k$ (this is a field extension after all), hence $B_\qq\otimes_A\ov{k}$ is free as an $B\otimes_Ak$-module. Since $B_\qq\otimes_Ak\cong B_\qq/\pp B_\qq$ is local with maximal ideal $\qq B_\qq\otimes_Ak$, we see that $\qq B_\qq\otimes_A\ov{k}$ is the only maximal ideal over $\qq B_\qq\otimes_Ak$. Then the going-up theorem, or more precisely \cite[Theorem~7\itememph{d}]{alg1}, shows that $\qq B_\qq\otimes_A\ov{k}$ is the only maximal ideal of $B_\qq\otimes_A\ov{k}$, which means we're done.
\end{proof}
\begin{prop}\lbl{prop:GeometricFibres101}
	Let $f\colon X\morphism Y$ be a morphism of preschemes, $y\in Y$ a point and $X_y$, $X_{\ov{y}}$ its ordinary and geometric fibre respectively.
	\begin{alphanumerate}
		\item Let $x$ be a point in the fibre $X_y$ over $y$. Then
		\begin{align*}
			\dim\Oo_{X_y,x}=\sup\left\{\dim\Oo_{X_{\ov{y}},\ov{x}}\st \ov{x}\in X_{\ov{y}}\text{ lies over }x\right\}\;.
		\end{align*}
		\item If $X_y$ and $X_{\ov{y}}$ are locally noetherian (this is e.g.\ the case if $f$ is of locally finite type) and the local rings $\Oo_{X_{\ov{y}},\ov{x}}$ are regular for all $\ov{x}$ above $x$, then $\Oo_{X_y,x}$ is regular itself.
	\end{alphanumerate}
\end{prop}
\begin{proof}
	Let $k=\KK(y)$. Both assertions are local on $X$ and $Y$, so we may work with $B\otimes_Ak$ and $B\otimes_A\ov{k}$ for some ring $A$ and some $A$-algebra $B$. Let $\qq\in\Spec B$ and $\pp\in\Spec A$ be the prime ideals corresponding to $x$ and $y$. Then $\Oo_{X_y,x}\cong B_\qq\otimes_Ak$. As in the proof of Lemma~\reff{lem:stalkOfGeometricFibres} we see that $B\otimes_A\ov{k}$ is integral/free over $B\otimes_Ak$. Hence the same is true after localizing the multiplicative subset $S\otimes 1$ (where $S=B\setminus \qq$), so $B_\qq\otimes_A\ov{k}\cong \Oo_{X_y,x}\otimes_k\ov{k}$ is integral over $B\otimes_Ak\cong \Oo_{X_y,x}$ and contains all relevant prime ideals (again, as in the proof of Lemma~\reff{lem:stalkOfGeometricFibres}).
	
	In particular, the $\Oo_{X_{\ov{y}},\ov{x}}$ are the localizations of $\Oo_{X_y,x}\otimes_k\ov{k}$ at its prime ideals. Then what part \itememph{a} actually claims is that 
	\begin{align*}
		\dim\Oo_{X_y,x}=\dim\Oo_{X_y,x}\otimes_k\ov{k}\;.
	\end{align*}
	But this is a well-known fact about integral ring extensions which follows from the going-up theorem (cf.\ \cite[Theorem~7]{alg1}).
	
	To deduce \itememph{b} we use Serre's regularity criterion (cf.\ \cite[Theorem~1]{homalg}). Let $M$ and $N$ be $\Oo_{X_y,x}$-modules and let $P_\bullet\epimorphism M$ be a projective resolution. Then $P_\bullet\otimes_k\ov{k}\epimorphism M\otimes_k\ov{k}$ is a projective resolution over $\Oo_{X_y,x}\otimes_k\ov{k}$. Indeed, $P_i\otimes_k\ov{k}$ is (as a $\Oo_{X_y,x}$-module or as an abelian group, this doesn't matter) a direct sum of copies of $P_i$, so exactness is preserved. Also, if $P_i\oplus Q_i$ is free over $\Oo_{X_y,x}$, then $(P_i\otimes_k\ov{k})\oplus (Q_i\otimes_k\ov{k})$ is free over $\Oo_{X_y,x}\otimes_k\ov{k}$, so $P_i\otimes_k\ov{k}$ is still projective. Hence the homology of 
	\begin{align*}
		\left(P_\bullet\otimes_k\ov{k}\right)\otimes_{\Oo_{X_y,x}\otimes_k\ov{k}}\left(M\otimes_k\ov{k}\right)\cong \left(P_\bullet\otimes_{\Oo_{X_y,x}}M\right)\otimes_k\ov{k}
	\end{align*}
	computes $\Tor_\bullet^{\Oo_{X_y,x}\otimes_k\ov{k}}(M\otimes_k\ov{k},N\otimes_k\ov{k})$.
	
	 Note that $d=\dim\Oo_{X_y,x}$ is finite (because $\Oo_{X_y,x}$ is a noetherian local ring), hence $\dim\Oo_{X_{\ov{y}},\ov{x}}\leq d$ by for all $\ov{x}$ above $x$ by part \itememph{a}. But then $\gdim\Oo_{X_{\ov{y}},\ov{x}}\leq d$ for all $\ov{x}$ over $x$ by Serre's regularity criterion. Since the $\Oo_{X_{\ov{y}},\ov{x}}$ are precisely the localizations of $\Oo_{X_y,x}\otimes_k\ov{k}$ at its prime ideals, this shows $\gdim\big(\Oo_{X_y,x}\otimes_k\ov{k}\big)\leq d$. But then
	 \begin{align*}
	 	H_p\left(P_\bullet\otimes_{\Oo_{X_y,x}}M\otimes_k\ov{k}\right)=0\quad\text{for all }p\geq d\;.
	 \end{align*}
	 Since $\ov{k}$ is free over $k$, this already shows $H_p\big(P_\bullet\otimes_{\Oo_{X_y,x}}M\big)=0$ for $p\geq d$. In other words, 
	 \begin{align*}
	 	\Tor_p^{\Oo_{X_y,x}}(M,N)=0\quad\text{for }p\geq d\;,
	 \end{align*}
	 hence $\gdim\Oo_{X_y,x}\leq d$. Then $\Oo_{X_y,x}$ is regular by Serre's regularity criterion.
\end{proof}
\begin{rem}
	\begin{alphanumerate}
		\item Some people like to use Kähler differentials together with \cite[Proposition~1.6.3]{alg2} to prove Proposition~\reff{prop:GeometricFibres101}\itememph{b}. However, it seems quite delicate to bypass the separability requirements there, so the proof by Serre's regularity criterion seemed more clean and straightforward to me.
		\item Both Lemma~\reff{lem:stalkOfGeometricFibres} and Proposition~\reff{prop:GeometricFibres101} still hold when $\ov{k}$ is replaced by an arbitrary algebraic extension $\ell/k$ -- in fact, that's all we needed.
	\end{alphanumerate}
\end{rem}
\begin{cor}\lbl{cor:FibresAreCurvesToo}
	If $f\colon C\morphism X$ is a morphism of preschemes whose geometric fibres are regular connected curves, then so are its ordinary fibres.
\end{cor}
\begin{proof}
	By Proposition~\reff{prop:GeometricFibres101} it's clear that the fibres $C_x$ for $x\in X$ are one-dimensional and regular. To show connectedness, note that $C_{\ov{x}}\morphism C_x$ is surjective as a map of topological spaces. Indeed, locally the question becomes whether $\Spec(B\otimes_A\ov{k})\morphism\Spec (B\otimes_Ak)$ is surjective. But we've seen several times now that $B\otimes_A\ov{k}$ is integral over $B\otimes_Ak$, so surjectivity follows from \cite[Theorem~7\itememph{a}]{alg1}. Therefore, $C_x$ is the surjective image of the connected space $C_{\ov{x}}$, hence connected itself.
\end{proof}

\printbibliography

\end{document}          
